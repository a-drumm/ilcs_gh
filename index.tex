% Options for packages loaded elsewhere
\PassOptionsToPackage{unicode}{hyperref}
\PassOptionsToPackage{hyphens}{url}
\PassOptionsToPackage{dvipsnames,svgnames,x11names}{xcolor}
%
\documentclass[
  letterpaper,
  DIV=11,
  numbers=noendperiod]{scrreprt}

\usepackage{amsmath,amssymb}
\usepackage{iftex}
\ifPDFTeX
  \usepackage[T1]{fontenc}
  \usepackage[utf8]{inputenc}
  \usepackage{textcomp} % provide euro and other symbols
\else % if luatex or xetex
  \usepackage{unicode-math}
  \defaultfontfeatures{Scale=MatchLowercase}
  \defaultfontfeatures[\rmfamily]{Ligatures=TeX,Scale=1}
\fi
\usepackage{lmodern}
\ifPDFTeX\else  
    % xetex/luatex font selection
\fi
% Use upquote if available, for straight quotes in verbatim environments
\IfFileExists{upquote.sty}{\usepackage{upquote}}{}
\IfFileExists{microtype.sty}{% use microtype if available
  \usepackage[]{microtype}
  \UseMicrotypeSet[protrusion]{basicmath} % disable protrusion for tt fonts
}{}
\makeatletter
\@ifundefined{KOMAClassName}{% if non-KOMA class
  \IfFileExists{parskip.sty}{%
    \usepackage{parskip}
  }{% else
    \setlength{\parindent}{0pt}
    \setlength{\parskip}{6pt plus 2pt minus 1pt}}
}{% if KOMA class
  \KOMAoptions{parskip=half}}
\makeatother
\usepackage{xcolor}
\setlength{\emergencystretch}{3em} % prevent overfull lines
\setcounter{secnumdepth}{5}
% Make \paragraph and \subparagraph free-standing
\ifx\paragraph\undefined\else
  \let\oldparagraph\paragraph
  \renewcommand{\paragraph}[1]{\oldparagraph{#1}\mbox{}}
\fi
\ifx\subparagraph\undefined\else
  \let\oldsubparagraph\subparagraph
  \renewcommand{\subparagraph}[1]{\oldsubparagraph{#1}\mbox{}}
\fi


\providecommand{\tightlist}{%
  \setlength{\itemsep}{0pt}\setlength{\parskip}{0pt}}\usepackage{longtable,booktabs,array}
\usepackage{calc} % for calculating minipage widths
% Correct order of tables after \paragraph or \subparagraph
\usepackage{etoolbox}
\makeatletter
\patchcmd\longtable{\par}{\if@noskipsec\mbox{}\fi\par}{}{}
\makeatother
% Allow footnotes in longtable head/foot
\IfFileExists{footnotehyper.sty}{\usepackage{footnotehyper}}{\usepackage{footnote}}
\makesavenoteenv{longtable}
\usepackage{graphicx}
\makeatletter
\def\maxwidth{\ifdim\Gin@nat@width>\linewidth\linewidth\else\Gin@nat@width\fi}
\def\maxheight{\ifdim\Gin@nat@height>\textheight\textheight\else\Gin@nat@height\fi}
\makeatother
% Scale images if necessary, so that they will not overflow the page
% margins by default, and it is still possible to overwrite the defaults
% using explicit options in \includegraphics[width, height, ...]{}
\setkeys{Gin}{width=\maxwidth,height=\maxheight,keepaspectratio}
% Set default figure placement to htbp
\makeatletter
\def\fps@figure{htbp}
\makeatother

\KOMAoption{captions}{tableheading}
\makeatletter
\makeatother
\makeatletter
\@ifpackageloaded{bookmark}{}{\usepackage{bookmark}}
\makeatother
\makeatletter
\@ifpackageloaded{caption}{}{\usepackage{caption}}
\AtBeginDocument{%
\ifdefined\contentsname
  \renewcommand*\contentsname{Table of contents}
\else
  \newcommand\contentsname{Table of contents}
\fi
\ifdefined\listfigurename
  \renewcommand*\listfigurename{List of Figures}
\else
  \newcommand\listfigurename{List of Figures}
\fi
\ifdefined\listtablename
  \renewcommand*\listtablename{List of Tables}
\else
  \newcommand\listtablename{List of Tables}
\fi
\ifdefined\figurename
  \renewcommand*\figurename{Figure}
\else
  \newcommand\figurename{Figure}
\fi
\ifdefined\tablename
  \renewcommand*\tablename{Table}
\else
  \newcommand\tablename{Table}
\fi
}
\@ifpackageloaded{float}{}{\usepackage{float}}
\floatstyle{ruled}
\@ifundefined{c@chapter}{\newfloat{codelisting}{h}{lop}}{\newfloat{codelisting}{h}{lop}[chapter]}
\floatname{codelisting}{Listing}
\newcommand*\listoflistings{\listof{codelisting}{List of Listings}}
\makeatother
\makeatletter
\@ifpackageloaded{caption}{}{\usepackage{caption}}
\@ifpackageloaded{subcaption}{}{\usepackage{subcaption}}
\makeatother
\makeatletter
\@ifpackageloaded{tcolorbox}{}{\usepackage[skins,breakable]{tcolorbox}}
\makeatother
\makeatletter
\@ifundefined{shadecolor}{\definecolor{shadecolor}{rgb}{.97, .97, .97}}
\makeatother
\makeatletter
\makeatother
\makeatletter
\makeatother
\ifLuaTeX
  \usepackage{selnolig}  % disable illegal ligatures
\fi
\IfFileExists{bookmark.sty}{\usepackage{bookmark}}{\usepackage{hyperref}}
\IfFileExists{xurl.sty}{\usepackage{xurl}}{} % add URL line breaks if available
\urlstyle{same} % disable monospaced font for URLs
\hypersetup{
  pdftitle={720 ILCS 5/},
  pdfauthor={ILGA},
  colorlinks=true,
  linkcolor={blue},
  filecolor={Maroon},
  citecolor={Blue},
  urlcolor={Blue},
  pdfcreator={LaTeX via pandoc}}

\title{720 ILCS 5/}
\author{ILGA}
\date{2023-05-08}

\begin{document}
\maketitle
\ifdefined\Shaded\renewenvironment{Shaded}{\begin{tcolorbox}[frame hidden, sharp corners, enhanced, interior hidden, borderline west={3pt}{0pt}{shadecolor}, boxrule=0pt, breakable]}{\end{tcolorbox}}\fi

\renewcommand*\contentsname{Table of contents}
{
\hypersetup{linkcolor=}
\setcounter{tocdepth}{2}
\tableofcontents
}
\bookmarksetup{startatroot}

\hypertarget{home}{%
\chapter*{Home}\label{home}}
\addcontentsline{toc}{chapter}{Home}

\markboth{Home}{Home}

\hypertarget{ilcs-5-criminal-code-of-2012}{%
\section*{720 ILCS 5/ Criminal Code of
2012}\label{ilcs-5-criminal-code-of-2012}}
\addcontentsline{toc}{section}{720 ILCS 5/ Criminal Code of 2012}

\markright{720 ILCS 5/ Criminal Code of 2012}

\bookmarksetup{startatroot}

\hypertarget{article-1.-title-and-construction-of-act}{%
\chapter*{Article 1. Title And Construction Of
Act;}\label{article-1.-title-and-construction-of-act}}
\addcontentsline{toc}{chapter}{Article 1. Title And Construction Of
Act;}

\markboth{Article 1. Title And Construction Of Act;}{Article 1. Title
And Construction Of Act;}

\hypertarget{ilcs-51-1-from-ch.-38-par.-1-1}{%
\subsection*{(720 ILCS 5/1-1) (from Ch. 38, par.
1-1)}\label{ilcs-51-1-from-ch.-38-par.-1-1}}
\addcontentsline{toc}{subsection}{(720 ILCS 5/1-1) (from Ch. 38, par.
1-1)}

\hypertarget{sec.-1-1.-short-title.}{%
\section*{Sec. 1-1. Short title.}\label{sec.-1-1.-short-title.}}
\addcontentsline{toc}{section}{Sec. 1-1. Short title.}

\markright{Sec. 1-1. Short title.}

This Act may be cited as the

Criminal Code of 2012.

(Source: P.A. 97-1108, eff. 1-1-13.)

\hypertarget{ilcs-51-2-from-ch.-38-par.-1-2}{%
\subsection*{(720 ILCS 5/1-2) (from Ch. 38, par.
1-2)}\label{ilcs-51-2-from-ch.-38-par.-1-2}}
\addcontentsline{toc}{subsection}{(720 ILCS 5/1-2) (from Ch. 38, par.
1-2)}

\hypertarget{sec.-1-2.-general-purposes.}{%
\section*{Sec. 1-2. General
purposes.}\label{sec.-1-2.-general-purposes.}}
\addcontentsline{toc}{section}{Sec. 1-2. General purposes.}

\markright{Sec. 1-2. General purposes.}

The provisions of this Code shall be construed in accordance with the
general purposes hereof, to:

(a) Forbid and prevent the commission of offenses;

(b) Define adequately the act and mental state which constitute each
offense, and limit the condemnation of conduct as criminal when it is
without fault;

(c) Prescribe penalties which are proportionate to the seriousness of
offenses and which permit recognition of differences in rehabilitation
possibilities among individual offenders;

(d) Prevent arbitrary or oppressive treatment of persons accused or
convicted of offenses.

(Source: Laws 1961, p.~1983.)

\hypertarget{ilcs-51-3-from-ch.-38-par.-1-3}{%
\subsection*{(720 ILCS 5/1-3) (from Ch. 38, par.
1-3)}\label{ilcs-51-3-from-ch.-38-par.-1-3}}
\addcontentsline{toc}{subsection}{(720 ILCS 5/1-3) (from Ch. 38, par.
1-3)}

\hypertarget{sec.-1-3.-applicability-of-common-law.}{%
\section*{Sec. 1-3. Applicability of common
law.}\label{sec.-1-3.-applicability-of-common-law.}}
\addcontentsline{toc}{section}{Sec. 1-3. Applicability of common law.}

\markright{Sec. 1-3. Applicability of common law.}

No conduct constitutes an offense unless it is described as an offense
in this Code or in another statute of this State. However, this
provision does not affect the power of a court to punish for contempt or
to employ any sanction authorized by law for the enforcement of an order
or civil judgment.

(Source: P.A. 79-1360.)

\hypertarget{ilcs-51-4-from-ch.-38-par.-1-4}{%
\subsection*{(720 ILCS 5/1-4) (from Ch. 38, par.
1-4)}\label{ilcs-51-4-from-ch.-38-par.-1-4}}
\addcontentsline{toc}{subsection}{(720 ILCS 5/1-4) (from Ch. 38, par.
1-4)}

\hypertarget{sec.-1-4.-civil-remedies-preserved.}{%
\section*{Sec. 1-4. Civil remedies
preserved.}\label{sec.-1-4.-civil-remedies-preserved.}}
\addcontentsline{toc}{section}{Sec. 1-4. Civil remedies preserved.}

\markright{Sec. 1-4. Civil remedies preserved.}

This Code does not bar, suspend, or otherwise affect any right or
liability to damages, penalty, forfeiture, or other remedy authorized by
law to be recovered or enforced in a civil action, for any conduct which
this Code makes punishable; and the civil injury is not merged in the
offense.

(Source: Laws 1961, p.~1983.)

\hypertarget{ilcs-51-5-from-ch.-38-par.-1-5}{%
\subsection*{(720 ILCS 5/1-5) (from Ch. 38, par.
1-5)}\label{ilcs-51-5-from-ch.-38-par.-1-5}}
\addcontentsline{toc}{subsection}{(720 ILCS 5/1-5) (from Ch. 38, par.
1-5)}

\hypertarget{sec.-1-5.-state-criminal-jurisdiction.}{%
\section*{Sec. 1-5. State criminal
jurisdiction.}\label{sec.-1-5.-state-criminal-jurisdiction.}}
\addcontentsline{toc}{section}{Sec. 1-5. State criminal jurisdiction.}

\markright{Sec. 1-5. State criminal jurisdiction.}

(a) A person is subject to prosecution in this State for an offense
which he commits, while either within or outside the State, by his own
conduct or that of another for which he is legally accountable, if:

(1) the offense is committed either wholly or partly within the State;
or

(2) the conduct outside the State constitutes an attempt to commit an
offense within the State; or

(3) the conduct outside the State constitutes a conspiracy to commit an
offense within the State, and an act in furtherance of the conspiracy
occurs in the State; or

(4) the conduct within the State constitutes an attempt, solicitation or
conspiracy to commit in another jurisdiction an offense under the laws
of both this State and such other jurisdiction.

(b) An offense is committed partly within this State, if either the
conduct which is an element of the offense, or the result which is such
an element, occurs within the State. In a prosecution pursuant to
paragraph (3) of subsection (a) of Section 9-1, the attempt or
commission of a forcible felony other than second degree murder within
this State is conduct which is an element of the offense for which a
person is subject to prosecution in this State. In homicide, the
``result'' is either the physical contact which causes death, or the
death itself; and if the body of a homicide victim is found within the
State, the death is presumed to have occurred within the State.

(c) An offense which is based on an omission to perform a duty imposed
by the law of this State is committed within the State, regardless of
the location of the offender at the time of the omission.

(Source: P.A. 91-357, eff. 7-29-99.)

\hypertarget{ilcs-51-6-from-ch.-38-par.-1-6}{%
\subsection*{(720 ILCS 5/1-6) (from Ch. 38, par.
1-6)}\label{ilcs-51-6-from-ch.-38-par.-1-6}}
\addcontentsline{toc}{subsection}{(720 ILCS 5/1-6) (from Ch. 38, par.
1-6)}

\hypertarget{sec.-1-6.-place-of-trial.}{%
\section*{Sec. 1-6. Place of trial.}\label{sec.-1-6.-place-of-trial.}}
\addcontentsline{toc}{section}{Sec. 1-6. Place of trial.}

\markright{Sec. 1-6. Place of trial.}

(a) Generally.

Criminal actions shall be tried in the county where the offense was
committed, except as otherwise provided by law. The State is not
required to prove during trial that the alleged offense occurred in any
particular county in this State. When a defendant contests the place of
trial under this Section, all proceedings regarding this issue shall be
conducted under Section 114-1 of the Code of Criminal Procedure of 1963.
All objections of improper place of trial are waived by a defendant
unless made before trial.

(b) Assailant and Victim in Different Counties.

If a person committing an offense upon the person of another is located
in one county and his victim is located in another county at the time of
the commission of the offense, trial may be had in either of said
counties.

(c) Death and Cause of Death in Different Places or Undetermined.

If cause of death is inflicted in one county and death ensues in another
county, the offender may be tried in either county. If neither the
county in which the cause of death was inflicted nor the county in which
death ensued are known before trial, the offender may be tried in the
county where the body was found.

(d) Offense Commenced Outside the State.

If the commission of an offense commenced outside the State is
consummated within this State, the offender shall be tried in the county
where the offense is consummated.

(e) Offenses Committed in Bordering Navigable Waters.

If an offense is committed on any of the navigable waters bordering on
this State, the offender may be tried in any county adjacent to such
navigable water.

(f) Offenses Committed while in Transit.

If an offense is committed upon any railroad car, vehicle, watercraft or
aircraft passing within this State, and it cannot readily be determined
in which county the offense was committed, the offender may be tried in
any county through which such railroad car, vehicle, watercraft or
aircraft has passed.

(g) Theft.

A person who commits theft of property may be tried in any county in
which he exerted control over such property.

(h) Bigamy.

A person who commits the offense of bigamy may be tried in any county
where the bigamous marriage or bigamous cohabitation has occurred.

(i) Kidnaping.

A person who commits the offense of kidnaping may be tried in any county
in which his victim has traveled or has been confined during the course
of the offense.

(j) Pandering.

A person who commits the offense of pandering as set forth in
subdivision (a)(2)(A) or (a)(2)(B) of Section 11-14.3 may be tried in
any county in which the prostitution was practiced or in any county in
which any act in furtherance of the offense shall have been committed.

(k) Treason.

A person who commits the offense of treason may be tried in any county.

(l) Criminal Defamation.

If criminal defamation is spoken, printed or written in one county and
is received or circulated in another or other counties, the offender
shall be tried in the county where the defamation is spoken, printed or
written. If the defamation is spoken, printed or written outside this
state, or the offender resides outside this state, the offender may be
tried in any county in this state in which the defamation was circulated
or received.

(m) Inchoate Offenses.

A person who commits an inchoate offense may be tried in any county in
which any act which is an element of the offense, including the
agreement in conspiracy, is committed.

(n) Accountability for Conduct of Another.

Where a person in one county solicits, aids, abets, agrees, or attempts
to aid another in the planning or commission of an offense in another
county, he may be tried for the offense in either county.

(o) Child Abduction.

A person who commits the offense of child abduction may be tried in any
county in which his victim has traveled, been detained, concealed or
removed to during the course of the offense. Notwithstanding the
foregoing, unless for good cause shown, the preferred place of trial
shall be the county of the residence of the lawful custodian.

(p) A person who commits the offense of narcotics racketeering may be
tried in any county where cannabis or a controlled substance which is
the basis for the charge of narcotics racketeering was used; acquired;
transferred or distributed to, from or through; or any county where any
act was performed to further the use; acquisition, transfer or
distribution of said cannabis or controlled substance; any money,
property, property interest, or any other asset generated by narcotics
activities was acquired, used, sold, transferred or distributed to, from
or through; or, any enterprise interest obtained as a result of
narcotics racketeering was acquired, used, transferred or distributed
to, from or through, or where any activity was conducted by the
enterprise or any conduct to further the interests of such an
enterprise.

(q) A person who commits the offense of money laundering may be tried in
any county where any part of a financial transaction in criminally
derived property took place or in any county where any money or monetary
instrument which is the basis for the offense was acquired, used, sold,
transferred or distributed to, from or through.

(r) A person who commits the offense of cannabis trafficking or
controlled substance trafficking may be tried in any county.

(s) A person who commits the offense of online sale of stolen property,
online theft by deception, or electronic fencing may be tried in any
county where any one or more elements of the offense took place,
regardless of whether the element of the offense was the result of acts
by the accused, the victim or by another person, and regardless of
whether the defendant was ever physically present within the boundaries
of the county.

(t) A person who commits the offense of identity theft or aggravated
identity theft may be tried in any one of the following counties in
which: (1) the offense occurred; (2) the information used to commit the
offense was illegally used; or (3) the victim resides.

(u) A person who commits the offense of financial exploitation of an
elderly person or a person with a disability may be tried in any one of
the following counties in which: (1) any part of the offense occurred;
or (2) the victim or one of the victims reside.

If a person is charged with more than one violation of identity theft or
aggravated identity theft and those violations may be tried in more than
one county, any of those counties is a proper venue for all of the
violations.

(Source: P.A. 101-394, eff. 1-1-20 .)

\hypertarget{ilcs-51-8-from-ch.-38-par.-1-8}{%
\subsection*{(720 ILCS 5/1-8) (from Ch. 38, par.
1-8)}\label{ilcs-51-8-from-ch.-38-par.-1-8}}
\addcontentsline{toc}{subsection}{(720 ILCS 5/1-8) (from Ch. 38, par.
1-8)}

\hypertarget{sec.-1-8.-order-of-protection-status.}{%
\section*{Sec. 1-8. Order of protection;
status.}\label{sec.-1-8.-order-of-protection-status.}}
\addcontentsline{toc}{section}{Sec. 1-8. Order of protection; status.}

\markright{Sec. 1-8. Order of protection; status.}

Whenever relief sought under this Code is based on allegations of
domestic violence, as defined in the Illinois Domestic Violence Act of
1986, the court, before granting relief, shall determine whether any
order of protection has previously been entered in the instant
proceeding or any other proceeding in which any party, or a child of any
party, or both, if relevant, has been designated as either a respondent
or a protected person.

(Source: P.A. 87-743.)

\bookmarksetup{startatroot}

\hypertarget{article-2.-general-definitions}{%
\chapter*{Article 2. General
Definitions}\label{article-2.-general-definitions}}
\addcontentsline{toc}{chapter}{Article 2. General Definitions}

\markboth{Article 2. General Definitions}{Article 2. General
Definitions}

\hypertarget{ilcs-52-0.5-was-720-ilcs-52-.5}{%
\subsection*{(720 ILCS 5/2-0.5) (was 720 ILCS
5/2-.5)}\label{ilcs-52-0.5-was-720-ilcs-52-.5}}
\addcontentsline{toc}{subsection}{(720 ILCS 5/2-0.5) (was 720 ILCS
5/2-.5)}

\hypertarget{sec.-2-0.5.-definitions.}{%
\section*{Sec. 2-0.5. Definitions.}\label{sec.-2-0.5.-definitions.}}
\addcontentsline{toc}{section}{Sec. 2-0.5. Definitions.}

\markright{Sec. 2-0.5. Definitions.}

For the purposes of this Code, the words and phrases described in this
Article have the meanings designated in this Article, except when a
particular context clearly requires a different meaning.

(Source: P.A. 95-331, eff. 8-21-07.)

\hypertarget{ilcs-52-1-from-ch.-38-par.-2-1}{%
\subsection*{(720 ILCS 5/2-1) (from Ch. 38, par.
2-1)}\label{ilcs-52-1-from-ch.-38-par.-2-1}}
\addcontentsline{toc}{subsection}{(720 ILCS 5/2-1) (from Ch. 38, par.
2-1)}

\hypertarget{sec.-2-1.-acquittal.}{%
\section*{Sec. 2-1. ``Acquittal''.}\label{sec.-2-1.-acquittal.}}
\addcontentsline{toc}{section}{Sec. 2-1. ``Acquittal''.}

\markright{Sec. 2-1. ``Acquittal''.}

``Acquittal'' means a verdict or finding of not guilty of an offense,
rendered by a legally constituted jury or by a court of competent
jurisdiction authorized to try the case without a jury.

(Source: Laws 1961, p.~1983.)

\hypertarget{ilcs-52-2-from-ch.-38-par.-2-2}{%
\subsection*{(720 ILCS 5/2-2) (from Ch. 38, par.
2-2)}\label{ilcs-52-2-from-ch.-38-par.-2-2}}
\addcontentsline{toc}{subsection}{(720 ILCS 5/2-2) (from Ch. 38, par.
2-2)}

\hypertarget{sec.-2-2.-act.}{%
\section*{Sec. 2-2. ``Act''.}\label{sec.-2-2.-act.}}
\addcontentsline{toc}{section}{Sec. 2-2. ``Act''.}

\markright{Sec. 2-2. ``Act''.}

``Act'' includes a failure or omission to take action.

(Source: Laws 1961, p.~1983.)

\hypertarget{ilcs-52-3-from-ch.-38-par.-2-3}{%
\subsection*{(720 ILCS 5/2-3) (from Ch. 38, par.
2-3)}\label{ilcs-52-3-from-ch.-38-par.-2-3}}
\addcontentsline{toc}{subsection}{(720 ILCS 5/2-3) (from Ch. 38, par.
2-3)}

\hypertarget{sec.-2-3.-another.}{%
\section*{Sec. 2-3. ``Another''.}\label{sec.-2-3.-another.}}
\addcontentsline{toc}{section}{Sec. 2-3. ``Another''.}

\markright{Sec. 2-3. ``Another''.}

``Another'' means a person or persons as defined in this Code other than
the offender.

(Source: Laws 1961, p.~1983.)

\hypertarget{ilcs-52-3.5}{%
\subsection*{(720 ILCS 5/2-3.5)}\label{ilcs-52-3.5}}
\addcontentsline{toc}{subsection}{(720 ILCS 5/2-3.5)}

\hypertarget{sec.-2-3.5.}{%
\section*{Sec. 2-3.5.}\label{sec.-2-3.5.}}
\addcontentsline{toc}{section}{Sec. 2-3.5.}

\markright{Sec. 2-3.5.}

``Community policing volunteer'' means a person who is summoned or
directed by a peace officer or any person actively participating in a
community policing program and who is engaged in lawful conduct intended
to assist any unit of government in enforcing any criminal or civil law.
For the purpose of this Section, ``community policing program'' means
any plan, system or strategy established by and conducted under the
auspices of a law enforcement agency in which citizens participate with
and are guided by the law enforcement agency and work with members of
that agency to reduce or prevent crime within a defined geographic area.

(Source: P.A. 90-651, eff. 1-1-99.)

\hypertarget{ilcs-52-3.6}{%
\subsection*{(720 ILCS 5/2-3.6)}\label{ilcs-52-3.6}}
\addcontentsline{toc}{subsection}{(720 ILCS 5/2-3.6)}

\hypertarget{sec.-2-3.6.-armed-with-a-firearm.}{%
\section*{Sec. 2-3.6. ``Armed with a
firearm''.}\label{sec.-2-3.6.-armed-with-a-firearm.}}
\addcontentsline{toc}{section}{Sec. 2-3.6. ``Armed with a firearm''.}

\markright{Sec. 2-3.6. ``Armed with a firearm''.}

Except as otherwise provided in a specific Section, a person is
considered ``armed with a firearm'' when he or she carries on or about
his or her person or is otherwise armed with a firearm.

(Source: P.A. 91-404, eff. 1-1-00.)

\hypertarget{ilcs-52-4-from-ch.-38-par.-2-4}{%
\subsection*{(720 ILCS 5/2-4) (from Ch. 38, par.
2-4)}\label{ilcs-52-4-from-ch.-38-par.-2-4}}
\addcontentsline{toc}{subsection}{(720 ILCS 5/2-4) (from Ch. 38, par.
2-4)}

\hypertarget{sec.-2-4.-conduct.}{%
\section*{Sec. 2-4. ``Conduct''.}\label{sec.-2-4.-conduct.}}
\addcontentsline{toc}{section}{Sec. 2-4. ``Conduct''.}

\markright{Sec. 2-4. ``Conduct''.}

``Conduct'' means an act or a series of acts, and the accompanying
mental state.

(Source: Laws 1961, p.~1983.)

\hypertarget{ilcs-52-5-from-ch.-38-par.-2-5}{%
\subsection*{(720 ILCS 5/2-5) (from Ch. 38, par.
2-5)}\label{ilcs-52-5-from-ch.-38-par.-2-5}}
\addcontentsline{toc}{subsection}{(720 ILCS 5/2-5) (from Ch. 38, par.
2-5)}

\hypertarget{sec.-2-5.-conviction.}{%
\section*{Sec. 2-5. ``Conviction''.}\label{sec.-2-5.-conviction.}}
\addcontentsline{toc}{section}{Sec. 2-5. ``Conviction''.}

\markright{Sec. 2-5. ``Conviction''.}

``Conviction'' means a judgment of conviction or sentence entered upon a
plea of guilty or upon a verdict or finding of guilty of an offense,
rendered by a legally constituted jury or by a court of competent
jurisdiction authorized to try the case without a jury.

(Source: Laws 1961, p.~1983 .)

\hypertarget{ilcs-52-5.1}{%
\subsection*{(720 ILCS 5/2-5.1)}\label{ilcs-52-5.1}}
\addcontentsline{toc}{subsection}{(720 ILCS 5/2-5.1)}

\hypertarget{sec.-2-5.1.-day-care-center.}{%
\section*{Sec. 2-5.1. Day care
center.}\label{sec.-2-5.1.-day-care-center.}}
\addcontentsline{toc}{section}{Sec. 2-5.1. Day care center.}

\markright{Sec. 2-5.1. Day care center.}

``Day care center'' has the meaning ascribed to it in Section 2.09 of
the Child Care Act of 1969.

(Source: P.A. 96-556, eff. 1-1-10.)

\hypertarget{ilcs-52-5.2}{%
\subsection*{(720 ILCS 5/2-5.2)}\label{ilcs-52-5.2}}
\addcontentsline{toc}{subsection}{(720 ILCS 5/2-5.2)}

\hypertarget{sec.-2-5.2.-day-care-home.}{%
\section*{Sec. 2-5.2. Day care home.}\label{sec.-2-5.2.-day-care-home.}}
\addcontentsline{toc}{section}{Sec. 2-5.2. Day care home.}

\markright{Sec. 2-5.2. Day care home.}

``Day care home'' has the meaning ascribed to it in Section 2.18 of the
Child Care Act of 1969.

(Source: P.A. 96-556, eff. 1-1-10.)

\hypertarget{ilcs-52-6-from-ch.-38-par.-2-6}{%
\subsection*{(720 ILCS 5/2-6) (from Ch. 38, par.
2-6)}\label{ilcs-52-6-from-ch.-38-par.-2-6}}
\addcontentsline{toc}{subsection}{(720 ILCS 5/2-6) (from Ch. 38, par.
2-6)}

\hypertarget{sec.-2-6.-dwelling.}{%
\section*{Sec. 2-6. ``Dwelling''.}\label{sec.-2-6.-dwelling.}}
\addcontentsline{toc}{section}{Sec. 2-6. ``Dwelling''.}

\markright{Sec. 2-6. ``Dwelling''.}

(a) Except as otherwise provided in subsection (b) of this Section,
``dwelling'' means a building or portion thereof, a tent, a vehicle, or
other enclosed space which is used or intended for use as a human
habitation, home or residence.

(b) For the purposes of Section 19-3 of this Code, ``dwelling'' means a
house, apartment, mobile home, trailer, or other living quarters in
which at the time of the alleged offense the owners or occupants
actually reside or in their absence intend within a reasonable period of
time to reside.

(Source: P.A. 84-1289.)

\hypertarget{ilcs-52-6.5}{%
\subsection*{(720 ILCS 5/2-6.5)}\label{ilcs-52-6.5}}
\addcontentsline{toc}{subsection}{(720 ILCS 5/2-6.5)}

\hypertarget{sec.-2-6.5.-emergency-medical-technician.}{%
\section*{Sec. 2-6.5. Emergency medical
technician.}\label{sec.-2-6.5.-emergency-medical-technician.}}
\addcontentsline{toc}{section}{Sec. 2-6.5. Emergency medical
technician.}

\markright{Sec. 2-6.5. Emergency medical technician.}

``Emergency medical technician-ambulance'', ``emergency medical
technician-intermediate'', and ``emergency medical
technician-paramedic'' have the meanings ascribed to them in the
Emergency Medical Services (EMS) Systems Act.

(Source: P.A. 88-433.)

\hypertarget{ilcs-52-6.6}{%
\subsection*{(720 ILCS 5/2-6.6)}\label{ilcs-52-6.6}}
\addcontentsline{toc}{subsection}{(720 ILCS 5/2-6.6)}

\hypertarget{sec.-2-6.6.-emergency-management-worker.}{%
\section*{Sec. 2-6.6. Emergency management
worker.}\label{sec.-2-6.6.-emergency-management-worker.}}
\addcontentsline{toc}{section}{Sec. 2-6.6. Emergency management worker.}

\markright{Sec. 2-6.6. Emergency management worker.}

``Emergency management worker'' shall include the following:

(a) any person, paid or unpaid, who is a member of a local or county
emergency services and disaster agency as defined by the Illinois
Emergency Management Agency Act, or who is an employee of the Illinois
Emergency Management Agency or the Federal Emergency Management Agency;

(b) any employee or volunteer of the American Red Cross;

(c) any employee of a federal, State, county, or local government agency
assisting an emergency services and disaster agency, the Illinois
Emergency Management Agency, or the Federal Emergency Management Agency
through mutual aid or as otherwise requested or directed in time of
disaster or emergency; and

(d) any person volunteering or directed to assist an emergency services
and disaster agency, the Illinois Emergency Management Agency, or the
Federal Emergency Management Agency.

(Source: P.A. 94-243, eff. 1-1-06; 94-323, eff. 1-1-06; 95-331, eff.
8-21-07.)

\hypertarget{ilcs-52-7-from-ch.-38-par.-2-7}{%
\subsection*{(720 ILCS 5/2-7) (from Ch. 38, par.
2-7)}\label{ilcs-52-7-from-ch.-38-par.-2-7}}
\addcontentsline{toc}{subsection}{(720 ILCS 5/2-7) (from Ch. 38, par.
2-7)}

\hypertarget{sec.-2-7.-felony.}{%
\section*{Sec. 2-7. ``Felony''.}\label{sec.-2-7.-felony.}}
\addcontentsline{toc}{section}{Sec. 2-7. ``Felony''.}

\markright{Sec. 2-7. ``Felony''.}

``Felony'' means an offense for which a sentence to death or to a term
of imprisonment in a penitentiary for one year or more is provided.

(Source: P.A. 77-2638.)

\hypertarget{ilcs-52-7.1}{%
\subsection*{(720 ILCS 5/2-7.1)}\label{ilcs-52-7.1}}
\addcontentsline{toc}{subsection}{(720 ILCS 5/2-7.1)}

\hypertarget{sec.-2-7.1.-firearm-and-firearm-ammunition.}{%
\section*{Sec. 2-7.1. ``Firearm'' and ``firearm
ammunition''.}\label{sec.-2-7.1.-firearm-and-firearm-ammunition.}}
\addcontentsline{toc}{section}{Sec. 2-7.1. ``Firearm'' and ``firearm
ammunition''.}

\markright{Sec. 2-7.1. ``Firearm'' and ``firearm ammunition''.}

``Firearm'' and ``firearm ammunition'' have the meanings ascribed to
them in Section 1.1 of the Firearm Owners Identification Card Act.

(Source: P.A. 91-544, eff. 1-1-00.)

\hypertarget{ilcs-52-7.5}{%
\subsection*{(720 ILCS 5/2-7.5)}\label{ilcs-52-7.5}}
\addcontentsline{toc}{subsection}{(720 ILCS 5/2-7.5)}

\hypertarget{sec.-2-7.5.-firearm.}{%
\section*{Sec. 2-7.5. ``Firearm''.}\label{sec.-2-7.5.-firearm.}}
\addcontentsline{toc}{section}{Sec. 2-7.5. ``Firearm''.}

\markright{Sec. 2-7.5. ``Firearm''.}

Except as otherwise provided in a specific Section, ``firearm'' has the
meaning ascribed to it in Section 1.1 of the Firearm Owners
Identification Card Act.

(Source: P.A. 95-331, eff. 8-21-07.)

\hypertarget{ilcs-52-8-from-ch.-38-par.-2-8}{%
\subsection*{(720 ILCS 5/2-8) (from Ch. 38, par.
2-8)}\label{ilcs-52-8-from-ch.-38-par.-2-8}}
\addcontentsline{toc}{subsection}{(720 ILCS 5/2-8) (from Ch. 38, par.
2-8)}

\hypertarget{sec.-2-8.-forcible-felony.}{%
\section*{Sec. 2-8. ``Forcible
felony''.}\label{sec.-2-8.-forcible-felony.}}
\addcontentsline{toc}{section}{Sec. 2-8. ``Forcible felony''.}

\markright{Sec. 2-8. ``Forcible felony''.}

``Forcible felony'' means treason, first degree murder, second degree
murder, predatory criminal sexual assault of a child, aggravated
criminal sexual assault, criminal sexual assault, robbery, burglary,
residential burglary, aggravated arson, arson, aggravated kidnaping,
kidnaping, aggravated battery resulting in great bodily harm or
permanent disability or disfigurement and any other felony which
involves the use or threat of physical force or violence against any
individual.

(Source: P.A. 88-277; 89-428, eff. 12-13-95; 89-462, eff. 5-29-96.)

\hypertarget{ilcs-52-8.1}{%
\subsection*{(720 ILCS 5/2-8.1)}\label{ilcs-52-8.1}}
\addcontentsline{toc}{subsection}{(720 ILCS 5/2-8.1)}

\hypertarget{sec.-2-8.1.-group-day-care-home.}{%
\section*{Sec. 2-8.1. Group day care
home.}\label{sec.-2-8.1.-group-day-care-home.}}
\addcontentsline{toc}{section}{Sec. 2-8.1. Group day care home.}

\markright{Sec. 2-8.1. Group day care home.}

``Group day care home'' has the meaning ascribed to it in Section 2.20
of the Child Care Act of 1969.

(Source: P.A. 96-556, eff. 1-1-10.)

\hypertarget{ilcs-52-9-from-ch.-38-par.-2-9}{%
\subsection*{(720 ILCS 5/2-9) (from Ch. 38, par.
2-9)}\label{ilcs-52-9-from-ch.-38-par.-2-9}}
\addcontentsline{toc}{subsection}{(720 ILCS 5/2-9) (from Ch. 38, par.
2-9)}

\hypertarget{sec.-2-9.-included-offense.}{%
\section*{Sec. 2-9. ``Included
offense''.}\label{sec.-2-9.-included-offense.}}
\addcontentsline{toc}{section}{Sec. 2-9. ``Included offense''.}

\markright{Sec. 2-9. ``Included offense''.}

``Included offense'' means an offense which

(a) Is established by proof of the same or less than all of the facts or
a less culpable mental state (or both), than that which is required to
establish the commission of the offense charged, or

(b) Consists of an attempt to commit the offense charged or an offense
included therein.

(Source: Laws 1961, p.~1983.)

\hypertarget{ilcs-52-10-from-ch.-38-par.-2-10}{%
\subsection*{(720 ILCS 5/2-10) (from Ch. 38, par.
2-10)}\label{ilcs-52-10-from-ch.-38-par.-2-10}}
\addcontentsline{toc}{subsection}{(720 ILCS 5/2-10) (from Ch. 38, par.
2-10)}

\hypertarget{sec.-2-10.-includes.}{%
\section*{Sec. 2-10. ``Includes''.}\label{sec.-2-10.-includes.}}
\addcontentsline{toc}{section}{Sec. 2-10. ``Includes''.}

\markright{Sec. 2-10. ``Includes''.}

``Includes'' or ``including'' means comprehending among other
particulars, without limiting the generality of the foregoing word or
phrase.

(Source: Laws 1961, p.~1983.)

\hypertarget{ilcs-52-10.1-from-ch.-38-par.-2-10.1}{%
\subsection*{(720 ILCS 5/2-10.1) (from Ch. 38, par.
2-10.1)}\label{ilcs-52-10.1-from-ch.-38-par.-2-10.1}}
\addcontentsline{toc}{subsection}{(720 ILCS 5/2-10.1) (from Ch. 38, par.
2-10.1)}

\hypertarget{sec.-2-10.1.}{%
\section*{Sec. 2-10.1.}\label{sec.-2-10.1.}}
\addcontentsline{toc}{section}{Sec. 2-10.1.}

\markright{Sec. 2-10.1.}

``Person with a severe or profound intellectual disability'' means a
person (i) whose intelligence quotient does not exceed 40 or (ii) whose
intelligence quotient does not exceed 55 and who suffers from
significant mental illness to the extent that the person's ability to
exercise rational judgment is impaired. In any proceeding in which the
defendant is charged with committing a violation of Section 10-2, 10-5,
11-1.30, 11-1.60, 11-14.4, 11-15.1, 11-19.1, 11-19.2, 11-20.1, 11-20.1B,
11-20.3, 12-4.3, 12-14, or 12-16, or subdivision (b)(1) of Section
12-3.05, of this Code against a victim who is alleged to be a person
with a severe or profound intellectual disability, any findings
concerning the victim's status as a person with a severe or profound
intellectual disability, made by a court after a judicial admission
hearing concerning the victim under Articles V and VI of Chapter IV of
the Mental Health and Developmental Disabilities Code shall be
admissible.

(Source: P.A. 98-756, eff. 7-16-14; 99-143, eff. 7-27-15.)

\hypertarget{ilcs-52-10.2}{%
\subsection*{(720 ILCS 5/2-10.2)}\label{ilcs-52-10.2}}
\addcontentsline{toc}{subsection}{(720 ILCS 5/2-10.2)}

\hypertarget{sec.-2-10.2.-laser-or-laser-device.}{%
\section*{Sec. 2-10.2. Laser or laser
device.}\label{sec.-2-10.2.-laser-or-laser-device.}}
\addcontentsline{toc}{section}{Sec. 2-10.2. Laser or laser device.}

\markright{Sec. 2-10.2. Laser or laser device.}

``Laser'' or ``laser device'' means any small or hand-held battery
powered device which converts incident electromagnetic radiation of
mixed frequencies to one or more discrete frequencies of highly
amplified and coherent visible radiation or light. Proof that a
particular device casts a small red dot or other similar small and
discrete image or small and discrete visual signal upon a target surface
at least 15 feet away creates a rebuttable presumption that the device
is a laser. Flashlights and similar lamps, lanterns, lights, and
penlights are not laser devices.

(Source: P.A. 91-672, eff. 1-1-00.)

\hypertarget{ilcs-52-10.3}{%
\subsection*{(720 ILCS 5/2-10.3)}\label{ilcs-52-10.3}}
\addcontentsline{toc}{subsection}{(720 ILCS 5/2-10.3)}

\hypertarget{sec.-2-10.3.-laser-gunsight.}{%
\section*{Sec. 2-10.3. Laser
gunsight.}\label{sec.-2-10.3.-laser-gunsight.}}
\addcontentsline{toc}{section}{Sec. 2-10.3. Laser gunsight.}

\markright{Sec. 2-10.3. Laser gunsight.}

``Laser gunsight'' means any battery powered laser device manufactured
to function as a firearm aiming device or sold as a firearm aiming
device.

(Source: P.A. 91-672, eff. 1-1-00.)

\hypertarget{ilcs-52-11-from-ch.-38-par.-2-11}{%
\subsection*{(720 ILCS 5/2-11) (from Ch. 38, par.
2-11)}\label{ilcs-52-11-from-ch.-38-par.-2-11}}
\addcontentsline{toc}{subsection}{(720 ILCS 5/2-11) (from Ch. 38, par.
2-11)}

\hypertarget{sec.-2-11.-misdemeanor.}{%
\section*{Sec. 2-11. ``Misdemeanor''.}\label{sec.-2-11.-misdemeanor.}}
\addcontentsline{toc}{section}{Sec. 2-11. ``Misdemeanor''.}

\markright{Sec. 2-11. ``Misdemeanor''.}

``Misdemeanor'' means any offense for which a sentence to a term of
imprisonment in other than a penitentiary for less than one year may be
imposed.

(Source: P.A. 77-2638.)

\hypertarget{ilcs-52-11.1}{%
\subsection*{(720 ILCS 5/2-11.1)}\label{ilcs-52-11.1}}
\addcontentsline{toc}{subsection}{(720 ILCS 5/2-11.1)}

\hypertarget{sec.-2-11.1.-motor-vehicle.}{%
\section*{Sec. 2-11.1. ``Motor
vehicle''.}\label{sec.-2-11.1.-motor-vehicle.}}
\addcontentsline{toc}{section}{Sec. 2-11.1. ``Motor vehicle''.}

\markright{Sec. 2-11.1. ``Motor vehicle''.}

``Motor vehicle'' has the meaning ascribed to it in the Illinois Vehicle
Code.

(Source: P.A. 97-1108, eff. 1-1-13.)

\hypertarget{ilcs-52-12-from-ch.-38-par.-2-12}{%
\subsection*{(720 ILCS 5/2-12) (from Ch. 38, par.
2-12)}\label{ilcs-52-12-from-ch.-38-par.-2-12}}
\addcontentsline{toc}{subsection}{(720 ILCS 5/2-12) (from Ch. 38, par.
2-12)}

\hypertarget{sec.-2-12.-offense.}{%
\section*{Sec. 2-12. ``Offense''.}\label{sec.-2-12.-offense.}}
\addcontentsline{toc}{section}{Sec. 2-12. ``Offense''.}

\markright{Sec. 2-12. ``Offense''.}

``Offense'' means a violation of any penal statute of this State.

(Source: Laws 1961, p.~1983.)

\hypertarget{ilcs-52-12.1}{%
\subsection*{(720 ILCS 5/2-12.1)}\label{ilcs-52-12.1}}
\addcontentsline{toc}{subsection}{(720 ILCS 5/2-12.1)}

\hypertarget{sec.-2-12.1.-part-day-child-care-facility.}{%
\section*{Sec. 2-12.1. Part day child care
facility.}\label{sec.-2-12.1.-part-day-child-care-facility.}}
\addcontentsline{toc}{section}{Sec. 2-12.1. Part day child care
facility.}

\markright{Sec. 2-12.1. Part day child care facility.}

``Part day child care facility'' has the meaning ascribed to it in
Section 2.10 of the Child Care Act of 1969.

(Source: P.A. 96-556, eff. 1-1-10.)

\hypertarget{ilcs-52-13-from-ch.-38-par.-2-13}{%
\subsection*{(720 ILCS 5/2-13) (from Ch. 38, par.
2-13)}\label{ilcs-52-13-from-ch.-38-par.-2-13}}
\addcontentsline{toc}{subsection}{(720 ILCS 5/2-13) (from Ch. 38, par.
2-13)}

\hypertarget{sec.-2-13.-peace-officer.}{%
\section*{Sec. 2-13. ``Peace
officer''.}\label{sec.-2-13.-peace-officer.}}
\addcontentsline{toc}{section}{Sec. 2-13. ``Peace officer''.}

\markright{Sec. 2-13. ``Peace officer''.}

``Peace officer'' means (i) any person who by virtue of his office or
public employment is vested by law with a duty to maintain public order
or to make arrests for offenses, whether that duty extends to all
offenses or is limited to specific offenses, or (ii) any person who, by
statute, is granted and authorized to exercise powers similar to those
conferred upon any peace officer employed by a law enforcement agency of
this State.

For purposes of Sections concerning unlawful use of weapons, for the
purposes of assisting an Illinois peace officer in an arrest, or when
the commission of any offense under Illinois law is directly observed by
the person, and statutes involving the false personation of a peace
officer, false personation of a peace officer while carrying a deadly
weapon, false personation of a peace officer in attempting or committing
a felony, and false personation of a peace officer in attempting or
committing a forcible felony, then officers, agents, or employees of the
federal government commissioned by federal statute to make arrests for
violations of federal criminal laws shall be considered ``peace
officers'' under this Code, including, but not limited to, all criminal
investigators of:

(1) the United States Department of Justice, the Federal Bureau of
Investigation, and the Drug Enforcement Administration and all United
States Marshals or Deputy United States Marshals whose duties involve
the enforcement of federal criminal laws;

(1.5) the United States Department of Homeland Security, United States
Citizenship and Immigration Services, United States Coast Guard, United
States Customs and Border Protection, and United States Immigration and
Customs Enforcement;

(2) the United States Department of the Treasury, the Alcohol and
Tobacco Tax and Trade Bureau, and the United States Secret Service;

(3) the United States Internal Revenue Service;

(4) the United States General Services Administration;

(5) the United States Postal Service;

(6) (blank); and

(7) the United States Department of Defense.

(Source: P.A. 102-558, eff. 8-20-21.)

\hypertarget{ilcs-52-14-from-ch.-38-par.-2-14}{%
\subsection*{(720 ILCS 5/2-14) (from Ch. 38, par.
2-14)}\label{ilcs-52-14-from-ch.-38-par.-2-14}}
\addcontentsline{toc}{subsection}{(720 ILCS 5/2-14) (from Ch. 38, par.
2-14)}

\hypertarget{sec.-2-14.-penal-institution.}{%
\section*{Sec. 2-14. ``Penal
institution''.}\label{sec.-2-14.-penal-institution.}}
\addcontentsline{toc}{section}{Sec. 2-14. ``Penal institution''.}

\markright{Sec. 2-14. ``Penal institution''.}

``Penal institution'' means a penitentiary, state farm, reformatory,
prison, jail, house of correction, or other institution for the
incarceration or custody of persons under sentence for offenses or
awaiting trial or sentence for offenses.

(Source: Laws 1961, p.~1983.)

\hypertarget{ilcs-52-15-from-ch.-38-par.-2-15}{%
\subsection*{(720 ILCS 5/2-15) (from Ch. 38, par.
2-15)}\label{ilcs-52-15-from-ch.-38-par.-2-15}}
\addcontentsline{toc}{subsection}{(720 ILCS 5/2-15) (from Ch. 38, par.
2-15)}

\hypertarget{sec.-2-15.-person.}{%
\section*{Sec. 2-15. ``Person''.}\label{sec.-2-15.-person.}}
\addcontentsline{toc}{section}{Sec. 2-15. ``Person''.}

\markright{Sec. 2-15. ``Person''.}

``Person'' means an individual, natural person, public or private
corporation, government, partnership, unincorporated association, or
other entity.

(Source: P.A. 97-597, eff. 1-1-12.)

\hypertarget{ilcs-52-15.5}{%
\subsection*{(720 ILCS 5/2-15.5)}\label{ilcs-52-15.5}}
\addcontentsline{toc}{subsection}{(720 ILCS 5/2-15.5)}

\hypertarget{sec.-2-15.5.-personally-discharged-a-firearm.}{%
\section*{Sec. 2-15.5. ``Personally discharged a
firearm''.}\label{sec.-2-15.5.-personally-discharged-a-firearm.}}
\addcontentsline{toc}{section}{Sec. 2-15.5. ``Personally discharged a
firearm''.}

\markright{Sec. 2-15.5. ``Personally discharged a firearm''.}

A person is considered to have ``personally discharged a firearm'' when
he or she, while armed with a firearm, knowingly and intentionally fires
a firearm causing the ammunition projectile to be forcefully expelled
from the firearm.

(Source: P.A. 91-404, eff. 1-1-00.)

\hypertarget{ilcs-52-15a-from-ch.-38-par.-2-15a}{%
\subsection*{(720 ILCS 5/2-15a) (from Ch. 38, par.
2-15a)}\label{ilcs-52-15a-from-ch.-38-par.-2-15a}}
\addcontentsline{toc}{subsection}{(720 ILCS 5/2-15a) (from Ch. 38, par.
2-15a)}

\hypertarget{sec.-2-15a.-person-with-a-physical-disability.}{%
\section*{Sec. 2-15a. ``Person with a physical
disability''.}\label{sec.-2-15a.-person-with-a-physical-disability.}}
\addcontentsline{toc}{section}{Sec. 2-15a. ``Person with a physical
disability''.}

\markright{Sec. 2-15a. ``Person with a physical disability''.}

``Person with a physical disability'' means a person who suffers from a
permanent and disabling physical characteristic, resulting from disease,
injury, functional disorder, or congenital condition.

(Source: P.A. 99-143, eff. 7-27-15.)

\hypertarget{ilcs-52-15b}{%
\subsection*{(720 ILCS 5/2-15b)}\label{ilcs-52-15b}}
\addcontentsline{toc}{subsection}{(720 ILCS 5/2-15b)}

\hypertarget{sec.-2-15b.}{%
\section*{Sec. 2-15b.}\label{sec.-2-15b.}}
\addcontentsline{toc}{section}{Sec. 2-15b.}

\markright{Sec. 2-15b.}

``Place of worship'' means a church, synagogue, mosque, temple, or other
building, structure, or place used primarily for religious worship and
includes the grounds of a place of worship.

(Source: P.A. 91-360, eff. 7-29-99.)

\hypertarget{ilcs-52-16-from-ch.-38-par.-2-16}{%
\subsection*{(720 ILCS 5/2-16) (from Ch. 38, par.
2-16)}\label{ilcs-52-16-from-ch.-38-par.-2-16}}
\addcontentsline{toc}{subsection}{(720 ILCS 5/2-16) (from Ch. 38, par.
2-16)}

\hypertarget{sec.-2-16.-prosecution.}{%
\section*{Sec. 2-16. ``Prosecution''.}\label{sec.-2-16.-prosecution.}}
\addcontentsline{toc}{section}{Sec. 2-16. ``Prosecution''.}

\markright{Sec. 2-16. ``Prosecution''.}

``Prosecution'' means all legal proceedings by which a person's
liability for an offense is determined, commencing with the return of
the indictment or the issuance of the information, and including the
final disposition of the case upon appeal.

(Source: Laws 1961, p.~1983.)

\hypertarget{ilcs-52-17-from-ch.-38-par.-2-17}{%
\subsection*{(720 ILCS 5/2-17) (from Ch. 38, par.
2-17)}\label{ilcs-52-17-from-ch.-38-par.-2-17}}
\addcontentsline{toc}{subsection}{(720 ILCS 5/2-17) (from Ch. 38, par.
2-17)}

\hypertarget{sec.-2-17.-public-employee.}{%
\section*{Sec. 2-17. ``Public
employee''.}\label{sec.-2-17.-public-employee.}}
\addcontentsline{toc}{section}{Sec. 2-17. ``Public employee''.}

\markright{Sec. 2-17. ``Public employee''.}

``Public employee'' means a person, other than a public officer, who is
authorized to perform any official function on behalf of, and is paid
by, the State or any of its political subdivisions.

(Source: Laws 1961, p.~1983.)

\hypertarget{ilcs-52-18-from-ch.-38-par.-2-18}{%
\subsection*{(720 ILCS 5/2-18) (from Ch. 38, par.
2-18)}\label{ilcs-52-18-from-ch.-38-par.-2-18}}
\addcontentsline{toc}{subsection}{(720 ILCS 5/2-18) (from Ch. 38, par.
2-18)}

\hypertarget{sec.-2-18.-public-officer.}{%
\section*{Sec. 2-18. ``Public
officer''.}\label{sec.-2-18.-public-officer.}}
\addcontentsline{toc}{section}{Sec. 2-18. ``Public officer''.}

\markright{Sec. 2-18. ``Public officer''.}

``Public officer'' means a person who is elected to office pursuant to
statute, or who is appointed to an office which is established, and the
qualifications and duties of which are prescribed, by statute, to
discharge a public duty for the State or any of its political
subdivisions.

(Source: Laws 1961, p.~1983.)

\hypertarget{ilcs-52-19-from-ch.-38-par.-2-19}{%
\subsection*{(720 ILCS 5/2-19) (from Ch. 38, par.
2-19)}\label{ilcs-52-19-from-ch.-38-par.-2-19}}
\addcontentsline{toc}{subsection}{(720 ILCS 5/2-19) (from Ch. 38, par.
2-19)}

\hypertarget{sec.-2-19.-reasonable-belief.}{%
\section*{Sec. 2-19. ``Reasonable
belief''.}\label{sec.-2-19.-reasonable-belief.}}
\addcontentsline{toc}{section}{Sec. 2-19. ``Reasonable belief''.}

\markright{Sec. 2-19. ``Reasonable belief''.}

``Reasonable belief'' or ``reasonably believes'' means that the person
concerned, acting as a reasonable man, believes that the described facts
exist.

(Source: Laws 1961, p.~1983.)

\hypertarget{ilcs-52-19.5}{%
\subsection*{(720 ILCS 5/2-19.5)}\label{ilcs-52-19.5}}
\addcontentsline{toc}{subsection}{(720 ILCS 5/2-19.5)}

\hypertarget{sec.-2-19.5.-school-means-a-public-private-or-parochial-elementary-or-secondary-school-community-college-college-or-university-and-includes-the-grounds-of-a-school.}{%
\section*{Sec. 2-19.5. ``School'' means a public, private, or parochial
elementary or secondary school, community college, college, or
university and includes the grounds of a
school.}\label{sec.-2-19.5.-school-means-a-public-private-or-parochial-elementary-or-secondary-school-community-college-college-or-university-and-includes-the-grounds-of-a-school.}}
\addcontentsline{toc}{section}{Sec. 2-19.5. ``School'' means a public,
private, or parochial elementary or secondary school, community college,
college, or university and includes the grounds of a school.}

\markright{Sec. 2-19.5. ``School'' means a public, private, or parochial
elementary or secondary school, community college, college, or
university and includes the grounds of a school.}

(Source: P.A. 91-360, eff. 7-29-99.)

\hypertarget{ilcs-52-20-from-ch.-38-par.-2-20}{%
\subsection*{(720 ILCS 5/2-20) (from Ch. 38, par.
2-20)}\label{ilcs-52-20-from-ch.-38-par.-2-20}}
\addcontentsline{toc}{subsection}{(720 ILCS 5/2-20) (from Ch. 38, par.
2-20)}

\hypertarget{sec.-2-20.-solicit.}{%
\section*{Sec. 2-20. ``Solicit''.}\label{sec.-2-20.-solicit.}}
\addcontentsline{toc}{section}{Sec. 2-20. ``Solicit''.}

\markright{Sec. 2-20. ``Solicit''.}

``Solicit'' or ``solicitation'' means to command, authorize, urge,
incite, request, or advise another to commit an offense.

(Source: Laws 1961, p.~1983.)

\hypertarget{ilcs-52-21-from-ch.-38-par.-2-21}{%
\subsection*{(720 ILCS 5/2-21) (from Ch. 38, par.
2-21)}\label{ilcs-52-21-from-ch.-38-par.-2-21}}
\addcontentsline{toc}{subsection}{(720 ILCS 5/2-21) (from Ch. 38, par.
2-21)}

\hypertarget{sec.-2-21.-state.}{%
\section*{Sec. 2-21. ``State''.}\label{sec.-2-21.-state.}}
\addcontentsline{toc}{section}{Sec. 2-21. ``State''.}

\markright{Sec. 2-21. ``State''.}

``State'' or ``this State'' means the State of Illinois, and all land
and water in respect to which the State of Illinois has either exclusive
or concurrent jurisdiction, and the air space above such land and water.
``Other state'' means any state or territory of the United States, the
District of Columbia and the Commonwealth of Puerto Rico.

(Source: Laws 1961, p.~1983.)

\hypertarget{ilcs-52-22-from-ch.-38-par.-2-22}{%
\subsection*{(720 ILCS 5/2-22) (from Ch. 38, par.
2-22)}\label{ilcs-52-22-from-ch.-38-par.-2-22}}
\addcontentsline{toc}{subsection}{(720 ILCS 5/2-22) (from Ch. 38, par.
2-22)}

\hypertarget{sec.-2-22.-statute.}{%
\section*{Sec. 2-22. ``Statute''.}\label{sec.-2-22.-statute.}}
\addcontentsline{toc}{section}{Sec. 2-22. ``Statute''.}

\markright{Sec. 2-22. ``Statute''.}

``Statute'' means the Constitution or an Act of the General Assembly of
this State.

(Source: Laws 1961, p.~1983.)

\bookmarksetup{startatroot}

\hypertarget{article-3.-rights-of-defendant}{%
\chapter*{Article 3. Rights Of
Defendant}\label{article-3.-rights-of-defendant}}
\addcontentsline{toc}{chapter}{Article 3. Rights Of Defendant}

\markboth{Article 3. Rights Of Defendant}{Article 3. Rights Of
Defendant}

\hypertarget{ilcs-53-1-from-ch.-38-par.-3-1}{%
\subsection*{(720 ILCS 5/3-1) (from Ch. 38, par.
3-1)}\label{ilcs-53-1-from-ch.-38-par.-3-1}}
\addcontentsline{toc}{subsection}{(720 ILCS 5/3-1) (from Ch. 38, par.
3-1)}

\hypertarget{sec.-3-1.-presumption-of-innocence-and-proof-of-guilt.}{%
\section*{Sec. 3-1. Presumption of innocence and proof of
guilt.}\label{sec.-3-1.-presumption-of-innocence-and-proof-of-guilt.}}
\addcontentsline{toc}{section}{Sec. 3-1. Presumption of innocence and
proof of guilt.}

\markright{Sec. 3-1. Presumption of innocence and proof of guilt.}

Every person is presumed innocent until proved guilty. No person shall
be convicted of any offense unless his guilt thereof is proved beyond a
reasonable doubt.

(Source: Laws 1961, p.~1983 .)

\hypertarget{ilcs-53-2-from-ch.-38-par.-3-2}{%
\subsection*{(720 ILCS 5/3-2) (from Ch. 38, par.
3-2)}\label{ilcs-53-2-from-ch.-38-par.-3-2}}
\addcontentsline{toc}{subsection}{(720 ILCS 5/3-2) (from Ch. 38, par.
3-2)}

\hypertarget{sec.-3-2.-affirmative-defense.}{%
\section*{Sec. 3-2. Affirmative
defense.}\label{sec.-3-2.-affirmative-defense.}}
\addcontentsline{toc}{section}{Sec. 3-2. Affirmative defense.}

\markright{Sec. 3-2. Affirmative defense.}

(a) ``Affirmative defense'' means that unless the State's evidence
raises the issue involving the alleged defense, the defendant, to raise
the issue, must present some evidence thereon.

(b) If the issue involved in an affirmative defense, other than
insanity, is raised then the State must sustain the burden of proving
the defendant guilty beyond a reasonable doubt as to that issue together
with all the other elements of the offense. If the affirmative defense
of insanity is raised, the defendant bears the burden of proving by
clear and convincing evidence his insanity at the time of the offense.

(Source: P.A. 89-404, eff. 8-20-95; 90-593, eff. 6-19-98.)

\hypertarget{ilcs-53-3-from-ch.-38-par.-3-3}{%
\subsection*{(720 ILCS 5/3-3) (from Ch. 38, par.
3-3)}\label{ilcs-53-3-from-ch.-38-par.-3-3}}
\addcontentsline{toc}{subsection}{(720 ILCS 5/3-3) (from Ch. 38, par.
3-3)}

\hypertarget{sec.-3-3.-multiple-prosecutions-for-same-act.}{%
\section*{Sec. 3-3. Multiple prosecutions for same
act.}\label{sec.-3-3.-multiple-prosecutions-for-same-act.}}
\addcontentsline{toc}{section}{Sec. 3-3. Multiple prosecutions for same
act.}

\markright{Sec. 3-3. Multiple prosecutions for same act.}

(a) When the same conduct of a defendant may establish the commission of
more than one offense, the defendant may be prosecuted for each such
offense.

(b) If the several offenses are known to the proper prosecuting officer
at the time of commencing the prosecution and are within the
jurisdiction of a single court, they must be prosecuted in a single
prosecution, except as provided in Subsection (c), if they are based on
the same act.

(c) When 2 or more offenses are charged as required by Subsection (b),
the court in the interest of justice may order that one or more of such
charges shall be tried separately.

(Source: Laws 1961, p.~1983.)

\hypertarget{ilcs-53-4-from-ch.-38-par.-3-4}{%
\subsection*{(720 ILCS 5/3-4) (from Ch. 38, par.
3-4)}\label{ilcs-53-4-from-ch.-38-par.-3-4}}
\addcontentsline{toc}{subsection}{(720 ILCS 5/3-4) (from Ch. 38, par.
3-4)}

\hypertarget{sec.-3-4.-effect-of-former-prosecution.}{%
\section*{Sec. 3-4. Effect of former
prosecution.}\label{sec.-3-4.-effect-of-former-prosecution.}}
\addcontentsline{toc}{section}{Sec. 3-4. Effect of former prosecution.}

\markright{Sec. 3-4. Effect of former prosecution.}

(a) A prosecution is barred if the defendant was formerly prosecuted for
the same offense, based upon the same facts, if that former prosecution:

(1) resulted in either a conviction or an acquittal or in a
determination that the evidence was insufficient to warrant a
conviction;

(2) was terminated by a final order or judgment, even if entered before
trial, that required a determination inconsistent with any fact or legal
proposition necessary to a conviction in the subsequent prosecution; or

(3) was terminated improperly after the jury was impaneled and sworn or,
in a trial before a court without a jury, after the first witness was
sworn but before findings were rendered by the trier of facts, or after
a plea of guilty was accepted by the court.

A conviction of an included offense, other than through a plea of
guilty, is an acquittal of the offense charged.

(b) A prosecution is barred if the defendant was formerly prosecuted for
a different offense, or for the same offense based upon different facts,
if that former prosecution:

(1) resulted in either a conviction or an acquittal, and the subsequent
prosecution is for an offense of which the defendant could have been
convicted on the former prosecution; or was for an offense with which
the defendant should have been charged on the former prosecution, as
provided in Section 3-3 of this Code (unless the court ordered a
separate trial of that charge); or was for an offense that involves the
same conduct, unless each prosecution requires proof of a fact not
required on the other prosecution, or the offense was not consummated
when the former trial began;

(2) was terminated by a final order or judgment, even if entered before
trial, that required a determination inconsistent with any fact
necessary to a conviction in the subsequent prosecution; or

(3) was terminated improperly under the circumstances stated in
subsection (a), and the subsequent prosecution is for an offense of
which the defendant could have been convicted if the former prosecution
had not been terminated improperly.

(c) A prosecution is barred if the defendant was formerly prosecuted in
a District Court of the United States or in a sister state for an
offense that is within the concurrent jurisdiction of this State, if
that former prosecution:

(1) resulted in either a conviction or an acquittal, and the subsequent
prosecution is for the same conduct, unless each prosecution requires
proof of a fact not required in the other prosecution, or the offense
was not consummated when the former trial began; or

(2) was terminated by a final order or judgment, even if entered before
trial, that required a determination inconsistent with any fact
necessary to a conviction in the prosecution in this State.

(d) A prosecution is not barred within the meaning of this Section 3-4,
however, if the former prosecution:

(1) was before a court that lacked jurisdiction over the defendant or
the offense; or

(2) was procured by the defendant without the knowledge of the proper
prosecuting officer, and with the purpose of avoiding the sentence that
otherwise might be imposed; or if subsequent proceedings resulted in the
invalidation, setting aside, reversal, or vacating of the conviction,
unless the defendant was thereby adjudged not guilty.

(Source: P.A. 96-710, eff. 1-1-10.)

\hypertarget{ilcs-53-5-from-ch.-38-par.-3-5}{%
\subsection*{(720 ILCS 5/3-5) (from Ch. 38, par.
3-5)}\label{ilcs-53-5-from-ch.-38-par.-3-5}}
\addcontentsline{toc}{subsection}{(720 ILCS 5/3-5) (from Ch. 38, par.
3-5)}

(Text of Section before amendment by P.A. 102-982

)

\hypertarget{sec.-3-5.-general-limitations.}{%
\section*{Sec. 3-5. General
limitations.}\label{sec.-3-5.-general-limitations.}}
\addcontentsline{toc}{section}{Sec. 3-5. General limitations.}

\markright{Sec. 3-5. General limitations.}

(a) A prosecution for: (1) first degree murder, attempt to commit first
degree murder, second degree murder, involuntary manslaughter, reckless
homicide, a violation of subparagraph (F) of paragraph (1) of subsection
(d) of Section 11-501 of the Illinois Vehicle Code for the offense of
aggravated driving under the influence of alcohol, other drug or drugs,
or intoxicating compound or compounds, or any combination thereof when
the violation was a proximate cause of a death, leaving the scene of a
motor vehicle accident involving death or personal injuries under
Section 11-401 of the Illinois Vehicle Code, failing to give information
and render aid under Section 11-403 of the Illinois Vehicle Code,
concealment of homicidal death, treason, arson, residential arson,
aggravated arson, forgery, child pornography under paragraph (1) of
subsection (a) of Section 11-20.1, or aggravated child pornography under
paragraph (1) of subsection (a) of Section 11-20.1B, or (2) any offense
involving sexual conduct or sexual penetration, as defined by Section
11-0.1 of this Code may be commenced at any time.

(a-5) A prosecution for theft of property exceeding \$100,000 in value
under Section 16-1, identity theft under subsection (a) of Section
16-30, aggravated identity theft under subsection (b) of Section 16-30,
financial exploitation of an elderly person or a person with a
disability under Section 17-56; theft by deception of a victim 60 years
of age or older or a person with a disability under Section 16-1; or any
offense set forth in Article 16H or Section 17-10.6 may be commenced
within 7 years of the last act committed in furtherance of the crime.

(b) Unless the statute describing the offense provides otherwise, or the
period of limitation is extended by Section 3-6, a prosecution for any
offense not designated in subsection (a) or (a-5) must be commenced
within 3 years after the commission of the offense if it is a felony, or
within one year and 6 months after its commission if it is a
misdemeanor.

(Source: P.A. 101-130, eff. 1-1-20; 102-244, eff. 1-1-22 .)

(Text of Section after amendment by P.A. 102-982

)

\hypertarget{sec.-3-5.-general-limitations.-1}{%
\section*{Sec. 3-5. General
limitations.}\label{sec.-3-5.-general-limitations.-1}}
\addcontentsline{toc}{section}{Sec. 3-5. General limitations.}

\markright{Sec. 3-5. General limitations.}

(a) A prosecution for: (1) first degree murder, attempt to commit first
degree murder, second degree murder, involuntary manslaughter, reckless
homicide, a violation of subparagraph (F) of paragraph (1) of subsection
(d) of Section 11-501 of the Illinois Vehicle Code for the offense of
aggravated driving under the influence of alcohol, other drug or drugs,
or intoxicating compound or compounds, or any combination thereof when
the violation was a proximate cause of a death, leaving the scene of a
motor vehicle crash involving death or personal injuries under Section
11-401 of the Illinois Vehicle Code, failing to give information and
render aid under Section 11-403 of the Illinois Vehicle Code,
concealment of homicidal death, treason, arson, residential arson,
aggravated arson, forgery, child pornography under paragraph (1) of
subsection (a) of Section 11-20.1, or aggravated child pornography under
paragraph (1) of subsection (a) of Section 11-20.1B, or (2) any offense
involving sexual conduct or sexual penetration, as defined by Section
11-0.1 of this Code may be commenced at any time.

(a-5) A prosecution for theft of property exceeding \$100,000 in value
under Section 16-1, identity theft under subsection (a) of Section
16-30, aggravated identity theft under subsection (b) of Section 16-30,
financial exploitation of an elderly person or a person with a
disability under Section 17-56; theft by deception of a victim 60 years
of age or older or a person with a disability under Section 16-1; or any
offense set forth in Article 16H or Section 17-10.6 may be commenced
within 7 years of the last act committed in furtherance of the crime.

(b) Unless the statute describing the offense provides otherwise, or the
period of limitation is extended by Section 3-6, a prosecution for any
offense not designated in subsection (a) or (a-5) must be commenced
within 3 years after the commission of the offense if it is a felony, or
within one year and 6 months after its commission if it is a
misdemeanor.

(Source: P.A. 101-130, eff. 1-1-20; 102-244, eff. 1-1-22; 102-982, eff.
7-1-23.)

\hypertarget{ilcs-53-6-from-ch.-38-par.-3-6}{%
\subsection*{(720 ILCS 5/3-6) (from Ch. 38, par.
3-6)}\label{ilcs-53-6-from-ch.-38-par.-3-6}}
\addcontentsline{toc}{subsection}{(720 ILCS 5/3-6) (from Ch. 38, par.
3-6)}

\hypertarget{sec.-3-6.-extended-limitations.}{%
\section*{Sec. 3-6. Extended
limitations.}\label{sec.-3-6.-extended-limitations.}}
\addcontentsline{toc}{section}{Sec. 3-6. Extended limitations.}

\markright{Sec. 3-6. Extended limitations.}

The period within which a prosecution must be commenced under the
provisions of Section 3-5 or other applicable statute is extended under
the following conditions:

(a) A prosecution for theft involving a breach of a fiduciary obligation
to the aggrieved person may be commenced as follows:

(1) If the aggrieved person is a minor or a person under legal
disability, then during the minority or legal disability or within one
year after the termination thereof.

(2) In any other instance, within one year after the discovery of the
offense by an aggrieved person, or by a person who has legal capacity to
represent an aggrieved person or has a legal duty to report the offense,
and is not himself or herself a party to the offense; or in the absence
of such discovery, within one year after the proper prosecuting officer
becomes aware of the offense. However, in no such case is the period of
limitation so extended more than 3 years beyond the expiration of the
period otherwise applicable.

(b) A prosecution for any offense based upon misconduct in office by a
public officer or employee may be commenced within one year after
discovery of the offense by a person having a legal duty to report such
offense, or in the absence of such discovery, within one year after the
proper prosecuting officer becomes aware of the offense. However, in no
such case is the period of limitation so extended more than 3 years
beyond the expiration of the period otherwise applicable.

(b-5) When the victim is under 18 years of age at the time of the
offense, a prosecution for involuntary servitude, involuntary sexual
servitude of a minor, or trafficking in persons and related offenses
under Section 10-9 of this Code may be commenced within 25 years of the
victim attaining the age of 18 years.

(b-6) When the victim is 18 years of age or over at the time of the
offense, a prosecution for involuntary servitude, involuntary sexual
servitude of a minor, or trafficking in persons and related offenses
under Section 10-9 of this Code may be commenced within 25 years after
the commission of the offense.

(b-7) When the victim is under 18 years of age at the time of the
offense, a prosecution for female genital mutilation may be commenced at
any time.

(c) (Blank).

(d) A prosecution for child pornography, aggravated child pornography,
indecent solicitation of a child, soliciting for a juvenile prostitute,
juvenile pimping, exploitation of a child, or promoting juvenile
prostitution except for keeping a place of juvenile prostitution may be
commenced within one year of the victim attaining the age of 18 years.
However, in no such case shall the time period for prosecution expire
sooner than 3 years after the commission of the offense.

(e) Except as otherwise provided in subdivision (j), a prosecution for
any offense involving sexual conduct or sexual penetration, as defined
in Section 11-0.1 of this Code, where the defendant was within a
professional or fiduciary relationship or a purported professional or
fiduciary relationship with the victim at the time of the commission of
the offense may be commenced within one year after the discovery of the
offense by the victim.

(f) A prosecution for any offense set forth in Section 44 of the
Environmental Protection Act may be commenced within 5 years after the
discovery of such an offense by a person or agency having the legal duty
to report the offense or in the absence of such discovery, within 5
years after the proper prosecuting officer becomes aware of the offense.

(f-5) A prosecution for any offense set forth in Section 16-30 of this
Code may be commenced within 5 years after the discovery of the offense
by the victim of that offense.

(g) (Blank).

(h) (Blank).

(i) Except as otherwise provided in subdivision (j), a prosecution for
criminal sexual assault, aggravated criminal sexual assault, or
aggravated criminal sexual abuse may be commenced at any time. If the
victim consented to the collection of evidence using an Illinois State
Police Sexual Assault Evidence Collection Kit under the Sexual Assault
Survivors Emergency Treatment Act, it shall constitute reporting for
purposes of this Section.

Nothing in this subdivision (i) shall be construed to shorten a period
within which a prosecution must be commenced under any other provision
of this Section.

(i-5) A prosecution for armed robbery, home invasion, kidnapping, or
aggravated kidnaping may be commenced within 10 years of the commission
of the offense if it arises out of the same course of conduct and meets
the criteria under one of the offenses in subsection (i) of this
Section.

(j) (1) When the victim is under 18 years of age at the time of the
offense, a prosecution for criminal sexual assault, aggravated criminal
sexual assault, predatory criminal sexual assault of a child, aggravated
criminal sexual abuse, felony criminal sexual abuse, or female genital
mutilation may be commenced at any time.

(2) When in circumstances other than as described in paragraph (1) of
this subsection (j), when the victim is under 18 years of age at the
time of the offense, a prosecution for failure of a person who is
required to report an alleged or suspected commission of criminal sexual
assault, aggravated criminal sexual assault, predatory criminal sexual
assault of a child, aggravated criminal sexual abuse, or felony criminal
sexual abuse under the Abused and Neglected Child Reporting Act may be
commenced within 20 years after the child victim attains 18 years of
age.

(3) When the victim is under 18 years of age at the time of the offense,
a prosecution for misdemeanor criminal sexual abuse may be commenced
within 10 years after the child victim attains 18 years of age.

(4) Nothing in this subdivision (j) shall be construed to shorten a
period within which a prosecution must be commenced under any other
provision of this Section.

(j-5) A prosecution for armed robbery, home invasion, kidnapping, or
aggravated kidnaping may be commenced at any time if it arises out of
the same course of conduct and meets the criteria under one of the
offenses in subsection (j) of this Section.

(k) (Blank).

(l) A prosecution for any offense set forth in Section 26-4 of this Code
may be commenced within one year after the discovery of the offense by
the victim of that offense.

(l-5) A prosecution for any offense involving sexual conduct or sexual
penetration, as defined in Section 11-0.1 of this Code, in which the
victim was 18 years of age or older at the time of the offense, may be
commenced within one year after the discovery of the offense by the
victim when corroborating physical evidence is available. The charging
document shall state that the statute of limitations is extended under
this subsection (l-5) and shall state the circumstances justifying the
extension. Nothing in this subsection (l-5) shall be construed to
shorten a period within which a prosecution must be commenced under any
other provision of this Section or Section 3-5 of this Code.

(m) The prosecution shall not be required to prove at trial facts which
extend the general limitations in Section 3-5 of this Code when the
facts supporting extension of the period of general limitations are
properly pled in the charging document. Any challenge relating to the
extension of the general limitations period as defined in this Section
shall be exclusively conducted under Section 114-1 of the Code of
Criminal Procedure of 1963.

(n) A prosecution for any offense set forth in subsection (a), (b), or
(c) of Section 8A-3 or Section 8A-13 of the Illinois Public Aid Code, in
which the total amount of money involved is \$5,000 or more, including
the monetary value of food stamps and the value of commodities under
Section 16-1 of this Code may be commenced within 5 years of the last
act committed in furtherance of the offense.

(Source: P.A. 101-18, eff. 1-1-20; 101-81, eff. 7-12-19; 101-130, eff.
1-1-20; 101-285, eff. 1-1-20; 102-558, eff. 8-20-21.)

\hypertarget{ilcs-53-7-from-ch.-38-par.-3-7}{%
\subsection*{(720 ILCS 5/3-7) (from Ch. 38, par.
3-7)}\label{ilcs-53-7-from-ch.-38-par.-3-7}}
\addcontentsline{toc}{subsection}{(720 ILCS 5/3-7) (from Ch. 38, par.
3-7)}

\hypertarget{sec.-3-7.-periods-excluded-from-limitation.}{%
\section*{Sec. 3-7. Periods excluded from
limitation.}\label{sec.-3-7.-periods-excluded-from-limitation.}}
\addcontentsline{toc}{section}{Sec. 3-7. Periods excluded from
limitation.}

\markright{Sec. 3-7. Periods excluded from limitation.}

(a) The period within which a prosecution must be commenced does not
include any period in which:

(1) the defendant is not usually and publicly resident within this
State; or

(2) the defendant is a public officer and the offense charged is theft
of public funds while in public office; or

(3) a prosecution is pending against the defendant for the same conduct,
even if the indictment or information which commences the prosecution is
quashed or the proceedings thereon are set aside, or are reversed on
appeal; or

(4) a proceeding or an appeal from a proceeding relating to the quashing
or enforcement of a Grand Jury subpoena issued in connection with an
investigation of a violation of a criminal law of this State is pending.
However, the period within which a prosecution must be commenced
includes any period in which the State brings a proceeding or an appeal
from a proceeding specified in this paragraph (4); or

(5) a material witness is placed on active military duty or leave. In
this paragraph (5), ``material witness'' includes, but is not limited
to, the arresting officer, occurrence witness, or the alleged victim of
the offense; or

(6) the victim of unlawful force or threat of imminent bodily harm to
obtain information or a confession is incarcerated, and the victim's
incarceration, in whole or in part, is a consequence of the unlawful
force or threats; or

(7) the sexual assault evidence is collected and submitted to the
Illinois State Police until the completion of the analysis of the
submitted evidence.

(a-5) The prosecution shall not be required to prove at trial facts
establishing periods excluded from the general limitations in Section
3-5 of this Code when the facts supporting periods being excluded from
the general limitations are properly pled in the charging document. Any
challenge relating to periods of exclusion as defined in this Section
shall be exclusively conducted under Section 114-1 of the Code of
Criminal Procedure of 1963.

(b) For the purposes of this Section:

``Completion of the analysis of the submitted evidence'' means analysis
of the collected evidence and conducting of laboratory tests and the
comparison of the collected evidence with the genetic marker grouping
analysis information maintained by the Illinois State Police under
Section 5-4-3 of the Unified Code of Corrections and with the
information contained in the Federal Bureau of Investigation's National
DNA database.

``Sexual assault'' has the meaning ascribed to it in

Section 1a of the Sexual Assault Survivors Emergency Treatment Act.

``Sexual assault evidence'' has the meaning ascribed to it in Section 5
of the Sexual Assault Evidence Submission Act.

(Source: P.A. 102-538, eff. 8-20-21.)

\hypertarget{ilcs-53-8-from-ch.-38-par.-3-8}{%
\subsection*{(720 ILCS 5/3-8) (from Ch. 38, par.
3-8)}\label{ilcs-53-8-from-ch.-38-par.-3-8}}
\addcontentsline{toc}{subsection}{(720 ILCS 5/3-8) (from Ch. 38, par.
3-8)}

\hypertarget{sec.-3-8.-limitation-on-offense-based-on-series-of-acts.}{%
\section*{Sec. 3-8. Limitation on offense based on series of
acts.}\label{sec.-3-8.-limitation-on-offense-based-on-series-of-acts.}}
\addcontentsline{toc}{section}{Sec. 3-8. Limitation on offense based on
series of acts.}

\markright{Sec. 3-8. Limitation on offense based on series of acts.}

When an offense is based on a series of acts performed at different
times, the period of limitation prescribed by this Article starts at the
time when the last such act is committed.

(Source: Laws 1961, p.~1983.)

\hypertarget{ilcs-5tit.-ii-heading}{%
\subsection*{(720 ILCS 5/Tit. II heading)}\label{ilcs-5tit.-ii-heading}}
\addcontentsline{toc}{subsection}{(720 ILCS 5/Tit. II heading)}

TITLE II.

PRINCIPLES OF CRIMINAL LIABILITY

\bookmarksetup{startatroot}

\hypertarget{article-4.-criminal-act-and-mental-state}{%
\chapter*{Article 4. Criminal Act And Mental
State}\label{article-4.-criminal-act-and-mental-state}}
\addcontentsline{toc}{chapter}{Article 4. Criminal Act And Mental State}

\markboth{Article 4. Criminal Act And Mental State}{Article 4. Criminal
Act And Mental State}

\hypertarget{ilcs-54-1-from-ch.-38-par.-4-1}{%
\subsection*{(720 ILCS 5/4-1) (from Ch. 38, par.
4-1)}\label{ilcs-54-1-from-ch.-38-par.-4-1}}
\addcontentsline{toc}{subsection}{(720 ILCS 5/4-1) (from Ch. 38, par.
4-1)}

\hypertarget{sec.-4-1.-voluntary-act.}{%
\section*{Sec. 4-1. Voluntary act.}\label{sec.-4-1.-voluntary-act.}}
\addcontentsline{toc}{section}{Sec. 4-1. Voluntary act.}

\markright{Sec. 4-1. Voluntary act.}

A material element of every offense is a voluntary act, which includes
an omission to perform a duty which the law imposes on the offender and
which he is physically capable of performing.

(Source: Laws 1961, p.~1983.)

\hypertarget{ilcs-54-2-from-ch.-38-par.-4-2}{%
\subsection*{(720 ILCS 5/4-2) (from Ch. 38, par.
4-2)}\label{ilcs-54-2-from-ch.-38-par.-4-2}}
\addcontentsline{toc}{subsection}{(720 ILCS 5/4-2) (from Ch. 38, par.
4-2)}

\hypertarget{sec.-4-2.-possession-as-voluntary-act.}{%
\section*{Sec. 4-2. Possession as voluntary
act.}\label{sec.-4-2.-possession-as-voluntary-act.}}
\addcontentsline{toc}{section}{Sec. 4-2. Possession as voluntary act.}

\markright{Sec. 4-2. Possession as voluntary act.}

Possession is a voluntary act if the offender knowingly procured or
received the thing possessed, or was aware of his control thereof for a
sufficient time to have been able to terminate his possession.

(Source: Laws 1961, p.~1983.)

\hypertarget{ilcs-54-3-from-ch.-38-par.-4-3}{%
\subsection*{(720 ILCS 5/4-3) (from Ch. 38, par.
4-3)}\label{ilcs-54-3-from-ch.-38-par.-4-3}}
\addcontentsline{toc}{subsection}{(720 ILCS 5/4-3) (from Ch. 38, par.
4-3)}

\hypertarget{sec.-4-3.-mental-state.}{%
\section*{Sec. 4-3. Mental state.}\label{sec.-4-3.-mental-state.}}
\addcontentsline{toc}{section}{Sec. 4-3. Mental state.}

\markright{Sec. 4-3. Mental state.}

(a) A person is not guilty of an offense, other than an offense which
involves absolute liability, unless, with respect to each element
described by the statute defining the offense, he acts while having one
of the mental states described in Sections 4-4 through 4-7.

(b) If the statute defining an offense prescribed a particular mental
state with respect to the offense as a whole, without distinguishing
among the elements thereof, the prescribed mental state applies to each
such element. If the statute does not prescribe a particular mental
state applicable to an element of an offense (other than an offense
which involves absolute liability), any mental state defined in Sections
4-4, 4-5 or 4-6 is applicable.

(c) Knowledge that certain conduct constitutes an offense, or knowledge
of the existence, meaning, or application of the statute defining an
offense, is not an element of the offense unless the statute clearly
defines it as such.

(Source: Laws 1961, p.~1983 .)

\hypertarget{ilcs-54-4-from-ch.-38-par.-4-4}{%
\subsection*{(720 ILCS 5/4-4) (from Ch. 38, par.
4-4)}\label{ilcs-54-4-from-ch.-38-par.-4-4}}
\addcontentsline{toc}{subsection}{(720 ILCS 5/4-4) (from Ch. 38, par.
4-4)}

\hypertarget{sec.-4-4.-intent.}{%
\section*{Sec. 4-4. Intent.}\label{sec.-4-4.-intent.}}
\addcontentsline{toc}{section}{Sec. 4-4. Intent.}

\markright{Sec. 4-4. Intent.}

A person intends, or acts intentionally or with intent, to accomplish a
result or engage in conduct described by the statute defining the
offense, when his conscious objective or purpose is to accomplish that
result or engage in that conduct.

(Source: Laws 1961, p.~1983.)

\hypertarget{ilcs-54-5-from-ch.-38-par.-4-5}{%
\subsection*{(720 ILCS 5/4-5) (from Ch. 38, par.
4-5)}\label{ilcs-54-5-from-ch.-38-par.-4-5}}
\addcontentsline{toc}{subsection}{(720 ILCS 5/4-5) (from Ch. 38, par.
4-5)}

\hypertarget{sec.-4-5.-knowledge.}{%
\section*{Sec. 4-5. Knowledge.}\label{sec.-4-5.-knowledge.}}
\addcontentsline{toc}{section}{Sec. 4-5. Knowledge.}

\markright{Sec. 4-5. Knowledge.}

A person knows, or acts knowingly or with knowledge of:

(a) The nature or attendant circumstances of his or her conduct,
described by the statute defining the offense, when he or she is
consciously aware that his or her conduct is of that nature or that
those circumstances exist. Knowledge of a material fact includes
awareness of the substantial probability that the fact exists.

(b) The result of his or her conduct, described by the statute defining
the offense, when he or she is consciously aware that that result is
practically certain to be caused by his conduct.

Conduct performed knowingly or with knowledge is performed wilfully,
within the meaning of a statute using the term ``willfully'', unless the
statute clearly requires another meaning.

When the law provides that acting knowingly suffices to establish an
element of an offense, that element also is established if a person acts
intentionally.

(Source: P.A. 96-710, eff. 1-1-10.)

\hypertarget{ilcs-54-6-from-ch.-38-par.-4-6}{%
\subsection*{(720 ILCS 5/4-6) (from Ch. 38, par.
4-6)}\label{ilcs-54-6-from-ch.-38-par.-4-6}}
\addcontentsline{toc}{subsection}{(720 ILCS 5/4-6) (from Ch. 38, par.
4-6)}

\hypertarget{sec.-4-6.-recklessness.}{%
\section*{Sec. 4-6. Recklessness.}\label{sec.-4-6.-recklessness.}}
\addcontentsline{toc}{section}{Sec. 4-6. Recklessness.}

\markright{Sec. 4-6. Recklessness.}

A person is reckless or acts recklessly when that person consciously
disregards a substantial and unjustifiable risk that circumstances exist
or that a result will follow, described by the statute defining the
offense, and that disregard constitutes a gross deviation from the
standard of care that a reasonable person would exercise in the
situation. An act performed recklessly is performed wantonly, within the
meaning of a statute using the term ``wantonly'', unless the statute
clearly requires another meaning.

(Source: P.A. 96-710, eff. 1-1-10.)

\hypertarget{ilcs-54-7-from-ch.-38-par.-4-7}{%
\subsection*{(720 ILCS 5/4-7) (from Ch. 38, par.
4-7)}\label{ilcs-54-7-from-ch.-38-par.-4-7}}
\addcontentsline{toc}{subsection}{(720 ILCS 5/4-7) (from Ch. 38, par.
4-7)}

\hypertarget{sec.-4-7.-negligence.}{%
\section*{Sec. 4-7. Negligence.}\label{sec.-4-7.-negligence.}}
\addcontentsline{toc}{section}{Sec. 4-7. Negligence.}

\markright{Sec. 4-7. Negligence.}

A person is negligent, or acts negligently, when that person fails to be
aware of a substantial and unjustifiable risk that circumstances exist
or a result will follow, described by the statute defining the offense,
and that failure constitutes a substantial deviation from the standard
of care that a reasonable person would exercise in the situation.

(Source: P.A. 96-710, eff. 1-1-10.)

\hypertarget{ilcs-54-8-from-ch.-38-par.-4-8}{%
\subsection*{(720 ILCS 5/4-8) (from Ch. 38, par.
4-8)}\label{ilcs-54-8-from-ch.-38-par.-4-8}}
\addcontentsline{toc}{subsection}{(720 ILCS 5/4-8) (from Ch. 38, par.
4-8)}

\hypertarget{sec.-4-8.-ignorance-or-mistake.}{%
\section*{Sec. 4-8. Ignorance or
mistake.}\label{sec.-4-8.-ignorance-or-mistake.}}
\addcontentsline{toc}{section}{Sec. 4-8. Ignorance or mistake.}

\markright{Sec. 4-8. Ignorance or mistake.}

(a) A person's ignorance or mistake as to a matter of either fact or
law, except as provided in Section 4-3(c) above, is a defense if it
negatives the existence of the mental state which the statute prescribes
with respect to an element of the offense.

(b) A person's reasonable belief that his conduct does not constitute an
offense is a defense if:

(1) the offense is defined by an administrative regulation or order
which is not known to him and has not been published or otherwise made
reasonably available to him, and he could not have acquired such
knowledge by the exercise of due diligence pursuant to facts known to
him; or

(2) he acts in reliance upon a statute which later is determined to be
invalid; or

(3) he acts in reliance upon an order or opinion of an Illinois
Appellate or Supreme Court, or a United States appellate court later
overruled or reversed; or

(4) he acts in reliance upon an official interpretation of the statute,
regulation or order defining the offense, made by a public officer or
agency legally authorized to interpret such statute.

(c) Although a person's ignorance or mistake of fact or law, or
reasonable belief, described in this Section 4-8 is a defense to the
offense charged, he may be convicted of an included offense of which he
would be guilty if the fact or law were as he believed it to be.

(d) A defense based upon this Section 4-8 is an affirmative defense.

(Source: P.A. 98-463, eff. 8-16-13.)

\hypertarget{ilcs-54-9-from-ch.-38-par.-4-9}{%
\subsection*{(720 ILCS 5/4-9) (from Ch. 38, par.
4-9)}\label{ilcs-54-9-from-ch.-38-par.-4-9}}
\addcontentsline{toc}{subsection}{(720 ILCS 5/4-9) (from Ch. 38, par.
4-9)}

\hypertarget{sec.-4-9.-absolute-liability.}{%
\section*{Sec. 4-9. Absolute
liability.}\label{sec.-4-9.-absolute-liability.}}
\addcontentsline{toc}{section}{Sec. 4-9. Absolute liability.}

\markright{Sec. 4-9. Absolute liability.}

A person may be guilty of an offense without having, as to each element
thereof, one of the mental states described in Sections 4-4 through 4-7
if the offense is a misdemeanor which is not punishable by incarceration
or by a fine exceeding \$1,000, or the statute defining the offense
clearly indicates a legislative purpose to impose absolute liability for
the conduct described.

(Source: P.A. 96-1198, eff. 1-1-11.)

\bookmarksetup{startatroot}

\hypertarget{article-5.-parties-to-crime}{%
\chapter*{Article 5. Parties To
Crime}\label{article-5.-parties-to-crime}}
\addcontentsline{toc}{chapter}{Article 5. Parties To Crime}

\markboth{Article 5. Parties To Crime}{Article 5. Parties To Crime}

\hypertarget{ilcs-55-1-from-ch.-38-par.-5-1}{%
\subsection*{(720 ILCS 5/5-1) (from Ch. 38, par.
5-1)}\label{ilcs-55-1-from-ch.-38-par.-5-1}}
\addcontentsline{toc}{subsection}{(720 ILCS 5/5-1) (from Ch. 38, par.
5-1)}

\hypertarget{sec.-5-1.-accountability-for-conduct-of-another.}{%
\section*{Sec. 5-1. Accountability for conduct of
another.}\label{sec.-5-1.-accountability-for-conduct-of-another.}}
\addcontentsline{toc}{section}{Sec. 5-1. Accountability for conduct of
another.}

\markright{Sec. 5-1. Accountability for conduct of another.}

A person is responsible for conduct which is an element of an offense if
the conduct is either that of the person himself, or that of another and
he is legally accountable for such conduct as provided in Section 5-2,
or both.

(Source: Laws 1961, p.~1983 .)

\hypertarget{ilcs-55-2-from-ch.-38-par.-5-2}{%
\subsection*{(720 ILCS 5/5-2) (from Ch. 38, par.
5-2)}\label{ilcs-55-2-from-ch.-38-par.-5-2}}
\addcontentsline{toc}{subsection}{(720 ILCS 5/5-2) (from Ch. 38, par.
5-2)}

\hypertarget{sec.-5-2.-when-accountability-exists.}{%
\section*{Sec. 5-2. When accountability
exists.}\label{sec.-5-2.-when-accountability-exists.}}
\addcontentsline{toc}{section}{Sec. 5-2. When accountability exists.}

\markright{Sec. 5-2. When accountability exists.}

A person is legally accountable for the conduct of another when:

(a) having a mental state described by the statute defining the offense,
he or she causes another to perform the conduct, and the other person in
fact or by reason of legal incapacity lacks such a mental state;

(b) the statute defining the offense makes him or her so accountable; or

(c) either before or during the commission of an offense, and with the
intent to promote or facilitate that commission, he or she solicits,
aids, abets, agrees, or attempts to aid that other person in the
planning or commission of the offense.

When 2 or more persons engage in a common criminal design or agreement,
any acts in the furtherance of that common design committed by one party
are considered to be the acts of all parties to the common design or
agreement and all are equally responsible for the consequences of those
further acts. Mere presence at the scene of a crime does not render a
person accountable for an offense; a person's presence at the scene of a
crime, however, may be considered with other circumstances by the trier
of fact when determining accountability.

A person is not so accountable, however, unless the statute defining the
offense provides otherwise, if:

(1) he or she is a victim of the offense committed;

(2) the offense is so defined that his or her conduct was inevitably
incident to its commission; or

(3) before the commission of the offense, he or she terminates his or
her effort to promote or facilitate that commission and does one of the
following: (i) wholly deprives his or her prior efforts of effectiveness
in that commission, (ii) gives timely warning to the proper law
enforcement authorities, or (iii) otherwise makes proper effort to
prevent the commission of the offense.

(Source: P.A. 96-710, eff. 1-1-10.)

\hypertarget{ilcs-55-3-from-ch.-38-par.-5-3}{%
\subsection*{(720 ILCS 5/5-3) (from Ch. 38, par.
5-3)}\label{ilcs-55-3-from-ch.-38-par.-5-3}}
\addcontentsline{toc}{subsection}{(720 ILCS 5/5-3) (from Ch. 38, par.
5-3)}

\hypertarget{sec.-5-3.-separate-conviction-of-person-accountable.}{%
\section*{Sec. 5-3. Separate conviction of person
accountable.}\label{sec.-5-3.-separate-conviction-of-person-accountable.}}
\addcontentsline{toc}{section}{Sec. 5-3. Separate conviction of person
accountable.}

\markright{Sec. 5-3. Separate conviction of person accountable.}

A person who is legally accountable for the conduct of another which is
an element of an offense may be convicted upon proof that the offense
was committed and that he was so accountable, although the other person
claimed to have committed the offense has not been prosecuted or
convicted, or has been convicted of a different offense or degree of
offense, or is not amenable to justice, or has been acquitted.

(Source: Laws 1961, p.~1983.)

\hypertarget{ilcs-55-4-from-ch.-38-par.-5-4}{%
\subsection*{(720 ILCS 5/5-4) (from Ch. 38, par.
5-4)}\label{ilcs-55-4-from-ch.-38-par.-5-4}}
\addcontentsline{toc}{subsection}{(720 ILCS 5/5-4) (from Ch. 38, par.
5-4)}

\hypertarget{sec.-5-4.-responsibility-of-corporation.}{%
\section*{Sec. 5-4. Responsibility of
corporation.}\label{sec.-5-4.-responsibility-of-corporation.}}
\addcontentsline{toc}{section}{Sec. 5-4. Responsibility of corporation.}

\markright{Sec. 5-4. Responsibility of corporation.}

(a) A corporation may be prosecuted for the commission of an offense if,
but only if:

(1) The offense is a misdemeanor, or is defined by Sections 11-20,
11-20.1 or 24-1 of this Code, or Section 44 of the ``Environmental
Protection Act'', approved June 29, 1970, as amended or is defined by
another statute which clearly indicates a legislative purpose to impose
liability on a corporation; and an agent of the corporation performs the
conduct which is an element of the offense while acting within the scope
of his or her office or employment and in behalf of the corporation,
except that any limitation in the defining statute, concerning the
corporation's accountability for certain agents or under certain
circumstances, is applicable; or

(2) The commission of the offense is authorized, requested, commanded,
or performed, by the board of directors or by a high managerial agent
who is acting within the scope of his or her employment in behalf of the
corporation.

(b) A corporation's proof, by a preponderance of the evidence, that the
high managerial agent having supervisory responsibility over the conduct
which is the subject matter of the offense exercised due diligence to
prevent the commission of the offense, is a defense to a prosecution for
any offense to which Subsection (a) (1) refers, other than an offense
for which absolute liability is imposed. This Subsection is inapplicable
if the legislative purpose of the statute defining the offense is
inconsistent with the provisions of this Subsection.

(c) For the purpose of this Section:

(1) ``Agent'' means any director, officer, servant, employee, or other
person who is authorized to act in behalf of the corporation.

(2) ``High managerial agent'' means an officer of the corporation, or
any other agent who has a position of comparable authority for the
formulation of corporate policy or the supervision of subordinate
employees in a managerial capacity.

(Source: P.A. 85-1440.)

\hypertarget{ilcs-55-5-from-ch.-38-par.-5-5}{%
\subsection*{(720 ILCS 5/5-5) (from Ch. 38, par.
5-5)}\label{ilcs-55-5-from-ch.-38-par.-5-5}}
\addcontentsline{toc}{subsection}{(720 ILCS 5/5-5) (from Ch. 38, par.
5-5)}

\hypertarget{sec.-5-5.-accountability-for-conduct-of-corporation.}{%
\section*{Sec. 5-5. Accountability for conduct of
corporation.}\label{sec.-5-5.-accountability-for-conduct-of-corporation.}}
\addcontentsline{toc}{section}{Sec. 5-5. Accountability for conduct of
corporation.}

\markright{Sec. 5-5. Accountability for conduct of corporation.}

(a) A person is legally accountable for conduct which is an element of
an offense and which, in the name or in behalf of a corporation, he
performs or causes to be performed, to the same extent as if the conduct
were performed in his own name or behalf.

(b) An individual who has been convicted of an offense by reason of his
legal accountability for the conduct of a corporation is subject to the
punishment authorized by law for an individual upon conviction of such
offense, although only a lesser or different punishment is authorized
for the corporation.

(Source: Laws 1961, p.~1983.)

\bookmarksetup{startatroot}

\hypertarget{article-6.-responsibility}{%
\chapter*{Article 6. Responsibility}\label{article-6.-responsibility}}
\addcontentsline{toc}{chapter}{Article 6. Responsibility}

\markboth{Article 6. Responsibility}{Article 6. Responsibility}

\hypertarget{ilcs-56-1-from-ch.-38-par.-6-1}{%
\subsection*{(720 ILCS 5/6-1) (from Ch. 38, par.
6-1)}\label{ilcs-56-1-from-ch.-38-par.-6-1}}
\addcontentsline{toc}{subsection}{(720 ILCS 5/6-1) (from Ch. 38, par.
6-1)}

\hypertarget{sec.-6-1.-infancy.}{%
\section*{Sec. 6-1. Infancy.}\label{sec.-6-1.-infancy.}}
\addcontentsline{toc}{section}{Sec. 6-1. Infancy.}

\markright{Sec. 6-1. Infancy.}

No person shall be convicted of any offense unless he had attained his
13th birthday at the time the offense was committed.

(Source: Laws 1961, p.~1983.)

\hypertarget{ilcs-56-2-from-ch.-38-par.-6-2}{%
\subsection*{(720 ILCS 5/6-2) (from Ch. 38, par.
6-2)}\label{ilcs-56-2-from-ch.-38-par.-6-2}}
\addcontentsline{toc}{subsection}{(720 ILCS 5/6-2) (from Ch. 38, par.
6-2)}

\hypertarget{sec.-6-2.-insanity.}{%
\section*{Sec. 6-2. Insanity.}\label{sec.-6-2.-insanity.}}
\addcontentsline{toc}{section}{Sec. 6-2. Insanity.}

\markright{Sec. 6-2. Insanity.}

(a) A person is not criminally responsible for conduct if at the time of
such conduct, as a result of mental disease or mental defect, he lacks
substantial capacity to appreciate the criminality of his conduct.

(b) The terms ``mental disease or mental defect'' do not include an
abnormality manifested only by repeated criminal or otherwise antisocial
conduct.

(c) A person who, at the time of the commission of a criminal offense,
was not insane but was suffering from a mental illness, is not relieved
of criminal responsibility for his conduct and may be found guilty but
mentally ill.

(d) For purposes of this Section, ``mental illness'' or ``mentally ill''
means a substantial disorder of thought, mood, or behavior which
afflicted a person at the time of the commission of the offense and
which impaired that person's judgment, but not to the extent that he is
unable to appreciate the wrongfulness of his behavior.

(e) When the defense of insanity has been presented during the trial,
the burden of proof is on the defendant to prove by clear and convincing
evidence that the defendant is not guilty by reason of insanity.
However, the burden of proof remains on the State to prove beyond a
reasonable doubt each of the elements of each of the offenses charged,
and, in a jury trial where the insanity defense has been presented, the
jury must be instructed that it may not consider whether the defendant
has met his burden of proving that he is not guilty by reason of
insanity until and unless it has first determined that the State has
proven the defendant guilty beyond a reasonable doubt of the offense
with which he is charged.

(Source: P.A. 89-404, eff. 8-20-95; 90-593, eff. 6-19-98.)

\hypertarget{ilcs-56-3-from-ch.-38-par.-6-3}{%
\subsection*{(720 ILCS 5/6-3) (from Ch. 38, par.
6-3)}\label{ilcs-56-3-from-ch.-38-par.-6-3}}
\addcontentsline{toc}{subsection}{(720 ILCS 5/6-3) (from Ch. 38, par.
6-3)}

\hypertarget{sec.-6-3.-intoxicated-or-drugged-condition.}{%
\section*{Sec. 6-3. Intoxicated or drugged
condition.}\label{sec.-6-3.-intoxicated-or-drugged-condition.}}
\addcontentsline{toc}{section}{Sec. 6-3. Intoxicated or drugged
condition.}

\markright{Sec. 6-3. Intoxicated or drugged condition.}

A person who is in an intoxicated or drugged condition is criminally
responsible for conduct unless such condition is involuntarily produced
and deprives him of substantial capacity either to appreciate the
criminality of his conduct or to conform his conduct to the requirements
of law.

(Source: P.A. 92-466, eff. 1-1-02.)

\hypertarget{ilcs-56-4-from-ch.-38-par.-6-4}{%
\subsection*{(720 ILCS 5/6-4) (from Ch. 38, par.
6-4)}\label{ilcs-56-4-from-ch.-38-par.-6-4}}
\addcontentsline{toc}{subsection}{(720 ILCS 5/6-4) (from Ch. 38, par.
6-4)}

\hypertarget{sec.-6-4.-affirmative-defense.}{%
\section*{Sec. 6-4. Affirmative
Defense.}\label{sec.-6-4.-affirmative-defense.}}
\addcontentsline{toc}{section}{Sec. 6-4. Affirmative Defense.}

\markright{Sec. 6-4. Affirmative Defense.}

A defense based upon any of the provisions of Article 6 is an
affirmative defense except that mental illness is not an affirmative
defense, but an alternative plea or finding that may be accepted, under
appropriate evidence, when the affirmative defense of insanity is raised
or the plea of guilty but mentally ill is made.

(Source: P.A. 82-553.)

\bookmarksetup{startatroot}

\hypertarget{article-7.-justifiable-use-of-force-exoneration}{%
\chapter*{Article 7. Justifiable Use Of Force;
Exoneration}\label{article-7.-justifiable-use-of-force-exoneration}}
\addcontentsline{toc}{chapter}{Article 7. Justifiable Use Of Force;
Exoneration}

\markboth{Article 7. Justifiable Use Of Force; Exoneration}{Article 7.
Justifiable Use Of Force; Exoneration}

\hypertarget{ilcs-57-1-from-ch.-38-par.-7-1}{%
\subsection*{(720 ILCS 5/7-1) (from Ch. 38, par.
7-1)}\label{ilcs-57-1-from-ch.-38-par.-7-1}}
\addcontentsline{toc}{subsection}{(720 ILCS 5/7-1) (from Ch. 38, par.
7-1)}

\hypertarget{sec.-7-1.-use-of-force-in-defense-of-person.}{%
\section*{Sec. 7-1. Use of force in defense of
person.}\label{sec.-7-1.-use-of-force-in-defense-of-person.}}
\addcontentsline{toc}{section}{Sec. 7-1. Use of force in defense of
person.}

\markright{Sec. 7-1. Use of force in defense of person.}

(a) A person is justified in the use of force against another when and
to the extent that he reasonably believes that such conduct is necessary
to defend himself or another against such other's imminent use of
unlawful force. However, he is justified in the use of force which is
intended or likely to cause death or great bodily harm only if he
reasonably believes that such force is necessary to prevent imminent
death or great bodily harm to himself or another, or the commission of a
forcible felony.

(b) In no case shall any act involving the use of force justified under
this Section give rise to any claim or liability brought by or on behalf
of any person acting within the definition of ``aggressor'' set forth in
Section 7-4 of this Article, or the estate, spouse, or other family
member of such a person, against the person or estate of the person
using such justified force, unless the use of force involves willful or
wanton misconduct.

(Source: P.A. 93-832, eff. 7-28-04 .)

\hypertarget{ilcs-57-2-from-ch.-38-par.-7-2}{%
\subsection*{(720 ILCS 5/7-2) (from Ch. 38, par.
7-2)}\label{ilcs-57-2-from-ch.-38-par.-7-2}}
\addcontentsline{toc}{subsection}{(720 ILCS 5/7-2) (from Ch. 38, par.
7-2)}

\hypertarget{sec.-7-2.-use-of-force-in-defense-of-dwelling.}{%
\section*{Sec. 7-2. Use of force in defense of
dwelling.}\label{sec.-7-2.-use-of-force-in-defense-of-dwelling.}}
\addcontentsline{toc}{section}{Sec. 7-2. Use of force in defense of
dwelling.}

\markright{Sec. 7-2. Use of force in defense of dwelling.}

(a) A person is justified in the use of force against another when and
to the extent that he reasonably believes that such conduct is necessary
to prevent or terminate such other's unlawful entry into or attack upon
a dwelling. However, he is justified in the use of force which is
intended or likely to cause death or great bodily harm only if:

(1) The entry is made or attempted in a violent, riotous, or tumultuous
manner, and he reasonably believes that such force is necessary to
prevent an assault upon, or offer of personal violence to, him or
another then in the dwelling, or

(2) He reasonably believes that such force is necessary to prevent the
commission of a felony in the dwelling.

(b) In no case shall any act involving the use of force justified under
this Section give rise to any claim or liability brought by or on behalf
of any person acting within the definition of ``aggressor'' set forth in
Section 7-4 of this Article, or the estate, spouse, or other family
member of such a person, against the person or estate of the person
using such justified force, unless the use of force involves willful or
wanton misconduct.

(Source: P.A. 93-832, eff. 7-28-04.)

\hypertarget{ilcs-57-3-from-ch.-38-par.-7-3}{%
\subsection*{(720 ILCS 5/7-3) (from Ch. 38, par.
7-3)}\label{ilcs-57-3-from-ch.-38-par.-7-3}}
\addcontentsline{toc}{subsection}{(720 ILCS 5/7-3) (from Ch. 38, par.
7-3)}

\hypertarget{sec.-7-3.-use-of-force-in-defense-of-other-property.}{%
\section*{Sec. 7-3. Use of force in defense of other
property.}\label{sec.-7-3.-use-of-force-in-defense-of-other-property.}}
\addcontentsline{toc}{section}{Sec. 7-3. Use of force in defense of
other property.}

\markright{Sec. 7-3. Use of force in defense of other property.}

(a) A person is justified in the use of force against another when and
to the extent that he reasonably believes that such conduct is necessary
to prevent or terminate such other's trespass on or other tortious or
criminal interference with either real property (other than a dwelling)
or personal property, lawfully in his possession or in the possession of
another who is a member of his immediate family or household or of a
person whose property he has a legal duty to protect. However, he is
justified in the use of force which is intended or likely to cause death
or great bodily harm only if he reasonably believes that such force is
necessary to prevent the commission of a forcible felony.

(b) In no case shall any act involving the use of force justified under
this Section give rise to any claim or liability brought by or on behalf
of any person acting within the definition of ``aggressor'' set forth in
Section 7-4 of this Article, or the estate, spouse, or other family
member of such a person, against the person or estate of the person
using such justified force, unless the use of force involves willful or
wanton misconduct.

(Source: P.A. 93-832, eff. 7-28-04.)

\hypertarget{ilcs-57-4-from-ch.-38-par.-7-4}{%
\subsection*{(720 ILCS 5/7-4) (from Ch. 38, par.
7-4)}\label{ilcs-57-4-from-ch.-38-par.-7-4}}
\addcontentsline{toc}{subsection}{(720 ILCS 5/7-4) (from Ch. 38, par.
7-4)}

\hypertarget{sec.-7-4.-use-of-force-by-aggressor.}{%
\section*{Sec. 7-4. Use of force by
aggressor.}\label{sec.-7-4.-use-of-force-by-aggressor.}}
\addcontentsline{toc}{section}{Sec. 7-4. Use of force by aggressor.}

\markright{Sec. 7-4. Use of force by aggressor.}

The justification described in the preceding Sections of this Article is
not available to a person who:

(a) is attempting to commit, committing, or escaping after the
commission of, a forcible felony; or

(b) initially provokes the use of force against himself, with the intent
to use such force as an excuse to inflict bodily harm upon the
assailant; or

(c) otherwise initially provokes the use of force against himself,
unless:

(1) such force is so great that he reasonably believes that he is in
imminent danger of death or great bodily harm, and that he has exhausted
every reasonable means to escape such danger other than the use of force
which is likely to cause death or great bodily harm to the assailant; or

(2) in good faith, he withdraws from physical contact with the assailant
and indicates clearly to the assailant that he desires to withdraw and
terminate the use of force, but the assailant continues or resumes the
use of force.

(Source: Laws 1961, p.~1983 .)

\hypertarget{ilcs-57-5-from-ch.-38-par.-7-5}{%
\subsection*{(720 ILCS 5/7-5) (from Ch. 38, par.
7-5)}\label{ilcs-57-5-from-ch.-38-par.-7-5}}
\addcontentsline{toc}{subsection}{(720 ILCS 5/7-5) (from Ch. 38, par.
7-5)}

\hypertarget{sec.-7-5.-peace-officers-use-of-force-in-making-arrest.}{%
\section*{Sec. 7-5. Peace officer's use of force in making
arrest.}\label{sec.-7-5.-peace-officers-use-of-force-in-making-arrest.}}
\addcontentsline{toc}{section}{Sec. 7-5. Peace officer's use of force in
making arrest.}

\markright{Sec. 7-5. Peace officer's use of force in making arrest.}

(a) A peace officer, or any person whom he has summoned or directed to
assist him, need not retreat or desist from efforts to make a lawful
arrest because of resistance or threatened resistance to the arrest. He
is justified in the use of any force which he reasonably believes, based
on the totality of the circumstances, to be necessary to effect the
arrest and of any force which he reasonably believes, based on the
totality of the circumstances, to be necessary to defend himself or
another from bodily harm while making the arrest. However, he is
justified in using force likely to cause death or great bodily harm only
when: (i) he reasonably believes, based on the totality of the
circumstances, that such force is necessary to prevent death or great
bodily harm to himself or such other person; or (ii) when he reasonably
believes, based on the totality of the circumstances, both that:

(1) Such force is necessary to prevent the arrest from being defeated by
resistance or escape and the officer reasonably believes that the person
to be arrested is likely to cause great bodily harm to another; and

(2) The person to be arrested committed or attempted a forcible felony
which involves the infliction or threatened infliction of great bodily
harm or is attempting to escape by use of a deadly weapon, or otherwise
indicates that he will endanger human life or inflict great bodily harm
unless arrested without delay.

As used in this subsection, ``retreat'' does not mean tactical
repositioning or other de-escalation tactics.

A peace officer is not justified in using force likely to cause death or
great bodily harm when there is no longer an imminent threat of great
bodily harm to the officer or another.

(a-5) Where feasible, a peace officer shall, prior to the use of force,
make reasonable efforts to identify himself or herself as a peace
officer and to warn that deadly force may be used.

(a-10) A peace officer shall not use deadly force against a person based
on the danger that the person poses to himself or herself if a
reasonable officer would believe the person does not pose an imminent
threat of death or great bodily harm to the peace officer or to another
person.

(a-15) A peace officer shall not use deadly force against a person who
is suspected of committing a property offense, unless that offense is
terrorism or unless deadly force is otherwise authorized by law.

(b) A peace officer making an arrest pursuant to an invalid warrant is
justified in the use of any force which he would be justified in using
if the warrant were valid, unless he knows that the warrant is invalid.

(c) The authority to use physical force conferred on peace officers by
this Article is a serious responsibility that shall be exercised
judiciously and with respect for human rights and dignity and for the
sanctity of every human life.

(d) Peace officers shall use deadly force only when reasonably necessary
in defense of human life. In determining whether deadly force is
reasonably necessary, officers shall evaluate each situation in light of
the totality of circumstances of each case, including, but not limited
to, the proximity in time of the use of force to the commission of a
forcible felony, and the reasonable feasibility of safely apprehending a
subject at a later time, and shall use other available resources and
techniques, if reasonably safe and feasible to a reasonable officer.

(e) The decision by a peace officer to use force shall be evaluated
carefully and thoroughly, in a manner that reflects the gravity of that
authority and the serious consequences of the use of force by peace
officers, in order to ensure that officers use force consistent with law
and agency policies.

(f) The decision by a peace officer to use force shall be evaluated from
the perspective of a reasonable officer in the same situation, based on
the totality of the circumstances known to or perceived by the officer
at the time of the decision, rather than with the benefit of hindsight,
and that the totality of the circumstances shall account for occasions
when officers may be forced to make quick judgments about using force.

(g) Law enforcement agencies are encouraged to adopt and develop
policies designed to protect individuals with physical, mental health,
developmental, or intellectual disabilities, or individuals who are
significantly more likely to experience greater levels of physical force
during police interactions, as these disabilities may affect the ability
of a person to understand or comply with commands from peace officers.

(h) As used in this Section:

(1) ``Deadly force'' means any use of force that creates a substantial
risk of causing death or great bodily harm, including, but not limited
to, the discharge of a firearm.

(2) A threat of death or serious bodily injury is

``imminent'' when, based on the totality of the circumstances, a
reasonable officer in the same situation would believe that a person has
the present ability, opportunity, and apparent intent to immediately
cause death or great bodily harm to the peace officer or another person.
An imminent harm is not merely a fear of future harm, no matter how
great the fear and no matter how great the likelihood of the harm, but
is one that, from appearances, must be instantly confronted and
addressed.

(3) ``Totality of the circumstances'' means all facts known to the peace
officer at the time, or that would be known to a reasonable officer in
the same situation, including the conduct of the officer and the subject
leading up to the use of deadly force.

(Source: P.A. 101-652, eff. 7-1-21; 102-28, eff. 6-25-21; 102-687, eff.
12-17-21.)

\hypertarget{ilcs-57-5.5}{%
\subsection*{(720 ILCS 5/7-5.5)}\label{ilcs-57-5.5}}
\addcontentsline{toc}{subsection}{(720 ILCS 5/7-5.5)}

\hypertarget{sec.-7-5.5.-prohibited-use-of-force-by-a-peace-officer.}{%
\section*{Sec. 7-5.5. Prohibited use of force by a peace
officer.}\label{sec.-7-5.5.-prohibited-use-of-force-by-a-peace-officer.}}
\addcontentsline{toc}{section}{Sec. 7-5.5. Prohibited use of force by a
peace officer.}

\markright{Sec. 7-5.5. Prohibited use of force by a peace officer.}

(a) A peace officer, or any other person acting under the color of law,
shall not use a chokehold or restraint above the shoulders with risk of
asphyxiation in the performance of his or her duties, unless deadly
force is justified under this Article.

(b) A peace officer, or any other person acting under the color of law,
shall not use a chokehold or restraint above the shoulders with risk of
asphyxiation, or any lesser contact with the throat or neck area of
another, in order to prevent the destruction of evidence by ingestion.

(c) As used in this Section, ``chokehold'' means applying any direct
pressure to the throat, windpipe, or airway of another. ``Chokehold''
does not include any holding involving contact with the neck that is not
intended to reduce the intake of air such as a headlock where the only
pressure applied is to the head.

(d) As used in this Section, ``restraint above the shoulders with risk
of positional asphyxiation'' means a use of a technique used to restrain
a person above the shoulders, including the neck or head, in a position
which interferes with the person's ability to breathe after the person
no longer poses a threat to the officer or any other person.

(e) A peace officer, or any other person acting under the color of law,
shall not:

(i) use force as punishment or retaliation;

(ii) discharge kinetic impact projectiles and all other non-lethal or
less-lethal projectiles in a manner that targets the head, neck, groin,
anterior pelvis, or back;

(iii) discharge conducted electrical weapons in a manner that targets
the head, chest, neck, groin, or anterior pelvis;

(iv) discharge firearms or kinetic impact projectiles indiscriminately
into a crowd;

(v) use chemical agents or irritants for crowd control, including pepper
spray and tear gas, prior to issuing an order to disperse in a
sufficient manner to allow for the order to be heard and repeated if
necessary, followed by sufficient time and space to allow compliance
with the order unless providing such time and space would unduly place
an officer or another person at risk of death or great bodily harm; or

(vi) use chemical agents or irritants, including pepper spray and tear
gas, prior to issuing an order in a sufficient manner to ensure the
order is heard, and repeated if necessary, to allow compliance with the
order unless providing such time and space would unduly place an officer
or another person at risk of death or great bodily harm.

(Source: P.A. 101-652, eff. 7-1-21; 102-28, eff. 6-25-21; 102-687, eff.
12-17-21.)

\hypertarget{ilcs-57-6-from-ch.-38-par.-7-6}{%
\subsection*{(720 ILCS 5/7-6) (from Ch. 38, par.
7-6)}\label{ilcs-57-6-from-ch.-38-par.-7-6}}
\addcontentsline{toc}{subsection}{(720 ILCS 5/7-6) (from Ch. 38, par.
7-6)}

\hypertarget{sec.-7-6.-private-persons-use-of-force-in-making-arrest.}{%
\section*{Sec. 7-6. Private person's use of force in making
arrest.}\label{sec.-7-6.-private-persons-use-of-force-in-making-arrest.}}
\addcontentsline{toc}{section}{Sec. 7-6. Private person's use of force
in making arrest.}

\markright{Sec. 7-6. Private person's use of force in making arrest.}

(a) A private person who makes, or assists another private person in
making a lawful arrest is justified in the use of any force which he
would be justified in using if he were summoned or directed by a peace
officer to make such arrest, except that he is justified in the use of
force likely to cause death or great bodily harm only when he reasonably
believes that such force is necessary to prevent death or great bodily
harm to himself or another.

(b) A private person who is summoned or directed by a peace officer to
assist in making an arrest which is unlawful, is justified in the use of
any force which he would be justified in using if the arrest were
lawful, unless he knows that the arrest is unlawful.

(Source: Laws 1961, p.~1983.)

\hypertarget{ilcs-57-7-from-ch.-38-par.-7-7}{%
\subsection*{(720 ILCS 5/7-7) (from Ch. 38, par.
7-7)}\label{ilcs-57-7-from-ch.-38-par.-7-7}}
\addcontentsline{toc}{subsection}{(720 ILCS 5/7-7) (from Ch. 38, par.
7-7)}

\hypertarget{sec.-7-7.-private-persons-use-of-force-in-resisting-arrest.}{%
\section*{Sec. 7-7. Private person's use of force in resisting
arrest.}\label{sec.-7-7.-private-persons-use-of-force-in-resisting-arrest.}}
\addcontentsline{toc}{section}{Sec. 7-7. Private person's use of force
in resisting arrest.}

\markright{Sec. 7-7. Private person's use of force in resisting arrest.}

A person is not authorized to use force to resist an arrest which he
knows is being made either by a peace officer or by a private person
summoned and directed by a peace officer to make the arrest, even if he
believes that the arrest is unlawful and the arrest in fact is unlawful.

(Source: P.A. 86-1475.)

\hypertarget{ilcs-57-8-from-ch.-38-par.-7-8}{%
\subsection*{(720 ILCS 5/7-8) (from Ch. 38, par.
7-8)}\label{ilcs-57-8-from-ch.-38-par.-7-8}}
\addcontentsline{toc}{subsection}{(720 ILCS 5/7-8) (from Ch. 38, par.
7-8)}

\hypertarget{sec.-7-8.-force-likely-to-cause-death-or-great-bodily-harm.}{%
\section*{Sec. 7-8. Force likely to cause death or great bodily
harm.}\label{sec.-7-8.-force-likely-to-cause-death-or-great-bodily-harm.}}
\addcontentsline{toc}{section}{Sec. 7-8. Force likely to cause death or
great bodily harm.}

\markright{Sec. 7-8. Force likely to cause death or great bodily harm.}

(a) Force which is likely to cause death or great bodily harm, within
the meaning of Sections 7-5 and 7-6 includes:

(1) The firing of a firearm in the direction of the person to be
arrested, even though no intent exists to kill or inflict great bodily
harm; and

(2) The firing of a firearm at a vehicle in which the person to be
arrested is riding.

(b) A peace officer's discharge of a firearm using ammunition designed
to disable or control an individual without creating the likelihood of
death or great bodily harm shall not be considered force likely to cause
death or great bodily harm within the meaning of Sections 7-5 and 7-6.

(Source: P.A. 90-138, eff. 1-1-98.)

\hypertarget{ilcs-57-9-from-ch.-38-par.-7-9}{%
\subsection*{(720 ILCS 5/7-9) (from Ch. 38, par.
7-9)}\label{ilcs-57-9-from-ch.-38-par.-7-9}}
\addcontentsline{toc}{subsection}{(720 ILCS 5/7-9) (from Ch. 38, par.
7-9)}

\hypertarget{sec.-7-9.-use-of-force-to-prevent-escape.}{%
\section*{Sec. 7-9. Use of force to prevent
escape.}\label{sec.-7-9.-use-of-force-to-prevent-escape.}}
\addcontentsline{toc}{section}{Sec. 7-9. Use of force to prevent
escape.}

\markright{Sec. 7-9. Use of force to prevent escape.}

(a) A peace officer or other person who has an arrested person in his
custody is justified in the use of force, except deadly force, to
prevent the escape of the arrested person from custody as he would be
justified in using if he were arresting such person.

(b) A guard or other peace officer is justified in the use of force
which he reasonably believes to be necessary to prevent the escape from
a penal institution of a person whom the officer reasonably believes to
be lawfully detained in such institution under sentence for an offense
or awaiting trial or commitment for an offense.

(c) Deadly force shall not be used to prevent escape under this Section
unless, based on the totality of the circumstances, deadly force is
necessary to prevent death or great bodily harm to himself or such other
person.

(Source: P.A. 101-652, eff. 7-1-21 .)

\hypertarget{ilcs-57-10-from-ch.-38-par.-7-10}{%
\subsection*{(720 ILCS 5/7-10) (from Ch. 38, par.
7-10)}\label{ilcs-57-10-from-ch.-38-par.-7-10}}
\addcontentsline{toc}{subsection}{(720 ILCS 5/7-10) (from Ch. 38, par.
7-10)}

\hypertarget{sec.-7-10.-execution-of-death-sentence.}{%
\section*{Sec. 7-10. Execution of death
sentence.}\label{sec.-7-10.-execution-of-death-sentence.}}
\addcontentsline{toc}{section}{Sec. 7-10. Execution of death sentence.}

\markright{Sec. 7-10. Execution of death sentence.}

A public officer who, in the exercise of his official duty, puts a
person to death pursuant to a sentence of a court of competent
jurisdiction, is justified if he acts in accordance with the sentence
pronounced and the law prescribing the procedure for execution of a
death sentence.

(Source: Laws 1961, p.~1983.)

\hypertarget{ilcs-57-11-from-ch.-38-par.-7-11}{%
\subsection*{(720 ILCS 5/7-11) (from Ch. 38, par.
7-11)}\label{ilcs-57-11-from-ch.-38-par.-7-11}}
\addcontentsline{toc}{subsection}{(720 ILCS 5/7-11) (from Ch. 38, par.
7-11)}

\hypertarget{sec.-7-11.-compulsion.}{%
\section*{Sec. 7-11. Compulsion.}\label{sec.-7-11.-compulsion.}}
\addcontentsline{toc}{section}{Sec. 7-11. Compulsion.}

\markright{Sec. 7-11. Compulsion.}

(a) A person is not guilty of an offense, other than an offense
punishable with death, by reason of conduct that he or she performs
under the compulsion of threat or menace of the imminent infliction of
death or great bodily harm, if he or she reasonably believes death or
great bodily harm will be inflicted upon him or her, or upon his or her
spouse or child, if he or she does not perform that conduct.

(b) A married woman is not entitled, by reason of the presence of her
husband, to any presumption of compulsion or to any defense of
compulsion, except that stated in subsection (a).

(Source: P.A. 96-710, eff. 1-1-10.)

\hypertarget{ilcs-57-12-from-ch.-38-par.-7-12}{%
\subsection*{(720 ILCS 5/7-12) (from Ch. 38, par.
7-12)}\label{ilcs-57-12-from-ch.-38-par.-7-12}}
\addcontentsline{toc}{subsection}{(720 ILCS 5/7-12) (from Ch. 38, par.
7-12)}

\hypertarget{sec.-7-12.-entrapment.}{%
\section*{Sec. 7-12. Entrapment.}\label{sec.-7-12.-entrapment.}}
\addcontentsline{toc}{section}{Sec. 7-12. Entrapment.}

\markright{Sec. 7-12. Entrapment.}

A person is not guilty of an offense if his or her conduct is incited or
induced by a public officer or employee, or agent of either, for the
purpose of obtaining evidence for the prosecution of that person.
However, this Section is inapplicable if the person was pre-disposed to
commit the offense and the public officer or employee, or agent of
either, merely affords to that person the opportunity or facility for
committing an offense.

(Source: P.A. 89-332, eff. 1-1-96.)

\hypertarget{ilcs-57-13-from-ch.-38-par.-7-13}{%
\subsection*{(720 ILCS 5/7-13) (from Ch. 38, par.
7-13)}\label{ilcs-57-13-from-ch.-38-par.-7-13}}
\addcontentsline{toc}{subsection}{(720 ILCS 5/7-13) (from Ch. 38, par.
7-13)}

\hypertarget{sec.-7-13.-necessity.}{%
\section*{Sec. 7-13. Necessity.}\label{sec.-7-13.-necessity.}}
\addcontentsline{toc}{section}{Sec. 7-13. Necessity.}

\markright{Sec. 7-13. Necessity.}

Conduct which would otherwise be an offense is justifiable by reason of
necessity if the accused was without blame in occasioning or developing
the situation and reasonably believed such conduct was necessary to
avoid a public or private injury greater than the injury which might
reasonably result from his own conduct.

(Source: Laws 1961, p.~1983.)

\hypertarget{ilcs-57-14-from-ch.-38-par.-7-14}{%
\subsection*{(720 ILCS 5/7-14) (from Ch. 38, par.
7-14)}\label{ilcs-57-14-from-ch.-38-par.-7-14}}
\addcontentsline{toc}{subsection}{(720 ILCS 5/7-14) (from Ch. 38, par.
7-14)}

\hypertarget{sec.-7-14.-affirmative-defense.}{%
\section*{Sec. 7-14. Affirmative
defense.}\label{sec.-7-14.-affirmative-defense.}}
\addcontentsline{toc}{section}{Sec. 7-14. Affirmative defense.}

\markright{Sec. 7-14. Affirmative defense.}

A defense of justifiable use of force, or of exoneration, based on the
provisions of this Article is an affirmative defense.

(Source: Laws 1961, p.~1983.)

\hypertarget{ilcs-57-15}{%
\subsection*{(720 ILCS 5/7-15)}\label{ilcs-57-15}}
\addcontentsline{toc}{subsection}{(720 ILCS 5/7-15)}

\hypertarget{sec.-7-15.-duty-to-render-aid.}{%
\section*{Sec. 7-15. Duty to render
aid.}\label{sec.-7-15.-duty-to-render-aid.}}
\addcontentsline{toc}{section}{Sec. 7-15. Duty to render aid.}

\markright{Sec. 7-15. Duty to render aid.}

It is the policy of the State of Illinois that all law enforcement
officers must, as soon as reasonably practical, determine if a person is
injured, whether as a result of a use of force or otherwise, and render
medical aid and assistance consistent with training and request
emergency medical assistance if necessary. ``Render medical aid and
assistance'' includes, but is not limited to, (i) performing emergency
life-saving procedures such as cardiopulmonary resuscitation or the
administration of an automated external defibrillator; and (ii) the
making of arrangements for the carrying of such person to a physician,
surgeon, or hospital for medical or surgical treatment if it is apparent
that treatment is necessary, or if such carrying is requested by the
injured person.

(Source: P.A. 101-652, eff. 7-1-21; 102-28, eff. 6-25-21.)

\hypertarget{ilcs-57-16}{%
\subsection*{(720 ILCS 5/7-16)}\label{ilcs-57-16}}
\addcontentsline{toc}{subsection}{(720 ILCS 5/7-16)}

\hypertarget{sec.-7-16.-duty-to-intervene.}{%
\section*{Sec. 7-16. Duty to
intervene.}\label{sec.-7-16.-duty-to-intervene.}}
\addcontentsline{toc}{section}{Sec. 7-16. Duty to intervene.}

\markright{Sec. 7-16. Duty to intervene.}

(a) A peace officer, or any other person acting under the color of law
who has an opportunity to intervene, shall have an affirmative duty to
intervene to prevent or stop another peace officer in his or her
presence from using any unauthorized force or force that exceeds the
degree of force permitted, if any, without regard for chain of command.

(b) A peace officer, or any other person acting under the color of law,
who intervenes as required by this Section shall report the intervention
to the person designated/identified by the law enforcement entity in a
manner prescribed by the agency. The report required by this Section
must include the date, time, and place of the occurrence; the identity,
if known, and description of the participants; and a description of the
intervention actions taken and whether they were successful. In no event
shall the report be submitted more than 5 days after the incident.

(c) A member of a law enforcement agency shall not discipline nor
retaliate in any way against a peace officer for intervening as required
in this Section or for reporting unconstitutional or unlawful conduct,
or for failing to follow what the officer reasonably believes is an
unconstitutional or unlawful directive.

(Source: P.A. 101-652, eff. 7-1-21; 102-28, eff. 6-25-21.)

\hypertarget{ilcs-5tit.-iii-heading}{%
\subsection*{(720 ILCS 5/Tit. III
heading)}\label{ilcs-5tit.-iii-heading}}
\addcontentsline{toc}{subsection}{(720 ILCS 5/Tit. III heading)}

TITLE III.

SPECIFIC OFFENSES

\hypertarget{ilcs-5tit.-iii-pt.-a-heading}{%
\subsection*{(720 ILCS 5/Tit. III Pt. A
heading)}\label{ilcs-5tit.-iii-pt.-a-heading}}
\addcontentsline{toc}{subsection}{(720 ILCS 5/Tit. III Pt. A heading)}

PART A.

INCHOATE OFFENSES

\bookmarksetup{startatroot}

\hypertarget{article-8.-solicitation-conspiracy-and-attempt}{%
\chapter*{Article 8. Solicitation, Conspiracy And
Attempt}\label{article-8.-solicitation-conspiracy-and-attempt}}
\addcontentsline{toc}{chapter}{Article 8. Solicitation, Conspiracy And
Attempt}

\markboth{Article 8. Solicitation, Conspiracy And Attempt}{Article 8.
Solicitation, Conspiracy And Attempt}

\hypertarget{ilcs-58-1-from-ch.-38-par.-8-1}{%
\subsection*{(720 ILCS 5/8-1) (from Ch. 38, par.
8-1)}\label{ilcs-58-1-from-ch.-38-par.-8-1}}
\addcontentsline{toc}{subsection}{(720 ILCS 5/8-1) (from Ch. 38, par.
8-1)}

\hypertarget{sec.-8-1.-solicitation-and-solicitation-of-murder.}{%
\section*{Sec. 8-1. Solicitation and solicitation of
murder.}\label{sec.-8-1.-solicitation-and-solicitation-of-murder.}}
\addcontentsline{toc}{section}{Sec. 8-1. Solicitation and solicitation
of murder.}

\markright{Sec. 8-1. Solicitation and solicitation of murder.}

(a) Solicitation. A person commits the offense of solicitation when,
with intent that an offense be committed, other than first degree
murder, he or she commands, encourages, or requests another to commit
that offense.

(b) Solicitation of murder. A person commits the offense of solicitation
of murder when he or she commits solicitation with the intent that the
offense of first degree murder be committed.

(c) Sentence. A person convicted of solicitation may be fined or
imprisoned or both not to exceed the maximum provided for the offense
solicited, except that the penalty shall not exceed the corresponding
maximum limit provided by subparagraph (c) of Section 8-4 of this Code.
Solicitation of murder is a Class X felony, and a person convicted of
solicitation of murder shall be sentenced to a term of imprisonment of
not less than 15 years and not more than 30 years, except that a person
convicted of solicitation of murder when the person solicited was a
person under the age of 17 years shall be sentenced to a term of
imprisonment of not less than 20 years and not more than 60 years.

(Source: P.A. 96-710, eff. 1-1-10.)

\hypertarget{ilcs-58-1.1}{%
\subsection*{(720 ILCS 5/8-1.1)}\label{ilcs-58-1.1}}
\addcontentsline{toc}{subsection}{(720 ILCS 5/8-1.1)}

\hypertarget{sec.-8-1.1.-repealed.}{%
\section*{Sec. 8-1.1. (Repealed).}\label{sec.-8-1.1.-repealed.}}
\addcontentsline{toc}{section}{Sec. 8-1.1. (Repealed).}

\markright{Sec. 8-1.1. (Repealed).}

(Source: P.A. 89-689, eff. 12-31-96. Repealed by P.A. 96-710, eff.
1-1-10.)

\hypertarget{ilcs-58-1.2-from-ch.-38-par.-8-1.2}{%
\subsection*{(720 ILCS 5/8-1.2) (from Ch. 38, par.
8-1.2)}\label{ilcs-58-1.2-from-ch.-38-par.-8-1.2}}
\addcontentsline{toc}{subsection}{(720 ILCS 5/8-1.2) (from Ch. 38, par.
8-1.2)}

\hypertarget{sec.-8-1.2.-solicitation-of-murder-for-hire.}{%
\section*{Sec. 8-1.2. Solicitation of murder for
hire.}\label{sec.-8-1.2.-solicitation-of-murder-for-hire.}}
\addcontentsline{toc}{section}{Sec. 8-1.2. Solicitation of murder for
hire.}

\markright{Sec. 8-1.2. Solicitation of murder for hire.}

(a) A person commits the offense of solicitation of murder for hire
when, with the intent that the offense of first degree murder be
committed, he or she procures another to commit that offense pursuant to
any contract, agreement, understanding, command, or request for money or
anything of value.

(b) Sentence. Solicitation of murder for hire is a Class X felony, and a
person convicted of solicitation of murder for hire shall be sentenced
to a term of imprisonment of not less than 20 years and not more than 40
years, except that a person convicted of solicitation of murder for hire
when the person solicited was a person under the age of 17 years shall
be sentenced to a term of imprisonment of not less than 25 years and not
more than 60 years.

(Source: P.A. 96-710, eff. 1-1-10.)

\hypertarget{ilcs-58-2-from-ch.-38-par.-8-2}{%
\subsection*{(720 ILCS 5/8-2) (from Ch. 38, par.
8-2)}\label{ilcs-58-2-from-ch.-38-par.-8-2}}
\addcontentsline{toc}{subsection}{(720 ILCS 5/8-2) (from Ch. 38, par.
8-2)}

\hypertarget{sec.-8-2.-conspiracy.}{%
\section*{Sec. 8-2. Conspiracy.}\label{sec.-8-2.-conspiracy.}}
\addcontentsline{toc}{section}{Sec. 8-2. Conspiracy.}

\markright{Sec. 8-2. Conspiracy.}

(a) Elements of the offense. A person commits the offense of conspiracy
when, with intent that an offense be committed, he or she agrees with
another to the commission of that offense. No person may be convicted of
conspiracy to commit an offense unless an act in furtherance of that
agreement is alleged and proved to have been committed by him or her or
by a co-conspirator.

(b) Co-conspirators. It is not a defense to conspiracy that the person
or persons with whom the accused is alleged to have conspired:

(1) have not been prosecuted or convicted,

(2) have been convicted of a different offense,

(3) are not amenable to justice,

(4) have been acquitted, or

(5) lacked the capacity to commit an offense.

(c) Sentence.

(1) Except as otherwise provided in this subsection or Code, a person
convicted of conspiracy to commit:

(A) a Class X felony shall be sentenced for a Class 1 felony;

(B) a Class 1 felony shall be sentenced for a Class 2 felony;

(C) a Class 2 felony shall be sentenced for a Class 3 felony;

(D) a Class 3 felony shall be sentenced for a Class 4 felony;

(E) a Class 4 felony shall be sentenced for a Class 4 felony; and

(F) a misdemeanor may be fined or imprisoned or both not to exceed the
maximum provided for the offense that is the object of the conspiracy.

(2) A person convicted of conspiracy to commit any of the following
offenses shall be sentenced for a Class X felony:

\hypertarget{a-aggravated-insurance-fraud-conspiracy-when-the-person-is-an-organizer-of-the-conspiracy-720-ilcs-546-4-or}{%
\subsection*{(A) aggravated insurance fraud conspiracy when the person
is an organizer of the conspiracy (720 ILCS 5/46-4);
or}\label{a-aggravated-insurance-fraud-conspiracy-when-the-person-is-an-organizer-of-the-conspiracy-720-ilcs-546-4-or}}
\addcontentsline{toc}{subsection}{(A) aggravated insurance fraud
conspiracy when the person is an organizer of the conspiracy (720 ILCS
5/46-4); or}

\hypertarget{b-aggravated-governmental-entity-insurance-fraud-conspiracy-when-the-person-is-an-organizer-of-the-conspiracy-720-ilcs-546-4.}{%
\subsection*{(B) aggravated governmental entity insurance fraud
conspiracy when the person is an organizer of the conspiracy (720 ILCS
5/46-4).}\label{b-aggravated-governmental-entity-insurance-fraud-conspiracy-when-the-person-is-an-organizer-of-the-conspiracy-720-ilcs-546-4.}}
\addcontentsline{toc}{subsection}{(B) aggravated governmental entity
insurance fraud conspiracy when the person is an organizer of the
conspiracy (720 ILCS 5/46-4).}

(3) A person convicted of conspiracy to commit any of the following
offenses shall be sentenced for a Class 1 felony:

\hypertarget{a-first-degree-murder-720-ilcs-59-1-or}{%
\subsection*{(A) first degree murder (720 ILCS 5/9-1);
or}\label{a-first-degree-murder-720-ilcs-59-1-or}}
\addcontentsline{toc}{subsection}{(A) first degree murder (720 ILCS
5/9-1); or}

\hypertarget{b-aggravated-insurance-fraud-720-ilcs-546-3-or-aggravated-governmental-insurance-fraud-720-ilcs-546-3.}{%
\subsection*{(B) aggravated insurance fraud (720 ILCS 5/46-3) or
aggravated governmental insurance fraud (720 ILCS
5/46-3).}\label{b-aggravated-insurance-fraud-720-ilcs-546-3-or-aggravated-governmental-insurance-fraud-720-ilcs-546-3.}}
\addcontentsline{toc}{subsection}{(B) aggravated insurance fraud (720
ILCS 5/46-3) or aggravated governmental insurance fraud (720 ILCS
5/46-3).}

\hypertarget{a-person-convicted-of-conspiracy-to-commit-insurance-fraud-720-ilcs-546-3-or-governmental-entity-insurance-fraud-720-ilcs-546-3-shall-be-sentenced-for-a-class-2-felony.}{%
\subsection*{(4) A person convicted of conspiracy to commit insurance
fraud (720 ILCS 5/46-3) or governmental entity insurance fraud (720 ILCS
5/46-3) shall be sentenced for a Class 2
felony.}\label{a-person-convicted-of-conspiracy-to-commit-insurance-fraud-720-ilcs-546-3-or-governmental-entity-insurance-fraud-720-ilcs-546-3-shall-be-sentenced-for-a-class-2-felony.}}
\addcontentsline{toc}{subsection}{(4) A person convicted of conspiracy
to commit insurance fraud (720 ILCS 5/46-3) or governmental entity
insurance fraud (720 ILCS 5/46-3) shall be sentenced for a Class 2
felony.}

(5) A person convicted of conspiracy to commit any of the following
offenses shall be sentenced for a Class 3 felony:

\hypertarget{a-soliciting-for-a-prostitute-720-ilcs}{%
\subsection*{(A) soliciting for a prostitute (720
ILCS}\label{a-soliciting-for-a-prostitute-720-ilcs}}
\addcontentsline{toc}{subsection}{(A) soliciting for a prostitute (720
ILCS}

5/11-14.3(a)(1));

\hypertarget{b-pandering-720-ilcs-511-14.3a2a-or}{%
\subsection*{(B) pandering (720 ILCS 5/11-14.3(a)(2)(A)
or}\label{b-pandering-720-ilcs-511-14.3a2a-or}}
\addcontentsline{toc}{subsection}{(B) pandering (720 ILCS
5/11-14.3(a)(2)(A) or}

5/11-14.3(a)(2)(B));

\hypertarget{c-keeping-a-place-of-prostitution-720-ilcs}{%
\subsection*{(C) keeping a place of prostitution (720
ILCS}\label{c-keeping-a-place-of-prostitution-720-ilcs}}
\addcontentsline{toc}{subsection}{(C) keeping a place of prostitution
(720 ILCS}

5/11-14.3(a)(1));

\hypertarget{d-pimping-720-ilcs-511-14.3a2c}{%
\subsection*{(D) pimping (720 ILCS
5/11-14.3(a)(2)(C));}\label{d-pimping-720-ilcs-511-14.3a2c}}
\addcontentsline{toc}{subsection}{(D) pimping (720 ILCS
5/11-14.3(a)(2)(C));}

(E) unlawful use of weapons under Section

\hypertarget{a1-720-ilcs-524-1a1}{%
\subsection*{24-1(a)(1) (720 ILCS
5/24-1(a)(1));}\label{a1-720-ilcs-524-1a1}}
\addcontentsline{toc}{subsection}{24-1(a)(1) (720 ILCS 5/24-1(a)(1));}

(F) unlawful use of weapons under Section

\hypertarget{a7-720-ilcs-524-1a7}{%
\subsection*{24-1(a)(7) (720 ILCS
5/24-1(a)(7));}\label{a7-720-ilcs-524-1a7}}
\addcontentsline{toc}{subsection}{24-1(a)(7) (720 ILCS 5/24-1(a)(7));}

\hypertarget{g-gambling-720-ilcs-528-1}{%
\subsection*{(G) gambling (720 ILCS
5/28-1);}\label{g-gambling-720-ilcs-528-1}}
\addcontentsline{toc}{subsection}{(G) gambling (720 ILCS 5/28-1);}

\hypertarget{h-keeping-a-gambling-place-720-ilcs-528-3}{%
\subsection*{(H) keeping a gambling place (720 ILCS
5/28-3);}\label{h-keeping-a-gambling-place-720-ilcs-528-3}}
\addcontentsline{toc}{subsection}{(H) keeping a gambling place (720 ILCS
5/28-3);}

\hypertarget{i-registration-of-federal-gambling-stamps-violation-720-ilcs-528-4}{%
\subsection*{(I) registration of federal gambling stamps violation (720
ILCS
5/28-4);}\label{i-registration-of-federal-gambling-stamps-violation-720-ilcs-528-4}}
\addcontentsline{toc}{subsection}{(I) registration of federal gambling
stamps violation (720 ILCS 5/28-4);}

\hypertarget{j-look-alike-substances-violation-720-ilcs}{%
\subsection*{(J) look-alike substances violation (720
ILCS}\label{j-look-alike-substances-violation-720-ilcs}}
\addcontentsline{toc}{subsection}{(J) look-alike substances violation
(720 ILCS}

570/404);

\hypertarget{k-miscellaneous-controlled-substance-violation-under-section-406b-720-ilcs-570406b-or}{%
\subsection*{(K) miscellaneous controlled substance violation under
Section 406(b) (720 ILCS 570/406(b));
or}\label{k-miscellaneous-controlled-substance-violation-under-section-406b-720-ilcs-570406b-or}}
\addcontentsline{toc}{subsection}{(K) miscellaneous controlled substance
violation under Section 406(b) (720 ILCS 570/406(b)); or}

(L) an inchoate offense related to any of the principal offenses set
forth in this item (5).

(Source: P.A. 96-710, eff. 1-1-10; 96-1551, eff. 7-1-11 .)

\hypertarget{ilcs-58-2.1}{%
\subsection*{(720 ILCS 5/8-2.1)}\label{ilcs-58-2.1}}
\addcontentsline{toc}{subsection}{(720 ILCS 5/8-2.1)}

\hypertarget{sec.-8-2.1.-conspiracy-against-civil-rights.}{%
\section*{Sec. 8-2.1. Conspiracy against civil
rights.}\label{sec.-8-2.1.-conspiracy-against-civil-rights.}}
\addcontentsline{toc}{section}{Sec. 8-2.1. Conspiracy against civil
rights.}

\markright{Sec. 8-2.1. Conspiracy against civil rights.}

(a) Offense. A person commits conspiracy against civil rights when,
without legal justification, he or she, with the intent to interfere
with the free exercise of any right or privilege secured by the
Constitution of the United States, the Constitution of the State of
Illinois, the laws of the United States, or the laws of the State of
Illinois by any person or persons, agrees with another to inflict
physical harm on any other person or the threat of physical harm on any
other person and either the accused or a co-conspirator has committed
any act in furtherance of that agreement.

(b) Co-conspirators. It shall not be a defense to conspiracy against
civil rights that a person or persons with whom the accused is alleged
to have conspired:

(1) has not been prosecuted or convicted; or

(2) has been convicted of a different offense; or

(3) is not amenable to justice; or

(4) has been acquitted; or

(5) lacked the capacity to commit an offense.

(c) Sentence. Conspiracy against civil rights is a Class 4 felony for a
first offense and a Class 2 felony for a second or subsequent offense.

(Source: P.A. 92-830, eff. 1-1-03.)

\hypertarget{ilcs-58-3-from-ch.-38-par.-8-3}{%
\subsection*{(720 ILCS 5/8-3) (from Ch. 38, par.
8-3)}\label{ilcs-58-3-from-ch.-38-par.-8-3}}
\addcontentsline{toc}{subsection}{(720 ILCS 5/8-3) (from Ch. 38, par.
8-3)}

\hypertarget{sec.-8-3.-defense.}{%
\section*{Sec. 8-3. Defense.}\label{sec.-8-3.-defense.}}
\addcontentsline{toc}{section}{Sec. 8-3. Defense.}

\markright{Sec. 8-3. Defense.}

It is a defense to a charge of solicitation or conspiracy that if the
criminal object were achieved the accused would not be guilty of an
offense.

(Source: Laws 1961, p.~1983.)

\hypertarget{ilcs-58-4-from-ch.-38-par.-8-4}{%
\subsection*{(720 ILCS 5/8-4) (from Ch. 38, par.
8-4)}\label{ilcs-58-4-from-ch.-38-par.-8-4}}
\addcontentsline{toc}{subsection}{(720 ILCS 5/8-4) (from Ch. 38, par.
8-4)}

\hypertarget{sec.-8-4.-attempt.}{%
\section*{Sec. 8-4. Attempt.}\label{sec.-8-4.-attempt.}}
\addcontentsline{toc}{section}{Sec. 8-4. Attempt.}

\markright{Sec. 8-4. Attempt.}

(a) Elements of the offense.

A person commits the offense of attempt when, with intent to commit a
specific offense, he or she does any act that constitutes a substantial
step toward the commission of that offense.

(b) Impossibility.

It is not a defense to a charge of attempt that because of a
misapprehension of the circumstances it would have been impossible for
the accused to commit the offense attempted.

(c) Sentence.

A person convicted of attempt may be fined or imprisoned or both not to
exceed the maximum provided for the offense attempted but, except for an
attempt to commit the offense defined in Section 33A-2 of this Code:

(1) the sentence for attempt to commit first degree murder is the
sentence for a Class X felony, except that

(A) an attempt to commit first degree murder when at least one of the
aggravating factors specified in paragraphs (1), (2), and (12) of
subsection (b) of Section 9-1 is present is a Class X felony for which
the sentence shall be a term of imprisonment of not less than 20 years
and not more than 80 years;

(B) an attempt to commit first degree murder while armed with a firearm
is a Class X felony for which 15 years shall be added to the term of
imprisonment imposed by the court;

(C) an attempt to commit first degree murder during which the person
personally discharged a firearm is a Class X felony for which 20 years
shall be added to the term of imprisonment imposed by the court;

(D) an attempt to commit first degree murder during which the person
personally discharged a firearm that proximately caused great bodily
harm, permanent disability, permanent disfigurement, or death to another
person is a Class X felony for which 25 years or up to a term of natural
life shall be added to the term of imprisonment imposed by the court;
and

(E) if the defendant proves by a preponderance of the evidence at
sentencing that, at the time of the attempted murder, he or she was
acting under a sudden and intense passion resulting from serious
provocation by the individual whom the defendant endeavored to kill, or
another, and, had the individual the defendant endeavored to kill died,
the defendant would have negligently or accidentally caused that death,
then the sentence for the attempted murder is the sentence for a Class 1
felony;

(2) the sentence for attempt to commit a Class X felony is the sentence
for a Class 1 felony;

(3) the sentence for attempt to commit a Class 1 felony is the sentence
for a Class 2 felony;

(4) the sentence for attempt to commit a Class 2 felony is the sentence
for a Class 3 felony; and

(5) the sentence for attempt to commit any felony other than those
specified in items (1), (2), (3), and (4) of this subsection (c) is the
sentence for a Class A misdemeanor.

(Source: P.A. 96-710, eff. 1-1-10.)

\hypertarget{ilcs-58-5-from-ch.-38-par.-8-5}{%
\subsection*{(720 ILCS 5/8-5) (from Ch. 38, par.
8-5)}\label{ilcs-58-5-from-ch.-38-par.-8-5}}
\addcontentsline{toc}{subsection}{(720 ILCS 5/8-5) (from Ch. 38, par.
8-5)}

\hypertarget{sec.-8-5.-multiple-convictions.}{%
\section*{Sec. 8-5. Multiple
convictions.}\label{sec.-8-5.-multiple-convictions.}}
\addcontentsline{toc}{section}{Sec. 8-5. Multiple convictions.}

\markright{Sec. 8-5. Multiple convictions.}

No person shall be convicted of both the inchoate and the principal
offense.

(Source: Laws 1961, p.~1983.)

\hypertarget{ilcs-58-6-from-ch.-38-par.-8-6}{%
\subsection*{(720 ILCS 5/8-6) (from Ch. 38, par.
8-6)}\label{ilcs-58-6-from-ch.-38-par.-8-6}}
\addcontentsline{toc}{subsection}{(720 ILCS 5/8-6) (from Ch. 38, par.
8-6)}

\hypertarget{sec.-8-6.-offense.}{%
\section*{Sec. 8-6. Offense.}\label{sec.-8-6.-offense.}}
\addcontentsline{toc}{section}{Sec. 8-6. Offense.}

\markright{Sec. 8-6. Offense.}

For the purposes of this Article, ``offense'' shall include conduct
which if performed in another State would be criminal by the laws of
that State and which conduct if performed in this State would be an
offense under the laws of this State.

(Source: Laws 1961, p.~1983.)

\hypertarget{ilcs-5tit.-iii-pt.-b-heading}{%
\subsection*{(720 ILCS 5/Tit. III Pt. B
heading)}\label{ilcs-5tit.-iii-pt.-b-heading}}
\addcontentsline{toc}{subsection}{(720 ILCS 5/Tit. III Pt. B heading)}

PART B.

OFFENSES DIRECTED AGAINST THE PERSON

\bookmarksetup{startatroot}

\hypertarget{article-9.-homicide}{%
\chapter*{Article 9. Homicide}\label{article-9.-homicide}}
\addcontentsline{toc}{chapter}{Article 9. Homicide}

\markboth{Article 9. Homicide}{Article 9. Homicide}

\hypertarget{ilcs-59-1-from-ch.-38-par.-9-1}{%
\subsection*{(720 ILCS 5/9-1) (from Ch. 38, par.
9-1)}\label{ilcs-59-1-from-ch.-38-par.-9-1}}
\addcontentsline{toc}{subsection}{(720 ILCS 5/9-1) (from Ch. 38, par.
9-1)}

\hypertarget{sec.-9-1.-first-degree-murder-death-penalties-exceptions-separate-hearings-proof-findings-appellate-procedures-reversals.}{%
\section*{Sec. 9-1. First degree murder; death penalties; exceptions;
separate hearings; proof; findings; appellate procedures;
reversals.}\label{sec.-9-1.-first-degree-murder-death-penalties-exceptions-separate-hearings-proof-findings-appellate-procedures-reversals.}}
\addcontentsline{toc}{section}{Sec. 9-1. First degree murder; death
penalties; exceptions; separate hearings; proof; findings; appellate
procedures; reversals.}

\markright{Sec. 9-1. First degree murder; death penalties; exceptions;
separate hearings; proof; findings; appellate procedures; reversals.}

(a) A person who kills an individual without lawful justification
commits first degree murder if, in performing the acts which cause the
death:

(1) he or she either intends to kill or do great bodily harm to that
individual or another, or knows that such acts will cause death to that
individual or another; or

(2) he or she knows that such acts create a strong probability of death
or great bodily harm to that individual or another; or

(3) he or she, acting alone or with one or more participants, commits or
attempts to commit a forcible felony other than second degree murder,
and in the course of or in furtherance of such crime or flight
therefrom, he or she or another participant causes the death of a
person.

(b) Aggravating Factors. A defendant who at the time of the commission
of the offense has attained the age of 18 or more and who has been found
guilty of first degree murder may be sentenced to death if:

(1) the murdered individual was a peace officer or fireman killed in the
course of performing his official duties, to prevent the performance of
his or her official duties, or in retaliation for performing his or her
official duties, and the defendant knew or should have known that the
murdered individual was a peace officer or fireman; or

(2) the murdered individual was an employee of an institution or
facility of the Department of Corrections, or any similar local
correctional agency, killed in the course of performing his or her
official duties, to prevent the performance of his or her official
duties, or in retaliation for performing his or her official duties, or
the murdered individual was an inmate at such institution or facility
and was killed on the grounds thereof, or the murdered individual was
otherwise present in such institution or facility with the knowledge and
approval of the chief administrative officer thereof; or

(3) the defendant has been convicted of murdering two or more
individuals under subsection (a) of this Section or under any law of the
United States or of any state which is substantially similar to
subsection (a) of this Section regardless of whether the deaths occurred
as the result of the same act or of several related or unrelated acts so
long as the deaths were the result of either an intent to kill more than
one person or of separate acts which the defendant knew would cause
death or create a strong probability of death or great bodily harm to
the murdered individual or another; or

(4) the murdered individual was killed as a result of the hijacking of
an airplane, train, ship, bus, or other public conveyance; or

(5) the defendant committed the murder pursuant to a contract,
agreement, or understanding by which he or she was to receive money or
anything of value in return for committing the murder or procured
another to commit the murder for money or anything of value; or

(6) the murdered individual was killed in the course of another felony
if:

(a) the murdered individual:

(i) was actually killed by the defendant, or

(ii) received physical injuries personally inflicted by the defendant
substantially contemporaneously with physical injuries caused by one or
more persons for whose conduct the defendant is legally accountable
under Section 5-2 of this Code, and the physical injuries inflicted by
either the defendant or the other person or persons for whose conduct he
is legally accountable caused the death of the murdered individual; and

(b) in performing the acts which caused the death of the murdered
individual or which resulted in physical injuries personally inflicted
by the defendant on the murdered individual under the circumstances of
subdivision (ii) of subparagraph (a) of paragraph (6) of subsection (b)
of this Section, the defendant acted with the intent to kill the
murdered individual or with the knowledge that his acts created a strong
probability of death or great bodily harm to the murdered individual or
another; and

(c) the other felony was an inherently violent crime or the attempt to
commit an inherently violent crime. In this subparagraph (c),
``inherently violent crime'' includes, but is not limited to, armed
robbery, robbery, predatory criminal sexual assault of a child,
aggravated criminal sexual assault, aggravated kidnapping, aggravated
vehicular hijacking, aggravated arson, aggravated stalking, residential
burglary, and home invasion; or

(7) the murdered individual was under 12 years of age and the death
resulted from exceptionally brutal or heinous behavior indicative of
wanton cruelty; or

(8) the defendant committed the murder with intent to prevent the
murdered individual from testifying or participating in any criminal
investigation or prosecution or giving material assistance to the State
in any investigation or prosecution, either against the defendant or
another; or the defendant committed the murder because the murdered
individual was a witness in any prosecution or gave material assistance
to the State in any investigation or prosecution, either against the
defendant or another; for purposes of this paragraph (8),
``participating in any criminal investigation or prosecution'' is
intended to include those appearing in the proceedings in any capacity
such as trial judges, prosecutors, defense attorneys, investigators,
witnesses, or jurors; or

(9) the defendant, while committing an offense punishable under Sections
401, 401.1, 401.2, 405, 405.2, 407 or 407.1 or subsection (b) of Section
404 of the Illinois Controlled Substances Act, or while engaged in a
conspiracy or solicitation to commit such offense, intentionally killed
an individual or counseled, commanded, induced, procured or caused the
intentional killing of the murdered individual; or

(10) the defendant was incarcerated in an institution or facility of the
Department of Corrections at the time of the murder, and while
committing an offense punishable as a felony under Illinois law, or
while engaged in a conspiracy or solicitation to commit such offense,
intentionally killed an individual or counseled, commanded, induced,
procured or caused the intentional killing of the murdered individual;
or

(11) the murder was committed in a cold, calculated and premeditated
manner pursuant to a preconceived plan, scheme or design to take a human
life by unlawful means, and the conduct of the defendant created a
reasonable expectation that the death of a human being would result
therefrom; or

(12) the murdered individual was an emergency medical technician -
ambulance, emergency medical technician - intermediate, emergency
medical technician - paramedic, ambulance driver, or other medical
assistance or first aid personnel, employed by a municipality or other
governmental unit, killed in the course of performing his official
duties, to prevent the performance of his official duties, or in
retaliation for performing his official duties, and the defendant knew
or should have known that the murdered individual was an emergency
medical technician - ambulance, emergency medical technician -
intermediate, emergency medical technician - paramedic, ambulance
driver, or other medical assistance or first aid personnel; or

(13) the defendant was a principal administrator, organizer, or leader
of a calculated criminal drug conspiracy consisting of a hierarchical
position of authority superior to that of all other members of the
conspiracy, and the defendant counseled, commanded, induced, procured,
or caused the intentional killing of the murdered person; or

(14) the murder was intentional and involved the infliction of torture.
For the purpose of this Section torture means the infliction of or
subjection to extreme physical pain, motivated by an intent to increase
or prolong the pain, suffering or agony of the victim; or

(15) the murder was committed as a result of the intentional discharge
of a firearm by the defendant from a motor vehicle and the victim was
not present within the motor vehicle; or

(16) the murdered individual was 60 years of age or older and the death
resulted from exceptionally brutal or heinous behavior indicative of
wanton cruelty; or

(17) the murdered individual was a person with a disability and the
defendant knew or should have known that the murdered individual was a
person with a disability. For purposes of this paragraph (17), ``person
with a disability'' means a person who suffers from a permanent physical
or mental impairment resulting from disease, an injury, a functional
disorder, or a congenital condition that renders the person incapable of
adequately providing for his or her own health or personal care; or

(18) the murder was committed by reason of any person's activity as a
community policing volunteer or to prevent any person from engaging in
activity as a community policing volunteer; or

(19) the murdered individual was subject to an order of protection and
the murder was committed by a person against whom the same order of
protection was issued under the Illinois Domestic Violence Act of 1986;
or

(20) the murdered individual was known by the defendant to be a teacher
or other person employed in any school and the teacher or other employee
is upon the grounds of a school or grounds adjacent to a school, or is
in any part of a building used for school purposes; or

(21) the murder was committed by the defendant in connection with or as
a result of the offense of terrorism as defined in Section 29D-14.9 of
this Code; or

(22) the murdered individual was a member of a congregation engaged in
prayer or other religious activities at a church, synagogue, mosque, or
other building, structure, or place used for religious worship.

(b-5) Aggravating Factor; Natural Life Imprisonment. A defendant who has
been found guilty of first degree murder and who at the time of the
commission of the offense had attained the age of 18 years or more may
be sentenced to natural life imprisonment if (i) the murdered individual
was a physician, physician assistant, psychologist, nurse, or advanced
practice registered nurse, (ii) the defendant knew or should have known
that the murdered individual was a physician, physician assistant,
psychologist, nurse, or advanced practice registered nurse, and (iii)
the murdered individual was killed in the course of acting in his or her
capacity as a physician, physician assistant, psychologist, nurse, or
advanced practice registered nurse, or to prevent him or her from acting
in that capacity, or in retaliation for his or her acting in that
capacity.

(c) Consideration of factors in Aggravation and Mitigation.

The court shall consider, or shall instruct the jury to consider any
aggravating and any mitigating factors which are relevant to the
imposition of the death penalty. Aggravating factors may include but
need not be limited to those factors set forth in subsection (b).
Mitigating factors may include but need not be limited to the following:

(1) the defendant has no significant history of prior criminal activity;

(2) the murder was committed while the defendant was under the influence
of extreme mental or emotional disturbance, although not such as to
constitute a defense to prosecution;

(3) the murdered individual was a participant in the defendant's
homicidal conduct or consented to the homicidal act;

(4) the defendant acted under the compulsion of threat or menace of the
imminent infliction of death or great bodily harm;

(5) the defendant was not personally present during commission of the
act or acts causing death;

(6) the defendant's background includes a history of extreme emotional
or physical abuse;

(7) the defendant suffers from a reduced mental capacity.

Provided, however, that an action that does not otherwise mitigate first
degree murder cannot qualify as a mitigating factor for first degree
murder because of the discovery, knowledge, or disclosure of the
victim's sexual orientation as defined in Section 1-103 of the Illinois
Human Rights Act.

(d) Separate sentencing hearing.

Where requested by the State, the court shall conduct a separate
sentencing proceeding to determine the existence of factors set forth in
subsection (b) and to consider any aggravating or mitigating factors as
indicated in subsection (c). The proceeding shall be conducted:

(1) before the jury that determined the defendant's guilt; or

(2) before a jury impanelled for the purpose of the proceeding if:

A. the defendant was convicted upon a plea of guilty; or

B. the defendant was convicted after a trial before the court sitting
without a jury; or

C. the court for good cause shown discharges the jury that determined
the defendant's guilt; or

(3) before the court alone if the defendant waives a jury for the
separate proceeding.

(e) Evidence and Argument.

During the proceeding any information relevant to any of the factors set
forth in subsection (b) may be presented by either the State or the
defendant under the rules governing the admission of evidence at
criminal trials. Any information relevant to any additional aggravating
factors or any mitigating factors indicated in subsection (c) may be
presented by the State or defendant regardless of its admissibility
under the rules governing the admission of evidence at criminal trials.
The State and the defendant shall be given fair opportunity to rebut any
information received at the hearing.

(f) Proof.

The burden of proof of establishing the existence of any of the factors
set forth in subsection (b) is on the State and shall not be satisfied
unless established beyond a reasonable doubt.

(g) Procedure - Jury.

If at the separate sentencing proceeding the jury finds that none of the
factors set forth in subsection (b) exists, the court shall sentence the
defendant to a term of imprisonment under Chapter V of the Unified Code
of Corrections. If there is a unanimous finding by the jury that one or
more of the factors set forth in subsection (b) exist, the jury shall
consider aggravating and mitigating factors as instructed by the court
and shall determine whether the sentence of death shall be imposed. If
the jury determines unanimously, after weighing the factors in
aggravation and mitigation, that death is the appropriate sentence, the
court shall sentence the defendant to death. If the court does not
concur with the jury determination that death is the appropriate
sentence, the court shall set forth reasons in writing including what
facts or circumstances the court relied upon, along with any relevant
documents, that compelled the court to non-concur with the sentence.
This document and any attachments shall be part of the record for
appellate review. The court shall be bound by the jury's sentencing
determination.

If after weighing the factors in aggravation and mitigation, one or more
jurors determines that death is not the appropriate sentence, the court
shall sentence the defendant to a term of imprisonment under Chapter V
of the Unified Code of Corrections.

(h) Procedure - No Jury.

In a proceeding before the court alone, if the court finds that none of
the factors found in subsection (b) exists, the court shall sentence the
defendant to a term of imprisonment under Chapter V of the Unified Code
of Corrections.

If the Court determines that one or more of the factors set forth in
subsection (b) exists, the Court shall consider any aggravating and
mitigating factors as indicated in subsection (c). If the Court
determines, after weighing the factors in aggravation and mitigation,
that death is the appropriate sentence, the Court shall sentence the
defendant to death.

If the court finds that death is not the appropriate sentence, the court
shall sentence the defendant to a term of imprisonment under Chapter V
of the Unified Code of Corrections.

(h-5) Decertification as a capital case.

In a case in which the defendant has been found guilty of first degree
murder by a judge or jury, or a case on remand for resentencing, and the
State seeks the death penalty as an appropriate sentence, on the court's
own motion or the written motion of the defendant, the court may
decertify the case as a death penalty case if the court finds that the
only evidence supporting the defendant's conviction is the
uncorroborated testimony of an informant witness, as defined in Section
115-21 of the Code of Criminal Procedure of 1963, concerning the
confession or admission of the defendant or that the sole evidence
against the defendant is a single eyewitness or single accomplice
without any other corroborating evidence. If the court decertifies the
case as a capital case under either of the grounds set forth above, the
court shall issue a written finding. The State may pursue its right to
appeal the decertification pursuant to Supreme Court Rule 604(a)(1). If
the court does not decertify the case as a capital case, the matter
shall proceed to the eligibility phase of the sentencing hearing.

(i) Appellate Procedure.

The conviction and sentence of death shall be subject to automatic
review by the Supreme Court. Such review shall be in accordance with
rules promulgated by the Supreme Court. The Illinois Supreme Court may
overturn the death sentence, and order the imposition of imprisonment
under Chapter V of the Unified Code of Corrections if the court finds
that the death sentence is fundamentally unjust as applied to the
particular case. If the Illinois Supreme Court finds that the death
sentence is fundamentally unjust as applied to the particular case,
independent of any procedural grounds for relief, the Illinois Supreme
Court shall issue a written opinion explaining this finding.

(j) Disposition of reversed death sentence.

In the event that the death penalty in this Act is held to be
unconstitutional by the Supreme Court of the United States or of the
State of Illinois, any person convicted of first degree murder shall be
sentenced by the court to a term of imprisonment under Chapter V of the
Unified Code of Corrections.

In the event that any death sentence pursuant to the sentencing
provisions of this Section is declared unconstitutional by the Supreme
Court of the United States or of the State of Illinois, the court having
jurisdiction over a person previously sentenced to death shall cause the
defendant to be brought before the court, and the court shall sentence
the defendant to a term of imprisonment under Chapter V of the Unified
Code of Corrections.

(k) Guidelines for seeking the death penalty.

The Attorney General and State's Attorneys Association shall consult on
voluntary guidelines for procedures governing whether or not to seek the
death penalty. The guidelines do not have the force of law and are only
advisory in nature.

(Source: P.A. 100-460, eff. 1-1-18; 100-513, eff. 1-1-18; 100-863, eff.
8-14-18; 101-223, eff. 1-1-20; 101-652, eff. 7-1-21 .)

\hypertarget{ilcs-59-1.2-from-ch.-38-par.-9-1.2}{%
\subsection*{(720 ILCS 5/9-1.2) (from Ch. 38, par.
9-1.2)}\label{ilcs-59-1.2-from-ch.-38-par.-9-1.2}}
\addcontentsline{toc}{subsection}{(720 ILCS 5/9-1.2) (from Ch. 38, par.
9-1.2)}

\hypertarget{sec.-9-1.2.-intentional-homicide-of-an-unborn-child.}{%
\section*{Sec. 9-1.2. Intentional homicide of an unborn
child.}\label{sec.-9-1.2.-intentional-homicide-of-an-unborn-child.}}
\addcontentsline{toc}{section}{Sec. 9-1.2. Intentional homicide of an
unborn child.}

\markright{Sec. 9-1.2. Intentional homicide of an unborn child.}

(a) A person commits the offense of intentional homicide of an unborn
child if, in performing acts which cause the death of an unborn child,
he without lawful justification:

(1) either intended to cause the death of or do great bodily harm to the
pregnant individual or unborn child or knew that such acts would cause
death or great bodily harm to the pregnant individual or unborn child;
or

(2) knew that his acts created a strong probability of death or great
bodily harm to the pregnant individual or unborn child; and

(3) knew that the individual was pregnant.

(b) For purposes of this Section, (1) ``unborn child'' shall mean any
individual of the human species from the implantation of an embryo until
birth, and (2) ``person'' shall not include the pregnant woman whose
unborn child is killed.

(c) This Section shall not apply to acts which cause the death of an
unborn child if those acts were committed during any abortion, as
defined in Section 1-10 of the Reproductive Health Act, to which the
pregnant individual has consented. This Section shall not apply to acts
which were committed pursuant to usual and customary standards of
medical practice during diagnostic testing or therapeutic treatment.

(d) Penalty. The sentence for intentional homicide of an unborn child
shall be the same as for first degree murder, except that:

(1) the death penalty may not be imposed;

(2) if the person committed the offense while armed with a firearm, 15
years shall be added to the term of imprisonment imposed by the court;

(3) if, during the commission of the offense, the person personally
discharged a firearm, 20 years shall be added to the term of
imprisonment imposed by the court;

(4) if, during the commission of the offense, the person personally
discharged a firearm that proximately caused great bodily harm,
permanent disability, permanent disfigurement, or death to another
person, 25 years or up to a term of natural life shall be added to the
term of imprisonment imposed by the court.

(e) The provisions of this Act shall not be construed to prohibit the
prosecution of any person under any other provision of law.

(Source: P.A. 101-13, eff. 6-12-19.)

\hypertarget{ilcs-59-2-from-ch.-38-par.-9-2}{%
\subsection*{(720 ILCS 5/9-2) (from Ch. 38, par.
9-2)}\label{ilcs-59-2-from-ch.-38-par.-9-2}}
\addcontentsline{toc}{subsection}{(720 ILCS 5/9-2) (from Ch. 38, par.
9-2)}

\hypertarget{sec.-9-2.-second-degree-murder.}{%
\section*{Sec. 9-2. Second degree
murder.}\label{sec.-9-2.-second-degree-murder.}}
\addcontentsline{toc}{section}{Sec. 9-2. Second degree murder.}

\markright{Sec. 9-2. Second degree murder.}

(a) A person commits the offense of second degree murder when he or she
commits the offense of first degree murder as defined in paragraph (1)
or (2) of subsection (a) of Section 9-1 of this Code and either of the
following mitigating factors are present:

(1) at the time of the killing he or she is acting under a sudden and
intense passion resulting from serious provocation by the individual
killed or another whom the offender endeavors to kill, but he or she
negligently or accidentally causes the death of the individual killed;
or

(2) at the time of the killing he or she believes the circumstances to
be such that, if they existed, would justify or exonerate the killing
under the principles stated in Article 7 of this Code, but his or her
belief is unreasonable.

(b) Serious provocation is conduct sufficient to excite an intense
passion in a reasonable person provided, however, that an action that
does not otherwise constitute serious provocation cannot qualify as
serious provocation because of the discovery, knowledge, or disclosure
of the victim's sexual orientation as defined in Section 1-103 of the
Illinois Human Rights Act.

(c) When evidence of either of the mitigating factors defined in
subsection (a) of this Section has been presented, the burden of proof
is on the defendant to prove either mitigating factor by a preponderance
of the evidence before the defendant can be found guilty of second
degree murder. The burden of proof, however, remains on the State to
prove beyond a reasonable doubt each of the elements of first degree
murder and, when appropriately raised, the absence of circumstances at
the time of the killing that would justify or exonerate the killing
under the principles stated in Article 7 of this Code.

(d) Sentence. Second degree murder is a Class 1 felony.

(Source: P.A. 100-460, eff. 1-1-18 .)

\hypertarget{ilcs-59-2.1-from-ch.-38-par.-9-2.1}{%
\subsection*{(720 ILCS 5/9-2.1) (from Ch. 38, par.
9-2.1)}\label{ilcs-59-2.1-from-ch.-38-par.-9-2.1}}
\addcontentsline{toc}{subsection}{(720 ILCS 5/9-2.1) (from Ch. 38, par.
9-2.1)}

\hypertarget{sec.-9-2.1.-voluntary-manslaughter-of-an-unborn-child.}{%
\section*{Sec. 9-2.1. Voluntary manslaughter of an unborn
child.}\label{sec.-9-2.1.-voluntary-manslaughter-of-an-unborn-child.}}
\addcontentsline{toc}{section}{Sec. 9-2.1. Voluntary manslaughter of an
unborn child.}

\markright{Sec. 9-2.1. Voluntary manslaughter of an unborn child.}

(a) A person who kills an unborn child without lawful justification
commits voluntary manslaughter of an unborn child if at the time of the
killing he is acting under a sudden and intense passion resulting from
serious provocation by another whom the offender endeavors to kill, but
he negligently or accidentally causes the death of the unborn child.

Serious provocation is conduct sufficient to excite an intense passion
in a reasonable person.

(b) A person who intentionally or knowingly kills an unborn child
commits voluntary manslaughter of an unborn child if at the time of the
killing he believes the circumstances to be such that, if they existed,
would justify or exonerate the killing under the principles stated in
Article 7 of this Code, but his belief is unreasonable.

(c) Sentence. Voluntary manslaughter of an unborn child is a Class 1
felony.

(d) For purposes of this Section, (1) ``unborn child'' shall mean any
individual of the human species from the implantation of an embryo until
birth, and (2) ``person'' shall not include the pregnant individual
whose unborn child is killed.

(e) This Section shall not apply to acts which cause the death of an
unborn child if those acts were committed during any abortion, as
defined in Section 1-10 of the Reproductive Health Act, to which the
pregnant individual has consented. This Section shall not apply to acts
which were committed pursuant to usual and customary standards of
medical practice during diagnostic testing or therapeutic treatment.

(Source: P.A. 101-13, eff. 6-12-19.)

\hypertarget{ilcs-59-3-from-ch.-38-par.-9-3}{%
\subsection*{(720 ILCS 5/9-3) (from Ch. 38, par.
9-3)}\label{ilcs-59-3-from-ch.-38-par.-9-3}}
\addcontentsline{toc}{subsection}{(720 ILCS 5/9-3) (from Ch. 38, par.
9-3)}

\hypertarget{sec.-9-3.-involuntary-manslaughter-and-reckless-homicide.}{%
\section*{Sec. 9-3. Involuntary Manslaughter and Reckless
Homicide.}\label{sec.-9-3.-involuntary-manslaughter-and-reckless-homicide.}}
\addcontentsline{toc}{section}{Sec. 9-3. Involuntary Manslaughter and
Reckless Homicide.}

\markright{Sec. 9-3. Involuntary Manslaughter and Reckless Homicide.}

(a) A person who unintentionally kills an individual without lawful
justification commits involuntary manslaughter if his acts whether
lawful or unlawful which cause the death are such as are likely to cause
death or great bodily harm to some individual, and he performs them
recklessly, except in cases in which the cause of the death consists of
the driving of a motor vehicle or operating a snowmobile, all-terrain
vehicle, or watercraft, in which case the person commits reckless
homicide. A person commits reckless homicide if he or she
unintentionally kills an individual while driving a vehicle and using an
incline in a roadway, such as a railroad crossing, bridge approach, or
hill, to cause the vehicle to become airborne.

(b) (Blank).

(c) (Blank).

(d) Sentence.

(1) Involuntary manslaughter is a Class 3 felony.

(2) Reckless homicide is a Class 3 felony.

(e) (Blank).

(e-2) Except as provided in subsection (e-3), in cases involving
reckless homicide in which the offense is committed upon a public
thoroughfare where children pass going to and from school when a school
crossing guard is performing official duties, the penalty is a Class 2
felony, for which a person, if sentenced to a term of imprisonment,
shall be sentenced to a term of not less than 3 years and not more than
14 years.

(e-3) In cases involving reckless homicide in which (i) the offense is
committed upon a public thoroughfare where children pass going to and
from school when a school crossing guard is performing official duties
and (ii) the defendant causes the deaths of 2 or more persons as part of
a single course of conduct, the penalty is a Class 2 felony, for which a
person, if sentenced to a term of imprisonment, shall be sentenced to a
term of not less than 6 years and not more than 28 years.

(e-5) (Blank).

(e-7) Except as otherwise provided in subsection (e-8), in cases
involving reckless homicide in which the defendant: (1) was driving in a
construction or maintenance zone, as defined in Section 11-605.1 of the
Illinois Vehicle Code, or (2) was operating a vehicle while failing or
refusing to comply with any lawful order or direction of any authorized
police officer or traffic control aide engaged in traffic control, the
penalty is a Class 2 felony, for which a person, if sentenced to a term
of imprisonment, shall be sentenced to a term of not less than 3 years
and not more than 14 years.

(e-8) In cases involving reckless homicide in which the defendant caused
the deaths of 2 or more persons as part of a single course of conduct
and: (1) was driving in a construction or maintenance zone, as defined
in Section 11-605.1 of the Illinois Vehicle Code, or (2) was operating a
vehicle while failing or refusing to comply with any lawful order or
direction of any authorized police officer or traffic control aide
engaged in traffic control, the penalty is a Class 2 felony, for which a
person, if sentenced to a term of imprisonment, shall be sentenced to a
term of not less than 6 years and not more than 28 years.

(e-9) In cases involving reckless homicide in which the defendant drove
a vehicle and used an incline in a roadway, such as a railroad crossing,
bridge approach, or hill, to cause the vehicle to become airborne, and
caused the deaths of 2 or more persons as part of a single course of
conduct, the penalty is a Class 2 felony.

(e-10) In cases involving involuntary manslaughter or reckless homicide
resulting in the death of a peace officer killed in the performance of
his or her duties as a peace officer, the penalty is a Class 2 felony.

(e-11) In cases involving reckless homicide in which the defendant
unintentionally kills an individual while driving in a posted school
zone, as defined in Section 11-605 of the Illinois Vehicle Code, while
children are present or in a construction or maintenance zone, as
defined in Section 11-605.1 of the Illinois Vehicle Code, when
construction or maintenance workers are present the trier of fact may
infer that the defendant's actions were performed recklessly where he or
she was also either driving at a speed of more than 20 miles per hour in
excess of the posted speed limit or violating Section 11-501 of the
Illinois Vehicle Code.

(e-12) Except as otherwise provided in subsection (e-13), in cases
involving reckless homicide in which the offense was committed as result
of a violation of subsection (c) of Section 11-907 of the Illinois
Vehicle Code, the penalty is a Class 2 felony, for which a person, if
sentenced to a term of imprisonment, shall be sentenced to a term of not
less than 3 years and not more than 14 years.

(e-13) In cases involving reckless homicide in which the offense was
committed as result of a violation of subsection (c) of Section 11-907
of the Illinois Vehicle Code and the defendant caused the deaths of 2 or
more persons as part of a single course of conduct, the penalty is a
Class 2 felony, for which a person, if sentenced to a term of
imprisonment, shall be sentenced to a term of not less than 6 years and
not more than 28 years.

(e-14) In cases involving reckless homicide in which the defendant
unintentionally kills an individual, the trier of fact may infer that
the defendant's actions were performed recklessly where he or she was
also violating subsection (c) of Section 11-907 of the Illinois Vehicle
Code. The penalty for a reckless homicide in which the driver also
violated subsection (c) of Section 11-907 of the Illinois Vehicle Code
is a Class 2 felony, for which a person, if sentenced to a term of
imprisonment, shall be sentenced to a term of not less than 3 years and
not more than 14 years.

(e-15) In cases involving reckless homicide in which the defendant was
operating a vehicle while failing or refusing to comply with subsection
(c) of Section 11-907 of the Illinois Vehicle Code resulting in the
death of a firefighter or emergency medical services personnel in the
performance of his or her official duties, the penalty is a Class 2
felony.

(f) In cases involving involuntary manslaughter in which the victim was
a family or household member as defined in paragraph (3) of Section
112A-3 of the Code of Criminal Procedure of 1963, the penalty shall be a
Class 2 felony, for which a person if sentenced to a term of
imprisonment, shall be sentenced to a term of not less than 3 years and
not more than 14 years.

(Source: P.A. 101-173, eff. 1-1-20 .)

\hypertarget{ilcs-59-3.1-from-ch.-38-par.-9-3.1}{%
\subsection*{(720 ILCS 5/9-3.1) (from Ch. 38, par.
9-3.1)}\label{ilcs-59-3.1-from-ch.-38-par.-9-3.1}}
\addcontentsline{toc}{subsection}{(720 ILCS 5/9-3.1) (from Ch. 38, par.
9-3.1)}

\hypertarget{sec.-9-3.1.-renumbered.}{%
\section*{Sec. 9-3.1. (Renumbered).}\label{sec.-9-3.1.-renumbered.}}
\addcontentsline{toc}{section}{Sec. 9-3.1. (Renumbered).}

\markright{Sec. 9-3.1. (Renumbered).}

(Source: Renumbered by P.A. 96-710, eff. 1-1-10.)

\hypertarget{ilcs-59-3-1.5}{%
\subsection*{(720 ILCS 5/9-3-1.5)}\label{ilcs-59-3-1.5}}
\addcontentsline{toc}{subsection}{(720 ILCS 5/9-3-1.5)}

\hypertarget{sec.-9-3-1.5.-renumbered-as-section-9-3.5.}{%
\section*{Sec. 9-3-1.5. (Renumbered as Section
9-3.5).}\label{sec.-9-3-1.5.-renumbered-as-section-9-3.5.}}
\addcontentsline{toc}{section}{Sec. 9-3-1.5. (Renumbered as Section
9-3.5).}

\markright{Sec. 9-3-1.5. (Renumbered as Section 9-3.5).}

(Source: Renumbered by P.A. 97-333, eff. 8-12-11.)

\hypertarget{ilcs-59-3.2-from-ch.-38-par.-9-3.2}{%
\subsection*{(720 ILCS 5/9-3.2) (from Ch. 38, par.
9-3.2)}\label{ilcs-59-3.2-from-ch.-38-par.-9-3.2}}
\addcontentsline{toc}{subsection}{(720 ILCS 5/9-3.2) (from Ch. 38, par.
9-3.2)}

\hypertarget{sec.-9-3.2.-involuntary-manslaughter-and-reckless-homicide-of-an-unborn-child.}{%
\section*{Sec. 9-3.2. Involuntary manslaughter and reckless homicide of
an unborn
child.}\label{sec.-9-3.2.-involuntary-manslaughter-and-reckless-homicide-of-an-unborn-child.}}
\addcontentsline{toc}{section}{Sec. 9-3.2. Involuntary manslaughter and
reckless homicide of an unborn child.}

\markright{Sec. 9-3.2. Involuntary manslaughter and reckless homicide of
an unborn child.}

(a) A person who unintentionally kills an unborn child without lawful
justification commits involuntary manslaughter of an unborn child if his
acts whether lawful or unlawful which cause the death are such as are
likely to cause death or great bodily harm to some individual, and he
performs them recklessly, except in cases in which the cause of death
consists of the driving of a motor vehicle, in which case the person
commits reckless homicide of an unborn child.

(b) Sentence.

(1) Involuntary manslaughter of an unborn child is a Class 3 felony.

(2) Reckless homicide of an unborn child is a Class 3 felony.

(c) For purposes of this Section, (1) ``unborn child'' shall mean any
individual of the human species from the implantation of an embryo until
birth, and (2) ``person'' shall not include the pregnant individual
whose unborn child is killed.

(d) This Section shall not apply to acts which cause the death of an
unborn child if those acts were committed during any abortion, as
defined in Section 1-10 of the Reproductive Health Act, to which the
pregnant individual has consented. This Section shall not apply to acts
which were committed pursuant to usual and customary standards of
medical practice during diagnostic testing or therapeutic treatment.

(e) The provisions of this Section shall not be construed to prohibit
the prosecution of any person under any other provision of law, nor
shall it be construed to preclude any civil cause of action.

(Source: P.A. 101-13, eff. 6-12-19; 102-558, eff. 8-20-21.)

\hypertarget{ilcs-59-3.3-from-ch.-38-par.-9-3.3}{%
\subsection*{(720 ILCS 5/9-3.3) (from Ch. 38, par.
9-3.3)}\label{ilcs-59-3.3-from-ch.-38-par.-9-3.3}}
\addcontentsline{toc}{subsection}{(720 ILCS 5/9-3.3) (from Ch. 38, par.
9-3.3)}

\hypertarget{sec.-9-3.3.-drug-induced-homicide.}{%
\section*{Sec. 9-3.3. Drug-induced
homicide.}\label{sec.-9-3.3.-drug-induced-homicide.}}
\addcontentsline{toc}{section}{Sec. 9-3.3. Drug-induced homicide.}

\markright{Sec. 9-3.3. Drug-induced homicide.}

(a) A person commits drug-induced homicide when he or she violates
Section 401 of the Illinois Controlled Substances Act or Section 55 of
the Methamphetamine Control and Community Protection Act by unlawfully
delivering a controlled substance to another, and any person's death is
caused by the injection, inhalation, absorption, or ingestion of any
amount of that controlled substance.

(a-5) A person commits drug-induced homicide when he or she violates the
law of another jurisdiction, which if the violation had been committed
in this State could be charged under Section 401 of the Illinois
Controlled Substances Act or Section 55 of the Methamphetamine Control
and Community Protection Act, by unlawfully delivering a controlled
substance to another, and any person's death is caused in this State by
the injection, inhalation, absorption, or ingestion of any amount of
that controlled substance.

(b) Sentence. Drug-induced homicide is a Class X felony, except:

(1) A person who commits drug-induced homicide by violating subsection
(a) or subsection (c) of Section 401 of the Illinois Controlled
Substances Act or Section 55 of the Methamphetamine Control and
Community Protection Act commits a Class X felony for which the
defendant shall in addition to a sentence authorized by law, be
sentenced to a term of imprisonment of not less than 15 years and not
more than 30 years or an extended term of not less than 30 years and not
more than 60 years.

(2) A person who commits drug-induced homicide by violating the law of
another jurisdiction, which if the violation had been committed in this
State could be charged under subsection (a) or subsection (c) of Section
401 of the Illinois Controlled Substances Act or Section 55 of the
Methamphetamine Control and Community Protection Act, commits a Class X
felony for which the defendant shall, in addition to a sentence
authorized by law, be sentenced to a term of imprisonment of not less
than 15 years and not more than 30 years or an extended term of not less
than 30 years and not more than 60 years.

(Source: P.A. 100-404, eff. 1-1-18 .)

\hypertarget{ilcs-59-3.4}{%
\subsection*{(720 ILCS 5/9-3.4)}\label{ilcs-59-3.4}}
\addcontentsline{toc}{subsection}{(720 ILCS 5/9-3.4)}

(was 720 ILCS 5/9-3.1)

\hypertarget{sec.-9-3.4.-concealment-of-homicidal-death.}{%
\section*{Sec. 9-3.4. Concealment of homicidal
death.}\label{sec.-9-3.4.-concealment-of-homicidal-death.}}
\addcontentsline{toc}{section}{Sec. 9-3.4. Concealment of homicidal
death.}

\markright{Sec. 9-3.4. Concealment of homicidal death.}

(a) A person commits the offense of concealment of homicidal death when
he or she knowingly conceals the death of any other person with
knowledge that such other person has died by homicidal means.

(b) Nothing in this Section prevents the defendant from also being
charged with and tried for the first degree murder, second degree
murder, or involuntary manslaughter of the person whose death is
concealed.

(b-5) For purposes of this Section:

``Conceal'' means the performing of some act or acts for the purpose of
preventing or delaying the discovery of a death by homicidal means.
``Conceal'' means something more than simply withholding knowledge or
failing to disclose information.

``Homicidal means'' means any act or acts, lawful or unlawful, of a
person that cause the death of another person.

(c) Sentence. Concealment of homicidal death is a Class 3 felony.

(Source: P.A. 96-710, eff. 1-1-10.)

\hypertarget{ilcs-59-3.5}{%
\subsection*{(720 ILCS 5/9-3.5)}\label{ilcs-59-3.5}}
\addcontentsline{toc}{subsection}{(720 ILCS 5/9-3.5)}

\hypertarget{sec.-9-3.5.-concealment-of-death.}{%
\section*{Sec. 9-3.5. Concealment of
death.}\label{sec.-9-3.5.-concealment-of-death.}}
\addcontentsline{toc}{section}{Sec. 9-3.5. Concealment of death.}

\markright{Sec. 9-3.5. Concealment of death.}

(a) For purposes of this Section, ``conceal'' means the performing of
some act or acts for the purpose of preventing or delaying the discovery
of a death. ``Conceal'' means something more than simply withholding
knowledge or failing to disclose information.

(b) A person commits the offense of concealment of death when he or she
knowingly conceals the death of any other person who died by other than
homicidal means.

(c) A person commits the offense of concealment of death when he or she
knowingly moves the body of a dead person from its place of death, with
the intent of concealing information regarding the place or manner of
death of that person, or the identity of any person with information
regarding the death of that person. This subsection shall not apply to
any movement of the body of a dead person by medical personnel, fire
fighters, law enforcement officers, coroners, medical examiners, or
licensed funeral directors, or by any person acting at the direction of
medical personnel, fire fighters, law enforcement officers, coroners,
medical examiners, or licensed funeral directors.

(d) Sentence. Concealment of death is a Class 4 felony.

(Source: P.A. 96-1361, eff. 1-1-11; 97-333, eff. 8-12-11.)

\bookmarksetup{startatroot}

\hypertarget{article-10.-kidnaping-and-related-offenses}{%
\chapter*{Article 10. Kidnaping And Related
Offenses}\label{article-10.-kidnaping-and-related-offenses}}
\addcontentsline{toc}{chapter}{Article 10. Kidnaping And Related
Offenses}

\markboth{Article 10. Kidnaping And Related Offenses}{Article 10.
Kidnaping And Related Offenses}

\hypertarget{ilcs-510-1-from-ch.-38-par.-10-1}{%
\subsection*{(720 ILCS 5/10-1) (from Ch. 38, par.
10-1)}\label{ilcs-510-1-from-ch.-38-par.-10-1}}
\addcontentsline{toc}{subsection}{(720 ILCS 5/10-1) (from Ch. 38, par.
10-1)}

\hypertarget{sec.-10-1.-kidnapping.}{%
\section*{Sec. 10-1. Kidnapping.}\label{sec.-10-1.-kidnapping.}}
\addcontentsline{toc}{section}{Sec. 10-1. Kidnapping.}

\markright{Sec. 10-1. Kidnapping.}

(a) A person commits the offense of kidnapping when he or she knowingly:

(1) and secretly confines another against his or her will;

(2) by force or threat of imminent force carries another from one place
to another with intent secretly to confine that other person against his
or her will; or

(3) by deceit or enticement induces another to go from one place to
another with intent secretly to confine that other person against his or
her will.

(b) Confinement of a child under the age of 13 years, or of a person
with a severe or profound intellectual disability, is against that
child's or person's will within the meaning of this Section if that
confinement is without the consent of that child's or person's parent or
legal guardian.

(c) Sentence. Kidnapping is a Class 2 felony.

(Source: P.A. 99-143, eff. 7-27-15.)

\hypertarget{ilcs-510-2-from-ch.-38-par.-10-2}{%
\subsection*{(720 ILCS 5/10-2) (from Ch. 38, par.
10-2)}\label{ilcs-510-2-from-ch.-38-par.-10-2}}
\addcontentsline{toc}{subsection}{(720 ILCS 5/10-2) (from Ch. 38, par.
10-2)}

\hypertarget{sec.-10-2.-aggravated-kidnaping.}{%
\section*{Sec. 10-2. Aggravated
kidnaping.}\label{sec.-10-2.-aggravated-kidnaping.}}
\addcontentsline{toc}{section}{Sec. 10-2. Aggravated kidnaping.}

\markright{Sec. 10-2. Aggravated kidnaping.}

(a) A person commits the offense of aggravated kidnaping when he or she
commits kidnapping and:

(1) kidnaps with the intent to obtain ransom from the person kidnaped or
from any other person;

(2) takes as his or her victim a child under the age of 13 years, or a
person with a severe or profound intellectual disability;

(3) inflicts great bodily harm, other than by the discharge of a
firearm, or commits another felony upon his or her victim;

(4) wears a hood, robe, or mask or conceals his or her identity;

(5) commits the offense of kidnaping while armed with a dangerous
weapon, other than a firearm, as defined in Section 33A-1 of this Code;

(6) commits the offense of kidnaping while armed with a firearm;

(7) during the commission of the offense of kidnaping, personally
discharges a firearm; or

(8) during the commission of the offense of kidnaping, personally
discharges a firearm that proximately causes great bodily harm,
permanent disability, permanent disfigurement, or death to another
person.

As used in this Section, ``ransom'' includes money, benefit, or other
valuable thing or concession.

(b) Sentence. Aggravated kidnaping in violation of paragraph (1), (2),
(3), (4), or (5) of subsection (a) is a Class X felony. A violation of
subsection (a)(6) is a Class X felony for which 15 years shall be added
to the term of imprisonment imposed by the court. A violation of
subsection (a)(7) is a Class X felony for which 20 years shall be added
to the term of imprisonment imposed by the court. A violation of
subsection (a)(8) is a Class X felony for which 25 years or up to a term
of natural life shall be added to the term of imprisonment imposed by
the court. An offender under the age of 18 years at the time of the
commission of aggravated kidnaping in violation of paragraphs (1)
through (8) of subsection (a) shall be sentenced under Section 5-4.5-105
of the Unified Code of Corrections.

A person who has attained the age of 18 years at the time of the
commission of the offense and who is convicted of a second or subsequent
offense of aggravated kidnaping shall be sentenced to a term of natural
life imprisonment; except that a sentence of natural life imprisonment
shall not be imposed under this Section unless the second or subsequent
offense was committed after conviction on the first offense. An offender
under the age of 18 years at the time of the commission of the second or
subsequent offense shall be sentenced under Section 5-4.5-105 of the
Unified Code of Corrections.

(Source: P.A. 99-69, eff. 1-1-16; 99-143, eff. 7-27-15; 99-642, eff.
7-28-16.)

\hypertarget{ilcs-510-3-from-ch.-38-par.-10-3}{%
\subsection*{(720 ILCS 5/10-3) (from Ch. 38, par.
10-3)}\label{ilcs-510-3-from-ch.-38-par.-10-3}}
\addcontentsline{toc}{subsection}{(720 ILCS 5/10-3) (from Ch. 38, par.
10-3)}

\hypertarget{sec.-10-3.-unlawful-restraint.}{%
\section*{Sec. 10-3. Unlawful
restraint.}\label{sec.-10-3.-unlawful-restraint.}}
\addcontentsline{toc}{section}{Sec. 10-3. Unlawful restraint.}

\markright{Sec. 10-3. Unlawful restraint.}

(a) A person commits the offense of unlawful restraint when he or she
knowingly without legal authority detains another.

(b) Sentence. Unlawful restraint is a Class 4 felony.

(Source: P.A. 96-710, eff. 1-1-10.)

\hypertarget{ilcs-510-3.1-from-ch.-38-par.-10-3.1}{%
\subsection*{(720 ILCS 5/10-3.1) (from Ch. 38, par.
10-3.1)}\label{ilcs-510-3.1-from-ch.-38-par.-10-3.1}}
\addcontentsline{toc}{subsection}{(720 ILCS 5/10-3.1) (from Ch. 38, par.
10-3.1)}

\hypertarget{sec.-10-3.1.-aggravated-unlawful-restraint.}{%
\section*{Sec. 10-3.1. Aggravated unlawful
restraint.}\label{sec.-10-3.1.-aggravated-unlawful-restraint.}}
\addcontentsline{toc}{section}{Sec. 10-3.1. Aggravated unlawful
restraint.}

\markright{Sec. 10-3.1. Aggravated unlawful restraint.}

(a) A person commits the offense of aggravated unlawful restraint when
he or she commits unlawful restraint while using a deadly weapon.

(b) Sentence. Aggravated unlawful restraint is a Class 3 felony.

(Source: P.A. 96-710, eff. 1-1-10.)

\hypertarget{ilcs-510-4-from-ch.-38-par.-10-4}{%
\subsection*{(720 ILCS 5/10-4) (from Ch. 38, par.
10-4)}\label{ilcs-510-4-from-ch.-38-par.-10-4}}
\addcontentsline{toc}{subsection}{(720 ILCS 5/10-4) (from Ch. 38, par.
10-4)}

\hypertarget{sec.-10-4.-forcible-detention.}{%
\section*{Sec. 10-4. Forcible
Detention.}\label{sec.-10-4.-forcible-detention.}}
\addcontentsline{toc}{section}{Sec. 10-4. Forcible Detention.}

\markright{Sec. 10-4. Forcible Detention.}

\begin{enumerate}
\def\labelenumi{(\alph{enumi})}
\tightlist
\item
  A person commits the offense of forcible detention when he holds an
  individual hostage without lawful authority for the purpose of
  obtaining performance by a third person of demands made by the person
  holding the hostage, and

  (1) the person holding the hostage is armed with a dangerous weapon as
  defined in Section 33A-1 of this Code, or

  (2) the hostage is known to the person holding him to be a peace
  officer or a correctional employee engaged in the performance of his
  official duties.

  (b) Forcible detention is a Class 2 felony.

  (Source: P.A. 79-941.)

  \hypertarget{ilcs-510-5-from-ch.-38-par.-10-5}{%
  \subsection*{(720 ILCS 5/10-5) (from Ch. 38, par.
  10-5)}\label{ilcs-510-5-from-ch.-38-par.-10-5}}
  \addcontentsline{toc}{subsection}{(720 ILCS 5/10-5) (from Ch. 38, par.
  10-5)}

  \hypertarget{sec.-10-5.-child-abduction.}{%
  \section*{Sec. 10-5. Child
  abduction.}\label{sec.-10-5.-child-abduction.}}
  \addcontentsline{toc}{section}{Sec. 10-5. Child abduction.}

  \markright{Sec. 10-5. Child abduction.}

  (a) For purposes of this Section, the following terms have the
  following meanings:

  (1) ``Child'' means a person who, at the time the alleged violation
  occurred, was under the age of 18 or was a person with a severe or
  profound intellectual disability.

  (2) ``Detains'' means taking or retaining physical custody of a child,
  whether or not the child resists or objects.

  (2.1) ``Express consent'' means oral or written permission that is
  positive, direct, and unequivocal, requiring no inference or
  implication to supply its meaning.

  (2.2) ``Luring'' means any knowing act to solicit, entice, tempt, or
  attempt to attract the minor.

  (3) ``Lawful custodian'' means a person or persons granted legal
  custody of a child or entitled to physical possession of a child
  pursuant to a court order. It is presumed that, when the parties have
  never been married to each other, the mother has legal custody of the
  child unless a valid court order states otherwise. If an adjudication
  of paternity has been completed and the father has been assigned
  support obligations or visitation rights, such a paternity order
  should, for the purposes of this Section, be considered a valid court
  order granting custody to the mother.

  (4) ``Putative father'' means a man who has a reasonable belief that
  he is the father of a child born of a woman who is not his wife.

  (5) ``Unlawful purpose'' means any misdemeanor or felony violation of
  State law or a similar federal or sister state law or local ordinance.

  (b) A person commits the offense of child abduction when he or she
  does any one of the following:

  (1) Intentionally violates any terms of a valid court order granting
  sole or joint custody, care, or possession to another by concealing or
  detaining the child or removing the child from the jurisdiction of the
  court.

  (2) Intentionally violates a court order prohibiting the person from
  concealing or detaining the child or removing the child from the
  jurisdiction of the court.

  (3) Intentionally conceals, detains, or removes the child without the
  consent of the mother or lawful custodian of the child if the person
  is a putative father and either: (A) the paternity of the child has
  not been legally established or (B) the paternity of the child has
  been legally established but no orders relating to custody have been
  entered. Notwithstanding the presumption created by paragraph (3) of
  subsection (a), however, a mother commits child abduction when she
  intentionally conceals or removes a child, whom she has abandoned or
  relinquished custody of, from an unadjudicated father who has provided
  sole ongoing care and custody of the child in her absence.

  (4) Intentionally conceals or removes the child from a parent after
  filing a petition or being served with process in an action affecting
  marriage or paternity but prior to the issuance of a temporary or
  final order determining custody.

  (5) At the expiration of visitation rights outside the State,
  intentionally fails or refuses to return or impedes the return of the
  child to the lawful custodian in Illinois.

  (6) Being a parent of the child, and if the parents of that child are
  or have been married and there has been no court order of custody,
  knowingly conceals the child for 15 days, and fails to make reasonable
  attempts within the 15-day period to notify the other parent as to the
  specific whereabouts of the child, including a means by which to
  contact the child, or to arrange reasonable visitation or contact with
  the child. It is not a violation of this Section for a person fleeing
  domestic violence to take the child with him or her to housing
  provided by a domestic violence program.

  (7) Being a parent of the child, and if the parents of the child are
  or have been married and there has been no court order of custody,
  knowingly conceals, detains, or removes the child with physical force
  or threat of physical force.

  (8) Knowingly conceals, detains, or removes the child for payment or
  promise of payment at the instruction of a person who has no legal
  right to custody.

  (9) Knowingly retains in this State for 30 days a child removed from
  another state without the consent of the lawful custodian or in
  violation of a valid court order of custody.

  (10) Intentionally lures or attempts to lure a child:

  (A) under the age of 17 or (B) while traveling to or from a primary or
  secondary school into a motor vehicle, building, housetrailer, or
  dwelling place without the consent of the child's parent or lawful
  custodian for other than a lawful purpose. For the purposes of this
  item (10), the trier of fact may infer that luring or attempted luring
  of a child under the age of 17 into a motor vehicle, building,
  housetrailer, or dwelling place without the express consent of the
  child's parent or lawful custodian or with the intent to avoid the
  express consent of the child's parent or lawful custodian was for
  other than a lawful purpose.

  (11) With the intent to obstruct or prevent efforts to locate the
  child victim of a child abduction, knowingly destroys, alters,
  conceals, or disguises physical evidence or furnishes false
  information.

  (c) It is an affirmative defense to subsections (b)(1) through (b)(10)
  of this Section that:

  (1) the person had custody of the child pursuant to a court order
  granting legal custody or visitation rights that existed at the time
  of the alleged violation;

  (2) the person had physical custody of the child pursuant to a court
  order granting legal custody or visitation rights and failed to return
  the child as a result of circumstances beyond his or her control, and
  the person notified and disclosed to the other parent or legal
  custodian the specific whereabouts of the child and a means by which
  the child could be contacted or made a reasonable attempt to notify
  the other parent or lawful custodian of the child of those
  circumstances and made the disclosure within 24 hours after the
  visitation period had expired and returned the child as soon as
  possible;

  (3) the person was fleeing an incidence or pattern of domestic
  violence; or

  (4) the person lured or attempted to lure a child under the age of 17
  into a motor vehicle, building, housetrailer, or dwelling place for a
  lawful purpose in prosecutions under paragraph (10) of subsection (b).

  (d) A person convicted of child abduction under this Section is guilty
  of a Class 4 felony. A person convicted of child abduction under
  subsection (b)(10) shall undergo a sex offender evaluation prior to a
  sentence being imposed. A person convicted of a second or subsequent
  violation of paragraph (10) of subsection (b) of this Section is
  guilty of a Class 3 felony. A person convicted of child abduction
  under subsection (b)(10) when the person has a prior conviction of a
  sex offense as defined in the Sex Offender Registration Act or any
  substantially similar federal, Uniform Code of Military Justice,
  sister state, or foreign government offense is guilty of a Class 2
  felony. It is a factor in aggravation under subsections (b)(1) through
  (b)(10) of this Section for which a court may impose a more severe
  sentence under Section 5-8-1 (730 ILCS 5/5-8-1) or Article 4.5 of
  Chapter V of the Unified Code of Corrections if, upon sentencing, the
  court finds evidence of any of the following aggravating factors:

  (1) that the defendant abused or neglected the child following the
  concealment, detention, or removal of the child;

  (2) that the defendant inflicted or threatened to inflict physical
  harm on a parent or lawful custodian of the child or on the child with
  intent to cause that parent or lawful custodian to discontinue
  criminal prosecution of the defendant under this Section;

  (3) that the defendant demanded payment in exchange for return of the
  child or demanded that he or she be relieved of the financial or legal
  obligation to support the child in exchange for return of the child;

  (4) that the defendant has previously been convicted of child
  abduction;

  (5) that the defendant committed the abduction while armed with a
  deadly weapon or the taking of the child resulted in serious bodily
  injury to another; or

  (6) that the defendant committed the abduction while in a school,
  regardless of the time of day or time of year; in a playground; on any
  conveyance owned, leased, or contracted by a school to transport
  students to or from school or a school related activity; on the real
  property of a school; or on a public way within 1,000 feet of the real
  property comprising any school or playground. For purposes of this
  paragraph (6), ``playground'' means a piece of land owned or
  controlled by a unit of local government that is designated by the
  unit of local government for use solely or primarily for children's
  recreation; and ``school'' means a public or private elementary or
  secondary school, community college, college, or university.

  (e) The court may order the child to be returned to the parent or
  lawful custodian from whom the child was concealed, detained, or
  removed. In addition to any sentence imposed, the court may assess any
  reasonable expense incurred in searching for or returning the child
  against any person convicted of violating this Section.

  (f) Nothing contained in this Section shall be construed to limit the
  court's contempt power.

  (g) Every law enforcement officer investigating an alleged incident of
  child abduction shall make a written police report of any bona fide
  allegation and the disposition of that investigation. Every police
  report completed pursuant to this Section shall be compiled and
  recorded within the meaning of Section 5.1 of the Criminal
  Identification Act.

  (h) Whenever a law enforcement officer has reasons to believe a child
  abduction has occurred, she or he shall provide the lawful custodian a
  summary of her or his rights under this Code, including the procedures
  and relief available to her or him.

  (i) If during the course of an investigation under this Section the
  child is found in the physical custody of the defendant or another,
  the law enforcement officer shall return the child to the parent or
  lawful custodian from whom the child was concealed, detained, or
  removed, unless there is good cause for the law enforcement officer or
  the Department of Children and Family Services to retain temporary
  protective custody of the child pursuant to the Abused and Neglected
  Child Reporting Act.

  (Source:

  P.A. 99-143, eff. 7-27-15.)

  \hypertarget{ilcs-510-5.1}{%
  \subsection*{(720 ILCS 5/10-5.1)}\label{ilcs-510-5.1}}
  \addcontentsline{toc}{subsection}{(720 ILCS 5/10-5.1)}

  \hypertarget{sec.-10-5.1.-luring-of-a-minor.}{%
  \section*{Sec. 10-5.1. Luring of a
  minor.}\label{sec.-10-5.1.-luring-of-a-minor.}}
  \addcontentsline{toc}{section}{Sec. 10-5.1. Luring of a minor.}

  \markright{Sec. 10-5.1. Luring of a minor.}

  (a) A person commits the offense of luring of a minor when the
  offender is 21 years of age or older and knowingly contacts or
  communicates electronically to the minor:

  (1) knowing the minor is under 15 years of age;

  (2) with the intent to persuade, lure or transport the minor away from
  his or her home, or other location known by the minor's parent or
  legal guardian to be the place where the minor is to be located;

  (3) for an unlawful purpose;

  (4) without the express consent of the person's parent or legal
  guardian;

  (5) with the intent to avoid the express consent of the person's
  parent or legal guardian;

  (6) after so communicating, commits any act in furtherance of the
  intent described in clause (a)(2); and

  (7) is a stranger to the parents or legal guardian of the minor.

  (b) A person commits the offense of luring of a minor when the
  offender is at least 18 years of age but under 21 years of age and
  knowingly contacts or communicates electronically to the minor:

  (1) knowing the minor is under 15 years of age;

  (2) with the intent to persuade, lure, or transport the minor away
  from his or her home or other location known by the minor's parent or
  legal guardian, to be the place where the minor is to be located;

  (3) for an unlawful purpose;

  (4) without the express consent of the person's parent or legal
  guardian;

  (5) with the intent to avoid the express consent of the person's
  parent or legal guardian;

  (6) after so communicating, commits any act in furtherance of the
  intent described in clause (b)(2); and

  (7) is a stranger to the parents or legal guardian of the minor.

  (c) Definitions. For purposes of this Section:

  (1) ``Emergency situation'' means a situation in which the minor is
  threatened with imminent bodily harm, emotional harm or psychological
  harm.

  (2) ``Express consent'' means oral or written permission that is
  positive, direct, and unequivocal, requiring no inference or
  implication to supply its meaning.

  (3) ``Contacts or communicates electronically'' includes but is not
  limited to, any attempt to make contact or communicate telephonically
  or through the Internet or text messages.

  (4) ``Luring'' shall mean any knowing act to solicit, entice, tempt,
  or attempt to attract the minor.

  (5) ``Minor'' shall mean any person under the age of 15.

  (6) ``Stranger'' shall have its common and ordinary meaning, including
  but not limited to, a person that is either not known by the parents
  of the minor or does not have any association with the parents of the
  minor.

  (7) ``Unlawful purpose'' shall mean any misdemeanor or felony
  violation of State law or a similar federal or sister state law or
  local ordinance.

  (d) This Section may not be interpreted to criminalize an act or
  person contacting a minor within the scope and course of his
  employment, or status as a volunteer of a recognized civic, charitable
  or youth organization.

  (e) This Section is intended to protect minors and to help parents and
  legal guardians exercise reasonable care, supervision, protection, and
  control over minor children.

  (f) Affirmative defenses.

  (1) It shall be an affirmative defense to any offense under this
  Section 10-5.1 that the accused reasonably believed that the minor was
  over the age of 15.

  (2) It shall be an affirmative defense to any offense under this
  Section 10-5.1 that the accused is assisting the minor in an emergency
  situation.

  (3) It shall not be a defense to the prosecution of any offense under
  this Section 10-5.1 if the person who is contacted by the offender is
  posing as a minor and is in actuality an adult law enforcement
  officer.

  (g) Penalties.

  (1) A first offense of luring of a minor under subsection (a) shall be
  a Class 4 felony. A person convicted of luring of a minor under
  subsection (a) shall undergo a sex offender evaluation prior to a
  sentence being imposed. An offense of luring of a minor under
  subsection (a) when a person has a prior conviction in Illinois of a
  sex offense as defined in the Sex Offender Registration Act, or any
  substantially similar federal, Uniform Code of Military Justice,
  sister state, or foreign government offense, is guilty of a Class 2
  felony.

  (2) A first offense of luring of a minor under subsection (b) is a
  Class B misdemeanor.

  (3) A second or subsequent offense of luring of a minor under
  subsection (a) is a Class 3 felony. A second or subsequent offense of
  luring of a minor under subsection (b) is a Class 4 felony. A second
  or subsequent offense when a person has a prior conviction in Illinois
  of a sex offense as defined in the Sex Offender Registration Act, or
  any substantially similar federal, Uniform Code of Military Justice,
  sister state, or foreign government offense, is a Class 1 felony. A
  defendant convicted a second time of an offense under subsection (a)
  or (b) shall register as a sexual predator of children pursuant to the
  Sex Offender Registration Act.

  (4) A third or subsequent offense is a Class 1 felony. A third or
  subsequent offense when a person has a prior conviction in Illinois of
  a sex offense as defined in the Sex Offender Registration Act, or any
  substantially similar federal, Uniform Code of Military Justice,
  sister state, or foreign government offense, is a Class X felony.

  (h) For violations of subsection (a), jurisdiction shall be
  established if the transmission that constitutes the offense either
  originates in this State or is received in this State and does not
  apply to emergency situations. For violations of subsection (b),
  jurisdiction shall be established in any county where the act in
  furtherance of the commission of the offense is committed, in the
  county where the minor resides, or in the county where the offender
  resides.

  (Source: P.A. 95-625, eff. 6-1-08 .)

  \hypertarget{ilcs-510-5.5}{%
  \subsection*{(720 ILCS 5/10-5.5)}\label{ilcs-510-5.5}}
  \addcontentsline{toc}{subsection}{(720 ILCS 5/10-5.5)}

  \hypertarget{sec.-10-5.5.-unlawful-visitation-or-parenting-time-interference.}{%
  \section*{Sec. 10-5.5. Unlawful visitation or parenting time
  interference.}\label{sec.-10-5.5.-unlawful-visitation-or-parenting-time-interference.}}
  \addcontentsline{toc}{section}{Sec. 10-5.5. Unlawful visitation or
  parenting time interference.}

  \markright{Sec. 10-5.5. Unlawful visitation or parenting time
  interference.}

  (a) As used in this Section, the terms ``child'', ``detain'', and
  ``lawful custodian'' have the meanings ascribed to them in Section
  10-5 of this Code.

  (b) Every person who, in violation of the visitation, parenting time,
  or custody time provisions of a court order relating to child custody,
  detains or conceals a child with the intent to deprive another person
  of his or her rights to visitation, parenting time, or custody time
  commits the offense of unlawful visitation or parenting time
  interference.

  (c) A person committing unlawful visitation or parenting time
  interference is guilty of a petty offense. Any person violating this
  Section after 2 prior convictions of unlawful visitation interference
  or unlawful visitation or parenting time interference, however, is
  guilty of a Class A misdemeanor.

  (d) Any law enforcement officer who has probable cause to believe that
  a person has committed or is committing an act in violation of this
  Section shall issue to that person a notice to appear.

  (e) The notice shall:

  (1) be in writing;

  (2) state the name of the person and his or her address, if known;

  (3) set forth the nature of the offense;

  (4) be signed by the officer issuing the notice; and

  (5) request the person to appear before a court at a certain time and
  place.

  (f) Upon failure of the person to appear, a summons or warrant of
  arrest may be issued.

  (g) It is an affirmative defense that:

  (1) a person or lawful custodian committed the act to protect the
  child from imminent physical harm, provided that the defendant's
  belief that there was physical harm imminent was reasonable and that
  the defendant's conduct in withholding visitation rights, parenting
  time, or custody time was a reasonable response to the harm believed
  imminent;

  (2) the act was committed with the mutual consent of all parties
  having a right to custody and visitation of the child or parenting
  time with the child; or

  (3) the act was otherwise authorized by law.

  (Source: P.A. 96-333, eff. 8-11-09; 96-675, eff. 8-25-09; 96-710, eff.
  1-1-10; 96-1000, eff. 7-2-10.)

  \hypertarget{ilcs-510-6-from-ch.-38-par.-10-6}{%
  \subsection*{(720 ILCS 5/10-6) (from Ch. 38, par.
  10-6)}\label{ilcs-510-6-from-ch.-38-par.-10-6}}
  \addcontentsline{toc}{subsection}{(720 ILCS 5/10-6) (from Ch. 38, par.
  10-6)}

  \hypertarget{sec.-10-6.-harboring-a-runaway.}{%
  \section*{Sec. 10-6. Harboring a
  runaway.}\label{sec.-10-6.-harboring-a-runaway.}}
  \addcontentsline{toc}{section}{Sec. 10-6. Harboring a runaway.}

  \markright{Sec. 10-6. Harboring a runaway.}

  (a) Any person, other than an agency or association providing crisis
  intervention services as defined in Section 3-5 of the Juvenile Court
  Act of 1987, or an operator of a youth emergency shelter as defined in
  Section 2.21 of the Child Care Act of 1969, who, without the knowledge
  and consent of the minor's parent or guardian, knowingly gives shelter
  to a minor, other than a mature minor who has been emancipated under
  the Emancipation of Minors Act, for more than 48 hours without the
  consent of the minor's parent or guardian, and without notifying the
  local law enforcement authorities of the minor's name and the fact
  that the minor is being provided shelter commits the offense of
  harboring a runaway.

  (b) Any person who commits the offense of harboring a runaway is
  guilty of a Class A misdemeanor.

  (Source: P.A. 95-331, eff. 8-21-07.)

  \hypertarget{ilcs-510-7-from-ch.-38-par.-10-7}{%
  \subsection*{(720 ILCS 5/10-7) (from Ch. 38, par.
  10-7)}\label{ilcs-510-7-from-ch.-38-par.-10-7}}
  \addcontentsline{toc}{subsection}{(720 ILCS 5/10-7) (from Ch. 38, par.
  10-7)}

  \hypertarget{sec.-10-7.-aiding-or-abetting-child-abduction.}{%
  \section*{Sec. 10-7. Aiding or abetting child
  abduction.}\label{sec.-10-7.-aiding-or-abetting-child-abduction.}}
  \addcontentsline{toc}{section}{Sec. 10-7. Aiding or abetting child
  abduction.}

  \markright{Sec. 10-7. Aiding or abetting child abduction.}

  (a) A person violates this Section when, before or during the
  commission of a child abduction as defined in Section 10-5 and with
  the intent to promote or facilitate such offense, he or she
  intentionally aids or abets another in the planning or commission of
  child abduction, unless before the commission of the offense he or she
  makes proper effort to prevent the commission of the offense.

  (b) Sentence. A person who violates this Section commits a Class 4
  felony.

  (Source: P.A. 96-710, eff. 1-1-10.)

  \hypertarget{ilcs-510-8-from-ch.-38-par.-10-8}{%
  \subsection*{(720 ILCS 5/10-8) (from Ch. 38, par.
  10-8)}\label{ilcs-510-8-from-ch.-38-par.-10-8}}
  \addcontentsline{toc}{subsection}{(720 ILCS 5/10-8) (from Ch. 38, par.
  10-8)}

  \hypertarget{sec.-10-8.-unlawful-sale-of-a-public-conveyance-travel-ticket-to-a-minor.}{%
  \section*{Sec. 10-8. Unlawful sale of a public conveyance travel
  ticket to a
  minor.}\label{sec.-10-8.-unlawful-sale-of-a-public-conveyance-travel-ticket-to-a-minor.}}
  \addcontentsline{toc}{section}{Sec. 10-8. Unlawful sale of a public
  conveyance travel ticket to a minor.}

  \markright{Sec. 10-8. Unlawful sale of a public conveyance travel
  ticket to a minor.}

  (a) A person commits the offense of unlawful sale of a public
  conveyance travel ticket to a minor when the person sells a ticket for
  travel on any public conveyance to an unemancipated minor under 17
  years of age without the consent of the minor's parents or guardian
  for passage to a destination outside this state and knows the minor's
  age or fails to take reasonable measures to ascertain the minor's age.

  (b) Evidence. The fact that the defendant demanded, was shown, and
  reasonably relied upon written evidence of a person's age in any
  transaction forbidden by this Section is competent evidence, and may
  be considered in any criminal prosecution for a violation of this
  Section.

  (c) Definition. ``Public Conveyance'', includes an airplane, boat,
  bus, railroad, train, taxicab or other vehicle used for the
  transportation of passengers for hire.

  (d) Sentence. Unlawful sale of a public conveyance travel ticket to a
  minor is a Class C misdemeanor.

  (Source: P.A. 86-336 .)

  \hypertarget{ilcs-510-8.1}{%
  \subsection*{(720 ILCS 5/10-8.1)}\label{ilcs-510-8.1}}
  \addcontentsline{toc}{subsection}{(720 ILCS 5/10-8.1)}

  \hypertarget{sec.-10-8.1.-unlawful-sending-of-a-public-conveyance-travel-ticket-to-a-minor.}{%
  \section*{Sec. 10-8.1. Unlawful sending of a public conveyance travel
  ticket to a
  minor.}\label{sec.-10-8.1.-unlawful-sending-of-a-public-conveyance-travel-ticket-to-a-minor.}}
  \addcontentsline{toc}{section}{Sec. 10-8.1. Unlawful sending of a
  public conveyance travel ticket to a minor.}

  \markright{Sec. 10-8.1. Unlawful sending of a public conveyance travel
  ticket to a minor.}

  (a) In this Section, ``public conveyance'' has the meaning ascribed to
  it in Section 10-8 of this Code.

  (b) A person commits the offense of unlawful sending of a public
  conveyance travel ticket to a minor when the person without the
  consent of the minor's parent or guardian:

  (1) knowingly sends, causes to be sent, or purchases a public
  conveyance travel ticket to any location for a person known by the
  offender to be an unemancipated minor under 17 years of age or a
  person he or she believes to be a minor under 17 years of age, other
  than for a lawful purpose under Illinois law; or

  (2) knowingly arranges for travel to any location on any public
  conveyance for a person known by the offender to be an unemancipated
  minor under 17 years of age or a person he or she believes to be a
  minor under 17 years of age, other than for a lawful purpose under
  Illinois law.

  (b-5) Telecommunications carriers, commercial mobile service
  providers, and providers of information services, including, but not
  limited to, Internet service providers and hosting service providers,
  are not liable under this Section, except for willful and wanton
  misconduct, by virtue of the transmission, storage, or caching of
  electronic communications or messages of others or by virtue of the
  provision of other related telecommunications, commercial mobile
  services, or information services used by others in violation of this
  Section.

  (c) Sentence. Unlawful sending of a public conveyance travel ticket to
  a minor is a Class A misdemeanor. A person who commits unlawful
  sending of a public conveyance travel ticket to a minor who believes
  that he or she is at least 5 years older than the minor is guilty of a
  Class 4 felony.

  (Source: P.A. 95-983, eff. 6-1-09 .)

  \hypertarget{ilcs-510-9}{%
  \subsection*{(720 ILCS 5/10-9)}\label{ilcs-510-9}}
  \addcontentsline{toc}{subsection}{(720 ILCS 5/10-9)}

  \hypertarget{sec.-10-9.-trafficking-in-persons-involuntary-servitude-and-related-offenses.}{%
  \section*{Sec. 10-9. Trafficking in persons, involuntary servitude,
  and related
  offenses.}\label{sec.-10-9.-trafficking-in-persons-involuntary-servitude-and-related-offenses.}}
  \addcontentsline{toc}{section}{Sec. 10-9. Trafficking in persons,
  involuntary servitude, and related offenses.}

  \markright{Sec. 10-9. Trafficking in persons, involuntary servitude,
  and related offenses.}

  (a) Definitions. In this Section:

  (1) ``Intimidation'' has the meaning prescribed in Section 12-6.

  (2) ``Commercial sexual activity'' means any sex act on account of
  which anything of value is given, promised to, or received by any
  person.

  (2.5) ``Company'' means any sole proprietorship, organization,
  association, corporation, partnership, joint venture, limited
  partnership, limited liability partnership, limited liability limited
  partnership, limited liability company, or other entity or business
  association, including all wholly owned subsidiaries, majority-owned
  subsidiaries, parent companies, or affiliates of those entities or
  business associations, that exist for the purpose of making profit.

  (3) ``Financial harm'' includes intimidation that brings about
  financial loss, criminal usury, or employment contracts that violate
  the Frauds Act.

  (4) (Blank).

  (5) ``Labor'' means work of economic or financial value.

  (6) ``Maintain'' means, in relation to labor or services, to secure
  continued performance thereof, regardless of any initial agreement on
  the part of the victim to perform that type of service.

  (7) ``Obtain'' means, in relation to labor or services, to secure
  performance thereof.

  (7.5) ``Serious harm'' means any harm, whether physical or
  nonphysical, including psychological, financial, or reputational harm,
  that is sufficiently serious, under all the surrounding circumstances,
  to compel a reasonable person of the same background and in the same
  circumstances to perform or to continue performing labor or services
  in order to avoid incurring that harm.

  (8) ``Services'' means activities resulting from a relationship
  between a person and the actor in which the person performs activities
  under the supervision of or for the benefit of the actor. Commercial
  sexual activity and sexually-explicit performances are forms of
  activities that are ``services'' under this Section. Nothing in this
  definition may be construed to legitimize or legalize prostitution.

  (9) ``Sexually-explicit performance'' means a live, recorded,
  broadcast (including over the Internet), or public act or show
  intended to arouse or satisfy the sexual desires or appeal to the
  prurient interests of patrons.

  (10) ``Trafficking victim'' means a person subjected to the practices
  set forth in subsection (b), (c), or (d).

  (b) Involuntary servitude. A person commits involuntary servitude when
  he or she knowingly subjects, attempts to subject, or engages in a
  conspiracy to subject another person to labor or services obtained or
  maintained through any of the following means, or any combination of
  these means:

  (1) causes or threatens to cause physical harm to any person;

  (2) physically restrains or threatens to physically restrain another
  person;

  (3) abuses or threatens to abuse the law or legal process;

  (4) knowingly destroys, conceals, removes, confiscates, or possesses
  any actual or purported passport or other immigration document, or any
  other actual or purported government identification document, of
  another person;

  (5) uses intimidation, or exerts financial control over any person; or

  (6) uses any scheme, plan, or pattern intended to cause the person to
  believe that, if the person did not perform the labor or services,
  that person or another person would suffer serious harm or physical
  restraint.

  Sentence. Except as otherwise provided in subsection (e) or (f), a
  violation of subsection (b)(1) is a Class X felony, (b)(2) is a Class
  1 felony, (b)(3) is a Class 2 felony, (b)(4) is a Class 3 felony,
  (b)(5) and (b)(6) is a Class 4 felony.

  (c) Involuntary sexual servitude of a minor. A person commits
  involuntary sexual servitude of a minor when he or she knowingly
  recruits, entices, harbors, transports, provides, or obtains by any
  means, or attempts to recruit, entice, harbor, provide, or obtain by
  any means, another person under 18 years of age, knowing that the
  minor will engage in commercial sexual activity, a sexually-explicit
  performance, or the production of pornography, or causes or attempts
  to cause a minor to engage in one or more of those activities and:

  (1) there is no overt force or threat and the minor is between the
  ages of 17 and 18 years;

  (2) there is no overt force or threat and the minor is under the age
  of 17 years; or

  (3) there is overt force or threat.

  Sentence. Except as otherwise provided in subsection (e) or (f), a
  violation of subsection (c)(1) is a Class 1 felony, (c)(2) is a Class
  X felony, and (c)(3) is a Class X felony.

  (d) Trafficking in persons. A person commits trafficking in persons
  when he or she knowingly: (1) recruits, entices, harbors, transports,
  provides, or obtains by any means, or attempts to recruit, entice,
  harbor, transport, provide, or obtain by any means, another person,
  intending or knowing that the person will be subjected to involuntary
  servitude; or (2) benefits, financially or by receiving anything of
  value, from participation in a venture that has engaged in an act of
  involuntary servitude or involuntary sexual servitude of a minor. A
  company commits trafficking in persons when the company knowingly
  benefits, financially or by receiving anything of value, from
  participation in a venture that has engaged in an act of involuntary
  servitude or involuntary sexual servitude of a minor.

  Sentence. Except as otherwise provided in subsection (e) or (f), a
  violation of this subsection by a person is a Class 1 felony. A
  violation of this subsection by a company is a business offense for
  which a fine of up to \$100,000 may be imposed.

  (e) Aggravating factors. A violation of this Section involving
  kidnapping or an attempt to kidnap, aggravated criminal sexual assault
  or an attempt to commit aggravated criminal sexual assault, or an
  attempt to commit first degree murder is a Class X felony.

  (f) Sentencing considerations.

  (1) Bodily injury. If, pursuant to a violation of this Section, a
  victim suffered bodily injury, the defendant may be sentenced to an
  extended-term sentence under Section 5-8-2 of the Unified Code of
  Corrections. The sentencing court must take into account the time in
  which the victim was held in servitude, with increased penalties for
  cases in which the victim was held for between 180 days and one year,
  and increased penalties for cases in which the victim was held for
  more than one year.

  (2) Number of victims. In determining sentences within statutory
  maximums, the sentencing court should take into account the number of
  victims, and may provide for substantially increased sentences in
  cases involving more than 10 victims.

  (g) Restitution. Restitution is mandatory under this Section. In
  addition to any other amount of loss identified, the court shall order
  restitution including the greater of (1) the gross income or value to
  the defendant of the victim's labor or services or (2) the value of
  the victim's labor as guaranteed under the Minimum Wage Law and
  overtime provisions of the Fair Labor Standards Act (FLSA) or the
  Minimum Wage Law, whichever is greater.

  (g-5) Fine distribution. If the court imposes a fine under subsection
  (b), (c), or (d) of this Section, it shall be collected and
  distributed to the Specialized Services for Survivors of Human
  Trafficking Fund in accordance with Section 5-9-1.21 of the Unified
  Code of Corrections.

  (h) Trafficking victim services. Subject to the availability of funds,
  the Department of Human Services may provide or fund emergency
  services and assistance to individuals who are victims of one or more
  offenses defined in this Section.

  (i) Certification. The Attorney General, a State's Attorney, or any
  law enforcement official shall certify in writing to the United States
  Department of Justice or other federal agency, such as the United
  States Department of Homeland Security, that an investigation or
  prosecution under this Section has begun and the individual who is a
  likely victim of a crime described in this Section is willing to
  cooperate or is cooperating with the investigation to enable the
  individual, if eligible under federal law, to qualify for an
  appropriate special immigrant visa and to access available federal
  benefits. Cooperation with law enforcement shall not be required of
  victims of a crime described in this Section who are under 18 years of
  age. This certification shall be made available to the victim and his
  or her designated legal representative.

  (j) A person who commits involuntary servitude, involuntary sexual
  servitude of a minor, or trafficking in persons under subsection (b),
  (c), or (d) of this Section is subject to the property forfeiture
  provisions set forth in Article 124B of the Code of Criminal Procedure
  of 1963.

  (Source: P.A. 101-18, eff. 1-1-20 .)

  \hypertarget{ilcs-510-10}{%
  \subsection*{(720 ILCS 5/10-10)}\label{ilcs-510-10}}
  \addcontentsline{toc}{subsection}{(720 ILCS 5/10-10)}

  \hypertarget{sec.-10-10.-failure-to-report-the-death-or-disappearance-of-a-child-under-13-years-of-age.}{%
  \section*{Sec. 10-10. Failure to report the death or disappearance of
  a child under 13 years of
  age.}\label{sec.-10-10.-failure-to-report-the-death-or-disappearance-of-a-child-under-13-years-of-age.}}
  \addcontentsline{toc}{section}{Sec. 10-10. Failure to report the death
  or disappearance of a child under 13 years of age.}

  \markright{Sec. 10-10. Failure to report the death or disappearance of
  a child under 13 years of age.}

  (a) A parent, legal guardian, or caretaker of a child under 13 years
  of age commits failure to report the death or disappearance of a child
  under 13 years of age when he or she knows or should know and fails to
  report the child as missing or deceased to a law enforcement agency
  within 24 hours if the parent, legal guardian, or caretaker reasonably
  believes that the child is missing or deceased. In the case of a child
  under the age of 2 years, the reporting requirement is reduced to no
  more than one hour.

  (b) A parent, legal guardian, or caretaker of a child under 13 years
  of age must report the death of the child to the law enforcement
  agency of the county where the child's corpse was found if the parent,
  legal guardian, or caretaker reasonably believes that the death of the
  child was caused by a homicide, accident, or other suspicious
  circumstance.

  (c) The Department of Children and Family Services Guardianship
  Administrator shall not personally be subject to the reporting
  requirements in subsection (a) or (b) of this Section.

  (d) A parent, legal guardian, or caretaker does not commit the offense
  of failure to report the death or disappearance of a child under 13
  years of age when:

  (1) the failure to report is due to an act of God, act of war, or
  inability of a law enforcement agency to receive a report of the
  disappearance of a child;

  (2) the parent, legal guardian, or caretaker calls

  911 to report the disappearance of the child;

  (3) the parent, legal guardian, or caretaker knows that the child is
  under the care of another parent, family member, relative, friend, or
  baby sitter; or

  (4) the parent, legal guardian, or caretaker is hospitalized, in a
  coma, or is otherwise seriously physically or mentally impaired as to
  prevent the person from reporting the death or disappearance.

  (e) Sentence. A violation of this Section is a Class 4 felony.

  (Source: P.A. 97-1079, eff. 1-1-13.)
\end{enumerate}

\bookmarksetup{startatroot}

\hypertarget{article-10a.-repealed}{%
\chapter*{Article 10a. (Repealed)}\label{article-10a.-repealed}}
\addcontentsline{toc}{chapter}{Article 10a. (Repealed)}

\markboth{Article 10a. (Repealed)}{Article 10a. (Repealed)}

(Source: Repealed by P.A. 96-710, eff. 1-1-10.)

\bookmarksetup{startatroot}

\hypertarget{article-11.-sex-offenses}{%
\chapter*{Article 11. Sex Offenses}\label{article-11.-sex-offenses}}
\addcontentsline{toc}{chapter}{Article 11. Sex Offenses}

\markboth{Article 11. Sex Offenses}{Article 11. Sex Offenses}

\hypertarget{ilcs-5art.-11-subdiv.-1-heading}{%
\subsection*{(720 ILCS 5/Art. 11 Subdiv. 1
heading)}\label{ilcs-5art.-11-subdiv.-1-heading}}
\addcontentsline{toc}{subsection}{(720 ILCS 5/Art. 11 Subdiv. 1
heading)}

SUBDIVISION 1.

GENERAL DEFINITIONS

(Source: P.A. 96-1551, eff. 7-1-11.)

\hypertarget{ilcs-511-0.1}{%
\subsection*{(720 ILCS 5/11-0.1)}\label{ilcs-511-0.1}}
\addcontentsline{toc}{subsection}{(720 ILCS 5/11-0.1)}

\hypertarget{sec.-11-0.1.-definitions.}{%
\section*{Sec. 11-0.1. Definitions.}\label{sec.-11-0.1.-definitions.}}
\addcontentsline{toc}{section}{Sec. 11-0.1. Definitions.}

\markright{Sec. 11-0.1. Definitions.}

In this Article, unless the context clearly requires otherwise, the
following terms are defined as indicated:

``Accused'' means a person accused of an offense prohibited by Section
11-1.20, 11-1.30, 11-1.40, 11-1.50, or 11-1.60 of this Code or a person
for whose conduct the accused is legally responsible under Article 5 of
this Code.

``Adult obscenity or child pornography Internet site''. See Section
11-23.

``Advance prostitution'' means:

(1) Soliciting for a prostitute by performing any of the following acts
when acting other than as a prostitute or a patron of a prostitute:

(A) Soliciting another for the purpose of prostitution.

(B) Arranging or offering to arrange a meeting of persons for the
purpose of prostitution.

(C) Directing another to a place knowing the direction is for the
purpose of prostitution.

(2) Keeping a place of prostitution by controlling or exercising control
over the use of any place that could offer seclusion or shelter for the
practice of prostitution and performing any of the following acts when
acting other than as a prostitute or a patron of a prostitute:

(A) Knowingly granting or permitting the use of the place for the
purpose of prostitution.

(B) Granting or permitting the use of the place under circumstances from
which he or she could reasonably know that the place is used or is to be
used for purposes of prostitution.

(C) Permitting the continued use of the place after becoming aware of
facts or circumstances from which he or she should reasonably know that
the place is being used for purposes of prostitution.

``Agency''. See Section 11-9.5.

``Arranges''. See Section 11-6.5.

``Bodily harm'' means physical harm, and includes, but is not limited
to, sexually transmitted disease, pregnancy, and impotence.

``Care and custody''. See Section 11-9.5.

``Child care institution''. See Section 11-9.3.

``Child pornography''. See Section 11-20.1.

``Child sex offender''. See Section 11-9.3.

``Community agency''. See Section 11-9.5.

``Conditional release''. See Section 11-9.2.

``Consent'' means a freely given agreement to the act of sexual
penetration or sexual conduct in question. Lack of verbal or physical
resistance or submission by the victim resulting from the use of force
or threat of force by the accused shall not constitute consent. The
manner of dress of the victim at the time of the offense shall not
constitute consent.

``Custody''. See Section 11-9.2.

``Day care center''. See Section 11-9.3.

``Depict by computer''. See Section 11-20.1.

``Depiction by computer''. See Section 11-20.1.

``Disseminate''. See Section 11-20.1.

``Distribute''. See Section 11-21.

``Family member'' means a parent, grandparent, child, aunt, uncle,
great-aunt, or great-uncle, whether by whole blood, half-blood, or
adoption, and includes a step-grandparent, step-parent, or step-child.
``Family member'' also means, if the victim is a child under 18 years of
age, an accused who has resided in the household with the child
continuously for at least 6 months.

``Force or threat of force'' means the use of force or violence or the
threat of force or violence, including, but not limited to, the
following situations:

(1) when the accused threatens to use force or violence on the victim or
on any other person, and the victim under the circumstances reasonably
believes that the accused has the ability to execute that threat; or

(2) when the accused overcomes the victim by use of superior strength or
size, physical restraint, or physical confinement.

``Harmful to minors''. See Section 11-21.

``Loiter''. See Section 9.3.

``Material''. See Section 11-21.

``Minor''. See Section 11-21.

``Nudity''. See Section 11-21.

``Obscene''. See Section 11-20.

``Part day child care facility''. See Section 11-9.3.

``Penal system''. See Section 11-9.2.

``Person responsible for the child's welfare''. See Section 11-9.1A.

``Person with a disability''. See Section 11-9.5.

``Playground''. See Section 11-9.3.

``Probation officer''. See Section 11-9.2.

``Produce''. See Section 11-20.1.

``Profit from prostitution'' means, when acting other than as a
prostitute, to receive anything of value for personally rendered
prostitution services or to receive anything of value from a prostitute,
if the thing received is not for lawful consideration and the person
knows it was earned in whole or in part from the practice of
prostitution.

``Public park''. See Section 11-9.3.

``Public place''. See Section 11-30.

``Reproduce''. See Section 11-20.1.

``Sado-masochistic abuse''. See Section 11-21.

``School''. See Section 11-9.3.

``School official''. See Section 11-9.3.

``Sexual abuse''. See Section 11-9.1A.

``Sexual act''. See Section 11-9.1.

``Sexual conduct'' means any knowing touching or fondling by the victim
or the accused, either directly or through clothing, of the sex organs,
anus, or breast of the victim or the accused, or any part of the body of
a child under 13 years of age, or any transfer or transmission of semen
by the accused upon any part of the clothed or unclothed body of the
victim, for the purpose of sexual gratification or arousal of the victim
or the accused.

``Sexual excitement''. See Section 11-21.

``Sexual penetration'' means any contact, however slight, between the
sex organ or anus of one person and an object or the sex organ, mouth,
or anus of another person, or any intrusion, however slight, of any part
of the body of one person or of any animal or object into the sex organ
or anus of another person, including, but not limited to, cunnilingus,
fellatio, or anal penetration. Evidence of emission of semen is not
required to prove sexual penetration.

``Solicit''. See Section 11-6.

``State-operated facility''. See Section 11-9.5.

``Supervising officer''. See Section 11-9.2.

``Surveillance agent''. See Section 11-9.2.

``Treatment and detention facility''. See Section 11-9.2.

``Unable to give knowing consent'' includes when the accused administers
any intoxicating or anesthetic substance, or any controlled substance
causing the victim to become unconscious of the nature of the act and
this condition was known, or reasonably should have been known by the
accused. ``Unable to give knowing consent'' also includes when the
victim has taken an intoxicating substance or any controlled substance
causing the victim to become unconscious of the nature of the act, and
this condition was known or reasonably should have been known by the
accused, but the accused did not provide or administer the intoxicating
substance. As used in this paragraph, ``unconscious of the nature of the
act'' means incapable of resisting because the victim meets any one of
the following conditions:

(1) was unconscious or asleep;

(2) was not aware, knowing, perceiving, or cognizant that the act
occurred;

(3) was not aware, knowing, perceiving, or cognizant of the essential
characteristics of the act due to the perpetrator's fraud in fact; or

(4) was not aware, knowing, perceiving, or cognizant of the essential
characteristics of the act due to the perpetrator's fraudulent
representation that the sexual penetration served a professional purpose
when it served no professional purpose.

A victim is presumed ``unable to give knowing consent'' when the victim:

(1) is committed to the care and custody or supervision of the Illinois
Department of Corrections (IDOC) and the accused is an employee or
volunteer who is not married to the victim who knows or reasonably
should know that the victim is committed to the care and custody or
supervision of such department;

(2) is committed to or placed with the Department of

Children and Family Services (DCFS) and in residential care, and the
accused employee is not married to the victim, and knows or reasonably
should know that the victim is committed to or placed with DCFS and in
residential care;

(3) is a client or patient and the accused is a health care provider or
mental health care provider and the sexual conduct or sexual penetration
occurs during a treatment session, consultation, interview, or
examination;

(4) is a resident or inpatient of a residential facility and the accused
is an employee of the facility who is not married to such resident or
inpatient who provides direct care services, case management services,
medical or other clinical services, habilitative services or direct
supervision of the residents in the facility in which the resident
resides; or an officer or other employee, consultant, contractor or
volunteer of the residential facility, who knows or reasonably should
know that the person is a resident of such facility; or

(5) is detained or otherwise in the custody of a police officer, peace
officer, or other law enforcement official who: (i) is detaining or
maintaining custody of such person; or (ii) knows, or reasonably should
know, that at the time of the offense, such person was detained or in
custody and the police officer, peace officer, or other law enforcement
official is not married to such detainee.

``Victim'' means a person alleging to have been subjected to an offense
prohibited by Section 11-1.20, 11-1.30, 11-1.40, 11-1.50, or 11-1.60 of
this Code.

(Source: P.A. 102-567, eff. 1-1-22; 102-1096, eff. 1-1-23 .)

\hypertarget{ilcs-5art.-11-subdiv.-5-heading}{%
\subsection*{(720 ILCS 5/Art. 11 Subdiv. 5
heading)}\label{ilcs-5art.-11-subdiv.-5-heading}}
\addcontentsline{toc}{subsection}{(720 ILCS 5/Art. 11 Subdiv. 5
heading)}

SUBDIVISION 5.

MAJOR SEX OFFENSES

(Source: P.A. 96-1551, eff. 7-1-11.)

\hypertarget{ilcs-511-1.10}{%
\subsection*{(720 ILCS 5/11-1.10)}\label{ilcs-511-1.10}}
\addcontentsline{toc}{subsection}{(720 ILCS 5/11-1.10)}

(was 720 ILCS 5/12-18)

\hypertarget{sec.-11-1.10.-general-provisions-concerning-offenses-described-in-sections-11-1.20-through-11-1.60.}{%
\section*{Sec. 11-1.10. General provisions concerning offenses described
in Sections 11-1.20 through
11-1.60.}\label{sec.-11-1.10.-general-provisions-concerning-offenses-described-in-sections-11-1.20-through-11-1.60.}}
\addcontentsline{toc}{section}{Sec. 11-1.10. General provisions
concerning offenses described in Sections 11-1.20 through 11-1.60.}

\markright{Sec. 11-1.10. General provisions concerning offenses
described in Sections 11-1.20 through 11-1.60.}

(a) No person accused of violating Section 11-1.20, 11-1.30, 11-1.40,
11-1.50, or 11-1.60 of this Code shall be presumed to be incapable of
committing an offense prohibited by Section 11-1.20, 11-1.30, 11-1.40,
11-1.50, or 11-1.60 of this Code because of age, physical condition or
relationship to the victim. Nothing in this Section shall be construed
to modify or abrogate the affirmative defense of infancy under Section
6-1 of this Code or the provisions of Section 5-805 of the Juvenile
Court Act of 1987.

(b) Any medical examination or procedure which is conducted by a
physician, nurse, medical or hospital personnel, parent, or caretaker
for purposes and in a manner consistent with reasonable medical
standards is not an offense under Section 11-1.20, 11-1.30, 11-1.40,
11-1.50, or 11-1.60 of this Code.

(c) (Blank).

(d) (Blank).

(e) The prosecuting State's Attorney shall seek an order from the court
to compel the accused to be tested for any sexually transmissible
disease, including a test for infection with human immunodeficiency
virus (HIV), within 48 hours:

(1) after a finding at a preliminary hearing that there is probable
cause to believe that an accused has committed a violation of Section
11-1.20, 11-1.30, or 11-1.40 of this Code, or

(2) after an indictment is returned charging an accused with a violation
of Section 11-1.20, 11-1.30, or 11-1.40 of this Code, or

(3) after a finding that a defendant charged with a violation of Section
11-1.20, 11-1.30, or 11-1.40 of this Code is unfit to stand trial
pursuant to Section 104-16 of the Code of Criminal Procedure of 1963
where the finding is made prior to the preliminary hearing, or

(4) after the request of the victim of the violation of Section 11-1.20,
11-1.30, or 11-1.40.

The medical tests shall be performed only by appropriately licensed
medical practitioners. The testing shall consist of a test approved by
the Illinois Department of Public Health to determine the presence of
HIV infection, based upon recommendations of the United States Centers
for Disease Control and Prevention; in the event of a positive result, a
reliable supplemental test based upon recommendations of the United
States Centers for Disease Control and Prevention shall be administered.
The results of the tests and any follow-up tests shall be kept strictly
confidential by all medical personnel involved in the testing and must
be personally delivered in a sealed envelope to the victim, to the
defendant, to the State's Attorney, and to the judge who entered the
order, for the judge's inspection in camera. The judge shall provide to
the victim a referral to the Illinois Department of Public Health
HIV/AIDS toll-free hotline for counseling and information in connection
with the test result. Acting in accordance with the best interests of
the victim and the public, the judge shall have the discretion to
determine to whom, if anyone, the result of the testing may be revealed;
however, in no case shall the identity of the victim be disclosed. The
court shall order that the cost of the tests shall be paid by the
county, and shall be taxed as costs against the accused if convicted.

(f) Whenever any law enforcement officer has reasonable cause to believe
that a person has been delivered a controlled substance without his or
her consent, the law enforcement officer shall advise the victim about
seeking medical treatment and preserving evidence.

(g) Every hospital providing emergency hospital services to an alleged
sexual assault survivor, when there is reasonable cause to believe that
a person has been delivered a controlled substance without his or her
consent, shall designate personnel to provide:

(1) An explanation to the victim about the nature and effects of
commonly used controlled substances and how such controlled substances
are administered.

(2) An offer to the victim of testing for the presence of such
controlled substances.

(3) A disclosure to the victim that all controlled substances or alcohol
ingested by the victim will be disclosed by the test.

(4) A statement that the test is completely voluntary.

(5) A form for written authorization for sample analysis of all
controlled substances and alcohol ingested by the victim.

A physician licensed to practice medicine in all its branches may agree
to be a designated person under this subsection.

No sample analysis may be performed unless the victim returns a signed
written authorization within 30 days after the sample was collected.

Any medical treatment or care under this subsection shall be only in
accordance with the order of a physician licensed to practice medicine
in all of its branches. Any testing under this subsection shall be only
in accordance with the order of a licensed individual authorized to
order the testing.

(Source: P.A. 97-1109, eff. 1-1-13; 98-761, eff. 7-16-14.)

\hypertarget{ilcs-511-1.20}{%
\subsection*{(720 ILCS 5/11-1.20)}\label{ilcs-511-1.20}}
\addcontentsline{toc}{subsection}{(720 ILCS 5/11-1.20)}

(was 720 ILCS 5/12-13)

\hypertarget{sec.-11-1.20.-criminal-sexual-assault.}{%
\section*{Sec. 11-1.20. Criminal sexual
assault.}\label{sec.-11-1.20.-criminal-sexual-assault.}}
\addcontentsline{toc}{section}{Sec. 11-1.20. Criminal sexual assault.}

\markright{Sec. 11-1.20. Criminal sexual assault.}

(a) A person commits criminal sexual assault if that person commits an
act of sexual penetration and:

(1) uses force or threat of force;

(2) knows that the victim is unable to understand the nature of the act
or is unable to give knowing consent;

(3) is a family member of the victim, and the victim is under 18 years
of age; or

(4) is 17 years of age or over and holds a position of trust, authority,
or supervision in relation to the victim, and the victim is at least 13
years of age but under 18 years of age.

(b) Sentence.

(1) Criminal sexual assault is a Class 1 felony, except that:

(A) A person who is convicted of the offense of criminal sexual assault
as defined in paragraph (a)(1) or (a)(2) after having previously been
convicted of the offense of criminal sexual assault or the offense of
exploitation of a child, or who is convicted of the offense of criminal
sexual assault as defined in paragraph (a)(1) or (a)(2) after having
previously been convicted under the laws of this State or any other
state of an offense that is substantially equivalent to the offense of
criminal sexual assault or to the offense of exploitation of a child,
commits a Class X felony for which the person shall be sentenced to a
term of imprisonment of not less than 30 years and not more than 60
years, except that if the person is under the age of 18 years at the
time of the offense, he or she shall be sentenced under Section
5-4.5-105 of the Unified Code of Corrections. The commission of the
second or subsequent offense is required to have been after the initial
conviction for this paragraph (A) to apply.

(B) A person who has attained the age of 18 years at the time of the
commission of the offense and who is convicted of the offense of
criminal sexual assault as defined in paragraph (a)(1) or (a)(2) after
having previously been convicted of the offense of aggravated criminal
sexual assault or the offense of predatory criminal sexual assault of a
child, or who is convicted of the offense of criminal sexual assault as
defined in paragraph (a)(1) or (a)(2) after having previously been
convicted under the laws of this State or any other state of an offense
that is substantially equivalent to the offense of aggravated criminal
sexual assault or the offense of predatory criminal sexual assault of a
child shall be sentenced to a term of natural life imprisonment. The
commission of the second or subsequent offense is required to have been
after the initial conviction for this paragraph (B) to apply. An
offender under the age of 18 years at the time of the commission of the
offense covered by this subparagraph (B) shall be sentenced under
Section 5-4.5-105 of the Unified Code of Corrections.

(C) A second or subsequent conviction for a violation of paragraph
(a)(3) or (a)(4) or under any similar statute of this State or any other
state for any offense involving criminal sexual assault that is
substantially equivalent to or more serious than the sexual assault
prohibited under paragraph (a)(3) or (a)(4) is a Class X felony.

(Source: P.A. 99-69, eff. 1-1-16 .)

\hypertarget{ilcs-511-1.30}{%
\subsection*{(720 ILCS 5/11-1.30)}\label{ilcs-511-1.30}}
\addcontentsline{toc}{subsection}{(720 ILCS 5/11-1.30)}

(was 720 ILCS 5/12-14)

\hypertarget{sec.-11-1.30.-aggravated-criminal-sexual-assault.}{%
\section*{Sec. 11-1.30. Aggravated Criminal Sexual
Assault.}\label{sec.-11-1.30.-aggravated-criminal-sexual-assault.}}
\addcontentsline{toc}{section}{Sec. 11-1.30. Aggravated Criminal Sexual
Assault.}

\markright{Sec. 11-1.30. Aggravated Criminal Sexual Assault.}

(a) A person commits aggravated criminal sexual assault if that person
commits criminal sexual assault and any of the following aggravating
circumstances exist during the commission of the offense or, for
purposes of paragraph (7), occur as part of the same course of conduct
as the commission of the offense:

(1) the person displays, threatens to use, or uses a dangerous weapon,
other than a firearm, or any other object fashioned or used in a manner
that leads the victim, under the circumstances, reasonably to believe
that the object is a dangerous weapon;

(2) the person causes bodily harm to the victim, except as provided in
paragraph (10);

(3) the person acts in a manner that threatens or endangers the life of
the victim or any other person;

(4) the person commits the criminal sexual assault during the course of
committing or attempting to commit any other felony;

(5) the victim is 60 years of age or older;

(6) the victim is a person with a physical disability;

(7) the person delivers (by injection, inhalation, ingestion, transfer
of possession, or any other means) any controlled substance to the
victim without the victim's consent or by threat or deception for other
than medical purposes;

(8) the person is armed with a firearm;

(9) the person personally discharges a firearm during the commission of
the offense; or

(10) the person personally discharges a firearm during the commission of
the offense, and that discharge proximately causes great bodily harm,
permanent disability, permanent disfigurement, or death to another
person.

(b) A person commits aggravated criminal sexual assault if that person
is under 17 years of age and: (i) commits an act of sexual penetration
with a victim who is under 9 years of age; or (ii) commits an act of
sexual penetration with a victim who is at least 9 years of age but
under 13 years of age and the person uses force or threat of force to
commit the act.

(c) A person commits aggravated criminal sexual assault if that person
commits an act of sexual penetration with a victim who is a person with
a severe or profound intellectual disability.

(d) Sentence.

(1) Aggravated criminal sexual assault in violation of paragraph (2),
(3), (4), (5), (6), or (7) of subsection (a) or in violation of
subsection (b) or (c) is a Class X felony. A violation of subsection
(a)(1) is a Class X felony for which 10 years shall be added to the term
of imprisonment imposed by the court. A violation of subsection (a)(8)
is a Class X felony for which 15 years shall be added to the term of
imprisonment imposed by the court. A violation of subsection (a)(9) is a
Class X felony for which 20 years shall be added to the term of
imprisonment imposed by the court. A violation of subsection (a)(10) is
a Class X felony for which 25 years or up to a term of natural life
imprisonment shall be added to the term of imprisonment imposed by the
court. An offender under the age of 18 years at the time of the
commission of aggravated criminal sexual assault in violation of
paragraphs (1) through (10) of subsection (a) shall be sentenced under
Section 5-4.5-105 of the Unified Code of Corrections.

(2) A person who has attained the age of 18 years at the time of the
commission of the offense and who is convicted of a second or subsequent
offense of aggravated criminal sexual assault, or who is convicted of
the offense of aggravated criminal sexual assault after having
previously been convicted of the offense of criminal sexual assault or
the offense of predatory criminal sexual assault of a child, or who is
convicted of the offense of aggravated criminal sexual assault after
having previously been convicted under the laws of this or any other
state of an offense that is substantially equivalent to the offense of
criminal sexual assault, the offense of aggravated criminal sexual
assault or the offense of predatory criminal sexual assault of a child,
shall be sentenced to a term of natural life imprisonment. The
commission of the second or subsequent offense is required to have been
after the initial conviction for this paragraph (2) to apply. An
offender under the age of 18 years at the time of the commission of the
offense covered by this paragraph (2) shall be sentenced under Section
5-4.5-105 of the Unified Code of Corrections.

(Source: P.A. 99-69, eff. 1-1-16; 99-143, eff. 7-27-15; 99-642, eff.
7-28-16.)

\hypertarget{ilcs-511-1.40}{%
\subsection*{(720 ILCS 5/11-1.40)}\label{ilcs-511-1.40}}
\addcontentsline{toc}{subsection}{(720 ILCS 5/11-1.40)}

(was 720 ILCS 5/12-14.1)

\hypertarget{sec.-11-1.40.-predatory-criminal-sexual-assault-of-a-child.}{%
\section*{Sec. 11-1.40. Predatory criminal sexual assault of a
child.}\label{sec.-11-1.40.-predatory-criminal-sexual-assault-of-a-child.}}
\addcontentsline{toc}{section}{Sec. 11-1.40. Predatory criminal sexual
assault of a child.}

\markright{Sec. 11-1.40. Predatory criminal sexual assault of a child.}

(a) A person commits predatory criminal sexual assault of a child if
that person is 17 years of age or older, and commits an act of contact,
however slight, between the sex organ or anus of one person and the part
of the body of another for the purpose of sexual gratification or
arousal of the victim or the accused, or an act of sexual penetration,
and:

(1) the victim is under 13 years of age; or

(2) the victim is under 13 years of age and that person:

(A) is armed with a firearm;

(B) personally discharges a firearm during the commission of the
offense;

(C) causes great bodily harm to the victim that:

(i) results in permanent disability; or

(ii) is life threatening; or

(D) delivers (by injection, inhalation, ingestion, transfer of
possession, or any other means) any controlled substance to the victim
without the victim's consent or by threat or deception, for other than
medical purposes.

(b) Sentence.

(1) A person convicted of a violation of subsection

(a)(1) commits a Class X felony, for which the person shall be sentenced
to a term of imprisonment of not less than 6 years and not more than 60
years. A person convicted of a violation of subsection (a)(2)(A) commits
a Class X felony for which 15 years shall be added to the term of
imprisonment imposed by the court. A person convicted of a violation of
subsection (a)(2)(B) commits a Class X felony for which 20 years shall
be added to the term of imprisonment imposed by the court. A person who
has attained the age of 18 years at the time of the commission of the
offense and who is convicted of a violation of subsection (a)(2)(C)
commits a Class X felony for which the person shall be sentenced to a
term of imprisonment of not less than 50 years or up to a term of
natural life imprisonment. An offender under the age of 18 years at the
time of the commission of predatory criminal sexual assault of a child
in violation of subsections (a)(1), (a)(2)(A), (a)(2)(B), and (a)(2)(C)
shall be sentenced under Section 5-4.5-105 of the Unified Code of
Corrections.

(1.1) A person convicted of a violation of subsection

(a)(2)(D) commits a Class X felony for which the person shall be
sentenced to a term of imprisonment of not less than 50 years and not
more than 60 years. An offender under the age of 18 years at the time of
the commission of predatory criminal sexual assault of a child in
violation of subsection (a)(2)(D) shall be sentenced under Section
5-4.5-105 of the Unified Code of Corrections.

(1.2) A person who has attained the age of 18 years at the time of the
commission of the offense and convicted of predatory criminal sexual
assault of a child committed against 2 or more persons regardless of
whether the offenses occurred as the result of the same act or of
several related or unrelated acts shall be sentenced to a term of
natural life imprisonment and an offender under the age of 18 years at
the time of the commission of the offense shall be sentenced under
Section 5-4.5-105 of the Unified Code of Corrections.

(2) A person who has attained the age of 18 years at the time of the
commission of the offense and who is convicted of a second or subsequent
offense of predatory criminal sexual assault of a child, or who is
convicted of the offense of predatory criminal sexual assault of a child
after having previously been convicted of the offense of criminal sexual
assault or the offense of aggravated criminal sexual assault, or who is
convicted of the offense of predatory criminal sexual assault of a child
after having previously been convicted under the laws of this State or
any other state of an offense that is substantially equivalent to the
offense of predatory criminal sexual assault of a child, the offense of
aggravated criminal sexual assault or the offense of criminal sexual
assault, shall be sentenced to a term of natural life imprisonment. The
commission of the second or subsequent offense is required to have been
after the initial conviction for this paragraph (2) to apply. An
offender under the age of 18 years at the time of the commission of the
offense covered by this paragraph (2) shall be sentenced under Section
5-4.5-105 of the Unified Code of Corrections.

(Source: P.A. 98-370, eff. 1-1-14; 98-756, eff. 7-16-14; 98-903, eff.
8-15-14; 99-69, eff. 1-1-16 .)

\hypertarget{ilcs-511-1.50}{%
\subsection*{(720 ILCS 5/11-1.50)}\label{ilcs-511-1.50}}
\addcontentsline{toc}{subsection}{(720 ILCS 5/11-1.50)}

(was 720 ILCS 5/12-15)

\hypertarget{sec.-11-1.50.-criminal-sexual-abuse.}{%
\section*{Sec. 11-1.50. Criminal sexual
abuse.}\label{sec.-11-1.50.-criminal-sexual-abuse.}}
\addcontentsline{toc}{section}{Sec. 11-1.50. Criminal sexual abuse.}

\markright{Sec. 11-1.50. Criminal sexual abuse.}

(a) A person commits criminal sexual abuse if that person:

(1) commits an act of sexual conduct by the use of force or threat of
force; or

(2) commits an act of sexual conduct and knows that the victim is unable
to understand the nature of the act or is unable to give knowing
consent.

(b) A person commits criminal sexual abuse if that person is under 17
years of age and commits an act of sexual penetration or sexual conduct
with a victim who is at least 9 years of age but under 17 years of age.

(c) A person commits criminal sexual abuse if that person commits an act
of sexual penetration or sexual conduct with a victim who is at least 13
years of age but under 17 years of age and the person is less than 5
years older than the victim.

(d) Sentence. Criminal sexual abuse for a violation of subsection (b) or
(c) of this Section is a Class A misdemeanor. Criminal sexual abuse for
a violation of paragraph (1) or (2) of subsection (a) of this Section is
a Class 4 felony. A second or subsequent conviction for a violation of
subsection (a) of this Section is a Class 2 felony. For purposes of this
Section it is a second or subsequent conviction if the accused has at
any time been convicted under this Section or under any similar statute
of this State or any other state for any offense involving sexual abuse
or sexual assault that is substantially equivalent to or more serious
than the sexual abuse prohibited under this Section.

(Source: P.A. 96-1551, eff. 7-1-11 .)

\hypertarget{ilcs-511-1.60}{%
\subsection*{(720 ILCS 5/11-1.60)}\label{ilcs-511-1.60}}
\addcontentsline{toc}{subsection}{(720 ILCS 5/11-1.60)}

(was 720 ILCS 5/12-16)

\hypertarget{sec.-11-1.60.-aggravated-criminal-sexual-abuse.}{%
\section*{Sec. 11-1.60. Aggravated criminal sexual
abuse.}\label{sec.-11-1.60.-aggravated-criminal-sexual-abuse.}}
\addcontentsline{toc}{section}{Sec. 11-1.60. Aggravated criminal sexual
abuse.}

\markright{Sec. 11-1.60. Aggravated criminal sexual abuse.}

(a) A person commits aggravated criminal sexual abuse if that person
commits criminal sexual abuse and any of the following aggravating
circumstances exist (i) during the commission of the offense or (ii) for
purposes of paragraph (7), as part of the same course of conduct as the
commission of the offense:

(1) the person displays, threatens to use, or uses a dangerous weapon or
any other object fashioned or used in a manner that leads the victim,
under the circumstances, reasonably to believe that the object is a
dangerous weapon;

(2) the person causes bodily harm to the victim;

(3) the victim is 60 years of age or older;

(4) the victim is a person with a physical disability;

(5) the person acts in a manner that threatens or endangers the life of
the victim or any other person;

(6) the person commits the criminal sexual abuse during the course of
committing or attempting to commit any other felony; or

(7) the person delivers (by injection, inhalation, ingestion, transfer
of possession, or any other means) any controlled substance to the
victim for other than medical purposes without the victim's consent or
by threat or deception.

(b) A person commits aggravated criminal sexual abuse if that person
commits an act of sexual conduct with a victim who is under 18 years of
age and the person is a family member.

(c) A person commits aggravated criminal sexual abuse if:

(1) that person is 17 years of age or over and: (i) commits an act of
sexual conduct with a victim who is under 13 years of age; or (ii)
commits an act of sexual conduct with a victim who is at least 13 years
of age but under 17 years of age and the person uses force or threat of
force to commit the act; or

(2) that person is under 17 years of age and: (i) commits an act of
sexual conduct with a victim who is under 9 years of age; or (ii)
commits an act of sexual conduct with a victim who is at least 9 years
of age but under 17 years of age and the person uses force or threat of
force to commit the act.

(d) A person commits aggravated criminal sexual abuse if that person
commits an act of sexual penetration or sexual conduct with a victim who
is at least 13 years of age but under 17 years of age and the person is
at least 5 years older than the victim.

(e) A person commits aggravated criminal sexual abuse if that person
commits an act of sexual conduct with a victim who is a person with a
severe or profound intellectual disability.

(f) A person commits aggravated criminal sexual abuse if that person
commits an act of sexual conduct with a victim who is but under 18 years
of age and the person is 17 years of age or over and holds a position of
trust, authority, or supervision in relation to the victim.

(g) Sentence. Aggravated criminal sexual abuse for a violation of
subsection (a), (b), (c), (d) or (e) of this Section is a Class 2
felony. Aggravated criminal sexual abuse for a violation of subsection
(f) of this Section is a Class 1 felony.

(Source: P.A. 102-567, eff. 1-1-22 .)

\hypertarget{ilcs-511-1.70}{%
\subsection*{(720 ILCS 5/11-1.70)}\label{ilcs-511-1.70}}
\addcontentsline{toc}{subsection}{(720 ILCS 5/11-1.70)}

(was 720 ILCS 5/12-17)

\hypertarget{sec.-11-1.70.-defenses-with-respect-to-offenses-described-in-sections-11-1.20-through-11-1.60.}{%
\section*{Sec. 11-1.70. Defenses with respect to offenses described in
Sections 11-1.20 through
11-1.60.}\label{sec.-11-1.70.-defenses-with-respect-to-offenses-described-in-sections-11-1.20-through-11-1.60.}}
\addcontentsline{toc}{section}{Sec. 11-1.70. Defenses with respect to
offenses described in Sections 11-1.20 through 11-1.60.}

\markright{Sec. 11-1.70. Defenses with respect to offenses described in
Sections 11-1.20 through 11-1.60.}

(a) It shall be a defense to any offense under Section 11-1.20, 11-1.30,
11-1.40, 11-1.50, or 11-1.60 of this Code where force or threat of force
is an element of the offense that the victim consented.

(b) It shall be a defense under subsection (b) and subsection (c) of
Section 11-1.50 and subsection (d) of Section 11-1.60 of this Code that
the accused reasonably believed the person to be 17 years of age or
over.

(c) A person who initially consents to sexual penetration or sexual
conduct is not deemed to have consented to any sexual penetration or
sexual conduct that occurs after he or she withdraws consent during the
course of that sexual penetration or sexual conduct.

(Source: P.A. 102-567, eff. 1-1-22 .)

\hypertarget{ilcs-511-1.80}{%
\subsection*{(720 ILCS 5/11-1.80)}\label{ilcs-511-1.80}}
\addcontentsline{toc}{subsection}{(720 ILCS 5/11-1.80)}

(was 720 ILCS 5/12-18.1)

\hypertarget{sec.-11-1.80.-civil-liability.}{%
\section*{Sec. 11-1.80. Civil
Liability.}\label{sec.-11-1.80.-civil-liability.}}
\addcontentsline{toc}{section}{Sec. 11-1.80. Civil Liability.}

\markright{Sec. 11-1.80. Civil Liability.}

(a) If any person has been convicted of any offense defined in Section
11-1.20, 11-1.30, 11-1.40, 11-1.50, 11-1.60, 12-13, 12-14, 12-14.1,
12-15, or 12-16 of this Act, a victim of such offense has a cause of
action for damages against any person or entity who, by the manufacture,
production, or wholesale distribution of any obscene material which was
possessed or viewed by the person convicted of the offense, proximately
caused such person, through his or her reading or viewing of the obscene
material, to commit the violation of Section 11-1.20, 11-1.30, 11-1.40,
11-1.50, 11-1.60, 12-13, 12-14, 12-14.1, 12-15, or 12-16. No victim may
recover in any such action unless he or she proves by a preponderance of
the evidence that: (1) the reading or viewing of the specific obscene
material manufactured, produced, or distributed wholesale by the
defendant proximately caused the person convicted of the violation of
Section 11-1.20, 11-1.30, 11-1.40, 11-1.50, 11-1.60, 12-13, 12-14,
12-14.1, 12-15, or 12-16 to commit such violation and (2) the defendant
knew or had reason to know that the manufacture, production, or
wholesale distribution of such material was likely to cause a violation
of an offense substantially of the type enumerated.

(b) The manufacturer, producer or wholesale distributor shall be liable
to the victim for:

(1) actual damages incurred by the victim, including medical costs;

(2) court costs and reasonable attorneys fees;

(3) infliction of emotional distress;

(4) pain and suffering; and

(5) loss of consortium.

(c) Every action under this Section shall be commenced within 3 years
after the conviction of the defendant for a violation of Section
11-1.20, 11-1.30, 11-1.40, 11-1.50, 11-1.60, 12-13, 12-14, 12-15 or
12-16 of this Code. However, if the victim was under the age of 18 years
at the time of the conviction of the defendant for a violation of
Section 11-1.20, 11-1.30, 11-1.40, 11-1.50, 11-1.60, 12-13, 12-14,
12-14.1, 12-15 or 12-16 of this Code, an action under this Section shall
be commenced within 3 years after the victim attains the age of 18
years.

(d) For the purposes of this Section:

(1) ``obscene'' has the meaning ascribed to it in subsection (b) of
Section 11-20 of this Code;

(2) ``wholesale distributor'' means any individual, partnership,
corporation, association, or other legal entity which stands between the
manufacturer and the retail seller in purchases, consignments, contracts
for sale or rental of the obscene material;

(3) ``producer'' means any individual, partnership, corporation,
association, or other legal entity which finances or supervises, to any
extent, the production or making of obscene material;

(4) ``manufacturer'' means any individual, partnership, corporation,
association, or other legal entity which manufacturers, assembles or
produces obscene material.

(Source: P.A. 96-1551, Article 2, Section 5, eff. 7-1-11; 96-1551,
Article 2, Section 1035, eff. 7-1-11; 97-1109, eff. 1-1-13.)

\hypertarget{ilcs-511-6-from-ch.-38-par.-11-6}{%
\subsection*{(720 ILCS 5/11-6) (from Ch. 38, par.
11-6)}\label{ilcs-511-6-from-ch.-38-par.-11-6}}
\addcontentsline{toc}{subsection}{(720 ILCS 5/11-6) (from Ch. 38, par.
11-6)}

\hypertarget{sec.-11-6.-indecent-solicitation-of-a-child.}{%
\section*{Sec. 11-6. Indecent solicitation of a
child.}\label{sec.-11-6.-indecent-solicitation-of-a-child.}}
\addcontentsline{toc}{section}{Sec. 11-6. Indecent solicitation of a
child.}

\markright{Sec. 11-6. Indecent solicitation of a child.}

(a) A person of the age of 17 years and upwards commits indecent
solicitation of a child if the person, with the intent that the offense
of aggravated criminal sexual assault, criminal sexual assault,
predatory criminal sexual assault of a child, or aggravated criminal
sexual abuse be committed, knowingly solicits a child or one whom he or
she believes to be a child to perform an act of sexual penetration or
sexual conduct as defined in Section 11-0.1 of this Code.

(a-5) A person of the age of 17 years and upwards commits indecent
solicitation of a child if the person knowingly discusses an act of
sexual conduct or sexual penetration with a child or with one whom he or
she believes to be a child by means of the Internet with the intent that
the offense of aggravated criminal sexual assault, predatory criminal
sexual assault of a child, or aggravated criminal sexual abuse be
committed.

(a-6) It is not a defense to subsection (a-5) that the person did not
solicit the child to perform sexual conduct or sexual penetration with
the person.

(b) Definitions. As used in this Section:

``Solicit'' means to command, authorize, urge, incite, request, or
advise another to perform an act by any means including, but not limited
to, in person, over the phone, in writing, by computer, or by
advertisement of any kind.

``Child'' means a person under 17 years of age.

``Internet'' has the meaning set forth in Section

16-0.1 of this Code.

``Sexual penetration'' or ``sexual conduct'' are defined in Section
11-0.1 of this Code.

(c) Sentence. Indecent solicitation of a child under subsection (a) is:

(1) a Class 1 felony when the act, if done, would be predatory criminal
sexual assault of a child or aggravated criminal sexual assault;

(2) a Class 2 felony when the act, if done, would be criminal sexual
assault;

(3) a Class 3 felony when the act, if done, would be aggravated criminal
sexual abuse.

Indecent solicitation of a child under subsection (a-5) is a Class 4
felony.

(Source: P.A. 96-1551, eff. 7-1-11; 97-1150, eff. 1-25-13.)

\hypertarget{ilcs-511-6.5}{%
\subsection*{(720 ILCS 5/11-6.5)}\label{ilcs-511-6.5}}
\addcontentsline{toc}{subsection}{(720 ILCS 5/11-6.5)}

\hypertarget{sec.-11-6.5.-indecent-solicitation-of-an-adult.}{%
\section*{Sec. 11-6.5. Indecent solicitation of an
adult.}\label{sec.-11-6.5.-indecent-solicitation-of-an-adult.}}
\addcontentsline{toc}{section}{Sec. 11-6.5. Indecent solicitation of an
adult.}

\markright{Sec. 11-6.5. Indecent solicitation of an adult.}

(a) A person commits indecent solicitation of an adult if the person
knowingly:

(1) Arranges for a person 17 years of age or over to commit an act of
sexual penetration as defined in Section 11-0.1 with a person:

(i) Under the age of 13 years; or

(ii) Thirteen years of age or over but under the age of 17 years; or

(2) Arranges for a person 17 years of age or over to commit an act of
sexual conduct as defined in Section 11-0.1 with a person:

(i) Under the age of 13 years; or

(ii) Thirteen years of age or older but under the age of 17 years.

(b) Sentence.

(1) Violation of paragraph (a)(1)(i) is a Class X felony.

(2) Violation of paragraph (a)(1)(ii) is a Class 1 felony.

(3) Violation of paragraph (a)(2)(i) is a Class 2 felony.

(4) Violation of paragraph (a)(2)(ii) is a Class A misdemeanor.

(c) For the purposes of this Section, ``arranges'' includes but is not
limited to oral or written communication and communication by telephone,
computer, or other electronic means. ``Computer'' has the meaning
ascribed to it in Section 17-0.5 of this Code.

(Source: P.A. 96-1551, eff. 7-1-11; 97-1150, eff. 1-25-13.)

\hypertarget{ilcs-511-6.6}{%
\subsection*{(720 ILCS 5/11-6.6)}\label{ilcs-511-6.6}}
\addcontentsline{toc}{subsection}{(720 ILCS 5/11-6.6)}

\hypertarget{sec.-11-6.6.-solicitation-to-meet-a-child.}{%
\section*{Sec. 11-6.6. Solicitation to meet a
child.}\label{sec.-11-6.6.-solicitation-to-meet-a-child.}}
\addcontentsline{toc}{section}{Sec. 11-6.6. Solicitation to meet a
child.}

\markright{Sec. 11-6.6. Solicitation to meet a child.}

(a) A person of the age of 18 or more years commits the offense of
solicitation to meet a child if the person while using a computer,
cellular telephone, or any other device, with the intent to meet a child
or one whom he or she believes to be a child, solicits, entices,
induces, or arranges with the child to meet at a location without the
knowledge of the child's parent or guardian and the meeting with the
child is arranged for a purpose other than a lawful purpose under
Illinois law.

(b) Sentence. Solicitation to meet a child is a Class A misdemeanor.
Solicitation to meet a child is a Class 4 felony when the solicitor
believes he or she is 5 or more years older than the child.

(c) For purposes of this Section, ``child'' means any person under 17
years of age; and ``computer'' has the meaning ascribed to it in Section
17-0.5 of this Code.

(Source: P.A. 101-87, eff. 1-1-20 .)

\hypertarget{ilcs-511-7-from-ch.-38-par.-11-7}{%
\subsection*{(720 ILCS 5/11-7) (from Ch. 38, par.
11-7)}\label{ilcs-511-7-from-ch.-38-par.-11-7}}
\addcontentsline{toc}{subsection}{(720 ILCS 5/11-7) (from Ch. 38, par.
11-7)}

(This Section was renumbered as Section 11-35 by P.A. 96-1551.)

\hypertarget{sec.-11-7.-renumbered.}{%
\section*{Sec. 11-7. (Renumbered).}\label{sec.-11-7.-renumbered.}}
\addcontentsline{toc}{section}{Sec. 11-7. (Renumbered).}

\markright{Sec. 11-7. (Renumbered).}

(Source: P.A. 86-490. Renumbered by P.A. 96-1551, eff. 7-1-11 .)

\hypertarget{ilcs-511-8-from-ch.-38-par.-11-8}{%
\subsection*{(720 ILCS 5/11-8) (from Ch. 38, par.
11-8)}\label{ilcs-511-8-from-ch.-38-par.-11-8}}
\addcontentsline{toc}{subsection}{(720 ILCS 5/11-8) (from Ch. 38, par.
11-8)}

(This Section was renumbered as Section 11-40 by P.A. 96-1551.)

\hypertarget{sec.-11-8.-renumbered.}{%
\section*{Sec. 11-8. (Renumbered).}\label{sec.-11-8.-renumbered.}}
\addcontentsline{toc}{section}{Sec. 11-8. (Renumbered).}

\markright{Sec. 11-8. (Renumbered).}

(Source: P.A. 86-490. Renumbered by P.A. 96-1551, eff. 7-1-11 .)

\hypertarget{ilcs-511-9-from-ch.-38-par.-11-9}{%
\subsection*{(720 ILCS 5/11-9) (from Ch. 38, par.
11-9)}\label{ilcs-511-9-from-ch.-38-par.-11-9}}
\addcontentsline{toc}{subsection}{(720 ILCS 5/11-9) (from Ch. 38, par.
11-9)}

(This Section was renumbered as Section 11-30 by P.A. 96-1551.)

\hypertarget{sec.-11-9.-renumbered.}{%
\section*{Sec. 11-9. (Renumbered).}\label{sec.-11-9.-renumbered.}}
\addcontentsline{toc}{section}{Sec. 11-9. (Renumbered).}

\markright{Sec. 11-9. (Renumbered).}

(Source: P.A. 96-1098, eff. 1-1-11. Renumbered by P.A. 96-1551, eff.
7-1-11 .)

\hypertarget{ilcs-5art.-11-subdiv.-10-heading}{%
\subsection*{(720 ILCS 5/Art. 11 Subdiv. 10
heading)}\label{ilcs-5art.-11-subdiv.-10-heading}}
\addcontentsline{toc}{subsection}{(720 ILCS 5/Art. 11 Subdiv. 10
heading)}

SUBDIVISION 10.

VULNERABLE VICTIM OFFENSES

(Source: P.A. 96-1551, eff. 7-1-11.)

\hypertarget{ilcs-511-9.1-from-ch.-38-par.-11-9.1}{%
\subsection*{(720 ILCS 5/11-9.1) (from Ch. 38, par.
11-9.1)}\label{ilcs-511-9.1-from-ch.-38-par.-11-9.1}}
\addcontentsline{toc}{subsection}{(720 ILCS 5/11-9.1) (from Ch. 38, par.
11-9.1)}

\hypertarget{sec.-11-9.1.-sexual-exploitation-of-a-child.}{%
\section*{Sec. 11-9.1. Sexual exploitation of a
child.}\label{sec.-11-9.1.-sexual-exploitation-of-a-child.}}
\addcontentsline{toc}{section}{Sec. 11-9.1. Sexual exploitation of a
child.}

\markright{Sec. 11-9.1. Sexual exploitation of a child.}

(a) A person commits sexual exploitation of a child if in the presence
or virtual presence, or both, of a child and with knowledge that a child
or one whom he or she believes to be a child would view his or her acts,
that person:

(1) engages in a sexual act; or

(2) exposes his or her sex organs, anus or breast for the purpose of
sexual arousal or gratification of such person or the child or one whom
he or she believes to be a child.

(a-5) A person commits sexual exploitation of a child who knowingly
entices, coerces, or persuades a child to remove the child's clothing
for the purpose of sexual arousal or gratification of the person or the
child, or both.

(b) Definitions. As used in this Section:

``Sexual act'' means masturbation, sexual conduct or sexual penetration
as defined in Section 11-0.1 of this Code.

``Sex offense'' means any violation of Article 11 of this Code.

``Child'' means a person under 17 years of age.

``Virtual presence'' means an environment that is created with software
and presented to the user and or receiver via the Internet, in such a
way that the user appears in front of the receiver on the computer
monitor or screen or hand-held portable electronic device, usually
through a web camming program. ``Virtual presence'' includes primarily
experiencing through sight or sound, or both, a video image that can be
explored interactively at a personal computer or hand-held communication
device, or both.

``Webcam'' means a video capturing device connected to a computer or
computer network that is designed to take digital photographs or live or
recorded video which allows for the live transmission to an end user
over the Internet.

(c) Sentence.

(1) Sexual exploitation of a child is a Class A misdemeanor. A second or
subsequent violation of this Section or a substantially similar law of
another state is a Class 4 felony.

(2) Sexual exploitation of a child is a Class 4 felony if the person has
been previously convicted of a sex offense.

(3) Sexual exploitation of a child is a Class 4 felony if the victim was
under 13 years of age at the time of the commission of the offense.

(4) Sexual exploitation of a child is a Class 4 felony if committed by a
person 18 years of age or older who is on or within 500 feet of
elementary or secondary school grounds when children are present on the
grounds.

(Source: P.A. 102-168, eff. 7-27-21.)

\hypertarget{ilcs-511-9.1a}{%
\subsection*{(720 ILCS 5/11-9.1A)}\label{ilcs-511-9.1a}}
\addcontentsline{toc}{subsection}{(720 ILCS 5/11-9.1A)}

\hypertarget{sec.-11-9.1a.-permitting-sexual-abuse-of-a-child.}{%
\section*{Sec. 11-9.1A. Permitting sexual abuse of a
child.}\label{sec.-11-9.1a.-permitting-sexual-abuse-of-a-child.}}
\addcontentsline{toc}{section}{Sec. 11-9.1A. Permitting sexual abuse of
a child.}

\markright{Sec. 11-9.1A. Permitting sexual abuse of a child.}

(a) A person responsible for a child's welfare commits permitting sexual
abuse of a child if the person has actual knowledge of and permits an
act of sexual abuse upon the child, or permits the child to engage in
prostitution as defined in Section 11-14 of this Code.

(b) In this Section:

``Actual knowledge'' includes credible allegations made by the child.

``Child'' means a minor under the age of 17 years.

``Person responsible for the child's welfare'' means the child's parent,
step-parent, legal guardian, or other person having custody of a child,
who is responsible for the child's care at the time of the alleged
sexual abuse.

``Prostitution'' means prostitution as defined in Section 11-14 of this
Code.

``Sexual abuse'' includes criminal sexual abuse or criminal sexual
assault as defined in Section 11-1.20, 11-1.30, 11-1.40, 11-1.50, or
11-1.60 of this Code.

(c) This Section does not apply to a person responsible for the child's
welfare who, having reason to believe that sexual abuse has occurred,
makes timely and reasonable efforts to stop the sexual abuse by
reporting the sexual abuse in conformance with the Abused and Neglected
Child Reporting Act or by reporting the sexual abuse, or causing a
report to be made, to medical or law enforcement authorities or anyone
who is a mandated reporter under Section 4 of the Abused and Neglected
Child Reporting Act.

(d) Whenever a law enforcement officer has reason to believe that the
child or the person responsible for the child's welfare has been abused
by a family or household member as defined by the Illinois Domestic
Violence Act of 1986, the officer shall immediately use all reasonable
means to prevent further abuse under Section 112A-30 of the Code of
Criminal Procedure of 1963.

(e) An order of protection under Section 111-8 of the Code of Criminal
Procedure of 1963 shall be sought in all cases where there is reason to
believe that a child has been sexually abused by a family or household
member. In considering appropriate available remedies, it shall be
presumed that awarding physical care or custody to the abuser is not in
the child's best interest.

(f) A person may not be charged with the offense of permitting sexual
abuse of a child under this Section until the person who committed the
offense is charged with criminal sexual assault, aggravated criminal
sexual assault, predatory criminal sexual assault of a child, criminal
sexual abuse, aggravated criminal sexual abuse, or prostitution.

(g) A person convicted of permitting the sexual abuse of a child is
guilty of a Class 1 felony. As a condition of any sentence of
supervision, probation, conditional discharge, or mandatory supervised
release, any person convicted under this Section shall be ordered to
undergo child sexual abuse, domestic violence, or other appropriate
counseling for a specified duration with a qualified social or mental
health worker.

(h) It is an affirmative defense to a charge of permitting sexual abuse
of a child under this Section that the person responsible for the
child's welfare had a reasonable apprehension that timely action to stop
the abuse or prostitution would result in the imminent infliction of
death, great bodily harm, permanent disfigurement, or permanent
disability to that person or another in retaliation for reporting.

(Source: P.A. 96-1551, eff. 7-1-11; 97-1150, eff. 1-25-13.)

\hypertarget{ilcs-511-9.1b}{%
\subsection*{(720 ILCS 5/11-9.1B)}\label{ilcs-511-9.1b}}
\addcontentsline{toc}{subsection}{(720 ILCS 5/11-9.1B)}

\hypertarget{sec.-11-9.1b.-failure-to-report-sexual-abuse-of-a-child.}{%
\section*{Sec. 11-9.1B. Failure to report sexual abuse of a
child.}\label{sec.-11-9.1b.-failure-to-report-sexual-abuse-of-a-child.}}
\addcontentsline{toc}{section}{Sec. 11-9.1B. Failure to report sexual
abuse of a child.}

\markright{Sec. 11-9.1B. Failure to report sexual abuse of a child.}

(a) For the purposes of this Section:

``Child'' means any person under the age of 13.

``Sexual abuse'' means any contact, however slight, between the sex
organ or anus of the victim or the accused and an object or body part,
including, but not limited to, the sex organ, mouth, or anus of the
victim or the accused, or any intrusion, however slight, of any part of
the body of the victim or the accused or of any animal or object into
the sex organ or anus of the victim or the accused, including, but not
limited to, cunnilingus, fellatio, or anal penetration. Evidence of
emission of semen is not required to prove sexual abuse.

(b) A person over the age of 18 commits failure to report sexual abuse
of a child when he or she personally observes sexual abuse, as defined
by this Section, between a person who he or she knows is over the age of
18 and a person he or she knows is a child, and knowingly fails to
report the sexual abuse to law enforcement.

(c) This Section does not apply to a person who makes timely and
reasonable efforts to stop the sexual abuse by reporting the sexual
abuse in conformance with the Abused and Neglected Child Reporting Act
or by reporting the sexual abuse or causing a report to be made, to
medical or law enforcement authorities or anyone who is a mandated
reporter under Section 4 of the Abused and Neglected Child Reporting
Act.

(d) A person may not be charged with the offense of failure to report
sexual abuse of a child under this Section until the person who
committed the offense is charged with criminal sexual assault,
aggravated criminal sexual assault, predatory criminal sexual assault of
a child, criminal sexual abuse, or aggravated criminal sexual abuse.

(e) It is an affirmative defense to a charge of failure to report sexual
abuse of a child under this Section that the person who personally
observed the sexual abuse had a reasonable apprehension that timely
action to stop the abuse would result in the imminent infliction of
death, great bodily harm, permanent disfigurement, or permanent
disability to that person or another in retaliation for reporting.

(f) Sentence. A person who commits failure to report sexual abuse of a
child is guilty of a Class A misdemeanor for the first violation and a
Class 4 felony for a second or subsequent violation.

(g) Nothing in this Section shall be construed to allow prosecution of a
person who personally observes the act of sexual abuse and assists with
an investigation and any subsequent prosecution of the offender.

(Source: P.A. 98-370, eff. 1-1-14; 98-756, eff. 7-16-14.)

\hypertarget{ilcs-511-9.2}{%
\subsection*{(720 ILCS 5/11-9.2)}\label{ilcs-511-9.2}}
\addcontentsline{toc}{subsection}{(720 ILCS 5/11-9.2)}

\hypertarget{sec.-11-9.2.-custodial-sexual-misconduct.}{%
\section*{Sec. 11-9.2. Custodial sexual
misconduct.}\label{sec.-11-9.2.-custodial-sexual-misconduct.}}
\addcontentsline{toc}{section}{Sec. 11-9.2. Custodial sexual
misconduct.}

\markright{Sec. 11-9.2. Custodial sexual misconduct.}

(a) A person commits custodial sexual misconduct when: (1) he or she is
an employee of a penal system and engages in sexual conduct or sexual
penetration with a person who is in the custody of that penal system;
(2) he or she is an employee of a treatment and detention facility and
engages in sexual conduct or sexual penetration with a person who is in
the custody of that treatment and detention facility; or (3) he or she
is an employee of a law enforcement agency and engages in sexual conduct
or sexual penetration with a person who is in the custody of a law
enforcement agency or employee.

(b) A probation or supervising officer, surveillance agent, or aftercare
specialist commits custodial sexual misconduct when the probation or
supervising officer, surveillance agent, or aftercare specialist engages
in sexual conduct or sexual penetration with a probationer, parolee, or
releasee or person serving a term of conditional release who is under
the supervisory, disciplinary, or custodial authority of the officer or
agent or employee so engaging in the sexual conduct or sexual
penetration.

(c) Custodial sexual misconduct is a Class 3 felony.

(d) Any person convicted of violating this Section immediately shall
forfeit his or her employment with a law enforcement agency, a penal
system, a treatment and detention facility, or a conditional release
program.

(e) In this Section, the consent of the probationer, parolee, releasee,
inmate in custody of the penal system or person detained or civilly
committed under the Sexually Violent Persons Commitment Act, or a person
in the custody of a law enforcement agency or employee shall not be a
defense to a prosecution under this Section. A person is deemed
incapable of consent, for purposes of this Section, when he or she is a
probationer, parolee, releasee, inmate in custody of a penal system or
person detained or civilly committed under the Sexually Violent Persons
Commitment Act, or a person in the custody of a law enforcement agency
or employee.

(f) This Section does not apply to:

(1) Any employee, probation or supervising officer, surveillance agent,
or aftercare specialist who is lawfully married to a person in custody
if the marriage occurred before the date of custody.

(2) Any employee, probation or supervising officer, surveillance agent,
or aftercare specialist who has no knowledge, and would have no reason
to believe, that the person with whom he or she engaged in custodial
sexual misconduct was a person in custody.

(g) In this Section:

(0.5) ``Aftercare specialist'' means any person employed by the
Department of Juvenile Justice to supervise and facilitate services for
persons placed on aftercare release.

(1) ``Custody'' means:

(i) pretrial incarceration or detention;

(ii) incarceration or detention under a sentence or commitment to a
State or local penal institution;

(iii) parole, aftercare release, or mandatory supervised release;

(iv) electronic monitoring or home detention;

(v) probation;

(vi) detention or civil commitment either in secure care or in the
community under the Sexually Violent Persons Commitment Act; or

(vii) detention or arrest by a law enforcement agency or employee.

(2) ``Penal system'' means any system which includes institutions as
defined in Section 2-14 of this Code or a county shelter care or
detention home established under Section 1 of the County Shelter Care
and Detention Home Act.

(2.1) ``Treatment and detention facility'' means any

Department of Human Services facility established for the detention or
civil commitment of persons under the Sexually Violent Persons
Commitment Act.

(2.2) ``Conditional release'' means a program of treatment and services,
vocational services, and alcohol or other drug abuse treatment provided
to any person civilly committed and conditionally released to the
community under the Sexually Violent Persons Commitment Act;

(3) ``Employee'' means:

(i) an employee of any governmental agency of this State or any county
or municipal corporation that has by statute, ordinance, or court order
the responsibility for the care, control, or supervision of pretrial or
sentenced persons in a penal system or persons detained or civilly
committed under the Sexually Violent Persons Commitment Act;

(ii) a contractual employee of a penal system as defined in paragraph
(g)(2) of this Section who works in a penal institution as defined in
Section 2-14 of this Code;

(iii) a contractual employee of a ``treatment and detention facility''
as defined in paragraph (g)(2.1) of this Code or a contractual employee
of the Department of Human Services who provides supervision of persons
serving a term of conditional release as defined in paragraph (g)(2.2)
of this Code; or

(iv) an employee of a law enforcement agency.

(3.5) ``Law enforcement agency'' means an agency of the

State or of a unit of local government charged with enforcement of
State, county, or municipal laws or with managing custody of detained
persons in the State, but not including a State's Attorney.

(4) ``Sexual conduct'' or ``sexual penetration'' means any act of sexual
conduct or sexual penetration as defined in Section 11-0.1 of this Code.

(5) ``Probation officer'' means any person employed in a probation or
court services department as defined in Section 9b of the Probation and
Probation Officers Act.

(6) ``Supervising officer'' means any person employed to supervise
persons placed on parole or mandatory supervised release with the duties
described in Section 3-14-2 of the Unified Code of Corrections.

(7) ``Surveillance agent'' means any person employed or contracted to
supervise persons placed on conditional release in the community under
the Sexually Violent Persons Commitment Act.

(Source: P.A. 100-431, eff. 8-25-17; 100-693, eff. 8-3-18; 101-81, eff.
7-12-19.)

\hypertarget{ilcs-511-9.3}{%
\subsection*{(720 ILCS 5/11-9.3)}\label{ilcs-511-9.3}}
\addcontentsline{toc}{subsection}{(720 ILCS 5/11-9.3)}

\hypertarget{sec.-11-9.3.-presence-within-school-zone-by-child-sex-offenders-prohibited-approaching-contacting-residing-with-or-communicating-with-a-child-within-certain-places-by-child-sex-offenders-prohibited.}{%
\section*{Sec. 11-9.3. Presence within school zone by child sex
offenders prohibited; approaching, contacting, residing with, or
communicating with a child within certain places by child sex offenders
prohibited.}\label{sec.-11-9.3.-presence-within-school-zone-by-child-sex-offenders-prohibited-approaching-contacting-residing-with-or-communicating-with-a-child-within-certain-places-by-child-sex-offenders-prohibited.}}
\addcontentsline{toc}{section}{Sec. 11-9.3. Presence within school zone
by child sex offenders prohibited; approaching, contacting, residing
with, or communicating with a child within certain places by child sex
offenders prohibited.}

\markright{Sec. 11-9.3. Presence within school zone by child sex
offenders prohibited; approaching, contacting, residing with, or
communicating with a child within certain places by child sex offenders
prohibited.}

(a) It is unlawful for a child sex offender to knowingly be present in
any school building, on real property comprising any school, or in any
conveyance owned, leased, or contracted by a school to transport
students to or from school or a school related activity when persons
under the age of 18 are present in the building, on the grounds or in
the conveyance, unless the offender is a parent or guardian of a student
attending the school and the parent or guardian is: (i) attending a
conference at the school with school personnel to discuss the progress
of his or her child academically or socially, (ii) participating in
child review conferences in which evaluation and placement decisions may
be made with respect to his or her child regarding special education
services, or (iii) attending conferences to discuss other student issues
concerning his or her child such as retention and promotion and notifies
the principal of the school of his or her presence at the school or
unless the offender has permission to be present from the superintendent
or the school board or in the case of a private school from the
principal. In the case of a public school, if permission is granted, the
superintendent or school board president must inform the principal of
the school where the sex offender will be present. Notification includes
the nature of the sex offender's visit and the hours in which the sex
offender will be present in the school. The sex offender is responsible
for notifying the principal's office when he or she arrives on school
property and when he or she departs from school property. If the sex
offender is to be present in the vicinity of children, the sex offender
has the duty to remain under the direct supervision of a school
official.

(a-5) It is unlawful for a child sex offender to knowingly be present
within 100 feet of a site posted as a pick-up or discharge stop for a
conveyance owned, leased, or contracted by a school to transport
students to or from school or a school related activity when one or more
persons under the age of 18 are present at the site.

(a-10) It is unlawful for a child sex offender to knowingly be present
in any public park building, a playground or recreation area within any
publicly accessible privately owned building, or on real property
comprising any public park when persons under the age of 18 are present
in the building or on the grounds and to approach, contact, or
communicate with a child under 18 years of age, unless the offender is a
parent or guardian of a person under 18 years of age present in the
building or on the grounds.

(b) It is unlawful for a child sex offender to knowingly loiter within
500 feet of a school building or real property comprising any school
while persons under the age of 18 are present in the building or on the
grounds, unless the offender is a parent or guardian of a student
attending the school and the parent or guardian is: (i) attending a
conference at the school with school personnel to discuss the progress
of his or her child academically or socially, (ii) participating in
child review conferences in which evaluation and placement decisions may
be made with respect to his or her child regarding special education
services, or (iii) attending conferences to discuss other student issues
concerning his or her child such as retention and promotion and notifies
the principal of the school of his or her presence at the school or has
permission to be present from the superintendent or the school board or
in the case of a private school from the principal. In the case of a
public school, if permission is granted, the superintendent or school
board president must inform the principal of the school where the sex
offender will be present. Notification includes the nature of the sex
offender's visit and the hours in which the sex offender will be present
in the school. The sex offender is responsible for notifying the
principal's office when he or she arrives on school property and when he
or she departs from school property. If the sex offender is to be
present in the vicinity of children, the sex offender has the duty to
remain under the direct supervision of a school official.

(b-2) It is unlawful for a child sex offender to knowingly loiter on a
public way within 500 feet of a public park building or real property
comprising any public park while persons under the age of 18 are present
in the building or on the grounds and to approach, contact, or
communicate with a child under 18 years of age, unless the offender is a
parent or guardian of a person under 18 years of age present in the
building or on the grounds.

(b-5) It is unlawful for a child sex offender to knowingly reside within
500 feet of a school building or the real property comprising any school
that persons under the age of 18 attend. Nothing in this subsection
(b-5) prohibits a child sex offender from residing within 500 feet of a
school building or the real property comprising any school that persons
under 18 attend if the property is owned by the child sex offender and
was purchased before July 7, 2000 (the effective date of Public Act
91-911).

(b-10) It is unlawful for a child sex offender to knowingly reside
within 500 feet of a playground, child care institution, day care
center, part day child care facility, day care home, group day care
home, or a facility providing programs or services exclusively directed
toward persons under 18 years of age. Nothing in this subsection (b-10)
prohibits a child sex offender from residing within 500 feet of a
playground or a facility providing programs or services exclusively
directed toward persons under 18 years of age if the property is owned
by the child sex offender and was purchased before July 7, 2000. Nothing
in this subsection (b-10) prohibits a child sex offender from residing
within 500 feet of a child care institution, day care center, or part
day child care facility if the property is owned by the child sex
offender and was purchased before June 26, 2006. Nothing in this
subsection (b-10) prohibits a child sex offender from residing within
500 feet of a day care home or group day care home if the property is
owned by the child sex offender and was purchased before August 14, 2008
(the effective date of Public Act 95-821).

(b-15) It is unlawful for a child sex offender to knowingly reside
within 500 feet of the victim of the sex offense. Nothing in this
subsection (b-15) prohibits a child sex offender from residing within
500 feet of the victim if the property in which the child sex offender
resides is owned by the child sex offender and was purchased before
August 22, 2002.

This subsection (b-15) does not apply if the victim of the sex offense
is 21 years of age or older.

(b-20) It is unlawful for a child sex offender to knowingly communicate,
other than for a lawful purpose under Illinois law, using the Internet
or any other digital media, with a person under 18 years of age or with
a person whom he or she believes to be a person under 18 years of age,
unless the offender is a parent or guardian of the person under 18 years
of age.

(c) It is unlawful for a child sex offender to knowingly operate,
manage, be employed by, volunteer at, be associated with, or knowingly
be present at any: (i) facility providing programs or services
exclusively directed toward persons under the age of 18; (ii) day care
center; (iii) part day child care facility; (iv) child care institution;
(v) school providing before and after school programs for children under
18 years of age; (vi) day care home; or (vii) group day care home. This
does not prohibit a child sex offender from owning the real property
upon which the programs or services are offered or upon which the day
care center, part day child care facility, child care institution, or
school providing before and after school programs for children under 18
years of age is located, provided the child sex offender refrains from
being present on the premises for the hours during which: (1) the
programs or services are being offered or (2) the day care center, part
day child care facility, child care institution, or school providing
before and after school programs for children under 18 years of age, day
care home, or group day care home is operated.

(c-2) It is unlawful for a child sex offender to participate in a
holiday event involving children under 18 years of age, including but
not limited to distributing candy or other items to children on
Halloween, wearing a Santa Claus costume on or preceding Christmas,
being employed as a department store Santa Claus, or wearing an Easter
Bunny costume on or preceding Easter. For the purposes of this
subsection, child sex offender has the meaning as defined in this
Section, but does not include as a sex offense under paragraph (2) of
subsection (d) of this Section, the offense under subsection (c) of
Section 11-1.50 of this Code. This subsection does not apply to a child
sex offender who is a parent or guardian of children under 18 years of
age that are present in the home and other non-familial minors are not
present.

(c-5) It is unlawful for a child sex offender to knowingly operate,
manage, be employed by, or be associated with any carnival, amusement
enterprise, or county or State fair when persons under the age of 18 are
present.

(c-6) It is unlawful for a child sex offender who owns and resides at
residential real estate to knowingly rent any residential unit within
the same building in which he or she resides to a person who is the
parent or guardian of a child or children under 18 years of age. This
subsection shall apply only to leases or other rental arrangements
entered into after January 1, 2009 (the effective date of Public Act
95-820).

(c-7) It is unlawful for a child sex offender to knowingly offer or
provide any programs or services to persons under 18 years of age in his
or her residence or the residence of another or in any facility for the
purpose of offering or providing such programs or services, whether such
programs or services are offered or provided by contract, agreement,
arrangement, or on a volunteer basis.

(c-8) It is unlawful for a child sex offender to knowingly operate,
whether authorized to do so or not, any of the following vehicles: (1) a
vehicle which is specifically designed, constructed or modified and
equipped to be used for the retail sale of food or beverages, including
but not limited to an ice cream truck; (2) an authorized emergency
vehicle; or (3) a rescue vehicle.

(d) Definitions. In this Section:

(1) ``Child sex offender'' means any person who:

(i) has been charged under Illinois law, or any substantially similar
federal law or law of another state, with a sex offense set forth in
paragraph (2) of this subsection (d) or the attempt to commit an
included sex offense, and the victim is a person under 18 years of age
at the time of the offense; and:

(A) is convicted of such offense or an attempt to commit such offense;
or

(B) is found not guilty by reason of insanity of such offense or an
attempt to commit such offense; or

(C) is found not guilty by reason of insanity pursuant to subsection (c)
of Section 104-25 of the Code of Criminal Procedure of 1963 of such
offense or an attempt to commit such offense; or

(D) is the subject of a finding not resulting in an acquittal at a
hearing conducted pursuant to subsection (a) of Section 104-25 of the
Code of Criminal Procedure of 1963 for the alleged commission or
attempted commission of such offense; or

(E) is found not guilty by reason of insanity following a hearing
conducted pursuant to a federal law or the law of another state
substantially similar to subsection (c) of Section 104-25 of the Code of
Criminal Procedure of 1963 of such offense or of the attempted
commission of such offense; or

(F) is the subject of a finding not resulting in an acquittal at a
hearing conducted pursuant to a federal law or the law of another state
substantially similar to subsection (a) of Section 104-25 of the Code of
Criminal Procedure of 1963 for the alleged violation or attempted
commission of such offense; or

(ii) is certified as a sexually dangerous person pursuant to the
Illinois Sexually Dangerous Persons Act, or any substantially similar
federal law or the law of another state, when any conduct giving rise to
such certification is committed or attempted against a person less than
18 years of age; or

(iii) is subject to the provisions of Section 2 of the Interstate
Agreements on Sexually Dangerous Persons Act.

Convictions that result from or are connected with the same act, or
result from offenses committed at the same time, shall be counted for
the purpose of this Section as one conviction. Any conviction set aside
pursuant to law is not a conviction for purposes of this Section.

(2) Except as otherwise provided in paragraph (2.5),

``sex offense'' means:

(i) A violation of any of the following Sections of the Criminal Code of
1961 or the Criminal Code of 2012: 10-4 (forcible detention), 10-7
(aiding or abetting child abduction under Section 10-5(b)(10)),
10-5(b)(10) (child luring), 11-1.40 (predatory criminal sexual assault
of a child), 11-6 (indecent solicitation of a child), 11-6.5 (indecent
solicitation of an adult), 11-9.1 (sexual exploitation of a child),
11-9.2 (custodial sexual misconduct), 11-9.5 (sexual misconduct with a
person with a disability), 11-11 (sexual relations within families),
11-14.3(a)(1) (promoting prostitution by advancing prostitution),
11-14.3(a)(2)(A) (promoting prostitution by profiting from prostitution
by compelling a person to be a prostitute), 11-14.3(a)(2)(C) (promoting
prostitution by profiting from prostitution by means other than as
described in subparagraphs (A) and (B) of paragraph (2) of subsection
(a) of Section 11-14.3), 11-14.4 (promoting juvenile prostitution),
11-18.1 (patronizing a juvenile prostitute), 11-20.1 (child
pornography), 11-20.1B (aggravated child pornography), 11-21 (harmful
material), 11-25 (grooming), 11-26 (traveling to meet a minor or
traveling to meet a child), 12-33 (ritualized abuse of a child), 11-20
(obscenity) (when that offense was committed in any school, on real
property comprising any school, in any conveyance owned, leased, or
contracted by a school to transport students to or from school or a
school related activity, or in a public park), 11-30 (public indecency)
(when committed in a school, on real property comprising a school, in
any conveyance owned, leased, or contracted by a school to transport
students to or from school or a school related activity, or in a public
park). An attempt to commit any of these offenses.

(ii) A violation of any of the following Sections of the Criminal Code
of 1961 or the Criminal Code of 2012, when the victim is a person under
18 years of age: 11-1.20 (criminal sexual assault), 11-1.30 (aggravated
criminal sexual assault), 11-1.50 (criminal sexual abuse), 11-1.60
(aggravated criminal sexual abuse). An attempt to commit any of these
offenses.

(iii) A violation of any of the following

Sections of the Criminal Code of 1961 or the Criminal Code of 2012, when
the victim is a person under 18 years of age and the defendant is not a
parent of the victim:

10-1 (kidnapping),

10-2 (aggravated kidnapping),

10-3 (unlawful restraint),

10-3.1 (aggravated unlawful restraint),

11-9.1(A) (permitting sexual abuse of a child).

An attempt to commit any of these offenses.

(iv) A violation of any former law of this State substantially
equivalent to any offense listed in clause (2)(i) or (2)(ii) of
subsection (d) of this Section.

(2.5) For the purposes of subsections (b-5) and

(b-10) only, a sex offense means:

(i) A violation of any of the following Sections of the Criminal Code of
1961 or the Criminal Code of 2012:

10-5(b)(10) (child luring), 10-7 (aiding or abetting child abduction
under Section 10-5(b)(10)), 11-1.40 (predatory criminal sexual assault
of a child), 11-6 (indecent solicitation of a child), 11-6.5 (indecent
solicitation of an adult), 11-9.2 (custodial sexual misconduct), 11-9.5
(sexual misconduct with a person with a disability), 11-11 (sexual
relations within families), 11-14.3(a)(1) (promoting prostitution by
advancing prostitution), 11-14.3(a)(2)(A) (promoting prostitution by
profiting from prostitution by compelling a person to be a prostitute),
11-14.3(a)(2)(C) (promoting prostitution by profiting from prostitution
by means other than as described in subparagraphs (A) and (B) of
paragraph (2) of subsection (a) of Section 11-14.3), 11-14.4 (promoting
juvenile prostitution), 11-18.1 (patronizing a juvenile prostitute),
11-20.1 (child pornography), 11-20.1B (aggravated child pornography),
11-25 (grooming), 11-26 (traveling to meet a minor or traveling to meet
a child), or 12-33 (ritualized abuse of a child). An attempt to commit
any of these offenses.

(ii) A violation of any of the following Sections of the Criminal Code
of 1961 or the Criminal Code of 2012, when the victim is a person under
18 years of age: 11-1.20 (criminal sexual assault), 11-1.30 (aggravated
criminal sexual assault), 11-1.60 (aggravated criminal sexual abuse),
and subsection (a) of Section 11-1.50 (criminal sexual abuse). An
attempt to commit any of these offenses.

(iii) A violation of any of the following

Sections of the Criminal Code of 1961 or the Criminal Code of 2012, when
the victim is a person under 18 years of age and the defendant is not a
parent of the victim:

10-1 (kidnapping),

10-2 (aggravated kidnapping),

10-3 (unlawful restraint),

10-3.1 (aggravated unlawful restraint),

11-9.1(A) (permitting sexual abuse of a child).

An attempt to commit any of these offenses.

(iv) A violation of any former law of this State substantially
equivalent to any offense listed in this paragraph (2.5) of this
subsection.

(3) A conviction for an offense of federal law or the law of another
state that is substantially equivalent to any offense listed in
paragraph (2) of subsection (d) of this Section shall constitute a
conviction for the purpose of this Section. A finding or adjudication as
a sexually dangerous person under any federal law or law of another
state that is substantially equivalent to the Sexually Dangerous Persons
Act shall constitute an adjudication for the purposes of this Section.

(4) ``Authorized emergency vehicle'', ``rescue vehicle'', and
``vehicle'' have the meanings ascribed to them in Sections 1-105,
1-171.8 and 1-217, respectively, of the Illinois Vehicle Code.

(5) ``Child care institution'' has the meaning ascribed to it in Section
2.06 of the Child Care Act of 1969.

(6) ``Day care center'' has the meaning ascribed to it in Section 2.09
of the Child Care Act of 1969.

(7) ``Day care home'' has the meaning ascribed to it in

Section 2.18 of the Child Care Act of 1969.

(8) ``Facility providing programs or services directed towards persons
under the age of 18'' means any facility providing programs or services
exclusively directed towards persons under the age of 18.

(9) ``Group day care home'' has the meaning ascribed to it in Section
2.20 of the Child Care Act of 1969.

(10) ``Internet'' has the meaning set forth in Section

16-0.1 of this Code.

(11) ``Loiter'' means:

(i) Standing, sitting idly, whether or not the person is in a vehicle,
or remaining in or around school or public park property.

(ii) Standing, sitting idly, whether or not the person is in a vehicle,
or remaining in or around school or public park property, for the
purpose of committing or attempting to commit a sex offense.

(iii) Entering or remaining in a building in or around school property,
other than the offender's residence.

(12) ``Part day child care facility'' has the meaning ascribed to it in
Section 2.10 of the Child Care Act of 1969.

(13) ``Playground'' means a piece of land owned or controlled by a unit
of local government that is designated by the unit of local government
for use solely or primarily for children's recreation.

(14) ``Public park'' includes a park, forest preserve, bikeway, trail,
or conservation area under the jurisdiction of the State or a unit of
local government.

(15) ``School'' means a public or private preschool or elementary or
secondary school.

(16) ``School official'' means the principal, a teacher, or any other
certified employee of the school, the superintendent of schools or a
member of the school board.

(e) For the purposes of this Section, the 500 feet distance shall be
measured from: (1) the edge of the property of the school building or
the real property comprising the school that is closest to the edge of
the property of the child sex offender's residence or where he or she is
loitering, and (2) the edge of the property comprising the public park
building or the real property comprising the public park, playground,
child care institution, day care center, part day child care facility,
or facility providing programs or services exclusively directed toward
persons under 18 years of age, or a victim of the sex offense who is
under 21 years of age, to the edge of the child sex offender's place of
residence or place where he or she is loitering.

(f) Sentence. A person who violates this Section is guilty of a Class 4
felony.

(Source: P.A. 102-997, eff. 1-1-23 .)

\hypertarget{ilcs-511-9.4}{%
\subsection*{(720 ILCS 5/11-9.4)}\label{ilcs-511-9.4}}
\addcontentsline{toc}{subsection}{(720 ILCS 5/11-9.4)}

\hypertarget{sec.-11-9.4.-repealed.}{%
\section*{Sec. 11-9.4. (Repealed).}\label{sec.-11-9.4.-repealed.}}
\addcontentsline{toc}{section}{Sec. 11-9.4. (Repealed).}

\markright{Sec. 11-9.4. (Repealed).}

(Source: P.A. 96-1000, eff. 7-2-10. Repealed by P.A. 96-1551, eff.
7-1-11.)

\hypertarget{ilcs-511-9.4-1}{%
\subsection*{(720 ILCS 5/11-9.4-1)}\label{ilcs-511-9.4-1}}
\addcontentsline{toc}{subsection}{(720 ILCS 5/11-9.4-1)}

\hypertarget{sec.-11-9.4-1.-sexual-predator-and-child-sex-offender-presence-or-loitering-in-or-near-public-parks-prohibited.}{%
\section*{Sec. 11-9.4-1. Sexual predator and child sex offender;
presence or loitering in or near public parks
prohibited.}\label{sec.-11-9.4-1.-sexual-predator-and-child-sex-offender-presence-or-loitering-in-or-near-public-parks-prohibited.}}
\addcontentsline{toc}{section}{Sec. 11-9.4-1. Sexual predator and child
sex offender; presence or loitering in or near public parks prohibited.}

\markright{Sec. 11-9.4-1. Sexual predator and child sex offender;
presence or loitering in or near public parks prohibited.}

(a) For the purposes of this Section:

``Child sex offender'' has the meaning ascribed to it in subsection (d)
of Section 11-9.3 of this Code, but does not include as a sex offense
under paragraph (2) of subsection (d) of Section 11-9.3, the offenses
under subsections (b) and (c) of Section 11-1.50 or subsections (b) and
(c) of Section 12-15 of this Code.

``Public park'' includes a park, forest preserve, bikeway, trail, or
conservation area under the jurisdiction of the State or a unit of local
government.

``Loiter'' means:

(i) Standing, sitting idly, whether or not the person is in a vehicle or
remaining in or around public park property.

(ii) Standing, sitting idly, whether or not the person is in a vehicle
or remaining in or around public park property, for the purpose of
committing or attempting to commit a sex offense.

``Sexual predator'' has the meaning ascribed to it in subsection (E) of
Section 2 of the Sex Offender Registration Act.

(b) It is unlawful for a sexual predator or a child sex offender to
knowingly be present in any public park building or on real property
comprising any public park.

(c) It is unlawful for a sexual predator or a child sex offender to
knowingly loiter on a public way within 500 feet of a public park
building or real property comprising any public park. For the purposes
of this subsection (c), the 500 feet distance shall be measured from the
edge of the property comprising the public park building or the real
property comprising the public park.

(d) Sentence. A person who violates this Section is guilty of a Class A
misdemeanor, except that a second or subsequent violation is a Class 4
felony.

(Source: P.A. 96-1099, eff. 1-1-11; 97-698, eff. 1-1-13; 97-1109, eff.
1-1-13.)

\hypertarget{ilcs-511-9.5}{%
\subsection*{(720 ILCS 5/11-9.5)}\label{ilcs-511-9.5}}
\addcontentsline{toc}{subsection}{(720 ILCS 5/11-9.5)}

\hypertarget{sec.-11-9.5.-sexual-misconduct-with-a-person-with-a-disability.}{%
\section*{Sec. 11-9.5. Sexual misconduct with a person with a
disability.}\label{sec.-11-9.5.-sexual-misconduct-with-a-person-with-a-disability.}}
\addcontentsline{toc}{section}{Sec. 11-9.5. Sexual misconduct with a
person with a disability.}

\markright{Sec. 11-9.5. Sexual misconduct with a person with a
disability.}

(a) Definitions. As used in this Section:

(1) ``Person with a disability'' means:

(i) a person diagnosed with a developmental disability as defined in
Section 1-106 of the Mental Health and Developmental Disabilities Code;
or

(ii) a person diagnosed with a mental illness as defined in Section
1-129 of the Mental Health and Developmental Disabilities Code.

(2) ``State-operated facility'' means:

(i) a developmental disability facility as defined in the Mental Health
and Developmental Disabilities Code; or

(ii) a mental health facility as defined in the

Mental Health and Developmental Disabilities Code.

(3) ``Community agency'' or ``agency'' means any community entity or
program providing residential mental health or developmental
disabilities services that is licensed, certified, or funded by the
Department of Human Services and not licensed or certified by any other
human service agency of the State such as the Departments of Public
Health, Healthcare and Family Services, and Children and Family
Services.

(4) ``Care and custody'' means admission to a

State-operated facility.

(5) ``Employee'' means:

(i) any person employed by the Illinois

Department of Human Services;

(ii) any person employed by a community agency providing services at the
direction of the owner or operator of the agency on or off site; or

(iii) any person who is a contractual employee or contractual agent of
the Department of Human Services or the community agency. This includes
but is not limited to payroll personnel, contractors, subcontractors,
and volunteers.

(6) ``Sexual conduct'' or ``sexual penetration'' means any act of sexual
conduct or sexual penetration as defined in Section 11-0.1 of this Code.

(b) A person commits sexual misconduct with a person with a disability
when:

(1) he or she is an employee and knowingly engages in sexual conduct or
sexual penetration with a person with a disability who is under the care
and custody of the Department of Human Services at a State-operated
facility; or

(2) he or she is an employee of a community agency funded by the
Department of Human Services and knowingly engages in sexual conduct or
sexual penetration with a person with a disability who is in a
residential program operated or supervised by a community agency.

(c) For purposes of this Section, the consent of a person with a
disability in custody of the Department of Human Services residing at a
State-operated facility or receiving services from a community agency
shall not be a defense to a prosecution under this Section. A person is
deemed incapable of consent, for purposes of this Section, when he or
she is a person with a disability and is receiving services at a
State-operated facility or is a person with a disability who is in a
residential program operated or supervised by a community agency.

(d) This Section does not apply to:

(1) any State employee or any community agency employee who is lawfully
married to a person with a disability in custody of the Department of
Human Services or receiving services from a community agency if the
marriage occurred before the date of custody or the initiation of
services at a community agency; or

(2) any State employee or community agency employee who has no
knowledge, and would have no reason to believe, that the person with
whom he or she engaged in sexual misconduct was a person with a
disability in custody of the Department of Human Services or was
receiving services from a community agency.

(e) Sentence. Sexual misconduct with a person with a disability is a
Class 3 felony.

(f) Any person convicted of violating this Section shall immediately
forfeit his or her employment with the State or the community agency.

(Source: P.A. 96-1551, eff. 7-1-11 .)

\hypertarget{ilcs-511-11-from-ch.-38-par.-11-11}{%
\subsection*{(720 ILCS 5/11-11) (from Ch. 38, par.
11-11)}\label{ilcs-511-11-from-ch.-38-par.-11-11}}
\addcontentsline{toc}{subsection}{(720 ILCS 5/11-11) (from Ch. 38, par.
11-11)}

\hypertarget{sec.-11-11.-sexual-relations-within-families.}{%
\section*{Sec. 11-11. Sexual Relations Within
Families.}\label{sec.-11-11.-sexual-relations-within-families.}}
\addcontentsline{toc}{section}{Sec. 11-11. Sexual Relations Within
Families.}

\markright{Sec. 11-11. Sexual Relations Within Families.}

(a) A person commits sexual relations within families if he or she:

(1) Commits an act of sexual penetration as defined in Section 11-0.1 of
this Code; and

(2) The person knows that he or she is related to the other person as
follows: (i) Brother or sister, either of the whole blood or the half
blood; or (ii) Father or mother, when the child, regardless of
legitimacy and regardless of whether the child was of the whole blood or
half-blood or was adopted, was 18 years of age or over when the act was
committed; or (iii) Stepfather or stepmother, when the stepchild was 18
years of age or over when the act was committed; or (iv) Aunt or uncle,
when the niece or nephew was 18 years of age or over when the act was
committed; or (v) Great-aunt or great-uncle, when the grand-niece or
grand-nephew was 18 years of age or over when the act was committed; or
(vi) Grandparent or step-grandparent, when the grandchild or
step-grandchild was 18 years of age or over when the act was committed.

(b) Sentence. Sexual relations within families is a Class 3 felony.

(Source: P.A. 96-233, eff. 1-1-10; 96-1551, eff. 7-1-11 .)

\hypertarget{ilcs-511-12-from-ch.-38-par.-11-12}{%
\subsection*{(720 ILCS 5/11-12) (from Ch. 38, par.
11-12)}\label{ilcs-511-12-from-ch.-38-par.-11-12}}
\addcontentsline{toc}{subsection}{(720 ILCS 5/11-12) (from Ch. 38, par.
11-12)}

(This Section was renumbered as Section 11-45 by P.A. 96-1551.)

\hypertarget{sec.-11-12.-renumbered.}{%
\section*{Sec. 11-12. (Renumbered).}\label{sec.-11-12.-renumbered.}}
\addcontentsline{toc}{section}{Sec. 11-12. (Renumbered).}

\markright{Sec. 11-12. (Renumbered).}

(Source: P.A. 81-230. Renumbered by P.A. 96-1551, eff. 7-1-11 .)

\hypertarget{ilcs-511-13-from-ch.-38-par.-11-13}{%
\subsection*{(720 ILCS 5/11-13) (from Ch. 38, par.
11-13)}\label{ilcs-511-13-from-ch.-38-par.-11-13}}
\addcontentsline{toc}{subsection}{(720 ILCS 5/11-13) (from Ch. 38, par.
11-13)}

\hypertarget{sec.-11-13.-repealed.}{%
\section*{Sec. 11-13. (Repealed).}\label{sec.-11-13.-repealed.}}
\addcontentsline{toc}{section}{Sec. 11-13. (Repealed).}

\markright{Sec. 11-13. (Repealed).}

(Source: P.A. 77-2638. Repealed by P.A. 96-1551, eff. 7-1-11 .)

\hypertarget{ilcs-5art.-11-subdiv.-15-heading}{%
\subsection*{(720 ILCS 5/Art. 11 Subdiv. 15
heading)}\label{ilcs-5art.-11-subdiv.-15-heading}}
\addcontentsline{toc}{subsection}{(720 ILCS 5/Art. 11 Subdiv. 15
heading)}

SUBDIVISION 15.

PROSTITUTION OFFENSES

(Source: P.A. 96-1551, eff. 7-1-11.)

\hypertarget{ilcs-511-14-from-ch.-38-par.-11-14}{%
\subsection*{(720 ILCS 5/11-14) (from Ch. 38, par.
11-14)}\label{ilcs-511-14-from-ch.-38-par.-11-14}}
\addcontentsline{toc}{subsection}{(720 ILCS 5/11-14) (from Ch. 38, par.
11-14)}

\hypertarget{sec.-11-14.-prostitution.}{%
\section*{Sec. 11-14. Prostitution.}\label{sec.-11-14.-prostitution.}}
\addcontentsline{toc}{section}{Sec. 11-14. Prostitution.}

\markright{Sec. 11-14. Prostitution.}

(a) Any person who knowingly performs, offers or agrees to perform any
act of sexual penetration as defined in Section 11-0.1 of this Code for
anything of value, or any touching or fondling of the sex organs of one
person by another person, for anything of value, for the purpose of
sexual arousal or gratification commits an act of prostitution.

(b) Sentence. A violation of this Section is a Class A misdemeanor.

(c) (Blank).

(c-5) It is an affirmative defense to a charge under this Section that
the accused engaged in or performed prostitution as a result of being a
victim of involuntary servitude or trafficking in persons as defined in
Section 10-9 of this Code.

(d) Notwithstanding the foregoing, if it is determined, after a
reasonable detention for investigative purposes, that a person suspected
of or charged with a violation of this Section is a person under the age
of 18, that person shall be immune from prosecution for a prostitution
offense under this Section, and shall be subject to the temporary
protective custody provisions of Sections 2-5 and 2-6 of the Juvenile
Court Act of 1987. Pursuant to the provisions of Section 2-6 of the
Juvenile Court Act of 1987, a law enforcement officer who takes a person
under 18 years of age into custody under this Section shall immediately
report an allegation of a violation of Section 10-9 of this Code to the
Illinois Department of Children and Family Services State Central
Register, which shall commence an initial investigation into child abuse
or child neglect within 24 hours pursuant to Section 7.4 of the Abused
and Neglected Child Reporting Act.

(Source: P.A. 98-164, eff. 1-1-14; 98-538, eff. 8-23-13; 98-756, eff.
7-16-14; 99-109, eff. 7-22-15.)

\hypertarget{ilcs-511-14.1}{%
\subsection*{(720 ILCS 5/11-14.1)}\label{ilcs-511-14.1}}
\addcontentsline{toc}{subsection}{(720 ILCS 5/11-14.1)}

\hypertarget{sec.-11-14.1.-solicitation-of-a-sexual-act.}{%
\section*{Sec. 11-14.1. Solicitation of a sexual
act.}\label{sec.-11-14.1.-solicitation-of-a-sexual-act.}}
\addcontentsline{toc}{section}{Sec. 11-14.1. Solicitation of a sexual
act.}

\markright{Sec. 11-14.1. Solicitation of a sexual act.}

(a) Any person who offers a person not his or her spouse any money,
property, token, object, or article or anything of value for that person
or any other person not his or her spouse to perform any act of sexual
penetration as defined in Section 11-0.1 of this Code, or any touching
or fondling of the sex organs of one person by another person for the
purpose of sexual arousal or gratification, commits solicitation of a
sexual act.

(b) Sentence. Solicitation of a sexual act is a Class A misdemeanor.
Solicitation of a sexual act from a person who is under the age of 18 or
who is a person with a severe or profound intellectual disability is a
Class 4 felony. If the court imposes a fine under this subsection (b),
it shall be collected and distributed to the Specialized Services for
Survivors of Human Trafficking Fund in accordance with Section 5-9-1.21
of the Unified Code of Corrections.

(b-5) (Blank).

(c) This Section does not apply to a person engaged in prostitution who
is under 18 years of age.

(d) A person cannot be convicted under this Section if the practice of
prostitution underlying the offense consists exclusively of the
accused's own acts of prostitution under Section 11-14 of this Code.

(Source: P.A. 102-939, eff. 1-1-23 .)

\hypertarget{ilcs-511-14.2}{%
\subsection*{(720 ILCS 5/11-14.2)}\label{ilcs-511-14.2}}
\addcontentsline{toc}{subsection}{(720 ILCS 5/11-14.2)}

\hypertarget{sec.-11-14.2.-repealed.}{%
\section*{Sec. 11-14.2. (Repealed).}\label{sec.-11-14.2.-repealed.}}
\addcontentsline{toc}{section}{Sec. 11-14.2. (Repealed).}

\markright{Sec. 11-14.2. (Repealed).}

(Source: P.A. 96-1464, eff. 8-20-10. Repealed by P.A. 96-1551, eff.
7-1-11.)

\hypertarget{ilcs-511-14.3}{%
\subsection*{(720 ILCS 5/11-14.3)}\label{ilcs-511-14.3}}
\addcontentsline{toc}{subsection}{(720 ILCS 5/11-14.3)}

\hypertarget{sec.-11-14.3.-promoting-prostitution.}{%
\section*{Sec. 11-14.3. Promoting
prostitution.}\label{sec.-11-14.3.-promoting-prostitution.}}
\addcontentsline{toc}{section}{Sec. 11-14.3. Promoting prostitution.}

\markright{Sec. 11-14.3. Promoting prostitution.}

(a) Any person who knowingly performs any of the following acts commits
promoting prostitution:

(1) advances prostitution as defined in Section

11-0.1;

(2) profits from prostitution by:

(A) compelling a person to become a prostitute;

(B) arranging or offering to arrange a situation in which a person may
practice prostitution; or

(C) any means other than those described in subparagraph (A) or (B),
including from a person who patronizes a prostitute. This paragraph (C)
does not apply to a person engaged in prostitution who is under 18 years
of age. A person cannot be convicted of promoting prostitution under
this paragraph (C) if the practice of prostitution underlying the
offense consists exclusively of the accused's own acts of prostitution
under Section 11-14 of this Code.

(b) Sentence.

(1) A violation of subdivision (a)(1) is a Class 4 felony, unless
committed within 1,000 feet of real property comprising a school, in
which case it is a Class 3 felony. A second or subsequent violation of
subdivision (a)(1), or any combination of convictions under subdivision
(a)(1), (a)(2)(A), or (a)(2)(B) and Section 11-14 (prostitution),
11-14.1 (solicitation of a sexual act), 11-14.4 (promoting juvenile
prostitution), 11-15 (soliciting for a prostitute), 11-15.1 (soliciting
for a juvenile prostitute), 11-16 (pandering), 11-17 (keeping a place of
prostitution), 11-17.1 (keeping a place of juvenile prostitution), 11-18
(patronizing a prostitute), 11-18.1 (patronizing a juvenile prostitute),
11-19 (pimping), 11-19.1 (juvenile pimping or aggravated juvenile
pimping), or 11-19.2 (exploitation of a child), is a Class 3 felony.

(2) A violation of subdivision (a)(2)(A) or (a)(2)(B) is a Class 4
felony, unless committed within 1,000 feet of real property comprising a
school, in which case it is a Class 3 felony.

(3) A violation of subdivision (a)(2)(C) is a Class 4 felony, unless
committed within 1,000 feet of real property comprising a school, in
which case it is a Class 3 felony. A second or subsequent violation of
subdivision (a)(2)(C), or any combination of convictions under
subdivision (a)(2)(C) and subdivision (a)(1), (a)(2)(A), or (a)(2)(B) of
this Section (promoting prostitution), 11-14 (prostitution), 11-14.1
(solicitation of a sexual act), 11-14.4 (promoting juvenile
prostitution), 11-15 (soliciting for a prostitute), 11-15.1 (soliciting
for a juvenile prostitute), 11-16 (pandering), 11-17 (keeping a place of
prostitution), 11-17.1 (keeping a place of juvenile prostitution), 11-18
(patronizing a prostitute), 11-18.1 (patronizing a juvenile prostitute),
11-19 (pimping), 11-19.1 (juvenile pimping or aggravated juvenile
pimping), or 11-19.2 (exploitation of a child), is a Class 3 felony.

If the court imposes a fine under this subsection (b), it shall be
collected and distributed to the Specialized Services for Survivors of
Human Trafficking Fund in accordance with Section 5-9-1.21 of the
Unified Code of Corrections.

(Source: P.A. 98-1013, eff. 1-1-15 .)

\hypertarget{ilcs-511-14.4}{%
\subsection*{(720 ILCS 5/11-14.4)}\label{ilcs-511-14.4}}
\addcontentsline{toc}{subsection}{(720 ILCS 5/11-14.4)}

\hypertarget{sec.-11-14.4.-promoting-juvenile-prostitution.}{%
\section*{Sec. 11-14.4. Promoting juvenile
prostitution.}\label{sec.-11-14.4.-promoting-juvenile-prostitution.}}
\addcontentsline{toc}{section}{Sec. 11-14.4. Promoting juvenile
prostitution.}

\markright{Sec. 11-14.4. Promoting juvenile prostitution.}

(a) Any person who knowingly performs any of the following acts commits
promoting juvenile prostitution:

(1) advances prostitution as defined in Section

11-0.1, where the minor engaged in prostitution, or any person engaged
in prostitution in the place, is under 18 years of age or is a person
with a severe or profound intellectual disability at the time of the
offense;

(2) profits from prostitution by any means where the prostituted person
is under 18 years of age or is a person with a severe or profound
intellectual disability at the time of the offense;

(3) profits from prostitution by any means where the prostituted person
is under 13 years of age at the time of the offense;

(4) confines a child under the age of 18 or a person with a severe or
profound intellectual disability against his or her will by the
infliction or threat of imminent infliction of great bodily harm or
permanent disability or disfigurement or by administering to the child
or the person with a severe or profound intellectual disability, without
his or her consent or by threat or deception and for other than medical
purposes, any alcoholic intoxicant or a drug as defined in the Illinois
Controlled Substances Act or the Cannabis Control Act or methamphetamine
as defined in the Methamphetamine Control and Community Protection Act
and:

(A) compels the child or the person with a severe or profound
intellectual disability to engage in prostitution;

(B) arranges a situation in which the child or the person with a severe
or profound intellectual disability may practice prostitution; or

(C) profits from prostitution by the child or the person with a severe
or profound intellectual disability.

(b) For purposes of this Section, administering drugs, as defined in
subdivision (a)(4), or an alcoholic intoxicant to a child under the age
of 13 or a person with a severe or profound intellectual disability
shall be deemed to be without consent if the administering is done
without the consent of the parents or legal guardian or if the
administering is performed by the parents or legal guardian for other
than medical purposes.

(c) If the accused did not have a reasonable opportunity to observe the
prostituted person, it is an affirmative defense to a charge of
promoting juvenile prostitution, except for a charge under subdivision
(a)(4), that the accused reasonably believed the person was of the age
of 18 years or over or was not a person with a severe or profound
intellectual disability at the time of the act giving rise to the
charge.

(d) Sentence. A violation of subdivision (a)(1) is a Class 1 felony,
unless committed within 1,000 feet of real property comprising a school,
in which case it is a Class X felony. A violation of subdivision (a)(2)
is a Class 1 felony. A violation of subdivision (a)(3) is a Class X
felony. A violation of subdivision (a)(4) is a Class X felony, for which
the person shall be sentenced to a term of imprisonment of not less than
6 years and not more than 60 years. A second or subsequent violation of
subdivision (a)(1), (a)(2), or (a)(3), or any combination of convictions
under subdivision (a)(1), (a)(2), or (a)(3) and Sections 11-14
(prostitution), 11-14.1 (solicitation of a sexual act), 11-14.3
(promoting prostitution), 11-15 (soliciting for a prostitute), 11-15.1
(soliciting for a juvenile prostitute), 11-16 (pandering), 11-17
(keeping a place of prostitution), 11-17.1 (keeping a place of juvenile
prostitution), 11-18 (patronizing a prostitute), 11-18.1 (patronizing a
juvenile prostitute), 11-19 (pimping), 11-19.1 (juvenile pimping or
aggravated juvenile pimping), or 11-19.2 (exploitation of a child) of
this Code, is a Class X felony.

(e) Forfeiture. Any person convicted of a violation of this Section that
involves promoting juvenile prostitution by keeping a place of juvenile
prostitution or convicted of a violation of subdivision (a)(4) is
subject to the property forfeiture provisions set forth in Article 124B
of the Code of Criminal Procedure of 1963.

(f) For the purposes of this Section, ``prostituted person'' means any
person who engages in, or agrees or offers to engage in, any act of
sexual penetration as defined in Section 11-0.1 of this Code for any
money, property, token, object, or article or anything of value, or any
touching or fondling of the sex organs of one person by another person,
for any money, property, token, object, or article or anything of value,
for the purpose of sexual arousal or gratification.

(Source: P.A. 99-143, eff. 7-27-15.)

\hypertarget{ilcs-511-15-from-ch.-38-par.-11-15}{%
\subsection*{(720 ILCS 5/11-15) (from Ch. 38, par.
11-15)}\label{ilcs-511-15-from-ch.-38-par.-11-15}}
\addcontentsline{toc}{subsection}{(720 ILCS 5/11-15) (from Ch. 38, par.
11-15)}

\hypertarget{sec.-11-15.-repealed.}{%
\section*{Sec. 11-15. (Repealed).}\label{sec.-11-15.-repealed.}}
\addcontentsline{toc}{section}{Sec. 11-15. (Repealed).}

\markright{Sec. 11-15. (Repealed).}

(Source: P.A. 96-1464, eff. 8-20-10. Repealed by P.A. 96-1551, eff.
7-1-11 .)

\hypertarget{ilcs-511-15.1-from-ch.-38-par.-11-15.1}{%
\subsection*{(720 ILCS 5/11-15.1) (from Ch. 38, par.
11-15.1)}\label{ilcs-511-15.1-from-ch.-38-par.-11-15.1}}
\addcontentsline{toc}{subsection}{(720 ILCS 5/11-15.1) (from Ch. 38,
par. 11-15.1)}

\hypertarget{sec.-11-15.1.-repealed.}{%
\section*{Sec. 11-15.1. (Repealed).}\label{sec.-11-15.1.-repealed.}}
\addcontentsline{toc}{section}{Sec. 11-15.1. (Repealed).}

\markright{Sec. 11-15.1. (Repealed).}

(Source: P.A. 97-227, eff. 1-1-12. Repealed by P.A. 96-1551, eff.
7-1-11.)

\hypertarget{ilcs-511-16-from-ch.-38-par.-11-16}{%
\subsection*{(720 ILCS 5/11-16) (from Ch. 38, par.
11-16)}\label{ilcs-511-16-from-ch.-38-par.-11-16}}
\addcontentsline{toc}{subsection}{(720 ILCS 5/11-16) (from Ch. 38, par.
11-16)}

\hypertarget{sec.-11-16.-repealed.}{%
\section*{Sec. 11-16. (Repealed).}\label{sec.-11-16.-repealed.}}
\addcontentsline{toc}{section}{Sec. 11-16. (Repealed).}

\markright{Sec. 11-16. (Repealed).}

(Source: P.A. 91-696, eff. 4-13-00. Repealed by P.A. 96-1551, eff.
7-1-11 .)

\hypertarget{ilcs-511-17-from-ch.-38-par.-11-17}{%
\subsection*{(720 ILCS 5/11-17) (from Ch. 38, par.
11-17)}\label{ilcs-511-17-from-ch.-38-par.-11-17}}
\addcontentsline{toc}{subsection}{(720 ILCS 5/11-17) (from Ch. 38, par.
11-17)}

\hypertarget{sec.-11-17.-repealed.}{%
\section*{Sec. 11-17. (Repealed).}\label{sec.-11-17.-repealed.}}
\addcontentsline{toc}{section}{Sec. 11-17. (Repealed).}

\markright{Sec. 11-17. (Repealed).}

(Source: P.A. 96-1464, eff. 8-20-10. Repealed by P.A. 96-1551, eff.
7-1-11 .)

\hypertarget{ilcs-511-17.1-from-ch.-38-par.-11-17.1}{%
\subsection*{(720 ILCS 5/11-17.1) (from Ch. 38, par.
11-17.1)}\label{ilcs-511-17.1-from-ch.-38-par.-11-17.1}}
\addcontentsline{toc}{subsection}{(720 ILCS 5/11-17.1) (from Ch. 38,
par. 11-17.1)}

\hypertarget{sec.-11-17.1.-repealed.}{%
\section*{Sec. 11-17.1. (Repealed).}\label{sec.-11-17.1.-repealed.}}
\addcontentsline{toc}{section}{Sec. 11-17.1. (Repealed).}

\markright{Sec. 11-17.1. (Repealed).}

(Source: P.A. 97-227, eff. 1-1-12. Repealed by P.A. 96-1551, eff.
7-1-11.)

\hypertarget{ilcs-511-18-from-ch.-38-par.-11-18}{%
\subsection*{(720 ILCS 5/11-18) (from Ch. 38, par.
11-18)}\label{ilcs-511-18-from-ch.-38-par.-11-18}}
\addcontentsline{toc}{subsection}{(720 ILCS 5/11-18) (from Ch. 38, par.
11-18)}

\hypertarget{sec.-11-18.-patronizing-a-prostitute.}{%
\section*{Sec. 11-18. Patronizing a
prostitute.}\label{sec.-11-18.-patronizing-a-prostitute.}}
\addcontentsline{toc}{section}{Sec. 11-18. Patronizing a prostitute.}

\markright{Sec. 11-18. Patronizing a prostitute.}

(a) Any person who knowingly performs any of the following acts with a
person not his or her spouse commits patronizing a prostitute:

(1) Engages in an act of sexual penetration as defined in Section 11-0.1
of this Code with a prostitute; or

(2) Enters or remains in a place of prostitution with intent to engage
in an act of sexual penetration as defined in Section 11-0.1 of this
Code; or

(3) Engages in any touching or fondling with a prostitute of the sex
organs of one person by the other person, with the intent to achieve
sexual arousal or gratification.

(b) Sentence.

Patronizing a prostitute is a Class 4 felony, unless committed within
1,000 feet of real property comprising a school, in which case it is a
Class 3 felony. A person convicted of a second or subsequent violation
of this Section, or of any combination of such number of convictions
under this Section and Sections 11-14 (prostitution), 11-14.1
(solicitation of a sexual act), 11-14.3 (promoting prostitution),
11-14.4 (promoting juvenile prostitution), 11-15 (soliciting for a
prostitute), 11-15.1 (soliciting for a juvenile prostitute), 11-16
(pandering), 11-17 (keeping a place of prostitution), 11-17.1 (keeping a
place of juvenile prostitution), 11-18.1 (patronizing a juvenile
prostitute), 11-19 (pimping), 11-19.1 (juvenile pimping or aggravated
juvenile pimping), or 11-19.2 (exploitation of a child) of this Code, is
guilty of a Class 3 felony. If the court imposes a fine under this
subsection (b), it shall be collected and distributed to the Specialized
Services for Survivors of Human Trafficking Fund in accordance with
Section 5-9-1.21 of the Unified Code of Corrections.

(c) (Blank).

(Source: P.A. 98-1013, eff. 1-1-15 .)

\hypertarget{ilcs-511-18.1-from-ch.-38-par.-11-18.1}{%
\subsection*{(720 ILCS 5/11-18.1) (from Ch. 38, par.
11-18.1)}\label{ilcs-511-18.1-from-ch.-38-par.-11-18.1}}
\addcontentsline{toc}{subsection}{(720 ILCS 5/11-18.1) (from Ch. 38,
par. 11-18.1)}

\hypertarget{sec.-11-18.1.-patronizing-a-minor-engaged-in-prostitution.}{%
\section*{Sec. 11-18.1. Patronizing a minor engaged in
prostitution.}\label{sec.-11-18.1.-patronizing-a-minor-engaged-in-prostitution.}}
\addcontentsline{toc}{section}{Sec. 11-18.1. Patronizing a minor engaged
in prostitution.}

\markright{Sec. 11-18.1. Patronizing a minor engaged in prostitution.}

(a) Any person who engages in an act of sexual penetration as defined in
Section 11-0.1 of this Code with a person engaged in prostitution who is
under 18 years of age or is a person with a severe or profound
intellectual disability commits patronizing a minor engaged in
prostitution.

(a-5) Any person who engages in any touching or fondling, with a person
engaged in prostitution who either is under 18 years of age or is a
person with a severe or profound intellectual disability, of the sex
organs of one person by the other person, with the intent to achieve
sexual arousal or gratification, commits patronizing a minor engaged in
prostitution.

(b) It is an affirmative defense to the charge of patronizing a minor
engaged in prostitution that the accused reasonably believed that the
person was of the age of 18 years or over or was not a person with a
severe or profound intellectual disability at the time of the act giving
rise to the charge.

(c) Sentence. A person who commits patronizing a juvenile prostitute is
guilty of a Class 3 felony, unless committed within 1,000 feet of real
property comprising a school, in which case it is a Class 2 felony. A
person convicted of a second or subsequent violation of this Section, or
of any combination of such number of convictions under this Section and
Sections 11-14 (prostitution), 11-14.1 (solicitation of a sexual act),
11-14.3 (promoting prostitution), 11-14.4 (promoting juvenile
prostitution), 11-15 (soliciting for a prostitute), 11-15.1 (soliciting
for a juvenile prostitute), 11-16 (pandering), 11-17 (keeping a place of
prostitution), 11-17.1 (keeping a place of juvenile prostitution), 11-18
(patronizing a prostitute), 11-19 (pimping), 11-19.1 (juvenile pimping
or aggravated juvenile pimping), or 11-19.2 (exploitation of a child) of
this Code, is guilty of a Class 2 felony. The fact of such conviction is
not an element of the offense and may not be disclosed to the jury
during trial unless otherwise permitted by issues properly raised during
such trial.

(Source: P.A. 99-143, eff. 7-27-15.)

\hypertarget{ilcs-511-19-from-ch.-38-par.-11-19}{%
\subsection*{(720 ILCS 5/11-19) (from Ch. 38, par.
11-19)}\label{ilcs-511-19-from-ch.-38-par.-11-19}}
\addcontentsline{toc}{subsection}{(720 ILCS 5/11-19) (from Ch. 38, par.
11-19)}

\hypertarget{sec.-11-19.-repealed.}{%
\section*{Sec. 11-19. (Repealed).}\label{sec.-11-19.-repealed.}}
\addcontentsline{toc}{section}{Sec. 11-19. (Repealed).}

\markright{Sec. 11-19. (Repealed).}

(Source: P.A. 96-1464, eff. 8-20-10. Repealed by P.A. 96-1551, eff.
7-1-11 .)

\hypertarget{ilcs-511-19.1-from-ch.-38-par.-11-19.1}{%
\subsection*{(720 ILCS 5/11-19.1) (from Ch. 38, par.
11-19.1)}\label{ilcs-511-19.1-from-ch.-38-par.-11-19.1}}
\addcontentsline{toc}{subsection}{(720 ILCS 5/11-19.1) (from Ch. 38,
par. 11-19.1)}

\hypertarget{sec.-11-19.1.-repealed.}{%
\section*{Sec. 11-19.1. (Repealed).}\label{sec.-11-19.1.-repealed.}}
\addcontentsline{toc}{section}{Sec. 11-19.1. (Repealed).}

\markright{Sec. 11-19.1. (Repealed).}

(Source: P.A. 97-227, eff. 1-1-12. Repealed by P.A. 96-1551, eff.
7-1-11.)

\hypertarget{ilcs-511-19.2-from-ch.-38-par.-11-19.2}{%
\subsection*{(720 ILCS 5/11-19.2) (from Ch. 38, par.
11-19.2)}\label{ilcs-511-19.2-from-ch.-38-par.-11-19.2}}
\addcontentsline{toc}{subsection}{(720 ILCS 5/11-19.2) (from Ch. 38,
par. 11-19.2)}

\hypertarget{sec.-11-19.2.-repealed.}{%
\section*{Sec. 11-19.2. (Repealed).}\label{sec.-11-19.2.-repealed.}}
\addcontentsline{toc}{section}{Sec. 11-19.2. (Repealed).}

\markright{Sec. 11-19.2. (Repealed).}

(Source: P.A. 97-227, eff. 1-1-12. Repealed by P.A. 96-1551, eff.
7-1-11.)

\hypertarget{ilcs-511-19.3}{%
\subsection*{(720 ILCS 5/11-19.3)}\label{ilcs-511-19.3}}
\addcontentsline{toc}{subsection}{(720 ILCS 5/11-19.3)}

\hypertarget{sec.-11-19.3.-repealed.}{%
\section*{Sec. 11-19.3. (Repealed).}\label{sec.-11-19.3.-repealed.}}
\addcontentsline{toc}{section}{Sec. 11-19.3. (Repealed).}

\markright{Sec. 11-19.3. (Repealed).}

(Source: P.A. 97-333, eff. 8-12-11. Repealed by P.A. 96-1551, eff.
7-1-11.)

\hypertarget{ilcs-5art.-11-subdiv.-20-heading}{%
\subsection*{(720 ILCS 5/Art. 11 Subdiv. 20
heading)}\label{ilcs-5art.-11-subdiv.-20-heading}}
\addcontentsline{toc}{subsection}{(720 ILCS 5/Art. 11 Subdiv. 20
heading)}

SUBDIVISION 20.

PORNOGRAPHY OFFENSES

(Source: P.A. 96-1551, eff. 7-1-11.)

\hypertarget{ilcs-511-20-from-ch.-38-par.-11-20}{%
\subsection*{(720 ILCS 5/11-20) (from Ch. 38, par.
11-20)}\label{ilcs-511-20-from-ch.-38-par.-11-20}}
\addcontentsline{toc}{subsection}{(720 ILCS 5/11-20) (from Ch. 38, par.
11-20)}

\hypertarget{sec.-11-20.-obscenity.}{%
\section*{Sec. 11-20. Obscenity.}\label{sec.-11-20.-obscenity.}}
\addcontentsline{toc}{section}{Sec. 11-20. Obscenity.}

\markright{Sec. 11-20. Obscenity.}

(a) Elements of the Offense. A person commits obscenity when, with
knowledge of the nature or content thereof, or recklessly failing to
exercise reasonable inspection which would have disclosed the nature or
content thereof, he or she:

(1) Sells, delivers or provides, or offers or agrees to sell, deliver or
provide any obscene writing, picture, record or other representation or
embodiment of the obscene; or

(2) Presents or directs an obscene play, dance or other performance or
participates directly in that portion thereof which makes it obscene; or

(3) Publishes, exhibits or otherwise makes available anything obscene;
or

(4) Performs an obscene act or otherwise presents an obscene exhibition
of his or her body for gain; or

(5) Creates, buys, procures or possesses obscene matter or material with
intent to disseminate it in violation of this Section, or of the penal
laws or regulations of any other jurisdiction; or

(6) Advertises or otherwise promotes the sale of material represented or
held out by him or her to be obscene, whether or not it is obscene.

(b) Obscene Defined.

Any material or performance is obscene if: (1) the average person,
applying contemporary adult community standards, would find that, taken
as a whole, it appeals to the prurient interest; and (2) the average
person, applying contemporary adult community standards, would find that
it depicts or describes, in a patently offensive way, ultimate sexual
acts or sadomasochistic sexual acts, whether normal or perverted, actual
or simulated, or masturbation, excretory functions or lewd exhibition of
the genitals; and (3) taken as a whole, it lacks serious literary,
artistic, political or scientific value.

(c) Interpretation of Evidence.

Obscenity shall be judged with reference to ordinary adults, except that
it shall be judged with reference to children or other specially
susceptible audiences if it appears from the character of the material
or the circumstances of its dissemination to be specially designed for
or directed to such an audience.

Where circumstances of production, presentation, sale, dissemination,
distribution, or publicity indicate that material is being commercially
exploited for the sake of its prurient appeal, such evidence is
probative with respect to the nature of the matter and can justify the
conclusion that the matter is lacking in serious literary, artistic,
political or scientific value.

In any prosecution for an offense under this Section evidence shall be
admissible to show:

(1) The character of the audience for which the material was designed or
to which it was directed;

(2) What the predominant appeal of the material would be for ordinary
adults or a special audience, and what effect, if any, it would probably
have on the behavior of such people;

(3) The artistic, literary, scientific, educational or other merits of
the material, or absence thereof;

(4) The degree, if any, of public acceptance of the material in this
State;

(5) Appeal to prurient interest, or absence thereof, in advertising or
other promotion of the material;

(6) Purpose of the author, creator, publisher or disseminator.

(d) Sentence.

Obscenity is a Class A misdemeanor. A second or subsequent offense is a
Class 4 felony.

(e) Permissive Inference.

The trier of fact may infer an intent to disseminate from the creation,
purchase, procurement or possession of a mold, engraved plate or other
embodiment of obscenity specially adapted for reproducing multiple
copies, or the possession of more than 3 copies of obscene material.

(f) Affirmative Defenses.

It shall be an affirmative defense to obscenity that the dissemination:

(1) Was not for gain and was made to personal associates other than
children under 18 years of age;

(2) Was to institutions or individuals having scientific or other
special justification for possession of such material.

(g) Forfeiture of property. A person who has been convicted previously
of the offense of obscenity and who is convicted of a second or
subsequent offense of obscenity is subject to the property forfeiture
provisions set forth in Article 124B of the Code of Criminal Procedure
of 1963.

(Source: P.A. 96-712, eff. 1-1-10; 96-1551, eff. 7-1-11 .)

\hypertarget{ilcs-511-20.1-from-ch.-38-par.-11-20.1}{%
\subsection*{(720 ILCS 5/11-20.1) (from Ch. 38, par.
11-20.1)}\label{ilcs-511-20.1-from-ch.-38-par.-11-20.1}}
\addcontentsline{toc}{subsection}{(720 ILCS 5/11-20.1) (from Ch. 38,
par. 11-20.1)}

\hypertarget{sec.-11-20.1.-child-pornography.}{%
\section*{Sec. 11-20.1. Child
pornography.}\label{sec.-11-20.1.-child-pornography.}}
\addcontentsline{toc}{section}{Sec. 11-20.1. Child pornography.}

\markright{Sec. 11-20.1. Child pornography.}

(a) A person commits child pornography who:

(1) films, videotapes, photographs, or otherwise depicts or portrays by
means of any similar visual medium or reproduction or depicts by
computer any child whom he or she knows or reasonably should know to be
under the age of 18 or any person with a severe or profound intellectual
disability where such child or person with a severe or profound
intellectual disability is:

(i) actually or by simulation engaged in any act of sexual penetration
or sexual conduct with any person or animal; or

(ii) actually or by simulation engaged in any act of sexual penetration
or sexual conduct involving the sex organs of the child or person with a
severe or profound intellectual disability and the mouth, anus, or sex
organs of another person or animal; or which involves the mouth, anus or
sex organs of the child or person with a severe or profound intellectual
disability and the sex organs of another person or animal; or

(iii) actually or by simulation engaged in any act of masturbation; or

(iv) actually or by simulation portrayed as being the object of, or
otherwise engaged in, any act of lewd fondling, touching, or caressing
involving another person or animal; or

(v) actually or by simulation engaged in any act of excretion or
urination within a sexual context; or

(vi) actually or by simulation portrayed or depicted as bound, fettered,
or subject to sadistic, masochistic, or sadomasochistic abuse in any
sexual context; or

(vii) depicted or portrayed in any pose, posture or setting involving a
lewd exhibition of the unclothed or transparently clothed genitals,
pubic area, buttocks, or, if such person is female, a fully or partially
developed breast of the child or other person; or

(2) with the knowledge of the nature or content thereof, reproduces,
disseminates, offers to disseminate, exhibits or possesses with intent
to disseminate any film, videotape, photograph or other similar visual
reproduction or depiction by computer of any child or person with a
severe or profound intellectual disability whom the person knows or
reasonably should know to be under the age of 18 or to be a person with
a severe or profound intellectual disability, engaged in any activity
described in subparagraphs (i) through (vii) of paragraph (1) of this
subsection; or

(3) with knowledge of the subject matter or theme thereof, produces any
stage play, live performance, film, videotape or other similar visual
portrayal or depiction by computer which includes a child whom the
person knows or reasonably should know to be under the age of 18 or a
person with a severe or profound intellectual disability engaged in any
activity described in subparagraphs (i) through (vii) of paragraph (1)
of this subsection; or

(4) solicits, uses, persuades, induces, entices, or coerces any child
whom he or she knows or reasonably should know to be under the age of 18
or a person with a severe or profound intellectual disability to appear
in any stage play, live presentation, film, videotape, photograph or
other similar visual reproduction or depiction by computer in which the
child or person with a severe or profound intellectual disability is or
will be depicted, actually or by simulation, in any act, pose or setting
described in subparagraphs (i) through (vii) of paragraph (1) of this
subsection; or

(5) is a parent, step-parent, legal guardian or other person having care
or custody of a child whom the person knows or reasonably should know to
be under the age of 18 or a person with a severe or profound
intellectual disability and who knowingly permits, induces, promotes, or
arranges for such child or person with a severe or profound intellectual
disability to appear in any stage play, live performance, film,
videotape, photograph or other similar visual presentation, portrayal or
simulation or depiction by computer of any act or activity described in
subparagraphs (i) through (vii) of paragraph (1) of this subsection; or

(6) with knowledge of the nature or content thereof, possesses any film,
videotape, photograph or other similar visual reproduction or depiction
by computer of any child or person with a severe or profound
intellectual disability whom the person knows or reasonably should know
to be under the age of 18 or to be a person with a severe or profound
intellectual disability, engaged in any activity described in
subparagraphs (i) through (vii) of paragraph (1) of this subsection; or

(7) solicits, or knowingly uses, persuades, induces, entices, or
coerces, a person to provide a child under the age of 18 or a person
with a severe or profound intellectual disability to appear in any
videotape, photograph, film, stage play, live presentation, or other
similar visual reproduction or depiction by computer in which the child
or person with a severe or profound intellectual disability will be
depicted, actually or by simulation, in any act, pose, or setting
described in subparagraphs (i) through (vii) of paragraph (1) of this
subsection.

(a-5) The possession of each individual film, videotape, photograph, or
other similar visual reproduction or depiction by computer in violation
of this Section constitutes a single and separate violation. This
subsection (a-5) does not apply to multiple copies of the same film,
videotape, photograph, or other similar visual reproduction or depiction
by computer that are identical to each other.

(b)(1) It shall be an affirmative defense to a charge of child
pornography that the defendant reasonably believed, under all of the
circumstances, that the child was 18 years of age or older or that the
person was not a person with a severe or profound intellectual
disability but only where, prior to the act or acts giving rise to a
prosecution under this Section, he or she took some affirmative action
or made a bonafide inquiry designed to ascertain whether the child was
18 years of age or older or that the person was not a person with a
severe or profound intellectual disability and his or her reliance upon
the information so obtained was clearly reasonable.

(1.5) Telecommunications carriers, commercial mobile service providers,
and providers of information services, including, but not limited to,
Internet service providers and hosting service providers, are not liable
under this Section by virtue of the transmission, storage, or caching of
electronic communications or messages of others or by virtue of the
provision of other related telecommunications, commercial mobile
services, or information services used by others in violation of this
Section.

(2) (Blank).

(3) The charge of child pornography shall not apply to the performance
of official duties by law enforcement or prosecuting officers or persons
employed by law enforcement or prosecuting agencies, court personnel or
attorneys, nor to bonafide treatment or professional education programs
conducted by licensed physicians, psychologists or social workers. In
any criminal proceeding, any property or material that constitutes child
pornography shall remain in the care, custody, and control of either the
State or the court. A motion to view the evidence shall comply with
subsection (e-5) of this Section.

(4) If the defendant possessed more than one of the same film, videotape
or visual reproduction or depiction by computer in which child
pornography is depicted, then the trier of fact may infer that the
defendant possessed such materials with the intent to disseminate them.

(5) The charge of child pornography does not apply to a person who does
not voluntarily possess a film, videotape, or visual reproduction or
depiction by computer in which child pornography is depicted. Possession
is voluntary if the defendant knowingly procures or receives a film,
videotape, or visual reproduction or depiction for a sufficient time to
be able to terminate his or her possession.

(6) Any violation of paragraph (1), (2), (3), (4), (5), or (7) of
subsection (a) that includes a child engaged in, solicited for, depicted
in, or posed in any act of sexual penetration or bound, fettered, or
subject to sadistic, masochistic, or sadomasochistic abuse in a sexual
context shall be deemed a crime of violence.

(c) If the violation does not involve a film, videotape, or other moving
depiction, a violation of paragraph (1), (4), (5), or (7) of subsection
(a) is a Class 1 felony with a mandatory minimum fine of \$2,000 and a
maximum fine of \$100,000. If the violation involves a film, videotape,
or other moving depiction, a violation of paragraph (1), (4), (5), or
(7) of subsection (a) is a Class X felony with a mandatory minimum fine
of \$2,000 and a maximum fine of \$100,000. If the violation does not
involve a film, videotape, or other moving depiction, a violation of
paragraph (3) of subsection (a) is a Class 1 felony with a mandatory
minimum fine of \$1500 and a maximum fine of \$100,000. If the violation
involves a film, videotape, or other moving depiction, a violation of
paragraph (3) of subsection (a) is a Class X felony with a mandatory
minimum fine of \$1500 and a maximum fine of \$100,000. If the violation
does not involve a film, videotape, or other moving depiction, a
violation of paragraph (2) of subsection (a) is a Class 1 felony with a
mandatory minimum fine of \$1000 and a maximum fine of \$100,000. If the
violation involves a film, videotape, or other moving depiction, a
violation of paragraph (2) of subsection (a) is a Class X felony with a
mandatory minimum fine of \$1000 and a maximum fine of \$100,000. If the
violation does not involve a film, videotape, or other moving depiction,
a violation of paragraph (6) of subsection (a) is a Class 3 felony with
a mandatory minimum fine of \$1000 and a maximum fine of \$100,000. If
the violation involves a film, videotape, or other moving depiction, a
violation of paragraph (6) of subsection (a) is a Class 2 felony with a
mandatory minimum fine of \$1000 and a maximum fine of \$100,000.

(c-5) Where the child depicted is under the age of 13, a violation of
paragraph (1), (2), (3), (4), (5), or (7) of subsection (a) is a Class X
felony with a mandatory minimum fine of \$2,000 and a maximum fine of
\$100,000. Where the child depicted is under the age of 13, a violation
of paragraph (6) of subsection (a) is a Class 2 felony with a mandatory
minimum fine of \$1,000 and a maximum fine of \$100,000. Where the child
depicted is under the age of 13, a person who commits a violation of
paragraph (1), (2), (3), (4), (5), or (7) of subsection (a) where the
defendant has previously been convicted under the laws of this State or
any other state of the offense of child pornography, aggravated child
pornography, aggravated criminal sexual abuse, aggravated criminal
sexual assault, predatory criminal sexual assault of a child, or any of
the offenses formerly known as rape, deviate sexual assault, indecent
liberties with a child, or aggravated indecent liberties with a child
where the victim was under the age of 18 years or an offense that is
substantially equivalent to those offenses, is guilty of a Class X
felony for which the person shall be sentenced to a term of imprisonment
of not less than 9 years with a mandatory minimum fine of \$2,000 and a
maximum fine of \$100,000. Where the child depicted is under the age of
13, a person who commits a violation of paragraph (6) of subsection (a)
where the defendant has previously been convicted under the laws of this
State or any other state of the offense of child pornography, aggravated
child pornography, aggravated criminal sexual abuse, aggravated criminal
sexual assault, predatory criminal sexual assault of a child, or any of
the offenses formerly known as rape, deviate sexual assault, indecent
liberties with a child, or aggravated indecent liberties with a child
where the victim was under the age of 18 years or an offense that is
substantially equivalent to those offenses, is guilty of a Class 1
felony with a mandatory minimum fine of \$1,000 and a maximum fine of
\$100,000. The issue of whether the child depicted is under the age of
13 is an element of the offense to be resolved by the trier of fact.

(d) If a person is convicted of a second or subsequent violation of this
Section within 10 years of a prior conviction, the court shall order a
presentence psychiatric examination of the person. The examiner shall
report to the court whether treatment of the person is necessary.

(e) Any film, videotape, photograph or other similar visual reproduction
or depiction by computer which includes a child under the age of 18 or a
person with a severe or profound intellectual disability engaged in any
activity described in subparagraphs (i) through (vii) or paragraph 1 of
subsection (a), and any material or equipment used or intended for use
in photographing, filming, printing, producing, reproducing,
manufacturing, projecting, exhibiting, depiction by computer, or
disseminating such material shall be seized and forfeited in the manner,
method and procedure provided by Section 36-1 of this Code for the
seizure and forfeiture of vessels, vehicles and aircraft.

In addition, any person convicted under this Section is subject to the
property forfeiture provisions set forth in Article 124B of the Code of
Criminal Procedure of 1963.

(e-5) Upon the conclusion of a case brought under this Section, the
court shall seal all evidence depicting a victim or witness that is
sexually explicit. The evidence may be unsealed and viewed, on a motion
of the party seeking to unseal and view the evidence, only for good
cause shown and in the discretion of the court. The motion must
expressly set forth the purpose for viewing the material. The State's
attorney and the victim, if possible, shall be provided reasonable
notice of the hearing on the motion to unseal the evidence. Any person
entitled to notice of a hearing under this subsection (e-5) may object
to the motion.

(f) Definitions. For the purposes of this Section:

(1) ``Disseminate'' means (i) to sell, distribute, exchange or transfer
possession, whether with or without consideration or (ii) to make a
depiction by computer available for distribution or downloading through
the facilities of any telecommunications network or through any other
means of transferring computer programs or data to a computer.

(2) ``Produce'' means to direct, promote, advertise, publish,
manufacture, issue, present or show.

(3) ``Reproduce'' means to make a duplication or copy.

(4) ``Depict by computer'' means to generate or create, or cause to be
created or generated, a computer program or data that, after being
processed by a computer either alone or in conjunction with one or more
computer programs, results in a visual depiction on a computer monitor,
screen, or display.

(5) ``Depiction by computer'' means a computer program or data that,
after being processed by a computer either alone or in conjunction with
one or more computer programs, results in a visual depiction on a
computer monitor, screen, or display.

(6) ``Computer'', ``computer program'', and ``data'' have the meanings
ascribed to them in Section 17.05 of this Code.

(7) For the purposes of this Section, ``child pornography'' includes a
film, videotape, photograph, or other similar visual medium or
reproduction or depiction by computer that is, or appears to be, that of
a person, either in part, or in total, under the age of 18 or a person
with a severe or profound intellectual disability, regardless of the
method by which the film, videotape, photograph, or other similar visual
medium or reproduction or depiction by computer is created, adopted, or
modified to appear as such. ``Child pornography'' also includes a film,
videotape, photograph, or other similar visual medium or reproduction or
depiction by computer that is advertised, promoted, presented,
described, or distributed in such a manner that conveys the impression
that the film, videotape, photograph, or other similar visual medium or
reproduction or depiction by computer is of a person under the age of 18
or a person with a severe or profound intellectual disability.

(g) Re-enactment; findings; purposes.

(1) The General Assembly finds and declares that:

(i) Section 50-5 of Public Act 88-680, effective

January 1, 1995, contained provisions amending the child pornography
statute, Section 11-20.1 of the Criminal Code of 1961. Section 50-5 also
contained other provisions.

(ii) In addition, Public Act 88-680 was entitled

``AN ACT to create a Safe Neighborhoods Law''. (A) Article 5 was
entitled JUVENILE JUSTICE and amended the Juvenile Court Act of 1987.
(B) Article 15 was entitled GANGS and amended various provisions of the
Criminal Code of 1961 and the Unified Code of Corrections. (C) Article
20 was entitled ALCOHOL ABUSE and amended various provisions of the
Illinois Vehicle Code. (D) Article 25 was entitled DRUG ABUSE and
amended the Cannabis Control Act and the Illinois Controlled Substances
Act. (E) Article 30 was entitled FIREARMS and amended the Criminal Code
of 1961 and the Code of Criminal Procedure of 1963. (F) Article 35
amended the Criminal Code of 1961, the Rights of Crime Victims and
Witnesses Act, and the Unified Code of Corrections. (G) Article 40
amended the Criminal Code of 1961 to increase the penalty for compelling
organization membership of persons. (H) Article 45 created the Secure
Residential Youth Care Facility Licensing Act and amended the State
Finance Act, the Juvenile Court Act of 1987, the Unified Code of
Corrections, and the Private Correctional Facility Moratorium Act. (I)
Article 50 amended the WIC Vendor Management Act, the Firearm Owners
Identification Card Act, the Juvenile Court Act of 1987, the Criminal
Code of 1961, the Wrongs to Children Act, and the Unified Code of
Corrections.

(iii) On September 22, 1998, the Third District

Appellate Court in People v. Dainty, 701 N.E. 2d 118, ruled that Public
Act 88-680 violates the single subject clause of the Illinois
Constitution (Article IV, Section 8 (d)) and was unconstitutional in its
entirety. As of the time this amendatory Act of 1999 was prepared,
People v. Dainty was still subject to appeal.

(iv) Child pornography is a vital concern to the people of this State
and the validity of future prosecutions under the child pornography
statute of the Criminal Code of 1961 is in grave doubt.

(2) It is the purpose of this amendatory Act of 1999 to prevent or
minimize any problems relating to prosecutions for child pornography
that may result from challenges to the constitutional validity of Public
Act 88-680 by re-enacting the Section relating to child pornography that
was included in Public Act 88-680.

(3) This amendatory Act of 1999 re-enacts Section

11-20.1 of the Criminal Code of 1961, as it has been amended. This
re-enactment is intended to remove any question as to the validity or
content of that Section; it is not intended to supersede any other
Public Act that amends the text of the Section as set forth in this
amendatory Act of 1999. The material is shown as existing text (i.e.,
without underscoring) because, as of the time this amendatory Act of
1999 was prepared, People v. Dainty was subject to appeal to the
Illinois Supreme Court.

(4) The re-enactment by this amendatory Act of 1999 of Section 11-20.1
of the Criminal Code of 1961 relating to child pornography that was
amended by Public Act 88-680 is not intended, and shall not be
construed, to imply that Public Act 88-680 is invalid or to limit or
impair any legal argument concerning whether those provisions were
substantially re-enacted by other Public Acts.

(Source:

P.A. 101-87, eff. 1-1-20; 102-567, eff. 1-1-22 .)

\hypertarget{ilcs-511-20.1a}{%
\subsection*{(720 ILCS 5/11-20.1A)}\label{ilcs-511-20.1a}}
\addcontentsline{toc}{subsection}{(720 ILCS 5/11-20.1A)}

\hypertarget{sec.-11-20.1a.-repealed.}{%
\section*{Sec. 11-20.1A. (Repealed).}\label{sec.-11-20.1a.-repealed.}}
\addcontentsline{toc}{section}{Sec. 11-20.1A. (Repealed).}

\markright{Sec. 11-20.1A. (Repealed).}

(Source: P.A. 95-579, eff. 6-1-08. Repealed by P.A. 96-712, eff.
1-1-10.)

\hypertarget{ilcs-511-20.1b}{%
\subsection*{(720 ILCS 5/11-20.1B)}\label{ilcs-511-20.1b}}
\addcontentsline{toc}{subsection}{(720 ILCS 5/11-20.1B)}

\hypertarget{sec.-11-20.1b.-repealed.}{%
\section*{Sec. 11-20.1B. (Repealed).}\label{sec.-11-20.1b.-repealed.}}
\addcontentsline{toc}{section}{Sec. 11-20.1B. (Repealed).}

\markright{Sec. 11-20.1B. (Repealed).}

(Source: P.A. 97-1109, eff. 1-1-13. Repealed by P.A. 97-995, eff.
1-1-13.)

\hypertarget{ilcs-511-20.2-from-ch.-38-par.-11-20.2}{%
\subsection*{(720 ILCS 5/11-20.2) (from Ch. 38, par.
11-20.2)}\label{ilcs-511-20.2-from-ch.-38-par.-11-20.2}}
\addcontentsline{toc}{subsection}{(720 ILCS 5/11-20.2) (from Ch. 38,
par. 11-20.2)}

\hypertarget{sec.-11-20.2.-duty-of-commercial-film-and-photographic-print-processors-or-computer-technicians-to-report-sexual-depiction-of-children.}{%
\section*{Sec. 11-20.2. Duty of commercial film and photographic print
processors or computer technicians to report sexual depiction of
children.}\label{sec.-11-20.2.-duty-of-commercial-film-and-photographic-print-processors-or-computer-technicians-to-report-sexual-depiction-of-children.}}
\addcontentsline{toc}{section}{Sec. 11-20.2. Duty of commercial film and
photographic print processors or computer technicians to report sexual
depiction of children.}

\markright{Sec. 11-20.2. Duty of commercial film and photographic print
processors or computer technicians to report sexual depiction of
children.}

(a) Any commercial film and photographic print processor or computer
technician who has knowledge of or observes, within the scope of his
professional capacity or employment, any film, photograph, videotape,
negative, slide, computer hard drive or any other magnetic or optical
media which depicts a child whom the processor or computer technician
knows or reasonably should know to be under the age of 18 where such
child is:

(i) actually or by simulation engaged in any act of sexual penetration
or sexual conduct with any person or animal; or

(ii) actually or by simulation engaged in any act of sexual penetration
or sexual conduct involving the sex organs of the child and the mouth,
anus, or sex organs of another person or animal; or which involves the
mouth, anus or sex organs of the child and the sex organs of another
person or animal; or

(iii) actually or by simulation engaged in any act of masturbation; or

(iv) actually or by simulation portrayed as being the object of, or
otherwise engaged in, any act of lewd fondling, touching, or caressing
involving another person or animal; or

(v) actually or by simulation engaged in any act of excretion or
urination within a sexual context; or

(vi) actually or by simulation portrayed or depicted as bound, fettered,
or subject to sadistic, masochistic, or sadomasochistic abuse in any
sexual context; or

(vii) depicted or portrayed in any pose, posture or setting involving a
lewd exhibition of the unclothed or transparently clothed genitals,
pubic area, buttocks, or, if such person is female, a fully or partially
developed breast of the child or other person;

shall report or cause a report to be made pursuant to subsections (b)
and (c) as soon as reasonably possible. Failure to make such report
shall be a business offense with a fine of \$1,000.

(b) Commercial film and photographic film processors shall report or
cause a report to be made to the local law enforcement agency of the
jurisdiction in which the image or images described in subsection (a)
are discovered.

(c) Computer technicians shall report or cause the report to be made to
the local law enforcement agency of the jurisdiction in which the image
or images described in subsection (a) are discovered or to the Illinois
Child Exploitation e-Tipline at reportchildporn@atg.state.il.us.

(d) Reports required by this Act shall include the following
information: (i) name, address, and telephone number of the person
filing the report; (ii) the employer of the person filing the report, if
any; (iii) the name, address and telephone number of the person whose
property is the subject of the report, if known; (iv) the circumstances
which led to the filing of the report, including a description of the
reported content.

(e) If a report is filed with the Cyber Tipline at the National Center
for Missing and Exploited Children or in accordance with the
requirements of 42 U.S.C. 13032, the requirements of this Act will be
deemed to have been met.

(f) A computer technician or an employer caused to report child
pornography under this Section is immune from any criminal, civil, or
administrative liability in connection with making the report, except
for willful or wanton misconduct.

(g) For the purposes of this Section, a ``computer technician'' is a
person who installs, maintains, troubleshoots, repairs or upgrades
computer hardware, software, computer networks, peripheral equipment,
electronic mail systems, or provides user assistance for any of the
aforementioned tasks.

(Source: P.A. 95-983, eff. 6-1-09; 96-1551, eff. 7-1-11 .)

\hypertarget{ilcs-511-20.3}{%
\subsection*{(720 ILCS 5/11-20.3)}\label{ilcs-511-20.3}}
\addcontentsline{toc}{subsection}{(720 ILCS 5/11-20.3)}

(This Section was renumbered as Section 11-20.1B by P.A. 96-1551.)

\hypertarget{sec.-11-20.3.-renumbered.}{%
\section*{Sec. 11-20.3. (Renumbered).}\label{sec.-11-20.3.-renumbered.}}
\addcontentsline{toc}{section}{Sec. 11-20.3. (Renumbered).}

\markright{Sec. 11-20.3. (Renumbered).}

(Source: P.A. 97-227, eff. 1-1-12. Renumbered by P.A. 96-1551, eff.
7-1-11.)

\hypertarget{ilcs-511-21-from-ch.-38-par.-11-21}{%
\subsection*{(720 ILCS 5/11-21) (from Ch. 38, par.
11-21)}\label{ilcs-511-21-from-ch.-38-par.-11-21}}
\addcontentsline{toc}{subsection}{(720 ILCS 5/11-21) (from Ch. 38, par.
11-21)}

\hypertarget{sec.-11-21.-harmful-material.}{%
\section*{Sec. 11-21. Harmful
material.}\label{sec.-11-21.-harmful-material.}}
\addcontentsline{toc}{section}{Sec. 11-21. Harmful material.}

\markright{Sec. 11-21. Harmful material.}

(a) As used in this Section:

``Distribute'' means to transfer possession of, whether with or without
consideration.

``Harmful to minors'' means that quality of any description or
representation, in whatever form, of nudity, sexual conduct, sexual
excitement, or sado-masochistic abuse, when, taken as a whole, it (i)
predominately appeals to the prurient interest in sex of minors, (ii) is
patently offensive to prevailing standards in the adult community in the
State as a whole with respect to what is suitable material for minors,
and (iii) lacks serious literary, artistic, political, or scientific
value for minors.

``Knowingly'' means having knowledge of the contents of the subject
matter, or recklessly failing to exercise reasonable inspection which
would have disclosed the contents.

``Material'' means (i) any picture, photograph, drawing, sculpture,
film, video game, computer game, video or similar visual depiction,
including any such representation or image which is stored
electronically, or (ii) any book, magazine, printed matter however
reproduced, or recorded audio of any sort.

``Minor'' means any person under the age of 18.

``Nudity'' means the showing of the human male or female genitals, pubic
area or buttocks with less than a fully opaque covering, or the showing
of the female breast with less than a fully opaque covering of any
portion below the top of the nipple, or the depiction of covered male
genitals in a discernibly turgid state.

``Sado-masochistic abuse'' means flagellation or torture by or upon a
person clad in undergarments, a mask or bizarre costume, or the
condition of being fettered, bound or otherwise physically restrained on
the part of one clothed for sexual gratification or stimulation.

``Sexual conduct'' means acts of masturbation, sexual intercourse, or
physical contact with a person's clothed or unclothed genitals, pubic
area, buttocks or, if such person be a female, breast.

``Sexual excitement'' means the condition of human male or female
genitals when in a state of sexual stimulation or arousal.

(b) A person is guilty of distributing harmful material to a minor when
he or she:

(1) knowingly sells, lends, distributes, exhibits to, depicts to, or
gives away to a minor, knowing that the minor is under the age of 18 or
failing to exercise reasonable care in ascertaining the person's true
age:

(A) any material which depicts nudity, sexual conduct or
sado-masochistic abuse, or which contains explicit and detailed verbal
descriptions or narrative accounts of sexual excitement, sexual conduct
or sado-masochistic abuse, and which taken as a whole is harmful to
minors;

(B) a motion picture, show, or other presentation which depicts nudity,
sexual conduct or sado-masochistic abuse and is harmful to minors; or

(C) an admission ticket or pass to premises where there is exhibited or
to be exhibited such a motion picture, show, or other presentation; or

(2) admits a minor to premises where there is exhibited or to be
exhibited such a motion picture, show, or other presentation, knowing
that the minor is a person under the age of 18 or failing to exercise
reasonable care in ascertaining the person's true age.

(c) In any prosecution arising under this Section, it is an affirmative
defense:

(1) that the minor as to whom the offense is alleged to have been
committed exhibited to the accused a draft card, driver's license, birth
certificate or other official or apparently official document purporting
to establish that the minor was 18 years of age or older, which was
relied upon by the accused;

(2) that the defendant was in a parental or guardianship relationship
with the minor or that the minor was accompanied by a parent or legal
guardian;

(3) that the defendant was a bona fide school, museum, or public
library, or was a person acting in the course of his or her employment
as an employee or official of such organization or retail outlet
affiliated with and serving the educational purpose of such
organization;

(4) that the act charged was committed in aid of legitimate scientific
or educational purposes; or

(5) that an advertisement of harmful material as defined in this Section
culminated in the sale or distribution of such harmful material to a
child under circumstances where there was no personal confrontation of
the child by the defendant, his or her employees, or agents, as where
the order or request for such harmful material was transmitted by mail,
telephone, Internet or similar means of communication, and delivery of
such harmful material to the child was by mail, freight, Internet or
similar means of transport, which advertisement contained the following
statement, or a substantially similar statement, and that the defendant
required the purchaser to certify that he or she was not under the age
of 18 and that the purchaser falsely stated that he or she was not under
the age of 18: ``NOTICE: It is unlawful for any person under the age of
18 to purchase the matter advertised. Any person under the age of 18
that falsely states that he or she is not under the age of 18 for the
purpose of obtaining the material advertised is guilty of a Class B
misdemeanor under the laws of the State.''

(d) The predominant appeal to prurient interest of the material shall be
judged with reference to average children of the same general age of the
child to whom such material was sold, lent, distributed or given, unless
it appears from the nature of the matter or the circumstances of its
dissemination or distribution that it is designed for specially
susceptible groups, in which case the predominant appeal of the material
shall be judged with reference to its intended or probable recipient
group.

(e) Distribution of harmful material in violation of this Section is a
Class A misdemeanor. A second or subsequent offense is a Class 4 felony.

(f) Any person under the age of 18 who falsely states, either orally or
in writing, that he or she is not under the age of 18, or who presents
or offers to any person any evidence of age and identity that is false
or not actually his or her own with the intent of ordering, obtaining,
viewing, or otherwise procuring or attempting to procure or view any
harmful material is guilty of a Class B misdemeanor.

(g) A person over the age of 18 who fails to exercise reasonable care in
ascertaining the true age of a minor, knowingly distributes to, or
sends, or causes to be sent, or exhibits to, or offers to distribute, or
exhibits any harmful material to a person that he or she believes is a
minor is guilty of a Class A misdemeanor. If that person utilized a
computer web camera, cellular telephone, or any other type of device to
manufacture the harmful material, then each offense is a Class 4 felony.

(h) Telecommunications carriers, commercial mobile service providers,
and providers of information services, including, but not limited to,
Internet service providers and hosting service providers, are not liable
under this Section, except for willful and wanton misconduct, by virtue
of the transmission, storage, or caching of electronic communications or
messages of others or by virtue of the provision of other related
telecommunications, commercial mobile services, or information services
used by others in violation of this Section.

(Source: P.A. 99-642, eff. 7-28-16.)

\hypertarget{ilcs-511-22-from-ch.-38-par.-11-22}{%
\subsection*{(720 ILCS 5/11-22) (from Ch. 38, par.
11-22)}\label{ilcs-511-22-from-ch.-38-par.-11-22}}
\addcontentsline{toc}{subsection}{(720 ILCS 5/11-22) (from Ch. 38, par.
11-22)}

\hypertarget{sec.-11-22.-tie-in-sales-of-obscene-publications-to-distributors.}{%
\section*{Sec. 11-22. Tie-in sales of obscene publications to
distributors.}\label{sec.-11-22.-tie-in-sales-of-obscene-publications-to-distributors.}}
\addcontentsline{toc}{section}{Sec. 11-22. Tie-in sales of obscene
publications to distributors.}

\markright{Sec. 11-22. Tie-in sales of obscene publications to
distributors.}

Any person, firm or corporation, or any agent, officer or employee
thereof, engaged in the business of distributing books, magazines,
periodicals, comic books or other publications to retail dealers, who
shall refuse to furnish to any retail dealer such quantity of books,
magazines, periodicals, comic books or other publications as such retail
dealer normally sells because the retail dealer refuses to sell, or
offer for sale, any books, magazines, periodicals, comic books or other
publications which are obscene, lewd, lascivious, filthy or indecent is
guilty of a petty offense. Each publication sold or delivered in
violation of this Act shall constitute a separate petty offense.

(Source: P.A. 77-2638.)

\hypertarget{ilcs-511-23}{%
\subsection*{(720 ILCS 5/11-23)}\label{ilcs-511-23}}
\addcontentsline{toc}{subsection}{(720 ILCS 5/11-23)}

\hypertarget{sec.-11-23.-posting-of-identifying-or-graphic-information-on-a-pornographic-internet-site-or-possessing-graphic-information-with-pornographic-material.}{%
\section*{Sec. 11-23. Posting of identifying or graphic information on a
pornographic Internet site or possessing graphic information with
pornographic
material.}\label{sec.-11-23.-posting-of-identifying-or-graphic-information-on-a-pornographic-internet-site-or-possessing-graphic-information-with-pornographic-material.}}
\addcontentsline{toc}{section}{Sec. 11-23. Posting of identifying or
graphic information on a pornographic Internet site or possessing
graphic information with pornographic material.}

\markright{Sec. 11-23. Posting of identifying or graphic information on
a pornographic Internet site or possessing graphic information with
pornographic material.}

(a) A person at least 17 years of age who knowingly discloses on an
adult obscenity or child pornography Internet site the name, address,
telephone number, or e-mail address of a person under 17 years of age at
the time of the commission of the offense or of a person at least 17
years of age without the consent of the person at least 17 years of age
is guilty of posting of identifying information on a pornographic
Internet site.

(a-5) Any person who knowingly places, posts, reproduces, or maintains
on an adult obscenity or child pornography Internet site a photograph,
video, or digital image of a person under 18 years of age that is not
child pornography under Section 11-20.1, without the knowledge and
consent of the person under 18 years of age, is guilty of posting of
graphic information on a pornographic Internet site. This provision
applies even if the person under 18 years of age is fully or properly
clothed in the photograph, video, or digital image.

(a-10) Any person who knowingly places, posts, reproduces, or maintains
on an adult obscenity or child pornography Internet site, or possesses
with obscene or child pornographic material a photograph, video, or
digital image of a person under 18 years of age in which the child is
posed in a suggestive manner with the focus or concentration of the
image on the child's clothed genitals, clothed pubic area, clothed
buttocks area, or if the child is female, the breast exposed through
transparent clothing, and the photograph, video, or digital image is not
child pornography under Section 11-20.1, is guilty of posting of graphic
information on a pornographic Internet site or possessing graphic
information with pornographic material.

(b) Sentence. A person who violates subsection (a) of this Section is
guilty of a Class 4 felony if the victim is at least 17 years of age at
the time of the offense and a Class 3 felony if the victim is under 17
years of age at the time of the offense. A person who violates
subsection (a-5) of this Section is guilty of a Class 4 felony. A person
who violates subsection (a-10) of this Section is guilty of a Class 3
felony.

(c) Definitions. For purposes of this Section:

(1) ``Adult obscenity or child pornography Internet site'' means a site
on the Internet that contains material that is obscene as defined in
Section 11-20 of this Code or that is child pornography as defined in
Section 11-20.1 of this Code.

(2) ``Internet'' has the meaning set forth in Section

16-0.1 of this Code.

(Source: P.A. 96-1551, eff. 7-1-11; 97-1150, eff. 1-25-13.)

\hypertarget{ilcs-511-23.5}{%
\subsection*{(720 ILCS 5/11-23.5)}\label{ilcs-511-23.5}}
\addcontentsline{toc}{subsection}{(720 ILCS 5/11-23.5)}

\hypertarget{sec.-11-23.5.-non-consensual-dissemination-of-private-sexual-images.}{%
\section*{Sec. 11-23.5. Non-consensual dissemination of private sexual
images.}\label{sec.-11-23.5.-non-consensual-dissemination-of-private-sexual-images.}}
\addcontentsline{toc}{section}{Sec. 11-23.5. Non-consensual
dissemination of private sexual images.}

\markright{Sec. 11-23.5. Non-consensual dissemination of private sexual
images.}

(a) Definitions. For the purposes of this Section:

``Computer'', ``computer program'', and ``data'' have the meanings
ascribed to them in Section 17-0.5 of this Code.

``Image'' includes a photograph, film, videotape, digital recording, or
other depiction or portrayal of an object, including a human body.

``Intimate parts'' means the fully unclothed, partially unclothed or
transparently clothed genitals, pubic area, anus, or if the person is
female, a partially or fully exposed nipple, including exposure through
transparent clothing.

``Sexual act'' means sexual penetration, masturbation, or sexual
activity.

``Sexual activity'' means any:

(1) knowing touching or fondling by the victim or another person or
animal, either directly or through clothing, of the sex organs, anus, or
breast of the victim or another person or animal for the purpose of
sexual gratification or arousal; or

(2) any transfer or transmission of semen upon any part of the clothed
or unclothed body of the victim, for the purpose of sexual gratification
or arousal of the victim or another; or

(3) an act of urination within a sexual context; or

(4) any bondage, fetter, or sadism masochism; or

(5) sadomasochism abuse in any sexual context.

(b) A person commits non-consensual dissemination of private sexual
images when he or she:

(1) intentionally disseminates an image of another person:

(A) who is at least 18 years of age; and

(B) who is identifiable from the image itself or information displayed
in connection with the image; and

(C) who is engaged in a sexual act or whose intimate parts are exposed,
in whole or in part; and

(2) obtains the image under circumstances in which a reasonable person
would know or understand that the image was to remain private; and

(3) knows or should have known that the person in the image has not
consented to the dissemination.

(c) The following activities are exempt from the provisions of this
Section:

(1) The intentional dissemination of an image of another identifiable
person who is engaged in a sexual act or whose intimate parts are
exposed when the dissemination is made for the purpose of a criminal
investigation that is otherwise lawful.

(2) The intentional dissemination of an image of another identifiable
person who is engaged in a sexual act or whose intimate parts are
exposed when the dissemination is for the purpose of, or in connection
with, the reporting of unlawful conduct.

(3) The intentional dissemination of an image of another identifiable
person who is engaged in a sexual act or whose intimate parts are
exposed when the images involve voluntary exposure in public or
commercial settings.

(4) The intentional dissemination of an image of another identifiable
person who is engaged in a sexual act or whose intimate parts are
exposed when the dissemination serves a lawful public purpose.

(d) Nothing in this Section shall be construed to impose liability upon
the following entities solely as a result of content or information
provided by another person:

(1) an interactive computer service, as defined in 47

U.S.C. 230(f)(2);

(2) a provider of public mobile services or private radio services, as
defined in Section 13-214 of the Public Utilities Act; or

(3) a telecommunications network or broadband provider.

(e) A person convicted under this Section is subject to the forfeiture
provisions in Article 124B of the Code of Criminal Procedure of 1963.

(f) Sentence. Non-consensual dissemination of private sexual images is a
Class 4 felony.

(Source: P.A. 98-1138, eff. 6-1-15 .)

\hypertarget{ilcs-511-24}{%
\subsection*{(720 ILCS 5/11-24)}\label{ilcs-511-24}}
\addcontentsline{toc}{subsection}{(720 ILCS 5/11-24)}

\hypertarget{sec.-11-24.-child-photography-by-sex-offender.}{%
\section*{Sec. 11-24. Child photography by sex
offender.}\label{sec.-11-24.-child-photography-by-sex-offender.}}
\addcontentsline{toc}{section}{Sec. 11-24. Child photography by sex
offender.}

\markright{Sec. 11-24. Child photography by sex offender.}

(a) In this Section:

``Child'' means a person under 18 years of age.

``Child sex offender'' has the meaning ascribed to it in Section 11-0.1
of this Code.

(b) It is unlawful for a child sex offender to knowingly:

(1) conduct or operate any type of business in which he or she
photographs, videotapes, or takes a digital image of a child; or

(2) conduct or operate any type of business in which he or she instructs
or directs another person to photograph, videotape, or take a digital
image of a child; or

(3) photograph, videotape, or take a digital image of a child, or
instruct or direct another person to photograph, videotape, or take a
digital image of a child without the consent of the parent or guardian.

(c) Sentence. A violation of this Section is a Class 2 felony. A person
who violates this Section at a playground, park facility, school, forest
preserve, day care facility, or at a facility providing programs or
services directed to persons under 17 years of age is guilty of a Class
1 felony.

(Source: P.A. 95-983, eff. 6-1-09; 96-1551, eff. 7-1-11 .)

\hypertarget{ilcs-5art.-11-subdiv.-25-heading}{%
\subsection*{(720 ILCS 5/Art. 11 Subdiv. 25
heading)}\label{ilcs-5art.-11-subdiv.-25-heading}}
\addcontentsline{toc}{subsection}{(720 ILCS 5/Art. 11 Subdiv. 25
heading)}

SUBDIVISION 25.

OTHER OFFENSES

(Source: P.A. 96-1551, eff. 7-1-11.)

\hypertarget{ilcs-511-25}{%
\subsection*{(720 ILCS 5/11-25)}\label{ilcs-511-25}}
\addcontentsline{toc}{subsection}{(720 ILCS 5/11-25)}

\hypertarget{sec.-11-25.-grooming.}{%
\section*{Sec. 11-25. Grooming.}\label{sec.-11-25.-grooming.}}
\addcontentsline{toc}{section}{Sec. 11-25. Grooming.}

\markright{Sec. 11-25. Grooming.}

(a) A person commits grooming when he or she knowingly uses a computer
on-line service, Internet service, local bulletin board service, or any
other device capable of electronic data storage or transmission,
performs an act in person or by conduct through a third party, or uses
written communication to seduce, solicit, lure, or entice, or attempt to
seduce, solicit, lure, or entice, a child, a child's guardian, or
another person believed by the person to be a child or a child's
guardian, to commit any sex offense as defined in Section 2 of the Sex
Offender Registration Act, to distribute photographs depicting the sex
organs of the child, or to otherwise engage in any unlawful sexual
conduct with a child or with another person believed by the person to be
a child. As used in this Section, ``child'' means a person under 17
years of age.

(b) Sentence. Grooming is a Class 4 felony.

(Source: P.A. 102-676, eff. 6-1-22 .)

\hypertarget{ilcs-511-26}{%
\subsection*{(720 ILCS 5/11-26)}\label{ilcs-511-26}}
\addcontentsline{toc}{subsection}{(720 ILCS 5/11-26)}

\hypertarget{sec.-11-26.-traveling-to-meet-a-child.}{%
\section*{Sec. 11-26. Traveling to meet a
child.}\label{sec.-11-26.-traveling-to-meet-a-child.}}
\addcontentsline{toc}{section}{Sec. 11-26. Traveling to meet a child.}

\markright{Sec. 11-26. Traveling to meet a child.}

(a) A person commits traveling to meet a child when he or she travels
any distance either within this State, to this State, or from this State
by any means, attempts to do so, or causes another to do so or attempt
to do so for the purpose of engaging in any sex offense as defined in
Section 2 of the Sex Offender Registration Act, or to otherwise engage
in other unlawful sexual conduct with a child or with another person
believed by the person to be a child after using a computer on-line
service, Internet service, local bulletin board service, or any other
device capable of electronic data storage or transmission to seduce,
solicit, lure, or entice, or to attempt to seduce, solicit, lure, or
entice, a child or a child's guardian, or another person believed by the
person to be a child or a child's guardian, for such purpose. As used in
this Section, ``child'' means a person under 17 years of age.

(b) Sentence. Traveling to meet a child is a Class 3 felony.

(Source: P.A. 100-428, eff. 1-1-18 .)

\hypertarget{ilcs-511-30}{%
\subsection*{(720 ILCS 5/11-30)}\label{ilcs-511-30}}
\addcontentsline{toc}{subsection}{(720 ILCS 5/11-30)}

(was 720 ILCS 5/11-9)

\hypertarget{sec.-11-30.-public-indecency.}{%
\section*{Sec. 11-30. Public
indecency.}\label{sec.-11-30.-public-indecency.}}
\addcontentsline{toc}{section}{Sec. 11-30. Public indecency.}

\markright{Sec. 11-30. Public indecency.}

(a) Any person of the age of 17 years and upwards who performs any of
the following acts in a public place commits a public indecency:

(1) An act of sexual penetration or sexual conduct; or

(2) A lewd exposure of the body done with intent to arouse or to satisfy
the sexual desire of the person.

Breast-feeding of infants is not an act of public indecency.

(b) ``Public place'' for purposes of this Section means any place where
the conduct may reasonably be expected to be viewed by others.

(c) Sentence.

Public indecency is a Class A misdemeanor. A person convicted of a third
or subsequent violation for public indecency is guilty of a Class 4
felony. Public indecency is a Class 4 felony if committed by a person 18
years of age or older who is on or within 500 feet of elementary or
secondary school grounds when children are present on the grounds.

(Source: P.A. 96-1098, eff. 1-1-11; 96-1551, eff. 7-1-11 .)

\hypertarget{ilcs-511-35}{%
\subsection*{(720 ILCS 5/11-35)}\label{ilcs-511-35}}
\addcontentsline{toc}{subsection}{(720 ILCS 5/11-35)}

(was 720 ILCS 5/11-7)

\hypertarget{sec.-11-35.-adultery.}{%
\section*{Sec. 11-35. Adultery.}\label{sec.-11-35.-adultery.}}
\addcontentsline{toc}{section}{Sec. 11-35. Adultery.}

\markright{Sec. 11-35. Adultery.}

(a) A person commits adultery when he or she has sexual intercourse with
another not his or her spouse, if the behavior is open and notorious,
and

(1) The person is married and knows the other person involved in such
intercourse is not his spouse; or

(2) The person is not married and knows that the other person involved
in such intercourse is married.

A person shall be exempt from prosecution under this Section if his
liability is based solely on evidence he has given in order to comply
with the requirements of Section 4-1.7 of ``The Illinois Public Aid
Code'', approved April 11, 1967, as amended.

(b) Sentence.

Adultery is a Class A misdemeanor.

(Source: P.A. 96-1551, eff. 7-1-11 .)

\hypertarget{ilcs-511-40}{%
\subsection*{(720 ILCS 5/11-40)}\label{ilcs-511-40}}
\addcontentsline{toc}{subsection}{(720 ILCS 5/11-40)}

(was 720 ILCS 5/11-8)

\hypertarget{sec.-11-40.-fornication.}{%
\section*{Sec. 11-40. Fornication.}\label{sec.-11-40.-fornication.}}
\addcontentsline{toc}{section}{Sec. 11-40. Fornication.}

\markright{Sec. 11-40. Fornication.}

(a) A person commits fornication when he or she knowingly has sexual
intercourse with another not his or her spouse if the behavior is open
and notorious.

A person shall be exempt from prosecution under this Section if his
liability is based solely on evidence he has given in order to comply
with the requirements of Section 4-1.7 of ``The Illinois Public Aid
Code'', approved April 11, 1967, as amended.

(b) Sentence.

Fornication is a Class B misdemeanor.

(Source: P.A. 96-1551, eff. 7-1-11 .)

\hypertarget{ilcs-511-45}{%
\subsection*{(720 ILCS 5/11-45)}\label{ilcs-511-45}}
\addcontentsline{toc}{subsection}{(720 ILCS 5/11-45)}

(was 720 ILCS 5/11-12)

\hypertarget{sec.-11-45.-bigamy-and-marrying-a-bigamist.}{%
\section*{Sec. 11-45. Bigamy and Marrying a
bigamist.}\label{sec.-11-45.-bigamy-and-marrying-a-bigamist.}}
\addcontentsline{toc}{section}{Sec. 11-45. Bigamy and Marrying a
bigamist.}

\markright{Sec. 11-45. Bigamy and Marrying a bigamist.}

(a) Bigamy. A person commits bigamy when that person has a husband or
wife and subsequently knowingly marries another.

(a-5) Marrying a bigamist. An unmarried person commits marrying a
bigamist when that person knowingly marries another under circumstances
known to him or her which would render the other person guilty of bigamy
under the laws of this State.

(b) It shall be an affirmative defense to bigamy and marrying a bigamist
that:

(1) The prior marriage was dissolved or declared invalid; or

(2) The accused reasonably believed the prior spouse to be dead; or

(3) The prior spouse had been continually absent for a period of 5 years
during which time the accused did not know the prior spouse to be alive;
or

(4) The accused reasonably believed that he or she or the person he or
she marries was legally eligible to be married.

(c) Sentence.

Bigamy is a Class 4 felony. Marrying a bigamist is a Class A
misdemeanor.

(Source: P.A. 96-1551, eff. 7-1-11 .)

\bookmarksetup{startatroot}

\hypertarget{article-12.-bodily-harm}{%
\chapter*{Article 12. Bodily Harm}\label{article-12.-bodily-harm}}
\addcontentsline{toc}{chapter}{Article 12. Bodily Harm}

\markboth{Article 12. Bodily Harm}{Article 12. Bodily Harm}

\hypertarget{ilcs-5art.-12-subdiv.-1-heading}{%
\subsection*{(720 ILCS 5/Art. 12, Subdiv. 1
heading)}\label{ilcs-5art.-12-subdiv.-1-heading}}
\addcontentsline{toc}{subsection}{(720 ILCS 5/Art. 12, Subdiv. 1
heading)}

SUBDIVISION 1.

DEFINITIONS

(Source: P.A. 96-1551, eff. 7-1-11.)

\hypertarget{ilcs-512-0.1}{%
\subsection*{(720 ILCS 5/12-0.1)}\label{ilcs-512-0.1}}
\addcontentsline{toc}{subsection}{(720 ILCS 5/12-0.1)}

\hypertarget{sec.-12-0.1.-definitions.}{%
\section*{Sec. 12-0.1. Definitions.}\label{sec.-12-0.1.-definitions.}}
\addcontentsline{toc}{section}{Sec. 12-0.1. Definitions.}

\markright{Sec. 12-0.1. Definitions.}

In this Article, unless the context clearly requires otherwise:

``Bona fide labor dispute'' means any controversy concerning wages,
salaries, hours, working conditions, or benefits, including health and
welfare, sick leave, insurance, and pension or retirement provisions,
the making or maintaining of collective bargaining agreements, and the
terms to be included in those agreements.

``Coach'' means a person recognized as a coach by the sanctioning
authority that conducts an athletic contest.

``Correctional institution employee'' means a person employed by a penal
institution.

``Emergency medical services personnel'' has the meaning specified in
Section 3.5 of the Emergency Medical Services (EMS) Systems Act and
shall include all ambulance crew members, including drivers or pilots.

``Family or household members'' include spouses, former spouses,
parents, children, stepchildren, and other persons related by blood or
by present or prior marriage, persons who share or formerly shared a
common dwelling, persons who have or allegedly have a child in common,
persons who share or allegedly share a blood relationship through a
child, persons who have or have had a dating or engagement relationship,
persons with disabilities and their personal assistants, and caregivers
as defined in Section 12-4.4a of this Code. For purposes of this
Article, neither a casual acquaintanceship nor ordinary fraternization
between 2 individuals in business or social contexts shall be deemed to
constitute a dating relationship.

``In the presence of a child'' means in the physical presence of a child
or knowing or having reason to know that a child is present and may see
or hear an act constituting an offense.

``Park district employee'' means a supervisor, director, instructor, or
other person employed by a park district.

``Person with a physical disability'' means a person who suffers from a
permanent and disabling physical characteristic, resulting from disease,
injury, functional disorder, or congenital condition.

``Private security officer'' means a registered employee of a private
security contractor agency under the Private Detective, Private Alarm,
Private Security, Fingerprint Vendor, and Locksmith Act of 2004.

``Probation officer'' means a person as defined in the Probation and
Probation Officers Act.

``Sports official'' means a person at an athletic contest who enforces
the rules of the contest, such as an umpire or referee.

``Sports venue'' means a publicly or privately owned sports or
entertainment arena, stadium, community or convention hall, special
event center, or amusement facility, or a special event center in a
public park, during the 12 hours before or after the sanctioned sporting
event.

``Streetgang'', ``streetgang member'', and ``criminal street gang'' have
the meanings ascribed to those terms in Section 10 of the Illinois
Streetgang Terrorism Omnibus Prevention Act.

``Transit employee'' means a driver, operator, or employee of any
transportation facility or system engaged in the business of
transporting the public for hire.

``Transit passenger'' means a passenger of any transportation facility
or system engaged in the business of transporting the public for hire,
including a passenger using any area designated by a transportation
facility or system as a vehicle boarding, departure, or transfer
location.

``Utility worker'' means any of the following:

(1) A person employed by a public utility as defined in Section 3-105 of
the Public Utilities Act.

(2) An employee of a municipally owned utility.

(3) An employee of a cable television company.

(4) An employee of an electric cooperative as defined in Section 3-119
of the Public Utilities Act.

(5) An independent contractor or an employee of an independent
contractor working on behalf of a cable television company, public
utility, municipally owned utility, or electric cooperative.

(6) An employee of a telecommunications carrier as defined in Section
13-202 of the Public Utilities Act, or an independent contractor or an
employee of an independent contractor working on behalf of a
telecommunications carrier.

(7) An employee of a telephone or telecommunications cooperative as
defined in Section 13-212 of the Public Utilities Act, or an independent
contractor or an employee of an independent contractor working on behalf
of a telephone or telecommunications cooperative.

(Source: P.A. 99-143, eff. 7-27-15; 99-816, eff. 8-15-16.)

\hypertarget{ilcs-5art.-12-subdiv.-5-heading}{%
\subsection*{(720 ILCS 5/Art. 12, Subdiv. 5
heading)}\label{ilcs-5art.-12-subdiv.-5-heading}}
\addcontentsline{toc}{subsection}{(720 ILCS 5/Art. 12, Subdiv. 5
heading)}

SUBDIVISION 5.

ASSAULT AND BATTERY

(Source: P.A. 96-1551, eff. 7-1-11.)

\hypertarget{ilcs-512-1-from-ch.-38-par.-12-1}{%
\subsection*{(720 ILCS 5/12-1) (from Ch. 38, par.
12-1)}\label{ilcs-512-1-from-ch.-38-par.-12-1}}
\addcontentsline{toc}{subsection}{(720 ILCS 5/12-1) (from Ch. 38, par.
12-1)}

\hypertarget{sec.-12-1.-assault.}{%
\section*{Sec. 12-1. Assault.}\label{sec.-12-1.-assault.}}
\addcontentsline{toc}{section}{Sec. 12-1. Assault.}

\markright{Sec. 12-1. Assault.}

(a) A person commits an assault when, without lawful authority, he or
she knowingly engages in conduct which places another in reasonable
apprehension of receiving a battery.

(b) Sentence. Assault is a Class C misdemeanor.

(c) In addition to any other sentence that may be imposed, a court shall
order any person convicted of assault to perform community service for
not less than 30 and not more than 120 hours, if community service is
available in the jurisdiction and is funded and approved by the county
board of the county where the offense was committed. In addition,
whenever any person is placed on supervision for an alleged offense
under this Section, the supervision shall be conditioned upon the
performance of the community service.

This subsection does not apply when the court imposes a sentence of
incarceration.

(Source: P.A. 96-1551, eff. 7-1-11 .)

\hypertarget{ilcs-512-2-from-ch.-38-par.-12-2}{%
\subsection*{(720 ILCS 5/12-2) (from Ch. 38, par.
12-2)}\label{ilcs-512-2-from-ch.-38-par.-12-2}}
\addcontentsline{toc}{subsection}{(720 ILCS 5/12-2) (from Ch. 38, par.
12-2)}

\hypertarget{sec.-12-2.-aggravated-assault.}{%
\section*{Sec. 12-2. Aggravated
assault.}\label{sec.-12-2.-aggravated-assault.}}
\addcontentsline{toc}{section}{Sec. 12-2. Aggravated assault.}

\markright{Sec. 12-2. Aggravated assault.}

(a) Offense based on location of conduct. A person commits aggravated
assault when he or she commits an assault against an individual who is
on or about a public way, public property, a public place of
accommodation or amusement, or a sports venue, or in a church,
synagogue, mosque, or other building, structure, or place used for
religious worship.

(b) Offense based on status of victim. A person commits aggravated
assault when, in committing an assault, he or she knows the individual
assaulted to be any of the following:

(1) A person with a physical disability or a person

60 years of age or older and the assault is without legal justification.

(2) A teacher or school employee upon school grounds or grounds adjacent
to a school or in any part of a building used for school purposes.

(3) A park district employee upon park grounds or grounds adjacent to a
park or in any part of a building used for park purposes.

(4) A community policing volunteer, private security officer, or utility
worker:

(i) performing his or her official duties;

(ii) assaulted to prevent performance of his or her official duties; or

(iii) assaulted in retaliation for performing his or her official
duties.

(4.1) A peace officer, fireman, emergency management worker, or
emergency medical services personnel:

(i) performing his or her official duties;

(ii) assaulted to prevent performance of his or her official duties; or

(iii) assaulted in retaliation for performing his or her official
duties.

(5) A correctional officer or probation officer:

(i) performing his or her official duties;

(ii) assaulted to prevent performance of his or her official duties; or

(iii) assaulted in retaliation for performing his or her official
duties.

(6) A correctional institution employee, a county juvenile detention
center employee who provides direct and continuous supervision of
residents of a juvenile detention center, including a county juvenile
detention center employee who supervises recreational activity for
residents of a juvenile detention center, or a Department of Human
Services employee, Department of Human Services officer, or employee of
a subcontractor of the Department of Human Services supervising or
controlling sexually dangerous persons or sexually violent persons:

(i) performing his or her official duties;

(ii) assaulted to prevent performance of his or her official duties; or

(iii) assaulted in retaliation for performing his or her official
duties.

(7) An employee of the State of Illinois, a municipal corporation
therein, or a political subdivision thereof, performing his or her
official duties.

(8) A transit employee performing his or her official duties, or a
transit passenger.

(9) A sports official or coach actively participating in any level of
athletic competition within a sports venue, on an indoor playing field
or outdoor playing field, or within the immediate vicinity of such a
facility or field.

(10) A person authorized to serve process under

Section 2-202 of the Code of Civil Procedure or a special process server
appointed by the circuit court, while that individual is in the
performance of his or her duties as a process server.

(c) Offense based on use of firearm, device, or motor vehicle. A person
commits aggravated assault when, in committing an assault, he or she
does any of the following:

(1) Uses a deadly weapon, an air rifle as defined in

Section 24.8-0.1 of this Act, or any device manufactured and designed to
be substantially similar in appearance to a firearm, other than by
discharging a firearm.

(2) Discharges a firearm, other than from a motor vehicle.

(3) Discharges a firearm from a motor vehicle.

(4) Wears a hood, robe, or mask to conceal his or her identity.

(5) Knowingly and without lawful justification shines or flashes a laser
gun sight or other laser device attached to a firearm, or used in
concert with a firearm, so that the laser beam strikes near or in the
immediate vicinity of any person.

(6) Uses a firearm, other than by discharging the firearm, against a
peace officer, community policing volunteer, fireman, private security
officer, emergency management worker, emergency medical services
personnel, employee of a police department, employee of a sheriff's
department, or traffic control municipal employee:

(i) performing his or her official duties;

(ii) assaulted to prevent performance of his or her official duties; or

(iii) assaulted in retaliation for performing his or her official
duties.

(7) Without justification operates a motor vehicle in a manner which
places a person, other than a person listed in subdivision (b)(4), in
reasonable apprehension of being struck by the moving motor vehicle.

(8) Without justification operates a motor vehicle in a manner which
places a person listed in subdivision (b)(4), in reasonable apprehension
of being struck by the moving motor vehicle.

(9) Knowingly video or audio records the offense with the intent to
disseminate the recording.

(d) Sentence. Aggravated assault as defined in subdivision (a), (b)(1),
(b)(2), (b)(3), (b)(4), (b)(7), (b)(8), (b)(9), (c)(1), (c)(4), or
(c)(9) is a Class A misdemeanor, except that aggravated assault as
defined in subdivision (b)(4) and (b)(7) is a Class 4 felony if a
Category I, Category II, or Category III weapon is used in the
commission of the assault. Aggravated assault as defined in subdivision
(b)(4.1), (b)(5), (b)(6), (b)(10), (c)(2), (c)(5), (c)(6), or (c)(7) is
a Class 4 felony. Aggravated assault as defined in subdivision (c)(3) or
(c)(8) is a Class 3 felony.

(e) For the purposes of this Section, ``Category I weapon'', ``Category
II weapon'', and ``Category III weapon'' have the meanings ascribed to
those terms in Section 33A-1 of this Code.

(Source: P.A. 101-223, eff. 1-1-20; 102-558, eff. 8-20-21.)

\hypertarget{ilcs-512-2.5}{%
\subsection*{(720 ILCS 5/12-2.5)}\label{ilcs-512-2.5}}
\addcontentsline{toc}{subsection}{(720 ILCS 5/12-2.5)}

(This Section was renumbered as Section 12-5.02 by P.A. 96-1551.)

\hypertarget{sec.-12-2.5.-renumbered.}{%
\section*{Sec. 12-2.5. (Renumbered).}\label{sec.-12-2.5.-renumbered.}}
\addcontentsline{toc}{section}{Sec. 12-2.5. (Renumbered).}

\markright{Sec. 12-2.5. (Renumbered).}

(Source: P.A. 88-467. Renumbered by P.A. 96-1551, eff. 7-1-11 .)

\hypertarget{ilcs-512-2.6}{%
\subsection*{(720 ILCS 5/12-2.6)}\label{ilcs-512-2.6}}
\addcontentsline{toc}{subsection}{(720 ILCS 5/12-2.6)}

(This Section was renumbered as Section 12-5.3 by P.A. 96-1551.)

\hypertarget{sec.-12-2.6.-renumbered.}{%
\section*{Sec. 12-2.6. (Renumbered).}\label{sec.-12-2.6.-renumbered.}}
\addcontentsline{toc}{section}{Sec. 12-2.6. (Renumbered).}

\markright{Sec. 12-2.6. (Renumbered).}

(Source: P.A. 94-743, eff. 5-8-06. Renumbered by P.A. 96-1551, eff.
7-1-11 .)

\hypertarget{ilcs-512-3-from-ch.-38-par.-12-3}{%
\subsection*{(720 ILCS 5/12-3) (from Ch. 38, par.
12-3)}\label{ilcs-512-3-from-ch.-38-par.-12-3}}
\addcontentsline{toc}{subsection}{(720 ILCS 5/12-3) (from Ch. 38, par.
12-3)}

\hypertarget{sec.-12-3.-battery.}{%
\section*{Sec. 12-3. Battery.}\label{sec.-12-3.-battery.}}
\addcontentsline{toc}{section}{Sec. 12-3. Battery.}

\markright{Sec. 12-3. Battery.}

(a) A person commits battery if he or she knowingly without legal
justification by any means (1) causes bodily harm to an individual or
(2) makes physical contact of an insulting or provoking nature with an
individual.

(b) Sentence.

Battery is a Class A misdemeanor.

(Source: P.A. 96-1551, eff. 7-1-11 .)

\hypertarget{ilcs-512-3.05}{%
\subsection*{(720 ILCS 5/12-3.05)}\label{ilcs-512-3.05}}
\addcontentsline{toc}{subsection}{(720 ILCS 5/12-3.05)}

(was 720 ILCS 5/12-4)

\hypertarget{sec.-12-3.05.-aggravated-battery.}{%
\section*{Sec. 12-3.05. Aggravated
battery.}\label{sec.-12-3.05.-aggravated-battery.}}
\addcontentsline{toc}{section}{Sec. 12-3.05. Aggravated battery.}

\markright{Sec. 12-3.05. Aggravated battery.}

(a) Offense based on injury. A person commits aggravated battery when,
in committing a battery, other than by the discharge of a firearm, he or
she knowingly does any of the following:

(1) Causes great bodily harm or permanent disability or disfigurement.

(2) Causes severe and permanent disability, great bodily harm, or
disfigurement by means of a caustic or flammable substance, a poisonous
gas, a deadly biological or chemical contaminant or agent, a radioactive
substance, or a bomb or explosive compound.

(3) Causes great bodily harm or permanent disability or disfigurement to
an individual whom the person knows to be a peace officer, community
policing volunteer, fireman, private security officer, correctional
institution employee, or Department of Human Services employee
supervising or controlling sexually dangerous persons or sexually
violent persons:

(i) performing his or her official duties;

(ii) battered to prevent performance of his or her official duties; or

(iii) battered in retaliation for performing his or her official duties.

(4) Causes great bodily harm or permanent disability or disfigurement to
an individual 60 years of age or older.

(5) Strangles another individual.

(b) Offense based on injury to a child or person with an intellectual
disability. A person who is at least 18 years of age commits aggravated
battery when, in committing a battery, he or she knowingly and without
legal justification by any means:

(1) causes great bodily harm or permanent disability or disfigurement to
any child under the age of 13 years, or to any person with a severe or
profound intellectual disability; or

(2) causes bodily harm or disability or disfigurement to any child under
the age of 13 years or to any person with a severe or profound
intellectual disability.

(c) Offense based on location of conduct. A person commits aggravated
battery when, in committing a battery, other than by the discharge of a
firearm, he or she is or the person battered is on or about a public
way, public property, a public place of accommodation or amusement, a
sports venue, or a domestic violence shelter, or in a church, synagogue,
mosque, or other building, structure, or place used for religious
worship.

(d) Offense based on status of victim. A person commits aggravated
battery when, in committing a battery, other than by discharge of a
firearm, he or she knows the individual battered to be any of the
following:

(1) A person 60 years of age or older.

(2) A person who is pregnant or has a physical disability.

(3) A teacher or school employee upon school grounds or grounds adjacent
to a school or in any part of a building used for school purposes.

(4) A peace officer, community policing volunteer, fireman, private
security officer, correctional institution employee, or Department of
Human Services employee supervising or controlling sexually dangerous
persons or sexually violent persons:

(i) performing his or her official duties;

(ii) battered to prevent performance of his or her official duties; or

(iii) battered in retaliation for performing his or her official duties.

(5) A judge, emergency management worker, emergency medical services
personnel, or utility worker:

(i) performing his or her official duties;

(ii) battered to prevent performance of his or her official duties; or

(iii) battered in retaliation for performing his or her official duties.

(6) An officer or employee of the State of Illinois, a unit of local
government, or a school district, while performing his or her official
duties.

(7) A transit employee performing his or her official duties, or a
transit passenger.

(8) A taxi driver on duty.

(9) A merchant who detains the person for an alleged commission of
retail theft under Section 16-26 of this Code and the person without
legal justification by any means causes bodily harm to the merchant.

(10) A person authorized to serve process under

Section 2-202 of the Code of Civil Procedure or a special process server
appointed by the circuit court while that individual is in the
performance of his or her duties as a process server.

(11) A nurse while in the performance of his or her duties as a nurse.

(12) A merchant: (i) while performing his or her duties, including, but
not limited to, relaying directions for healthcare or safety from his or
her supervisor or employer or relaying health or safety guidelines,
recommendations, regulations, or rules from a federal, State, or local
public health agency; and (ii) during a disaster declared by the
Governor, or a state of emergency declared by the mayor of the
municipality in which the merchant is located, due to a public health
emergency and for a period of 6 months after such declaration.

(e) Offense based on use of a firearm. A person commits aggravated
battery when, in committing a battery, he or she knowingly does any of
the following:

(1) Discharges a firearm, other than a machine gun or a firearm equipped
with a silencer, and causes any injury to another person.

(2) Discharges a firearm, other than a machine gun or a firearm equipped
with a silencer, and causes any injury to a person he or she knows to be
a peace officer, community policing volunteer, person summoned by a
police officer, fireman, private security officer, correctional
institution employee, or emergency management worker:

(i) performing his or her official duties;

(ii) battered to prevent performance of his or her official duties; or

(iii) battered in retaliation for performing his or her official duties.

(3) Discharges a firearm, other than a machine gun or a firearm equipped
with a silencer, and causes any injury to a person he or she knows to be
emergency medical services personnel:

(i) performing his or her official duties;

(ii) battered to prevent performance of his or her official duties; or

(iii) battered in retaliation for performing his or her official duties.

(4) Discharges a firearm and causes any injury to a person he or she
knows to be a teacher, a student in a school, or a school employee, and
the teacher, student, or employee is upon school grounds or grounds
adjacent to a school or in any part of a building used for school
purposes.

(5) Discharges a machine gun or a firearm equipped with a silencer, and
causes any injury to another person.

(6) Discharges a machine gun or a firearm equipped with a silencer, and
causes any injury to a person he or she knows to be a peace officer,
community policing volunteer, person summoned by a police officer,
fireman, private security officer, correctional institution employee or
emergency management worker:

(i) performing his or her official duties;

(ii) battered to prevent performance of his or her official duties; or

(iii) battered in retaliation for performing his or her official duties.

(7) Discharges a machine gun or a firearm equipped with a silencer, and
causes any injury to a person he or she knows to be emergency medical
services personnel:

(i) performing his or her official duties;

(ii) battered to prevent performance of his or her official duties; or

(iii) battered in retaliation for performing his or her official duties.

(8) Discharges a machine gun or a firearm equipped with a silencer, and
causes any injury to a person he or she knows to be a teacher, or a
student in a school, or a school employee, and the teacher, student, or
employee is upon school grounds or grounds adjacent to a school or in
any part of a building used for school purposes.

(f) Offense based on use of a weapon or device. A person commits
aggravated battery when, in committing a battery, he or she does any of
the following:

(1) Uses a deadly weapon other than by discharge of a firearm, or uses
an air rifle as defined in Section 24.8-0.1 of this Code.

(2) Wears a hood, robe, or mask to conceal his or her identity.

(3) Knowingly and without lawful justification shines or flashes a laser
gunsight or other laser device attached to a firearm, or used in concert
with a firearm, so that the laser beam strikes upon or against the
person of another.

(4) Knowingly video or audio records the offense with the intent to
disseminate the recording.

(g) Offense based on certain conduct. A person commits aggravated
battery when, other than by discharge of a firearm, he or she does any
of the following:

(1) Violates Section 401 of the Illinois Controlled

Substances Act by unlawfully delivering a controlled substance to
another and any user experiences great bodily harm or permanent
disability as a result of the injection, inhalation, or ingestion of any
amount of the controlled substance.

(2) Knowingly administers to an individual or causes him or her to take,
without his or her consent or by threat or deception, and for other than
medical purposes, any intoxicating, poisonous, stupefying, narcotic,
anesthetic, or controlled substance, or gives to another person any food
containing any substance or object intended to cause physical injury if
eaten.

(3) Knowingly causes or attempts to cause a correctional institution
employee or Department of Human Services employee to come into contact
with blood, seminal fluid, urine, or feces by throwing, tossing, or
expelling the fluid or material, and the person is an inmate of a penal
institution or is a sexually dangerous person or sexually violent person
in the custody of the Department of Human Services.

(h) Sentence. Unless otherwise provided, aggravated battery is a Class 3
felony.

Aggravated battery as defined in subdivision (a)(4), (d)(4), or (g)(3)
is a Class 2 felony.

Aggravated battery as defined in subdivision (a)(3) or (g)(1) is a Class
1 felony.

Aggravated battery as defined in subdivision (a)(1) is a Class 1 felony
when the aggravated battery was intentional and involved the infliction
of torture, as defined in paragraph (14) of subsection (b) of Section
9-1 of this Code, as the infliction of or subjection to extreme physical
pain, motivated by an intent to increase or prolong the pain, suffering,
or agony of the victim.

Aggravated battery as defined in subdivision (a)(1) is a Class 2 felony
when the person causes great bodily harm or permanent disability to an
individual whom the person knows to be a member of a congregation
engaged in prayer or other religious activities at a church, synagogue,
mosque, or other building, structure, or place used for religious
worship.

Aggravated battery under subdivision (a)(5) is a Class 1 felony if:

(A) the person used or attempted to use a dangerous instrument while
committing the offense;

(B) the person caused great bodily harm or permanent disability or
disfigurement to the other person while committing the offense; or

(C) the person has been previously convicted of a violation of
subdivision (a)(5) under the laws of this State or laws similar to
subdivision (a)(5) of any other state.

Aggravated battery as defined in subdivision (e)(1) is a Class X felony.

Aggravated battery as defined in subdivision (a)(2) is a Class X felony
for which a person shall be sentenced to a term of imprisonment of a
minimum of 6 years and a maximum of 45 years.

Aggravated battery as defined in subdivision (e)(5) is a Class X felony
for which a person shall be sentenced to a term of imprisonment of a
minimum of 12 years and a maximum of 45 years.

Aggravated battery as defined in subdivision (e)(2), (e)(3), or (e)(4)
is a Class X felony for which a person shall be sentenced to a term of
imprisonment of a minimum of 15 years and a maximum of 60 years.

Aggravated battery as defined in subdivision (e)(6), (e)(7), or (e)(8)
is a Class X felony for which a person shall be sentenced to a term of
imprisonment of a minimum of 20 years and a maximum of 60 years.

Aggravated battery as defined in subdivision (b)(1) is a Class X felony,
except that:

(1) if the person committed the offense while armed with a firearm, 15
years shall be added to the term of imprisonment imposed by the court;

(2) if, during the commission of the offense, the person personally
discharged a firearm, 20 years shall be added to the term of
imprisonment imposed by the court;

(3) if, during the commission of the offense, the person personally
discharged a firearm that proximately caused great bodily harm,
permanent disability, permanent disfigurement, or death to another
person, 25 years or up to a term of natural life shall be added to the
term of imprisonment imposed by the court.

(i) Definitions. In this Section:

``Building or other structure used to provide shelter'' has the meaning
ascribed to ``shelter'' in Section 1 of the Domestic Violence Shelters
Act.

``Domestic violence'' has the meaning ascribed to it in Section 103 of
the Illinois Domestic Violence Act of 1986.

``Domestic violence shelter'' means any building or other structure used
to provide shelter or other services to victims or to the dependent
children of victims of domestic violence pursuant to the Illinois
Domestic Violence Act of 1986 or the Domestic Violence Shelters Act, or
any place within 500 feet of such a building or other structure in the
case of a person who is going to or from such a building or other
structure.

``Firearm'' has the meaning provided under Section 1.1 of the Firearm
Owners Identification Card Act, and does not include an air rifle as
defined by Section 24.8-0.1 of this Code.

``Machine gun'' has the meaning ascribed to it in Section 24-1 of this
Code.

``Merchant'' has the meaning ascribed to it in Section 16-0.1 of this
Code.

``Strangle'' means intentionally impeding the normal breathing or
circulation of the blood of an individual by applying pressure on the
throat or neck of that individual or by blocking the nose or mouth of
that individual.

(Source: P.A. 101-223, eff. 1-1-20; 101-651, eff. 8-7-20.)

\hypertarget{ilcs-512-3.1-from-ch.-38-par.-12-3.1}{%
\subsection*{(720 ILCS 5/12-3.1) (from Ch. 38, par.
12-3.1)}\label{ilcs-512-3.1-from-ch.-38-par.-12-3.1}}
\addcontentsline{toc}{subsection}{(720 ILCS 5/12-3.1) (from Ch. 38, par.
12-3.1)}

\hypertarget{sec.-12-3.1.-battery-of-an-unborn-child-aggravated-battery-of-an-unborn-child.}{%
\section*{Sec. 12-3.1. Battery of an unborn child; aggravated battery of
an unborn
child.}\label{sec.-12-3.1.-battery-of-an-unborn-child-aggravated-battery-of-an-unborn-child.}}
\addcontentsline{toc}{section}{Sec. 12-3.1. Battery of an unborn child;
aggravated battery of an unborn child.}

\markright{Sec. 12-3.1. Battery of an unborn child; aggravated battery
of an unborn child.}

(a) A person commits battery of an unborn child if he or she knowingly
without legal justification and by any means causes bodily harm to an
unborn child.

(a-5) A person commits aggravated battery of an unborn child when, in
committing a battery of an unborn child, he or she knowingly causes
great bodily harm or permanent disability or disfigurement to an unborn
child.

(b) For purposes of this Section, (1) ``unborn child'' shall mean any
individual of the human species from the implantation of an embryo until
birth, and (2) ``person'' shall not include the pregnant individual
whose unborn child is harmed.

(c) Sentence. Battery of an unborn child is a Class A misdemeanor.
Aggravated battery of an unborn child is a Class 2 felony.

(d) This Section shall not apply to acts which cause bodily harm to an
unborn child if those acts were committed during any abortion, as
defined in Section 1-10 of the Reproductive Health Act, to which the
pregnant individual has consented. This Section shall not apply to acts
which were committed pursuant to usual and customary standards of
medical practice during diagnostic testing or therapeutic treatment.

(Source: P.A. 101-13, eff. 6-12-19.)

\hypertarget{ilcs-512-3.2-from-ch.-38-par.-12-3.2}{%
\subsection*{(720 ILCS 5/12-3.2) (from Ch. 38, par.
12-3.2)}\label{ilcs-512-3.2-from-ch.-38-par.-12-3.2}}
\addcontentsline{toc}{subsection}{(720 ILCS 5/12-3.2) (from Ch. 38, par.
12-3.2)}

\hypertarget{sec.-12-3.2.-domestic-battery.}{%
\section*{Sec. 12-3.2. Domestic
battery.}\label{sec.-12-3.2.-domestic-battery.}}
\addcontentsline{toc}{section}{Sec. 12-3.2. Domestic battery.}

\markright{Sec. 12-3.2. Domestic battery.}

(a) A person commits domestic battery if he or she knowingly without
legal justification by any means:

(1) causes bodily harm to any family or household member;

(2) makes physical contact of an insulting or provoking nature with any
family or household member.

(b) Sentence. Domestic battery is a Class A misdemeanor. Domestic
battery is a Class 4 felony if the defendant has any prior conviction
under this Code for violation of an order of protection (Section 12-3.4
or 12-30), or any prior conviction under the law of another jurisdiction
for an offense which is substantially similar. Domestic battery is a
Class 4 felony if the defendant has any prior conviction under this Code
for first degree murder (Section 9-1), attempt to commit first degree
murder (Section 8-4), aggravated domestic battery (Section 12-3.3),
aggravated battery (Section 12-3.05 or 12-4), heinous battery (Section
12-4.1), aggravated battery with a firearm (Section 12-4.2), aggravated
battery with a machine gun or a firearm equipped with a silencer
(Section 12-4.2-5), aggravated battery of a child (Section 12-4.3),
aggravated battery of an unborn child (subsection (a-5) of Section
12-3.1, or Section 12-4.4), aggravated battery of a senior citizen
(Section 12-4.6), stalking (Section 12-7.3), aggravated stalking
(Section 12-7.4), criminal sexual assault (Section 11-1.20 or 12-13),
aggravated criminal sexual assault (Section 11-1.30 or 12-14),
kidnapping (Section 10-1), aggravated kidnapping (Section 10-2),
predatory criminal sexual assault of a child (Section 11-1.40 or
12-14.1), aggravated criminal sexual abuse (Section 11-1.60 or 12-16),
unlawful restraint (Section 10-3), aggravated unlawful restraint
(Section 10-3.1), aggravated arson (Section 20-1.1), or aggravated
discharge of a firearm (Section 24-1.2), or any prior conviction under
the law of another jurisdiction for any offense that is substantially
similar to the offenses listed in this Section, when any of these
offenses have been committed against a family or household member.
Domestic battery is a Class 4 felony if the defendant has one or 2 prior
convictions under this Code for domestic battery (Section 12-3.2), or
one or 2 prior convictions under the law of another jurisdiction for any
offense which is substantially similar. Domestic battery is a Class 3
felony if the defendant had 3 prior convictions under this Code for
domestic battery (Section 12-3.2), or 3 prior convictions under the law
of another jurisdiction for any offense which is substantially similar.
Domestic battery is a Class 2 felony if the defendant had 4 or more
prior convictions under this Code for domestic battery (Section 12-3.2),
or 4 or more prior convictions under the law of another jurisdiction for
any offense which is substantially similar. In addition to any other
sentencing alternatives, for any second or subsequent conviction of
violating this Section, the offender shall be mandatorily sentenced to a
minimum of 72 consecutive hours of imprisonment. The imprisonment shall
not be subject to suspension, nor shall the person be eligible for
probation in order to reduce the sentence.

(c) Domestic battery committed in the presence of a child. In addition
to any other sentencing alternatives, a defendant who commits, in the
presence of a child, a felony domestic battery (enhanced under
subsection (b)), aggravated domestic battery (Section 12-3.3),
aggravated battery (Section 12-3.05 or 12-4), unlawful restraint
(Section 10-3), or aggravated unlawful restraint (Section 10-3.1)
against a family or household member shall be required to serve a
mandatory minimum imprisonment of 10 days or perform 300 hours of
community service, or both. The defendant shall further be liable for
the cost of any counseling required for the child at the discretion of
the court in accordance with subsection (b) of Section 5-5-6 of the
Unified Code of Corrections. For purposes of this Section, ``child''
means a person under 18 years of age who is the defendant's or victim's
child or step-child or who is a minor child residing within or visiting
the household of the defendant or victim.

(d) Upon conviction of domestic battery, the court shall advise the
defendant orally or in writing, substantially as follows: ``An
individual convicted of domestic battery may be subject to federal
criminal penalties for possessing, transporting, shipping, or receiving
any firearm or ammunition in violation of the federal Gun Control Act of
1968 (18 U.S.C. 922(g)(8) and (9)).'' A notation shall be made in the
court file that the admonition was given.

(Source: P.A. 97-1109, eff. 1-1-13; 98-187, eff. 1-1-14; 98-994, eff.
1-1-15 .)

\hypertarget{ilcs-512-3.3}{%
\subsection*{(720 ILCS 5/12-3.3)}\label{ilcs-512-3.3}}
\addcontentsline{toc}{subsection}{(720 ILCS 5/12-3.3)}

\hypertarget{sec.-12-3.3.-aggravated-domestic-battery.}{%
\section*{Sec. 12-3.3. Aggravated domestic
battery.}\label{sec.-12-3.3.-aggravated-domestic-battery.}}
\addcontentsline{toc}{section}{Sec. 12-3.3. Aggravated domestic
battery.}

\markright{Sec. 12-3.3. Aggravated domestic battery.}

(a) A person who, in committing a domestic battery, knowingly causes
great bodily harm, or permanent disability or disfigurement commits
aggravated domestic battery.

(a-5) A person who, in committing a domestic battery, strangles another
individual commits aggravated domestic battery. For the purposes of this
subsection (a-5), ``strangle'' means intentionally impeding the normal
breathing or circulation of the blood of an individual by applying
pressure on the throat or neck of that individual or by blocking the
nose or mouth of that individual.

(b) Sentence. Aggravated domestic battery is a Class 2 felony. Any order
of probation or conditional discharge entered following a conviction for
an offense under this Section must include, in addition to any other
condition of probation or conditional discharge, a condition that the
offender serve a mandatory term of imprisonment of not less than 60
consecutive days. A person convicted of a second or subsequent violation
of this Section must be sentenced to a mandatory term of imprisonment of
not less than 3 years and not more than 7 years or an extended term of
imprisonment of not less than 7 years and not more than 14 years.

(c) Upon conviction of aggravated domestic battery, the court shall
advise the defendant orally or in writing, substantially as follows:
``An individual convicted of aggravated domestic battery may be subject
to federal criminal penalties for possessing, transporting, shipping, or
receiving any firearm or ammunition in violation of the federal Gun
Control Act of 1968 (18 U.S.C. 922(g)(8) and (9)).'' A notation shall be
made in the court file that the admonition was given.

(Source: P.A. 96-287, eff. 8-11-09; 96-363, eff. 8-13-09; 96-1000, eff.
7-2-10; 96-1551, eff. 7-1-11 .)

\hypertarget{ilcs-512-3.4}{%
\subsection*{(720 ILCS 5/12-3.4)}\label{ilcs-512-3.4}}
\addcontentsline{toc}{subsection}{(720 ILCS 5/12-3.4)}

(was 720 ILCS 5/12-30)

\hypertarget{sec.-12-3.4.-violation-of-an-order-of-protection.}{%
\section*{Sec. 12-3.4. Violation of an order of
protection.}\label{sec.-12-3.4.-violation-of-an-order-of-protection.}}
\addcontentsline{toc}{section}{Sec. 12-3.4. Violation of an order of
protection.}

\markright{Sec. 12-3.4. Violation of an order of protection.}

(a) A person commits violation of an order of protection if:

(1) He or she knowingly commits an act which was prohibited by a court
or fails to commit an act which was ordered by a court in violation of:

(i) a remedy in a valid order of protection authorized under paragraphs
(1), (2), (3), (14), or (14.5) of subsection (b) of Section 214 of the
Illinois Domestic Violence Act of 1986,

(ii) a remedy, which is substantially similar to the remedies authorized
under paragraphs (1), (2), (3), (14) or (14.5) of subsection (b) of
Section 214 of the Illinois Domestic Violence Act of 1986, in a valid
order of protection, which is authorized under the laws of another
state, tribe or United States territory,

(iii) any other remedy when the act constitutes a crime against the
protected parties as the term protected parties is defined in Section
112A-4 of the Code of Criminal Procedure of 1963; and

(2) Such violation occurs after the offender has been served notice of
the contents of the order, pursuant to the Illinois Domestic Violence
Act of 1986 or any substantially similar statute of another state, tribe
or United States territory, or otherwise has acquired actual knowledge
of the contents of the order.

An order of protection issued by a state, tribal or territorial court
related to domestic or family violence shall be deemed valid if the
issuing court had jurisdiction over the parties and matter under the law
of the state, tribe or territory. There shall be a presumption of
validity where an order is certified and appears authentic on its face.
For purposes of this Section, an ``order of protection'' may have been
issued in a criminal or civil proceeding.

(a-5) Failure to provide reasonable notice and opportunity to be heard
shall be an affirmative defense to any charge or process filed seeking
enforcement of a foreign order of protection.

(b) Nothing in this Section shall be construed to diminish the inherent
authority of the courts to enforce their lawful orders through civil or
criminal contempt proceedings.

(c) The limitations placed on law enforcement liability by Section 305
of the Illinois Domestic Violence Act of 1986 apply to actions taken
under this Section.

(d) Violation of an order of protection is a Class A misdemeanor.
Violation of an order of protection is a Class 4 felony if the defendant
has any prior conviction under this Code for domestic battery (Section
12-3.2) or violation of an order of protection (Section 12-3.4 or 12-30)
or any prior conviction under the law of another jurisdiction for an
offense that could be charged in this State as a domestic battery or
violation of an order of protection. Violation of an order of protection
is a Class 4 felony if the defendant has any prior conviction under this
Code for first degree murder (Section 9-1), attempt to commit first
degree murder (Section 8-4), aggravated domestic battery (Section
12-3.3), aggravated battery (Section 12-3.05 or 12-4), heinous battery
(Section 12-4.1), aggravated battery with a firearm (Section 12-4.2),
aggravated battery with a machine gun or a firearm equipped with a
silencer (Section 12-4.2-5), aggravated battery of a child (Section
12-4.3), aggravated battery of an unborn child (subsection (a-5) of
Section 12-3.1, or Section 12-4.4), aggravated battery of a senior
citizen (Section 12-4.6), stalking (Section 12-7.3), aggravated stalking
(Section 12-7.4), criminal sexual assault (Section 11-1.20 or 12-13),
aggravated criminal sexual assault (Section 11-1.30 or 12-14),
kidnapping (Section 10-1), aggravated kidnapping (Section 10-2),
predatory criminal sexual assault of a child (Section 11-1.40 or
12-14.1), aggravated criminal sexual abuse (Section 11-1.60 or 12-16),
unlawful restraint (Section 10-3), aggravated unlawful restraint
(Section 10-3.1), aggravated arson (Section 20-1.1), aggravated
discharge of a firearm (Section 24-1.2), or a violation of any former
law of this State that is substantially similar to any listed offense,
or any prior conviction under the law of another jurisdiction for an
offense that could be charged in this State as one of the offenses
listed in this Section, when any of these offenses have been committed
against a family or household member as defined in Section 112A-3 of the
Code of Criminal Procedure of 1963. The court shall impose a minimum
penalty of 24 hours imprisonment for defendant's second or subsequent
violation of any order of protection; unless the court explicitly finds
that an increased penalty or such period of imprisonment would be
manifestly unjust. In addition to any other penalties, the court may
order the defendant to pay a fine as authorized under Section 5-9-1 of
the Unified Code of Corrections or to make restitution to the victim
under Section 5-5-6 of the Unified Code of Corrections.

(e) (Blank).

(f) A defendant who directed the actions of a third party to violate
this Section, under the principles of accountability set forth in
Article 5 of this Code, is guilty of violating this Section as if the
same had been personally done by the defendant, without regard to the
mental state of the third party acting at the direction of the
defendant.

(Source: P.A. 100-987, eff. 7-1-19 .)

\hypertarget{ilcs-512-3.5}{%
\subsection*{(720 ILCS 5/12-3.5)}\label{ilcs-512-3.5}}
\addcontentsline{toc}{subsection}{(720 ILCS 5/12-3.5)}

(was 720 ILCS 5/12-6.3)

\hypertarget{sec.-12-3.5.-interfering-with-the-reporting-of-domestic-violence.}{%
\section*{Sec. 12-3.5. Interfering with the reporting of domestic
violence.}\label{sec.-12-3.5.-interfering-with-the-reporting-of-domestic-violence.}}
\addcontentsline{toc}{section}{Sec. 12-3.5. Interfering with the
reporting of domestic violence.}

\markright{Sec. 12-3.5. Interfering with the reporting of domestic
violence.}

(a) A person commits interfering with the reporting of domestic violence
when, after having committed an act of domestic violence, he or she
knowingly prevents or attempts to prevent the victim of or a witness to
the act of domestic violence from calling a 9-1-1 emergency telephone
system, obtaining medical assistance, or making a report to any law
enforcement official.

(b) For the purposes of this Section:

``Domestic violence'' shall have the meaning ascribed to it in Section
112A-3 of the Code of Criminal Procedure of 1963.

(c) Sentence. Interfering with the reporting of domestic violence is a
Class A misdemeanor.

(Source: P.A. 96-1551, eff. 7-1-11 .)

\hypertarget{ilcs-512-3.6}{%
\subsection*{(720 ILCS 5/12-3.6)}\label{ilcs-512-3.6}}
\addcontentsline{toc}{subsection}{(720 ILCS 5/12-3.6)}

(was 720 ILCS 5/45-1 and 5/45-2)

\hypertarget{sec.-12-3.6.-disclosing-location-of-domestic-violence-victim.}{%
\section*{Sec. 12-3.6. Disclosing location of domestic violence
victim.}\label{sec.-12-3.6.-disclosing-location-of-domestic-violence-victim.}}
\addcontentsline{toc}{section}{Sec. 12-3.6. Disclosing location of
domestic violence victim.}

\markright{Sec. 12-3.6. Disclosing location of domestic violence
victim.}

(a) As used in this Section:

``Domestic violence'' means attempting to cause or causing abuse of a
family or household member or high-risk adult with disabilities, or
attempting to cause or causing neglect or exploitation of a high-risk
adult with disabilities which threatens the adult's health and safety.

``Family or household member'' means a spouse, person living as a
spouse, parent, or other adult person related by consanguinity or
affinity, who is residing or has resided with the person committing
domestic violence. ``Family or household member'' includes a high-risk
adult with disabilities who resides with or receives care from any
person who has the responsibility for a high-risk adult as a result of a
family relationship or who has assumed responsibility for all or a
portion of the care of an adult with disabilities voluntarily, by
express or implied contract, or by court order.

``High-risk adult with disabilities'' means a person aged 18 or over
whose physical or mental disability impairs his or her ability to seek
or obtain protection from abuse, neglect, or exploitation.

``Abuse'', ``exploitation'', and ``neglect'' have the meanings ascribed
to those terms in Section 103 of the Illinois Domestic Violence Act of
1986.

(b) A person commits disclosure of location of domestic violence victim
when he or she publishes, disseminates or otherwise discloses the
location of any domestic violence victim, without that person's
authorization, knowing the disclosure will result in, or has the
substantial likelihood of resulting in, the threat of bodily harm.

(c) Nothing in this Section shall apply to confidential communications
between an attorney and his or her client.

(d) Sentence. Disclosure of location of domestic violence victim is a
Class A misdemeanor.

(Source: P.A. 96-1551, eff. 7-1-11 .)

\hypertarget{ilcs-512-3.8}{%
\subsection*{(720 ILCS 5/12-3.8)}\label{ilcs-512-3.8}}
\addcontentsline{toc}{subsection}{(720 ILCS 5/12-3.8)}

\hypertarget{sec.-12-3.8.-violation-of-a-civil-no-contact-order.}{%
\section*{Sec. 12-3.8. Violation of a civil no contact
order.}\label{sec.-12-3.8.-violation-of-a-civil-no-contact-order.}}
\addcontentsline{toc}{section}{Sec. 12-3.8. Violation of a civil no
contact order.}

\markright{Sec. 12-3.8. Violation of a civil no contact order.}

(a) A person commits violation of a civil no contact order if:

(1) he or she knowingly commits an act which was prohibited by a court
or fails to commit an act which was ordered in violation of:

(A) a remedy of a valid civil no contact order authorized under Section
213 of the Civil No Contact Order Act or Section 112A-14.5 of the Code
of Criminal Procedure of 1963; or

(B) a remedy, which is substantially similar to the remedies authorized
under Section 213 of the Civil No Contact Order Act or Section 112A-14.5
of the Code of Criminal Procedure of 1963, or in a valid civil no
contact order, which is authorized under the laws of another state,
tribe, or United States territory; and

(2) the violation occurs after the offender has been served notice of
the contents of the order under the Civil No Contact Order Act, Article
112A of the Code of Criminal Procedure of 1963, or any substantially
similar statute of another state, tribe, or United States territory, or
otherwise has acquired actual knowledge of the contents of the order.

A civil no contact order issued by a state, tribal, or territorial court
shall be deemed valid if the issuing court had jurisdiction over the
parties and matter under the law of the state, tribe, or territory.
There shall be a presumption of validity when an order is certified and
appears authentic on its face.

(a-3) For purposes of this Section, a ``civil no contact order'' may
have been issued in a criminal or civil proceeding.

(a-5) Failure to provide reasonable notice and opportunity to be heard
shall be an affirmative defense to any charge or process filed seeking
enforcement of a foreign civil no contact order.

(b) Prosecution for a violation of a civil no contact order shall not
bar a concurrent prosecution for any other crime, including any crime
that may have been committed at the time of the violation of the civil
no contact order.

(c) Nothing in this Section shall be construed to diminish the inherent
authority of the courts to enforce their lawful orders through civil or
criminal contempt proceedings.

(d) A defendant who directed the actions of a third party to violate
this Section, under the principles of accountability set forth in
Article 5 of this Code, is guilty of violating this Section as if the
same had been personally done by the defendant, without regard to the
mental state of the third party acting at the direction of the
defendant.

(e) Sentence. A violation of a civil no contact order is a Class A
misdemeanor for a first violation, and a Class 4 felony for a second or
subsequent violation.

(Source: P.A. 100-199, eff. 1-1-18 .)

\hypertarget{ilcs-512-3.9}{%
\subsection*{(720 ILCS 5/12-3.9)}\label{ilcs-512-3.9}}
\addcontentsline{toc}{subsection}{(720 ILCS 5/12-3.9)}

\hypertarget{sec.-12-3.9.-violation-of-a-stalking-no-contact-order.}{%
\section*{Sec. 12-3.9. Violation of a stalking no contact
order.}\label{sec.-12-3.9.-violation-of-a-stalking-no-contact-order.}}
\addcontentsline{toc}{section}{Sec. 12-3.9. Violation of a stalking no
contact order.}

\markright{Sec. 12-3.9. Violation of a stalking no contact order.}

(a) A person commits violation of a stalking no contact order if:

(1) he or she knowingly commits an act which was prohibited by a court
or fails to commit an act which was ordered by a court in violation of:

(A) a remedy in a valid stalking no contact order of protection
authorized under Section 80 of the Stalking No Contact Order Act or
Section 112A-14.7 of the Code of Criminal Procedure of 1963; or

(B) a remedy, which is substantially similar to the remedies authorized
under Section 80 of the Stalking No Contact Order Act or Section
112A-14.7 of the Code of Criminal Procedure of 1963, or in a valid
stalking no contact order, which is authorized under the laws of another
state, tribe, or United States territory; and

(2) the violation occurs after the offender has been served notice of
the contents of the order, under the Stalking No Contact Order Act,
Article 112A of the Code of Criminal Procedure of 1963, or any
substantially similar statute of another state, tribe, or United States
territory, or otherwise has acquired actual knowledge of the contents of
the order.

A stalking no contact order issued by a state, tribal, or territorial
court shall be deemed valid if the issuing court had jurisdiction over
the parties and matter under the law of the state, tribe, or territory.
There shall be a presumption of validity when an order is certified and
appears authentic on its face.

(a-3) For purposes of this Section, a ``stalking no contact order'' may
have been issued in a criminal or civil proceeding.

(a-5) Failure to provide reasonable notice and opportunity to be heard
shall be an affirmative defense to any charge or process filed seeking
enforcement of a foreign stalking no contact order.

(b) Prosecution for a violation of a stalking no contact order shall not
bar a concurrent prosecution for any other crime, including any crime
that may have been committed at the time of the violation of the civil
no contact order.

(c) Nothing in this Section shall be construed to diminish the inherent
authority of the courts to enforce their lawful orders through civil or
criminal contempt proceedings.

(d) A defendant who directed the actions of a third party to violate
this Section, under the principles of accountability set forth in
Article 5 of this Code, is guilty of violating this Section as if the
same had been personally done by the defendant, without regard to the
mental state of the third party acting at the direction of the
defendant.

(e) Sentence. A violation of a stalking no contact order is a Class A
misdemeanor for a first violation, and a Class 4 felony for a second or
subsequent violation.

(Source: P.A. 100-199, eff. 1-1-18 .)

\hypertarget{ilcs-512-4}{%
\subsection*{(720 ILCS 5/12-4)}\label{ilcs-512-4}}
\addcontentsline{toc}{subsection}{(720 ILCS 5/12-4)}

(This Section was renumbered as Section 12-3.05 by P.A. 96-1551.)

\hypertarget{sec.-12-4.-renumbered.}{%
\section*{Sec. 12-4. (Renumbered).}\label{sec.-12-4.-renumbered.}}
\addcontentsline{toc}{section}{Sec. 12-4. (Renumbered).}

\markright{Sec. 12-4. (Renumbered).}

(Source: P.A. 97-467, eff. 1-1-12. Renumbered by P.A. 96-1551, eff.
7-1-11.)

\hypertarget{ilcs-512-4.1-from-ch.-38-par.-12-4.1}{%
\subsection*{(720 ILCS 5/12-4.1) (from Ch. 38, par.
12-4.1)}\label{ilcs-512-4.1-from-ch.-38-par.-12-4.1}}
\addcontentsline{toc}{subsection}{(720 ILCS 5/12-4.1) (from Ch. 38, par.
12-4.1)}

\hypertarget{sec.-12-4.1.-repealed.}{%
\section*{Sec. 12-4.1. (Repealed).}\label{sec.-12-4.1.-repealed.}}
\addcontentsline{toc}{section}{Sec. 12-4.1. (Repealed).}

\markright{Sec. 12-4.1. (Repealed).}

(Source: P.A. 91-121, eff. 7-15-99. Repealed by P.A. 96-1551, eff.
7-1-11 .)

\hypertarget{ilcs-512-4.2-from-ch.-38-par.-12-4.2}{%
\subsection*{(720 ILCS 5/12-4.2) (from Ch. 38, par.
12-4.2)}\label{ilcs-512-4.2-from-ch.-38-par.-12-4.2}}
\addcontentsline{toc}{subsection}{(720 ILCS 5/12-4.2) (from Ch. 38, par.
12-4.2)}

\hypertarget{sec.-12-4.2.-repealed.}{%
\section*{Sec. 12-4.2. (Repealed).}\label{sec.-12-4.2.-repealed.}}
\addcontentsline{toc}{section}{Sec. 12-4.2. (Repealed).}

\markright{Sec. 12-4.2. (Repealed).}

(Source: P.A. 96-328, eff. 8-11-09. Repealed by P.A. 96-1551, eff.
7-1-11 .)

\hypertarget{ilcs-512-4.2-5}{%
\subsection*{(720 ILCS 5/12-4.2-5)}\label{ilcs-512-4.2-5}}
\addcontentsline{toc}{subsection}{(720 ILCS 5/12-4.2-5)}

\hypertarget{sec.-12-4.2-5.-repealed.}{%
\section*{Sec. 12-4.2-5. (Repealed).}\label{sec.-12-4.2-5.-repealed.}}
\addcontentsline{toc}{section}{Sec. 12-4.2-5. (Repealed).}

\markright{Sec. 12-4.2-5. (Repealed).}

(Source: P.A. 96-328, eff. 8-11-09. Repealed by P.A. 96-1551, eff.
7-1-11 .)

\hypertarget{ilcs-512-4.3-from-ch.-38-par.-12-4.3}{%
\subsection*{(720 ILCS 5/12-4.3) (from Ch. 38, par.
12-4.3)}\label{ilcs-512-4.3-from-ch.-38-par.-12-4.3}}
\addcontentsline{toc}{subsection}{(720 ILCS 5/12-4.3) (from Ch. 38, par.
12-4.3)}

\hypertarget{sec.-12-4.3.-repealed.}{%
\section*{Sec. 12-4.3. (Repealed).}\label{sec.-12-4.3.-repealed.}}
\addcontentsline{toc}{section}{Sec. 12-4.3. (Repealed).}

\markright{Sec. 12-4.3. (Repealed).}

(Source: P.A. 97-227, eff. 1-1-12. Repealed by P.A. 96-1551, eff.
7-1-11.)

\hypertarget{ilcs-512-4.4-from-ch.-38-par.-12-4.4}{%
\subsection*{(720 ILCS 5/12-4.4) (from Ch. 38, par.
12-4.4)}\label{ilcs-512-4.4-from-ch.-38-par.-12-4.4}}
\addcontentsline{toc}{subsection}{(720 ILCS 5/12-4.4) (from Ch. 38, par.
12-4.4)}

\hypertarget{sec.-12-4.4.-repealed.}{%
\section*{Sec. 12-4.4. (Repealed).}\label{sec.-12-4.4.-repealed.}}
\addcontentsline{toc}{section}{Sec. 12-4.4. (Repealed).}

\markright{Sec. 12-4.4. (Repealed).}

(Source: P.A. 84-1414. Repealed by P.A. 96-1551, eff. 7-1-11 .)

\hypertarget{ilcs-5art.-12-subdiv.-10-heading}{%
\subsection*{(720 ILCS 5/Art. 12, Subdiv. 10
heading)}\label{ilcs-5art.-12-subdiv.-10-heading}}
\addcontentsline{toc}{subsection}{(720 ILCS 5/Art. 12, Subdiv. 10
heading)}

SUBDIVISION 10.

ENDANGERMENT

(Source: P.A. 96-1551, eff. 7-1-11.)

\hypertarget{ilcs-512-4.4a}{%
\subsection*{(720 ILCS 5/12-4.4a)}\label{ilcs-512-4.4a}}
\addcontentsline{toc}{subsection}{(720 ILCS 5/12-4.4a)}

\hypertarget{sec.-12-4.4a.-abuse-or-criminal-neglect-of-a-long-term-care-facility-resident-criminal-abuse-or-neglect-of-an-elderly-person-or-person-with-a-disability.}{%
\section*{Sec. 12-4.4a. Abuse or criminal neglect of a long term care
facility resident; criminal abuse or neglect of an elderly person or
person with a
disability.}\label{sec.-12-4.4a.-abuse-or-criminal-neglect-of-a-long-term-care-facility-resident-criminal-abuse-or-neglect-of-an-elderly-person-or-person-with-a-disability.}}
\addcontentsline{toc}{section}{Sec. 12-4.4a. Abuse or criminal neglect
of a long term care facility resident; criminal abuse or neglect of an
elderly person or person with a disability.}

\markright{Sec. 12-4.4a. Abuse or criminal neglect of a long term care
facility resident; criminal abuse or neglect of an elderly person or
person with a disability.}

(a) Abuse or criminal neglect of a long term care facility resident.

(1) A person or an owner or licensee commits abuse of a long term care
facility resident when he or she knowingly causes any physical or mental
injury to, or commits any sexual offense in this Code against, a
resident.

(2) A person or an owner or licensee commits criminal neglect of a long
term care facility resident when he or she recklessly:

(A) performs acts that cause a resident's life to be endangered, health
to be injured, or pre-existing physical or mental condition to
deteriorate, or that create the substantial likelihood that an elderly
person's or person with a disability's life will be endangered, health
will be injured, or pre-existing physical or mental condition will
deteriorate;

(B) fails to perform acts that he or she knows or reasonably should know
are necessary to maintain or preserve the life or health of a resident,
and that failure causes the resident's life to be endangered, health to
be injured, or pre-existing physical or mental condition to deteriorate,
or that create the substantial likelihood that an elderly person's or
person with a disability's life will be endangered, health will be
injured, or pre-existing physical or mental condition will deteriorate;
or

(C) abandons a resident.

(3) A person or an owner or licensee commits neglect of a long term care
facility resident when he or she negligently fails to provide adequate
medical care, personal care, or maintenance to the resident which
results in physical or mental injury or deterioration of the resident's
physical or mental condition. An owner or licensee is guilty under this
subdivision (a)(3), however, only if the owner or licensee failed to
exercise reasonable care in the hiring, training, supervising, or
providing of staff or other related routine administrative
responsibilities.

(b) Criminal abuse or neglect of an elderly person or person with a
disability.

(1) A caregiver commits criminal abuse or neglect of an elderly person
or person with a disability when he or she knowingly does any of the
following:

(A) performs acts that cause the person's life to be endangered, health
to be injured, or pre-existing physical or mental condition to
deteriorate;

(B) fails to perform acts that he or she knows or reasonably should know
are necessary to maintain or preserve the life or health of the person,
and that failure causes the person's life to be endangered, health to be
injured, or pre-existing physical or mental condition to deteriorate;

(C) abandons the person;

(D) physically abuses, harasses, intimidates, or interferes with the
personal liberty of the person; or

(E) exposes the person to willful deprivation.

(2) It is not a defense to criminal abuse or neglect of an elderly
person or person with a disability that the caregiver reasonably
believed that the victim was not an elderly person or person with a
disability.

(c) Offense not applicable.

(1) Nothing in this Section applies to a physician licensed to practice
medicine in all its branches or a duly licensed nurse providing care
within the scope of his or her professional judgment and within the
accepted standards of care within the community.

(2) Nothing in this Section imposes criminal liability on a caregiver
who made a good faith effort to provide for the health and personal care
of an elderly person or person with a disability, but through no fault
of his or her own was unable to provide such care.

(3) Nothing in this Section applies to the medical supervision,
regulation, or control of the remedial care or treatment of residents in
a long term care facility conducted for those who rely upon treatment by
prayer or spiritual means in accordance with the creed or tenets of any
well-recognized church or religious denomination as described in Section
3-803 of the Nursing Home Care Act, Section 1-102 of the Specialized
Mental Health Rehabilitation Act of 2013, Section 3-803 of the ID/DD
Community Care Act, or Section 3-803 of the MC/DD Act.

(4) Nothing in this Section prohibits a caregiver from providing
treatment to an elderly person or person with a disability by spiritual
means through prayer alone and care consistent therewith in lieu of
medical care and treatment in accordance with the tenets and practices
of any church or religious denomination of which the elderly person or
person with a disability is a member.

(5) Nothing in this Section limits the remedies available to the victim
under the Illinois Domestic Violence Act of 1986.

(d) Sentence.

(1) Long term care facility. Abuse of a long term care facility resident
is a Class 3 felony. Criminal neglect of a long term care facility
resident is a Class 4 felony, unless it results in the resident's death
in which case it is a Class 3 felony. Neglect of a long term care
facility resident is a petty offense.

(2) Caregiver. Criminal abuse or neglect of an elderly person or person
with a disability is a Class 3 felony, unless it results in the person's
death in which case it is a Class 2 felony, and if imprisonment is
imposed it shall be for a minimum term of 3 years and a maximum term of
14 years.

(e) Definitions. For the purposes of this Section:

``Abandon'' means to desert or knowingly forsake a resident or an
elderly person or person with a disability under circumstances in which
a reasonable person would continue to provide care and custody.

``Caregiver'' means a person who has a duty to provide for an elderly
person or person with a disability's health and personal care, at the
elderly person or person with a disability's place of residence,
including, but not limited to, food and nutrition, shelter, hygiene,
prescribed medication, and medical care and treatment, and includes any
of the following:

(1) A parent, spouse, adult child, or other relative by blood or
marriage who resides with or resides in the same building with or
regularly visits the elderly person or person with a disability, knows
or reasonably should know of such person's physical or mental
impairment, and knows or reasonably should know that such person is
unable to adequately provide for his or her own health and personal
care.

(2) A person who is employed by the elderly person or person with a
disability or by another to reside with or regularly visit the elderly
person or person with a disability and provide for such person's health
and personal care.

(3) A person who has agreed for consideration to reside with or
regularly visit the elderly person or person with a disability and
provide for such person's health and personal care.

(4) A person who has been appointed by a private or public agency or by
a court of competent jurisdiction to provide for the elderly person or
person with a disability's health and personal care.

``Caregiver'' does not include a long-term care facility licensed or
certified under the Nursing Home Care Act or a facility licensed or
certified under the ID/DD Community Care Act, the MC/DD Act, or the
Specialized Mental Health Rehabilitation Act of 2013, or any
administrative, medical, or other personnel of such a facility, or a
health care provider who is licensed under the Medical Practice Act of
1987 and renders care in the ordinary course of his or her profession.

``Elderly person'' means a person 60 years of age or older who is
incapable of adequately providing for his or her own health and personal
care.

``Licensee'' means the individual or entity licensed to operate a
facility under the Nursing Home Care Act, the Specialized Mental Health
Rehabilitation Act of 2013, the ID/DD Community Care Act, the MC/DD Act,
or the Assisted Living and Shared Housing Act.

``Long term care facility'' means a private home, institution, building,
residence, or other place, whether operated for profit or not, or a
county home for the infirm and chronically ill operated pursuant to
Division 5-21 or 5-22 of the Counties Code, or any similar institution
operated by the State of Illinois or a political subdivision thereof,
which provides, through its ownership or management, personal care,
sheltered care, or nursing for 3 or more persons not related to the
owner by blood or marriage. The term also includes skilled nursing
facilities and intermediate care facilities as defined in Titles XVIII
and XIX of the federal Social Security Act and assisted living
establishments and shared housing establishments licensed under the
Assisted Living and Shared Housing Act.

``Owner'' means the owner of a long term care facility as provided in
the Nursing Home Care Act, the owner of a facility as provided under the
Specialized Mental Health Rehabilitation Act of 2013, the owner of a
facility as provided in the ID/DD Community Care Act, the owner of a
facility as provided in the MC/DD Act, or the owner of an assisted
living or shared housing establishment as provided in the Assisted
Living and Shared Housing Act.

``Person with a disability'' means a person who suffers from a permanent
physical or mental impairment, resulting from disease, injury,
functional disorder, or congenital condition, which renders the person
incapable of adequately providing for his or her own health and personal
care.

``Resident'' means a person residing in a long term care facility.

``Willful deprivation'' has the meaning ascribed to it in paragraph (15)
of Section 103 of the Illinois Domestic Violence Act of 1986.

(Source: P.A. 98-104, eff. 7-22-13; 99-180, eff. 7-29-15; 99-642, eff.
7-28-16.)

\hypertarget{ilcs-512-4.5-from-ch.-38-par.-12-4.5}{%
\subsection*{(720 ILCS 5/12-4.5) (from Ch. 38, par.
12-4.5)}\label{ilcs-512-4.5-from-ch.-38-par.-12-4.5}}
\addcontentsline{toc}{subsection}{(720 ILCS 5/12-4.5) (from Ch. 38, par.
12-4.5)}

\hypertarget{sec.-12-4.5.-tampering-with-food-drugs-or-cosmetics.}{%
\section*{Sec. 12-4.5. Tampering with food, drugs or
cosmetics.}\label{sec.-12-4.5.-tampering-with-food-drugs-or-cosmetics.}}
\addcontentsline{toc}{section}{Sec. 12-4.5. Tampering with food, drugs
or cosmetics.}

\markright{Sec. 12-4.5. Tampering with food, drugs or cosmetics.}

(a) A person who knowingly puts any substance capable of causing death
or great bodily harm to a human being into any food, drug or cosmetic
offered for sale or consumption commits tampering with food, drugs or
cosmetics.

(b) Sentence. Tampering with food, drugs or cosmetics is a Class 2
felony.

(Source: P.A. 96-1551, eff. 7-1-11 .)

\hypertarget{ilcs-512-4.6-from-ch.-38-par.-12-4.6}{%
\subsection*{(720 ILCS 5/12-4.6) (from Ch. 38, par.
12-4.6)}\label{ilcs-512-4.6-from-ch.-38-par.-12-4.6}}
\addcontentsline{toc}{subsection}{(720 ILCS 5/12-4.6) (from Ch. 38, par.
12-4.6)}

\hypertarget{sec.-12-4.6.-repealed.}{%
\section*{Sec. 12-4.6. (Repealed).}\label{sec.-12-4.6.-repealed.}}
\addcontentsline{toc}{section}{Sec. 12-4.6. (Repealed).}

\markright{Sec. 12-4.6. (Repealed).}

(Source: P.A. 85-1177. Repealed by P.A. 96-1551, eff. 7-1-11 .)

\hypertarget{ilcs-512-4.7-from-ch.-38-par.-12-4.7}{%
\subsection*{(720 ILCS 5/12-4.7) (from Ch. 38, par.
12-4.7)}\label{ilcs-512-4.7-from-ch.-38-par.-12-4.7}}
\addcontentsline{toc}{subsection}{(720 ILCS 5/12-4.7) (from Ch. 38, par.
12-4.7)}

\hypertarget{sec.-12-4.7.-repealed.}{%
\section*{Sec. 12-4.7. (Repealed).}\label{sec.-12-4.7.-repealed.}}
\addcontentsline{toc}{section}{Sec. 12-4.7. (Repealed).}

\markright{Sec. 12-4.7. (Repealed).}

(Source: P.A. 92-256, eff. 1-1-02. Repealed by P.A. 96-1551, eff. 7-1-11
.)

\hypertarget{ilcs-512-4.8}{%
\subsection*{(720 ILCS 5/12-4.8)}\label{ilcs-512-4.8}}
\addcontentsline{toc}{subsection}{(720 ILCS 5/12-4.8)}

\hypertarget{sec.-12-4.8.-repealed.}{%
\section*{Sec. 12-4.8. (Repealed).}\label{sec.-12-4.8.-repealed.}}
\addcontentsline{toc}{section}{Sec. 12-4.8. (Repealed).}

\markright{Sec. 12-4.8. (Repealed).}

(Source: P.A. 89-234, eff. 1-1-96. Repealed by P.A. 96-1551, eff. 7-1-11
.)

\hypertarget{ilcs-512-4.9}{%
\subsection*{(720 ILCS 5/12-4.9)}\label{ilcs-512-4.9}}
\addcontentsline{toc}{subsection}{(720 ILCS 5/12-4.9)}

(This Section was renumbered as Section 12C-45 by P.A. 97-1109.)

\hypertarget{sec.-12-4.9.-renumbered.}{%
\section*{Sec. 12-4.9. (Renumbered).}\label{sec.-12-4.9.-renumbered.}}
\addcontentsline{toc}{section}{Sec. 12-4.9. (Renumbered).}

\markright{Sec. 12-4.9. (Renumbered).}

(Source: P.A. 89-632, eff. 1-1-97. Renumbered by P.A. 97-1109, eff.
1-1-13.)

\hypertarget{ilcs-512-4.10}{%
\subsection*{(720 ILCS 5/12-4.10)}\label{ilcs-512-4.10}}
\addcontentsline{toc}{subsection}{(720 ILCS 5/12-4.10)}

\hypertarget{sec.-12-4.10.-repealed.}{%
\section*{Sec. 12-4.10. (Repealed).}\label{sec.-12-4.10.-repealed.}}
\addcontentsline{toc}{section}{Sec. 12-4.10. (Repealed).}

\markright{Sec. 12-4.10. (Repealed).}

(Source: P.A. 95-331, eff. 8-21-07. Repealed by P.A. 94-556, eff.
9-11-05.)

\hypertarget{ilcs-512-4.11}{%
\subsection*{(720 ILCS 5/12-4.11)}\label{ilcs-512-4.11}}
\addcontentsline{toc}{subsection}{(720 ILCS 5/12-4.11)}

\hypertarget{sec.-12-4.11.-repealed.}{%
\section*{Sec. 12-4.11. (Repealed).}\label{sec.-12-4.11.-repealed.}}
\addcontentsline{toc}{section}{Sec. 12-4.11. (Repealed).}

\markright{Sec. 12-4.11. (Repealed).}

(Source: P.A. 93-340, eff. 7-24-03. Repealed by P.A. 94-556, eff.
9-11-05.)

\hypertarget{ilcs-512-4.12}{%
\subsection*{(720 ILCS 5/12-4.12)}\label{ilcs-512-4.12}}
\addcontentsline{toc}{subsection}{(720 ILCS 5/12-4.12)}

\hypertarget{sec.-12-4.12.-repealed.}{%
\section*{Sec. 12-4.12. (Repealed).}\label{sec.-12-4.12.-repealed.}}
\addcontentsline{toc}{section}{Sec. 12-4.12. (Repealed).}

\markright{Sec. 12-4.12. (Repealed).}

(Source: P.A. 95-331, eff. 8-21-07. Repealed by P.A. 94-556, eff.
9-11-05.)

\hypertarget{ilcs-512-5-from-ch.-38-par.-12-5}{%
\subsection*{(720 ILCS 5/12-5) (from Ch. 38, par.
12-5)}\label{ilcs-512-5-from-ch.-38-par.-12-5}}
\addcontentsline{toc}{subsection}{(720 ILCS 5/12-5) (from Ch. 38, par.
12-5)}

\hypertarget{sec.-12-5.-reckless-conduct.}{%
\section*{Sec. 12-5. Reckless
conduct.}\label{sec.-12-5.-reckless-conduct.}}
\addcontentsline{toc}{section}{Sec. 12-5. Reckless conduct.}

\markright{Sec. 12-5. Reckless conduct.}

(a) A person commits reckless conduct when he or she, by any means
lawful or unlawful, recklessly performs an act or acts that:

(1) cause bodily harm to or endanger the safety of another person; or

(2) cause great bodily harm or permanent disability or disfigurement to
another person.

(b) Sentence.

Reckless conduct under subdivision (a)(1) is a Class A misdemeanor.
Reckless conduct under subdivision (a)(2) is a Class 4 felony.

(Source: P.A. 96-1551, eff. 7-1-11 .)

\hypertarget{ilcs-512-5.01}{%
\subsection*{(720 ILCS 5/12-5.01)}\label{ilcs-512-5.01}}
\addcontentsline{toc}{subsection}{(720 ILCS 5/12-5.01)}

\hypertarget{sec.-12-5.01.-repealed.}{%
\section*{Sec. 12-5.01. (Repealed).}\label{sec.-12-5.01.-repealed.}}
\addcontentsline{toc}{section}{Sec. 12-5.01. (Repealed).}

\markright{Sec. 12-5.01. (Repealed).}

(Source: P.A. 97-1046, eff. 8-21-12. Repealed by P.A. 102-168, eff.
7-27-21.)

\hypertarget{ilcs-512-5.02}{%
\subsection*{(720 ILCS 5/12-5.02)}\label{ilcs-512-5.02}}
\addcontentsline{toc}{subsection}{(720 ILCS 5/12-5.02)}

(was 720 ILCS 5/12-2.5)

\hypertarget{sec.-12-5.02.-vehicular-endangerment.}{%
\section*{Sec. 12-5.02. Vehicular
endangerment.}\label{sec.-12-5.02.-vehicular-endangerment.}}
\addcontentsline{toc}{section}{Sec. 12-5.02. Vehicular endangerment.}

\markright{Sec. 12-5.02. Vehicular endangerment.}

(a) A person commits vehicular endangerment when he or she strikes a
motor vehicle by causing an object to fall from an overpass or other
elevated location in the direction of a moving motor vehicle with the
intent to strike a motor vehicle while it is traveling upon a highway in
this State.

(b) Sentence. Vehicular endangerment is a Class 2 felony, unless death
results, in which case vehicular endangerment is a Class 1 felony.

(c) Definitions. For purposes of this Section:

``Elevated location'' means a bridge, overpass, highway ramp, building,
artificial structure, hill, mound, or natural elevation above or
adjacent to and above a highway.

``Object'' means any object or substance that by its size, weight, or
consistency is likely to cause great bodily harm to any occupant of a
motor vehicle.

``Overpass'' means any structure that passes over a highway.

``Motor vehicle'' and ``highway'' have the meanings as defined in the
Illinois Vehicle Code.

(Source: P.A. 99-656, eff. 1-1-17 .)

\hypertarget{ilcs-512-5.1-from-ch.-38-par.-12-5.1}{%
\subsection*{(720 ILCS 5/12-5.1) (from Ch. 38, par.
12-5.1)}\label{ilcs-512-5.1-from-ch.-38-par.-12-5.1}}
\addcontentsline{toc}{subsection}{(720 ILCS 5/12-5.1) (from Ch. 38, par.
12-5.1)}

\hypertarget{sec.-12-5.1.-criminal-housing-management.}{%
\section*{Sec. 12-5.1. Criminal housing
management.}\label{sec.-12-5.1.-criminal-housing-management.}}
\addcontentsline{toc}{section}{Sec. 12-5.1. Criminal housing
management.}

\markright{Sec. 12-5.1. Criminal housing management.}

(a) A person commits criminal housing management when, having personal
management or control of residential real estate, whether as a legal or
equitable owner or as a managing agent or otherwise, he or she
recklessly permits the physical condition or facilities of the
residential real estate to become or remain in any condition which
endangers the health or safety of a person other than the defendant.

(b) Sentence.

Criminal housing management is a Class A misdemeanor, and a subsequent
conviction is a Class 4 felony.

(Source: P.A. 96-1551, eff. 7-1-11 .)

\hypertarget{ilcs-512-5.1a}{%
\subsection*{(720 ILCS 5/12-5.1a)}\label{ilcs-512-5.1a}}
\addcontentsline{toc}{subsection}{(720 ILCS 5/12-5.1a)}

(was 720 ILCS 5/12-5.15)

\hypertarget{sec.-12-5.1a.-aggravated-criminal-housing-management.}{%
\section*{Sec. 12-5.1a. Aggravated criminal housing
management.}\label{sec.-12-5.1a.-aggravated-criminal-housing-management.}}
\addcontentsline{toc}{section}{Sec. 12-5.1a. Aggravated criminal housing
management.}

\markright{Sec. 12-5.1a. Aggravated criminal housing management.}

(a) A person commits aggravated criminal housing management when he or
she commits criminal housing management and:

(1) the condition endangering the health or safety of a person other
than the defendant is determined to be a contributing factor in the
death of that person; and

(2) the person recklessly conceals or attempts to conceal the condition
that endangered the health or safety of the person other than the
defendant that is found to be a contributing factor in that death.

(b) Sentence. Aggravated criminal housing management is a Class 4
felony.

(Source: P.A. 96-1551, eff. 7-1-11 .)

\hypertarget{ilcs-512-5.2-from-ch.-38-par.-12-5.2}{%
\subsection*{(720 ILCS 5/12-5.2) (from Ch. 38, par.
12-5.2)}\label{ilcs-512-5.2-from-ch.-38-par.-12-5.2}}
\addcontentsline{toc}{subsection}{(720 ILCS 5/12-5.2) (from Ch. 38, par.
12-5.2)}

\hypertarget{sec.-12-5.2.-injunction-in-connection-with-criminal-housing-management-or-aggravated-criminal-housing-management.}{%
\section*{Sec. 12-5.2. Injunction in connection with criminal housing
management or aggravated criminal housing
management.}\label{sec.-12-5.2.-injunction-in-connection-with-criminal-housing-management-or-aggravated-criminal-housing-management.}}
\addcontentsline{toc}{section}{Sec. 12-5.2. Injunction in connection
with criminal housing management or aggravated criminal housing
management.}

\markright{Sec. 12-5.2. Injunction in connection with criminal housing
management or aggravated criminal housing management.}

(a) In addition to any other remedies, the State's Attorney of the
county where the residential property which endangers the health or
safety of any person exists is authorized to file a complaint and apply
to the circuit court for a temporary restraining order, and such circuit
court shall upon hearing grant a temporary restraining order or a
preliminary or permanent injunction, without bond, restraining any
person who owns, manages, or has any equitable interest in the property,
from collecting, receiving or benefiting from any rents or other monies
available from the property, so long as the property remains in a
condition which endangers the health or safety of any person.

(b) The court may order any rents or other monies owed to be paid into
an escrow account. The funds are to be paid out of the escrow account
only to satisfy the reasonable cost of necessary repairs of the property
which had been incurred or will be incurred in ameliorating the
condition of the property as described in subsection (a), payment of
delinquent real estate taxes on the property or payment of other legal
debts relating to the property. The court may order that funds remain in
escrow for a reasonable time after the completion of all necessary
repairs to assure continued upkeep of the property and satisfaction of
other outstanding legal debts of the property.

(c) The owner shall be responsible for contracting to have necessary
repairs completed and shall be required to submit all bills, together
with certificates of completion, to the manager of the escrow account
within 30 days after their receipt by the owner.

(d) In contracting for any repairs required pursuant to this Section the
owner of the property shall enter into a contract only after receiving
bids from at least 3 independent contractors capable of making the
necessary repairs. If the owner does not contract for the repairs with
the lowest bidder, he shall file an affidavit with the court explaining
why the lowest bid was not acceptable. At no time, under the provisions
of this Section, shall the owner contract with anyone who is not a
licensed contractor, except that a contractor need not be licensed if
neither the State nor the county, township, or municipality where the
residential real estate is located requires that the contractor be
licensed. The court may order release of those funds in the escrow
account that are in excess of the monies that the court determines to
its satisfaction are needed to correct the condition of the property as
described in subsection (a).

For the purposes of this Section, ``licensed contractor'' means: (i) a
contractor licensed by the State, if the State requires the licensure of
the contractor; or (ii) a contractor licensed by the county, township,
or municipality where the residential real estate is located, if that
jurisdiction requires the licensure of the contractor.

(e) The Clerk of the Circuit Court shall maintain a separate trust
account entitled ``Property Improvement Trust Account'', which shall
serve as the depository for the escrowed funds prescribed by this
Section. The Clerk of the Court shall be responsible for the receipt,
disbursement, monitoring and maintenance of all funds entrusted to this
account, and shall provide to the court a quarterly accounting of the
activities for any property, with funds in such account, unless the
court orders accountings on a more frequent basis.

The Clerk of the Circuit Court shall promulgate rules and procedures to
administer the provisions of this Act.

(f) Nothing in this Section shall in any way be construed to limit or
alter any existing liability incurred, or to be incurred, by the owner
or manager except as expressly provided in this Act. Nor shall anything
in this Section be construed to create any liability on behalf of the
Clerk of the Court, the State's Attorney's office or any other
governmental agency involved in this action.

Nor shall anything in this Section be construed to authorize tenants to
refrain from paying rent.

(g) Costs. As part of the costs of an action under this Section, the
court shall assess a reasonable fee against the defendant to be paid to
the Clerk of the Circuit Court. This amount is to be used solely for the
maintenance of the Property Improvement Trust Account. No money obtained
directly or indirectly from the property subject to the case may be used
to satisfy this cost.

(h) The municipal building department or other entity responsible for
inspection of property and the enforcement of such local requirements
shall, within 5 business days of a request by the State's Attorney,
provide all documents requested, which shall include, but not be limited
to, all records of inspections, permits and other information relating
to any property.

(Source: P.A. 96-1551, eff. 7-1-11 .)

\hypertarget{ilcs-512-5.3}{%
\subsection*{(720 ILCS 5/12-5.3)}\label{ilcs-512-5.3}}
\addcontentsline{toc}{subsection}{(720 ILCS 5/12-5.3)}

(was 720 ILCS 5/12-2.6)

\hypertarget{sec.-12-5.3.-use-of-a-dangerous-place-for-the-commission-of-a-controlled-substance-or-cannabis-offense.}{%
\section*{Sec. 12-5.3. Use of a dangerous place for the commission of a
controlled substance or cannabis
offense.}\label{sec.-12-5.3.-use-of-a-dangerous-place-for-the-commission-of-a-controlled-substance-or-cannabis-offense.}}
\addcontentsline{toc}{section}{Sec. 12-5.3. Use of a dangerous place for
the commission of a controlled substance or cannabis offense.}

\markright{Sec. 12-5.3. Use of a dangerous place for the commission of a
controlled substance or cannabis offense.}

(a) A person commits use of a dangerous place for the commission of a
controlled substance or cannabis offense when that person knowingly
exercises control over any place with the intent to use that place to
manufacture, produce, deliver, or possess with intent to deliver a
controlled or counterfeit substance or controlled substance analog in
violation of Section 401 of the Illinois Controlled Substances Act or to
manufacture, produce, deliver, or possess with intent to deliver
cannabis in violation of Section 5, 5.1, 5.2, 7, or 8 of the Cannabis
Control Act and:

(1) the place, by virtue of the presence of the substance or substances
used or intended to be used to manufacture a controlled or counterfeit
substance, controlled substance analog, or cannabis, presents a
substantial risk of injury to any person from fire, explosion, or
exposure to toxic or noxious chemicals or gas; or

(2) the place used or intended to be used to manufacture, produce,
deliver, or possess with intent to deliver a controlled or counterfeit
substance, controlled substance analog, or cannabis has located within
it or surrounding it devices, weapons, chemicals, or explosives
designed, hidden, or arranged in a manner that would cause a person to
be exposed to a substantial risk of great bodily harm.

(b) It may be inferred that a place was intended to be used to
manufacture a controlled or counterfeit substance or controlled
substance analog if a substance containing a controlled or counterfeit
substance or controlled substance analog or a substance containing a
chemical important to the manufacture of a controlled or counterfeit
substance or controlled substance analog is found at the place of the
alleged illegal controlled substance manufacturing in close proximity to
equipment or a chemical used for facilitating the manufacture of the
controlled or counterfeit substance or controlled substance analog that
is alleged to have been intended to be manufactured.

(c) As used in this Section, ``place'' means a premises, conveyance, or
location that offers seclusion, shelter, means, or facilitation for
manufacturing, producing, possessing, or possessing with intent to
deliver a controlled or counterfeit substance, controlled substance
analog, or cannabis.

(d) Use of a dangerous place for the commission of a controlled
substance or cannabis offense is a Class 1 felony.

(Source: P.A. 96-1551, eff. 7-1-11 .)

\hypertarget{ilcs-512-5.5}{%
\subsection*{(720 ILCS 5/12-5.5)}\label{ilcs-512-5.5}}
\addcontentsline{toc}{subsection}{(720 ILCS 5/12-5.5)}

\hypertarget{sec.-12-5.5.-common-carrier-recklessness.}{%
\section*{Sec. 12-5.5. Common carrier
recklessness.}\label{sec.-12-5.5.-common-carrier-recklessness.}}
\addcontentsline{toc}{section}{Sec. 12-5.5. Common carrier
recklessness.}

\markright{Sec. 12-5.5. Common carrier recklessness.}

(a) A person commits common carrier recklessness when he or she, having
personal management or control of or over a public conveyance used for
the common carriage of persons, recklessly endangers the safety of
others.

(b) Sentence. Common carrier recklessness is a Class 4 felony.

(Source: P.A. 96-1551, eff. 7-1-11 .)

\hypertarget{ilcs-512-5.15}{%
\subsection*{(720 ILCS 5/12-5.15)}\label{ilcs-512-5.15}}
\addcontentsline{toc}{subsection}{(720 ILCS 5/12-5.15)}

(This Section was renumbered as Section 12-5.1a by P.A. 96-1551.)

\hypertarget{sec.-12-5.15.-renumbered.}{%
\section*{Sec. 12-5.15. (Renumbered).}\label{sec.-12-5.15.-renumbered.}}
\addcontentsline{toc}{section}{Sec. 12-5.15. (Renumbered).}

\markright{Sec. 12-5.15. (Renumbered).}

(Source: P.A. 93-852, eff. 8-2-04. Renumbered by P.A. 96-1551, eff.
7-1-11 .)

\hypertarget{ilcs-5art.-12-subdiv.-15-heading}{%
\subsection*{(720 ILCS 5/Art. 12, Subdiv. 15
heading)}\label{ilcs-5art.-12-subdiv.-15-heading}}
\addcontentsline{toc}{subsection}{(720 ILCS 5/Art. 12, Subdiv. 15
heading)}

SUBDIVISION 15.

INTIMIDATION

(Source: P.A. 96-1551, eff. 7-1-11.)

\hypertarget{ilcs-512-6-from-ch.-38-par.-12-6}{%
\subsection*{(720 ILCS 5/12-6) (from Ch. 38, par.
12-6)}\label{ilcs-512-6-from-ch.-38-par.-12-6}}
\addcontentsline{toc}{subsection}{(720 ILCS 5/12-6) (from Ch. 38, par.
12-6)}

\hypertarget{sec.-12-6.-intimidation.}{%
\section*{Sec. 12-6. Intimidation.}\label{sec.-12-6.-intimidation.}}
\addcontentsline{toc}{section}{Sec. 12-6. Intimidation.}

\markright{Sec. 12-6. Intimidation.}

(a) A person commits intimidation when, with intent to cause another to
perform or to omit the performance of any act, he or she communicates to
another, directly or indirectly by any means, a threat to perform
without lawful authority any of the following acts:

(1) Inflict physical harm on the person threatened or any other person
or on property; or

(2) Subject any person to physical confinement or restraint; or

(3) Commit a felony or Class A misdemeanor; or

(4) Accuse any person of an offense; or

(5) Expose any person to hatred, contempt or ridicule; or

(6) Take action as a public official against anyone or anything, or
withhold official action, or cause such action or withholding; or

(7) Bring about or continue a strike, boycott or other collective
action.

(b) Sentence.

Intimidation is a Class 3 felony for which an offender may be sentenced
to a term of imprisonment of not less than 2 years and not more than 10
years.

(Source: P.A. 96-1551, eff. 7-1-11 .)

\hypertarget{ilcs-512-6.1-from-ch.-38-par.-12-6.1}{%
\subsection*{(720 ILCS 5/12-6.1) (from Ch. 38, par.
12-6.1)}\label{ilcs-512-6.1-from-ch.-38-par.-12-6.1}}
\addcontentsline{toc}{subsection}{(720 ILCS 5/12-6.1) (from Ch. 38, par.
12-6.1)}

(This Section was renumbered as Section 12-6.5 by P.A. 96-1551.)

\hypertarget{sec.-12-6.1.-renumbered.}{%
\section*{Sec. 12-6.1. (Renumbered).}\label{sec.-12-6.1.-renumbered.}}
\addcontentsline{toc}{section}{Sec. 12-6.1. (Renumbered).}

\markright{Sec. 12-6.1. (Renumbered).}

(Source: P.A. 91-696, eff. 4-13-00. Renumbered by P.A. 96-1551, eff.
7-1-11 .)

\hypertarget{ilcs-512-6.2}{%
\subsection*{(720 ILCS 5/12-6.2)}\label{ilcs-512-6.2}}
\addcontentsline{toc}{subsection}{(720 ILCS 5/12-6.2)}

\hypertarget{sec.-12-6.2.-aggravated-intimidation.}{%
\section*{Sec. 12-6.2. Aggravated
intimidation.}\label{sec.-12-6.2.-aggravated-intimidation.}}
\addcontentsline{toc}{section}{Sec. 12-6.2. Aggravated intimidation.}

\markright{Sec. 12-6.2. Aggravated intimidation.}

(a) A person commits aggravated intimidation when he or she commits
intimidation and:

(1) the person committed the offense in furtherance of the activities of
an organized gang or because of the person's membership in or allegiance
to an organized gang; or

(2) the offense is committed with the intent to prevent any person from
becoming a community policing volunteer; or

(3) the following conditions are met:

(A) the person knew that the victim was a peace officer, a correctional
institution employee, a fireman, a community policing volunteer, or a
civilian reporting information regarding a forcible felony to a law
enforcement agency; and

(B) the offense was committed:

(i) while the victim was engaged in the execution of his or her official
duties; or

(ii) to prevent the victim from performing his or her official duties;

(iii) in retaliation for the victim's performance of his or her official
duties;

(iv) by reason of any person's activity as a community policing
volunteer; or

(v) because the person reported information regarding a forcible felony
to a law enforcement agency.

(b) Sentence. Aggravated intimidation as defined in paragraph (a)(1) is
a Class 1 felony. Aggravated intimidation as defined in paragraph (a)(2)
or (a)(3) is a Class 2 felony for which the offender may be sentenced to
a term of imprisonment of not less than 3 years nor more than 14 years.

(c) (Blank).

(Source: P.A. 96-1551, eff. 7-1-11; 97-162, eff. 1-1-12; 97-1109, eff.
1-1-13.)

\hypertarget{ilcs-512-6.3}{%
\subsection*{(720 ILCS 5/12-6.3)}\label{ilcs-512-6.3}}
\addcontentsline{toc}{subsection}{(720 ILCS 5/12-6.3)}

(This Section was renumbered as Section 12-3.5 by P.A. 96-1551.)

\hypertarget{sec.-12-6.3.-renumbered.}{%
\section*{Sec. 12-6.3. (Renumbered).}\label{sec.-12-6.3.-renumbered.}}
\addcontentsline{toc}{section}{Sec. 12-6.3. (Renumbered).}

\markright{Sec. 12-6.3. (Renumbered).}

(Source: P.A. 90-118, eff. 1-1-98. Renumbered by P.A. 96-1551, eff.
7-1-11 .)

\hypertarget{ilcs-512-6.4}{%
\subsection*{(720 ILCS 5/12-6.4)}\label{ilcs-512-6.4}}
\addcontentsline{toc}{subsection}{(720 ILCS 5/12-6.4)}

\hypertarget{sec.-12-6.4.-criminal-street-gang-recruitment-on-school-grounds-or-public-property-adjacent-to-school-grounds-and-criminal-street-gang-recruitment-of-a-minor.}{%
\section*{Sec. 12-6.4. Criminal street gang recruitment on school
grounds or public property adjacent to school grounds and criminal
street gang recruitment of a
minor.}\label{sec.-12-6.4.-criminal-street-gang-recruitment-on-school-grounds-or-public-property-adjacent-to-school-grounds-and-criminal-street-gang-recruitment-of-a-minor.}}
\addcontentsline{toc}{section}{Sec. 12-6.4. Criminal street gang
recruitment on school grounds or public property adjacent to school
grounds and criminal street gang recruitment of a minor.}

\markright{Sec. 12-6.4. Criminal street gang recruitment on school
grounds or public property adjacent to school grounds and criminal
street gang recruitment of a minor.}

(a) A person commits criminal street gang recruitment on school grounds
or public property adjacent to school grounds when on school grounds or
public property adjacent to school grounds, he or she knowingly
threatens the use of physical force to coerce, solicit, recruit, or
induce another person to join or remain a member of a criminal street
gang, or conspires to do so.

(a-5) A person commits the offense of criminal street gang recruitment
of a minor when he or she threatens the use of physical force to coerce,
solicit, recruit, or induce another person to join or remain a member of
a criminal street gang, or conspires to do so, whether or not such
threat is communicated in person, by means of the Internet, or by means
of a telecommunications device.

(b) Sentence. Criminal street gang recruitment on school grounds or
public property adjacent to school grounds is a Class 1 felony and
criminal street gang recruitment of a minor is a Class 1 felony.

(c) In this Section:

``School grounds'' means the building or buildings or real property
comprising a public or private elementary or secondary school, community
college, college, or university and includes a school yard, school
playing field, or school playground.

``Minor'' means any person under 18 years of age.

``Internet'' means an interactive computer service or system or an
information service, system, or access software provider that provides
or enables computer access by multiple users to a computer server, and
includes, but is not limited to, an information service, system, or
access software provider that provides access to a network system
commonly known as the Internet, or any comparable system or service and
also includes, but is not limited to, a World Wide Web page, newsgroup,
message board, mailing list, or chat area on any interactive computer
service or system or other online service.

``Telecommunications device'' means a device that is capable of
receiving or transmitting speech, data, signals, text, images, sounds,
codes, or other information including, but not limited to, paging
devices, telephones, and cellular and mobile telephones.

(Source: P.A. 96-199, eff. 1-1-10; 96-1551, eff. 7-1-11 .)

\hypertarget{ilcs-512-6.5}{%
\subsection*{(720 ILCS 5/12-6.5)}\label{ilcs-512-6.5}}
\addcontentsline{toc}{subsection}{(720 ILCS 5/12-6.5)}

(was 720 ILCS 5/12-6.1)

\hypertarget{sec.-12-6.5.-compelling-organization-membership-of-persons.}{%
\section*{Sec. 12-6.5. Compelling organization membership of
persons.}\label{sec.-12-6.5.-compelling-organization-membership-of-persons.}}
\addcontentsline{toc}{section}{Sec. 12-6.5. Compelling organization
membership of persons.}

\markright{Sec. 12-6.5. Compelling organization membership of persons.}

A person who knowingly, expressly or impliedly, threatens to do bodily
harm or does bodily harm to an individual or to that individual's family
or uses any other criminally unlawful means to solicit or cause any
person to join, or deter any person from leaving, any organization or
association regardless of the nature of such organization or
association, is guilty of a Class 2 felony.

Any person of the age of 18 years or older who knowingly, expressly or
impliedly, threatens to do bodily harm or does bodily harm to a person
under 18 years of age or uses any other criminally unlawful means to
solicit or cause any person under 18 years of age to join, or deter any
person under 18 years of age from leaving, any organization or
association regardless of the nature of such organization or association
is guilty of a Class 1 felony.

A person convicted of an offense under this Section shall not be
eligible to receive a sentence of probation, conditional discharge, or
periodic imprisonment.

(Source: P.A. 96-1551, eff. 7-1-11 .)

\hypertarget{ilcs-512-7-from-ch.-38-par.-12-7}{%
\subsection*{(720 ILCS 5/12-7) (from Ch. 38, par.
12-7)}\label{ilcs-512-7-from-ch.-38-par.-12-7}}
\addcontentsline{toc}{subsection}{(720 ILCS 5/12-7) (from Ch. 38, par.
12-7)}

\hypertarget{sec.-12-7.-compelling-confession-or-information-by-force-or-threat.}{%
\section*{Sec. 12-7. Compelling confession or information by force or
threat.}\label{sec.-12-7.-compelling-confession-or-information-by-force-or-threat.}}
\addcontentsline{toc}{section}{Sec. 12-7. Compelling confession or
information by force or threat.}

\markright{Sec. 12-7. Compelling confession or information by force or
threat.}

(a) A person who, with intent to obtain a confession, statement or
information regarding any offense, knowingly inflicts or threatens
imminent bodily harm upon the person threatened or upon any other person
commits compelling a confession or information by force or threat.

(b) Sentence.

Compelling a confession or information is a: (1) Class 4 felony if the
defendant threatens imminent bodily harm to obtain a confession,
statement, or information but does not inflict bodily harm on the
victim, (2) Class 3 felony if the defendant inflicts bodily harm on the
victim to obtain a confession, statement, or information, and (3) Class
2 felony if the defendant inflicts great bodily harm to obtain a
confession, statement, or information.

(Source: P.A. 96-1551, eff. 7-1-11 .)

\hypertarget{ilcs-512-7.1-from-ch.-38-par.-12-7.1}{%
\subsection*{(720 ILCS 5/12-7.1) (from Ch. 38, par.
12-7.1)}\label{ilcs-512-7.1-from-ch.-38-par.-12-7.1}}
\addcontentsline{toc}{subsection}{(720 ILCS 5/12-7.1) (from Ch. 38, par.
12-7.1)}

\hypertarget{sec.-12-7.1.-hate-crime.}{%
\section*{Sec. 12-7.1. Hate crime.}\label{sec.-12-7.1.-hate-crime.}}
\addcontentsline{toc}{section}{Sec. 12-7.1. Hate crime.}

\markright{Sec. 12-7.1. Hate crime.}

(a) A person commits hate crime when, by reason of the actual or
perceived race, color, creed, religion, ancestry, gender, sexual
orientation, physical or mental disability, citizenship, immigration
status, or national origin of another individual or group of
individuals, regardless of the existence of any other motivating factor
or factors, he or she commits assault, battery, aggravated assault,
intimidation, stalking, cyberstalking, misdemeanor theft, criminal
trespass to residence, misdemeanor criminal damage to property, criminal
trespass to vehicle, criminal trespass to real property, mob action,
disorderly conduct, transmission of obscene messages, harassment by
telephone, or harassment through electronic communications as these
crimes are defined in Sections 12-1, 12-2, 12-3(a), 12-7.3, 12-7.5,
16-1, 19-4, 21-1, 21-2, 21-3, 25-1, 26-1, 26.5-1, 26.5-2, paragraphs
(a)(1), (a)(2), and (a)(3) of Section 12-6, and paragraphs (a)(2) and
(a)(5) of Section 26.5-3 of this Code, respectively.

(b) Except as provided in subsection (b-5), hate crime is a Class 4
felony for a first offense and a Class 2 felony for a second or
subsequent offense.

(b-5) Hate crime is a Class 3 felony for a first offense and a Class 2
felony for a second or subsequent offense if committed:

(1) in, or upon the exterior or grounds of, a church, synagogue, mosque,
or other building, structure, or place identified or associated with a
particular religion or used for religious worship or other religious
purpose;

(2) in a cemetery, mortuary, or other facility used for the purpose of
burial or memorializing the dead;

(3) in a school or other educational facility, including an
administrative facility or public or private dormitory facility of or
associated with the school or other educational facility;

(4) in a public park or an ethnic or religious community center;

(5) on the real property comprising any location specified in clauses
(1) through (4) of this subsection (b-5); or

(6) on a public way within 1,000 feet of the real property comprising
any location specified in clauses (1) through (4) of this subsection
(b-5).

(b-10) Upon imposition of any sentence, the trial court shall also
either order restitution paid to the victim or impose a fine in an
amount to be determined by the court based on the severity of the crime
and the injury or damages suffered by the victim. In addition, any order
of probation or conditional discharge entered following a conviction or
an adjudication of delinquency shall include a condition that the
offender perform public or community service of no less than 200 hours
if that service is established in the county where the offender was
convicted of hate crime. In addition, any order of probation or
conditional discharge entered following a conviction or an adjudication
of delinquency shall include a condition that the offender enroll in an
educational program discouraging hate crimes involving the protected
class identified in subsection (a) that gave rise to the offense the
offender committed. The educational program must be attended by the
offender in-person and may be administered, as determined by the court,
by a university, college, community college, non-profit organization,
the Illinois Holocaust and Genocide Commission, or any other
organization that provides educational programs discouraging hate
crimes, except that programs administered online or that can otherwise
be attended remotely are prohibited. The court may also impose any other
condition of probation or conditional discharge under this Section. If
the court sentences the offender to imprisonment or periodic
imprisonment for a violation of this Section, as a condition of the
offender's mandatory supervised release, the court shall require that
the offender perform public or community service of no less than 200
hours and enroll in an educational program discouraging hate crimes
involving the protected class identified in subsection (a) that gave
rise to the offense the offender committed.

(c) Independent of any criminal prosecution or the result of a criminal
prosecution, any person suffering injury to his or her person, damage to
his or her property, intimidation as defined in paragraphs (a)(1),
(a)(2), and (a)(3) of Section 12-6 of this Code, stalking as defined in
Section 12-7.3 of this Code, cyberstalking as defined in Section 12-7.5
of this Code, disorderly conduct as defined in paragraph (a)(1), (a)(4),
(a)(5), or (a)(6) of Section 26-1 of this Code, transmission of obscene
messages as defined in Section 26.5-1 of this Code, harassment by
telephone as defined in Section 26.5-2 of this Code, or harassment
through electronic communications as defined in paragraphs (a)(2) and
(a)(5) of Section 26.5-3 of this Code as a result of a hate crime may
bring a civil action for damages, injunction or other appropriate
relief. The court may award actual damages, including damages for
emotional distress, as well as punitive damages. The court may impose a
civil penalty up to \$25,000 for each violation of this subsection (c).
A judgment in favor of a person who brings a civil action under this
subsection (c) shall include attorney's fees and costs. After consulting
with the local State's Attorney, the Attorney General may bring a civil
action in the name of the People of the State for an injunction or other
equitable relief under this subsection (c). In addition, the Attorney
General may request and the court may impose a civil penalty up to
\$25,000 for each violation under this subsection (c). The parents or
legal guardians, other than guardians appointed pursuant to the Juvenile
Court Act or the Juvenile Court Act of 1987, of an unemancipated minor
shall be liable for the amount of any judgment for all damages rendered
against such minor under this subsection (c) in any amount not exceeding
the amount provided under Section 5 of the Parental Responsibility Law.

(d) ``Sexual orientation'' has the meaning ascribed to it in paragraph
(O-1) of Section 1-103 of the Illinois Human Rights Act.

(Source: P.A. 102-235, eff. 1-1-22; 102-468, eff. 1-1-22; 102-813, eff.
5-13-22.)

\hypertarget{ilcs-512-7.2-from-ch.-38-par.-12-7.2}{%
\subsection*{(720 ILCS 5/12-7.2) (from Ch. 38, par.
12-7.2)}\label{ilcs-512-7.2-from-ch.-38-par.-12-7.2}}
\addcontentsline{toc}{subsection}{(720 ILCS 5/12-7.2) (from Ch. 38, par.
12-7.2)}

\hypertarget{sec.-12-7.2.-educational-intimidation.}{%
\section*{Sec. 12-7.2. Educational
intimidation.}\label{sec.-12-7.2.-educational-intimidation.}}
\addcontentsline{toc}{section}{Sec. 12-7.2. Educational intimidation.}

\markright{Sec. 12-7.2. Educational intimidation.}

(a) A person commits educational intimidation when he knowingly
interferes with the right of any child who is or is believed to be
afflicted with a chronic infectious disease to attend or participate in
the activities of an elementary or secondary school in this State:

(1) by actual or threatened physical harm to the person or property of
the child or the child's family; or

(2) by impeding or obstructing the child's right of ingress to, egress
from, or freedom of movement at school facilities or activities; or

(3) by exposing or threatening to expose the child, or the family or
friends of the child, to public hatred, contempt or ridicule.

(b) Subsection (a) does not apply to the actions of school officials or
the school's infectious disease review team who are acting within the
course of their professional duties and in accordance with applicable
law.

(c) Educational intimidation is a Class C misdemeanor, except that a
second or subsequent offense shall be a Class A misdemeanor.

(d) Independent of any criminal prosecution or the result thereof, any
person suffering injury to his person or damage to his property as a
result of educational intimidation may bring a civil action for damages,
injunction or other appropriate relief. The court may award actual
damages, including damages for emotional distress, or punitive damages.
A judgment may include attorney's fees and costs. The parents or legal
guardians of an unemancipated minor, other than guardians appointed
pursuant to the Juvenile Court Act or the Juvenile Court Act of 1987,
shall be liable for the amount of any judgment for actual damages
awarded against such minor under this subsection (d) in any amount not
exceeding the amount provided under Section of the Parental
Responsibility Law.

(Source: P.A. 86-890.)

\hypertarget{ilcs-512-7.3-from-ch.-38-par.-12-7.3}{%
\subsection*{(720 ILCS 5/12-7.3) (from Ch. 38, par.
12-7.3)}\label{ilcs-512-7.3-from-ch.-38-par.-12-7.3}}
\addcontentsline{toc}{subsection}{(720 ILCS 5/12-7.3) (from Ch. 38, par.
12-7.3)}

\hypertarget{sec.-12-7.3.-stalking.}{%
\section*{Sec. 12-7.3. Stalking.}\label{sec.-12-7.3.-stalking.}}
\addcontentsline{toc}{section}{Sec. 12-7.3. Stalking.}

\markright{Sec. 12-7.3. Stalking.}

(a) A person commits stalking when he or she knowingly engages in a
course of conduct directed at a specific person, and he or she knows or
should know that this course of conduct would cause a reasonable person
to:

(1) fear for his or her safety or the safety of a third person; or

(2) suffer other emotional distress.

(a-3) A person commits stalking when he or she, knowingly and without
lawful justification, on at least 2 separate occasions follows another
person or places the person under surveillance or any combination
thereof and:

(1) at any time transmits a threat of immediate or future bodily harm,
sexual assault, confinement or restraint and the threat is directed
towards that person or a family member of that person; or

(2) places that person in reasonable apprehension of immediate or future
bodily harm, sexual assault, confinement or restraint to or of that
person or a family member of that person.

(a-5) A person commits stalking when he or she has previously been
convicted of stalking another person and knowingly and without lawful
justification on one occasion:

(1) follows that same person or places that same person under
surveillance; and

(2) transmits a threat of immediate or future bodily harm, sexual
assault, confinement or restraint to that person or a family member of
that person.

(a-7) A person commits stalking when he or she knowingly makes threats
that are a part of a course of conduct and is aware of the threatening
nature of his or her speech.

(b) Sentence. Stalking is a Class 4 felony; a second or subsequent
conviction is a Class 3 felony.

(c) Definitions. For purposes of this Section:

(1) ``Course of conduct'' means 2 or more acts, including but not
limited to acts in which a defendant directly, indirectly, or through
third parties, by any action, method, device, or means follows,
monitors, observes, surveils, threatens, or communicates to or about, a
person, engages in other non-consensual contact, or interferes with or
damages a person's property or pet. A course of conduct may include
contact via electronic communications.

(2) ``Electronic communication'' means any transfer of signs, signals,
writings, sounds, data, or intelligence of any nature transmitted in
whole or in part by a wire, radio, electromagnetic, photoelectric, or
photo-optical system. ``Electronic communication'' includes
transmissions by a computer through the Internet to another computer.

(3) ``Emotional distress'' means significant mental suffering, anxiety
or alarm.

(4) ``Family member'' means a parent, grandparent, brother, sister, or
child, whether by whole blood, half-blood, or adoption and includes a
step-grandparent, step-parent, step-brother, step-sister or step-child.
``Family member'' also means any other person who regularly resides in
the household, or who, within the prior 6 months, regularly resided in
the household.

(5) ``Follows another person'' means (i) to move in relative proximity
to a person as that person moves from place to place or (ii) to remain
in relative proximity to a person who is stationary or whose movements
are confined to a small area. ``Follows another person'' does not
include a following within the residence of the defendant.

(6) ``Non-consensual contact'' means any contact with the victim that is
initiated or continued without the victim's consent, including but not
limited to being in the physical presence of the victim; appearing
within the sight of the victim; approaching or confronting the victim in
a public place or on private property; appearing at the workplace or
residence of the victim; entering onto or remaining on property owned,
leased, or occupied by the victim; or placing an object on, or
delivering an object to, property owned, leased, or occupied by the
victim.

(7) ``Places a person under surveillance'' means: (1) remaining present
outside the person's school, place of employment, vehicle, other place
occupied by the person, or residence other than the residence of the
defendant; or (2) placing an electronic tracking device on the person or
the person's property.

(8) ``Reasonable person'' means a person in the victim's situation.

(9) ``Transmits a threat'' means a verbal or written threat or a threat
implied by a pattern of conduct or a combination of verbal or written
statements or conduct.

(d) Exemptions.

(1) This Section does not apply to any individual or organization (i)
monitoring or attentive to compliance with public or worker safety laws,
wage and hour requirements, or other statutory requirements, or (ii)
picketing occurring at the workplace that is otherwise lawful and arises
out of a bona fide labor dispute, including any controversy concerning
wages, salaries, hours, working conditions or benefits, including health
and welfare, sick leave, insurance, and pension or retirement
provisions, the making or maintaining of collective bargaining
agreements, and the terms to be included in those agreements.

(2) This Section does not apply to an exercise of the right to free
speech or assembly that is otherwise lawful.

(3) Telecommunications carriers, commercial mobile service providers,
and providers of information services, including, but not limited to,
Internet service providers and hosting service providers, are not liable
under this Section, except for willful and wanton misconduct, by virtue
of the transmission, storage, or caching of electronic communications or
messages of others or by virtue of the provision of other related
telecommunications, commercial mobile services, or information services
used by others in violation of this Section.

(d-5) The incarceration of a person in a penal institution who commits
the course of conduct or transmits a threat is not a bar to prosecution
under this Section.

(d-10) A defendant who directed the actions of a third party to violate
this Section, under the principles of accountability set forth in
Article 5 of this Code, is guilty of violating this Section as if the
same had been personally done by the defendant, without regard to the
mental state of the third party acting at the direction of the
defendant.

(Source: P.A. 102-547, eff. 1-1-22 .)

\hypertarget{ilcs-512-7.4-from-ch.-38-par.-12-7.4}{%
\subsection*{(720 ILCS 5/12-7.4) (from Ch. 38, par.
12-7.4)}\label{ilcs-512-7.4-from-ch.-38-par.-12-7.4}}
\addcontentsline{toc}{subsection}{(720 ILCS 5/12-7.4) (from Ch. 38, par.
12-7.4)}

\hypertarget{sec.-12-7.4.-aggravated-stalking.}{%
\section*{Sec. 12-7.4. Aggravated
stalking.}\label{sec.-12-7.4.-aggravated-stalking.}}
\addcontentsline{toc}{section}{Sec. 12-7.4. Aggravated stalking.}

\markright{Sec. 12-7.4. Aggravated stalking.}

(a) A person commits aggravated stalking when he or she commits stalking
and:

(1) causes bodily harm to the victim;

(2) confines or restrains the victim; or

(3) violates a temporary restraining order, an order of protection, a
stalking no contact order, a civil no contact order, or an injunction
prohibiting the behavior described in subsection (b)(1) of Section 214
of the Illinois Domestic Violence Act of 1986.

(a-1) A person commits aggravated stalking when he or she is required to
register under the Sex Offender Registration Act or has been previously
required to register under that Act and commits the offense of stalking
when the victim of the stalking is also the victim of the offense for
which the sex offender is required to register under the Sex Offender
Registration Act or a family member of the victim.

(b) Sentence. Aggravated stalking is a Class 3 felony; a second or
subsequent conviction is a Class 2 felony.

(c) Exemptions.

(1) This Section does not apply to any individual or organization (i)
monitoring or attentive to compliance with public or worker safety laws,
wage and hour requirements, or other statutory requirements, or (ii)
picketing occurring at the workplace that is otherwise lawful and arises
out of a bona fide labor dispute including any controversy concerning
wages, salaries, hours, working conditions or benefits, including health
and welfare, sick leave, insurance, and pension or retirement
provisions, the managing or maintenance of collective bargaining
agreements, and the terms to be included in those agreements.

(2) This Section does not apply to an exercise of the right of free
speech or assembly that is otherwise lawful.

(3) Telecommunications carriers, commercial mobile service providers,
and providers of information services, including, but not limited to,
Internet service providers and hosting service providers, are not liable
under this Section, except for willful and wanton misconduct, by virtue
of the transmission, storage, or caching of electronic communications or
messages of others or by virtue of the provision of other related
telecommunications, commercial mobile services, or information services
used by others in violation of this Section.

(d) A defendant who directed the actions of a third party to violate
this Section, under the principles of accountability set forth in
Article 5 of this Code, is guilty of violating this Section as if the
same had been personally done by the defendant, without regard to the
mental state of the third party acting at the direction of the
defendant.

(Source: P.A. 96-686, eff. 1-1-10; 96-1551, eff. 7-1-11; 97-311, eff.
8-11-11; 97-468, eff. 1-1-12; 97-1109, eff. 1-1-13.)

\hypertarget{ilcs-512-7.5}{%
\subsection*{(720 ILCS 5/12-7.5)}\label{ilcs-512-7.5}}
\addcontentsline{toc}{subsection}{(720 ILCS 5/12-7.5)}

\hypertarget{sec.-12-7.5.-cyberstalking.}{%
\section*{Sec. 12-7.5.
Cyberstalking.}\label{sec.-12-7.5.-cyberstalking.}}
\addcontentsline{toc}{section}{Sec. 12-7.5. Cyberstalking.}

\markright{Sec. 12-7.5. Cyberstalking.}

(a) A person commits cyberstalking when he or she engages in a course of
conduct using electronic communication directed at a specific person,
and he or she knows or should know that would cause a reasonable person
to:

(1) fear for his or her safety or the safety of a third person; or

(2) suffer other emotional distress.

(a-3) A person commits cyberstalking when he or she, knowingly and
without lawful justification, on at least 2 separate occasions, harasses
another person through the use of electronic communication and:

(1) at any time transmits a threat of immediate or future bodily harm,
sexual assault, confinement, or restraint and the threat is directed
towards that person or a family member of that person; or

(2) places that person or a family member of that person in reasonable
apprehension of immediate or future bodily harm, sexual assault,
confinement, or restraint; or

(3) at any time knowingly solicits the commission of an act by any
person which would be a violation of this Code directed towards that
person or a family member of that person.

(a-4) A person commits cyberstalking when he or she knowingly,
surreptitiously, and without lawful justification, installs or otherwise
places electronic monitoring software or spyware on an electronic
communication device as a means to harass another person and:

(1) at any time transmits a threat of immediate or future bodily harm,
sexual assault, confinement, or restraint and the threat is directed
towards that person or a family member of that person;

(2) places that person or a family member of that person in reasonable
apprehension of immediate or future bodily harm, sexual assault,
confinement, or restraint; or

(3) at any time knowingly solicits the commission of an act by any
person which would be a violation of this Code directed towards that
person or a family member of that person.

For purposes of this Section, an installation or placement is not
surreptitious if:

(1) with respect to electronic software, hardware, or computer
applications, clear notice regarding the use of the specific type of
tracking software or spyware is provided by the installer in advance to
the owners and primary users of the electronic software, hardware, or
computer application; or

(2) written or electronic consent of all owners and primary users of the
electronic software, hardware, or computer application on which the
tracking software or spyware will be installed has been sought and
obtained through a mechanism that does not seek to obtain any other
approvals or acknowledgement from the owners and primary users.

(a-5) A person commits cyberstalking when he or she, knowingly and
without lawful justification, creates and maintains an Internet website
or webpage which is accessible to one or more third parties for a period
of at least 24 hours, and which contains statements harassing another
person and:

(1) which communicates a threat of immediate or future bodily harm,
sexual assault, confinement, or restraint, where the threat is directed
towards that person or a family member of that person, or

(2) which places that person or a family member of that person in
reasonable apprehension of immediate or future bodily harm, sexual
assault, confinement, or restraint, or

(3) which knowingly solicits the commission of an act by any person
which would be a violation of this Code directed towards that person or
a family member of that person.

(b) Sentence. Cyberstalking is a Class 4 felony; a second or subsequent
conviction is a Class 3 felony.

(c) For purposes of this Section:

(1) ``Course of conduct'' means 2 or more acts, including but not
limited to acts in which a defendant directly, indirectly, or through
third parties, by any action, method, device, or means follows,
monitors, observes, surveils, threatens, or communicates to or about, a
person, engages in other non-consensual contact, or interferes with or
damages a person's property or pet. The incarceration in a penal
institution of a person who commits the course of conduct is not a bar
to prosecution under this Section.

(2) ``Electronic communication'' means any transfer of signs, signals,
writings, sounds, data, or intelligence of any nature transmitted in
whole or in part by a wire, radio, electromagnetic, photoelectric, or
photo-optical system. ``Electronic communication'' includes
transmissions through an electronic device including, but not limited
to, a telephone, cellular phone, computer, or pager, which communication
includes, but is not limited to, e-mail, instant message, text message,
or voice mail.

(2.1) ``Electronic communication device'' means an electronic device,
including, but not limited to, a wireless telephone, personal digital
assistant, or a portable or mobile computer.

(2.2) ``Electronic monitoring software or spyware'' means software or an
application that surreptitiously tracks computer activity on a device
and records and transmits the information to third parties with the
intent to cause injury or harm. For the purposes of this paragraph
(2.2), ``intent to cause injury or harm'' does not include activities
carried out in furtherance of the prevention of fraud or crime or of
protecting the security of networks, online services, applications,
software, other computer programs, users, or electronic communication
devices or similar devices.

(3) ``Emotional distress'' means significant mental suffering, anxiety
or alarm.

(4) ``Harass'' means to engage in a knowing and willful course of
conduct directed at a specific person that alarms, torments, or
terrorizes that person.

(5) ``Non-consensual contact'' means any contact with the victim that is
initiated or continued without the victim's consent, including but not
limited to being in the physical presence of the victim; appearing
within the sight of the victim; approaching or confronting the victim in
a public place or on private property; appearing at the workplace or
residence of the victim; entering onto or remaining on property owned,
leased, or occupied by the victim; or placing an object on, or
delivering an object to, property owned, leased, or occupied by the
victim.

(6) ``Reasonable person'' means a person in the victim's circumstances,
with the victim's knowledge of the defendant and the defendant's prior
acts.

(7) ``Third party'' means any person other than the person violating
these provisions and the person or persons towards whom the violator's
actions are directed.

(d) Telecommunications carriers, commercial mobile service providers,
and providers of information services, including, but not limited to,
Internet service providers and hosting service providers, are not liable
under this Section, except for willful and wanton misconduct, by virtue
of the transmission, storage, or caching of electronic communications or
messages of others or by virtue of the provision of other related
telecommunications, commercial mobile services, or information services
used by others in violation of this Section.

(e) A defendant who directed the actions of a third party to violate
this Section, under the principles of accountability set forth in
Article 5 of this Code, is guilty of violating this Section as if the
same had been personally done by the defendant, without regard to the
mental state of the third party acting at the direction of the
defendant.

(f) It is not a violation of this Section to:

(1) provide, protect, maintain, update, or upgrade networks, online
services, applications, software, other computer programs, electronic
communication devices, or similar devices under the terms of use
applicable to those networks, services, applications, software,
programs, or devices;

(2) interfere with or prohibit terms or conditions in a contract or
license related to networks, online services, applications, software,
other computer programs, electronic communication devices, or similar
devices; or

(3) create any liability by reason of terms or conditions adopted, or
technical measures implemented, to prevent the transmission of
unsolicited electronic mail or communications.

(Source: P.A. 100-166, eff. 1-1-18 .)

\hypertarget{ilcs-512-7.6}{%
\subsection*{(720 ILCS 5/12-7.6)}\label{ilcs-512-7.6}}
\addcontentsline{toc}{subsection}{(720 ILCS 5/12-7.6)}

\hypertarget{sec.-12-7.6.-cross-burning.}{%
\section*{Sec. 12-7.6. Cross
burning.}\label{sec.-12-7.6.-cross-burning.}}
\addcontentsline{toc}{section}{Sec. 12-7.6. Cross burning.}

\markright{Sec. 12-7.6. Cross burning.}

(a) A person commits cross burning when he or she, with the intent to
intimidate any other person or group of persons, burns or causes to be
burned a cross.

(b) Sentence. Cross burning is a Class A misdemeanor for a first offense
and a Class 4 felony for a second or subsequent offense.

(c) For the purposes of this Section, a person acts with the ``intent to
intimidate'' when he or she intentionally places or attempts to place
another person in fear of physical injury or fear of damage to that
other person's property.

(Source: P.A. 96-1551, eff. 7-1-11 .)

\hypertarget{ilcs-512-8-from-ch.-38-par.-12-8}{%
\subsection*{(720 ILCS 5/12-8) (from Ch. 38, par.
12-8)}\label{ilcs-512-8-from-ch.-38-par.-12-8}}
\addcontentsline{toc}{subsection}{(720 ILCS 5/12-8) (from Ch. 38, par.
12-8)}

\hypertarget{sec.-12-8.-repealed.}{%
\section*{Sec. 12-8. (Repealed).}\label{sec.-12-8.-repealed.}}
\addcontentsline{toc}{section}{Sec. 12-8. (Repealed).}

\markright{Sec. 12-8. (Repealed).}

(Source: P.A. 77-2638. Repealed by P.A. 89-657, eff. 8-14-96.)

\hypertarget{ilcs-512-9-from-ch.-38-par.-12-9}{%
\subsection*{(720 ILCS 5/12-9) (from Ch. 38, par.
12-9)}\label{ilcs-512-9-from-ch.-38-par.-12-9}}
\addcontentsline{toc}{subsection}{(720 ILCS 5/12-9) (from Ch. 38, par.
12-9)}

\hypertarget{sec.-12-9.-threatening-public-officials-human-service-providers.}{%
\section*{Sec. 12-9. Threatening public officials; human service
providers.}\label{sec.-12-9.-threatening-public-officials-human-service-providers.}}
\addcontentsline{toc}{section}{Sec. 12-9. Threatening public officials;
human service providers.}

\markright{Sec. 12-9. Threatening public officials; human service
providers.}

(a) A person commits threatening a public official or human service
provider when:

(1) that person knowingly delivers or conveys, directly or indirectly,
to a public official or human service provider by any means a
communication:

(i) containing a threat that would place the public official or human
service provider or a member of his or her immediate family in
reasonable apprehension of immediate or future bodily harm, sexual
assault, confinement, or restraint; or

(ii) containing a threat that would place the public official or human
service provider or a member of his or her immediate family in
reasonable apprehension that damage will occur to property in the
custody, care, or control of the public official or his or her immediate
family; and

(2) the threat was conveyed because of the performance or nonperformance
of some public duty or duty as a human service provider, because of
hostility of the person making the threat toward the status or position
of the public official or the human service provider, or because of any
other factor related to the official's public existence.

(a-5) For purposes of a threat to a sworn law enforcement officer, the
threat must contain specific facts indicative of a unique threat to the
person, family or property of the officer and not a generalized threat
of harm.

(a-6) For purposes of a threat to a social worker, caseworker,
investigator, or human service provider, the threat must contain
specific facts indicative of a unique threat to the person, family or
property of the individual and not a generalized threat of harm.

(b) For purposes of this Section:

(1) ``Public official'' means a person who is elected to office in
accordance with a statute or who is appointed to an office which is
established, and the qualifications and duties of which are prescribed,
by statute, to discharge a public duty for the State or any of its
political subdivisions or in the case of an elective office any person
who has filed the required documents for nomination or election to such
office. ``Public official'' includes a duly appointed assistant State's
Attorney, assistant Attorney General, or Appellate Prosecutor; a sworn
law enforcement or peace officer; a social worker, caseworker, attorney,
or investigator employed by the Department of Healthcare and Family
Services, the Department of Human Services, the Department of Children
and Family Services, or the Guardianship and Advocacy Commission; or an
assistant public guardian, attorney, social worker, case manager, or
investigator employed by a duly appointed public guardian.

(1.5) ``Human service provider'' means a social worker, case worker, or
investigator employed by an agency or organization providing social
work, case work, or investigative services under a contract with or a
grant from the Department of Human Services, the Department of Children
and Family Services, the Department of Healthcare and Family Services,
or the Department on Aging.

(2) ``Immediate family'' means a public official's spouse or child or
children.

(c) Threatening a public official or human service provider is a Class 3
felony for a first offense and a Class 2 felony for a second or
subsequent offense.

(Source: P.A. 100-1, eff. 1-1-18 .)

\hypertarget{ilcs-5art.-12-subdiv.-20-heading}{%
\subsection*{(720 ILCS 5/Art. 12, Subdiv. 20
heading)}\label{ilcs-5art.-12-subdiv.-20-heading}}
\addcontentsline{toc}{subsection}{(720 ILCS 5/Art. 12, Subdiv. 20
heading)}

SUBDIVISION 20.

MUTILATION

(Source: P.A. 96-1551, eff. 7-1-11.)

\hypertarget{ilcs-512-10-from-ch.-38-par.-12-10}{%
\subsection*{(720 ILCS 5/12-10) (from Ch. 38, par.
12-10)}\label{ilcs-512-10-from-ch.-38-par.-12-10}}
\addcontentsline{toc}{subsection}{(720 ILCS 5/12-10) (from Ch. 38, par.
12-10)}

(This Section was renumbered as Section 12C-35 by P.A. 97-1109.)

\hypertarget{sec.-12-10.-renumbered.}{%
\section*{Sec. 12-10. (Renumbered).}\label{sec.-12-10.-renumbered.}}
\addcontentsline{toc}{section}{Sec. 12-10. (Renumbered).}

\markright{Sec. 12-10. (Renumbered).}

(Source: P.A. 94-684, eff. 1-1-06. Renumbered by P.A. 97-1109, eff.
1-1-13.)

\hypertarget{ilcs-512-10.1}{%
\subsection*{(720 ILCS 5/12-10.1)}\label{ilcs-512-10.1}}
\addcontentsline{toc}{subsection}{(720 ILCS 5/12-10.1)}

(This Section was renumbered as Section 12C-40 by P.A. 97-1109.)

\hypertarget{sec.-12-10.1.-renumbered.}{%
\section*{Sec. 12-10.1. (Renumbered).}\label{sec.-12-10.1.-renumbered.}}
\addcontentsline{toc}{section}{Sec. 12-10.1. (Renumbered).}

\markright{Sec. 12-10.1. (Renumbered).}

(Source: P.A. 93-449, eff. 1-1-04; 94-684, eff. 1-1-06. Renumbered by
P.A. 97-1109, eff. 1-1-13.)

\hypertarget{ilcs-512-10.2}{%
\subsection*{(720 ILCS 5/12-10.2)}\label{ilcs-512-10.2}}
\addcontentsline{toc}{subsection}{(720 ILCS 5/12-10.2)}

\hypertarget{sec.-12-10.2.-tongue-splitting.}{%
\section*{Sec. 12-10.2. Tongue
splitting.}\label{sec.-12-10.2.-tongue-splitting.}}
\addcontentsline{toc}{section}{Sec. 12-10.2. Tongue splitting.}

\markright{Sec. 12-10.2. Tongue splitting.}

(a) In this Section, ``tongue splitting'' means the cutting of a human
tongue into 2 or more parts.

(b) A person may not knowingly perform tongue splitting on another
person unless the person performing the tongue splitting is licensed to
practice medicine in all its branches under the Medical Practice Act of
1987 or licensed under the Illinois Dental Practice Act.

(c) Sentence. Tongue splitting performed in violation of this Section is
a Class A misdemeanor for a first offense and a Class 4 felony for a
second or subsequent offense.

(Source: P.A. 96-1551, eff. 7-1-11 .)

\hypertarget{ilcs-512-10.3}{%
\subsection*{(720 ILCS 5/12-10.3)}\label{ilcs-512-10.3}}
\addcontentsline{toc}{subsection}{(720 ILCS 5/12-10.3)}

\hypertarget{sec.-12-10.3.-false-representation-to-a-tattoo-or-body-piercing-business-as-the-parent-or-legal-guardian-of-a-minor.}{%
\section*{Sec. 12-10.3. False representation to a tattoo or body
piercing business as the parent or legal guardian of a
minor.}\label{sec.-12-10.3.-false-representation-to-a-tattoo-or-body-piercing-business-as-the-parent-or-legal-guardian-of-a-minor.}}
\addcontentsline{toc}{section}{Sec. 12-10.3. False representation to a
tattoo or body piercing business as the parent or legal guardian of a
minor.}

\markright{Sec. 12-10.3. False representation to a tattoo or body
piercing business as the parent or legal guardian of a minor.}

(a) A person, other than the parent or legal guardian of a minor,
commits the offense of false representation to a tattoo or body piercing
business as the parent or legal guardian of a minor when he or she
falsely represents himself or herself as the parent or legal guardian of
the minor to an owner or employee of a tattoo or body piercing business
for the purpose of:

(1) accompanying the minor to a business that provides tattooing as
required under Section 12-10 of this Code (tattooing body of minor);

(2) accompanying the minor to a business that provides body piercing as
required under Section 12-10.1 of this Code (piercing the body of a
minor); or

(3) furnishing the written consent required under

Section 12-10.1 of this Code (piercing the body of a minor).

(b) Sentence. False representation to a tattoo or body piercing business
as the parent or legal guardian of a minor is a Class C misdemeanor.

(Source: P.A. 96-1311, eff. 1-1-11.)

\hypertarget{ilcs-512-11-from-ch.-38-par.-12-11}{%
\subsection*{(720 ILCS 5/12-11) (from Ch. 38, par.
12-11)}\label{ilcs-512-11-from-ch.-38-par.-12-11}}
\addcontentsline{toc}{subsection}{(720 ILCS 5/12-11) (from Ch. 38, par.
12-11)}

(This Section was renumbered as Section 19-6 by P.A. 97-1108.)

\hypertarget{sec.-12-11.-renumbered.}{%
\section*{Sec. 12-11. (Renumbered).}\label{sec.-12-11.-renumbered.}}
\addcontentsline{toc}{section}{Sec. 12-11. (Renumbered).}

\markright{Sec. 12-11. (Renumbered).}

(Source: P.A. 96-1113, eff. 1-1-11; 96-1551, eff. 7-1-11. Renumbered by
P.A. 97-1108, eff. 1-1-13.)

\hypertarget{ilcs-512-11.1-from-ch.-38-par.-12-11.1}{%
\subsection*{(720 ILCS 5/12-11.1) (from Ch. 38, par.
12-11.1)}\label{ilcs-512-11.1-from-ch.-38-par.-12-11.1}}
\addcontentsline{toc}{subsection}{(720 ILCS 5/12-11.1) (from Ch. 38,
par. 12-11.1)}

(This Section was renumbered as Section 18-6 by P.A. 97-1108.)

\hypertarget{sec.-12-11.1.-renumbered.}{%
\section*{Sec. 12-11.1. (Renumbered).}\label{sec.-12-11.1.-renumbered.}}
\addcontentsline{toc}{section}{Sec. 12-11.1. (Renumbered).}

\markright{Sec. 12-11.1. (Renumbered).}

(Source: P.A. 86-1392. Renumbered by P.A. 97-1108, eff. 1-1-13.)

\hypertarget{ilcs-512-12-from-ch.-38-par.-12-12}{%
\subsection*{(720 ILCS 5/12-12) (from Ch. 38, par.
12-12)}\label{ilcs-512-12-from-ch.-38-par.-12-12}}
\addcontentsline{toc}{subsection}{(720 ILCS 5/12-12) (from Ch. 38, par.
12-12)}

\hypertarget{sec.-12-12.-repealed.}{%
\section*{Sec. 12-12. (Repealed).}\label{sec.-12-12.-repealed.}}
\addcontentsline{toc}{section}{Sec. 12-12. (Repealed).}

\markright{Sec. 12-12. (Repealed).}

(Source: P.A. 96-233, eff. 1-1-10. Repealed by P.A. 96-1551, eff. 7-1-11
.)

\hypertarget{ilcs-512-13-from-ch.-38-par.-12-13}{%
\subsection*{(720 ILCS 5/12-13) (from Ch. 38, par.
12-13)}\label{ilcs-512-13-from-ch.-38-par.-12-13}}
\addcontentsline{toc}{subsection}{(720 ILCS 5/12-13) (from Ch. 38, par.
12-13)}

(This Section was renumbered as Section 11-1.20 by P.A. 96-1551.)

\hypertarget{sec.-12-13.-renumbered.}{%
\section*{Sec. 12-13. (Renumbered).}\label{sec.-12-13.-renumbered.}}
\addcontentsline{toc}{section}{Sec. 12-13. (Renumbered).}

\markright{Sec. 12-13. (Renumbered).}

(Source: P.A. 95-640, eff. 6-1-08. Renumbered by P.A. 96-1551, eff.
7-1-11 .)

\hypertarget{ilcs-512-14-from-ch.-38-par.-12-14}{%
\subsection*{(720 ILCS 5/12-14) (from Ch. 38, par.
12-14)}\label{ilcs-512-14-from-ch.-38-par.-12-14}}
\addcontentsline{toc}{subsection}{(720 ILCS 5/12-14) (from Ch. 38, par.
12-14)}

(This Section was renumbered as Section 11-1.30 by P.A. 96-1551.)

\hypertarget{sec.-12-14.-renumbered.}{%
\section*{Sec. 12-14. (Renumbered).}\label{sec.-12-14.-renumbered.}}
\addcontentsline{toc}{section}{Sec. 12-14. (Renumbered).}

\markright{Sec. 12-14. (Renumbered).}

(Source: P.A. 97-227, eff. 1-1-12. Renumbered by P.A. 96-1551, eff.
7-1-11.)

\hypertarget{ilcs-512-14.1}{%
\subsection*{(720 ILCS 5/12-14.1)}\label{ilcs-512-14.1}}
\addcontentsline{toc}{subsection}{(720 ILCS 5/12-14.1)}

(This Section was renumbered as Section 11-1.40 by P.A. 96-1551.)

\hypertarget{sec.-12-14.1.-renumbered.}{%
\section*{Sec. 12-14.1. (Renumbered).}\label{sec.-12-14.1.-renumbered.}}
\addcontentsline{toc}{section}{Sec. 12-14.1. (Renumbered).}

\markright{Sec. 12-14.1. (Renumbered).}

(Source: P.A. 95-640, eff. 6-1-08. Renumbered by P.A. 96-1551, eff.
7-1-11 .)

\hypertarget{ilcs-512-15-from-ch.-38-par.-12-15}{%
\subsection*{(720 ILCS 5/12-15) (from Ch. 38, par.
12-15)}\label{ilcs-512-15-from-ch.-38-par.-12-15}}
\addcontentsline{toc}{subsection}{(720 ILCS 5/12-15) (from Ch. 38, par.
12-15)}

(This Section was renumbered as Section 11-1.50 by P.A. 96-1551.)

\hypertarget{sec.-12-15.-renumbered.}{%
\section*{Sec. 12-15. (Renumbered).}\label{sec.-12-15.-renumbered.}}
\addcontentsline{toc}{section}{Sec. 12-15. (Renumbered).}

\markright{Sec. 12-15. (Renumbered).}

(Source: P.A. 91-389, eff. 1-1-00. Renumbered by P.A. 96-1551, eff.
7-1-11 .)

\hypertarget{ilcs-512-16-from-ch.-38-par.-12-16}{%
\subsection*{(720 ILCS 5/12-16) (from Ch. 38, par.
12-16)}\label{ilcs-512-16-from-ch.-38-par.-12-16}}
\addcontentsline{toc}{subsection}{(720 ILCS 5/12-16) (from Ch. 38, par.
12-16)}

(This Section was renumbered as Section 11-1.60 by P.A. 96-1551.)

\hypertarget{sec.-12-16.-renumbered.}{%
\section*{Sec. 12-16. (Renumbered).}\label{sec.-12-16.-renumbered.}}
\addcontentsline{toc}{section}{Sec. 12-16. (Renumbered).}

\markright{Sec. 12-16. (Renumbered).}

(Source: P.A. 97-227, eff. 1-1-12. Renumbered by P.A. 96-1551, eff.
7-1-11.)

\hypertarget{ilcs-512-16.2-from-ch.-38-par.-12-16.2}{%
\subsection*{(720 ILCS 5/12-16.2) (from Ch. 38, par.
12-16.2)}\label{ilcs-512-16.2-from-ch.-38-par.-12-16.2}}
\addcontentsline{toc}{subsection}{(720 ILCS 5/12-16.2) (from Ch. 38,
par. 12-16.2)}

(This Section was renumbered as Section 12-5.01 by P.A. 96-1551.)

\hypertarget{sec.-12-16.2.-renumbered.}{%
\section*{Sec. 12-16.2. (Renumbered).}\label{sec.-12-16.2.-renumbered.}}
\addcontentsline{toc}{section}{Sec. 12-16.2. (Renumbered).}

\markright{Sec. 12-16.2. (Renumbered).}

(Source: P.A. 86-897. Renumbered by P.A. 96-1551, eff. 7-1-11 .)

\hypertarget{ilcs-512-17-from-ch.-38-par.-12-17}{%
\subsection*{(720 ILCS 5/12-17) (from Ch. 38, par.
12-17)}\label{ilcs-512-17-from-ch.-38-par.-12-17}}
\addcontentsline{toc}{subsection}{(720 ILCS 5/12-17) (from Ch. 38, par.
12-17)}

(This Section was renumbered as Section 11-1.70 by P.A. 96-1551.)

\hypertarget{sec.-12-17.-renumbered.}{%
\section*{Sec. 12-17. (Renumbered).}\label{sec.-12-17.-renumbered.}}
\addcontentsline{toc}{section}{Sec. 12-17. (Renumbered).}

\markright{Sec. 12-17. (Renumbered).}

(Source: P.A. 93-389, eff. 7-25-03. Renumbered by P.A. 96-1551, eff.
7-1-11 .)

\hypertarget{ilcs-512-18-from-ch.-38-par.-12-18}{%
\subsection*{(720 ILCS 5/12-18) (from Ch. 38, par.
12-18)}\label{ilcs-512-18-from-ch.-38-par.-12-18}}
\addcontentsline{toc}{subsection}{(720 ILCS 5/12-18) (from Ch. 38, par.
12-18)}

(This Section was renumbered as Section 11-1.10 by P.A. 96-1551.)

\hypertarget{sec.-12-18.-renumbered.}{%
\section*{Sec. 12-18. (Renumbered).}\label{sec.-12-18.-renumbered.}}
\addcontentsline{toc}{section}{Sec. 12-18. (Renumbered).}

\markright{Sec. 12-18. (Renumbered).}

(Source: P.A. 97-244, eff. 8-4-11. Renumbered by P.A. 96-1551, eff.
7-1-11.)

\hypertarget{ilcs-512-18.1-from-ch.-38-par.-12-18.1}{%
\subsection*{(720 ILCS 5/12-18.1) (from Ch. 38, par.
12-18.1)}\label{ilcs-512-18.1-from-ch.-38-par.-12-18.1}}
\addcontentsline{toc}{subsection}{(720 ILCS 5/12-18.1) (from Ch. 38,
par. 12-18.1)}

(This Section was renumbered as Section 11-1.80 by P.A. 96-1551.)

\hypertarget{sec.-12-18.1.-renumbered.}{%
\section*{Sec. 12-18.1. (Renumbered).}\label{sec.-12-18.1.-renumbered.}}
\addcontentsline{toc}{section}{Sec. 12-18.1. (Renumbered).}

\markright{Sec. 12-18.1. (Renumbered).}

(Source: P.A. 96-1551, Article 2, Section 1035, eff. 7-1-11. Renumbered
by P.A. 96-1551, Article 2, Section 5, eff. 7-1-11.)

\hypertarget{ilcs-512-19-from-ch.-38-par.-12-19}{%
\subsection*{(720 ILCS 5/12-19) (from Ch. 38, par.
12-19)}\label{ilcs-512-19-from-ch.-38-par.-12-19}}
\addcontentsline{toc}{subsection}{(720 ILCS 5/12-19) (from Ch. 38, par.
12-19)}

\hypertarget{sec.-12-19.-repealed.}{%
\section*{Sec. 12-19. (Repealed).}\label{sec.-12-19.-repealed.}}
\addcontentsline{toc}{section}{Sec. 12-19. (Repealed).}

\markright{Sec. 12-19. (Repealed).}

(Source: P.A. 97-227, eff. 1-1-12. Repealed by P.A. 96-1551, eff.
7-1-11.)

\hypertarget{ilcs-512-20-from-ch.-38-par.-12-20}{%
\subsection*{(720 ILCS 5/12-20) (from Ch. 38, par.
12-20)}\label{ilcs-512-20-from-ch.-38-par.-12-20}}
\addcontentsline{toc}{subsection}{(720 ILCS 5/12-20) (from Ch. 38, par.
12-20)}

\hypertarget{sec.-12-20.-sale-of-body-parts.}{%
\section*{Sec. 12-20. Sale of body
parts.}\label{sec.-12-20.-sale-of-body-parts.}}
\addcontentsline{toc}{section}{Sec. 12-20. Sale of body parts.}

\markright{Sec. 12-20. Sale of body parts.}

(a) Except as provided in subsection (b), any person who knowingly buys
or sells, or offers to buy or sell, a human body or any part of a human
body, is guilty of a Class A misdemeanor for the first conviction and a
Class 4 felony for subsequent convictions.

(b) This Section does not prohibit:

(1) An anatomical gift made in accordance with the

Illinois Anatomical Gift Act.

(2) (Blank).

(3) Reimbursement of actual expenses incurred by a living person in
donating an organ, tissue or other body part or fluid for
transplantation, implantation, infusion, injection, or other medical or
scientific purpose, including medical costs, loss of income, and travel
expenses.

(4) Payments provided under a plan of insurance or other health care
coverage.

(5) Reimbursement of reasonable costs associated with the removal,
storage or transportation of a human body or part thereof donated for
medical or scientific purposes.

(6) Purchase or sale of blood, plasma, blood products or derivatives,
other body fluids, or human hair.

(7) Purchase or sale of drugs, reagents or other substances made from
human bodies or body parts, for use in medical or scientific research,
treatment or diagnosis.

(Source: P.A. 96-1551, eff. 7-1-11 .)

\hypertarget{ilcs-512-20.5}{%
\subsection*{(720 ILCS 5/12-20.5)}\label{ilcs-512-20.5}}
\addcontentsline{toc}{subsection}{(720 ILCS 5/12-20.5)}

\hypertarget{sec.-12-20.5.-dismembering-a-human-body.}{%
\section*{Sec. 12-20.5. Dismembering a human
body.}\label{sec.-12-20.5.-dismembering-a-human-body.}}
\addcontentsline{toc}{section}{Sec. 12-20.5. Dismembering a human body.}

\markright{Sec. 12-20.5. Dismembering a human body.}

(a) A person commits dismembering a human body when he or she knowingly
dismembers, severs, separates, dissects, or mutilates any body part of a
deceased's body.

(b) This Section does not apply to:

(1) an anatomical gift made in accordance with the

Illinois Anatomical Gift Act;

(2) (blank);

(3) the purchase or sale of drugs, reagents, or other substances made
from human body parts, for the use in medical or scientific research,
treatment, or diagnosis;

(4) persons employed by a county medical examiner's office or coroner's
office acting within the scope of their employment while performing an
autopsy;

(5) the acts of a licensed funeral director or embalmer while performing
acts authorized by the Funeral Directors and Embalmers Licensing Code;

(6) the acts of emergency medical personnel or physicians performed in
good faith and according to the usual and customary standards of medical
practice in an attempt to resuscitate a life; or

(7) physicians licensed to practice medicine in all of its branches or
holding a visiting professor, physician, or resident permit under the
Medical Practice Act of 1987, performing acts in accordance with usual
and customary standards of medical practice, or a currently enrolled
student in an accredited medical school in furtherance of his or her
education at the accredited medical school.

(c) It is not a defense to a violation of this Section that the decedent
died due to natural, accidental, or suicidal causes.

(d) Sentence. Dismembering a human body is a Class X felony.

(Source: P.A. 95-331, eff. 8-21-07; 96-1551, eff. 7-1-11 .)

\hypertarget{ilcs-512-20.6}{%
\subsection*{(720 ILCS 5/12-20.6)}\label{ilcs-512-20.6}}
\addcontentsline{toc}{subsection}{(720 ILCS 5/12-20.6)}

\hypertarget{sec.-12-20.6.-abuse-of-a-corpse.}{%
\section*{Sec. 12-20.6. Abuse of a
corpse.}\label{sec.-12-20.6.-abuse-of-a-corpse.}}
\addcontentsline{toc}{section}{Sec. 12-20.6. Abuse of a corpse.}

\markright{Sec. 12-20.6. Abuse of a corpse.}

(a) In this Section:

``Corpse'' means the dead body of a human being.

``Sexual conduct'' has the meaning ascribed to the term in Section
11-0.1 of this Code.

(b) A person commits abuse of a corpse if he or she intentionally:

(1) engages in sexual conduct with a corpse or involving a corpse; or

(2) removes or carries away a corpse and is not authorized by law to do
so.

(c) Sentence.

(1) A person convicted of violating paragraph (1) of subsection (b) of
this Section is guilty of a Class 2 felony.

(2) A person convicted of violating paragraph (2) of subsection (b) of
this Section is guilty of a Class 4 felony.

(d) Paragraph (2) of subsection (b) of this Section does not apply to:

(1) persons employed by a county medical examiner's office or coroner's
office acting within the scope of their employment;

(2) the acts of a licensed funeral director or embalmer while performing
acts authorized by the Funeral Directors and Embalmers Licensing Code;

(3) cemeteries and cemetery personnel while performing acts pursuant to
a bona fide request from the involved cemetery consumer or his or her
heirs, or pursuant to an interment or disinterment permit or a court
order, or as authorized under Section 14.5 of the Cemetery Protection
Act, or any other actions legally authorized for cemetery employees;

(4) the acts of emergency medical personnel or physicians performed in
good faith and according to the usual and customary standards of medical
practice in an attempt to resuscitate a life;

(5) physicians licensed to practice medicine in all of its branches or
holding a visiting professor, physician, or resident permit under the
Medical Practice Act of 1987, performing acts in accordance with usual
and customary standards of medical practice, or a currently enrolled
student in an accredited medical school in furtherance of his or her
education at the accredited medical school; or

(6) removing or carrying away a corpse by the employees, independent
contractors, or other persons designated by the federally designated
organ procurement agency engaged in the organ and tissue procurement
process.

(Source: P.A. 97-1072, eff. 8-24-12 .)

\hypertarget{ilcs-512-21-from-ch.-38-par.-12-21}{%
\subsection*{(720 ILCS 5/12-21) (from Ch. 38, par.
12-21)}\label{ilcs-512-21-from-ch.-38-par.-12-21}}
\addcontentsline{toc}{subsection}{(720 ILCS 5/12-21) (from Ch. 38, par.
12-21)}

\hypertarget{sec.-12-21.-repealed.}{%
\section*{Sec. 12-21. (Repealed).}\label{sec.-12-21.-repealed.}}
\addcontentsline{toc}{section}{Sec. 12-21. (Repealed).}

\markright{Sec. 12-21. (Repealed).}

(Source: P.A. 97-227, eff. 1-1-12. Repealed by P.A. 96-1551, eff.
7-1-11.)

\hypertarget{ilcs-512-21.5}{%
\subsection*{(720 ILCS 5/12-21.5)}\label{ilcs-512-21.5}}
\addcontentsline{toc}{subsection}{(720 ILCS 5/12-21.5)}

(This Section was renumbered as Section 12C-10 by P.A. 97-1109.)

\hypertarget{sec.-12-21.5.-renumbered.}{%
\section*{Sec. 12-21.5. (Renumbered).}\label{sec.-12-21.5.-renumbered.}}
\addcontentsline{toc}{section}{Sec. 12-21.5. (Renumbered).}

\markright{Sec. 12-21.5. (Renumbered).}

(Source: P.A. 92-408, eff. 8-17-01; 92-432, eff. 8-17-01. Renumbered by
P.A. 97-1109, eff. 1-1-13.)

\hypertarget{ilcs-512-21.6}{%
\subsection*{(720 ILCS 5/12-21.6)}\label{ilcs-512-21.6}}
\addcontentsline{toc}{subsection}{(720 ILCS 5/12-21.6)}

(This Section was renumbered as Section 12C-5 by P.A. 97-1109.)

\hypertarget{sec.-12-21.6.-renumbered.}{%
\section*{Sec. 12-21.6. (Renumbered).}\label{sec.-12-21.6.-renumbered.}}
\addcontentsline{toc}{section}{Sec. 12-21.6. (Renumbered).}

\markright{Sec. 12-21.6. (Renumbered).}

(Source: P.A. 92-408, eff. 8-17-01; 92-432, eff. 8-17-01; 92-515, eff.
6-1-02; 92-651, eff. 7-11-02

. Renumbered by 97-1109, eff. 1-1-13.)

\hypertarget{ilcs-512-21.6-5}{%
\subsection*{(720 ILCS 5/12-21.6-5)}\label{ilcs-512-21.6-5}}
\addcontentsline{toc}{subsection}{(720 ILCS 5/12-21.6-5)}

\hypertarget{sec.-12-21.6-5.-parent-or-guardian-leaving-custody-or-control-of-child-with-child-sex-offender.}{%
\section*{Sec. 12-21.6-5. Parent or guardian leaving custody or control
of child with child sex
offender.}\label{sec.-12-21.6-5.-parent-or-guardian-leaving-custody-or-control-of-child-with-child-sex-offender.}}
\addcontentsline{toc}{section}{Sec. 12-21.6-5. Parent or guardian
leaving custody or control of child with child sex offender.}

\markright{Sec. 12-21.6-5. Parent or guardian leaving custody or control
of child with child sex offender.}

(a) For the purposes of this Section, ``minor'' means a person under 18
years of age; and ``child sex offender'' means a sex offender who is
required to register under the Sex Offender Registration Act and is a
child sex offender as defined in Sections 11-9.3 and 11-9.4 of this
Code.

(b) It is unlawful for a parent or guardian of a minor to knowingly
leave that minor in the custody or control of a child sex offender, or
allow the child sex offender unsupervised access to the minor.

(c) This Section does not apply to leaving the minor in the custody or
control of, or allowing unsupervised access to the minor by:

(1) a child sex offender who is the parent of the minor;

(2) a person convicted of a violation of subsection

(c) of Section 12-15 of this Code; or

(3) a child sex offender who is married to and living in the same
household with the parent or guardian of the minor.

This subsection (c) shall not be construed to allow a child sex offender
to knowingly reside within 500 feet of the minor victim of the sex
offense if prohibited by subsection (b-6) of Section 11-9.4 of this
Code.

(d) Sentence. A person who violates this Section is guilty of a Class A
misdemeanor.

(e) Nothing in this Section shall prohibit the filing of a petition or
the instituting of any proceeding under Article II of the Juvenile Court
Act of 1987 relating to abused minors.

(Source: P.A. 96-1094, eff. 1-1-11.)

\hypertarget{ilcs-512-21.7}{%
\subsection*{(720 ILCS 5/12-21.7)}\label{ilcs-512-21.7}}
\addcontentsline{toc}{subsection}{(720 ILCS 5/12-21.7)}

\hypertarget{sec.-12-21.7.-repealed.}{%
\section*{Sec. 12-21.7. (Repealed).}\label{sec.-12-21.7.-repealed.}}
\addcontentsline{toc}{section}{Sec. 12-21.7. (Repealed).}

\markright{Sec. 12-21.7. (Repealed).}

(Source: P.A. 94-12, eff. 1-1-06. Repealed by P.A. 97-1109, eff.
1-1-13.)

\hypertarget{ilcs-512-22}{%
\subsection*{(720 ILCS 5/12-22)}\label{ilcs-512-22}}
\addcontentsline{toc}{subsection}{(720 ILCS 5/12-22)}

(This Section was renumbered as Section 12C-15 by P.A. 97-1109.)

\hypertarget{sec.-12-22.-renumbered.}{%
\section*{Sec. 12-22. (Renumbered).}\label{sec.-12-22.-renumbered.}}
\addcontentsline{toc}{section}{Sec. 12-22. (Renumbered).}

\markright{Sec. 12-22. (Renumbered).}

(Source: P.A. 88-479. Renumbered by P.A. 97-1109, eff. 1-1-13.)

\hypertarget{ilcs-512-30-from-ch.-38-par.-12-30}{%
\subsection*{(720 ILCS 5/12-30) (from Ch. 38, par.
12-30)}\label{ilcs-512-30-from-ch.-38-par.-12-30}}
\addcontentsline{toc}{subsection}{(720 ILCS 5/12-30) (from Ch. 38, par.
12-30)}

(This Section was renumbered as Section 12-3.4 by P.A. 96-1551.)

\hypertarget{sec.-12-30.-renumbered.}{%
\section*{Sec. 12-30. (Renumbered).}\label{sec.-12-30.-renumbered.}}
\addcontentsline{toc}{section}{Sec. 12-30. (Renumbered).}

\markright{Sec. 12-30. (Renumbered).}

(Source: P.A. 97-311, eff. 8-11-11. Renumbered by P.A. 96-1551, Article
1, Section 5, eff. 7-1-11.)

\hypertarget{ilcs-512-31-from-ch.-38-par.-12-31}{%
\subsection*{(720 ILCS 5/12-31) (from Ch. 38, par.
12-31)}\label{ilcs-512-31-from-ch.-38-par.-12-31}}
\addcontentsline{toc}{subsection}{(720 ILCS 5/12-31) (from Ch. 38, par.
12-31)}

(This Section was renumbered as Section 12-34.5 by P.A. 96-1551.)

\hypertarget{sec.-12-31.-renumbered.}{%
\section*{Sec. 12-31. (Renumbered).}\label{sec.-12-31.-renumbered.}}
\addcontentsline{toc}{section}{Sec. 12-31. (Renumbered).}

\markright{Sec. 12-31. (Renumbered).}

(Source: P.A. 88-392. Renumbered by P.A. 96-1551, eff. 7-1-11 .)

\hypertarget{ilcs-512-32-from-ch.-38-par.-12-32}{%
\subsection*{(720 ILCS 5/12-32) (from Ch. 38, par.
12-32)}\label{ilcs-512-32-from-ch.-38-par.-12-32}}
\addcontentsline{toc}{subsection}{(720 ILCS 5/12-32) (from Ch. 38, par.
12-32)}

\hypertarget{sec.-12-32.-ritual-mutilation.}{%
\section*{Sec. 12-32. Ritual
mutilation.}\label{sec.-12-32.-ritual-mutilation.}}
\addcontentsline{toc}{section}{Sec. 12-32. Ritual mutilation.}

\markright{Sec. 12-32. Ritual mutilation.}

(a) A person commits ritual mutilation when he or she knowingly
mutilates, dismembers or tortures another person as part of a ceremony,
rite, initiation, observance, performance or practice, and the victim
did not consent or under such circumstances that the defendant knew or
should have known that the victim was unable to render effective
consent.

(b) Ritual mutilation does not include the practice of male circumcision
or a ceremony, rite, initiation, observance, or performance related
thereto.

(c) Sentence. Ritual mutilation is a Class 2 felony.

(Source: P.A. 96-1551, eff. 7-1-11 .)

\hypertarget{ilcs-512-33-from-ch.-38-par.-12-33}{%
\subsection*{(720 ILCS 5/12-33) (from Ch. 38, par.
12-33)}\label{ilcs-512-33-from-ch.-38-par.-12-33}}
\addcontentsline{toc}{subsection}{(720 ILCS 5/12-33) (from Ch. 38, par.
12-33)}

\hypertarget{sec.-12-33.-ritualized-abuse-of-a-child.}{%
\section*{Sec. 12-33. Ritualized abuse of a
child.}\label{sec.-12-33.-ritualized-abuse-of-a-child.}}
\addcontentsline{toc}{section}{Sec. 12-33. Ritualized abuse of a child.}

\markright{Sec. 12-33. Ritualized abuse of a child.}

(a) A person commits ritualized abuse of a child when he or she
knowingly commits any of the following acts with, upon, or in the
presence of a child as part of a ceremony, rite or any similar
observance:

(1) actually or in simulation, tortures, mutilates, or sacrifices any
warm-blooded animal or human being;

(2) forces ingestion, injection or other application of any narcotic,
drug, hallucinogen or anaesthetic for the purpose of dulling
sensitivity, cognition, recollection of, or resistance to any criminal
activity;

(3) forces ingestion, or external application, of human or animal urine,
feces, flesh, blood, bones, body secretions, nonprescribed drugs or
chemical compounds;

(4) involves the child in a mock, unauthorized or unlawful marriage
ceremony with another person or representation of any force or deity,
followed by sexual contact with the child;

(5) places a living child into a coffin or open grave containing a human
corpse or remains;

(6) threatens death or serious harm to a child, his or her parents,
family, pets, or friends that instills a well-founded fear in the child
that the threat will be carried out; or

(7) unlawfully dissects, mutilates, or incinerates a human corpse.

(b) The provisions of this Section shall not be construed to apply to:

(1) lawful agricultural, animal husbandry, food preparation, or wild
game hunting and fishing practices and specifically the branding or
identification of livestock;

(2) the lawful medical practice of male circumcision or any ceremony
related to male circumcision;

(3) any state or federally approved, licensed, or funded research
project; or

(4) the ingestion of animal flesh or blood in the performance of a
religious service or ceremony.

(b-5) For the purposes of this Section, ``child'' means any person under
18 years of age.

(c) Ritualized abuse of a child is a Class 1 felony for a first offense.
A second or subsequent conviction for ritualized abuse of a child is a
Class X felony for which an offender who has attained the age of 18
years at the time of the commission of the offense may be sentenced to a
term of natural life imprisonment and an offender under the age of 18
years at the time of the commission of the offense shall be sentenced
under Section 5-4.5-105 of the Unified Code of Corrections.

(d) (Blank).

(Source: P.A. 99-69, eff. 1-1-16 .)

\hypertarget{ilcs-512-34}{%
\subsection*{(720 ILCS 5/12-34)}\label{ilcs-512-34}}
\addcontentsline{toc}{subsection}{(720 ILCS 5/12-34)}

\hypertarget{sec.-12-34.-female-genital-mutilation.}{%
\section*{Sec. 12-34. Female genital
mutilation.}\label{sec.-12-34.-female-genital-mutilation.}}
\addcontentsline{toc}{section}{Sec. 12-34. Female genital mutilation.}

\markright{Sec. 12-34. Female genital mutilation.}

(a) Except as otherwise permitted in subsection (b), whoever knowingly
circumcises, excises, or infibulates, in whole or in part, the labia
majora, labia minora, or clitoris of another commits female genital
mutilation. Consent to the procedure by a minor on whom it is performed
or by the minor's parent or guardian is not a defense to a violation of
this Section.

(a-5) A parent, guardian, or other person having physical custody or
control of a child who knowingly facilitates or permits the
circumcision, excision, or infibulation, in whole or in part, of the
labia majora, labia minora, or clitoris of the child commits female
genital mutilation.

(b) A surgical procedure is not a violation of subsection (a) if the
procedure is performed by a physician licensed to practice medicine in
all its branches and:

(1) is necessary to the health of the person on whom it is performed; or

(2) is performed on a person who is in labor or who has just given birth
and is performed for medical purposes connected with that labor or
birth.

(c) Sentence. Female genital mutilation as described in subsection (a)
is a Class X felony. Female genital mutilation as described in
subsection (a-5) is a Class 1 felony.

(Source: P.A. 101-285, eff. 1-1-20 .)

\hypertarget{ilcs-5art.-12-subdiv.-25-heading}{%
\subsection*{(720 ILCS 5/Art. 12, Subdiv. 25
heading)}\label{ilcs-5art.-12-subdiv.-25-heading}}
\addcontentsline{toc}{subsection}{(720 ILCS 5/Art. 12, Subdiv. 25
heading)}

SUBDIVISION 25.

OTHER HARM OFFENSES

(Source: P.A. 96-1551, eff. 7-1-11.)

\hypertarget{ilcs-512-34.5}{%
\subsection*{(720 ILCS 5/12-34.5)}\label{ilcs-512-34.5}}
\addcontentsline{toc}{subsection}{(720 ILCS 5/12-34.5)}

(was 720 ILCS 5/12-31)

\hypertarget{sec.-12-34.5.-inducement-to-commit-suicide.}{%
\section*{Sec. 12-34.5. Inducement to commit
suicide.}\label{sec.-12-34.5.-inducement-to-commit-suicide.}}
\addcontentsline{toc}{section}{Sec. 12-34.5. Inducement to commit
suicide.}

\markright{Sec. 12-34.5. Inducement to commit suicide.}

(a) A person commits inducement to commit suicide when he or she does
either of the following:

(1) Knowingly coerces another to commit suicide and the other person
commits or attempts to commit suicide as a direct result of the
coercion, and he or she exercises substantial control over the other
person through (i) control of the other person's physical location or
circumstances; (ii) use of psychological pressure; or (iii) use of
actual or ostensible religious, political, social, philosophical or
other principles.

(2) With knowledge that another person intends to commit or attempt to
commit suicide, intentionally (i) offers and provides the physical means
by which another person commits or attempts to commit suicide, or (ii)
participates in a physical act by which another person commits or
attempts to commit suicide.

For the purposes of this Section, ``attempts to commit suicide'' means
any act done with the intent to commit suicide and which constitutes a
substantial step toward commission of suicide.

(b) Sentence. Inducement to commit suicide under paragraph (a)(1) when
the other person commits suicide as a direct result of the coercion is a
Class 2 felony. Inducement to commit suicide under paragraph (a)(2) when
the other person commits suicide as a direct result of the assistance
provided is a Class 4 felony. Inducement to commit suicide under
paragraph (a)(1) when the other person attempts to commit suicide as a
direct result of the coercion is a Class 3 felony. Inducement to commit
suicide under paragraph (a)(2) when the other person attempts to commit
suicide as a direct result of the assistance provided is a Class A
misdemeanor.

(c) The lawful compliance or a good-faith attempt at lawful compliance
with the Illinois Living Will Act, the Health Care Surrogate Act, or the
Powers of Attorney for Health Care Law is not inducement to commit
suicide under paragraph (a)(2) of this Section.

(Source: P.A. 96-1551, eff. 7-1-11 .)

\hypertarget{ilcs-512-35}{%
\subsection*{(720 ILCS 5/12-35)}\label{ilcs-512-35}}
\addcontentsline{toc}{subsection}{(720 ILCS 5/12-35)}

\hypertarget{sec.-12-35.-sexual-conduct-or-sexual-contact-with-an-animal.}{%
\section*{Sec. 12-35. Sexual conduct or sexual contact with an
animal.}\label{sec.-12-35.-sexual-conduct-or-sexual-contact-with-an-animal.}}
\addcontentsline{toc}{section}{Sec. 12-35. Sexual conduct or sexual
contact with an animal.}

\markright{Sec. 12-35. Sexual conduct or sexual contact with an animal.}

(a) A person may not knowingly engage in any sexual conduct or sexual
contact with an animal.

(b) A person may not knowingly cause, aid, or abet another person to
engage in any sexual conduct or sexual contact with an animal.

(c) A person may not knowingly permit any sexual conduct or sexual
contact with an animal to be conducted on any premises under his or her
charge or control.

(d) A person may not knowingly engage in, promote, aid, or abet any
activity involving any sexual conduct or sexual contact with an animal
for a commercial or recreational purpose.

(e) Sentence. A person who violates this Section is guilty of a Class 4
felony. A person who violates this Section in the presence of a person
under 18 years of age or causes the animal serious physical injury or
death is guilty of a Class 3 felony.

(f) In addition to the penalty imposed in subsection (e), the court may
order that the defendant do any of the following:

(1) Not harbor animals or reside in any household where animals are
present for a reasonable period of time or permanently, if necessary.

(2) Relinquish and permanently forfeit all animals residing in the
household to a recognized or duly organized animal shelter or humane
society.

(3) Undergo a psychological evaluation and counseling at defendant's
expense.

(4) Reimburse the animal shelter or humane society for any reasonable
costs incurred for the care and maintenance of the animal involved in
the sexual conduct or sexual contact in addition to any animals
relinquished to the animal shelter or humane society.

(g) Nothing in this Section shall be construed to prohibit accepted
animal husbandry practices or accepted veterinary medical practices by a
licensed veterinarian or certified veterinary technician.

(h) If the court has reasonable grounds to believe that a violation of
this Section has occurred, the court may order the seizure of all
animals involved in the alleged violation as a condition of bond of a
person charged with a violation of this Section.

(i) In this Section:

``Animal'' means every creature, either alive or dead, other than a
human being.

``Sexual conduct'' means any knowing touching or fondling by a person,
either directly or through clothing, of the sex organs or anus of an
animal or any transfer or transmission of semen by the person upon any
part of the animal, for the purpose of sexual gratification or arousal
of the person.

``Sexual contact'' means any contact, however slight, between the sex
organ or anus of a person and the sex organ, mouth, or anus of an
animal, or any intrusion, however slight, of any part of the body of the
person into the sex organ or anus of an animal, for the purpose of
sexual gratification or arousal of the person. Evidence of emission of
semen is not required to prove sexual contact.

(Source: P.A. 96-1551, eff. 7-1-11 .)

\hypertarget{ilcs-512-36}{%
\subsection*{(720 ILCS 5/12-36)}\label{ilcs-512-36}}
\addcontentsline{toc}{subsection}{(720 ILCS 5/12-36)}

\hypertarget{sec.-12-36.-possession-of-unsterilized-or-vicious-dogs-by-felons-prohibited.}{%
\section*{Sec. 12-36. Possession of unsterilized or vicious dogs by
felons
prohibited.}\label{sec.-12-36.-possession-of-unsterilized-or-vicious-dogs-by-felons-prohibited.}}
\addcontentsline{toc}{section}{Sec. 12-36. Possession of unsterilized or
vicious dogs by felons prohibited.}

\markright{Sec. 12-36. Possession of unsterilized or vicious dogs by
felons prohibited.}

(a) For a period of 10 years commencing upon the release of a person
from incarceration, it is unlawful for a person convicted of a forcible
felony, a felony violation of the Humane Care for Animals Act, a felony
violation of Section 26-5 or 48-1 of this Code, a felony violation of
Article 24 of this Code, a felony violation of Class 3 or higher of the
Illinois Controlled Substances Act, a felony violation of Class 3 or
higher of the Cannabis Control Act, or a felony violation of Class 2 or
higher of the Methamphetamine Control and Community Protection Act, to
knowingly own, possess, have custody of, or reside in a residence with,
either:

(1) an unspayed or unneutered dog or puppy older than

12 weeks of age; or

(2) irrespective of whether the dog has been spayed or neutered, any dog
that has been determined to be a vicious dog under Section 15 of the
Animal Control Act.

(b) Any dog owned, possessed by, or in the custody of a person convicted
of a felony, as described in subsection (a), must be microchipped for
permanent identification.

(c) Sentence. A person who violates this Section is guilty of a Class A
misdemeanor.

(d) It is an affirmative defense to prosecution under this Section that
the dog in question is neutered or spayed, or that the dog in question
was neutered or spayed within 7 days of the defendant being charged with
a violation of this Section. Medical records from, or the certificate
of, a doctor of veterinary medicine licensed to practice in the State of
Illinois who has personally examined or operated upon the dog,
unambiguously indicating whether the dog in question has been spayed or
neutered, shall be prima facie true and correct, and shall be sufficient
evidence of whether the dog in question has been spayed or neutered.
This subsection (d) is not applicable to any dog that has been
determined to be a vicious dog under Section 15 of the Animal Control
Act.

(Source: P.A. 96-185, eff. 1-1-10; 97-1108, eff. 1-1-13.)

\hypertarget{ilcs-512-37}{%
\subsection*{(720 ILCS 5/12-37)}\label{ilcs-512-37}}
\addcontentsline{toc}{subsection}{(720 ILCS 5/12-37)}

\hypertarget{sec.-12-37.-possession-and-sale-of-caustic-and-noxious-substances.}{%
\section*{Sec. 12-37. Possession and sale of caustic and noxious
substances.}\label{sec.-12-37.-possession-and-sale-of-caustic-and-noxious-substances.}}
\addcontentsline{toc}{section}{Sec. 12-37. Possession and sale of
caustic and noxious substances.}

\markright{Sec. 12-37. Possession and sale of caustic and noxious
substances.}

(a) Except as provided in subsection (b), it is unlawful for any person
knowingly to have in his or her possession or to carry about any of the
substances which are regulated by Title 16 CFR Section 1500.129 of the
Federal Caustic Poison Act and are required to contain the words
``causes severe burns'' as the affirmative statement of principal hazard
on its label.

(b) Provided that the product is not used to threaten, intimidate,
injure, or cause distress to another, the restrictions of subsection (a)
do not apply to:

(1) persons while engaged in the legitimate commercial manufacture,
distribution, storage, or use of the specified substances;

(2) persons while engaged in legitimate scientific or medical research,
study, teaching or treatment involving the use of such substances,
including without limitation physicians, pharmacists, scientists,
teachers, students, and employees of regularly established laboratories,
manufacturing and wholesale pharmacies, retail pharmacies, medical
treatment facilities, schools, colleges, and universities;

(3) persons who have procured any of the specified substances for
medicinal purposes upon a prescription of a physician licensed to
practice medicine in all its branches under the Medical Practice Act of
1987;

(4) commercial or consumer products that contain any of the specified
substances found in subsection (a) including, but not limited to,
batteries;

(5) production agriculture as defined in Section 3-5 of the Use Tax Act;

(6) persons while engaged in the possession or transportation, or both,
of a commercial product containing any of the substances specified in
subsection (a) for retail sale;

(7) persons while engaged in the possession, transportation, or use,
unrelated to a retail sale, of any of the substances specified in
subsection (a); or

(8) persons engaged in the possession, transportation, or use of a
commercial product containing any of the substances specified in
subsection (a).

(c) Sentence. A violation of this Section is a Class 4 felony.

(d) The regulation of the possession and carrying of caustic and noxious
substances under this Section is an exclusive power and function of the
State. A home rule unit may not regulate the possession and carrying of
caustic and noxious substances and any ordinance or local law contrary
to this Section is declared void. This is a denial and limitation of
home rule powers and functions under subsection (h) of Section 6 of
Article VII of the Illinois Constitution.

(Source: P.A. 97-565, eff. 1-1-12.)

\hypertarget{ilcs-512-38}{%
\subsection*{(720 ILCS 5/12-38)}\label{ilcs-512-38}}
\addcontentsline{toc}{subsection}{(720 ILCS 5/12-38)}

\hypertarget{sec.-12-38.-restrictions-on-purchase-or-acquisition-of-corrosive-or-caustic-acid.}{%
\section*{Sec. 12-38. Restrictions on purchase or acquisition of
corrosive or caustic
acid.}\label{sec.-12-38.-restrictions-on-purchase-or-acquisition-of-corrosive-or-caustic-acid.}}
\addcontentsline{toc}{section}{Sec. 12-38. Restrictions on purchase or
acquisition of corrosive or caustic acid.}

\markright{Sec. 12-38. Restrictions on purchase or acquisition of
corrosive or caustic acid.}

(a) A person seeking to purchase a substance which is regulated by Title
16 CFR Section 1500.129 of the Federal Caustic Poison Act and is
required to contain the words ``causes severe burns'' as the affirmative
statement of principal hazard on its label, must prior to taking
possession:

(1) provide a valid driver's license or other government-issued
identification showing the person's name, date of birth, and photograph;
and

(2) sign a log documenting the name and address of the person, date and
time of the transaction, and the brand, product name and net weight of
the item.

(b) Exemption. The requirements of subsection (a) do not apply to
batteries or household products. For the purposes of this Section,
``household product'' means any product which is customarily produced or
distributed for sale for consumption or use, or customarily stored, by
individuals in or about the household, including, but not limited to,
products which are customarily produced and distributed for use in or
about a household as a cleaning agent, drain cleaner, pesticide, epoxy,
paint, stain, or similar substance.

(c) Rules and Regulations. The Illinois State Police shall have the
authority to promulgate rules for the implementation and enforcement of
this Section.

(d) Sentence. Any violation of this Section is a business offense for
which a fine not exceeding \$150 for the first violation, \$500 for the
second violation, or \$1,500 for the third and subsequent violations
within a 12-month period shall be imposed.

(e) Preemption. The regulation of the purchase or acquisition, or both,
of a caustic or corrosive substance and any registry regarding the sale
or possession, or both, of a caustic or corrosive substance is an
exclusive power and function of the State. A home rule unit may not
regulate the purchase or acquisition of caustic or corrosive substances
and any ordinance or local law contrary to this Section is declared
void. This is a denial and limitation of home rule powers and functions
under subsection (h) of Section 6 of Article VII of the Illinois
Constitution.

(Source: P.A. 102-538, eff. 8-20-21.)

\bookmarksetup{startatroot}

\hypertarget{article-12a.-violent-video-games}{%
\chapter*{Article 12a. Violent Video
Games}\label{article-12a.-violent-video-games}}
\addcontentsline{toc}{chapter}{Article 12a. Violent Video Games}

\markboth{Article 12a. Violent Video Games}{Article 12a. Violent Video
Games}

(Source: P.A. 94-315, eff. 1-1-06 .)

\hypertarget{ilcs-512a-1}{%
\subsection*{(720 ILCS 5/12A-1)}\label{ilcs-512a-1}}
\addcontentsline{toc}{subsection}{(720 ILCS 5/12A-1)}

\hypertarget{sec.-12a-1.-short-title.}{%
\section*{Sec. 12A-1. Short title.}\label{sec.-12a-1.-short-title.}}
\addcontentsline{toc}{section}{Sec. 12A-1. Short title.}

\markright{Sec. 12A-1. Short title.}

This Article may be cited as the Violent Video Games Law.

(Source: P.A. 94-315, eff. 1-1-06 .)

\hypertarget{ilcs-512a-5}{%
\subsection*{(720 ILCS 5/12A-5)}\label{ilcs-512a-5}}
\addcontentsline{toc}{subsection}{(720 ILCS 5/12A-5)}

\hypertarget{sec.-12a-5.-findings.}{%
\section*{Sec. 12A-5. Findings.}\label{sec.-12a-5.-findings.}}
\addcontentsline{toc}{section}{Sec. 12A-5. Findings.}

\markright{Sec. 12A-5. Findings.}

(a) The General Assembly finds that minors who play violent video games
are more likely to:

(1) Exhibit violent, asocial, or aggressive behavior.

(2) Experience feelings of aggression.

(3) Experience a reduction of activity in the frontal lobes of the brain
which is responsible for controlling behavior.

(b) While the video game industry has adopted its own voluntary
standards describing which games are appropriate for minors, those
standards are not adequately enforced.

(c) Minors are capable of purchasing and do purchase violent video
games.

(d) The State has a compelling interest in assisting parents in
protecting their minor children from violent video games.

(e) The State has a compelling interest in preventing violent,
aggressive, and asocial behavior.

(f) The State has a compelling interest in preventing psychological harm
to minors who play violent video games.

(g) The State has a compelling interest in eliminating any societal
factors that may inhibit the physiological and neurological development
of its youth.

(h) The State has a compelling interest in facilitating the maturation
of Illinois' children into law-abiding, productive adults.

(Source: P.A. 94-315, eff. 1-1-06 .)

\hypertarget{ilcs-512a-10}{%
\subsection*{(720 ILCS 5/12A-10)}\label{ilcs-512a-10}}
\addcontentsline{toc}{subsection}{(720 ILCS 5/12A-10)}

\hypertarget{sec.-12a-10.-definitions.}{%
\section*{Sec. 12A-10. Definitions.}\label{sec.-12a-10.-definitions.}}
\addcontentsline{toc}{section}{Sec. 12A-10. Definitions.}

\markright{Sec. 12A-10. Definitions.}

For the purposes of this Article, the following terms have the following
meanings:

(a) ``Video game retailer'' means a person who sells or rents video
games to the public.

(b) ``Video game'' means an object or device that stores recorded data
or instructions, receives data or instructions generated by a person who
uses it, and, by processing the data or instructions, creates an
interactive game capable of being played, viewed, or experienced on or
through a computer, gaming system, console, or other technology.

(c) ``Minor'' means a person under 18 years of age.

(d) ``Person'' includes but is not limited to an individual,
corporation, partnership, and association.

(e) ``Violent'' video games include depictions of or simulations of
human-on-human violence in which the player kills or otherwise causes
serious physical harm to another human. ``Serious physical harm''
includes depictions of death, dismemberment, amputation, decapitation,
maiming, disfigurement, mutilation of body parts, or rape.

(Source: P.A. 94-315, eff. 1-1-06 .)

\hypertarget{ilcs-512a-15}{%
\subsection*{(720 ILCS 5/12A-15)}\label{ilcs-512a-15}}
\addcontentsline{toc}{subsection}{(720 ILCS 5/12A-15)}

\hypertarget{sec.-12a-15.-restricted-sale-or-rental-of-violent-video-games.}{%
\section*{Sec. 12A-15. Restricted sale or rental of violent video
games.}\label{sec.-12a-15.-restricted-sale-or-rental-of-violent-video-games.}}
\addcontentsline{toc}{section}{Sec. 12A-15. Restricted sale or rental of
violent video games.}

\markright{Sec. 12A-15. Restricted sale or rental of violent video
games.}

(a) A person who sells, rents, or permits to be sold or rented, any
violent video game to any minor, commits a petty offense for which a
fine of \$1,000 may be imposed.

(b) A person who sells, rents, or permits to be sold or rented any
violent video game via electronic scanner must program the electronic
scanner to prompt sales clerks to check identification before the sale
or rental transaction is completed. A person who violates this
subsection (b) commits a petty offense for which a fine of \$1,000 may
be imposed.

(c) A person may not sell or rent, or permit to be sold or rented, any
violent video game through a self-scanning checkout mechanism. A person
who violates this subsection (c) commits a petty offense for which a
fine of \$1,000 may be imposed.

(d) A retail sales clerk shall not be found in violation of this Section
unless he or she has complete knowledge that the party to whom he or she
sold or rented a violent video game was a minor and the clerk sold or
rented the video game to the minor with the specific intent to do so.

(Source: P.A. 94-315, eff. 1-1-06 .)

\hypertarget{ilcs-512a-20}{%
\subsection*{(720 ILCS 5/12A-20)}\label{ilcs-512a-20}}
\addcontentsline{toc}{subsection}{(720 ILCS 5/12A-20)}

\hypertarget{sec.-12a-20.-affirmative-defenses.}{%
\section*{Sec. 12A-20. Affirmative
defenses.}\label{sec.-12a-20.-affirmative-defenses.}}
\addcontentsline{toc}{section}{Sec. 12A-20. Affirmative defenses.}

\markright{Sec. 12A-20. Affirmative defenses.}

In any prosecution arising under this Article, it is an affirmative
defense:

(1) that the defendant was a family member of the minor for whom the
video game was purchased. ``Family member'' for the purpose of this
Section, includes a parent, sibling, grandparent, aunt, uncle, or first
cousin;

(2) that the minor who purchased the video game exhibited a draft card,
driver's license, birth certificate or other official or apparently
official document purporting to establish that the minor was 18 years of
age or older, which the defendant reasonably relied on and reasonably
believed to be authentic;

(3) for the video game retailer, if the retail sales clerk had complete
knowledge that the party to whom he or she sold or rented a violent
video game was a minor and the clerk sold or rented the video game to
the minor with the specific intent to do so; or

(4) that the video game sold or rented was pre-packaged and rated EC,
E10+, E, or T by the Entertainment Software Ratings Board.

(Source: P.A. 94-315, eff. 1-1-06 .)

\hypertarget{ilcs-512a-25}{%
\subsection*{(720 ILCS 5/12A-25)}\label{ilcs-512a-25}}
\addcontentsline{toc}{subsection}{(720 ILCS 5/12A-25)}

\hypertarget{sec.-12a-25.-labeling-of-violent-video-games.}{%
\section*{Sec. 12A-25. Labeling of violent video
games.}\label{sec.-12a-25.-labeling-of-violent-video-games.}}
\addcontentsline{toc}{section}{Sec. 12A-25. Labeling of violent video
games.}

\markright{Sec. 12A-25. Labeling of violent video games.}

(a) Video game retailers shall label all violent video games as defined
in this Article, with a solid white ``18'' outlined in black. The ``18''
shall have dimensions of no less than 2 inches by 2 inches. The ``18''
shall be displayed on the front face of the video game package.

(b) A retailer's failure to comply with this Section is a petty offense
punishable by a fine of \$500 for the first 3 violations, and \$1,000
for every subsequent violation.

(Source: P.A. 94-315, eff. 1-1-06 .)

\bookmarksetup{startatroot}

\hypertarget{article-12b.-sexually-explicit-video-games}{%
\chapter*{Article 12b. Sexually Explicit Video
Games}\label{article-12b.-sexually-explicit-video-games}}
\addcontentsline{toc}{chapter}{Article 12b. Sexually Explicit Video
Games}

\markboth{Article 12b. Sexually Explicit Video Games}{Article 12b.
Sexually Explicit Video Games}

(Source: P.A. 94-315, eff. 1-1-06 .)

\hypertarget{ilcs-512b-1}{%
\subsection*{(720 ILCS 5/12B-1)}\label{ilcs-512b-1}}
\addcontentsline{toc}{subsection}{(720 ILCS 5/12B-1)}

\hypertarget{sec.-12b-1.-short-title.}{%
\section*{Sec. 12B-1. Short title.}\label{sec.-12b-1.-short-title.}}
\addcontentsline{toc}{section}{Sec. 12B-1. Short title.}

\markright{Sec. 12B-1. Short title.}

This Article may be cited as the

Sexually Explicit Video Games Law.

(Source: P.A. 94-315, eff. 1-1-06 .)

\hypertarget{ilcs-512b-5}{%
\subsection*{(720 ILCS 5/12B-5)}\label{ilcs-512b-5}}
\addcontentsline{toc}{subsection}{(720 ILCS 5/12B-5)}

\hypertarget{sec.-12b-5.-findings.}{%
\section*{Sec. 12B-5. Findings.}\label{sec.-12b-5.-findings.}}
\addcontentsline{toc}{section}{Sec. 12B-5. Findings.}

\markright{Sec. 12B-5. Findings.}

The General Assembly finds sexually explicit video games inappropriate
for minors and that the State has a compelling interest in assisting
parents in protecting their minor children from sexually explicit video
games.

(Source: P.A. 94-315, eff. 1-1-06 .)

\hypertarget{ilcs-512b-10}{%
\subsection*{(720 ILCS 5/12B-10)}\label{ilcs-512b-10}}
\addcontentsline{toc}{subsection}{(720 ILCS 5/12B-10)}

\hypertarget{sec.-12b-10.-definitions.}{%
\section*{Sec. 12B-10. Definitions.}\label{sec.-12b-10.-definitions.}}
\addcontentsline{toc}{section}{Sec. 12B-10. Definitions.}

\markright{Sec. 12B-10. Definitions.}

For the purposes of this Article, the following terms have the following
meanings:

(a) ``Video game retailer'' means a person who sells or rents video
games to the public.

(b) ``Video game'' means an object or device that stores recorded data
or instructions, receives data or instructions generated by a person who
uses it, and, by processing the data or instructions, creates an
interactive game capable of being played, viewed, or experienced on or
through a computer, gaming system, console, or other technology.

(c) ``Minor'' means a person under 18 years of age.

(d) ``Person'' includes but is not limited to an individual,
corporation, partnership, and association.

(e) ``Sexually explicit'' video games include those that the average
person, applying contemporary community standards would find, with
respect to minors, is designed to appeal or pander to the prurient
interest and depict or represent in a manner patently offensive with
respect to minors, an actual or simulated sexual act or sexual contact,
an actual or simulated normal or perverted sexual act or a lewd
exhibition of the genitals or post-pubescent female breast.

(Source: P.A. 94-315, eff. 1-1-06 .)

\hypertarget{ilcs-512b-15}{%
\subsection*{(720 ILCS 5/12B-15)}\label{ilcs-512b-15}}
\addcontentsline{toc}{subsection}{(720 ILCS 5/12B-15)}

\hypertarget{sec.-12b-15.-restricted-sale-or-rental-of-sexually-explicit-video-games.}{%
\section*{Sec. 12B-15. Restricted sale or rental of sexually explicit
video
games.}\label{sec.-12b-15.-restricted-sale-or-rental-of-sexually-explicit-video-games.}}
\addcontentsline{toc}{section}{Sec. 12B-15. Restricted sale or rental of
sexually explicit video games.}

\markright{Sec. 12B-15. Restricted sale or rental of sexually explicit
video games.}

(a) A person who sells, rents, or permits to be sold or rented, any
sexually explicit video game to any minor, commits a petty offense for
which a fine of \$1,000 may be imposed.

(b) A person who sells, rents, or permits to be sold or rented any
sexually explicit video game via electronic scanner must program the
electronic scanner to prompt sales clerks to check identification before
the sale or rental transaction is completed. A person who violates this
subsection (b) commits a petty offense for which a fine of \$1,000 may
be imposed.

(c) A person may not sell or rent, or permit to be sold or rented, any
sexually explicit video game through a self-scanning checkout mechanism.
A person who violates this subsection (c) commits a petty offense for
which a fine of \$1,000 may be imposed.

(d) A retail sales clerk shall not be found in violation of this Section
unless he or she has complete knowledge that the party to whom he or she
sold or rented a sexually explicit video game was a minor and the clerk
sold or rented the video game to the minor with the specific intent to
do so.

(Source: P.A. 94-315, eff. 1-1-06 .)

\hypertarget{ilcs-512b-20}{%
\subsection*{(720 ILCS 5/12B-20)}\label{ilcs-512b-20}}
\addcontentsline{toc}{subsection}{(720 ILCS 5/12B-20)}

\hypertarget{sec.-12b-20.-affirmative-defenses.}{%
\section*{Sec. 12B-20. Affirmative
defenses.}\label{sec.-12b-20.-affirmative-defenses.}}
\addcontentsline{toc}{section}{Sec. 12B-20. Affirmative defenses.}

\markright{Sec. 12B-20. Affirmative defenses.}

In any prosecution arising under this Article, it is an affirmative
defense:

(1) that the defendant was a family member of the minor for whom the
video game was purchased. ``Family member'' for the purpose of this
Section, includes a parent, sibling, grandparent, aunt, uncle, or first
cousin;

(2) that the minor who purchased the video game exhibited a draft card,
driver's license, birth certificate or other official or apparently
official document purporting to establish that the minor was 18 years of
age or older, which the defendant reasonably relied on and reasonably
believed to be authentic;

(3) for the video game retailer, if the retail sales clerk had complete
knowledge that the party to whom he or she sold or rented a violent
video game was a minor and the clerk sold or rented the video game to
the minor with the specific intent to do so; or

(4) that the video game sold or rented was pre-packaged and rated EC,
E10+, E, or T by the Entertainment Software Ratings Board.

(Source: P.A. 94-315, eff. 1-1-06 .)

\hypertarget{ilcs-512b-25}{%
\subsection*{(720 ILCS 5/12B-25)}\label{ilcs-512b-25}}
\addcontentsline{toc}{subsection}{(720 ILCS 5/12B-25)}

\hypertarget{sec.-12b-25.-labeling-of-sexually-explicit-video-games.}{%
\section*{Sec. 12B-25. Labeling of sexually explicit video
games.}\label{sec.-12b-25.-labeling-of-sexually-explicit-video-games.}}
\addcontentsline{toc}{section}{Sec. 12B-25. Labeling of sexually
explicit video games.}

\markright{Sec. 12B-25. Labeling of sexually explicit video games.}

(a) Video game retailers shall label all sexually explicit video games
as defined in this Act, with a solid white ``18'' outlined in black. The
``18'' shall have dimensions of no less than 2 inches by 2 inches. The
``18'' shall be displayed on the front face of the video game package.

(b) A retailer who fails to comply with this Section is guilty of a
petty offense punishable by a fine of \$500 for the first 3 violations,
and \$1,000 for every subsequent violation.

(Source: P.A. 94-315, eff. 1-1-06 .)

\hypertarget{ilcs-512b-30}{%
\subsection*{(720 ILCS 5/12B-30)}\label{ilcs-512b-30}}
\addcontentsline{toc}{subsection}{(720 ILCS 5/12B-30)}

\hypertarget{sec.-12b-30.-posting-notification-of-video-games-rating-system.}{%
\section*{Sec. 12B-30. Posting notification of video games rating
system.}\label{sec.-12b-30.-posting-notification-of-video-games-rating-system.}}
\addcontentsline{toc}{section}{Sec. 12B-30. Posting notification of
video games rating system.}

\markright{Sec. 12B-30. Posting notification of video games rating
system.}

(a) A retailer who sells or rents video games shall post a sign that
notifies customers that a video game rating system, created by the
Entertainment Software Ratings Board, is available to aid in the
selection of a game. The sign shall be prominently posted in, or within
5 feet of, the area in which games are displayed for sale or rental, at
the information desk if one exists, and at the point of purchase.

(b) The lettering of each sign shall be printed, at a minimum, in
36-point type and shall be in black ink against a light colored
background, with dimensions of no less than 18 by 24 inches.

(c) A retailer's failure to comply with this Section is a petty offense
punishable by a fine of \$500 for the first 3 violations, and \$1,000
for every subsequent violation.

(Source: P.A. 94-315, eff. 1-1-06 .)

\hypertarget{ilcs-512b-35}{%
\subsection*{(720 ILCS 5/12B-35)}\label{ilcs-512b-35}}
\addcontentsline{toc}{subsection}{(720 ILCS 5/12B-35)}

\hypertarget{sec.-12b-35.-availability-of-brochure-describing-rating-system.}{%
\section*{Sec. 12B-35. Availability of brochure describing rating
system.}\label{sec.-12b-35.-availability-of-brochure-describing-rating-system.}}
\addcontentsline{toc}{section}{Sec. 12B-35. Availability of brochure
describing rating system.}

\markright{Sec. 12B-35. Availability of brochure describing rating
system.}

(a) A video game retailer shall make available upon request a brochure
to customers that explains the Entertainment Software Ratings Board
ratings system.

(b) A retailer who fails to comply with this Section shall receive the
punishment described in subsection (b) of Section 12B-25.

(Source: P.A. 94-315, eff. 1-1-06 .)

\bookmarksetup{startatroot}

\hypertarget{article-12c.-harms-to-children}{%
\chapter*{Article 12c. Harms To
Children}\label{article-12c.-harms-to-children}}
\addcontentsline{toc}{chapter}{Article 12c. Harms To Children}

\markboth{Article 12c. Harms To Children}{Article 12c. Harms To
Children}

(Source: P.A. 97-1109, eff. 1-1-13.)

\hypertarget{ilcs-5art.-12c-subdiv.-1-heading}{%
\subsection*{(720 ILCS 5/Art. 12C, Subdiv. 1
heading)}\label{ilcs-5art.-12c-subdiv.-1-heading}}
\addcontentsline{toc}{subsection}{(720 ILCS 5/Art. 12C, Subdiv. 1
heading)}

SUBDIVISION 1.

ENDANGERMENT AND NEGLECT OFFENSES

(Source: P.A. 97-1109, eff. 1-1-13.)

\hypertarget{ilcs-512c-5}{%
\subsection*{(720 ILCS 5/12C-5)}\label{ilcs-512c-5}}
\addcontentsline{toc}{subsection}{(720 ILCS 5/12C-5)}

(was 720 ILCS 5/12-21.6)

\hypertarget{sec.-12c-5.-endangering-the-life-or-health-of-a-child.}{%
\section*{Sec. 12C-5. Endangering the life or health of a
child.}\label{sec.-12c-5.-endangering-the-life-or-health-of-a-child.}}
\addcontentsline{toc}{section}{Sec. 12C-5. Endangering the life or
health of a child.}

\markright{Sec. 12C-5. Endangering the life or health of a child.}

(a) A person commits endangering the life or health of a child when he
or she knowingly: (1) causes or permits the life or health of a child
under the age of 18 to be endangered; or (2) causes or permits a child
to be placed in circumstances that endanger the child's life or health.
It is not a violation of this Section for a person to relinquish a child
in accordance with the Abandoned Newborn Infant Protection Act.

(b) A trier of fact may infer that a child 6 years of age or younger is
unattended if that child is left in a motor vehicle for more than 10
minutes.

(c) ``Unattended'' means either: (i) not accompanied by a person 14
years of age or older; or (ii) if accompanied by a person 14 years of
age or older, out of sight of that person.

(d) Sentence. A violation of this Section is a Class A misdemeanor. A
second or subsequent violation of this Section is a Class 3 felony. A
violation of this Section that is a proximate cause of the death of the
child is a Class 3 felony for which a person, if sentenced to a term of
imprisonment, shall be sentenced to a term of not less than 2 years and
not more than 10 years. A parent, who is found to be in violation of
this Section with respect to his or her child, may be sentenced to
probation for this offense pursuant to Section 12C-15.

(Source:

P.A. 97-1109, eff. 1-1-13.)

\hypertarget{ilcs-512c-10}{%
\subsection*{(720 ILCS 5/12C-10)}\label{ilcs-512c-10}}
\addcontentsline{toc}{subsection}{(720 ILCS 5/12C-10)}

(was 720 ILCS 5/12-21.5)

\hypertarget{sec.-12c-10.-child-abandonment.}{%
\section*{Sec. 12C-10. Child
abandonment.}\label{sec.-12c-10.-child-abandonment.}}
\addcontentsline{toc}{section}{Sec. 12C-10. Child abandonment.}

\markright{Sec. 12C-10. Child abandonment.}

(a) A person commits child abandonment when he or she, as a parent,
guardian, or other person having physical custody or control of a child,
without regard for the mental or physical health, safety, or welfare of
that child, knowingly leaves that child who is under the age of 13
without supervision by a responsible person over the age of 14 for a
period of 24 hours or more. It is not a violation of this Section for a
person to relinquish a child in accordance with the Abandoned Newborn
Infant Protection Act.

(b) For the purposes of determining whether the child was left without
regard for the mental or physical health, safety, or welfare of that
child, the trier of fact shall consider the following factors:

(1) the age of the child;

(2) the number of children left at the location;

(3) special needs of the child, including whether the child is a person
with a physical or mental disability, or otherwise in need of ongoing
prescribed medical treatment such as periodic doses of insulin or other
medications;

(4) the duration of time in which the child was left without
supervision;

(5) the condition and location of the place where the child was left
without supervision;

(6) the time of day or night when the child was left without
supervision;

(7) the weather conditions, including whether the child was left in a
location with adequate protection from the natural elements such as
adequate heat or light;

(8) the location of the parent, guardian, or other person having
physical custody or control of the child at the time the child was left
without supervision, the physical distance the child was from the
parent, guardian, or other person having physical custody or control of
the child at the time the child was without supervision;

(9) whether the child's movement was restricted, or the child was
otherwise locked within a room or other structure;

(10) whether the child was given a phone number of a person or location
to call in the event of an emergency and whether the child was capable
of making an emergency call;

(11) whether there was food and other provision left for the child;

(12) whether any of the conduct is attributable to economic hardship or
illness and the parent, guardian or other person having physical custody
or control of the child made a good faith effort to provide for the
health and safety of the child;

(13) the age and physical and mental capabilities of the person or
persons who provided supervision for the child;

(14) any other factor that would endanger the health or safety of that
particular child;

(15) whether the child was left under the supervision of another person.

(c) Child abandonment is a Class 4 felony. A second or subsequent
offense after a prior conviction is a Class 3 felony. A parent, who is
found to be in violation of this Section with respect to his or her
child, may be sentenced to probation for this offense pursuant to
Section 12C-15.

(Source: P.A. 98-756, eff. 7-16-14; 99-143, eff. 7-27-15.)

\hypertarget{ilcs-512c-15}{%
\subsection*{(720 ILCS 5/12C-15)}\label{ilcs-512c-15}}
\addcontentsline{toc}{subsection}{(720 ILCS 5/12C-15)}

(was 720 ILCS 5/12-22)

\hypertarget{sec.-12c-15.-child-abandonment-or-endangerment-probation.}{%
\section*{Sec. 12C-15. Child abandonment or endangerment;
probation.}\label{sec.-12c-15.-child-abandonment-or-endangerment-probation.}}
\addcontentsline{toc}{section}{Sec. 12C-15. Child abandonment or
endangerment; probation.}

\markright{Sec. 12C-15. Child abandonment or endangerment; probation.}

(a) Whenever a parent of a child as determined by the court on the facts
before it, pleads guilty to or is found guilty of, with respect to his
or her child, child abandonment under Section 12C-10 of this Article or
endangering the life or health of a child under Section 12C-5 of this
Article, the court may, without entering a judgment of guilt and with
the consent of the person, defer further proceedings and place the
person upon probation upon the reasonable terms and conditions as the
court may require. At least one term of the probation shall require the
person to cooperate with the Department of Children and Family Services
at the times and in the programs that the Department of Children and
Family Services may require.

(b) Upon fulfillment of the terms and conditions imposed under
subsection (a), the court shall discharge the person and dismiss the
proceedings. Discharge and dismissal under this Section shall be without
court adjudication of guilt and shall not be considered a conviction for
purposes of disqualification or disabilities imposed by law upon
conviction of a crime. However, a record of the disposition shall be
reported by the clerk of the circuit court to the Illinois State Police
under Section 2.1 of the Criminal Identification Act, and the record
shall be maintained and provided to any civil authority in connection
with a determination of whether the person is an acceptable candidate
for the care, custody and supervision of children.

(c) Discharge and dismissal under this Section may occur only once.

(d) Probation under this Section may not be for a period of less than 2
years.

(e) If the child dies of the injuries alleged, this Section shall be
inapplicable.

(Source: P.A. 102-538, eff. 8-20-21.)

\hypertarget{ilcs-512c-20}{%
\subsection*{(720 ILCS 5/12C-20)}\label{ilcs-512c-20}}
\addcontentsline{toc}{subsection}{(720 ILCS 5/12C-20)}

\hypertarget{sec.-12c-20.-abandonment-of-a-school-bus-containing-children.}{%
\section*{Sec. 12C-20. Abandonment of a school bus containing
children.}\label{sec.-12c-20.-abandonment-of-a-school-bus-containing-children.}}
\addcontentsline{toc}{section}{Sec. 12C-20. Abandonment of a school bus
containing children.}

\markright{Sec. 12C-20. Abandonment of a school bus containing
children.}

(a) A school bus driver commits abandonment of a school bus containing
children when he or she knowingly abandons the school bus while it
contains any children who are without other adult supervision, except in
an emergency where the driver is seeking help or otherwise acting in the
best interests of the children.

(b) Sentence. A violation of this Section is a Class A misdemeanor for a
first offense, and a Class 4 felony for a second or subsequent offense.

(Source: P.A. 97-1109, eff. 1-1-13.)

\hypertarget{ilcs-512c-25}{%
\subsection*{(720 ILCS 5/12C-25)}\label{ilcs-512c-25}}
\addcontentsline{toc}{subsection}{(720 ILCS 5/12C-25)}

\hypertarget{sec.-12c-25.-contributing-to-the-dependency-and-neglect-of-a-minor.}{%
\section*{Sec. 12C-25. Contributing to the dependency and neglect of a
minor.}\label{sec.-12c-25.-contributing-to-the-dependency-and-neglect-of-a-minor.}}
\addcontentsline{toc}{section}{Sec. 12C-25. Contributing to the
dependency and neglect of a minor.}

\markright{Sec. 12C-25. Contributing to the dependency and neglect of a
minor.}

(a) Any parent, legal guardian or person having the custody of a child
under the age of 18 years commits contributing to the dependency and
neglect of a minor when he or she knowingly: (1) causes, aids, or
encourages such minor to be or to become a dependent and neglected
minor; (2) does acts which directly tend to render any such minor so
dependent and neglected; or (3) fails to do that which will directly
tend to prevent such state of dependency and neglect. It is not a
violation of this Section for a person to relinquish a child in
accordance with the Abandoned Newborn Infant Protection Act.

(b) ``Dependent and neglected minor'' means any child who, while under
the age of 18 years, for any reason is destitute, homeless or abandoned;
or dependent upon the public for support; or has not proper parental
care or guardianship; or habitually begs or receives alms; or is found
living in any house of ill fame or with any vicious or disreputable
person; or has a home which by reason of neglect, cruelty or depravity
on the part of its parents, guardian or any other person in whose care
it may be is an unfit place for such child; and any child who while
under the age of 10 years is found begging, peddling or selling any
articles or singing or playing any musical instrument for gain upon the
street or giving any public entertainments or accompanies or is used in
aid of any person so doing.

(c) Sentence. A violation of this Section is a Class A misdemeanor.

(d) The husband or wife of the defendant shall be a competent witness to
testify in any case under this Section and to all matters relevant
thereto.

(Source: P.A. 97-1109, eff. 1-1-13.)

\hypertarget{ilcs-512c-30}{%
\subsection*{(720 ILCS 5/12C-30)}\label{ilcs-512c-30}}
\addcontentsline{toc}{subsection}{(720 ILCS 5/12C-30)}

(was 720 ILCS 5/33D-1)

\hypertarget{sec.-12c-30.-contributing-to-the-delinquency-or-criminal-delinquency-of-a-minor.}{%
\section*{Sec. 12C-30. Contributing to the delinquency or criminal
delinquency of a
minor.}\label{sec.-12c-30.-contributing-to-the-delinquency-or-criminal-delinquency-of-a-minor.}}
\addcontentsline{toc}{section}{Sec. 12C-30. Contributing to the
delinquency or criminal delinquency of a minor.}

\markright{Sec. 12C-30. Contributing to the delinquency or criminal
delinquency of a minor.}

(a) Contributing to the delinquency of a minor. A person commits
contributing to the delinquency of a minor when he or she knowingly: (1)
causes, aids, or encourages a minor to be or to become a delinquent
minor; or (2) does acts which directly tend to render any minor so
delinquent.

(b) Contributing to the criminal delinquency of a minor. A person of the
age of 21 years and upwards commits contributing to the criminal
delinquency of a minor when he or she, with the intent to promote or
facilitate the commission of an offense solicits, compels or directs a
minor in the commission of the offense that is either: (i) a felony when
the minor is under the age of 17 years; or (ii) a misdemeanor when the
minor is under the age of 18 years.

(c) ``Delinquent minor'' means any minor who prior to his or her 17th
birthday has violated or attempted to violate, regardless of where the
act occurred, any federal or State law or county or municipal ordinance,
and any minor who prior to his or her 18th birthday has violated or
attempted to violate, regardless of where the act occurred, any federal
or State law or county or municipal ordinance classified as a
misdemeanor offense.

(d) Sentence.

(1) A violation of subsection (a) is a Class A misdemeanor.

(2) A violation of subsection (b) is:

(i) a Class C misdemeanor if the offense committed is a petty offense or
a business offense;

(ii) a Class B misdemeanor if the offense committed is a Class C
misdemeanor;

(iii) a Class A misdemeanor if the offense committed is a Class B
misdemeanor;

(iv) a Class 4 felony if the offense committed is a Class A misdemeanor;

(v) a Class 3 felony if the offense committed is a Class 4 felony;

(vi) a Class 2 felony if the offense committed is a Class 3 felony;

(vii) a Class 1 felony if the offense committed is a Class 2 felony; and

(viii) a Class X felony if the offense committed is a Class 1 felony or
a Class X felony.

(3) A violation of subsection (b) incurs the same penalty as first
degree murder if the committed offense is first degree murder.

(e) The husband or wife of the defendant shall be a competent witness to
testify in any case under this Section and to all matters relevant
thereto.

(Source: P.A. 97-1109, eff. 1-1-13.)

\hypertarget{ilcs-5art.-12c-subdiv.-5-heading}{%
\subsection*{(720 ILCS 5/Art. 12C, Subdiv. 5
heading)}\label{ilcs-5art.-12c-subdiv.-5-heading}}
\addcontentsline{toc}{subsection}{(720 ILCS 5/Art. 12C, Subdiv. 5
heading)}

SUBDIVISION 5.

BODILY HARM OFFENSES

(Source: P.A. 97-1109, eff. 1-1-13.)

\hypertarget{ilcs-512c-35}{%
\subsection*{(720 ILCS 5/12C-35)}\label{ilcs-512c-35}}
\addcontentsline{toc}{subsection}{(720 ILCS 5/12C-35)}

(was 720 ILCS 5/12-10)

\hypertarget{sec.-12c-35.-tattooing-the-body-of-a-minor.}{%
\section*{Sec. 12C-35. Tattooing the body of a
minor.}\label{sec.-12c-35.-tattooing-the-body-of-a-minor.}}
\addcontentsline{toc}{section}{Sec. 12C-35. Tattooing the body of a
minor.}

\markright{Sec. 12C-35. Tattooing the body of a minor.}

(a) A person, other than a person licensed to practice medicine in all
its branches, commits tattooing the body of a minor when he or she
knowingly or recklessly tattoos or offers to tattoo a person under the
age of 18.

(b) A person who is an owner or employee of a business that performs
tattooing, other than a person licensed to practice medicine in all its
branches, may not permit a person under 18 years of age to enter or
remain on the premises where tattooing is being performed unless the
person under 18 years of age is accompanied by his or her parent or
legal guardian.

(c) ``Tattoo'' means to insert pigment under the surface of the skin of
a human being, by pricking with a needle or otherwise, so as to produce
an indelible mark or figure visible through the skin.

(d) Subsection (a) of this Section does not apply to a person under 18
years of age who tattoos or offers to tattoo another person under 18
years of age away from the premises of any business at which tattooing
is performed.

(d-5) Subsections (a) and (b) of this Section do not apply to the
removal of a tattoo from a person under 18 years of age, who is a victim
of a violation of Section 10-9 of this Code or who is or has been a
streetgang member as defined in Section 10 of the Illinois Streetgang
Terrorism Omnibus Prevention Act, if the removal of the tattoo is
performed in an establishment or multi-type establishment which has
received a certificate of registration from the Department of Public
Health or its agent under the Tattoo and Body Piercing Establishment
Registration Act and the removal of the tattoo is performed by the
operator or an authorized employee of the operator of the establishment
or multi-type establishment. For the purposes of this subsection (d-5),
``tattoo'' also means the indelible mark or figure visible through the
skin created by tattooing.

(e) Sentence. A violation of this Section is a Class A misdemeanor.

(Source: P.A. 97-1109, eff. 1-1-13; 98-936, eff. 8-15-14.)

\hypertarget{ilcs-512c-40}{%
\subsection*{(720 ILCS 5/12C-40)}\label{ilcs-512c-40}}
\addcontentsline{toc}{subsection}{(720 ILCS 5/12C-40)}

(was 720 ILCS 5/12-10.1)

\hypertarget{sec.-12c-40.-piercing-the-body-of-a-minor.}{%
\section*{Sec. 12C-40. Piercing the body of a
minor.}\label{sec.-12c-40.-piercing-the-body-of-a-minor.}}
\addcontentsline{toc}{section}{Sec. 12C-40. Piercing the body of a
minor.}

\markright{Sec. 12C-40. Piercing the body of a minor.}

(a)(1) A person commits piercing the body of a minor when he or she
knowingly or recklessly pierces the body of a person under 18 years of
age without written consent of a parent or legal guardian of that
person. Before the oral cavity of a person under 18 years of age may be
pierced, the written consent form signed by the parent or legal guardian
must contain a provision in substantially the following form:

``I understand that the oral piercing of the tongue, lips, cheeks, or
any other area of the oral cavity carries serious risk of infection or
damage to the mouth and teeth, or both infection and damage to those
areas, that could result but is not limited to nerve damage, numbness,
and life threatening blood clots.''.

A person who pierces the oral cavity of a person under 18 years of age
without obtaining a signed written consent form from a parent or legal
guardian of the person that includes the provision describing the health
risks of body piercing, violates this Section.

(2) A person who is an owner or employed by a business that performs
body piercing may not permit a person under 18 years of age to enter or
remain on the premises where body piercing is being performed unless the
person under 18 years of age is accompanied by his or her parent or
legal guardian.

(b) ``Pierce'' means to make a hole in the body in order to insert or
allow the insertion of any ring, hoop, stud, or other object for the
purpose of ornamentation of the body. ``Piercing'' does not include
tongue splitting as defined in Section 12-10.2. The term ``body''
includes the oral cavity.

(c) Exceptions. This Section may not be construed in any way to prohibit
any injection, incision, acupuncture, or similar medical or dental
procedure performed by a licensed health care professional or other
person authorized to perform that procedure or the presence on the
premises where that procedure is being performed by a health care
professional or other person authorized to perform that procedure of a
person under 18 years of age who is not accompanied by a parent or legal
guardian. This Section does not prohibit ear piercing. This Section does
not apply to a minor emancipated under the Juvenile Court Act of 1987 or
the Emancipation of Minors Act or by marriage. This Section does not
apply to a person under 18 years of age who pierces the body or oral
cavity of another person under 18 years of age away from the premises of
any business at which body piercing or oral cavity piercing is
performed.

(d) Sentence. A violation of this Section is a Class A misdemeanor.

(Source: P.A. 97-1109, eff. 1-1-13.)

\hypertarget{ilcs-512c-45}{%
\subsection*{(720 ILCS 5/12C-45)}\label{ilcs-512c-45}}
\addcontentsline{toc}{subsection}{(720 ILCS 5/12C-45)}

(was 720 ILCS 5/12-4.9)

\hypertarget{sec.-12c-45.-drug-induced-infliction-of-harm-to-a-child-athlete.}{%
\section*{Sec. 12C-45. Drug induced infliction of harm to a child
athlete.}\label{sec.-12c-45.-drug-induced-infliction-of-harm-to-a-child-athlete.}}
\addcontentsline{toc}{section}{Sec. 12C-45. Drug induced infliction of
harm to a child athlete.}

\markright{Sec. 12C-45. Drug induced infliction of harm to a child
athlete.}

(a) A person commits drug induced infliction of harm to a child athlete
when he or she knowingly distributes a drug to or encourages the
ingestion of a drug by a person under the age of 18 with the intent that
the person under the age of 18 ingest the drug for the purpose of a
quick weight gain or loss in connection with participation in athletics.

(b) This Section does not apply to care under usual and customary
standards of medical practice by a physician licensed to practice
medicine in all its branches or to the sale of drugs or products by a
retail merchant.

(c) Drug induced infliction of harm to a child athlete is a Class A
misdemeanor. A second or subsequent violation is a Class 4 felony.

(Source: P.A. 97-1109, eff. 1-1-13.)

\hypertarget{ilcs-512c-50}{%
\subsection*{(720 ILCS 5/12C-50)}\label{ilcs-512c-50}}
\addcontentsline{toc}{subsection}{(720 ILCS 5/12C-50)}

\hypertarget{sec.-12c-50.-hazing.}{%
\section*{Sec. 12C-50. Hazing.}\label{sec.-12c-50.-hazing.}}
\addcontentsline{toc}{section}{Sec. 12C-50. Hazing.}

\markright{Sec. 12C-50. Hazing.}

(a) A person commits hazing when he or she knowingly requires the
performance of any act by a student or other person in a school,
college, university, or other educational institution of this State, for
the purpose of induction or admission into any group, organization, or
society associated or connected with that institution, if:

(1) the act is not sanctioned or authorized by that educational
institution; and

(2) the act results in bodily harm to any person.

(b) Sentence. Hazing is a Class A misdemeanor, except that hazing that
results in death or great bodily harm is a Class 4 felony.

(Source: P.A. 97-1109, eff. 1-1-13.)

\hypertarget{ilcs-512c-50.1}{%
\subsection*{(720 ILCS 5/12C-50.1)}\label{ilcs-512c-50.1}}
\addcontentsline{toc}{subsection}{(720 ILCS 5/12C-50.1)}

\hypertarget{sec.-12c-50.1.-failure-to-report-hazing.}{%
\section*{Sec. 12C-50.1. Failure to report
hazing.}\label{sec.-12c-50.1.-failure-to-report-hazing.}}
\addcontentsline{toc}{section}{Sec. 12C-50.1. Failure to report hazing.}

\markright{Sec. 12C-50.1. Failure to report hazing.}

(a) For purposes of this Section, ``school official'' includes any and
all paid school administrators, teachers, counselors, support staff, and
coaches and any and all volunteer coaches employed by a school, college,
university, or other educational institution of this State.

(b) A school official commits failure to report hazing when:

(1) while fulfilling his or her official responsibilities as a school
official, he or she personally observes an act which is not sanctioned
or authorized by that educational institution;

(2) the act results in bodily harm to any person; and

(3) the school official knowingly fails to report the act to supervising
educational authorities or, in the event of death or great bodily harm,
to law enforcement.

(c) Sentence. Failure to report hazing is a Class B misdemeanor. If the
act which the person failed to report resulted in death or great bodily
harm, the offense is a Class A misdemeanor.

(d) It is an affirmative defense to a charge of failure to report hazing
under this Section that the person who personally observed the act had a
reasonable apprehension that timely action to stop the act would result
in the imminent infliction of death, great bodily harm, permanent
disfigurement, or permanent disability to that person or another in
retaliation for reporting.

(e) Nothing in this Section shall be construed to allow prosecution of a
person who personally observes the act of hazing and assists with an
investigation and any subsequent prosecution of the offender.

(Source: P.A. 98-393, eff. 8-16-13.)

\hypertarget{ilcs-5art.-12c-subdiv.-10-heading}{%
\subsection*{(720 ILCS 5/Art. 12C, Subdiv. 10
heading)}\label{ilcs-5art.-12c-subdiv.-10-heading}}
\addcontentsline{toc}{subsection}{(720 ILCS 5/Art. 12C, Subdiv. 10
heading)}

SUBDIVISION 10.

CURFEW OFFENSES

(Source: P.A. 97-1109, eff. 1-1-13.)

\hypertarget{ilcs-512c-60}{%
\subsection*{(720 ILCS 5/12C-60)}\label{ilcs-512c-60}}
\addcontentsline{toc}{subsection}{(720 ILCS 5/12C-60)}

(Text of Section before amendment by P.A. 102-982

)

\hypertarget{sec.-12c-60.-curfew.}{%
\section*{Sec. 12C-60. Curfew.}\label{sec.-12c-60.-curfew.}}
\addcontentsline{toc}{section}{Sec. 12C-60. Curfew.}

\markright{Sec. 12C-60. Curfew.}

(a) Curfew offenses.

(1) A minor commits a curfew offense when he or she remains in any
public place or on the premises of any establishment during curfew
hours.

(2) A parent or guardian of a minor or other person in custody or
control of a minor commits a curfew offense when he or she knowingly
permits the minor to remain in any public place or on the premises of
any establishment during curfew hours.

(b) Curfew defenses. It is a defense to prosecution under subsection (a)
that the minor was:

(1) accompanied by the minor's parent or guardian or other person in
custody or control of the minor;

(2) on an errand at the direction of the minor's parent or guardian,
without any detour or stop;

(3) in a motor vehicle involved in interstate travel;

(4) engaged in an employment activity or going to or returning home from
an employment activity, without any detour or stop;

(5) involved in an emergency;

(6) on the sidewalk abutting the minor's residence or abutting the
residence of a next-door neighbor if the neighbor did not complain to
the police department about the minor's presence;

(7) attending an official school, religious, or other recreational
activity supervised by adults and sponsored by a government or
governmental agency, a civic organization, or another similar entity
that takes responsibility for the minor, or going to or returning home
from, without any detour or stop, an official school, religious, or
other recreational activity supervised by adults and sponsored by a
government or governmental agency, a civic organization, or another
similar entity that takes responsibility for the minor;

(8) exercising First Amendment rights protected by the United States
Constitution, such as the free exercise of religion, freedom of speech,
and the right of assembly; or

(9) married or had been married or is an emancipated minor under the
Emancipation of Minors Act.

(c) Enforcement. Before taking any enforcement action under this
Section, a law enforcement officer shall ask the apparent offender's age
and reason for being in the public place. The officer shall not issue a
citation or make an arrest under this Section unless the officer
reasonably believes that an offense has occurred and that, based on any
response and other circumstances, no defense in subsection (b) is
present.

(d) Definitions. In this Section:

(1) ``Curfew hours'' means:

(A) Between 12:01 a.m. and 6:00 a.m. on Saturday;

(B) Between 12:01 a.m. and 6:00 a.m. on Sunday; and

(C) Between 11:00 p.m. on Sunday to Thursday, inclusive, and 6:00 a.m.
on the following day.

(2) ``Emergency'' means an unforeseen combination of circumstances or
the resulting state that calls for immediate action. The term includes,
but is not limited to, a fire, a natural disaster, an automobile
accident, or any situation requiring immediate action to prevent serious
bodily injury or loss of life.

(3) ``Establishment'' means any privately-owned place of business
operated for a profit to which the public is invited, including, but not
limited to, any place of amusement or entertainment.

(4) ``Guardian'' means:

(A) a person who, under court order, is the guardian of the person of a
minor; or

(B) a public or private agency with whom a minor has been placed by a
court.

(5) ``Minor'' means any person under 17 years of age.

(6) ``Parent'' means a person who is:

(A) a natural parent, adoptive parent, or step-parent of another person;
or

(B) at least 18 years of age and authorized by a parent or guardian to
have the care and custody of a minor.

(7) ``Public place'' means any place to which the public or a
substantial group of the public has access and includes, but is not
limited to, streets, highways, and the common areas of schools,
hospitals, apartment houses, office buildings, transport facilities, and
shops.

(8) ``Remain'' means to:

(A) linger or stay; or

(B) fail to leave premises when requested to do so by a police officer
or the owner, operator, or other person in control of the premises.

(9) ``Serious bodily injury'' means bodily injury that creates a
substantial risk of death or that causes death, serious permanent
disfigurement, or protracted loss or impairment of the function of any
bodily member or organ.

(e) Sentence. A violation of this Section is a petty offense with a fine
of not less than \$10 nor more than \$500, except that neither a person
who has been made a ward of the court under the Juvenile Court Act of
1987, nor that person's legal guardian, shall be subject to any fine. In
addition to or instead of the fine imposed by this Section, the court
may order a parent, legal guardian, or other person convicted of a
violation of subsection (a) of this Section to perform community service
as determined by the court, except that the legal guardian of a person
who has been made a ward of the court under the Juvenile Court Act of
1987 may not be ordered to perform community service. The dates and
times established for the performance of community service by the
parent, legal guardian, or other person convicted of a violation of
subsection (a) of this Section shall not conflict with the dates and
times that the person is employed in his or her regular occupation.

(f) County, municipal and other local boards and bodies authorized to
adopt local police laws and regulations under the constitution and laws
of this State may exercise legislative or regulatory authority over this
subject matter by ordinance or resolution incorporating the substance of
this Section or increasing the requirements thereof or otherwise not in
conflict with this Section.

(Source: P.A. 97-1109, eff. 1-1-13.)

(Text of Section after amendment by P.A. 102-982

)

\hypertarget{sec.-12c-60.-curfew.-1}{%
\section*{Sec. 12C-60. Curfew.}\label{sec.-12c-60.-curfew.-1}}
\addcontentsline{toc}{section}{Sec. 12C-60. Curfew.}

\markright{Sec. 12C-60. Curfew.}

(a) Curfew offenses.

(1) A minor commits a curfew offense when he or she remains in any
public place or on the premises of any establishment during curfew
hours.

(2) A parent or guardian of a minor or other person in custody or
control of a minor commits a curfew offense when he or she knowingly
permits the minor to remain in any public place or on the premises of
any establishment during curfew hours.

(b) Curfew defenses. It is a defense to prosecution under subsection (a)
that the minor was:

(1) accompanied by the minor's parent or guardian or other person in
custody or control of the minor;

(2) on an errand at the direction of the minor's parent or guardian,
without any detour or stop;

(3) in a motor vehicle involved in interstate travel;

(4) engaged in an employment activity or going to or returning home from
an employment activity, without any detour or stop;

(5) involved in an emergency;

(6) on the sidewalk abutting the minor's residence or abutting the
residence of a next-door neighbor if the neighbor did not complain to
the police department about the minor's presence;

(7) attending an official school, religious, or other recreational
activity supervised by adults and sponsored by a government or
governmental agency, a civic organization, or another similar entity
that takes responsibility for the minor, or going to or returning home
from, without any detour or stop, an official school, religious, or
other recreational activity supervised by adults and sponsored by a
government or governmental agency, a civic organization, or another
similar entity that takes responsibility for the minor;

(8) exercising First Amendment rights protected by the United States
Constitution, such as the free exercise of religion, freedom of speech,
and the right of assembly; or

(9) married or had been married or is an emancipated minor under the
Emancipation of Minors Act.

(c) Enforcement. Before taking any enforcement action under this
Section, a law enforcement officer shall ask the apparent offender's age
and reason for being in the public place. The officer shall not issue a
citation or make an arrest under this Section unless the officer
reasonably believes that an offense has occurred and that, based on any
response and other circumstances, no defense in subsection (b) is
present.

(d) Definitions. In this Section:

(1) ``Curfew hours'' means:

(A) Between 12:01 a.m. and 6:00 a.m. on Saturday;

(B) Between 12:01 a.m. and 6:00 a.m. on Sunday; and

(C) Between 11:00 p.m. on Sunday to Thursday, inclusive, and 6:00 a.m.
on the following day.

(2) ``Emergency'' means an unforeseen combination of circumstances or
the resulting state that calls for immediate action. The term includes,
but is not limited to, a fire, a natural disaster, an automobile crash,
or any situation requiring immediate action to prevent serious bodily
injury or loss of life.

(3) ``Establishment'' means any privately-owned place of business
operated for a profit to which the public is invited, including, but not
limited to, any place of amusement or entertainment.

(4) ``Guardian'' means:

(A) a person who, under court order, is the guardian of the person of a
minor; or

(B) a public or private agency with whom a minor has been placed by a
court.

(5) ``Minor'' means any person under 17 years of age.

(6) ``Parent'' means a person who is:

(A) a natural parent, adoptive parent, or step-parent of another person;
or

(B) at least 18 years of age and authorized by a parent or guardian to
have the care and custody of a minor.

(7) ``Public place'' means any place to which the public or a
substantial group of the public has access and includes, but is not
limited to, streets, highways, and the common areas of schools,
hospitals, apartment houses, office buildings, transport facilities, and
shops.

(8) ``Remain'' means to:

(A) linger or stay; or

(B) fail to leave premises when requested to do so by a police officer
or the owner, operator, or other person in control of the premises.

(9) ``Serious bodily injury'' means bodily injury that creates a
substantial risk of death or that causes death, serious permanent
disfigurement, or protracted loss or impairment of the function of any
bodily member or organ.

(e) Sentence. A violation of this Section is a petty offense with a fine
of not less than \$10 nor more than \$500, except that neither a person
who has been made a ward of the court under the Juvenile Court Act of
1987, nor that person's legal guardian, shall be subject to any fine. In
addition to or instead of the fine imposed by this Section, the court
may order a parent, legal guardian, or other person convicted of a
violation of subsection (a) of this Section to perform community service
as determined by the court, except that the legal guardian of a person
who has been made a ward of the court under the Juvenile Court Act of
1987 may not be ordered to perform community service. The dates and
times established for the performance of community service by the
parent, legal guardian, or other person convicted of a violation of
subsection (a) of this Section shall not conflict with the dates and
times that the person is employed in his or her regular occupation.

(f) County, municipal and other local boards and bodies authorized to
adopt local police laws and regulations under the constitution and laws
of this State may exercise legislative or regulatory authority over this
subject matter by ordinance or resolution incorporating the substance of
this Section or increasing the requirements thereof or otherwise not in
conflict with this Section.

(Source: P.A. 102-982, eff. 7-1-23.)

\hypertarget{ilcs-5art.-12c-subdiv.-15-heading}{%
\subsection*{(720 ILCS 5/Art. 12C, Subdiv. 15
heading)}\label{ilcs-5art.-12c-subdiv.-15-heading}}
\addcontentsline{toc}{subsection}{(720 ILCS 5/Art. 12C, Subdiv. 15
heading)}

SUBDIVISION 15.

MISCELLANEOUS OFFENSES

(Source: P.A. 97-1109, eff. 1-1-13.)

\hypertarget{ilcs-512c-65}{%
\subsection*{(720 ILCS 5/12C-65)}\label{ilcs-512c-65}}
\addcontentsline{toc}{subsection}{(720 ILCS 5/12C-65)}

(was 720 ILCS 5/44-2 and 5/44-3)

\hypertarget{sec.-12c-65.-unlawful-transfer-of-a-telecommunications-device-to-a-minor.}{%
\section*{Sec. 12C-65. Unlawful transfer of a telecommunications device
to a
minor.}\label{sec.-12c-65.-unlawful-transfer-of-a-telecommunications-device-to-a-minor.}}
\addcontentsline{toc}{section}{Sec. 12C-65. Unlawful transfer of a
telecommunications device to a minor.}

\markright{Sec. 12C-65. Unlawful transfer of a telecommunications device
to a minor.}

(a) A person commits unlawful transfer of a telecommunications device to
a minor when he or she gives, sells or otherwise transfers possession of
a telecommunications device to a person under 18 years of age with the
intent that the device be used to commit any offense under this Code,
the Cannabis Control Act, the Illinois Controlled Substances Act, or the
Methamphetamine Control and Community Protection Act.

(b) ``Telecommunications device'' or ``device'' means a device which is
portable or which may be installed in a motor vehicle, boat or other
means of transportation, and which is capable of receiving or
transmitting speech, data, signals or other information, including but
not limited to paging devices, cellular and mobile telephones, and radio
transceivers, transmitters and receivers, but not including radios
designed to receive only standard AM and FM broadcasts.

(c) Sentence. A violation of this Section is a Class A misdemeanor.

(d) Seizure and forfeiture of property. Any person who commits the
offense of unlawful transfer of a telecommunications device to a minor
as set forth in this Section is subject to the property forfeiture
provisions in Article 124B of the Code of Criminal Procedure of 1963.

(Source: P.A. 97-1109, eff. 1-1-13.)

\hypertarget{ilcs-512c-70}{%
\subsection*{(720 ILCS 5/12C-70)}\label{ilcs-512c-70}}
\addcontentsline{toc}{subsection}{(720 ILCS 5/12C-70)}

\hypertarget{sec.-12c-70.-adoption-compensation-prohibited.}{%
\section*{Sec. 12C-70. Adoption compensation
prohibited.}\label{sec.-12c-70.-adoption-compensation-prohibited.}}
\addcontentsline{toc}{section}{Sec. 12C-70. Adoption compensation
prohibited.}

\markright{Sec. 12C-70. Adoption compensation prohibited.}

(a) Receipt of compensation for placing out prohibited; exception. No
person and no agency, association, corporation, institution, society, or
other organization, except a child welfare agency as defined by the
Child Care Act of 1969, shall knowingly request, receive or accept any
compensation or thing of value, directly or indirectly, for providing
adoption services, as defined in Section 2.24 of the Child Care Act of
1969.

(b) Payment of compensation for placing out prohibited. No person shall
knowingly pay or give any compensation or thing of value, directly or
indirectly, for providing adoption services, as defined in Section 2.24
of the Child Care Act of 1969, including placing out of a child to any
person or to any agency, association, corporation, institution, society,
or other organization except a child welfare agency as defined by the
Child Care Act of 1969.

(c) Certain payments of salaries and medical expenses not prevented.

(1) The provisions of this Section shall not be construed to prevent the
payment of salaries or other compensation by a licensed child welfare
agency providing adoption services, as that term is defined by the Child
Care Act of 1969, to the officers, employees, agents, contractors, or
any other persons acting on behalf of the child welfare agency, provided
that such salaries and compensation are consistent with subsection (a)
of Section 14.5 of the Child Care Act of 1969.

(2) The provisions of this Section shall not be construed to prevent the
payment by a prospective adoptive parent of reasonable and actual
medical fees or hospital charges for services rendered in connection
with the birth of such child, if such payment is made to the physician
or hospital who or which rendered the services or to the biological
mother of the child or to prevent the receipt of such payment by such
physician, hospital, or mother.

(3) The provisions of this Section shall not be construed to prevent a
prospective adoptive parent from giving a gift or gifts or other thing
or things of value to a biological parent provided that the total value
of such gift or gifts or thing or things of value does not exceed \$200.

(d) Payment of certain expenses.

(1) A prospective adoptive parent shall be permitted to pay the
reasonable living expenses of the biological parents of the child sought
to be adopted, in addition to those expenses set forth in subsection
(c), only in accordance with the provisions of this subsection (d).

``Reasonable living expenses'' means those expenses related to
activities of daily living and meeting basic needs, including, but not
limited to, lodging, food, and clothing for the biological parents
during the biological mother's pregnancy and for no more than 120 days
prior to the biological mother's expected date of delivery and for no
more than 60 days after the birth of the child. The term does not
include expenses for lost wages, gifts, educational expenses, or other
similar expenses of the biological parents.

(2)(A) The prospective adoptive parents may seek leave of the court to
pay the reasonable living expenses of the biological parents. They shall
be permitted to pay the reasonable living expenses of the biological
parents only upon prior order of the circuit court where the petition
for adoption will be filed, or if the petition for adoption has been
filed in the circuit court where the petition is pending.

(B) Notwithstanding clause (2)(A) of this subsection

(d), a prospective adoptive parent may advance a maximum of \$1,000 for
reasonable birth parent living expenses without prior order of court.
The prospective adoptive parents shall present a final accounting of all
expenses to the court prior to the entry of a final judgment order for
adoption.

(C) If the court finds an accounting by the prospective adoptive parents
to be incomplete or deceptive or to contain amounts which are
unauthorized or unreasonable, the court may order a new accounting or
the repayment of amounts found to be excessive or unauthorized or make
any other orders it deems appropriate.

(3) Payments under this subsection (d) shall be permitted only in those
circumstances where there is a demonstrated need for the payment of such
expenses to protect the health of the biological parents or the health
of the child sought to be adopted.

(4) Payment of their reasonable living expenses, as provided in this
subsection (d), shall not obligate the biological parents to place the
child for adoption. In the event the biological parents choose not to
place the child for adoption, the prospective adoptive parents shall
have no right to seek reimbursement from the biological parents, or from
any relative or associate of the biological parents, of moneys paid to,
or on behalf of, the biological parents pursuant to a court order under
this subsection (d).

(5) Notwithstanding paragraph (4) of this subsection

(d), a prospective adoptive parent may seek reimbursement of reasonable
living expenses from a person who receives such payments only if the
person who accepts payment of reasonable living expenses before the
child's birth, as described in paragraph (4) of this subsection (d),
knows that the person on whose behalf he or she is accepting payment is
not pregnant at the time of the receipt of such payments or the person
receives reimbursement for reasonable living expenses simultaneously
from more than one prospective adoptive parent without the knowledge of
the prospective adoptive parent.

(6) No person or entity shall offer, provide, or co-sign a loan or any
other credit accommodation, directly or indirectly, with a biological
parent or a relative or associate of a biological parent based on the
contingency of a surrender or placement of a child for adoption.

(7) Within 14 days after the completion of all payments for reasonable
living expenses of the biological parents under this subsection (d), the
prospective adoptive parents shall present a final accounting of all
those expenses to the court. The accounting shall also include the
verified statements of the prospective adoptive parents, each attorney
of record, and the biological parents or parents to whom or on whose
behalf the payments were made attesting to the accuracy of the
accounting.

(8) If the placement of a child for adoption is made in accordance with
the Interstate Compact on the Placement of Children, and if the sending
state permits the payment of any expenses of biological parents that are
not permitted under this Section, then the payment of those expenses
shall not be a violation of this Section. In that event, the prospective
adoptive parents shall file an accounting of all payments of the
expenses of the biological parent or parents with the court in which the
petition for adoption is filed or is to be filed. The accounting shall
include a copy of the statutory provisions of the sending state that
permit payments in addition to those permitted by this Section and a
copy of all orders entered in the sending state that relate to expenses
of the biological parents paid by the prospective adoptive parents in
the sending state.

(9) The prospective adoptive parents shall be permitted to pay the
reasonable attorney's fees of a biological parent's attorney in
connection with proceedings under this Section or in connection with
proceedings for the adoption of the child if the amount of fees of the
attorney is \$1,000 or less. If the amount of attorney's fees of each
biological parent exceeds \$1,000, the attorney's fees shall be paid
only after a petition seeking leave to pay those fees is filed with the
court in which the adoption proceeding is filed or to be filed. The
court shall review the petition for leave to pay attorney's fees, and if
the court determines that the fees requested are reasonable, the court
shall permit the petitioners to pay them. If the court determines that
the fees requested are not reasonable, the court shall determine and set
the reasonable attorney's fees of the biological parents' attorney which
may be paid by the petitioners. The prospective adoptive parents shall
present a final accounting of all those fees to the court prior to the
entry of a final judgment order for adoption.

(10) The court may appoint a guardian ad litem for an unborn child to
represent the interests of the child in proceedings under this
subsection (d).

(11) The provisions of this subsection (d) apply to a person who is a
prospective adoptive parent. This subsection (d) does not apply to a
licensed child welfare agency, as that term is defined in the Child Care
Act of 1969, whose payments are governed by the Child Care Act of 1969
and the Department of Children and Family Services rules adopted
thereunder.

(e) Injunctive relief.

(A) Whenever it appears that any person, agency, association,
corporation, institution, society, or other organization is engaged or
about to engage in any acts or practices that constitute or will
constitute a violation of this Section, the Department of Children and
Family Services shall inform the Attorney General and the State's
Attorney of the appropriate county. Under such circumstances, the
Attorney General or the State's Attorney may initiate injunction
proceedings. Upon a proper showing, any circuit court may enter a
permanent or preliminary injunction or temporary restraining order
without bond to enforce this Section or any rule adopted under this
Section in addition to any other penalties and other remedies provided
in this Section.

(B) Whenever it appears that any person, agency, association,
corporation, institution, society, or other organization is engaged or
is about to engage in any act or practice that constitutes or will
constitute a violation of any rule adopted under the authority of this
Section, the Department of Children and Family Services may inform the
Attorney General and the State's Attorney of the appropriate county.
Under such circumstances, the Attorney General or the State's Attorney
may initiate injunction proceedings. Upon a proper showing, any circuit
court may enter a permanent or preliminary injunction or a temporary
restraining order without bond to enforce this Section or any rule
adopted under this Section, in addition to any other penalties and
remedies provided in this Section.

(f) A violation of this Section on a first conviction is a Class 4
felony, and on a second or subsequent conviction is a Class 3 felony.

(g) ``Adoption services'' has the meaning given that term in the Child
Care Act of 1969.

(h) ``Placing out'' means to arrange for the free care or placement of a
child in a family other than that of the child's parent, stepparent,
grandparent, brother, sister, uncle or aunt or legal guardian, for the
purpose of adoption or for the purpose of providing care.

(i) ``Prospective adoptive parent'' means a person or persons who have
filed or intend to file a petition to adopt a child under the Adoption
Act.

(Source: P.A. 97-1109, eff. 1-1-13.)

\bookmarksetup{startatroot}

\hypertarget{article-14.-eavesdropping}{%
\chapter*{Article 14. Eavesdropping}\label{article-14.-eavesdropping}}
\addcontentsline{toc}{chapter}{Article 14. Eavesdropping}

\markboth{Article 14. Eavesdropping}{Article 14. Eavesdropping}

\hypertarget{ilcs-514-1-from-ch.-38-par.-14-1}{%
\subsection*{(720 ILCS 5/14-1) (from Ch. 38, par.
14-1)}\label{ilcs-514-1-from-ch.-38-par.-14-1}}
\addcontentsline{toc}{subsection}{(720 ILCS 5/14-1) (from Ch. 38, par.
14-1)}

\hypertarget{sec.-14-1.-definitions.}{%
\section*{Sec. 14-1. Definitions.}\label{sec.-14-1.-definitions.}}
\addcontentsline{toc}{section}{Sec. 14-1. Definitions.}

\markright{Sec. 14-1. Definitions.}

(a) Eavesdropping device.

An eavesdropping device is any device capable of being used to hear or
record oral conversation or intercept, or transcribe electronic
communications whether such conversation or electronic communication is
conducted in person, by telephone, or by any other means; Provided,
however, that this definition shall not include devices used for the
restoration of the deaf or hard-of-hearing to normal or partial hearing.

(b) Eavesdropper.

An eavesdropper is any person, including any law enforcement officer and
any party to a private conversation, who operates or participates in the
operation of any eavesdropping device contrary to the provisions of this
Article or who acts as a principal, as defined in this Article.

(c) Principal.

A principal is any person who:

(1) Knowingly employs another who illegally uses an eavesdropping device
in the course of such employment; or

(2) Knowingly derives any benefit or information from the illegal use of
an eavesdropping device by another; or

(3) Directs another to use an eavesdropping device illegally on his or
her behalf.

(d) Private conversation.

For the purposes of this Article, ``private conversation'' means any
oral communication between 2 or more persons, whether in person or
transmitted between the parties by wire or other means, when one or more
of the parties intended the communication to be of a private nature
under circumstances reasonably justifying that expectation. A reasonable
expectation shall include any expectation recognized by law, including,
but not limited to, an expectation derived from a privilege, immunity,
or right established by common law, Supreme Court rule, or the Illinois
or United States Constitution.

(e) Private electronic communication.

For purposes of this Article, ``private electronic communication'' means
any transfer of signs, signals, writing, images, sounds, data, or
intelligence of any nature transmitted in whole or part by a wire,
radio, pager, computer, electromagnetic, photo electronic or photo
optical system, when the sending or receiving party intends the
electronic communication to be private under circumstances reasonably
justifying that expectation. A reasonable expectation shall include any
expectation recognized by law, including, but not limited to, an
expectation derived from a privilege, immunity, or right established by
common law, Supreme Court rule, or the Illinois or United States
Constitution. Electronic communication does not include any
communication from a tracking device.

(f) Bait car.

For purposes of this Article, ``bait car'' means any motor vehicle that
is not occupied by a law enforcement officer and is used by a law
enforcement agency to deter, detect, identify, and assist in the
apprehension of an auto theft suspect in the act of stealing a motor
vehicle.

(g) Surreptitious.

For purposes of this Article, ``surreptitious'' means obtained or made
by stealth or deception, or executed through secrecy or concealment.

(Source: P.A. 98-1142, eff. 12-30-14.)

\hypertarget{ilcs-514-2-from-ch.-38-par.-14-2}{%
\subsection*{(720 ILCS 5/14-2) (from Ch. 38, par.
14-2)}\label{ilcs-514-2-from-ch.-38-par.-14-2}}
\addcontentsline{toc}{subsection}{(720 ILCS 5/14-2) (from Ch. 38, par.
14-2)}

\hypertarget{sec.-14-2.-elements-of-the-offense-affirmative-defense.}{%
\section*{Sec. 14-2. Elements of the offense; affirmative
defense.}\label{sec.-14-2.-elements-of-the-offense-affirmative-defense.}}
\addcontentsline{toc}{section}{Sec. 14-2. Elements of the offense;
affirmative defense.}

\markright{Sec. 14-2. Elements of the offense; affirmative defense.}

(a) A person commits eavesdropping when he or she knowingly and
intentionally:

(1) Uses an eavesdropping device, in a surreptitious manner, for the
purpose of overhearing, transmitting, or recording all or any part of
any private conversation to which he or she is not a party unless he or
she does so with the consent of all of the parties to the private
conversation;

(2) Uses an eavesdropping device, in a surreptitious manner, for the
purpose of transmitting or recording all or any part of any private
conversation to which he or she is a party unless he or she does so with
the consent of all other parties to the private conversation;

(3) Intercepts, records, or transcribes, in a surreptitious manner, any
private electronic communication to which he or she is not a party
unless he or she does so with the consent of all parties to the private
electronic communication;

(4) Manufactures, assembles, distributes, or possesses any electronic,
mechanical, eavesdropping, or other device knowing that or having reason
to know that the design of the device renders it primarily useful for
the purpose of the surreptitious overhearing, transmitting, or recording
of private conversations or the interception, or transcription of
private electronic communications and the intended or actual use of the
device is contrary to the provisions of this Article; or

(5) Uses or discloses any information which he or she knows or
reasonably should know was obtained from a private conversation or
private electronic communication in violation of this Article, unless he
or she does so with the consent of all of the parties.

(a-5) It does not constitute a violation of this Article to
surreptitiously use an eavesdropping device to overhear, transmit, or
record a private conversation, or to surreptitiously intercept, record,
or transcribe a private electronic communication, if the overhearing,
transmitting, recording, interception, or transcription is done in
accordance with Article 108A or Article 108B of the Code of Criminal
Procedure of 1963.

(b) It is an affirmative defense to a charge brought under this Article
relating to the interception of a privileged communication that the
person charged:

\begin{enumerate}
\def\labelenumi{\arabic{enumi}.}
\tightlist
\item
  was a law enforcement officer acting pursuant to an order of
  interception, entered pursuant to Section 108A-1 or 108B-5 of the Code
  of Criminal Procedure of 1963; and

  \begin{enumerate}
  \def\labelenumii{\arabic{enumii}.}
  \setcounter{enumii}{1}
  \tightlist
  \item
    at the time the communication was intercepted, the officer was
    unaware that the communication was privileged; and

    \begin{enumerate}
    \def\labelenumiii{\arabic{enumiii}.}
    \setcounter{enumiii}{2}
    \tightlist
    \item
      stopped the interception within a reasonable time after
      discovering that the communication was privileged; and

      \begin{enumerate}
      \def\labelenumiv{\arabic{enumiv}.}
      \setcounter{enumiv}{3}
      \tightlist
      \item
        did not disclose the contents of the communication.

        (c) It is not unlawful for a manufacturer or a supplier of
        eavesdropping devices, or a provider of wire or electronic
        communication services, their agents, employees, contractors, or
        venders to manufacture, assemble, sell, or possess an
        eavesdropping device within the normal course of their business
        for purposes not contrary to this Article or for law enforcement
        officers and employees of the Illinois Department of Corrections
        to manufacture, assemble, purchase, or possess an eavesdropping
        device in preparation for or within the course of their official
        duties.

        (d) The interception, recording, or transcription of an
        electronic communication by an employee of a penal institution
        is not prohibited under this Act, provided that the
        interception, recording, or transcription is:

        (1) otherwise legally permissible under Illinois law;

        (2) conducted with the approval of the penal institution for the
        purpose of investigating or enforcing a State criminal law or a
        penal institution rule or regulation with respect to inmates in
        the institution; and

        (3) within the scope of the employee's official duties.

        For the purposes of this subsection (d), ``penal institution''
        has the meaning ascribed to it in clause (c)(1) of Section
        31A-1.1.

        (e) Nothing in this Article shall prohibit any individual, not a
        law enforcement officer, from recording a law enforcement
        officer in the performance of his or her duties in a public
        place or in circumstances in which the officer has no reasonable
        expectation of privacy. However, an officer may take reasonable
        action to maintain safety and control, secure crime scenes and
        accident sites, protect the integrity and confidentiality of
        investigations, and protect the public safety and order.

        (Source: P.A. 98-1142, eff. 12-30-14; 99-352, eff. 1-1-16 .)

        \hypertarget{ilcs-514-3}{%
        \subsection*{(720 ILCS 5/14-3)}\label{ilcs-514-3}}
        \addcontentsline{toc}{subsection}{(720 ILCS 5/14-3)}

        \hypertarget{sec.-14-3.-exemptions.}{%
        \section*{Sec. 14-3. Exemptions.}\label{sec.-14-3.-exemptions.}}
        \addcontentsline{toc}{section}{Sec. 14-3. Exemptions.}

        \markright{Sec. 14-3. Exemptions.}

        The following activities shall be exempt from the provisions of
        this Article:

        (a) Listening to radio, wireless electronic communications, and
        television communications of any sort where the same are
        publicly made;

        (b) Hearing conversation when heard by employees of any common
        carrier by wire incidental to the normal course of their
        employment in the operation, maintenance or repair of the
        equipment of such common carrier by wire so long as no
        information obtained thereby is used or divulged by the hearer;

        (c) Any broadcast by radio, television or otherwise whether it
        be a broadcast or recorded for the purpose of later broadcasts
        of any function where the public is in attendance and the
        conversations are overheard incidental to the main purpose for
        which such broadcasts are then being made;

        (d) Recording or listening with the aid of any device to any
        emergency communication made in the normal course of operations
        by any federal, state or local law enforcement agency or
        institutions dealing in emergency services, including, but not
        limited to, hospitals, clinics, ambulance services, fire
        fighting agencies, any public utility, emergency repair
        facility, civilian defense establishment or military
        installation;

        (e) Recording the proceedings of any meeting required to be open
        by the Open Meetings Act, as amended;

        (f) Recording or listening with the aid of any device to
        incoming telephone calls of phone lines publicly listed or
        advertised as consumer ``hotlines'' by manufacturers or
        retailers of food and drug products. Such recordings must be
        destroyed, erased or turned over to local law enforcement
        authorities within 24 hours from the time of such recording and
        shall not be otherwise disseminated. Failure on the part of the
        individual or business operating any such recording or listening
        device to comply with the requirements of this subsection shall
        eliminate any civil or criminal immunity conferred upon that
        individual or business by the operation of this Section;

        (g) With prior notification to the State's Attorney of the
        county in which it is to occur, recording or listening with the
        aid of any device to any conversation where a law enforcement
        officer, or any person acting at the direction of law
        enforcement, is a party to the conversation and has consented to
        it being intercepted or recorded under circumstances where the
        use of the device is necessary for the protection of the law
        enforcement officer or any person acting at the direction of law
        enforcement, in the course of an investigation of a forcible
        felony, a felony offense of involuntary servitude, involuntary
        sexual servitude of a minor, or trafficking in persons under
        Section 10-9 of this Code, an offense involving prostitution,
        solicitation of a sexual act, or pandering, a felony violation
        of the Illinois Controlled Substances Act, a felony violation of
        the Cannabis Control Act, a felony violation of the
        Methamphetamine Control and Community Protection Act, any
        ``streetgang related'' or ``gang-related'' felony as those terms
        are defined in the Illinois Streetgang Terrorism Omnibus
        Prevention Act, or any felony offense involving any weapon
        listed in paragraphs (1) through (11) of subsection (a) of
        Section 24-1 of this Code. Any recording or evidence derived as
        the result of this exemption shall be inadmissible in any
        proceeding, criminal, civil or administrative, except (i) where
        a party to the conversation suffers great bodily injury or is
        killed during such conversation, or (ii) when used as direct
        impeachment of a witness concerning matters contained in the
        interception or recording. The Director of the Illinois State
        Police shall issue regulations as are necessary concerning the
        use of devices, retention of tape recordings, and reports
        regarding their use;

        (g-5) (Blank);

        (g-6) With approval of the State's Attorney of the county in
        which it is to occur, recording or listening with the aid of any
        device to any conversation where a law enforcement officer, or
        any person acting at the direction of law enforcement, is a
        party to the conversation and has consented to it being
        intercepted or recorded in the course of an investigation of
        child pornography, aggravated child pornography, indecent
        solicitation of a child, luring of a minor, sexual exploitation
        of a child, aggravated criminal sexual abuse in which the victim
        of the offense was at the time of the commission of the offense
        under 18 years of age, or criminal sexual abuse by force or
        threat of force in which the victim of the offense was at the
        time of the commission of the offense under 18 years of age. In
        all such cases, an application for an order approving the
        previous or continuing use of an eavesdropping device must be
        made within 48 hours of the commencement of such use. In the
        absence of such an order, or upon its denial, any continuing use
        shall immediately terminate. The Director of the Illinois State
        Police shall issue rules as are necessary concerning the use of
        devices, retention of recordings, and reports regarding their
        use. Any recording or evidence obtained or derived in the course
        of an investigation of child pornography, aggravated child
        pornography, indecent solicitation of a child, luring of a
        minor, sexual exploitation of a child, aggravated criminal
        sexual abuse in which the victim of the offense was at the time
        of the commission of the offense under 18 years of age, or
        criminal sexual abuse by force or threat of force in which the
        victim of the offense was at the time of the commission of the
        offense under 18 years of age shall, upon motion of the State's
        Attorney or Attorney General prosecuting any case involving
        child pornography, aggravated child pornography, indecent
        solicitation of a child, luring of a minor, sexual exploitation
        of a child, aggravated criminal sexual abuse in which the victim
        of the offense was at the time of the commission of the offense
        under 18 years of age, or criminal sexual abuse by force or
        threat of force in which the victim of the offense was at the
        time of the commission of the offense under 18 years of age be
        reviewed in camera with notice to all parties present by the
        court presiding over the criminal case, and, if ruled by the
        court to be relevant and otherwise admissible, it shall be
        admissible at the trial of the criminal case. Absent such a
        ruling, any such recording or evidence shall not be admissible
        at the trial of the criminal case;

        (h) Recordings made simultaneously with the use of an in-car
        video camera recording of an oral conversation between a
        uniformed peace officer, who has identified his or her office,
        and a person in the presence of the peace officer whenever (i)
        an officer assigned a patrol vehicle is conducting an
        enforcement stop; or (ii) patrol vehicle emergency lights are
        activated or would otherwise be activated if not for the need to
        conceal the presence of law enforcement.

        For the purposes of this subsection (h), ``enforcement stop''
        means an action by a law enforcement officer in relation to
        enforcement and investigation duties, including but not limited
        to, traffic stops, pedestrian stops, abandoned vehicle contacts,
        motorist assists, commercial motor vehicle stops, roadside
        safety checks, requests for identification, or responses to
        requests for emergency assistance;

        (h-5) Recordings of utterances made by a person while in the
        presence of a uniformed peace officer and while an occupant of a
        police vehicle including, but not limited to, (i) recordings
        made simultaneously with the use of an in-car video camera and
        (ii) recordings made in the presence of the peace officer
        utilizing video or audio systems, or both, authorized by the law
        enforcement agency;

        (h-10) Recordings made simultaneously with a video camera
        recording during the use of a taser or similar weapon or device
        by a peace officer if the weapon or device is equipped with such
        camera;

        (h-15) Recordings made under subsection (h), (h-5), or (h-10)
        shall be retained by the law enforcement agency that employs the
        peace officer who made the recordings for a storage period of 90
        days, unless the recordings are made as a part of an arrest or
        the recordings are deemed evidence in any criminal, civil, or
        administrative proceeding and then the recordings must only be
        destroyed upon a final disposition and an order from the court.
        Under no circumstances shall any recording be altered or erased
        prior to the expiration of the designated storage period. Upon
        completion of the storage period, the recording medium may be
        erased and reissued for operational use;

        (i) Recording of a conversation made by or at the request of a
        person, not a law enforcement officer or agent of a law
        enforcement officer, who is a party to the conversation, under
        reasonable suspicion that another party to the conversation is
        committing, is about to commit, or has committed a criminal
        offense against the person or a member of his or her immediate
        household, and there is reason to believe that evidence of the
        criminal offense may be obtained by the recording;

        (j) The use of a telephone monitoring device by either (1) a
        corporation or other business entity engaged in marketing or
        opinion research or (2) a corporation or other business entity
        engaged in telephone solicitation, as defined in this
        subsection, to record or listen to oral telephone solicitation
        conversations or marketing or opinion research conversations by
        an employee of the corporation or other business entity when:

        (i) the monitoring is used for the purpose of service quality
        control of marketing or opinion research or telephone
        solicitation, the education or training of employees or
        contractors engaged in marketing or opinion research or
        telephone solicitation, or internal research related to
        marketing or opinion research or telephone solicitation; and

        (ii) the monitoring is used with the consent of at least one
        person who is an active party to the marketing or opinion
        research conversation or telephone solicitation conversation
        being monitored.

        No communication or conversation or any part, portion, or aspect
        of the communication or conversation made, acquired, or
        obtained, directly or indirectly, under this exemption (j), may
        be, directly or indirectly, furnished to any law enforcement
        officer, agency, or official for any purpose or used in any
        inquiry or investigation, or used, directly or indirectly, in
        any administrative, judicial, or other proceeding, or divulged
        to any third party.

        When recording or listening authorized by this subsection (j) on
        telephone lines used for marketing or opinion research or
        telephone solicitation purposes results in recording or
        listening to a conversation that does not relate to marketing or
        opinion research or telephone solicitation; the person recording
        or listening shall, immediately upon determining that the
        conversation does not relate to marketing or opinion research or
        telephone solicitation, terminate the recording or listening and
        destroy any such recording as soon as is practicable.

        Business entities that use a telephone monitoring or telephone
        recording system pursuant to this exemption (j) shall provide
        current and prospective employees with notice that the
        monitoring or recordings may occur during the course of their
        employment. The notice shall include prominent signage
        notification within the workplace.

        Business entities that use a telephone monitoring or telephone
        recording system pursuant to this exemption (j) shall provide
        their employees or agents with access to personal-only telephone
        lines which may be pay telephones, that are not subject to
        telephone monitoring or telephone recording.

        For the purposes of this subsection (j), ``telephone
        solicitation'' means a communication through the use of a
        telephone by live operators:

        (i) soliciting the sale of goods or services;

        (ii) receiving orders for the sale of goods or services;

        (iii) assisting in the use of goods or services; or

        (iv) engaging in the solicitation, administration, or collection
        of bank or retail credit accounts.

        For the purposes of this subsection (j), ``marketing or opinion
        research'' means a marketing or opinion research interview
        conducted by a live telephone interviewer engaged by a
        corporation or other business entity whose principal business is
        the design, conduct, and analysis of polls and surveys measuring
        the opinions, attitudes, and responses of respondents toward
        products and services, or social or political issues, or both;

        (k) Electronic recordings, including but not limited to, a
        motion picture, videotape, digital, or other visual or audio
        recording, made of a custodial interrogation of an individual at
        a police station or other place of detention by a law
        enforcement officer under Section 5-401.5 of the Juvenile Court
        Act of 1987 or Section 103-2.1 of the Code of Criminal Procedure
        of 1963;

        (l) Recording the interview or statement of any person when the
        person knows that the interview is being conducted by a law
        enforcement officer or prosecutor and the interview takes place
        at a police station that is currently participating in the
        Custodial Interview Pilot Program established under the Illinois
        Criminal Justice Information Act;

        (m) An electronic recording, including but not limited to, a
        motion picture, videotape, digital, or other visual or audio
        recording, made of the interior of a school bus while the school
        bus is being used in the transportation of students to and from
        school and school-sponsored activities, when the school board
        has adopted a policy authorizing such recording, notice of such
        recording policy is included in student handbooks and other
        documents including the policies of the school, notice of the
        policy regarding recording is provided to parents of students,
        and notice of such recording is clearly posted on the door of
        and inside the school bus.

        Recordings made pursuant to this subsection (m) shall be
        confidential records and may only be used by school officials
        (or their designees) and law enforcement personnel for
        investigations, school disciplinary actions and hearings,
        proceedings under the Juvenile Court Act of 1987, and criminal
        prosecutions, related to incidents occurring in or around the
        school bus;

        (n) Recording or listening to an audio transmission from a
        microphone placed by a person under the authority of a law
        enforcement agency inside a bait car surveillance vehicle while
        simultaneously capturing a photographic or video image;

        (o) The use of an eavesdropping camera or audio device during an
        ongoing hostage or barricade situation by a law enforcement
        officer or individual acting on behalf of a law enforcement
        officer when the use of such device is necessary to protect the
        safety of the general public, hostages, or law enforcement
        officers or anyone acting on their behalf;

        (p) Recording or listening with the aid of any device to
        incoming telephone calls of phone lines publicly listed or
        advertised as the ``CPS Violence Prevention Hotline'', but only
        where the notice of recording is given at the beginning of each
        call as required by Section 34-21.8 of the School Code. The
        recordings may be retained only by the Chicago Police Department
        or other law enforcement authorities, and shall not be otherwise
        retained or disseminated;

        (q)(1) With prior request to and written or verbal approval of
        the State's Attorney of the county in which the conversation is
        anticipated to occur, recording or listening with the aid of an
        eavesdropping device to a conversation in which a law
        enforcement officer, or any person acting at the direction of a
        law enforcement officer, is a party to the conversation and has
        consented to the conversation being intercepted or recorded in
        the course of an investigation of a qualified offense. The
        State's Attorney may grant this approval only after determining
        that reasonable cause exists to believe that inculpatory
        conversations concerning a qualified offense will occur with a
        specified individual or individuals within a designated period
        of time.

        (2) Request for approval. To invoke the exception contained in
        this subsection (q), a law enforcement officer shall make a
        request for approval to the appropriate State's Attorney. The
        request may be written or verbal; however, a written
        memorialization of the request must be made by the State's
        Attorney. This request for approval shall include whatever
        information is deemed necessary by the State's Attorney but
        shall include, at a minimum, the following information about
        each specified individual whom the law enforcement officer
        believes will commit a qualified offense:

        (A) his or her full or partial name, nickname or alias;

        (B) a physical description; or

        (C) failing either (A) or (B) of this paragraph

        (2), any other supporting information known to the law
        enforcement officer at the time of the request that gives rise
        to reasonable cause to believe that the specified individual
        will participate in an inculpatory conversation concerning a
        qualified offense.

        (3) Limitations on approval. Each written approval by the
        State's Attorney under this subsection (q) shall be limited to:

        (A) a recording or interception conducted by a specified law
        enforcement officer or person acting at the direction of a law
        enforcement officer;

        (B) recording or intercepting conversations with the individuals
        specified in the request for approval, provided that the verbal
        approval shall be deemed to include the recording or
        intercepting of conversations with other individuals, unknown to
        the law enforcement officer at the time of the request for
        approval, who are acting in conjunction with or as
        co-conspirators with the individuals specified in the request
        for approval in the commission of a qualified offense;

        (C) a reasonable period of time but in no event longer than 24
        consecutive hours;

        (D) the written request for approval, if applicable, or the
        written memorialization must be filed, along with the written
        approval, with the circuit clerk of the jurisdiction on the next
        business day following the expiration of the authorized period
        of time, and shall be subject to review by the Chief Judge or
        his or her designee as deemed appropriate by the court.

        (3.5) The written memorialization of the request for approval
        and the written approval by the State's Attorney may be in any
        format, including via facsimile, email, or otherwise, so long as
        it is capable of being filed with the circuit clerk.

        (3.10) Beginning March 1, 2015, each State's Attorney shall
        annually submit a report to the General Assembly disclosing:

        (A) the number of requests for each qualified offense for
        approval under this subsection; and

        (B) the number of approvals for each qualified offense given by
        the State's Attorney.

        (4) Admissibility of evidence. No part of the contents of any
        wire, electronic, or oral communication that has been recorded
        or intercepted as a result of this exception may be received in
        evidence in any trial, hearing, or other proceeding in or before
        any court, grand jury, department, officer, agency, regulatory
        body, legislative committee, or other authority of this State,
        or a political subdivision of the State, other than in a
        prosecution of:

        (A) the qualified offense for which approval was given to record
        or intercept a conversation under this subsection (q);

        (B) a forcible felony committed directly in the course of the
        investigation of the qualified offense for which approval was
        given to record or intercept a conversation under this
        subsection (q); or

        (C) any other forcible felony committed while the recording or
        interception was approved in accordance with this subsection
        (q), but for this specific category of prosecutions, only if the
        law enforcement officer or person acting at the direction of a
        law enforcement officer who has consented to the conversation
        being intercepted or recorded suffers great bodily injury or is
        killed during the commission of the charged forcible felony.

        (5) Compliance with the provisions of this subsection is a
        prerequisite to the admissibility in evidence of any part of the
        contents of any wire, electronic or oral communication that has
        been intercepted as a result of this exception, but nothing in
        this subsection shall be deemed to prevent a court from
        otherwise excluding the evidence on any other ground recognized
        by State or federal law, nor shall anything in this subsection
        be deemed to prevent a court from independently reviewing the
        admissibility of the evidence for compliance with the Fourth
        Amendment to the U.S. Constitution or with Article I, Section 6
        of the Illinois Constitution.

        (6) Use of recordings or intercepts unrelated to qualified
        offenses. Whenever any private conversation or private
        electronic communication has been recorded or intercepted as a
        result of this exception that is not related to an offense for
        which the recording or intercept is admissible under paragraph
        (4) of this subsection (q), no part of the contents of the
        communication and evidence derived from the communication may be
        received in evidence in any trial, hearing, or other proceeding
        in or before any court, grand jury, department, officer, agency,
        regulatory body, legislative committee, or other authority of
        this State, or a political subdivision of the State, nor may it
        be publicly disclosed in any way.

        (6.5) The Illinois State Police shall adopt rules as are
        necessary concerning the use of devices, retention of
        recordings, and reports regarding their use under this
        subsection (q).

        (7) Definitions. For the purposes of this subsection

        (q) only:

        ``Forcible felony'' includes and is limited to those offenses
        contained in Section 2-8 of the Criminal Code of 1961 as of the
        effective date of this amendatory Act of the 97th General
        Assembly, and only as those offenses have been defined by law or
        judicial interpretation as of that date.

        ``Qualified offense'' means and is limited to:

        (A) a felony violation of the Cannabis

        Control Act, the Illinois Controlled Substances Act, or the
        Methamphetamine Control and Community Protection Act, except for
        violations of:

        (i) Section 4 of the Cannabis Control Act;

        (ii) Section 402 of the Illinois

        Controlled Substances Act; and

        (iii) Section 60 of the Methamphetamine

        Control and Community Protection Act; and

        (B) first degree murder, solicitation of murder for hire,
        predatory criminal sexual assault of a child, criminal sexual
        assault, aggravated criminal sexual assault, aggravated arson,
        kidnapping, aggravated kidnapping, child abduction, trafficking
        in persons, involuntary servitude, involuntary sexual servitude
        of a minor, or gunrunning.

        ``State's Attorney'' includes and is limited to the

        State's Attorney or an assistant State's Attorney designated by
        the State's Attorney to provide verbal approval to record or
        intercept conversations under this subsection (q).

        (8) Sunset. This subsection (q) is inoperative on and after
        January 1, 2027. No conversations intercepted pursuant to this
        subsection (q), while operative, shall be inadmissible in a
        court of law by virtue of the inoperability of this subsection
        (q) on January 1, 2027.

        (9) Recordings, records, and custody. Any private conversation
        or private electronic communication intercepted by a law
        enforcement officer or a person acting at the direction of law
        enforcement shall, if practicable, be recorded in such a way as
        will protect the recording from editing or other alteration. Any
        and all original recordings made under this subsection (q) shall
        be inventoried without unnecessary delay pursuant to the law
        enforcement agency's policies for inventorying evidence. The
        original recordings shall not be destroyed except upon an order
        of a court of competent jurisdiction; and

        (r) Electronic recordings, including but not limited to, motion
        picture, videotape, digital, or other visual or audio recording,
        made of a lineup under Section 107A-2 of the Code of Criminal
        Procedure of 1963.

        (Source: P.A. 101-80, eff. 7-12-19; 102-538, eff. 8-20-21;
        102-918, eff. 5-27-22.)

        \hypertarget{ilcs-514-3a}{%
        \subsection*{(720 ILCS 5/14-3A)}\label{ilcs-514-3a}}
        \addcontentsline{toc}{subsection}{(720 ILCS 5/14-3A)}

        \hypertarget{sec.-14-3a.-recordings-records-and-custody.}{%
        \section*{Sec. 14-3A. Recordings, records, and
        custody.}\label{sec.-14-3a.-recordings-records-and-custody.}}
        \addcontentsline{toc}{section}{Sec. 14-3A. Recordings, records,
        and custody.}

        \markright{Sec. 14-3A. Recordings, records, and custody.}

        (a) Any private oral communication intercepted in accordance
        with subsection (g) of Section 14-3 shall, if practicable, be
        recorded by tape or other comparable method. The recording
        shall, if practicable, be done in such a way as will protect it
        from editing or other alteration. During an interception, the
        interception shall be carried out by a law enforcement officer,
        and the officer shall keep a signed, written record, including:

        (1) The day and hours of interception or recording;

        (2) The time and duration of each intercepted communication;

        (3) The parties, if known, to each intercepted communication;
        and

        (4) A summary of the contents of each intercepted communication.

        (b) Both the written record of the interception or recording and
        any and all recordings of the interception or recording shall
        immediately be inventoried and shall be maintained where the
        chief law enforcement officer of the county in which the
        interception or recording occurred directs. The written records
        of the interception or recording conducted under subsection (g)
        of Section 14-3 shall not be destroyed except upon an order of a
        court of competent jurisdiction and in any event shall be kept
        for 10 years.

        (Source: P.A. 88-677, eff. 12-15-94.)

        \hypertarget{ilcs-514-3b}{%
        \subsection*{(720 ILCS 5/14-3B)}\label{ilcs-514-3b}}
        \addcontentsline{toc}{subsection}{(720 ILCS 5/14-3B)}

        \hypertarget{sec.-14-3b.-notice-of-interception-or-recording.}{%
        \section*{Sec. 14-3B. Notice of interception or
        recording.}\label{sec.-14-3b.-notice-of-interception-or-recording.}}
        \addcontentsline{toc}{section}{Sec. 14-3B. Notice of
        interception or recording.}

        \markright{Sec. 14-3B. Notice of interception or recording.}

        (a) Within a reasonable time, but not later than 60 days after
        the termination of the investigation for which the interception
        or recording was conducted, or immediately upon the initiation
        of criminal proceedings, the person who was the subject of an
        interception or recording under subsection (g) of Section 14-3
        shall be served with an inventory that shall include:

        (1) Notice to any person who was the subject of the interception
        or recording;

        (2) Notice of any interception or recording if the defendant was
        arrested or indicted or otherwise charged as a result of the
        interception of his or her private oral communication;

        (3) The date of the interception or recording;

        (4) The period of interception or recording; and

        (5) Notice of whether during the period of interception or
        recording devices were or were not used to overhear and record
        various conversations and whether or not the conversations are
        recorded.

        (b) A court of competent jurisdiction, upon filing of a motion,
        may in its discretion make available to those persons or their
        attorneys for inspection those portions of the intercepted
        communications as the court determines to be in the interest of
        justice.

        (Source: P.A. 88-677, eff. 12-15-94.)

        \hypertarget{ilcs-514-4-from-ch.-38-par.-14-4}{%
        \subsection*{(720 ILCS 5/14-4) (from Ch. 38, par.
        14-4)}\label{ilcs-514-4-from-ch.-38-par.-14-4}}
        \addcontentsline{toc}{subsection}{(720 ILCS 5/14-4) (from Ch.
        38, par. 14-4)}

        \hypertarget{sec.-14-4.-sentence.}{%
        \section*{Sec. 14-4. Sentence.}\label{sec.-14-4.-sentence.}}
        \addcontentsline{toc}{section}{Sec. 14-4. Sentence.}

        \markright{Sec. 14-4. Sentence.}

        (a) Eavesdropping, for a first offense, is a Class 4 felony and,
        for a second or subsequent offense, is a Class 3 felony.

        (b) The eavesdropping of an oral conversation or an electronic
        communication of any law enforcement officer, State's Attorney,
        Assistant State's Attorney, the Attorney General, Assistant
        Attorney General, or a judge, while in the performance of his or
        her official duties, if not authorized by this Article or proper
        court order, is a Class 3 felony, and for a second or subsequent
        offense, is a Class 2 felony.

        (Source: P.A. 98-1142, eff. 12-30-14.)

        \hypertarget{ilcs-514-5-from-ch.-38-par.-14-5}{%
        \subsection*{(720 ILCS 5/14-5) (from Ch. 38, par.
        14-5)}\label{ilcs-514-5-from-ch.-38-par.-14-5}}
        \addcontentsline{toc}{subsection}{(720 ILCS 5/14-5) (from Ch.
        38, par. 14-5)}

        \hypertarget{sec.-14-5.-evidence-inadmissible.}{%
        \section*{Sec. 14-5. Evidence
        inadmissible.}\label{sec.-14-5.-evidence-inadmissible.}}
        \addcontentsline{toc}{section}{Sec. 14-5. Evidence
        inadmissible.}

        \markright{Sec. 14-5. Evidence inadmissible.}

        Any evidence obtained in violation of this Article is not
        admissible in any civil or criminal trial, or any administrative
        or legislative inquiry or proceeding, nor in any grand jury
        proceedings; provided, however, that so much of the contents of
        an alleged unlawfully intercepted, overheard or recorded
        conversation as is clearly relevant, as determined as a matter
        of law by the court in chambers, to the proof of such allegation
        may be admitted into evidence in any criminal trial or grand
        jury proceeding brought against any person charged with
        violating any provision of this Article. Nothing in this Section
        bars admission of evidence if all parties to the private
        conversation or private electronic communication consent to
        admission of the evidence.

        (Source: P.A. 98-1142, eff. 12-30-14.)

        \hypertarget{ilcs-514-6-from-ch.-38-par.-14-6}{%
        \subsection*{(720 ILCS 5/14-6) (from Ch. 38, par.
        14-6)}\label{ilcs-514-6-from-ch.-38-par.-14-6}}
        \addcontentsline{toc}{subsection}{(720 ILCS 5/14-6) (from Ch.
        38, par. 14-6)}

        \hypertarget{sec.-14-6.-civil-remedies-to-injured-parties.}{%
        \section*{Sec. 14-6. Civil remedies to injured
        parties.}\label{sec.-14-6.-civil-remedies-to-injured-parties.}}
        \addcontentsline{toc}{section}{Sec. 14-6. Civil remedies to
        injured parties.}

        \markright{Sec. 14-6. Civil remedies to injured parties.}

        (1) Any or all parties to any conversation or electronic
        communication upon which eavesdropping is practiced contrary to
        this Article shall be entitled to the following remedies:

        (a) To an injunction by the circuit court prohibiting further
        eavesdropping by the eavesdropper and by or on behalf of his
        principal, or either;

        (b) To all actual damages against the eavesdropper or his
        principal or both;

        (c) To any punitive damages which may be awarded by the court or
        by a jury;

        (d) To all actual damages against any landlord, owner or
        building operator, or any common carrier by wire who aids,
        abets, or knowingly permits the eavesdropping concerned;

        (e) To any punitive damages which may be awarded by the court or
        by a jury against any landlord, owner or building operator, or
        common carrier by wire who aids, abets, or knowingly permits the
        eavesdropping concerned.

        (2) No cause of action shall lie in any court against any common
        carrier by wire or its officers, agents or employees for
        providing information, assistance or facilities in accordance
        with the terms of a court order entered under Article 108A of
        the Code of Criminal Procedure of 1963.

        (3) No civil claim, cause of action, or remedy shall lie against
        a parent, step-parent, guardian, or grandparent for
        eavesdropping of electronic communications through access to
        their minor's electronic accounts during that parent,
        step-parent, guardian, or grandparent's exercise of his or her
        parental rights to supervise, monitor, and control the
        activities of a minor in his or her care, custody, or control.
        This provision does not diminish the protections given to
        electronic accounts of a minor under any existing law other than
        this Article.

        (Source: P.A. 98-268, eff. 1-1-14.)

        \hypertarget{ilcs-514-7-from-ch.-38-par.-14-7}{%
        \subsection*{(720 ILCS 5/14-7) (from Ch. 38, par.
        14-7)}\label{ilcs-514-7-from-ch.-38-par.-14-7}}
        \addcontentsline{toc}{subsection}{(720 ILCS 5/14-7) (from Ch.
        38, par. 14-7)}

        \hypertarget{sec.-14-7.-common-carrier-to-aid-in-detection.}{%
        \section*{Sec. 14-7. Common carrier to aid in
        detection.}\label{sec.-14-7.-common-carrier-to-aid-in-detection.}}
        \addcontentsline{toc}{section}{Sec. 14-7. Common carrier to aid
        in detection.}

        \markright{Sec. 14-7. Common carrier to aid in detection.}

        Subject to regulation by the Illinois Commerce Commission, any
        common carrier by wire shall, upon request of any subscriber and
        upon responsible offer to pay the reasonable cost thereof,
        furnish whatever services may be within its command for the
        purpose of detecting any eavesdropping involving its wires which
        are used by said subscriber. All such requests by subscribers
        shall be kept confidential unless divulgence is authorized in
        writing by the requesting subscriber.

        (Source: Laws 1961, p.~1983.)

        \hypertarget{ilcs-514-8-from-ch.-38-par.-14-8}{%
        \subsection*{(720 ILCS 5/14-8) (from Ch. 38, par.
        14-8)}\label{ilcs-514-8-from-ch.-38-par.-14-8}}
        \addcontentsline{toc}{subsection}{(720 ILCS 5/14-8) (from Ch.
        38, par. 14-8)}

        \hypertarget{sec.-14-8.-discovery-of-eavesdropping-device-by-an-individual-common-carrier-private-investigative-agency-or-non-governmental-corporation.-any-agent-officer-or-employee-of-a-private-investigative-agency-or-non-governmental-corporation-or-of-a-common-carrier-by-wire-or-any-individual-who-discovers-any-physical-evidence-of-an-eavesdropping-device-being-used-which-such-person-does-not-know-to-be-a-legal-eavesdropping-device-shall-within-a-reasonable-time-after-such-discovery-disclose-the-existence-of-such-eavesdropping-device-to-the-states-attorney-of-the-county-where-such-device-was-found.-the-states-attorney-shall-within-a-reasonable-time-notify-the-person-or-persons-apparently-being-eavesdropped-upon-of-the-existence-of-that-device-if-the-device-is-illegal.-a-violation-of-this-section-is-a-business-offense-for-which-a-fine-shall-be-imposed-not-to-exceed-500.}{%
        \section*{Sec. 14-8. Discovery of eavesdropping device by an
        individual, common carrier, private investigative agency or
        non-governmental corporation). Any agent, officer or employee of
        a private investigative agency or non-governmental corporation,
        or of a common carrier by wire, or any individual, who discovers
        any physical evidence of an eavesdropping device being used
        which such person does not know to be a legal eavesdropping
        device shall, within a reasonable time after such discovery
        disclose the existence of such eavesdropping device to the
        State's Attorney of the county where such device was found. The
        State's Attorney shall within a reasonable time notify the
        person or persons apparently being eavesdropped upon of the
        existence of that device if the device is illegal. A violation
        of this Section is a Business Offense for which a fine shall be
        imposed not to exceed
        \$500.}\label{sec.-14-8.-discovery-of-eavesdropping-device-by-an-individual-common-carrier-private-investigative-agency-or-non-governmental-corporation.-any-agent-officer-or-employee-of-a-private-investigative-agency-or-non-governmental-corporation-or-of-a-common-carrier-by-wire-or-any-individual-who-discovers-any-physical-evidence-of-an-eavesdropping-device-being-used-which-such-person-does-not-know-to-be-a-legal-eavesdropping-device-shall-within-a-reasonable-time-after-such-discovery-disclose-the-existence-of-such-eavesdropping-device-to-the-states-attorney-of-the-county-where-such-device-was-found.-the-states-attorney-shall-within-a-reasonable-time-notify-the-person-or-persons-apparently-being-eavesdropped-upon-of-the-existence-of-that-device-if-the-device-is-illegal.-a-violation-of-this-section-is-a-business-offense-for-which-a-fine-shall-be-imposed-not-to-exceed-500.}}
        \addcontentsline{toc}{section}{Sec. 14-8. Discovery of
        eavesdropping device by an individual, common carrier, private
        investigative agency or non-governmental corporation). Any
        agent, officer or employee of a private investigative agency or
        non-governmental corporation, or of a common carrier by wire, or
        any individual, who discovers any physical evidence of an
        eavesdropping device being used which such person does not know
        to be a legal eavesdropping device shall, within a reasonable
        time after such discovery disclose the existence of such
        eavesdropping device to the State's Attorney of the county where
        such device was found. The State's Attorney shall within a
        reasonable time notify the person or persons apparently being
        eavesdropped upon of the existence of that device if the device
        is illegal. A violation of this Section is a Business Offense
        for which a fine shall be imposed not to exceed \$500.}

        \markright{Sec. 14-8. Discovery of eavesdropping device by an
        individual, common carrier, private investigative agency or
        non-governmental corporation). Any agent, officer or employee of
        a private investigative agency or non-governmental corporation,
        or of a common carrier by wire, or any individual, who discovers
        any physical evidence of an eavesdropping device being used
        which such person does not know to be a legal eavesdropping
        device shall, within a reasonable time after such discovery
        disclose the existence of such eavesdropping device to the
        State's Attorney of the county where such device was found. The
        State's Attorney shall within a reasonable time notify the
        person or persons apparently being eavesdropped upon of the
        existence of that device if the device is illegal. A violation
        of this Section is a Business Offense for which a fine shall be
        imposed not to exceed \$500.}

        (Source: P.A. 79-984; 79-1454.)

        \hypertarget{ilcs-514-9-from-ch.-38-par.-14-9}{%
        \subsection*{(720 ILCS 5/14-9) (from Ch. 38, par.
        14-9)}\label{ilcs-514-9-from-ch.-38-par.-14-9}}
        \addcontentsline{toc}{subsection}{(720 ILCS 5/14-9) (from Ch.
        38, par. 14-9)}

        \hypertarget{sec.-14-9.-discovery-of-eavesdropping-device-by-common-carrier-by-wire---disclosure-to-subscriber.-any-agent-officer-or-employee-of-any-common-carrier-by-wire-who-discovers-any-physical-evidence-of-an-eavesdropping-device-which-such-person-does-not-know-to-be-a-legal-eavesdropping-device-shall-within-a-reasonable-time-after-such-discovery-disclose-the-existence-of-the-eavesdropping-device-to-the-states-attorney-of-the-county-where-such-device-was-found.-the-states-attorney-shall-within-a-reasonable-time-notify-the-person-or-persons-apparently-being-eavesdropped-upon-of-the-existence-of-that-device-if-the-device-is-illegal.-a-violation-of-this-section-is-a-business-offense-for-which-a-fine-shall-be-imposed-not-to-exceed-500.}{%
        \section*{Sec. 14-9. Discovery of eavesdropping device by common
        carrier by wire - disclosure to subscriber.) Any agent, officer
        or employee of any common carrier by wire who discovers any
        physical evidence of an eavesdropping device which such person
        does not know to be a legal eavesdropping device shall, within a
        reasonable time after such discovery, disclose the existence of
        the eavesdropping device to the State's Attorney of the County
        where such device was found. The State's Attorney shall within a
        reasonable time notify the person or persons apparently being
        eavesdropped upon of the existence of that device if the device
        is illegal. A violation of this Section is a Business Offense
        for which a fine shall be imposed not to exceed
        \$500.}\label{sec.-14-9.-discovery-of-eavesdropping-device-by-common-carrier-by-wire---disclosure-to-subscriber.-any-agent-officer-or-employee-of-any-common-carrier-by-wire-who-discovers-any-physical-evidence-of-an-eavesdropping-device-which-such-person-does-not-know-to-be-a-legal-eavesdropping-device-shall-within-a-reasonable-time-after-such-discovery-disclose-the-existence-of-the-eavesdropping-device-to-the-states-attorney-of-the-county-where-such-device-was-found.-the-states-attorney-shall-within-a-reasonable-time-notify-the-person-or-persons-apparently-being-eavesdropped-upon-of-the-existence-of-that-device-if-the-device-is-illegal.-a-violation-of-this-section-is-a-business-offense-for-which-a-fine-shall-be-imposed-not-to-exceed-500.}}
        \addcontentsline{toc}{section}{Sec. 14-9. Discovery of
        eavesdropping device by common carrier by wire - disclosure to
        subscriber.) Any agent, officer or employee of any common
        carrier by wire who discovers any physical evidence of an
        eavesdropping device which such person does not know to be a
        legal eavesdropping device shall, within a reasonable time after
        such discovery, disclose the existence of the eavesdropping
        device to the State's Attorney of the County where such device
        was found. The State's Attorney shall within a reasonable time
        notify the person or persons apparently being eavesdropped upon
        of the existence of that device if the device is illegal. A
        violation of this Section is a Business Offense for which a fine
        shall be imposed not to exceed \$500.}

        \markright{Sec. 14-9. Discovery of eavesdropping device by
        common carrier by wire - disclosure to subscriber.) Any agent,
        officer or employee of any common carrier by wire who discovers
        any physical evidence of an eavesdropping device which such
        person does not know to be a legal eavesdropping device shall,
        within a reasonable time after such discovery, disclose the
        existence of the eavesdropping device to the State's Attorney of
        the County where such device was found. The State's Attorney
        shall within a reasonable time notify the person or persons
        apparently being eavesdropped upon of the existence of that
        device if the device is illegal. A violation of this Section is
        a Business Offense for which a fine shall be imposed not to
        exceed \$500.}

        (Source: P.A. 79-985.)

        \hypertarget{ilcs-5tit.-iii-pt.-c-heading}{%
        \subsection*{(720 ILCS 5/Tit. III Pt. C
        heading)}\label{ilcs-5tit.-iii-pt.-c-heading}}
        \addcontentsline{toc}{subsection}{(720 ILCS 5/Tit. III Pt. C
        heading)}

        PART C.

        OFFENSES DIRECTED AGAINST PROPERTY
      \end{enumerate}
    \end{enumerate}
  \end{enumerate}
\end{enumerate}

\bookmarksetup{startatroot}

\hypertarget{article-15.-definitions}{%
\chapter*{Article 15. Definitions}\label{article-15.-definitions}}
\addcontentsline{toc}{chapter}{Article 15. Definitions}

\markboth{Article 15. Definitions}{Article 15. Definitions}

\hypertarget{ilcs-515-1-from-ch.-38-par.-15-1}{%
\subsection*{(720 ILCS 5/15-1) (from Ch. 38, par.
15-1)}\label{ilcs-515-1-from-ch.-38-par.-15-1}}
\addcontentsline{toc}{subsection}{(720 ILCS 5/15-1) (from Ch. 38, par.
15-1)}

\hypertarget{sec.-15-1.-property.}{%
\section*{Sec. 15-1. Property.}\label{sec.-15-1.-property.}}
\addcontentsline{toc}{section}{Sec. 15-1. Property.}

\markright{Sec. 15-1. Property.}

As used in this Part C, ``property'' means anything of value. Property
includes real estate, money, commercial instruments, admission or
transportation tickets, written instruments representing or embodying
rights concerning anything of value, labor, or services, or otherwise of
value to the owner; things growing on, affixed to, or found on land, or
part of or affixed to any building; electricity, gas and water;
telecommunications services; birds, animals and fish, which ordinarily
are kept in a state of confinement; food and drink; samples, cultures,
microorganisms, specimens, records, recordings, documents, blueprints,
drawings, maps, and whole or partial copies, descriptions, photographs,
computer programs or data, prototypes or models thereof, or any other
articles, materials, devices, substances and whole or partial copies,
descriptions, photographs, prototypes, or models thereof which
constitute, represent, evidence, reflect or record a secret scientific,
technical, merchandising, production or management information, design,
process, procedure, formula, invention, or improvement.

(Source: P.A. 88-75.)

\hypertarget{ilcs-515-2-from-ch.-38-par.-15-2}{%
\subsection*{(720 ILCS 5/15-2) (from Ch. 38, par.
15-2)}\label{ilcs-515-2-from-ch.-38-par.-15-2}}
\addcontentsline{toc}{subsection}{(720 ILCS 5/15-2) (from Ch. 38, par.
15-2)}

\hypertarget{sec.-15-2.-owner.}{%
\section*{Sec. 15-2. Owner.}\label{sec.-15-2.-owner.}}
\addcontentsline{toc}{section}{Sec. 15-2. Owner.}

\markright{Sec. 15-2. Owner.}

As used in this Part C, ``owner'' means a person, other than the
offender, who has possession of or any other interest in the property
involved, even though such interest or possession is unlawful, and
without whose consent the offender has no authority to exert control
over the property.

(Source: Laws 1961, p.~1983.)

\hypertarget{ilcs-515-3-from-ch.-38-par.-15-3}{%
\subsection*{(720 ILCS 5/15-3) (from Ch. 38, par.
15-3)}\label{ilcs-515-3-from-ch.-38-par.-15-3}}
\addcontentsline{toc}{subsection}{(720 ILCS 5/15-3) (from Ch. 38, par.
15-3)}

\hypertarget{sec.-15-3.-permanent-deprivation.}{%
\section*{Sec. 15-3. Permanent
deprivation.}\label{sec.-15-3.-permanent-deprivation.}}
\addcontentsline{toc}{section}{Sec. 15-3. Permanent deprivation.}

\markright{Sec. 15-3. Permanent deprivation.}

As used in this Part C, to ``permanently deprive'' means to:

(a) Defeat all recovery of the property by the owner; or

(b) Deprive the owner permanently of the beneficial use of the property;
or

(c) Retain the property with intent to restore it to the owner only if
the owner purchases or leases it back, or pays a reward or other
compensation for its return; or

(d) Sell, give, pledge, or otherwise transfer any interest in the
property or subject it to the claim of a person other than the owner.

(Source: Laws 1961, p.~1983.)

\hypertarget{ilcs-515-4-from-ch.-38-par.-15-4}{%
\subsection*{(720 ILCS 5/15-4) (from Ch. 38, par.
15-4)}\label{ilcs-515-4-from-ch.-38-par.-15-4}}
\addcontentsline{toc}{subsection}{(720 ILCS 5/15-4) (from Ch. 38, par.
15-4)}

\hypertarget{sec.-15-4.-deception.}{%
\section*{Sec. 15-4. Deception.}\label{sec.-15-4.-deception.}}
\addcontentsline{toc}{section}{Sec. 15-4. Deception.}

\markright{Sec. 15-4. Deception.}

As used in this Part C ``deception'' means knowingly to:

(a) Create or confirm another's impression which is false and which the
offender does not believe to be true; or

(b) Fail to correct a false impression which the offender previously has
created or confirmed; or

(c) Prevent another from acquiring information pertinent to the
disposition of the property involved; or

(d) Sell or otherwise transfer or encumber property, failing to disclose
a lien, adverse claim, or other legal impediment to the enjoyment of the
property, whether such impediment is or is not valid, or is or is not a
matter of official record; or

(e) Promise performance which the offender does not intend to perform or
knows will not be performed. Failure to perform standing alone is not
evidence that the offender did not intend to perform.

(Source: Laws 1961, p.~1983.)

\hypertarget{ilcs-515-5-from-ch.-38-par.-15-5}{%
\subsection*{(720 ILCS 5/15-5) (from Ch. 38, par.
15-5)}\label{ilcs-515-5-from-ch.-38-par.-15-5}}
\addcontentsline{toc}{subsection}{(720 ILCS 5/15-5) (from Ch. 38, par.
15-5)}

\hypertarget{sec.-15-5.-threat.}{%
\section*{Sec. 15-5. Threat.}\label{sec.-15-5.-threat.}}
\addcontentsline{toc}{section}{Sec. 15-5. Threat.}

\markright{Sec. 15-5. Threat.}

As used in this Part C, ``threat'' means a menace, however communicated,
to:

(a) Inflict physical harm on the person threatened or any other person
or on property; or

(b) Subject any person to physical confinement or restraint; or

(c) Commit any criminal offense; or

(d) Accuse any person of a criminal offense; or

(e) Expose any person to hatred, contempt or ridicule; or

(f) Harm the credit or business repute of any person; or

(g) Reveal any information sought to be concealed by the person
threatened; or

(h) Take action as an official against anyone or anything, or withhold
official action, or cause such action or withholding; or

(i) Bring about or continue a strike, boycott or other similar
collective action if the property is not demanded or received for the
benefit of the group which he purports to represent; or

(j) Testify or provide information or withhold testimony or information
with respect to another's legal claim or defense; or

(k) Inflict any other harm which would not benefit the offender.

(Source: Laws 1961, p.~1983.)

\hypertarget{ilcs-515-6-from-ch.-38-par.-15-6}{%
\subsection*{(720 ILCS 5/15-6) (from Ch. 38, par.
15-6)}\label{ilcs-515-6-from-ch.-38-par.-15-6}}
\addcontentsline{toc}{subsection}{(720 ILCS 5/15-6) (from Ch. 38, par.
15-6)}

\hypertarget{sec.-15-6.-stolen-property.}{%
\section*{Sec. 15-6. Stolen
property.}\label{sec.-15-6.-stolen-property.}}
\addcontentsline{toc}{section}{Sec. 15-6. Stolen property.}

\markright{Sec. 15-6. Stolen property.}

As used in this Part C, ``stolen property'' means property over which
control has been obtained by theft.

(Source: Laws 1961, p.~1983 .)

\hypertarget{ilcs-515-7-from-ch.-38-par.-15-7}{%
\subsection*{(720 ILCS 5/15-7) (from Ch. 38, par.
15-7)}\label{ilcs-515-7-from-ch.-38-par.-15-7}}
\addcontentsline{toc}{subsection}{(720 ILCS 5/15-7) (from Ch. 38, par.
15-7)}

\hypertarget{sec.-15-7.-obtain.}{%
\section*{Sec. 15-7. Obtain.}\label{sec.-15-7.-obtain.}}
\addcontentsline{toc}{section}{Sec. 15-7. Obtain.}

\markright{Sec. 15-7. Obtain.}

As used in this Part C, ``obtain'' means:

(a) In relation to property, to bring about a transfer of interest or
possession, whether to the offender or to another, and

(b) In relation to labor or services, to secure the performance thereof.

(Source: Laws 1961, p.~1983.)

\hypertarget{ilcs-515-8-from-ch.-38-par.-15-8}{%
\subsection*{(720 ILCS 5/15-8) (from Ch. 38, par.
15-8)}\label{ilcs-515-8-from-ch.-38-par.-15-8}}
\addcontentsline{toc}{subsection}{(720 ILCS 5/15-8) (from Ch. 38, par.
15-8)}

\hypertarget{sec.-15-8.-obtains-control.}{%
\section*{Sec. 15-8. Obtains
control.}\label{sec.-15-8.-obtains-control.}}
\addcontentsline{toc}{section}{Sec. 15-8. Obtains control.}

\markright{Sec. 15-8. Obtains control.}

As used in this Part C, the phrase ``obtains or exerts control'' over
property, includes but is not limited to the taking, carrying away, or
the sale, conveyance, or transfer of title to, or interest in, or
possession of property.

(Source: Laws 1961, p.~1983 .)

\hypertarget{ilcs-515-9-from-ch.-38-par.-15-9}{%
\subsection*{(720 ILCS 5/15-9) (from Ch. 38, par.
15-9)}\label{ilcs-515-9-from-ch.-38-par.-15-9}}
\addcontentsline{toc}{subsection}{(720 ILCS 5/15-9) (from Ch. 38, par.
15-9)}

\hypertarget{sec.-15-9.-value.}{%
\section*{Sec. 15-9. Value.}\label{sec.-15-9.-value.}}
\addcontentsline{toc}{section}{Sec. 15-9. Value.}

\markright{Sec. 15-9. Value.}

As used in this Part C, the ``value'' of property consisting of any
commercial instrument or any written instrument representing or
embodying rights concerning anything of value, labor, or services or
otherwise of value to the owner shall be:

(a) The ``market value'' of such instrument if such instrument is
negotiable and has a market value; and

(b) The ``actual value'' of such instrument if such instrument is not
negotiable or is otherwise without a market value. For the purpose of
establishing such ``actual value'', the interest of any owner or owners
entitled to part or all of the property represented by such instrument,
by reason of such instrument, may be shown, even if another ``owner''
may be named in the complaint, information or indictment.

(Source: Laws 1967, p.~2849.)

\hypertarget{ilcs-515-10}{%
\subsection*{(720 ILCS 5/15-10)}\label{ilcs-515-10}}
\addcontentsline{toc}{subsection}{(720 ILCS 5/15-10)}

\hypertarget{sec.-15-10.-governmental-property.}{%
\section*{Sec. 15-10. Governmental
property.}\label{sec.-15-10.-governmental-property.}}
\addcontentsline{toc}{section}{Sec. 15-10. Governmental property.}

\markright{Sec. 15-10. Governmental property.}

As used in this Part C, ``governmental property'' means funds or other
property owned by the State, a unit of local government, or a school
district.

(Source: P.A. 94-134, eff. 1-1-06.)

\bookmarksetup{startatroot}

\hypertarget{article-16.-theft-and-related-offenses}{%
\chapter*{Article 16. Theft And Related
Offenses}\label{article-16.-theft-and-related-offenses}}
\addcontentsline{toc}{chapter}{Article 16. Theft And Related Offenses}

\markboth{Article 16. Theft And Related Offenses}{Article 16. Theft And
Related Offenses}

\hypertarget{ilcs-5art.-16-subdiv.-1-heading}{%
\subsection*{(720 ILCS 5/Art. 16, Subdiv. 1
heading)}\label{ilcs-5art.-16-subdiv.-1-heading}}
\addcontentsline{toc}{subsection}{(720 ILCS 5/Art. 16, Subdiv. 1
heading)}

SUBDIVISION 1.

DEFINITIONS

(Source: P.A. 97-597, eff. 1-1-12.)

\hypertarget{ilcs-516-0.1}{%
\subsection*{(720 ILCS 5/16-0.1)}\label{ilcs-516-0.1}}
\addcontentsline{toc}{subsection}{(720 ILCS 5/16-0.1)}

\hypertarget{sec.-16-0.1.-definitions.}{%
\section*{Sec. 16-0.1. Definitions.}\label{sec.-16-0.1.-definitions.}}
\addcontentsline{toc}{section}{Sec. 16-0.1. Definitions.}

\markright{Sec. 16-0.1. Definitions.}

In this Article, unless the context clearly requires otherwise, the
following terms are defined as indicated:

``Access'' means to use, instruct, communicate with, store data in,
retrieve or intercept data from, or otherwise utilize any services of a
computer.

``Coin-operated machine'' includes any automatic vending machine or any
part thereof, parking meter, coin telephone, coin-operated transit
turnstile, transit fare box, coin laundry machine, coin dry cleaning
machine, amusement machine, music machine, vending machine dispensing
goods or services, or money changer.

``Communication device'' means any type of instrument, device, machine,
or equipment which is capable of transmitting, acquiring, decrypting, or
receiving any telephonic, electronic, data, Internet access, audio,
video, microwave, or radio transmissions, signals, communications, or
services, including the receipt, acquisition, transmission, or
decryption of all such communications, transmissions, signals, or
services provided by or through any cable television, fiber optic,
telephone, satellite, microwave, radio, Internet-based, data
transmission, or wireless distribution network, system or facility; or
any part, accessory, or component thereof, including any computer
circuit, security module, smart card, software, computer chip,
electronic mechanism or other component, accessory or part of any
communication device which is capable of facilitating the transmission,
decryption, acquisition or reception of all such communications,
transmissions, signals, or services.

``Communication service'' means any service lawfully provided for a
charge or compensation to facilitate the lawful origination,
transmission, emission, or reception of signs, signals, data, writings,
images, and sounds or intelligence of any nature by telephone, including
cellular telephones or a wire, wireless, radio, electromagnetic,
photo-electronic or photo-optical system; and also any service lawfully
provided by any radio, telephone, cable television, fiber optic,
satellite, microwave, Internet-based or wireless distribution network,
system, facility or technology, including, but not limited to, any and
all electronic, data, video, audio, Internet access, telephonic,
microwave and radio communications, transmissions, signals and services,
and any such communications, transmissions, signals and services
lawfully provided directly or indirectly by or through any of those
networks, systems, facilities or technologies.

``Communication service provider'' means: (1) any person or entity
providing any communication service, whether directly or indirectly, as
a reseller, including, but not limited to, a cellular, paging or other
wireless communications company or other person or entity which, for a
fee, supplies the facility, cell site, mobile telephone switching office
or other equipment or communication service; (2) any person or entity
owning or operating any cable television, fiber optic, satellite,
telephone, wireless, microwave, radio, data transmission or
Internet-based distribution network, system or facility; and (3) any
person or entity providing any communication service directly or
indirectly by or through any such distribution system, network or
facility.

``Computer'' means a device that accepts, processes, stores, retrieves
or outputs data, and includes but is not limited to auxiliary storage
and telecommunications devices connected to computers.

``Continuing course of conduct'' means a series of acts, and the
accompanying mental state necessary for the crime in question,
irrespective of whether the series of acts are continuous or
intermittent.

``Delivery container'' means any bakery basket of wire or plastic used
to transport or store bread or bakery products, any dairy case of wire
or plastic used to transport or store dairy products, and any dolly or
cart of 2 or 4 wheels used to transport or store any bakery or dairy
product.

``Document-making implement'' means any implement, impression, template,
computer file, computer disc, electronic device, computer hardware,
computer software, instrument, or device that is used to make a real or
fictitious or fraudulent personal identification document.

``Financial transaction device'' means any of the following:

(1) An electronic funds transfer card.

(2) A credit card.

(3) A debit card.

(4) A point-of-sale card.

(5) Any instrument, device, card, plate, code, account number, personal
identification number, or a record or copy of a code, account number, or
personal identification number or other means of access to a credit
account or deposit account, or a driver's license or State
identification card used to access a proprietary account, other than
access originated solely by a paper instrument, that can be used alone
or in conjunction with another access device, for any of the following
purposes:

(A) Obtaining money, cash refund or credit account, credit, goods,
services, or any other thing of value.

(B) Certifying or guaranteeing to a person or business the availability
to the device holder of funds on deposit to honor a draft or check
payable to the order of that person or business.

(C) Providing the device holder access to a deposit account for the
purpose of making deposits, withdrawing funds, transferring funds
between deposit accounts, obtaining information pertaining to a deposit
account, or making an electronic funds transfer.

``Full retail value'' means the merchant's stated or advertised price of
the merchandise. ``Full retail value'' includes the aggregate value of
property obtained from retail thefts committed by the same person as
part of a continuing course of conduct from one or more mercantile
establishments in a single transaction or in separate transactions over
a period of one year.

``Internet'' means an interactive computer service or system or an
information service, system, or access software provider that provides
or enables computer access by multiple users to a computer server, and
includes, but is not limited to, an information service, system, or
access software provider that provides access to a network system
commonly known as the Internet, or any comparable system or service and
also includes, but is not limited to, a World Wide Web page, newsgroup,
message board, mailing list, or chat area on any interactive computer
service or system or other online service.

``Library card'' means a card or plate issued by a library facility for
purposes of identifying the person to whom the library card was issued
as authorized to borrow library material, subject to all limitations and
conditions imposed on the borrowing by the library facility issuing such
card.

``Library facility'' includes any public library or museum, or any
library or museum of an educational, historical or eleemosynary
institution, organization or society.

``Library material'' includes any book, plate, picture, photograph,
engraving, painting, sculpture, statue, artifact, drawing, map,
newspaper, pamphlet, broadside, magazine, manuscript, document, letter,
microfilm, sound recording, audiovisual material, magnetic or other
tape, electronic data processing record or other documentary, written or
printed material regardless of physical form or characteristics, or any
part thereof, belonging to, or on loan to or otherwise in the custody of
a library facility.

``Manufacture or assembly of an unlawful access device'' means to make,
produce or assemble an unlawful access device or to modify, alter,
program or re-program any instrument, device, machine, equipment or
software so that it is capable of defeating or circumventing any
technology, device or software used by the provider, owner or licensee
of a communication service or of any data, audio or video programs or
transmissions to protect any such communication, data, audio or video
services, programs or transmissions from unauthorized access,
acquisition, disclosure, receipt, decryption, communication,
transmission or re-transmission.

``Manufacture or assembly of an unlawful communication device'' means to
make, produce or assemble an unlawful communication or wireless device
or to modify, alter, program or reprogram a communication or wireless
device to be capable of acquiring, disrupting, receiving, transmitting,
decrypting, or facilitating the acquisition, disruption, receipt,
transmission or decryption of, a communication service without the
express consent or express authorization of the communication service
provider, or to knowingly assist others in those activities.

``Master sound recording'' means the original physical object on which a
given set of sounds were first recorded and which the original object
from which all subsequent sound recordings embodying the same set of
sounds are directly or indirectly derived.

``Merchandise'' means any item of tangible personal property, including
motor fuel.

``Merchant'' means an owner or operator of any retail mercantile
establishment or any agent, employee, lessee, consignee, officer,
director, franchisee, or independent contractor of the owner or
operator. ``Merchant'' also means a person who receives from an
authorized user of a payment card, or someone the person believes to be
an authorized user, a payment card or information from a payment card,
or what the person believes to be a payment card or information from a
payment card, as the instrument for obtaining, purchasing or receiving
goods, services, money, or anything else of value from the person.

``Motor fuel'' means a liquid, regardless of its properties, used to
propel a vehicle, including gasoline and diesel.

``Online'' means the use of any electronic or wireless device to access
the Internet.

``Payment card'' means a credit card, charge card, debit card, or any
other card that is issued to an authorized card user and that allows the
user to obtain, purchase, or receive goods, services, money, or anything
else of value from a merchant.

``Person with a disability'' means a person who suffers from a physical
or mental impairment resulting from disease, injury, functional disorder
or congenital condition that impairs the individual's mental or physical
ability to independently manage his or her property or financial
resources, or both.

``Personal identification document'' means a birth certificate, a
driver's license, a State identification card, a public, government, or
private employment identification card, a social security card, a
firearm owner's identification card, a credit card, a debit card, or a
passport issued to or on behalf of a person other than the offender, or
any document made or issued, or falsely purported to have been made or
issued, by or under the authority of the United States Government, the
State of Illinois, or any other state political subdivision of any
state, or any other governmental or quasi-governmental organization that
is of a type intended for the purpose of identification of an
individual, or any such document made or altered in a manner that it
falsely purports to have been made on behalf of or issued to another
person or by the authority of one who did not give that authority.

``Personal identifying information'' means any of the following
information:

(1) A person's name.

(2) A person's address.

(3) A person's date of birth.

(4) A person's telephone number.

(5) A person's driver's license number or State of

Illinois identification card as assigned by the Secretary of State of
the State of Illinois or a similar agency of another state.

(6) A person's social security number.

(7) A person's public, private, or government employer, place of
employment, or employment identification number.

(8) The maiden name of a person's mother.

(9) The number assigned to a person's depository account, savings
account, or brokerage account.

(10) The number assigned to a person's credit or debit card, commonly
known as a ``Visa Card'', ``MasterCard'', ``American Express Card'',
``Discover Card'', or other similar cards whether issued by a financial
institution, corporation, or business entity.

(11) Personal identification numbers.

(12) Electronic identification numbers.

(13) Digital signals.

(14) User names, passwords, and any other word, number, character or
combination of the same usable in whole or part to access information
relating to a specific individual, or to the actions taken,
communications made or received, or other activities or transactions of
a specific individual.

(15) Any other numbers or information which can be used to access a
person's financial resources, or to identify a specific individual, or
the actions taken, communications made or received, or other activities
or transactions of a specific individual.

``Premises of a retail mercantile establishment'' includes, but is not
limited to, the retail mercantile establishment; any common use areas in
shopping centers; and all parking areas set aside by a merchant or on
behalf of a merchant for the parking of vehicles for the convenience of
the patrons of such retail mercantile establishment.

``Public water, gas, or power supply, or other public services'' mean
any service subject to regulation by the Illinois Commerce Commission;
any service furnished by a public utility that is owned and operated by
any political subdivision, public institution of higher education or
municipal corporation of this State; any service furnished by any public
utility that is owned by such political subdivision, public institution
of higher education, or municipal corporation and operated by any of its
lessees or operating agents; any service furnished by an electric
cooperative as defined in Section 3.4 of the Electric Supplier Act; or
wireless service or other service regulated by the Federal
Communications Commission.

``Publish'' means to communicate or disseminate information to any one
or more persons, either orally, in person, or by telephone, radio or
television or in writing of any kind, including, without limitation, a
letter or memorandum, circular or handbill, newspaper or magazine
article or book.

``Radio frequency identification device'' means any implement, computer
file, computer disc, electronic device, computer hardware, computer
software, or instrument that is used to activate, read, receive, or
decode information stored on a RFID tag or transponder attached to a
personal identification document.

``RFID tag or transponder'' means a chip or device that contains
personal identifying information from which the personal identifying
information can be read or decoded by another device emitting a radio
frequency that activates or powers a radio frequency emission response
from the chip or transponder.

``Reencoder'' means an electronic device that places encoded information
from the magnetic strip or stripe of a payment card onto the magnetic
strip or stripe of a different payment card.

``Retail mercantile establishment'' means any place where merchandise is
displayed, held, stored or offered for sale to the public.

``Scanning device'' means a scanner, reader, or any other electronic
device that is used to access, read, scan, obtain, memorize, or store,
temporarily or permanently, information encoded on the magnetic strip or
stripe of a payment card.

``Shopping cart'' means those push carts of the type or types which are
commonly provided by grocery stores, drug stores or other retail
mercantile establishments for the use of the public in transporting
commodities in stores and markets and, incidentally, from the stores to
a place outside the store.

``Sound or audio visual recording'' means any sound or audio visual
phonograph record, disc, pre-recorded tape, film, wire, magnetic tape or
other object, device or medium, now known or hereafter invented, by
which sounds or images may be reproduced with or without the use of any
additional machine, equipment or device.

``Stored value card'' means any card, gift card, instrument, or device
issued with or without fee for the use of the cardholder to obtain
money, goods, services, or anything else of value. Stored value cards
include, but are not limited to, cards issued for use as a stored value
card or gift card, and an account identification number or symbol used
to identify a stored value card. ``Stored value card'' does not include
a prepaid card usable at multiple, unaffiliated merchants or at
automated teller machines, or both. ``Stored value card'' shall only
apply to Section 16-25.1 of this Act.

``Theft detection device remover'' means any tool or device specifically
designed and intended to be used to remove any theft detection device
from any merchandise.

``Under-ring'' means to cause the cash register or other sales recording
device to reflect less than the full retail value of the merchandise.

``Unidentified sound or audio visual recording'' means a sound or audio
visual recording without the actual name and full and correct street
address of the manufacturer, and the name of the actual performers or
groups prominently and legibly printed on the outside cover or jacket
and on the label of such sound or audio visual recording.

``Unlawful access device'' means any type of instrument, device,
machine, equipment, technology, or software which is primarily
possessed, used, designed, assembled, manufactured, sold, distributed or
offered, promoted or advertised for the purpose of defeating or
circumventing any technology, device or software, or any component or
part thereof, used by the provider, owner or licensee of any
communication service or of any data, audio or video programs or
transmissions to protect any such communication, audio or video
services, programs or transmissions from unauthorized access,
acquisition, receipt, decryption, disclosure, communication,
transmission or re-transmission.

``Unlawful communication device'' means any electronic serial number,
mobile identification number, personal identification number or any
communication or wireless device that is capable of acquiring or
facilitating the acquisition of a communication service without the
express consent or express authorization of the communication service
provider, or that has been altered, modified, programmed or
reprogrammed, alone or in conjunction with another communication or
wireless device or other equipment, to so acquire or facilitate the
unauthorized acquisition of a communication service. ``Unlawful
communication device'' also means:

(1) any phone altered to obtain service without the express consent or
express authorization of the communication service provider, tumbler
phone, counterfeit or clone phone, tumbler microchip, counterfeit or
clone microchip, scanning receiver of wireless communication service or
other instrument capable of disguising its identity or location or of
gaining unauthorized access to a communications or wireless system
operated by a communication service provider; and

(2) any communication or wireless device which is capable of, or has
been altered, designed, modified, programmed or reprogrammed, alone or
in conjunction with another communication or wireless device or devices,
so as to be capable of, facilitating the disruption, acquisition,
receipt, transmission or decryption of a communication service without
the express consent or express authorization of the communication
service provider, including, but not limited to, any device, technology,
product, service, equipment, computer software or component or part
thereof, primarily distributed, sold, designed, assembled, manufactured,
modified, programmed, reprogrammed or used for the purpose of providing
the unauthorized receipt of, transmission of, disruption of, decryption
of, access to or acquisition of any communication service provided by
any communication service provider.

``Vehicle'' means a motor vehicle, motorcycle, or farm implement that is
self-propelled and that uses motor fuel for propulsion.

``Wireless device'' includes any type of instrument, device, machine, or
equipment that is capable of transmitting or receiving telephonic,
electronic or radio communications, or any part of such instrument,
device, machine, or equipment, or any computer circuit, computer chip,
electronic mechanism, or other component that is capable of facilitating
the transmission or reception of telephonic, electronic, or radio
communications.

(Source: P.A. 102-757, eff. 5-13-22.)

\hypertarget{ilcs-5art.-16-subdiv.-5-heading}{%
\subsection*{(720 ILCS 5/Art. 16, Subdiv. 5
heading)}\label{ilcs-5art.-16-subdiv.-5-heading}}
\addcontentsline{toc}{subsection}{(720 ILCS 5/Art. 16, Subdiv. 5
heading)}

SUBDIVISION 5.

GENERAL THEFT

(Source: P.A. 97-597, eff. 1-1-12.)

\hypertarget{ilcs-516-1-from-ch.-38-par.-16-1}{%
\subsection*{(720 ILCS 5/16-1) (from Ch. 38, par.
16-1)}\label{ilcs-516-1-from-ch.-38-par.-16-1}}
\addcontentsline{toc}{subsection}{(720 ILCS 5/16-1) (from Ch. 38, par.
16-1)}

\hypertarget{sec.-16-1.-theft.}{%
\section*{Sec. 16-1. Theft.}\label{sec.-16-1.-theft.}}
\addcontentsline{toc}{section}{Sec. 16-1. Theft.}

\markright{Sec. 16-1. Theft.}

(a) A person commits theft when he or she knowingly:

(1) Obtains or exerts unauthorized control over property of the owner;
or

(2) Obtains by deception control over property of the owner; or

(3) Obtains by threat control over property of the owner; or

(4) Obtains control over stolen property knowing the property to have
been stolen or under such circumstances as would reasonably induce him
or her to believe that the property was stolen; or

(5) Obtains or exerts control over property in the custody of any law
enforcement agency which any law enforcement officer or any individual
acting in behalf of a law enforcement agency explicitly represents to
the person as being stolen or represents to the person such
circumstances as would reasonably induce the person to believe that the
property was stolen, and

(A) Intends to deprive the owner permanently of the use or benefit of
the property; or

(B) Knowingly uses, conceals or abandons the property in such manner as
to deprive the owner permanently of such use or benefit; or

(C) Uses, conceals, or abandons the property knowing such use,
concealment or abandonment probably will deprive the owner permanently
of such use or benefit.

(b) Sentence.

(1) Theft of property not from the person and not exceeding \$500 in
value is a Class A misdemeanor.

(1.1) Theft of property not from the person and not exceeding \$500 in
value is a Class 4 felony if the theft was committed in a school or
place of worship or if the theft was of governmental property.

(2) A person who has been convicted of theft of property not from the
person and not exceeding \$500 in value who has been previously
convicted of any type of theft, robbery, armed robbery, burglary,
residential burglary, possession of burglary tools, home invasion,
forgery, a violation of Section 4-103, 4-103.1, 4-103.2, or 4-103.3 of
the Illinois Vehicle Code relating to the possession of a stolen or
converted motor vehicle, or a violation of Section 17-36 of the Criminal
Code of 1961 or the Criminal Code of 2012, or Section 8 of the Illinois
Credit Card and Debit Card Act is guilty of a Class 4 felony.

(3) (Blank).

(4) Theft of property from the person not exceeding

\$500 in value, or theft of property exceeding \$500 and not exceeding
\$10,000 in value, is a Class 3 felony.

(4.1) Theft of property from the person not exceeding

\$500 in value, or theft of property exceeding \$500 and not exceeding
\$10,000 in value, is a Class 2 felony if the theft was committed in a
school or place of worship or if the theft was of governmental property.

(5) Theft of property exceeding \$10,000 and not exceeding \$100,000 in
value is a Class 2 felony.

(5.1) Theft of property exceeding \$10,000 and not exceeding \$100,000
in value is a Class 1 felony if the theft was committed in a school or
place of worship or if the theft was of governmental property.

(6) Theft of property exceeding \$100,000 and not exceeding \$500,000 in
value is a Class 1 felony.

(6.1) Theft of property exceeding \$100,000 in value is a Class X felony
if the theft was committed in a school or place of worship or if the
theft was of governmental property.

(6.2) Theft of property exceeding \$500,000 and not exceeding
\$1,000,000 in value is a Class 1 non-probationable felony.

(6.3) Theft of property exceeding \$1,000,000 in value is a Class X
felony.

(7) Theft by deception, as described by paragraph (2) of subsection (a)
of this Section, in which the offender obtained money or property valued
at \$5,000 or more from a victim 60 years of age or older or a person
with a disability is a Class 2 felony.

(8) Theft by deception, as described by paragraph

(2) of subsection (a) of this Section, in which the offender falsely
poses as a landlord or agent or employee of the landlord and obtains a
rent payment or a security deposit from a tenant is a Class 3 felony if
the rent payment or security deposit obtained does not exceed \$500.

(9) Theft by deception, as described by paragraph

(2) of subsection (a) of this Section, in which the offender falsely
poses as a landlord or agent or employee of the landlord and obtains a
rent payment or a security deposit from a tenant is a Class 2 felony if
the rent payment or security deposit obtained exceeds \$500 and does not
exceed \$10,000.

(10) Theft by deception, as described by paragraph

(2) of subsection (a) of this Section, in which the offender falsely
poses as a landlord or agent or employee of the landlord and obtains a
rent payment or a security deposit from a tenant is a Class 1 felony if
the rent payment or security deposit obtained exceeds \$10,000 and does
not exceed \$100,000.

(11) Theft by deception, as described by paragraph

(2) of subsection (a) of this Section, in which the offender falsely
poses as a landlord or agent or employee of the landlord and obtains a
rent payment or a security deposit from a tenant is a Class X felony if
the rent payment or security deposit obtained exceeds \$100,000.

(c) When a charge of theft of property exceeding a specified value is
brought, the value of the property involved is an element of the offense
to be resolved by the trier of fact as either exceeding or not exceeding
the specified value.

(d) Theft by lessee; permissive inference. The trier of fact may infer
evidence that a person intends to deprive the owner permanently of the
use or benefit of the property (1) if a lessee of the personal property
of another fails to return it to the owner within 10 days after written
demand from the owner for its return or (2) if a lessee of the personal
property of another fails to return it to the owner within 24 hours
after written demand from the owner for its return and the lessee had
presented identification to the owner that contained a materially
fictitious name, address, or telephone number. A notice in writing,
given after the expiration of the leasing agreement, addressed and
mailed, by registered mail, to the lessee at the address given by him
and shown on the leasing agreement shall constitute proper demand.

(e) Permissive inference; evidence of intent that a person obtains by
deception control over property. The trier of fact may infer that a
person ``knowingly obtains by deception control over property of the
owner'' when he or she fails to return, within 45 days after written
demand from the owner, the downpayment and any additional payments
accepted under a promise, oral or in writing, to perform services for
the owner for consideration of \$3,000 or more, and the promisor
knowingly without good cause failed to substantially perform pursuant to
the agreement after taking a down payment of 10\% or more of the agreed
upon consideration. This provision shall not apply where the owner
initiated the suspension of performance under the agreement, or where
the promisor responds to the notice within the 45-day notice period. A
notice in writing, addressed and mailed, by registered mail, to the
promisor at the last known address of the promisor, shall constitute
proper demand.

(f) Offender's interest in the property.

(1) It is no defense to a charge of theft of property that the offender
has an interest therein, when the owner also has an interest to which
the offender is not entitled.

(2) Where the property involved is that of the offender's spouse, no
prosecution for theft may be maintained unless the parties were not
living together as man and wife and were living in separate abodes at
the time of the alleged theft.

(Source: P.A. 101-394, eff. 1-1-20 .)

\hypertarget{ilcs-516-1.1}{%
\subsection*{(720 ILCS 5/16-1.1)}\label{ilcs-516-1.1}}
\addcontentsline{toc}{subsection}{(720 ILCS 5/16-1.1)}

\hypertarget{sec.-16-1.1.-repealed.}{%
\section*{Sec. 16-1.1. (Repealed).}\label{sec.-16-1.1.-repealed.}}
\addcontentsline{toc}{section}{Sec. 16-1.1. (Repealed).}

\markright{Sec. 16-1.1. (Repealed).}

(Source: P.A. 95-857, eff. 1-1-09. Repealed by P.A. 97-597, eff.
1-1-12.)

\hypertarget{ilcs-516-1.2}{%
\subsection*{(720 ILCS 5/16-1.2)}\label{ilcs-516-1.2}}
\addcontentsline{toc}{subsection}{(720 ILCS 5/16-1.2)}

\hypertarget{sec.-16-1.2.-repealed.}{%
\section*{Sec. 16-1.2. (Repealed).}\label{sec.-16-1.2.-repealed.}}
\addcontentsline{toc}{section}{Sec. 16-1.2. (Repealed).}

\markright{Sec. 16-1.2. (Repealed).}

(Source: P.A. 84-992. Repealed by P.A. 97-597, eff. 1-1-12.)

\hypertarget{ilcs-516-1.3-from-ch.-38-par.-16-1.3}{%
\subsection*{(720 ILCS 5/16-1.3) (from Ch. 38, par.
16-1.3)}\label{ilcs-516-1.3-from-ch.-38-par.-16-1.3}}
\addcontentsline{toc}{subsection}{(720 ILCS 5/16-1.3) (from Ch. 38, par.
16-1.3)}

(This Section was renumbered as Section 17-56 by P.A. 96-1551.)

\hypertarget{sec.-16-1.3.-renumbered.}{%
\section*{Sec. 16-1.3. (Renumbered).}\label{sec.-16-1.3.-renumbered.}}
\addcontentsline{toc}{section}{Sec. 16-1.3. (Renumbered).}

\markright{Sec. 16-1.3. (Renumbered).}

(Source: P.A. 95-798, eff. 1-1-09. Renumbered by P.A. 96-1551, eff.
7-1-11 .)

\hypertarget{ilcs-516-2-from-ch.-38-par.-16-2}{%
\subsection*{(720 ILCS 5/16-2) (from Ch. 38, par.
16-2)}\label{ilcs-516-2-from-ch.-38-par.-16-2}}
\addcontentsline{toc}{subsection}{(720 ILCS 5/16-2) (from Ch. 38, par.
16-2)}

\hypertarget{sec.-16-2.-theft-of-lost-or-mislaid-property.}{%
\section*{Sec. 16-2. Theft of lost or mislaid
property.}\label{sec.-16-2.-theft-of-lost-or-mislaid-property.}}
\addcontentsline{toc}{section}{Sec. 16-2. Theft of lost or mislaid
property.}

\markright{Sec. 16-2. Theft of lost or mislaid property.}

A person commits theft of lost or mislaid property when he or she
obtains control over the property and:

(a) Knows or learns the identity of the owner or knows, or is aware of,
or learns of a reasonable method of identifying the owner, and

(b) Fails to take reasonable measures to restore the property to the
owner, and

(c) Intends to deprive the owner permanently of the use or benefit of
the property.

(d) Sentence.

Theft of lost or mislaid property where:

(1) the value does not exceed \$500 is a Class B misdemeanor;

(2) the value exceeds \$500 but does not exceed

\$10,000 is a Class A misdemeanor; and

(3) the value exceeds \$10,000 is a Class 4 felony.

(Source: P.A. 97-597, eff. 1-1-12.)

\hypertarget{ilcs-516-3-from-ch.-38-par.-16-3}{%
\subsection*{(720 ILCS 5/16-3) (from Ch. 38, par.
16-3)}\label{ilcs-516-3-from-ch.-38-par.-16-3}}
\addcontentsline{toc}{subsection}{(720 ILCS 5/16-3) (from Ch. 38, par.
16-3)}

\hypertarget{sec.-16-3.-theft-of-labor-or-services-or-use-of-property.}{%
\section*{Sec. 16-3. Theft of labor or services or use of
property.}\label{sec.-16-3.-theft-of-labor-or-services-or-use-of-property.}}
\addcontentsline{toc}{section}{Sec. 16-3. Theft of labor or services or
use of property.}

\markright{Sec. 16-3. Theft of labor or services or use of property.}

(a) A person commits theft when he or she knowingly obtains the
temporary use of property, labor or services of another which are
available only for hire, by means of threat or deception or knowing that
such use is without the consent of the person providing the property,
labor or services. For the purposes of this subsection, library material
is available for hire.

(b) A person commits theft when after (1) renting or leasing a motor
vehicle, (2) obtaining a motor vehicle through a ``driveaway'' service
mode of transportation, (3) renting or leasing equipment exceeding \$500
in value including tools, construction or industry equipment, and such
items as linens, tableware, tents, tables, chairs and other equipment
specially rented for a party or special event, or (4) renting or leasing
any other type of personal property exceeding \$500 in value, under an
agreement in writing which provides for the return of the vehicle,
equipment, or other personal property to a particular place at a
particular time, he or she without good cause knowingly fails to return
the vehicle, equipment, or other personal property to that place within
the time specified, and is thereafter served or sent a written demand
mailed to the last known address, made by certified mail return receipt
requested, to return the vehicle, equipment, or other personal property
within 3 days from the mailing of the written demand, and who without
good cause knowingly fails to return the vehicle, equipment, or any
other personal property to any place of business of the lessor within
the return period. The trier of fact may infer evidence that the person
is without good cause if the person signs the agreement with a name or
address other than his or her own.

(c) A person commits theft when he or she borrows from a library
facility library material which has an aggregate value of \$50 or more
pursuant to an agreement with or procedure established by the library
facility for the return of such library material, and knowingly without
good cause fails to return the library material so borrowed in
accordance with such agreement or procedure, and further knowingly
without good cause fails to return such library material within 30 days
after receiving written notice by certified mail from the library
facility demanding the return of such library material.

(d) Sentence.

A person convicted of theft under subsection (a) is guilty of a Class A
misdemeanor, except that the theft of library material where the
aggregate value exceeds \$300 is a Class 3 felony. A person convicted of
theft under subsection (b) of this Section is guilty of a Class 4
felony. A person convicted of theft under subsection (c) is guilty of a
petty offense for which the offender may be fined an amount not to
exceed \$500 and shall be ordered to reimburse the library for postage
costs, attorney's fees, and actual replacement costs of the materials
not returned, except that theft under subsection (c) where the aggregate
value exceeds \$300 is a Class 3 felony. In addition to any other
penalty imposed, the court may order a person convicted under this
Section to make restitution to the victim of the offense.

For the purpose of sentencing on theft of library material, separate
transactions totalling more than \$300 within a 90-day period shall
constitute a single offense.

(Source: P.A. 99-534, eff. 1-1-17 .)

\hypertarget{ilcs-516-3.1}{%
\subsection*{(720 ILCS 5/16-3.1)}\label{ilcs-516-3.1}}
\addcontentsline{toc}{subsection}{(720 ILCS 5/16-3.1)}

\hypertarget{sec.-16-3.1.-repealed.}{%
\section*{Sec. 16-3.1. (Repealed).}\label{sec.-16-3.1.-repealed.}}
\addcontentsline{toc}{section}{Sec. 16-3.1. (Repealed).}

\markright{Sec. 16-3.1. (Repealed).}

(Source: P.A. 83-1004. Repealed by P.A. 97-597, eff. 1-1-12.)

\hypertarget{ilcs-516-4}{%
\subsection*{(720 ILCS 5/16-4)}\label{ilcs-516-4}}
\addcontentsline{toc}{subsection}{(720 ILCS 5/16-4)}

\hypertarget{sec.-16-4.-repealed.}{%
\section*{Sec. 16-4. (Repealed).}\label{sec.-16-4.-repealed.}}
\addcontentsline{toc}{section}{Sec. 16-4. (Repealed).}

\markright{Sec. 16-4. (Repealed).}

(Source: Laws 1961, p.~1983. Repealed by P.A. 97-597, eff. 1-1-12.)

\hypertarget{ilcs-516-5-from-ch.-38-par.-16-5}{%
\subsection*{(720 ILCS 5/16-5) (from Ch. 38, par.
16-5)}\label{ilcs-516-5-from-ch.-38-par.-16-5}}
\addcontentsline{toc}{subsection}{(720 ILCS 5/16-5) (from Ch. 38, par.
16-5)}

\hypertarget{sec.-16-5.-theft-from-coin-operated-machine.}{%
\section*{Sec. 16-5. Theft from coin-operated
machine.}\label{sec.-16-5.-theft-from-coin-operated-machine.}}
\addcontentsline{toc}{section}{Sec. 16-5. Theft from coin-operated
machine.}

\markright{Sec. 16-5. Theft from coin-operated machine.}

(a) A person commits theft from a coin-operated machine when he or she
knowingly and without authority opens, breaks into, tampers with,
triggers, or damages a coin-operated machine either:

(1) to operate or use the machine; or

(2) with the intent to commit a theft from the machine.

(b) Sentence.

(1) A violation of subdivision (a)(1) is a Class B misdemeanor.

(2) A violation of subdivision (a)(2) is a Class A misdemeanor.

(3) A person who has been convicted of theft from a coin-operated
machine in violation of subdivision (a)(2) and who has been previously
convicted of any type of theft, robbery, armed robbery, burglary,
residential burglary, possession of burglary tools, or home invasion is
guilty of a Class 4 felony.

(Source: P.A. 97-597, eff. 1-1-12.)

\hypertarget{ilcs-516-6-from-ch.-38-par.-16-6}{%
\subsection*{(720 ILCS 5/16-6) (from Ch. 38, par.
16-6)}\label{ilcs-516-6-from-ch.-38-par.-16-6}}
\addcontentsline{toc}{subsection}{(720 ILCS 5/16-6) (from Ch. 38, par.
16-6)}

\hypertarget{sec.-16-6.-theft-related-devices.}{%
\section*{Sec. 16-6. Theft-related
devices.}\label{sec.-16-6.-theft-related-devices.}}
\addcontentsline{toc}{section}{Sec. 16-6. Theft-related devices.}

\markright{Sec. 16-6. Theft-related devices.}

(a)(1) A person commits unlawful possession of a key or device for a
coin-operated machine when he or she possesses a key, drawing, print,
mold of a key, device, or substance designed to open, break into, tamper
with, or damage a coin-operated machine, with intent to commit a theft
from the machine.

(2) A person commits unlawful use of a key or device for a coin-operated
machine when he or she with the intent to commit a theft from a
coin-operated machine uses a key, drawing, print, mold of a key, device,
or substance and causes damage or loss to the coin-operated machine of
more than \$300.

(b)(1) A person commits unlawful use of a theft detection shielding
device when he or she knowingly manufactures, sells, offers for sale or
distributes any theft detection shielding device.

(2) A person commits unlawful possession of a theft detection shielding
device when he or she knowingly possesses a theft detection shielding
device with the intent to commit theft or retail theft.

(3) A person commits unlawful possession of a theft detection device
remover when he or she knowingly possesses a theft detection device
remover with the intent to use such tool to remove any theft detection
device from any merchandise without the permission of the merchant or
person owning or holding the merchandise.

(c) A person commits use of a scanning device or reencoder to defraud
when the person knowingly uses:

(1) a scanning device to access, read, obtain, memorize, or store,
temporarily or permanently, information encoded on the magnetic strip or
stripe of a payment card without the permission of the authorized user
of the payment card and with the intent to defraud the authorized user,
the issuer of the authorized user's payment card, or a merchant; or

(2) a reencoder to place information encoded on the magnetic strip or
stripe of a payment card onto the magnetic strip or stripe of a
different card without the permission of the authorized user of the card
from which the information is being reencoded and with the intent to
defraud the authorized user, the issuer of the authorized user's payment
card, or a merchant.

(d) Sentence. A violation of subdivision (a)(1), (b)(1), (b)(2), or
(b)(3) is a Class A misdemeanor. A second or subsequent violation of
subdivision (b)(1), (b)(2), or (b)(3) is a Class 4 felony. A violation
of subdivision (a)(2), (c)(1), or (c)(2) is a Class 4 felony. A second
or subsequent violation of subdivision (c)(1) or (c)(2) is a Class 3
felony.

(e) The owner of a coin-operated machine may maintain a civil cause of
action against a person engaged in the activities covered in
subdivisions (a)(1) and (a)(2) and may recover treble actual damages,
reasonable attorney's fees, and costs.

(f) As used in this Section, ``substance'' means a corrosive or acidic
liquid or solid but does not include items purchased through a
coin-operated machine at the location or acquired as condiments at the
location of the coin-operated machine.

(g) For the purposes of this Section, ``theft detection shielding
device'' means any laminated or coated bag or device peculiar to and
marketed for shielding and intended to shield merchandise from detection
by an electronic or magnetic theft alarm sensor.

(Source: P.A. 97-597, eff. 1-1-12.)

\hypertarget{ilcs-516-7-from-ch.-38-par.-16-7}{%
\subsection*{(720 ILCS 5/16-7) (from Ch. 38, par.
16-7)}\label{ilcs-516-7-from-ch.-38-par.-16-7}}
\addcontentsline{toc}{subsection}{(720 ILCS 5/16-7) (from Ch. 38, par.
16-7)}

\hypertarget{sec.-16-7.-unlawful-use-of-recorded-sounds-or-images.}{%
\section*{Sec. 16-7. Unlawful use of recorded sounds or
images.}\label{sec.-16-7.-unlawful-use-of-recorded-sounds-or-images.}}
\addcontentsline{toc}{section}{Sec. 16-7. Unlawful use of recorded
sounds or images.}

\markright{Sec. 16-7. Unlawful use of recorded sounds or images.}

(a) A person commits unlawful use of recorded sounds or images when he
or she knowingly or recklessly:

(1) transfers or causes to be transferred without the consent of the
owner, any sounds or images recorded on any sound or audio visual
recording with the intent of selling or causing to be sold, or using or
causing to be used for profit the article to which such sounds or
recordings of sound are transferred;

(2) sells, offers for sale, advertises for sale, uses or causes to be
used for profit any such article described in subdivision (a)(1) without
consent of the owner;

(3) offers or makes available for a fee, rental or any other form of
compensation, directly or indirectly, any equipment or machinery for the
purpose of use by another to reproduce or transfer, without the consent
of the owner, any sounds or images recorded on any sound or audio visual
recording to another sound or audio visual recording or for the purpose
of use by another to manufacture any sound or audio visual recording in
violation of subsection (b); or

(4) transfers or causes to be transferred without the consent of the
owner, any live performance with the intent of selling or causing to be
sold, or using or causing to be used for profit the sound or audio
visual recording to which the performance is transferred.

(b) A person commits unlawful use of unidentified sound or audio visual
recordings when he or she knowingly, recklessly, or negligently for
profit manufacturers, sells, distributes, vends, circulates, performs,
leases, possesses, or otherwise deals in and with unidentified sound or
audio visual recordings or causes the manufacture, sale, distribution,
vending, circulation, performance, lease, or other dealing in and with
unidentified sound or audio visual recordings.

(c) For the purposes of this Section, ``owner'' means the person who
owns the master sound recording on which sound is recorded and from
which the transferred recorded sounds are directly or indirectly
derived, or the person who owns the rights to record or authorize the
recording of a live performance.

For the purposes of this Section, ``manufacturer'' means the person who
actually makes or causes to be made a sound or audio visual recording.
``Manufacturer'' does not include a person who manufactures the medium
upon which sounds or visual images can be recorded or stored, or who
manufactures the cartridge or casing itself.

(d) Sentence. Unlawful use of recorded sounds or images or unidentified
sound or audio visual recordings is a Class 4 felony; however:

(1) If the offense involves more than 100 but not exceeding 1000
unidentified sound recordings or more than 7 but not exceeding 65
unidentified audio visual recordings during any 180 day period the
authorized fine is up to \$100,000; and

(2) If the offense involves more than 1,000 unidentified sound
recordings or more than 65 unidentified audio visual recordings during
any 180 day period the authorized fine is up to \$250,000.

(e) Upon conviction of any violation of subsection (b), the offender
shall be sentenced to make restitution to any owner or lawful producer
of a master sound or audio visual recording, or to the trade association
representing such owner or lawful producer, that has suffered injury
resulting from the crime. The order of restitution shall be based on the
aggregate wholesale value of lawfully manufactured and authorized sound
or audio visual recordings corresponding to the non-conforming recorded
devices involved in the offense, and shall include investigative costs
relating to the offense.

(f) Subsection (a) of this Section shall neither enlarge nor diminish
the rights of parties in private litigation.

(g) Subsection (a) of this Section does not apply to any person engaged
in the business of radio or television broadcasting who transfers, or
causes to be transferred, any sounds (other than from the sound track of
a motion picture) solely for the purpose of broadcast transmission.

(h) Each individual manufacture, distribution or sale or transfer for a
consideration of such recorded devices in contravention of subsection
(a) of this Section constitutes a separate violation of this Section.
Each individual manufacture, sale, distribution, vending, circulation,
performance, lease, possession, or other dealing in and with an
unidentified sound or audio visual recording under subsection (b) of
this Section constitutes a separate violation of this Section.

(i) Any sound or audio visual recordings containing transferred sounds
or a performance whose transfer was not authorized by the owner of the
master sound recording or performance, or any unidentified sound or
audio visual recording used, in violation of this Section, or in the
attempt to commit such violation as defined in Section 8-4, or in a
conspiracy to commit such violation as defined in Section 8-2, or in a
solicitation to commit such offense as defined in Section 8-1, may be
confiscated and destroyed upon conclusion of the case or cases to which
they are relevant, except that the court may enter an order preserving
them as evidence for use in other cases or pending the final
determination of an appeal.

(j) It is an affirmative defense to any charge of unlawful use of
recorded sounds or images that the recorded sounds or images so used are
public domain material. For purposes of this Section, recorded sounds
are deemed to be in the public domain if the recorded sounds were
copyrighted pursuant to the copyright laws of the United States, as the
same may be amended from time to time, and the term of the copyright and
any extensions or renewals thereof has expired.

(k) With respect to sound recordings (other than accompanying a motion
picture or other audiovisual work), this Section applies only to sound
recordings that were initially recorded before February 15, 1972.

(Source:

P.A. 97-538, eff. 1-1-12; 97-597, eff. 1-1-12; 97-1109, eff. 1-1-13.)

\hypertarget{ilcs-516-8}{%
\subsection*{(720 ILCS 5/16-8)}\label{ilcs-516-8}}
\addcontentsline{toc}{subsection}{(720 ILCS 5/16-8)}

\hypertarget{sec.-16-8.-repealed.}{%
\section*{Sec. 16-8. (Repealed).}\label{sec.-16-8.-repealed.}}
\addcontentsline{toc}{section}{Sec. 16-8. (Repealed).}

\markright{Sec. 16-8. (Repealed).}

(Source: P.A. 95-485, eff. 1-1-08. Repealed by P.A. 97-597, eff.
1-1-12.)

\hypertarget{ilcs-516-10-from-ch.-38-par.-16-10}{%
\subsection*{(720 ILCS 5/16-10) (from Ch. 38, par.
16-10)}\label{ilcs-516-10-from-ch.-38-par.-16-10}}
\addcontentsline{toc}{subsection}{(720 ILCS 5/16-10) (from Ch. 38, par.
16-10)}

\hypertarget{sec.-16-10.-repealed.}{%
\section*{Sec. 16-10. (Repealed).}\label{sec.-16-10.-repealed.}}
\addcontentsline{toc}{section}{Sec. 16-10. (Repealed).}

\markright{Sec. 16-10. (Repealed).}

(Source: P.A. 90-655, eff. 7-30-98. Repealed by P.A. 92-728, eff.
1-1-03.)

\hypertarget{ilcs-516-11-from-ch.-38-par.-16-11}{%
\subsection*{(720 ILCS 5/16-11) (from Ch. 38, par.
16-11)}\label{ilcs-516-11-from-ch.-38-par.-16-11}}
\addcontentsline{toc}{subsection}{(720 ILCS 5/16-11) (from Ch. 38, par.
16-11)}

\hypertarget{sec.-16-11.-repealed.}{%
\section*{Sec. 16-11. (Repealed).}\label{sec.-16-11.-repealed.}}
\addcontentsline{toc}{section}{Sec. 16-11. (Repealed).}

\markright{Sec. 16-11. (Repealed).}

(Source: P.A. 88-466. Repealed by P.A. 92-728, eff. 1-1-03.)

\hypertarget{ilcs-516-12-from-ch.-38-par.-16-12}{%
\subsection*{(720 ILCS 5/16-12) (from Ch. 38, par.
16-12)}\label{ilcs-516-12-from-ch.-38-par.-16-12}}
\addcontentsline{toc}{subsection}{(720 ILCS 5/16-12) (from Ch. 38, par.
16-12)}

\hypertarget{sec.-16-12.-repealed.}{%
\section*{Sec. 16-12. (Repealed).}\label{sec.-16-12.-repealed.}}
\addcontentsline{toc}{section}{Sec. 16-12. (Repealed).}

\markright{Sec. 16-12. (Repealed).}

(Source: P.A. 88-466. Repealed by P.A. 92-728, eff. 1-1-03.)

\hypertarget{ilcs-516-13-from-ch.-38-par.-16-13}{%
\subsection*{(720 ILCS 5/16-13) (from Ch. 38, par.
16-13)}\label{ilcs-516-13-from-ch.-38-par.-16-13}}
\addcontentsline{toc}{subsection}{(720 ILCS 5/16-13) (from Ch. 38, par.
16-13)}

\hypertarget{sec.-16-13.-repealed.}{%
\section*{Sec. 16-13. (Repealed).}\label{sec.-16-13.-repealed.}}
\addcontentsline{toc}{section}{Sec. 16-13. (Repealed).}

\markright{Sec. 16-13. (Repealed).}

(Source: P.A. 83-519. Repealed by P.A. 92-728, eff. 1-1-03.)

\hypertarget{ilcs-516-14-from-ch.-38-par.-16-14}{%
\subsection*{(720 ILCS 5/16-14) (from Ch. 38, par.
16-14)}\label{ilcs-516-14-from-ch.-38-par.-16-14}}
\addcontentsline{toc}{subsection}{(720 ILCS 5/16-14) (from Ch. 38, par.
16-14)}

\hypertarget{sec.-16-14.-theft-of-utility-services.}{%
\section*{Sec. 16-14. Theft of utility
services.}\label{sec.-16-14.-theft-of-utility-services.}}
\addcontentsline{toc}{section}{Sec. 16-14. Theft of utility services.}

\markright{Sec. 16-14. Theft of utility services.}

(a) A person commits theft of utility services when he or she knowingly,
without authority, diverts or interferes with any public water, gas,
power supply, or other public services or installs any device with the
intent to divert or interfere with any public water, gas, power supply,
or other public services without the authority of the owner or entity
furnishing or transmitting such product or services.

(b) Sentence.

(1) Except as provided in paragraph (3), a violation of this Section is
a Class A misdemeanor unless the offense was committed for remuneration,
in which case it is a Class 4 felony.

(2) Except as provided in paragraph (3), a second or subsequent
violation of this Section is a Class 4 felony.

(3) If the offense causes disruption of the public utility services or
delay in the restoration of the public utility services to 10 or more
customers or affects an area of more than one square mile, a violation
of this Section is a Class 2 felony.

(c) This Section does not apply to the theft of telecommunication
services.

(Source: P.A. 97-597, eff. 1-1-12.)

\hypertarget{ilcs-516-15}{%
\subsection*{(720 ILCS 5/16-15)}\label{ilcs-516-15}}
\addcontentsline{toc}{subsection}{(720 ILCS 5/16-15)}

\hypertarget{sec.-16-15.-repealed.}{%
\section*{Sec. 16-15. (Repealed).}\label{sec.-16-15.-repealed.}}
\addcontentsline{toc}{section}{Sec. 16-15. (Repealed).}

\markright{Sec. 16-15. (Repealed).}

(Source: P.A. 91-357, eff. 7-29-99. Repealed by P.A. 97-597, eff.
1-1-12.)

\hypertarget{ilcs-516-16}{%
\subsection*{(720 ILCS 5/16-16)}\label{ilcs-516-16}}
\addcontentsline{toc}{subsection}{(720 ILCS 5/16-16)}

\hypertarget{sec.-16-16.-repealed.}{%
\section*{Sec. 16-16. (Repealed).}\label{sec.-16-16.-repealed.}}
\addcontentsline{toc}{section}{Sec. 16-16. (Repealed).}

\markright{Sec. 16-16. (Repealed).}

(Source: P.A. 97-347, eff. 1-1-12. Repealed by P.A. 97-597, eff.
1-1-12.)

\hypertarget{ilcs-516-16.1}{%
\subsection*{(720 ILCS 5/16-16.1)}\label{ilcs-516-16.1}}
\addcontentsline{toc}{subsection}{(720 ILCS 5/16-16.1)}

\hypertarget{sec.-16-16.1.-repealed.}{%
\section*{Sec. 16-16.1. (Repealed).}\label{sec.-16-16.1.-repealed.}}
\addcontentsline{toc}{section}{Sec. 16-16.1. (Repealed).}

\markright{Sec. 16-16.1. (Repealed).}

(Source: P.A. 97-347, eff. 1-1-12. Repealed by P.A. 97-597, eff.
1-1-12.)

\hypertarget{ilcs-516-17}{%
\subsection*{(720 ILCS 5/16-17)}\label{ilcs-516-17}}
\addcontentsline{toc}{subsection}{(720 ILCS 5/16-17)}

\hypertarget{sec.-16-17.-theft-of-advertising-services.}{%
\section*{Sec. 16-17. Theft of advertising
services.}\label{sec.-16-17.-theft-of-advertising-services.}}
\addcontentsline{toc}{section}{Sec. 16-17. Theft of advertising
services.}

\markright{Sec. 16-17. Theft of advertising services.}

(a) A person commits theft of advertising services when he or she
knowingly attaches or inserts an unauthorized advertisement in a
newspaper or periodical, and redistributes it to the public or has the
intent to redistribute it to the public.

(b) This Section applies to any newspaper or periodical that is offered
for retail sale or is distributed without charge.

(c) This Section does not apply if the publisher or authorized
distributor of the newspaper or periodical consents to the attachment or
insertion of the advertisement.

(d) In this Section, ``unauthorized advertisement'' means any form of
representation or communication, including any handbill, newsletter,
pamphlet, or notice that contains any letters, words, or pictorial
representation that is attached to or inserted in a newspaper or
periodical without a contractual agreement between the publisher and an
advertiser.

(e) Sentence. Theft of advertising services is a Class A misdemeanor.

(Source: P.A. 97-597, eff. 1-1-12.)

\hypertarget{ilcs-516-18}{%
\subsection*{(720 ILCS 5/16-18)}\label{ilcs-516-18}}
\addcontentsline{toc}{subsection}{(720 ILCS 5/16-18)}

\hypertarget{sec.-16-18.-tampering-with-communication-services-theft-of-communication-services.}{%
\section*{Sec. 16-18. Tampering with communication services; theft of
communication
services.}\label{sec.-16-18.-tampering-with-communication-services-theft-of-communication-services.}}
\addcontentsline{toc}{section}{Sec. 16-18. Tampering with communication
services; theft of communication services.}

\markright{Sec. 16-18. Tampering with communication services; theft of
communication services.}

(a) Injury to wires or obtaining service with intent to defraud. A
person commits injury to wires or obtaining service with intent to
defraud when he or she knowingly:

(1) displaces, removes, injures or destroys any telegraph or telephone
line, wire, cable, pole or conduit, belonging to another, or the
material or property appurtenant thereto; or

(2) cuts, breaks, taps, or makes any connection with any telegraph or
telephone line, wire, cable or instrument belonging to another; or

(3) reads, takes or copies any message, communication or report intended
for another passing over any such telegraph line, wire or cable in this
State; or

(4) prevents, obstructs or delays by any means or contrivance
whatsoever, the sending, transmission, conveyance or delivery in this
State of any message, communication or report by or through any
telegraph or telephone line, wire or cable; or

(5) uses any apparatus to unlawfully do or cause to be done any of the
acts described in subdivisions (a)(1) through (a)(4) of this Section; or

(6) obtains, or attempts to obtain, any telecommunications service with
the intent to deprive any person of the lawful charge, in whole or in
part, for any telecommunications service:

(A) by charging such service to an existing telephone number without the
authority of the subscriber thereto; or

(B) by charging such service to a nonexistent, false, fictitious, or
counterfeit telephone number or to a suspended, terminated, expired,
canceled, or revoked telephone number; or

(C) by use of a code, prearranged scheme, or other similar stratagem or
device whereby said person, in effect, sends or receives information; or

(D) by publishing the number or code of an existing, canceled, revoked
or nonexistent telephone number, credit number or other credit device or
method of numbering or coding which is employed in the issuance of
telephone numbers, credit numbers or other credit devices which may be
used to avoid the payment of any lawful telephone toll charge; or

(E) by any other trick, stratagem, impersonation, false pretense, false
representation, false statement, contrivance, device, or means.

(b) Theft of communication services. A person commits theft of
communication services when he or she knowingly:

(1) obtains or uses a communication service without the authorization
of, or compensation paid to, the communication service provider;

(2) possesses, uses, manufactures, assembles, distributes, leases,
transfers, or sells, or offers, promotes or advertises for sale, lease,
use, or distribution, an unlawful communication device:

(A) for the commission of a theft of a communication service or to
receive, disrupt, transmit, decrypt, or acquire, or facilitate the
receipt, disruption, transmission, decryption or acquisition, of any
communication service without the express consent or express
authorization of the communication service provider; or

(B) to conceal or to assist another to conceal from any communication
service provider or from any lawful authority the existence or place of
origin or destination of any communication;

(3) modifies, alters, programs or reprograms a communication device for
the purposes described in subdivision (2)(A) or (2)(B);

(4) possesses, uses, manufactures, assembles, leases, distributes,
sells, or transfers, or offers, promotes or advertises for sale, use or
distribution, any unlawful access device; or

(5) possesses, uses, prepares, distributes, gives or otherwise transfers
to another or offers, promotes, or advertises for sale, use or
distribution, any:

(A) plans or instructions for making or assembling an unlawful
communication or access device, with the intent to use or employ the
unlawful communication or access device, or to allow the same to be used
or employed, for a purpose prohibited by this subsection (b), or knowing
or having reason to know that the plans or instructions are intended to
be used for manufacturing or assembling the unlawful communication or
access device for a purpose prohibited by this subsection (b); or

(B) material, including hardware, cables, tools, data, computer software
or other information or equipment, knowing that the purchaser or a third
person intends to use the material in the manufacture or assembly of an
unlawful communication or access device for a purpose prohibited by this
subsection (b).

(c) Sentence.

(1) A violation of subsection (a) is a Class A misdemeanor; provided,
however, that any of the following is a Class 4 felony:

(A) a second or subsequent conviction for a violation of subsection (a);
or

(B) an offense committed for remuneration; or

(C) an offense involving damage or destruction of property in an amount
in excess of \$300 or defrauding of services in excess of \$500.

(2) A violation of subsection (b) is a Class A misdemeanor, except that:

(A) A violation of subsection (b) is a Class 4 felony if:

(i) the violation of subsection (b) involves at least 10, but not more
than 50, unlawful communication or access devices; or

(ii) the defendant engages in conduct identified in subdivision (b)(3)
of this Section with the intention of substantially disrupting and
impairing the ability of a communication service provider to deliver
communication services to its lawful customers or subscribers; or

(iii) the defendant at the time of the commission of the offense is a
pre-trial detainee at a penal institution or is serving a sentence at a
penal institution; or

(iv) the defendant at the time of the commission of the offense is a
pre-trial detainee at a penal institution or is serving a sentence at a
penal institution and uses any means of electronic communication as
defined in Section 26.5-0.1 of this Code for fraud, theft, theft by
deception, identity theft, or any other unlawful purpose; or

(v) the aggregate value of the service obtained is \$300 or more; or

(vi) the violation is for a wired communication service or device and
the defendant has been convicted previously for an offense under
subsection (b) or for any other type of theft, robbery, armed robbery,
burglary, residential burglary, possession of burglary tools, home
invasion, or fraud, including violations of the Cable Communications
Policy Act of 1984 in this or any federal or other state jurisdiction.

(B) A violation of subsection (b) is a Class 3 felony if:

(i) the violation of subsection (b) involves more than 50 unlawful
communication or access devices; or

(ii) the defendant at the time of the commission of the offense is a
pre-trial detainee at a penal institution or is serving a sentence at a
penal institution and has been convicted previously of an offense under
subsection (b) committed by the defendant while serving as a pre-trial
detainee in a penal institution or while serving a sentence at a penal
institution; or

(iii) the defendant at the time of the commission of the offense is a
pre-trial detainee at a penal institution or is serving a sentence at a
penal institution and has been convicted previously of an offense under
subsection (b) committed by the defendant while serving as a pre-trial
detainee in a penal institution or while serving a sentence at a penal
institution and uses any means of electronic communication as defined in
Section 26.5-0.1 of this Code for fraud, theft, theft by deception,
identity theft, or any other unlawful purpose; or

(iv) the violation is for a wired communication service or device and
the defendant has been convicted previously on 2 or more occasions for
offenses under subsection (b) or for any other type of theft, robbery,
armed robbery, burglary, residential burglary, possession of burglary
tools, home invasion, or fraud, including violations of the Cable
Communications Policy Act of 1984 in this or any federal or other state
jurisdiction.

(C) A violation of subsection (b) is a Class 2 felony if the violation
is for a wireless communication service or device and the defendant has
been convicted previously for an offense under subsection (b) or for any
other type of theft, robbery, armed robbery, burglary, residential
burglary, possession of burglary tools, home invasion, or fraud,
including violations of the Cable Communications Policy Act of 1984 in
this or any federal or other state jurisdiction.

(3) Restitution. The court shall, in addition to any other sentence
authorized by law, sentence a person convicted of violating subsection
(b) to make restitution in the manner provided in Article 5 of Chapter V
of the Unified Code of Corrections.

(d) Grading of offense based on prior convictions. For purposes of
grading an offense based upon a prior conviction for an offense under
subsection (b) or for any other type of theft, robbery, armed robbery,
burglary, residential burglary, possession of burglary tools, home
invasion, or fraud, including violations of the Cable Communications
Policy Act of 1984 in this or any federal or other state jurisdiction
under subdivisions (c)(2)(A)(i) and (c)(2)(B)(i) of this Section, a
prior conviction shall consist of convictions upon separate indictments
or criminal complaints for offenses under subsection (b) or for any
other type of theft, robbery, armed robbery, burglary, residential
burglary, possession of burglary tools, home invasion, or fraud,
including violations of the Cable Communications Policy Act of 1984 in
this or any federal or other state jurisdiction.

(e) Separate offenses. For purposes of all criminal penalties or fines
established for violations of subsection (b), the prohibited activity
established in subsection (b) as it applies to each unlawful
communication or access device shall be deemed a separate offense.

(f) Forfeiture of unlawful communication or access devices. Upon
conviction of a defendant under subsection (b), the court may, in
addition to any other sentence authorized by law, direct that the
defendant forfeit any unlawful communication or access devices in the
defendant's possession or control which were involved in the violation
for which the defendant was convicted.

(g) Venue. An offense under subsection (b) may be deemed to have been
committed at either the place where the defendant manufactured or
assembled an unlawful communication or access device, or assisted others
in doing so, or the place where the unlawful communication or access
device was sold or delivered to a purchaser or recipient. It is not a
defense to a violation of subsection (b) that some of the acts
constituting the offense occurred outside of the State of Illinois.

(h) Civil action. For purposes of subsection (b):

(1) Bringing a civil action. Any person aggrieved by a violation may
bring a civil action in any court of competent jurisdiction.

(2) Powers of the court. The court may:

(A) grant preliminary and final injunctions to prevent or restrain
violations without a showing by the plaintiff of special damages,
irreparable harm or inadequacy of other legal remedies;

(B) at any time while an action is pending, order the impounding, on
such terms as it deems reasonable, of any unlawful communication or
access device that is in the custody or control of the violator and that
the court has reasonable cause to believe was involved in the alleged
violation;

(C) award damages as described in subdivision

(h)(3);

(D) award punitive damages;

(E) in its discretion, award reasonable attorney's fees and costs,
including, but not limited to, costs for investigation, testing and
expert witness fees, to an aggrieved party who prevails; and

(F) as part of a final judgment or decree finding a violation, order the
remedial modification or destruction of any unlawful communication or
access device involved in the violation that is in the custody or
control of the violator or has been impounded under subdivision
(h)(2)(B).

(3) Types of damages recoverable. Damages awarded by a court under this
Section shall be computed as either of the following:

(A) Upon his or her election of such damages at any time before final
judgment is entered, the complaining party may recover the actual
damages suffered by him or her as a result of the violation and any
profits of the violator that are attributable to the violation and are
not taken into account in computing the actual damages; in determining
the violator's profits, the complaining party shall be required to prove
only the violator's gross revenue, and the violator shall be required to
prove his or her deductible expenses and the elements of profit
attributable to factors other than the violation; or

(B) Upon election by the complaining party at any time before final
judgment is entered, that party may recover in lieu of actual damages an
award of statutory damages of not less than \$250 and not more than
\$10,000 for each unlawful communication or access device involved in
the action, with the amount of statutory damages to be determined by the
court, as the court considers just. In any case, if the court finds that
any of the violations were committed with the intent to obtain
commercial advantage or private financial gain, the court in its
discretion may increase the award of statutory damages by an amount of
not more than \$50,000 for each unlawful communication or access device
involved in the action.

(4) Separate violations. For purposes of all civil remedies established
for violations, the prohibited activity established in this Section
applies to each unlawful communication or access device and shall be
deemed a separate violation.

(Source: P.A. 97-597, eff. 1-1-12; 97-1108, eff. 1-1-13.)

\hypertarget{ilcs-516-19}{%
\subsection*{(720 ILCS 5/16-19)}\label{ilcs-516-19}}
\addcontentsline{toc}{subsection}{(720 ILCS 5/16-19)}

\hypertarget{sec.-16-19.-repealed.}{%
\section*{Sec. 16-19. (Repealed).}\label{sec.-16-19.-repealed.}}
\addcontentsline{toc}{section}{Sec. 16-19. (Repealed).}

\markright{Sec. 16-19. (Repealed).}

(Source: P.A. 92-728, eff. 1-1-03. Repealed by P.A. 97-597, eff.
1-1-12.)

\hypertarget{ilcs-516-20}{%
\subsection*{(720 ILCS 5/16-20)}\label{ilcs-516-20}}
\addcontentsline{toc}{subsection}{(720 ILCS 5/16-20)}

\hypertarget{sec.-16-20.-repealed.}{%
\section*{Sec. 16-20. (Repealed).}\label{sec.-16-20.-repealed.}}
\addcontentsline{toc}{section}{Sec. 16-20. (Repealed).}

\markright{Sec. 16-20. (Repealed).}

(Source: P.A. 96-497, eff. 1-1-10. Repealed by P.A. 97-597, eff.
1-1-12.)

\hypertarget{ilcs-516-21}{%
\subsection*{(720 ILCS 5/16-21)}\label{ilcs-516-21}}
\addcontentsline{toc}{subsection}{(720 ILCS 5/16-21)}

\hypertarget{sec.-16-21.-repealed.}{%
\section*{Sec. 16-21. (Repealed).}\label{sec.-16-21.-repealed.}}
\addcontentsline{toc}{section}{Sec. 16-21. (Repealed).}

\markright{Sec. 16-21. (Repealed).}

(Source: P.A. 92-728, eff. 1-1-03. Repealed by P.A. 97-597, eff.
1-1-12.)

\hypertarget{ilcs-516-22}{%
\subsection*{(720 ILCS 5/16-22)}\label{ilcs-516-22}}
\addcontentsline{toc}{subsection}{(720 ILCS 5/16-22)}

(This Section was renumbered as Section 17-11.5 by P.A. 96-1551.)

\hypertarget{sec.-16-22.-renumbered.}{%
\section*{Sec. 16-22. (Renumbered).}\label{sec.-16-22.-renumbered.}}
\addcontentsline{toc}{section}{Sec. 16-22. (Renumbered).}

\markright{Sec. 16-22. (Renumbered).}

(Source: P.A. 94-707, eff. 6-1-06. Renumbered by P.A. 96-1551, eff.
7-1-11 .)

\hypertarget{ilcs-5art.-16-subdiv.-10-heading}{%
\subsection*{(720 ILCS 5/Art. 16, Subdiv. 10
heading)}\label{ilcs-5art.-16-subdiv.-10-heading}}
\addcontentsline{toc}{subsection}{(720 ILCS 5/Art. 16, Subdiv. 10
heading)}

SUBDIVISION 10.

RETAIL THEFT

(Source: P.A. 97-597, eff. 1-1-12.)

\hypertarget{ilcs-516-25}{%
\subsection*{(720 ILCS 5/16-25)}\label{ilcs-516-25}}
\addcontentsline{toc}{subsection}{(720 ILCS 5/16-25)}

\hypertarget{sec.-16-25.-retail-theft.}{%
\section*{Sec. 16-25. Retail theft.}\label{sec.-16-25.-retail-theft.}}
\addcontentsline{toc}{section}{Sec. 16-25. Retail theft.}

\markright{Sec. 16-25. Retail theft.}

(a) A person commits retail theft when he or she knowingly:

(1) Takes possession of, carries away, transfers or causes to be carried
away or transferred any merchandise displayed, held, stored or offered
for sale in a retail mercantile establishment with the intention of
retaining such merchandise or with the intention of depriving the
merchant permanently of the possession, use or benefit of such
merchandise without paying the full retail value of such merchandise; or

(2) Alters, transfers, or removes any label, price tag, marking, indicia
of value or any other markings which aid in determining value affixed to
any merchandise displayed, held, stored or offered for sale in a retail
mercantile establishment and attempts to purchase such merchandise at
less than the full retail value with the intention of depriving the
merchant of the full retail value of such merchandise; or

(3) Transfers any merchandise displayed, held, stored or offered for
sale in a retail mercantile establishment from the container in or on
which such merchandise is displayed to any other container with the
intention of depriving the merchant of the full retail value of such
merchandise; or

(4) Under-rings with the intention of depriving the merchant of the full
retail value of the merchandise; or

(5) Removes a shopping cart from the premises of a retail mercantile
establishment without the consent of the merchant given at the time of
such removal with the intention of depriving the merchant permanently of
the possession, use or benefit of such cart; or

(6) Represents to a merchant that he, she, or another is the lawful
owner of property, knowing that such representation is false, and
conveys or attempts to convey that property to a merchant who is the
owner of the property in exchange for money, merchandise credit or other
property of the merchant; or

(7) Uses or possesses any theft detection shielding device or theft
detection device remover with the intention of using such device to
deprive the merchant permanently of the possession, use or benefit of
any merchandise displayed, held, stored or offered for sale in a retail
mercantile establishment without paying the full retail value of such
merchandise; or

(8) Obtains or exerts unauthorized control over property of the owner
and thereby intends to deprive the owner permanently of the use or
benefit of the property when a lessee of the personal property of
another fails to return it to the owner, or if the lessee fails to pay
the full retail value of such property to the lessor in satisfaction of
any contractual provision requiring such, within 10 days after written
demand from the owner for its return. A notice in writing, given after
the expiration of the leasing agreement, by registered mail, to the
lessee at the address given by the lessee and shown on the leasing
agreement shall constitute proper demand.

(b) Theft by emergency exit. A person commits theft by emergency exit
when he or she commits a retail theft as defined in subdivisions (a)(1)
through (a)(8) of this Section and to facilitate the theft he or she
leaves the retail mercantile establishment by use of a designated
emergency exit.

(c) Permissive inference. If any person:

(1) conceals upon his or her person or among his or her belongings
unpurchased merchandise displayed, held, stored or offered for sale in a
retail mercantile establishment; and

station for receiving payments for that merchandise in that retail
mercantile establishment, then the trier of fact may infer that the
person possessed, carried away or transferred such merchandise with the
intention of retaining it or with the intention of depriving the
merchant permanently of the possession, use or benefit of such
merchandise without paying the full retail value of such merchandise.

To ``conceal'' merchandise means that, although there may be some notice
of its presence, that merchandise is not visible through ordinary
observation.

(d) Venue. Multiple thefts committed by the same person as part of a
continuing course of conduct in different jurisdictions that have been
aggregated in one jurisdiction may be prosecuted in any jurisdiction in
which one or more of the thefts occurred.

(e) For the purposes of this Section, ``theft detection shielding
device'' means any laminated or coated bag or device designed and
intended to shield merchandise from detection by an electronic or
magnetic theft alarm sensor.

(f) Sentence.

(1) A violation of any of subdivisions (a)(1) through

(a)(6) and (a)(8) of this Section, the full retail value of which does
not exceed \$300 for property other than motor fuel or \$150 for motor
fuel, is a Class A misdemeanor. A violation of subdivision (a)(7) of
this Section is a Class A misdemeanor for a first offense and a Class 4
felony for a second or subsequent offense. Theft by emergency exit of
property, the full retail value of which does not exceed \$300, is a
Class 4 felony.

(2) A person who has been convicted of retail theft of property under
any of subdivisions (a)(1) through (a)(6) and (a)(8) of this Section,
the full retail value of which does not exceed \$300 for property other
than motor fuel or \$150 for motor fuel, and who has been previously
convicted of any type of theft, robbery, armed robbery, burglary,
residential burglary, possession of burglary tools, home invasion,
unlawful use of a credit card, or forgery is guilty of a Class 4 felony.
A person who has been convicted of theft by emergency exit of property,
the full retail value of which does not exceed \$300, and who has been
previously convicted of any type of theft, robbery, armed robbery,
burglary, residential burglary, possession of burglary tools, home
invasion, unlawful use of a credit card, or forgery is guilty of a Class
3 felony.

(3) Any retail theft of property under any of subdivisions (a)(1)
through (a)(6) and (a)(8) of this Section, the full retail value of
which exceeds \$300 for property other than motor fuel or \$150 for
motor fuel in a single transaction, or in separate transactions
committed by the same person as part of a continuing course of conduct
from one or more mercantile establishments over a period of one year, is
a Class 3 felony. Theft by emergency exit of property, the full retail
value of which exceeds \$300 in a single transaction, or in separate
transactions committed by the same person as part of a continuing course
of conduct from one or more mercantile establishments over a period of
one year, is a Class 2 felony. When a charge of retail theft of property
or theft by emergency exit of property, the full value of which exceeds
\$300, is brought, the value of the property involved is an element of
the offense to be resolved by the trier of fact as either exceeding or
not exceeding \$300.

(Source: P.A. 97-597, eff. 1-1-12.)

\hypertarget{ilcs-516-25.1}{%
\subsection*{(720 ILCS 5/16-25.1)}\label{ilcs-516-25.1}}
\addcontentsline{toc}{subsection}{(720 ILCS 5/16-25.1)}

\hypertarget{sec.-16-25.1.-organized-retail-crime.}{%
\section*{Sec. 16-25.1. Organized retail
crime.}\label{sec.-16-25.1.-organized-retail-crime.}}
\addcontentsline{toc}{section}{Sec. 16-25.1. Organized retail crime.}

\markright{Sec. 16-25.1. Organized retail crime.}

(a) An individual is guilty of organized retail crime when that
individual, in concert with another individual or any group of
individuals, knowingly commits the act of retail theft from one or more
retail mercantile establishments, and in the course of or in furtherance
of such crime or flight therefrom:

(1) knowingly commits assault as defined under

Section 12-1 or battery as defined under Section 12-3(a)(2) on the
premises of the retail mercantile establishment;

(2) knowingly commits a battery under Section

12-3(a)(1) on the premises of the retail mercantile establishment; or

(3) intentionally destroys or damages the property of the retail
mercantile establishment.

(b) An individual is guilty of being a manager of the organized retail
crime when that individual knowingly recruits, organizes, supervises,
finances, or otherwise manages or directs any other individual or
individuals to:

(1) commit the act of retail theft from one or more retail mercantile
establishments, if the aggregate value of the merchandise exceeds \$300,
and the manager or the individual has the intent to resell the
merchandise or otherwise have the merchandise reenter the stream of
commerce;

(2) commit theft of merchandise, the aggregate retail value of which
exceeds \$300, while the merchandise is in transit from the manufacturer
to the retail mercantile establishment, and the manager or the
individual has the intent to resell the merchandise;

(3) obtain control over property for sale or resale, the aggregate
retail value of which exceeds \$300, knowing the property to have been
stolen or under such circumstances as would reasonably induce the
individual to believe that the property was stolen; or

(4) receive, possess, or purchase any merchandise or stored value cards,
the aggregate retail value of which exceeds \$300, obtained from a
fraudulent return with the knowledge that the property was obtained in
violation of this Section or Section 16-25.

(c) If acts or omissions constituting any part of the commission of the
charged offense under the Section occurred in more than one county, each
county has concurrent venue. If the charged offenses under this Section
occurred in more than one county, the counties may join the offenses in
a single criminal pleading and have concurrent venue as to all charged
offenses. When counties have concurrent venue, the first county in which
a criminal complaint, information, or indictment is issued in the case
becomes the county with exclusive venue. A violation of organized retail
crime may be investigated, indicted, and prosecuted pursuant to the
Statewide Grand Jury Act.

(d) Sentence. A violation of paragraph (1) or (3) of subsection (a) is a
Class 3 felony. A violation of paragraph (2) of subsection (a) is a
Class 2 felony. A violation of subsection (b) is a Class 2 felony.

(Source: P.A. 102-757, eff. 5-13-22.)

\hypertarget{ilcs-516-25.2}{%
\subsection*{(720 ILCS 5/16-25.2)}\label{ilcs-516-25.2}}
\addcontentsline{toc}{subsection}{(720 ILCS 5/16-25.2)}

\hypertarget{sec.-16-25.2.-retail-loss-prevention-report-and-notice-requirements.}{%
\section*{Sec. 16-25.2. Retail loss prevention report and notice
requirements.}\label{sec.-16-25.2.-retail-loss-prevention-report-and-notice-requirements.}}
\addcontentsline{toc}{section}{Sec. 16-25.2. Retail loss prevention
report and notice requirements.}

\markright{Sec. 16-25.2. Retail loss prevention report and notice
requirements.}

(a) A retail mercantile establishment that is a victim of a violation of
Section 16-25, 16-25.1, 17-10.6, or 25-4 shall have the right:

(1) to timely notification of all court proceedings as defined under
subsection (e) of Section 3 of the Rights of Crime Victims and Witnesses
Act. Timely notice shall include 7 days' notice of any court
proceedings. Timely notice shall be sent to the location of the retail
mercantile establishment where the violation occurred and to the point
of contact as provided by the retail mercantile establishment. The point
of contact may be any employee of the retail mercantile establishment or
representative as provided by the retail mercantile establishment;

(2) to communicate with the prosecution;

(3) to be reasonably heard at any post-arraignment court proceeding in
which a right of the victim is at issue and any court proceeding
involving a post-arraignment release decision, plea, or sentencing;

(4) to be notified of the conviction, the sentence, the imprisonment,
and the release of the accused; and

(5) to have present at all court proceedings subject to the rules of
evidence an advocate of the retail mercantile establishment's choice.

(b) Unless a retail mercantile establishment refuses to file a report
regarding the incident, the law enforcement agency having jurisdiction
shall file a report concerning the incident with the State's Attorney.
No law enforcement agent shall discourage or attempt to discourage a
retail mercantile establishment from filing a police report concerning
the incident. Upon the request of the retail mercantile establishment,
the law enforcement agency having jurisdiction shall provide a free copy
of the police report concerning the incident, as soon as practicable,
but in no event later than 5 business days after the request. The
Illinois Law Enforcement Training Standards Board shall not consider any
allegation of a violation of this subsection that is contained in a
complaint made under Section 1-35 of the Police and Community Relations
Improvement Act.

(Source: P.A. 102-757, eff. 5-13-22.)

\hypertarget{ilcs-516-26}{%
\subsection*{(720 ILCS 5/16-26)}\label{ilcs-516-26}}
\addcontentsline{toc}{subsection}{(720 ILCS 5/16-26)}

\hypertarget{sec.-16-26.-detention-affirmative-defense.}{%
\section*{Sec. 16-26. Detention; affirmative
defense.}\label{sec.-16-26.-detention-affirmative-defense.}}
\addcontentsline{toc}{section}{Sec. 16-26. Detention; affirmative
defense.}

\markright{Sec. 16-26. Detention; affirmative defense.}

(a) Detention. Any merchant who has reasonable grounds to believe that a
person has committed retail theft may detain the person, on or off the
premises of a retail mercantile establishment, in a reasonable manner
and for a reasonable length of time for all or any of the following
purposes:

(1) To request identification;

(2) To verify such identification;

(3) To make reasonable inquiry as to whether such person has in his
possession unpurchased merchandise and to make reasonable investigation
of the ownership of such merchandise;

(4) To inform a peace officer of the detention of the person and
surrender that person to the custody of a peace officer;

(5) In the case of a minor, to immediately make a reasonable attempt to
inform the parents, guardian or other private person interested in the
welfare of that minor and, at the merchant's discretion, a peace
officer, of this detention and to surrender custody of such minor to
such person.

A merchant may make a detention as permitted in this Section off the
premises of a retail mercantile establishment only if such detention is
pursuant to an immediate pursuit of such person.

A merchant shall be deemed to have reasonable grounds to make a
detention for the purposes of this Section if the merchant detains a
person because such person has in his or her possession either a theft
detection shielding device or a theft detection device remover.

(b) Affirmative defense. A detention as permitted in this Section does
not constitute an arrest or an unlawful restraint, as defined in Section
10-3 of this Code, nor shall it render the merchant liable to the person
so detained.

(c) For the purposes of this Section, ``minor'' means a person who is
less than 19 years of age, is unemancipated, and resides with his or her
parent or parents or legal guardian.

(Source: P.A. 97-597, eff. 1-1-12.)

\hypertarget{ilcs-516-27}{%
\subsection*{(720 ILCS 5/16-27)}\label{ilcs-516-27}}
\addcontentsline{toc}{subsection}{(720 ILCS 5/16-27)}

\hypertarget{sec.-16-27.-civil-liability.}{%
\section*{Sec. 16-27. Civil
liability.}\label{sec.-16-27.-civil-liability.}}
\addcontentsline{toc}{section}{Sec. 16-27. Civil liability.}

\markright{Sec. 16-27. Civil liability.}

(a) A person who commits the offense of retail theft as defined in
subdivision (a)(1), (a)(2), (a)(3), or (a)(8) of Section 16-25 shall be
civilly liable to the merchant of the merchandise in an amount
consisting of:

(i) actual damages equal to the full retail value of the merchandise;
plus

(ii) an amount not less than \$100 nor more than

\$1,000; plus

(iii) attorney's fees and court costs.

(b) If a minor commits the offense of retail theft, the parents or
guardian of the minor shall be civilly liable as provided in this
Section; however, a guardian appointed pursuant to the Juvenile Court
Act of 1987 shall not be liable under this Section. Total recovery under
this Section shall not exceed the maximum recovery permitted under
Section 5 of the Parental Responsibility Law. For the purposes of this
Section, ``minor'' means a person who is less than 19 years of age, is
unemancipated, and resides with his or her parent or parents or legal
guardian.

(c) A conviction or a plea of guilty to the offense of retail theft is
not a prerequisite to the bringing of a civil suit under this Section.

(d) Judgments arising under this Section may be assigned.

(Source: P.A. 97-597, eff. 1-1-12.)

\hypertarget{ilcs-516-28}{%
\subsection*{(720 ILCS 5/16-28)}\label{ilcs-516-28}}
\addcontentsline{toc}{subsection}{(720 ILCS 5/16-28)}

\hypertarget{sec.-16-28.-delivery-container-theft.}{%
\section*{Sec. 16-28. Delivery container
theft.}\label{sec.-16-28.-delivery-container-theft.}}
\addcontentsline{toc}{section}{Sec. 16-28. Delivery container theft.}

\markright{Sec. 16-28. Delivery container theft.}

(a) A person commits delivery container theft when he or she knowingly
does any of the following:

(1) Uses for any purpose, when not on the premises of the owner or an
adjacent parking area, a delivery container of another person which is
marked by a name or mark unless the use is authorized by the owner.

(2) Sells, or offers for sale, a delivery container of another person
which is marked by a name or mark unless the sale is authorized by the
owner.

(3) Defaces, obliterates, destroys, covers up or otherwise removes or
conceals a name or mark on a delivery container of another person
without the written consent of the owner.

(4) Removes the delivery container of another person from the premises,
parking area or any other area under the control of any processor,
distributor or retail establishment, or from any delivery vehicle,
without the consent of the owner of the delivery container. If a person
possesses any marked or named delivery container without the consent of
the owner and while not on the premises, parking area or other area
under control of a processor, distributor or retail establishment doing
business with the owner, the trier of fact may infer that the person
removed the delivery container in violation of this paragraph.

(b) Any common carrier or private carrier for hire, except those engaged
in transporting bakery or dairy products to and from the places where
they are produced, that receives or transports any delivery container
marked with a name or mark without having in its possession a bill of
lading or invoice for that delivery container commits the offense of
delivery container theft.

(c) Sentence. Delivery container theft is a Class B misdemeanor. An
offender may be sentenced to pay a fine of \$150 for the first offense
and \$500 for a second or subsequent offense.

(Source: P.A. 97-597, eff. 1-1-12.)

\hypertarget{ilcs-5art.-16-subdiv.-15-heading}{%
\subsection*{(720 ILCS 5/Art. 16, Subdiv. 15
heading)}\label{ilcs-5art.-16-subdiv.-15-heading}}
\addcontentsline{toc}{subsection}{(720 ILCS 5/Art. 16, Subdiv. 15
heading)}

SUBDIVISION 15.

IDENTITY THEFT

(Source: P.A. 97-597, eff. 1-1-12.)

\hypertarget{ilcs-516-30}{%
\subsection*{(720 ILCS 5/16-30)}\label{ilcs-516-30}}
\addcontentsline{toc}{subsection}{(720 ILCS 5/16-30)}

\hypertarget{sec.-16-30.-identity-theft-aggravated-identity-theft.}{%
\section*{Sec. 16-30. Identity theft; aggravated identity
theft.}\label{sec.-16-30.-identity-theft-aggravated-identity-theft.}}
\addcontentsline{toc}{section}{Sec. 16-30. Identity theft; aggravated
identity theft.}

\markright{Sec. 16-30. Identity theft; aggravated identity theft.}

(a) A person commits identity theft when he or she knowingly:

(1) uses any personal identifying information or personal identification
document of another person to fraudulently obtain credit, money, goods,
services, or other property;

(2) uses any personal identifying information or personal identification
document of another with intent to commit any felony not set forth in
paragraph (1) of this subsection (a);

(3) obtains, records, possesses, sells, transfers, purchases, or
manufactures any personal identifying information or personal
identification document of another with intent to commit any felony;

(4) uses, obtains, records, possesses, sells, transfers, purchases, or
manufactures any personal identifying information or personal
identification document of another knowing that such personal
identifying information or personal identification documents were stolen
or produced without lawful authority;

(5) uses, transfers, or possesses document-making implements to produce
false identification or false documents with knowledge that they will be
used by the person or another to commit any felony;

(6) uses any personal identifying information or personal identification
document of another to portray himself or herself as that person, or
otherwise, for the purpose of gaining access to any personal identifying
information or personal identification document of that person, without
the prior express permission of that person;

(7) uses any personal identifying information or personal identification
document of another for the purpose of gaining access to any record of
the actions taken, communications made or received, or other activities
or transactions of that person, without the prior express permission of
that person;

(7.5) uses, possesses, or transfers a radio frequency identification
device capable of obtaining or processing personal identifying
information from a radio frequency identification (RFID) tag or
transponder with knowledge that the device will be used by the person or
another to commit a felony violation of State law or any violation of
this Article; or

(8) in the course of applying for a building permit with a unit of local
government, provides the license number of a roofing or fire sprinkler
contractor whom he or she does not intend to have perform the work on
the roofing or fire sprinkler portion of the project; it is an
affirmative defense to prosecution under this paragraph (8) that the
building permit applicant promptly informed the unit of local government
that issued the building permit of any change in the roofing or fire
sprinkler contractor.

(b) Aggravated identity theft. A person commits aggravated identity
theft when he or she commits identity theft as set forth in subsection
(a) of this Section:

(1) against a person 60 years of age or older or a person with a
disability; or

(2) in furtherance of the activities of an organized gang.

A defense to aggravated identity theft does not exist merely because the
accused reasonably believed the victim to be a person less than 60 years
of age. For the purposes of this subsection, ``organized gang'' has the
meaning ascribed in Section 10 of the Illinois Streetgang Terrorism
Omnibus Prevention Act.

(c) Knowledge shall be determined by an evaluation of all circumstances
surrounding the use of the other person's identifying information or
document.

(d) When a charge of identity theft or aggravated identity theft of
credit, money, goods, services, or other property exceeding a specified
value is brought, the value of the credit, money, goods, services, or
other property is an element of the offense to be resolved by the trier
of fact as either exceeding or not exceeding the specified value.

(e) Sentence.

(1) Identity theft.

(A) A person convicted of identity theft in violation of paragraph (1)
of subsection (a) shall be sentenced as follows:

(i) Identity theft of credit, money, goods, services, or other property
not exceeding \$300 in value is a Class 4 felony. A person who has been
previously convicted of identity theft of less than \$300 who is
convicted of a second or subsequent offense of identity theft of less
than \$300 is guilty of a Class 3 felony. A person who has been
convicted of identity theft of less than \$300 who has been previously
convicted of any type of theft, robbery, armed robbery, burglary,
residential burglary, possession of burglary tools, home invasion, home
repair fraud, aggravated home repair fraud, or financial exploitation of
an elderly person or person with a disability is guilty of a Class 3
felony. Identity theft of credit, money, goods, services, or other
property not exceeding \$300 in value when the victim of the identity
theft is an active duty member of the Armed Services or Reserve Forces
of the United States or of the Illinois National Guard serving in a
foreign country is a Class 3 felony. A person who has been previously
convicted of identity theft of less than \$300 who is convicted of a
second or subsequent offense of identity theft of less than \$300 when
the victim of the identity theft is an active duty member of the Armed
Services or Reserve Forces of the United States or of the Illinois
National Guard serving in a foreign country is guilty of a Class 2
felony. A person who has been convicted of identity theft of less than
\$300 when the victim of the identity theft is an active duty member of
the Armed Services or Reserve Forces of the United States or of the
Illinois National Guard serving in a foreign country who has been
previously convicted of any type of theft, robbery, armed robbery,
burglary, residential burglary, possession of burglary tools, home
invasion, home repair fraud, aggravated home repair fraud, or financial
exploitation of an elderly person or person with a disability is guilty
of a Class 2 felony.

(ii) Identity theft of credit, money, goods, services, or other property
exceeding \$300 and not exceeding \$2,000 in value is a Class 3 felony.
Identity theft of credit, money, goods, services, or other property
exceeding \$300 and not exceeding \$2,000 in value when the victim of
the identity theft is an active duty member of the Armed Services or
Reserve Forces of the United States or of the Illinois National Guard
serving in a foreign country is a Class 2 felony.

(iii) Identity theft of credit, money, goods, services, or other
property exceeding \$2,000 and not exceeding \$10,000 in value is a
Class 2 felony. Identity theft of credit, money, goods, services, or
other property exceeding \$2,000 and not exceeding \$10,000 in value
when the victim of the identity theft is an active duty member of the
Armed Services or Reserve Forces of the United States or of the Illinois
National Guard serving in a foreign country is a Class 1 felony.

(iv) Identity theft of credit, money, goods, services, or other property
exceeding \$10,000 and not exceeding \$100,000 in value is a Class 1
felony. Identity theft of credit, money, goods, services, or other
property exceeding \$10,000 and not exceeding \$100,000 in value when
the victim of the identity theft is an active duty member of the Armed
Services or Reserve Forces of the United States or of the Illinois
National Guard serving in a foreign country is a Class X felony.

(v) Identity theft of credit, money, goods, services, or other property
exceeding \$100,000 in value is a Class X felony.

(B) A person convicted of any offense enumerated in paragraphs (2)
through (7.5) of subsection (a) is guilty of a Class 3 felony. A person
convicted of any offense enumerated in paragraphs (2) through (7.5) of
subsection (a) when the victim of the identity theft is an active duty
member of the Armed Services or Reserve Forces of the United States or
of the Illinois National Guard serving in a foreign country is guilty of
a Class 2 felony.

(C) A person convicted of any offense enumerated in paragraphs (2)
through (5) and (7.5) of subsection (a) a second or subsequent time is
guilty of a Class 2 felony. A person convicted of any offense enumerated
in paragraphs (2) through (5) and (7.5) of subsection (a) a second or
subsequent time when the victim of the identity theft is an active duty
member of the Armed Services or Reserve Forces of the United States or
of the Illinois National Guard serving in a foreign country is guilty of
a Class 1 felony.

(D) A person who, within a 12-month period, is found in violation of any
offense enumerated in paragraphs (2) through (7.5) of subsection (a)
with respect to the identifiers of, or other information relating to, 3
or more separate individuals, at the same time or consecutively, is
guilty of a Class 2 felony. A person who, within a 12-month period, is
found in violation of any offense enumerated in paragraphs (2) through
(7.5) of subsection (a) with respect to the identifiers of, or other
information relating to, 3 or more separate individuals, at the same
time or consecutively, when the victim of the identity theft is an
active duty member of the Armed Services or Reserve Forces of the United
States or of the Illinois National Guard serving in a foreign country is
guilty of a Class 1 felony.

(E) A person convicted of identity theft in violation of paragraph (2)
of subsection (a) who uses any personal identifying information or
personal identification document of another to purchase methamphetamine
manufacturing material as defined in Section 10 of the Methamphetamine
Control and Community Protection Act with the intent to unlawfully
manufacture methamphetamine is guilty of a Class 2 felony for a first
offense and a Class 1 felony for a second or subsequent offense. A
person convicted of identity theft in violation of paragraph (2) of
subsection (a) who uses any personal identifying information or personal
identification document of another to purchase methamphetamine
manufacturing material as defined in Section 10 of the Methamphetamine
Control and Community Protection Act with the intent to unlawfully
manufacture methamphetamine when the victim of the identity theft is an
active duty member of the Armed Services or Reserve Forces of the United
States or of the Illinois National Guard serving in a foreign country is
guilty of a Class 1 felony for a first offense and a Class X felony for
a second or subsequent offense.

(F) A person convicted of identity theft in violation of paragraph (8)
of subsection (a) of this Section is guilty of a Class 4 felony.

(2) Aggravated identity theft.

(A) Aggravated identity theft of credit, money, goods, services, or
other property not exceeding \$300 in value is a Class 3 felony.

(B) Aggravated identity theft of credit, money, goods, services, or
other property exceeding \$300 and not exceeding \$10,000 in value is a
Class 2 felony.

(C) Aggravated identity theft of credit, money, goods, services, or
other property exceeding \$10,000 in value and not exceeding \$100,000
in value is a Class 1 felony.

(D) Aggravated identity theft of credit, money, goods, services, or
other property exceeding \$100,000 in value is a Class X felony.

(E) Aggravated identity theft for a violation of any offense enumerated
in paragraphs (2) through (7.5) of subsection (a) of this Section is a
Class 2 felony.

(F) Aggravated identity theft when a person who, within a 12-month
period, is found in violation of any offense enumerated in paragraphs
(2) through (7.5) of subsection (a) of this Section with identifiers of,
or other information relating to, 3 or more separate individuals, at the
same time or consecutively, is a Class 1 felony.

(G) A person who has been previously convicted of aggravated identity
theft regardless of the value of the property involved who is convicted
of a second or subsequent offense of aggravated identity theft
regardless of the value of the property involved is guilty of a Class X
felony.

(Source: P.A. 101-324, eff. 1-1-20 .)

\hypertarget{ilcs-516-31}{%
\subsection*{(720 ILCS 5/16-31)}\label{ilcs-516-31}}
\addcontentsline{toc}{subsection}{(720 ILCS 5/16-31)}

\hypertarget{sec.-16-31.-transmission-of-personal-identifying-information.}{%
\section*{Sec. 16-31. Transmission of personal identifying
information.}\label{sec.-16-31.-transmission-of-personal-identifying-information.}}
\addcontentsline{toc}{section}{Sec. 16-31. Transmission of personal
identifying information.}

\markright{Sec. 16-31. Transmission of personal identifying
information.}

(a) A person commits transmission of personal identifying information if
he or she is not a party to a transaction that involves the use of a
financial transaction device and knowingly: (i) secretly or
surreptitiously photographs, or otherwise captures or records,
electronically or by any other means, personal identifying information
from the transaction without the consent of the person whose information
is photographed or otherwise captured, recorded, distributed,
disseminated, or transmitted, or (ii) distributes, disseminates, or
transmits, electronically or by any other means, personal identifying
information from the transaction without the consent of the person whose
information is photographed, or otherwise captured, recorded,
distributed, disseminated, or transmitted.

(b) This Section does not:

(1) prohibit the capture or transmission of personal identifying
information in the ordinary and lawful course of business;

(2) apply to a peace officer of this State, or of the federal
government, or the officer's agent, while in the lawful performance of
the officer's duties;

(3) prohibit a person from being charged with, convicted of, or punished
for any other violation of law committed by that person while violating
or attempting to violate this Section.

(c) Sentence. A person who violates this Section is guilty of a Class A
misdemeanor.

(Source: P.A. 97-597, eff. 1-1-12.)

\hypertarget{ilcs-516-32}{%
\subsection*{(720 ILCS 5/16-32)}\label{ilcs-516-32}}
\addcontentsline{toc}{subsection}{(720 ILCS 5/16-32)}

\hypertarget{sec.-16-32.-facilitating-identity-theft.}{%
\section*{Sec. 16-32. Facilitating identity
theft.}\label{sec.-16-32.-facilitating-identity-theft.}}
\addcontentsline{toc}{section}{Sec. 16-32. Facilitating identity theft.}

\markright{Sec. 16-32. Facilitating identity theft.}

(a) A person commits facilitating identity theft when he or she, in the
course of his or her employment or official duties, has access to the
personal information of another person in the possession of the State of
Illinois, whether written, recorded, or on computer disk, and knowingly,
with the intent of committing identity theft, aggravated identity theft,
or any violation of the Illinois Financial Crime Law, disposes of that
written, recorded, or computerized information in any receptacle, trash
can, or other container that the public could gain access to, without
shredding that information, destroying the recording, or wiping the
computer disk so that the information is either unintelligible or
destroyed.

(b) Sentence. Facilitating identity theft is a Class A misdemeanor for a
first offense and a Class 4 felony for a second or subsequent offense.

(c) For purposes of this Section, ``personal information'' has the
meaning provided in the Personal Information Protection Act.

(Source: P.A. 97-597, eff. 1-1-12.)

\hypertarget{ilcs-516-33}{%
\subsection*{(720 ILCS 5/16-33)}\label{ilcs-516-33}}
\addcontentsline{toc}{subsection}{(720 ILCS 5/16-33)}

\hypertarget{sec.-16-33.-civil-remedies.}{%
\section*{Sec. 16-33. Civil
remedies.}\label{sec.-16-33.-civil-remedies.}}
\addcontentsline{toc}{section}{Sec. 16-33. Civil remedies.}

\markright{Sec. 16-33. Civil remedies.}

A person who is convicted of facilitating identity theft, identity
theft, or aggravated identity theft is liable in a civil action to the
person who suffered damages as a result of the violation. The person
suffering damages may recover court costs, attorney's fees, lost wages,
and actual damages. Where a person has been convicted of identity theft
in violation of subdivision (a)(6) or subdivision (a)(7) of Section
16-30, in the absence of proof of actual damages, the person whose
personal identification information or personal identification documents
were used in the violation in question may recover damages of \$2,000.

(Source: P.A. 97-597, eff. 1-1-12.)

\hypertarget{ilcs-516-34}{%
\subsection*{(720 ILCS 5/16-34)}\label{ilcs-516-34}}
\addcontentsline{toc}{subsection}{(720 ILCS 5/16-34)}

\hypertarget{sec.-16-34.-offenders-interest-in-the-property-consent.}{%
\section*{Sec. 16-34. Offender's interest in the property;
consent.}\label{sec.-16-34.-offenders-interest-in-the-property-consent.}}
\addcontentsline{toc}{section}{Sec. 16-34. Offender's interest in the
property; consent.}

\markright{Sec. 16-34. Offender's interest in the property; consent.}

(a) It is no defense to a charge of aggravated identity theft or
identity theft that the offender has an interest in the credit, money,
goods, services, or other property.

(b) It is no defense to a charge of aggravated identity theft or
identity theft that the offender received the consent of any person to
access any personal identification information or personal
identification document, other than the person described by the personal
identification information or personal identification document used by
the offender.

(Source: P.A. 97-597, eff. 1-1-12.)

\hypertarget{ilcs-516-35}{%
\subsection*{(720 ILCS 5/16-35)}\label{ilcs-516-35}}
\addcontentsline{toc}{subsection}{(720 ILCS 5/16-35)}

\hypertarget{sec.-16-35.-mandating-law-enforcement-agencies-to-accept-and-provide-reports-judicial-factual-determination.}{%
\section*{Sec. 16-35. Mandating law enforcement agencies to accept and
provide reports; judicial factual
determination.}\label{sec.-16-35.-mandating-law-enforcement-agencies-to-accept-and-provide-reports-judicial-factual-determination.}}
\addcontentsline{toc}{section}{Sec. 16-35. Mandating law enforcement
agencies to accept and provide reports; judicial factual determination.}

\markright{Sec. 16-35. Mandating law enforcement agencies to accept and
provide reports; judicial factual determination.}

(a) A person who has learned or reasonably suspects that his or her
personal identifying information has been unlawfully used by another may
initiate a law enforcement investigation by contacting the local law
enforcement agency that has jurisdiction over his or her actual
residence, which shall take a police report of the matter, provide the
complainant with a copy of that report, and begin an investigation of
the facts, or, if the suspected crime was committed in a different
jurisdiction, refer the matter to the law enforcement agency where the
suspected crime was committed for an investigation of the facts.

(b) A person who reasonably believes that he or she is the victim of
financial identity theft may petition a court, or upon application of
the prosecuting attorney or on its own motion, the court may move for an
expedited judicial determination of his or her factual innocence, where
the perpetrator of the financial identity theft was arrested for, cited
for, or convicted of a crime under the victim's identity, or where a
criminal complaint has been filed against the perpetrator in the
victim's name, or where the victim's identity has been mistakenly
associated with a criminal conviction. Any judicial determination of
factual innocence made pursuant to this subsection may be heard and
determined upon declarations, affidavits, police reports, or other
material, relevant, and reliable information submitted by the parties or
ordered to be part of the record by the court. If the court determines
that the petition or motion is meritorious and that there is no
reasonable cause to believe that the victim committed the offense for
which the perpetrator of the identity theft was arrested, cited,
convicted, or subject to a criminal complaint in the victim's name, or
that the victim's identity has been mistakenly associated with a record
of criminal conviction, the court shall find the victim factually
innocent of that offense. If the victim is found factually innocent, the
court shall issue an order certifying this determination.

(c) After a court has issued a determination of factual innocence under
this Section, the court may order the name and associated personal
identifying information contained in the court records, files, and
indexes accessible by the public sealed, deleted, or labeled to show
that the data is impersonated and does not reflect the defendant's
identity.

(d) A court that has issued a determination of factual innocence under
this Section may at any time vacate that determination if the petition,
or any information submitted in support of the petition, is found to
contain any material misrepresentation or fraud.

(e) Except for criminal and civil actions provided for by Sections 16-30
through 16-36, or for disciplinary or licensure-related proceedings
involving the violation of Sections 16-30 through 16-36, no information
acquired by, or as a result of, any violation of Section 16-30 shall be
discoverable or admissible in any court or other proceeding, or
otherwise subject to disclosure without the express permission of any
person or persons identified in that information.

(Source: P.A. 97-597, eff. 1-1-12.)

\hypertarget{ilcs-516-36}{%
\subsection*{(720 ILCS 5/16-36)}\label{ilcs-516-36}}
\addcontentsline{toc}{subsection}{(720 ILCS 5/16-36)}

\hypertarget{sec.-16-36.-venue.}{%
\section*{Sec. 16-36. Venue.}\label{sec.-16-36.-venue.}}
\addcontentsline{toc}{section}{Sec. 16-36. Venue.}

\markright{Sec. 16-36. Venue.}

In addition to any other venues provided for by statute or otherwise,
venue for any criminal prosecution or civil recovery action under
Sections 16-30 through 16-36 shall be proper in any county where the
person described in the personal identification information or personal
identification document in question resides or has his or her principal
place of business. Where a criminal prosecution or civil recovery action
under Sections 16-30 through 16-36 involves the personal identification
information or personal identification documents of more than one
person, venue shall be proper in any county where one or more of the
persons described in the personal identification information or personal
identification documents in question resides or has his or her principal
place of business.

(Source: P.A. 97-597, eff. 1-1-12.)

\hypertarget{ilcs-516-37}{%
\subsection*{(720 ILCS 5/16-37)}\label{ilcs-516-37}}
\addcontentsline{toc}{subsection}{(720 ILCS 5/16-37)}

\hypertarget{sec.-16-37.-exemptions-relation-to-other-laws.}{%
\section*{Sec. 16-37. Exemptions; relation to other
laws.}\label{sec.-16-37.-exemptions-relation-to-other-laws.}}
\addcontentsline{toc}{section}{Sec. 16-37. Exemptions; relation to other
laws.}

\markright{Sec. 16-37. Exemptions; relation to other laws.}

(a) Sections 16-30 through 16-36 do not:

(1) prohibit the capture or transmission of personal identifying
information in the ordinary and lawful course of business;

(2) apply to a peace officer of this State, or of the federal
government, or the officer's agent, while in the lawful performance of
the officer's duties;

(3) prohibit a licensed private detective or licensed private detective
agency from representing himself, herself, or itself as another person,
provided that he, she, or it may not portray himself, herself, or itself
as the person whose information he, she, or it is seeking except as
provided under Sections 16-30 through 16-36;

(4) apply to activities authorized under any other statute.

(b) No criminal prosecution or civil action brought under Sections 16-30
through 16-36 shall prohibit a person from being charged with, convicted
of, or punished for any other violation of law committed by that person
while violating or attempting to violate Sections 16-30 through 16-36.

(Source: P.A. 97-597, eff. 1-1-12.)

\hypertarget{ilcs-5art.-16-subdiv.-20-heading}{%
\subsection*{(720 ILCS 5/Art. 16, Subdiv. 20
heading)}\label{ilcs-5art.-16-subdiv.-20-heading}}
\addcontentsline{toc}{subsection}{(720 ILCS 5/Art. 16, Subdiv. 20
heading)}

SUBDIVISION 20.

MISCELLANEOUS THEFT-RELATED OFFENSES

(Source: P.A. 97-597, eff. 1-1-12.)

\hypertarget{ilcs-516-40}{%
\subsection*{(720 ILCS 5/16-40)}\label{ilcs-516-40}}
\addcontentsline{toc}{subsection}{(720 ILCS 5/16-40)}

\hypertarget{sec.-16-40.-internet-offenses.}{%
\section*{Sec. 16-40. Internet
offenses.}\label{sec.-16-40.-internet-offenses.}}
\addcontentsline{toc}{section}{Sec. 16-40. Internet offenses.}

\markright{Sec. 16-40. Internet offenses.}

(a) Online sale of stolen property. A person commits online sale of
stolen property when he or she uses or accesses the Internet with the
intent of selling property gained through unlawful means.

(b) Online theft by deception. A person commits online theft by
deception when he or she uses the Internet to purchase or attempt to
purchase property from a seller with a mode of payment that he or she
knows is fictitious, stolen, or lacking the consent of the valid account
holder.

(c) Electronic fencing. A person commits electronic fencing when he or
she sells stolen property using the Internet, knowing that the property
was stolen. A person who unknowingly purchases stolen property over the
Internet does not violate this Section.

(d) Sentence. A violation of this Section is a Class 4 felony if the
full retail value of the stolen property or property obtained by
deception does not exceed \$300. A violation of this Section is a Class
2 felony if the full retail value of the stolen property or property
obtained by deception exceeds \$300.

(Source: P.A. 97-597, eff. 1-1-12.)

\bookmarksetup{startatroot}

\hypertarget{article-16a.-retail-theft}{%
\chapter*{Article 16a. Retail Theft}\label{article-16a.-retail-theft}}
\addcontentsline{toc}{chapter}{Article 16a. Retail Theft}

\markboth{Article 16a. Retail Theft}{Article 16a. Retail Theft}

(Repealed)

(Source: Repealed by P.A. 97-597, eff. 1-1-12.)

\bookmarksetup{startatroot}

\hypertarget{article-16b.-protection-of-library-materials}{%
\chapter*{Article 16b. Protection Of Library
Materials}\label{article-16b.-protection-of-library-materials}}
\addcontentsline{toc}{chapter}{Article 16b. Protection Of Library
Materials}

\markboth{Article 16b. Protection Of Library Materials}{Article 16b.
Protection Of Library Materials}

(Repealed)

(Source: Repealed by P.A. 97-597, eff. 1-1-12.)

\bookmarksetup{startatroot}

\hypertarget{article-16c.-unlawful-sale-of-household-appliances}{%
\chapter*{Article 16c. Unlawful Sale Of Household
Appliances}\label{article-16c.-unlawful-sale-of-household-appliances}}
\addcontentsline{toc}{chapter}{Article 16c. Unlawful Sale Of Household
Appliances}

\markboth{Article 16c. Unlawful Sale Of Household Appliances}{Article
16c. Unlawful Sale Of Household Appliances}

(Source: Repealed by P.A. 97-1150, eff. 1-25-13.)

\bookmarksetup{startatroot}

\hypertarget{article-16d.-computer-crime}{%
\chapter*{Article 16d. Computer
Crime}\label{article-16d.-computer-crime}}
\addcontentsline{toc}{chapter}{Article 16d. Computer Crime}

\markboth{Article 16d. Computer Crime}{Article 16d. Computer Crime}

(Source: Repealed by P.A. 97-1150, eff. 1-25-13.)

\bookmarksetup{startatroot}

\hypertarget{article-16e.-delivery-container-crime}{%
\chapter*{Article 16e. Delivery Container
Crime}\label{article-16e.-delivery-container-crime}}
\addcontentsline{toc}{chapter}{Article 16e. Delivery Container Crime}

\markboth{Article 16e. Delivery Container Crime}{Article 16e. Delivery
Container Crime}

(Repealed)

(Source: Repealed by P.A. 97-597, eff. 1-1-12.)

\bookmarksetup{startatroot}

\hypertarget{article-16f.-wireless-service-theft}{%
\chapter*{Article 16f. Wireless Service
Theft}\label{article-16f.-wireless-service-theft}}
\addcontentsline{toc}{chapter}{Article 16f. Wireless Service Theft}

\markboth{Article 16f. Wireless Service Theft}{Article 16f. Wireless
Service Theft}

(Repealed)

(Source: Repealed by P.A. 97-597, eff. 1-1-12.)

\bookmarksetup{startatroot}

\hypertarget{article-16g.-identity-theft-law}{%
\chapter*{Article 16g. Identity Theft
Law}\label{article-16g.-identity-theft-law}}
\addcontentsline{toc}{chapter}{Article 16g. Identity Theft Law}

\markboth{Article 16g. Identity Theft Law}{Article 16g. Identity Theft
Law}

(Repealed)

(Source: Repealed by P.A. 97-597, eff. 1-1-12.)

\bookmarksetup{startatroot}

\hypertarget{article-16h.-illinois-financial-crime-law}{%
\chapter*{Article 16h. Illinois Financial Crime
Law}\label{article-16h.-illinois-financial-crime-law}}
\addcontentsline{toc}{chapter}{Article 16h. Illinois Financial Crime
Law}

\markboth{Article 16h. Illinois Financial Crime Law}{Article 16h.
Illinois Financial Crime Law}

(Repealed)

(Article repealed by P.A. 96-1551, eff. 7-1-11)

\bookmarksetup{startatroot}

\hypertarget{article-16j.-online-property-offenses}{%
\chapter*{Article 16j. Online Property
Offenses}\label{article-16j.-online-property-offenses}}
\addcontentsline{toc}{chapter}{Article 16j. Online Property Offenses}

\markboth{Article 16j. Online Property Offenses}{Article 16j. Online
Property Offenses}

(Repealed)

(Source: P.A. 95-331, eff. 8-21-07. Repealed by P.A. 97-597, eff.
1-1-12.)

\bookmarksetup{startatroot}

\hypertarget{article-16k.-theft-of-motor-fuel}{%
\chapter*{Article 16k. Theft Of Motor
Fuel}\label{article-16k.-theft-of-motor-fuel}}
\addcontentsline{toc}{chapter}{Article 16k. Theft Of Motor Fuel}

\markboth{Article 16k. Theft Of Motor Fuel}{Article 16k. Theft Of Motor
Fuel}

(Repealed)

(Source: P.A. 95-331, eff. 8-21-07. Repealed by P.A. 97-597, eff.
1-1-12.)

\bookmarksetup{startatroot}

\hypertarget{article-17.-deception-and-fraud}{%
\chapter*{Article 17. Deception And
Fraud}\label{article-17.-deception-and-fraud}}
\addcontentsline{toc}{chapter}{Article 17. Deception And Fraud}

\markboth{Article 17. Deception And Fraud}{Article 17. Deception And
Fraud}

(Source: P.A. 96-1551, eff. 7-1-11 .)

\hypertarget{ilcs-5art.-17-subdiv.-1-heading}{%
\subsection*{(720 ILCS 5/Art. 17, Subdiv. 1
heading)}\label{ilcs-5art.-17-subdiv.-1-heading}}
\addcontentsline{toc}{subsection}{(720 ILCS 5/Art. 17, Subdiv. 1
heading)}

SUBDIVISION 1.

GENERAL DEFINITIONS

(Source: P.A. 96-1551, eff. 7-1-11.)

\hypertarget{ilcs-517-0.5}{%
\subsection*{(720 ILCS 5/17-0.5)}\label{ilcs-517-0.5}}
\addcontentsline{toc}{subsection}{(720 ILCS 5/17-0.5)}

\hypertarget{sec.-17-0.5.-definitions.}{%
\section*{Sec. 17-0.5. Definitions.}\label{sec.-17-0.5.-definitions.}}
\addcontentsline{toc}{section}{Sec. 17-0.5. Definitions.}

\markright{Sec. 17-0.5. Definitions.}

In this Article:

``Altered credit card or debit card'' means any instrument or device,
whether known as a credit card or debit card, which has been changed in
any respect by addition or deletion of any material, except for the
signature by the person to whom the card is issued.

``Cardholder'' means the person or organization named on the face of a
credit card or debit card to whom or for whose benefit the credit card
or debit card is issued by an issuer.

``Computer'' means a device that accepts, processes, stores, retrieves,
or outputs data and includes, but is not limited to, auxiliary storage,
including cloud-based networks of remote services hosted on the
Internet, and telecommunications devices connected to computers.

``Computer network'' means a set of related, remotely connected devices
and any communications facilities including more than one computer with
the capability to transmit data between them through the communications
facilities.

``Computer program'' or ``program'' means a series of coded instructions
or statements in a form acceptable to a computer which causes the
computer to process data and supply the results of the data processing.

``Computer services'' means computer time or services, including data
processing services, Internet services, electronic mail services,
electronic message services, or information or data stored in connection
therewith.

``Counterfeit'' means to manufacture, produce or create, by any means, a
credit card or debit card without the purported issuer's consent or
authorization.

``Credit card'' means any instrument or device, whether known as a
credit card, credit plate, charge plate or any other name, issued with
or without fee by an issuer for the use of the cardholder in obtaining
money, goods, services or anything else of value on credit or in
consideration or an undertaking or guaranty by the issuer of the payment
of a check drawn by the cardholder.

``Data'' means a representation in any form of information, knowledge,
facts, concepts, or instructions, including program documentation, which
is prepared or has been prepared in a formalized manner and is stored or
processed in or transmitted by a computer or in a system or network.
Data is considered property and may be in any form, including, but not
limited to, printouts, magnetic or optical storage media, punch cards,
or data stored internally in the memory of the computer.

``Debit card'' means any instrument or device, known by any name, issued
with or without fee by an issuer for the use of the cardholder in
obtaining money, goods, services, and anything else of value, payment of
which is made against funds previously deposited by the cardholder. A
debit card which also can be used to obtain money, goods, services and
anything else of value on credit shall not be considered a debit card
when it is being used to obtain money, goods, services or anything else
of value on credit.

``Document'' includes, but is not limited to, any document,
representation, or image produced manually, electronically, or by
computer.

``Electronic fund transfer terminal'' means any machine or device that,
when properly activated, will perform any of the following services:

(1) Dispense money as a debit to the cardholder's account; or

(2) Print the cardholder's account balances on a statement; or

(3) Transfer funds between a cardholder's accounts; or

(4) Accept payments on a cardholder's loan; or

(5) Dispense cash advances on an open end credit or a revolving charge
agreement; or

(6) Accept deposits to a customer's account; or

(7) Receive inquiries of verification of checks and dispense information
that verifies that funds are available to cover such checks; or

(8) Cause money to be transferred electronically from a cardholder's
account to an account held by any business, firm, retail merchant,
corporation, or any other organization.

``Electronic funds transfer system'', hereafter referred to as ``EFT
System'', means that system whereby funds are transferred electronically
from a cardholder's account to any other account.

``Electronic mail service provider'' means any person who (i) is an
intermediary in sending or receiving electronic mail and (ii) provides
to end-users of electronic mail services the ability to send or receive
electronic mail.

``Expired credit card or debit card'' means a credit card or debit card
which is no longer valid because the term on it has elapsed.

``False academic degree'' means a certificate, diploma, transcript, or
other document purporting to be issued by an institution of higher
learning or purporting to indicate that a person has completed an
organized academic program of study at an institution of higher learning
when the person has not completed the organized academic program of
study indicated on the certificate, diploma, transcript, or other
document.

``False claim'' means any statement made to any insurer, purported
insurer, servicing corporation, insurance broker, or insurance agent, or
any agent or employee of one of those entities, and made as part of, or
in support of, a claim for payment or other benefit under a policy of
insurance, or as part of, or in support of, an application for the
issuance of, or the rating of, any insurance policy, when the statement
does any of the following:

(1) Contains any false, incomplete, or misleading information concerning
any fact or thing material to the claim.

(2) Conceals (i) the occurrence of an event that is material to any
person's initial or continued right or entitlement to any insurance
benefit or payment or (ii) the amount of any benefit or payment to which
the person is entitled.

``Financial institution'' means any bank, savings and loan association,
credit union, or other depository of money or medium of savings and
collective investment.

``Governmental entity'' means: each officer, board, commission, and
agency created by the Constitution, whether in the executive,
legislative, or judicial branch of State government; each officer,
department, board, commission, agency, institution, authority,
university, and body politic and corporate of the State; each
administrative unit or corporate outgrowth of State government that is
created by or pursuant to statute, including units of local government
and their officers, school districts, and boards of election
commissioners; and each administrative unit or corporate outgrowth of
the foregoing items and as may be created by executive order of the
Governor.

``Incomplete credit card or debit card'' means a credit card or debit
card which is missing part of the matter other than the signature of the
cardholder which an issuer requires to appear on the credit card or
debit card before it can be used by a cardholder, and this includes
credit cards or debit cards which have not been stamped, embossed,
imprinted or written on.

``Institution of higher learning'' means a public or private college,
university, or community college located in the State of Illinois that
is authorized by the Board of Higher Education or the Illinois Community
College Board to issue post-secondary degrees, or a public or private
college, university, or community college located anywhere in the United
States that is or has been legally constituted to offer degrees and
instruction in its state of origin or incorporation.

``Insurance company'' means ``company'' as defined under Section 2 of
the Illinois Insurance Code.

``Issuer'' means the business organization or financial institution
which issues a credit card or debit card, or its duly authorized agent.

``Merchant'' has the meaning ascribed to it in Section 16-0.1 of this
Code.

``Person'' means any individual, corporation, government, governmental
subdivision or agency, business trust, estate, trust, partnership or
association or any other entity.

``Receives'' or ``receiving'' means acquiring possession or control.

``Record of charge form'' means any document submitted or intended to be
submitted to an issuer as evidence of a credit transaction for which the
issuer has agreed to reimburse persons providing money, goods, property,
services or other things of value.

``Revoked credit card or debit card'' means a credit card or debit card
which is no longer valid because permission to use it has been suspended
or terminated by the issuer.

``Sale'' means any delivery for value.

``Scheme or artifice to defraud'' includes a scheme or artifice to
deprive another of the intangible right to honest services.

``Self-insured entity'' means any person, business, partnership,
corporation, or organization that sets aside funds to meet his, her, or
its losses or to absorb fluctuations in the amount of loss, the losses
being charged against the funds set aside or accumulated.

``Social networking website'' means an Internet website containing
profile web pages of the members of the website that include the names
or nicknames of such members, photographs placed on the profile web
pages by such members, or any other personal or personally identifying
information about such members and links to other profile web pages on
social networking websites of friends or associates of such members that
can be accessed by other members or visitors to the website. A social
networking website provides members of or visitors to such website the
ability to leave messages or comments on the profile web page that are
visible to all or some visitors to the profile web page and may also
include a form of electronic mail for members of the social networking
website.

``Statement'' means any assertion, oral, written, or otherwise, and
includes, but is not limited to: any notice, letter, or memorandum;
proof of loss; bill of lading; receipt for payment; invoice, account, or
other financial statement; estimate of property damage; bill for
services; diagnosis or prognosis; prescription; hospital, medical, or
dental chart or other record, x-ray, photograph, videotape, or movie
film; test result; other evidence of loss, injury, or expense;
computer-generated document; and data in any form.

``Universal Price Code Label'' means a unique symbol that consists of a
machine-readable code and human-readable numbers.

``With intent to defraud'' means to act knowingly, and with the specific
intent to deceive or cheat, for the purpose of causing financial loss to
another or bringing some financial gain to oneself, regardless of
whether any person was actually defrauded or deceived. This includes an
intent to cause another to assume, create, transfer, alter, or terminate
any right, obligation, or power with reference to any person or
property.

(Source: P.A. 101-87, eff. 1-1-20 .)

\hypertarget{ilcs-5art.-17-subdiv.-5-heading}{%
\subsection*{(720 ILCS 5/Art. 17, Subdiv. 5
heading)}\label{ilcs-5art.-17-subdiv.-5-heading}}
\addcontentsline{toc}{subsection}{(720 ILCS 5/Art. 17, Subdiv. 5
heading)}

SUBDIVISION 5.

DECEPTION

(Source: P.A. 96-1551, eff. 7-1-11.)

\hypertarget{ilcs-517-1-from-ch.-38-par.-17-1}{%
\subsection*{(720 ILCS 5/17-1) (from Ch. 38, par.
17-1)}\label{ilcs-517-1-from-ch.-38-par.-17-1}}
\addcontentsline{toc}{subsection}{(720 ILCS 5/17-1) (from Ch. 38, par.
17-1)}

\hypertarget{sec.-17-1.-deceptive-practices.}{%
\section*{Sec. 17-1. Deceptive
practices.}\label{sec.-17-1.-deceptive-practices.}}
\addcontentsline{toc}{section}{Sec. 17-1. Deceptive practices.}

\markright{Sec. 17-1. Deceptive practices.}

\begin{enumerate}
\def\labelenumi{(\Alph{enumi})}
\tightlist
\item
  General deception.

  A person commits a deceptive practice when, with intent to defraud,
  the person does any of the following:

  (1) He or she knowingly causes another, by deception or threat, to
  execute a document disposing of property or a document by which a
  pecuniary obligation is incurred.

  (2) Being an officer, manager or other person participating in the
  direction of a financial institution, he or she knowingly receives or
  permits the receipt of a deposit or other investment, knowing that the
  institution is insolvent.

  (3) He or she knowingly makes a false or deceptive statement addressed
  to the public for the purpose of promoting the sale of property or
  services.

  (B) Bad checks.

  A person commits a deceptive practice when:

  (1) With intent to obtain control over property or to pay for
  property, labor or services of another, or in satisfaction of an
  obligation for payment of tax under the Retailers' Occupation Tax Act
  or any other tax due to the State of Illinois, he or she issues or
  delivers a check or other order upon a real or fictitious depository
  for the payment of money, knowing that it will not be paid by the
  depository. The trier of fact may infer that the defendant knows that
  the check or other order will not be paid by the depository and that
  the defendant has acted with intent to defraud when the defendant
  fails to have sufficient funds or credit with the depository when the
  check or other order is issued or delivered, or when such check or
  other order is presented for payment and dishonored on each of 2
  occasions at least 7 days apart. In this paragraph (B)(1),
  ``property'' includes rental property (real or personal).

  (2) He or she issues or delivers a check or other order upon a real or
  fictitious depository in an amount exceeding \$150 in payment of an
  amount owed on any credit transaction for property, labor or services,
  or in payment of the entire amount owed on any credit transaction for
  property, labor or services, knowing that it will not be paid by the
  depository, and thereafter fails to provide funds or credit with the
  depository in the face amount of the check or order within 7 days of
  receiving actual notice from the depository or payee of the dishonor
  of the check or order.

  \begin{enumerate}
  \def\labelenumii{(\Alph{enumii})}
  \setcounter{enumii}{2}
  \tightlist
  \item
    Bank-related fraud.

    (1) False statement.

    A person commits false statement bank fraud if he or she, with
    intent to defraud, makes or causes to be made any false statement in
    writing in order to obtain an account with a bank or other financial
    institution, or to obtain credit from a bank or other financial
    institution, or to obtain services from a currency exchange, knowing
    such writing to be false, and with the intent that it be relied
    upon.

    For purposes of this subsection (C), a false statement means any
    false statement representing identity, address, or employment, or
    the identity, address, or employment of any person, firm, or
    corporation.

    (2) Possession of stolen or fraudulently obtained checks.

    A person commits possession of stolen or fraudulently obtained
    checks when he or she possesses, with the intent to obtain access to
    funds of another person held in a real or fictitious deposit account
    at a financial institution, makes a false statement or a
    misrepresentation to the financial institution, or possesses,
    transfers, negotiates, or presents for payment a check, draft, or
    other item purported to direct the financial institution to withdraw
    or pay funds out of the account holder's deposit account with
    knowledge that such possession, transfer, negotiation, or
    presentment is not authorized by the account holder or the issuing
    financial institution. A person shall be deemed to have been
    authorized to possess, transfer, negotiate, or present for payment
    such item if the person was otherwise entitled by law to withdraw or
    recover funds from the account in question and followed the
    requisite procedures under the law. If the account holder, upon
    discovery of the withdrawal or payment, claims that the withdrawal
    or payment was not authorized, the financial institution may require
    the account holder to submit an affidavit to that effect on a form
    satisfactory to the financial institution before the financial
    institution may be required to credit the account in an amount equal
    to the amount or amounts that were withdrawn or paid without
    authorization.

    (3) Possession of implements of check fraud.

    A person commits possession of implements of check fraud when he or
    she possesses, with the intent to defraud and without the authority
    of the account holder or financial institution, any check imprinter,
    signature imprinter, or ``certified'' stamp.

    (D) Sentence.

    (1) The commission of a deceptive practice in violation of this
    Section, except as otherwise provided by this subsection (D), is a
    Class A misdemeanor.

    (2) For purposes of paragraphs (A)(1) and (B)(1):

    (a) The commission of a deceptive practice in violation of paragraph
    (A)(1) or (B)(1), when the value of the property so obtained, in a
    single transaction or in separate transactions within a 90-day
    period, exceeds \$150, is a Class 4 felony. In the case of a
    prosecution for separate transactions totaling more than \$150
    within a 90-day period, those separate transactions shall be alleged
    in a single charge and prosecuted in a single prosecution.

    (b) The commission of a deceptive practice in violation of paragraph
    (B)(1) a second or subsequent time is a Class 4 felony.

    (3) For purposes of paragraph (C)(2), a person who, within any
    12-month period, violates paragraph (C)(2) with respect to 3 or more
    checks or orders for the payment of money at the same time or
    consecutively, each the property of a different account holder or
    financial institution, is guilty of a Class 4 felony.

    (4) For purposes of paragraph (C)(3), a person who within any
    12-month period violates paragraph (C)(3) as to possession of 3 or
    more such devices at the same time or consecutively is guilty of a
    Class 4 felony.

    (E) Civil liability. A person who issues a check or order to a payee
    in violation of paragraph (B)(1) and who fails to pay the amount of
    the check or order to the payee within 30 days following either
    delivery and acceptance by the addressee of a written demand both by
    certified mail and by first class mail to the person's last known
    address or attempted delivery of a written demand sent both by
    certified mail and by first class mail to the person's last known
    address and the demand by certified mail is returned to the sender
    with a notation that delivery was refused or unclaimed shall be
    liable to the payee or a person subrogated to the rights of the
    payee for, in addition to the amount owing upon such check or order,
    damages of treble the amount so owing, but in no case less than
    \$100 nor more than \$1,500, plus attorney's fees and court costs.
    An action under this subsection (E) may be brought in small claims
    court or in any other appropriate court. As part of the written
    demand required by this subsection (E), the plaintiff shall provide
    written notice to the defendant of the fact that prior to the
    hearing of any action under this subsection (E), the defendant may
    tender to the plaintiff and the plaintiff shall accept, as
    satisfaction of the claim, an amount of money equal to the sum of
    the amount of the check and the incurred court costs, including the
    cost of service of process, and attorney's fees.

    (Source: P.A. 96-1432, eff. 1-1-11; 96-1551, eff. 7-1-11 .)

    \hypertarget{ilcs-517-1a-from-ch.-38-par.-17-1a}{%
    \subsection*{(720 ILCS 5/17-1a) (from Ch. 38, par.
    17-1a)}\label{ilcs-517-1a-from-ch.-38-par.-17-1a}}
    \addcontentsline{toc}{subsection}{(720 ILCS 5/17-1a) (from Ch. 38,
    par. 17-1a)}

    \hypertarget{sec.-17-1a.-repealed.}{%
    \section*{Sec. 17-1a. (Repealed).}\label{sec.-17-1a.-repealed.}}
    \addcontentsline{toc}{section}{Sec. 17-1a. (Repealed).}

    \markright{Sec. 17-1a. (Repealed).}

    (Source: P.A. 90-721, eff. 1-1-99. Repealed by P.A. 96-1551, eff.
    7-1-11 .)

    \hypertarget{ilcs-517-1b}{%
    \subsection*{(720 ILCS 5/17-1b)}\label{ilcs-517-1b}}
    \addcontentsline{toc}{subsection}{(720 ILCS 5/17-1b)}

    \hypertarget{sec.-17-1b.-states-attorneys-bad-check-diversion-program.}{%
    \section*{Sec. 17-1b. State's Attorney's bad check diversion
    program.}\label{sec.-17-1b.-states-attorneys-bad-check-diversion-program.}}
    \addcontentsline{toc}{section}{Sec. 17-1b. State's Attorney's bad
    check diversion program.}

    \markright{Sec. 17-1b. State's Attorney's bad check diversion
    program.}

    (a) In this Section:

    ``Offender'' means a person charged with, or for whom probable cause
    exists to charge the person with, deceptive practices.

    ``Pretrial diversion'' means the decision of a prosecutor to refer
    an offender to a diversion program on condition that the criminal
    charges against the offender will be dismissed after a specified
    period of time, or the case will not be charged, if the offender
    successfully completes the program.

    ``Restitution'' means all amounts payable to a victim of deceptive
    practices under the bad check diversion program created under this
    Section, including the amount of the check and any transaction fees
    payable to a victim as set forth in subsection (g) but does not
    include amounts recoverable under Section 3-806 of the Uniform
    Commercial Code and subsection (E) of Section 17-1 of this Code.

    (b) A State's Attorney may create within his or her office a bad
    check diversion program for offenders who agree to voluntarily
    participate in the program instead of undergoing prosecution. The
    program may be conducted by the State's Attorney or by a private
    entity under contract with the State's Attorney. If the State's
    Attorney contracts with a private entity to perform any services in
    operating the program, the entity shall operate under the
    supervision, direction, and control of the State's Attorney. Any
    private entity providing services under this Section is not a
    ``collection agency'' as that term is defined under the Collection
    Agency Act.

    (c) If an offender is referred to the State's Attorney, the State's
    Attorney may determine whether the offender is appropriate for
    acceptance in the program. The State's Attorney may consider, but
    shall not be limited to consideration of, the following factors:

    (1) the amount of the check that was drawn or passed;

    (2) prior referrals of the offender to the program;

    (3) whether other charges of deceptive practices are pending against
    the offender;

    (4) the evidence presented to the State's Attorney regarding the
    facts and circumstances of the incident;

    (5) the offender's criminal history; and

    (6) the reason the check was dishonored by the financial
    institution.

    (d) The bad check diversion program may require an offender to do
    one or more of the following:

    (i) pay for, at his or her own expense, and successfully complete an
    educational class held by the State's Attorney or a private entity
    under contract with the State's Attorney;

    (ii) make full restitution for the offense;

    (iii) pay a per-check administrative fee as set forth in this
    Section.

    (e) If an offender is diverted to the program, the State's Attorney
    shall agree in writing not to prosecute the offender upon the
    offender's successful completion of the program conditions. The
    State's Attorney's agreement to divert the offender shall specify
    the offenses that will not be prosecuted by identifying the checks
    involved in the transactions.

    (f) The State's Attorney, or private entity under contract with the
    State's Attorney, may collect a fee from an offender diverted to the
    State's Attorney's bad check diversion program. This fee may be
    deposited in a bank account maintained by the State's Attorney for
    the purpose of depositing fees and paying the expenses of the
    program or for use in the enforcement and prosecution of criminal
    laws. The State's Attorney may require that the fee be paid directly
    to a private entity that administers the program under a contract
    with the State's Attorney. The amount of the administrative fees
    collected by the State's Attorney under the program may not exceed
    \$35 per check. The county board may, however, by ordinance,
    increase the fees allowed by this Section if the increase is
    justified by an acceptable cost study showing that the fees allowed
    by this Section are not sufficient to cover the cost of providing
    the service.

    (g)

    (1) The private entity shall be required to maintain adequate
    general liability insurance of \$1,000,000 per occurrence as well as
    adequate coverage for potential loss resulting from employee
    dishonesty. The State's Attorney may require a surety bond payable
    to the State's Attorney if in the State's Attorney's opinion it is
    determined that the private entity is not adequately insured or
    funded.

    (2)

    (A) Each private entity that has a contract with the State's
    Attorney to conduct a bad check diversion program shall at all times
    maintain a separate bank account in which all moneys received from
    the offenders participating in the program shall be deposited,
    referred to as a ``trust account'', except that negotiable
    instruments received may be forwarded directly to a victim of the
    deceptive practice committed by the offender if that procedure is
    provided for by a writing executed by the victim. Moneys received
    shall be so deposited within 5 business days after posting to the
    private entity's books of account. There shall be sufficient funds
    in the trust account at all times to pay the victims the amount due
    them.

    (B) The trust account shall be established in a financial
    institution which is federally or State insured or otherwise secured
    as defined by rule. If the account is interest bearing, the private
    entity shall pay to the victim interest earned on funds on deposit
    after the 60th day.

    (C) Each private entity shall keep on file the name of the financial
    institution in which each trust account is maintained, the name of
    each trust account, and the names of the persons authorized to
    withdraw funds from each account. The private entity, within 30 days
    of the time of a change of depository or person authorized to make
    withdrawal, shall update its files to reflect that change. An
    examination and audit of a private entity's trust accounts may be
    made by the State's Attorney as the State's Attorney deems
    appropriate. A trust account financial report shall be submitted
    annually on forms acceptable to the State's Attorney.

    (3) The State's Attorney may cancel a contract entered into with a
    private entity under this Section for any one or any combination of
    the following causes:

    (A) Conviction of the private entity or the principals of the
    private entity of any crime under the laws of any U.S. jurisdiction
    which is a felony, a misdemeanor an essential element of which is
    dishonesty, or of any crime which directly relates to the practice
    of the profession.

    (B) A determination that the private entity has engaged in conduct
    prohibited in item (4).

    (4) The State's Attorney may determine whether the private entity
    has engaged in the following prohibited conduct:

    (A) Using or threatening to use force or violence to cause physical
    harm to an offender, his or her family, or his or her property.

    (B) Threatening the seizure, attachment, or sale of an offender's
    property where such action can only be taken pursuant to court order
    without disclosing that prior court proceedings are required.

    (C) Disclosing or threatening to disclose information adversely
    affecting an offender's reputation for creditworthiness with
    knowledge the information is false.

    (D) Initiating or threatening to initiate communication with an
    offender's employer unless there has been a default of the payment
    of the obligation for at least 30 days and at least 5 days prior
    written notice, to the last known address of the offender, of the
    intention to communicate with the employer has been given to the
    employee, except as expressly permitted by law or court order.

    (E) Communicating with the offender or any member of the offender's
    family at such a time of day or night and with such frequency as to
    constitute harassment of the offender or any member of the
    offender's family. For purposes of this clause (E) the following
    conduct shall constitute harassment:

    (i) Communicating with the offender or any member of his or her
    family at any unusual time or place or a time or place known or
    which should be known to be inconvenient to the offender. In the
    absence of knowledge of circumstances to the contrary, a private
    entity shall assume that the convenient time for communicating with
    a consumer is after 8 o'clock a.m. and before 9 o'clock p.m. local
    time at the offender's residence.

    (ii) The threat of publication or publication of a list of offenders
    who allegedly refuse to pay restitution, except by the State's
    Attorney.

    (iii) The threat of advertisement or advertisement for sale of any
    restitution to coerce payment of the restitution.

    (iv) Causing a telephone to ring or engaging any person in telephone
    conversation repeatedly or continuously with intent to annoy, abuse,
    or harass any person at the called number.

    (v) Using profane, obscene or abusive language in communicating with
    an offender, his or her family, or others.

    (vi) Disclosing or threatening to disclose information relating to a
    offender's case to any other person except the victim and
    appropriate law enforcement personnel.

    (vii) Disclosing or threatening to disclose information concerning
    the alleged criminal act which the private entity knows to be
    reasonably disputed by the offender without disclosing the fact that
    the offender disputes the accusation.

    (viii) Engaging in any conduct which the

    State's Attorney finds was intended to cause and did cause mental or
    physical illness to the offender or his or her family.

    (ix) Attempting or threatening to enforce a right or remedy with
    knowledge or reason to know that the right or remedy does not exist.

    (x) Except as authorized by the State's

    Attorney, using any form of communication which simulates legal or
    judicial process or which gives the appearance of being authorized,
    issued or approved by a governmental agency or official or by an
    attorney at law when it is not.

    (xi) Using any badge, uniform, or other indicia of any governmental
    agency or official, except as authorized by law or by the State's
    Attorney.

    (xii) Except as authorized by the State's

    Attorney, conducting business under any name or in any manner which
    suggests or implies that the private entity is bonded if such
    private entity is or is a branch of or is affiliated with any
    governmental agency or court if such private entity is not.

    (xiii) Misrepresenting the amount of the restitution alleged to be
    owed.

    (xiv) Except as authorized by the State's

    Attorney, representing that an existing restitution amount may be
    increased by the addition of attorney's fees, investigation fees, or
    any other fees or charges when those fees or charges may not legally
    be added to the existing restitution.

    (xv) Except as authorized by the State's

    Attorney, representing that the private entity is an attorney at law
    or an agent for an attorney if the entity is not.

    (xvi) Collecting or attempting to collect any interest or other
    charge or fee in excess of the actual restitution or claim unless
    the interest or other charge or fee is expressly authorized by the
    State's Attorney, who shall determine what constitutes a reasonable
    collection fee.

    (xvii) Communicating or threatening to communicate with an offender
    when the private entity is informed in writing by an attorney that
    the attorney represents the offender concerning the claim, unless
    authorized by the attorney. If the attorney fails to respond within
    a reasonable period of time, the private entity may communicate with
    the offender. The private entity may communicate with the offender
    when the attorney gives his consent.

    (xviii) Engaging in dishonorable, unethical, or unprofessional
    conduct of a character likely to deceive, defraud, or harm the
    public.

    (5) The State's Attorney shall audit the accounts of the bad check
    diversion program after notice in writing to the private entity.

    (6) Any information obtained by a private entity that has a contract
    with the State's Attorney to conduct a bad check diversion program
    is confidential information between the State's Attorney and the
    private entity and may not be sold or used for any other purpose but
    may be shared with other authorized law enforcement agencies as
    determined by the State's Attorney.

    (h) The State's Attorney, or private entity under contract with the
    State's Attorney, shall recover, in addition to the face amount of
    the dishonored check or draft, a transaction fee to defray the costs
    and expenses incurred by a victim who received a dishonored check
    that was made or delivered by the offender. The face amount of the
    dishonored check or draft and the transaction fee shall be paid by
    the State's Attorney or private entity under contract with the
    State's Attorney to the victim as restitution for the offense. The
    amount of the transaction fee must not exceed: \$25 if the face
    amount of the check or draft does not exceed \$100; \$30 if the face
    amount of the check or draft is greater than \$100 but does not
    exceed \$250; \$35 if the face amount of the check or draft is
    greater than \$250 but does not exceed \$500; \$40 if the face
    amount of the check or draft is greater than \$500 but does not
    exceed \$1,000; and \$50 if the face amount of the check or draft is
    greater than \$1,000.

    (i) The offender, if aggrieved by an action of the private entity
    contracted to operate a bad check diversion program, may submit a
    grievance to the State's Attorney who may then resolve the
    grievance. The private entity must give notice to the offender that
    the grievance procedure is available. The grievance procedure shall
    be established by the State's Attorney.

    (Source: P.A. 95-41, eff. 1-1-08; 96-1551, eff. 7-1-11 .)

    \hypertarget{ilcs-517-2-from-ch.-38-par.-17-2}{%
    \subsection*{(720 ILCS 5/17-2) (from Ch. 38, par.
    17-2)}\label{ilcs-517-2-from-ch.-38-par.-17-2}}
    \addcontentsline{toc}{subsection}{(720 ILCS 5/17-2) (from Ch. 38,
    par. 17-2)}

    \hypertarget{sec.-17-2.-false-personation-solicitation.}{%
    \section*{Sec. 17-2. False personation;
    solicitation.}\label{sec.-17-2.-false-personation-solicitation.}}
    \addcontentsline{toc}{section}{Sec. 17-2. False personation;
    solicitation.}

    \markright{Sec. 17-2. False personation; solicitation.}

    (a) False personation; solicitation.

    (1) A person commits a false personation when he or she knowingly
    and falsely represents himself or herself to be a member or
    representative of any veterans' or public safety personnel
    organization or a representative of any charitable organization, or
    when he or she knowingly exhibits or uses in any manner any decal,
    badge or insignia of any charitable, public safety personnel, or
    veterans' organization when not authorized to do so by the
    charitable, public safety personnel, or veterans' organization.
    ``Public safety personnel organization'' has the meaning ascribed to
    that term in Section 1 of the Solicitation for Charity Act.

    (2) A person commits a false personation when he or she knowingly
    and falsely represents himself or herself to be a veteran in seeking
    employment or public office. In this paragraph, ``veteran'' means a
    person who has served in the Armed Services or Reserve Forces of the
    United States.

    (2.1) A person commits a false personation when he or she knowingly
    and falsely represents himself or herself to be:

    (A) an active-duty member of the Armed Services or Reserve Forces of
    the United States or the National Guard or a veteran of the Armed
    Services or Reserve Forces of the United States or the National
    Guard; and

    (B) obtains money, property, or another tangible benefit through
    that false representation.

    In this paragraph, ``member of the Armed Services or

    Reserve Forces of the United States'' means a member of the United
    States Navy, Army, Air Force, Marine Corps, or Coast Guard; and
    ``veteran'' means a person who has served in the Armed Services or
    Reserve Forces of the United States or the National Guard.

    (2.5) A person commits a false personation when he or she knowingly
    and falsely represents himself or herself to be:

    (A) another actual person and does an act in such assumed character
    with intent to intimidate, threaten, injure, defraud, or to obtain a
    benefit from another; or

    (B) a representative of an actual person or organization and does an
    act in such false capacity with intent to obtain a benefit or to
    injure or defraud another.

    (3) No person shall knowingly use the words ``Police'',

    ``Police Department'', ``Patrolman'', ``Sergeant'', ``Lieutenant'',
    ``Peace Officer'', ``Sheriff's Police'', ``Sheriff'', ``Officer'',
    ``Law Enforcement'', ``Trooper'', ``Deputy'', ``Deputy Sheriff'',
    ``State Police'', or any other words to the same effect (i) in the
    title of any organization, magazine, or other publication without
    the express approval of the named public safety personnel
    organization's governing board or (ii) in combination with the name
    of any state, state agency, public university, or unit of local
    government without the express written authorization of that state,
    state agency, public university, or unit of local government.

    (4) No person may knowingly claim or represent that he or she is
    acting on behalf of any public safety personnel organization when
    soliciting financial contributions or selling or delivering or
    offering to sell or deliver any merchandise, goods, services,
    memberships, or advertisements unless the chief of the police
    department, fire department, and the corporate or municipal
    authority thereof, or the sheriff has first entered into a written
    agreement with the person or with an organization with which the
    person is affiliated and the agreement permits the activity and
    specifies and states clearly and fully the purpose for which the
    proceeds of the solicitation, contribution, or sale will be used.

    (5) No person, when soliciting financial contributions or selling or
    delivering or offering to sell or deliver any merchandise, goods,
    services, memberships, or advertisements may claim or represent that
    he or she is representing or acting on behalf of any nongovernmental
    organization by any name which includes ``officer'', ``peace
    officer'', ``police'', ``law enforcement'', ``trooper'',
    ``sheriff'', ``deputy'', ``deputy sheriff'', ``State police'', or
    any other word or words which would reasonably be understood to
    imply that the organization is composed of law enforcement personnel
    unless:

    (A) the person is actually representing or acting on behalf of the
    nongovernmental organization;

    (B) the nongovernmental organization is controlled by and governed
    by a membership of and represents a group or association of active
    duty peace officers, retired peace officers, or injured peace
    officers; and

    (C) before commencing the solicitation or the sale or the offers to
    sell any merchandise, goods, services, memberships, or
    advertisements, a written contract between the soliciting or selling
    person and the nongovernmental organization, which specifies and
    states clearly and fully the purposes for which the proceeds of the
    solicitation, contribution, or sale will be used, has been entered
    into.

    (6) No person, when soliciting financial contributions or selling or
    delivering or offering to sell or deliver any merchandise, goods,
    services, memberships, or advertisements, may knowingly claim or
    represent that he or she is representing or acting on behalf of any
    nongovernmental organization by any name which includes the term
    ``fireman'', ``fire fighter'', ``paramedic'', or any other word or
    words which would reasonably be understood to imply that the
    organization is composed of fire fighter or paramedic personnel
    unless:

    (A) the person is actually representing or acting on behalf of the
    nongovernmental organization;

    (B) the nongovernmental organization is controlled by and governed
    by a membership of and represents a group or association of active
    duty, retired, or injured fire fighters (for the purposes of this
    Section, ``fire fighter'' has the meaning ascribed to that term in
    Section 2 of the Illinois Fire Protection Training Act) or active
    duty, retired, or injured emergency medical technicians - ambulance,
    emergency medical technicians - intermediate, emergency medical
    technicians - paramedic, ambulance drivers, or other medical
    assistance or first aid personnel; and

    (C) before commencing the solicitation or the sale or delivery or
    the offers to sell or deliver any merchandise, goods, services,
    memberships, or advertisements, the soliciting or selling person and
    the nongovernmental organization have entered into a written
    contract that specifies and states clearly and fully the purposes
    for which the proceeds of the solicitation, contribution, or sale
    will be used.

    (7) No person may knowingly claim or represent that he or she is an
    airman, airline employee, airport employee, or contractor at an
    airport in order to obtain the uniform, identification card,
    license, or other identification paraphernalia of an airman, airline
    employee, airport employee, or contractor at an airport.

    (8) No person, firm, copartnership, or corporation

    (except corporations organized and doing business under the Pawners
    Societies Act) shall knowingly use a name that contains in it the
    words ``Pawners' Society''.

    (b) False personation; public officials and employees. A person
    commits a false personation if he or she knowingly and falsely
    represents himself or herself to be any of the following:

    (1) An attorney authorized to practice law for purposes of
    compensation or consideration. This paragraph (b)(1) does not apply
    to a person who unintentionally fails to pay attorney registration
    fees established by Supreme Court Rule.

    (2) A public officer or a public employee or an official or employee
    of the federal government.

    (2.3) A public officer, a public employee, or an official or
    employee of the federal government, and the false representation is
    made in furtherance of the commission of felony.

    (2.7) A public officer or a public employee, and the false
    representation is for the purpose of effectuating identity theft as
    defined in Section 16-30 of this Code.

    (3) A peace officer.

    (4) A peace officer while carrying a deadly weapon.

    (5) A peace officer in attempting or committing a felony.

    (6) A peace officer in attempting or committing a forcible felony.

    (7) The parent, legal guardian, or other relation of a minor child
    to any public official, public employee, or elementary or secondary
    school employee or administrator.

    (7.5) The legal guardian, including any representative of a State or
    public guardian, of a person with a disability appointed under
    Article XIa of the Probate Act of 1975.

    (8) A fire fighter.

    (9) A fire fighter while carrying a deadly weapon.

    (10) A fire fighter in attempting or committing a felony.

    (11) An emergency management worker of any jurisdiction in this
    State.

    (12) An emergency management worker of any jurisdiction in this
    State in attempting or committing a felony. For the purposes of this
    subsection (b), ``emergency management worker'' has the meaning
    provided under Section 2-6.6 of this Code.

    (b-5) The trier of fact may infer that a person falsely represents
    himself or herself to be a public officer or a public employee or an
    official or employee of the federal government if the person:

    (1) wears or displays without authority any uniform, badge,
    insignia, or facsimile thereof by which a public officer or public
    employee or official or employee of the federal government is
    lawfully distinguished; or

    (2) falsely expresses by word or action that he or she is a public
    officer or public employee or official or employee of the federal
    government and is acting with approval or authority of a public
    agency or department.

    (c) Fraudulent advertisement of a corporate name.

    (1) A company, association, or individual commits fraudulent
    advertisement of a corporate name if he, she, or it, not being
    incorporated, puts forth a sign or advertisement and assumes, for
    the purpose of soliciting business, a corporate name.

    (2) Nothing contained in this subsection (c) prohibits a
    corporation, company, association, or person from using a divisional
    designation or trade name in conjunction with its corporate name or
    assumed name under Section 4.05 of the Business Corporation Act of
    1983 or, if it is a member of a partnership or joint venture, from
    doing partnership or joint venture business under the partnership or
    joint venture name. The name under which the joint venture or
    partnership does business may differ from the names of the members.
    Business may not be conducted or transacted under that joint venture
    or partnership name, however, unless all provisions of the Assumed
    Business Name Act have been complied with. Nothing in this
    subsection (c) permits a foreign corporation to do business in this
    State without complying with all Illinois laws regulating the doing
    of business by foreign corporations. No foreign corporation may
    conduct or transact business in this State as a member of a
    partnership or joint venture that violates any Illinois law
    regulating or pertaining to the doing of business by foreign
    corporations in Illinois.

    (3) The provisions of this subsection (c) do not apply to limited
    partnerships formed under the Revised Uniform Limited Partnership
    Act or under the Uniform Limited Partnership Act (2001).

    (d) False law enforcement badges.

    (1) A person commits false law enforcement badges if he or she
    knowingly produces, sells, or distributes a law enforcement badge
    without the express written consent of the law enforcement agency
    represented on the badge or, in case of a reorganized or defunct law
    enforcement agency, its successor law enforcement agency.

    (2) It is a defense to false law enforcement badges that the law
    enforcement badge is used or is intended to be used exclusively: (i)
    as a memento or in a collection or exhibit; (ii) for decorative
    purposes; or (iii) for a dramatic presentation, such as a
    theatrical, film, or television production.

    (e) False medals.

    (1) A person commits a false personation if he or she knowingly and
    falsely represents himself or herself to be a recipient of, or wears
    on his or her person, any of the following medals if that medal was
    not awarded to that person by the United States Government,
    irrespective of branch of service: The Congressional Medal of Honor,
    The Distinguished Service Cross, The Navy Cross, The Air Force
    Cross, The Silver Star, The Bronze Star, or the Purple Heart.

    (2) It is a defense to a prosecution under paragraph

    (e)(1) that the medal is used, or is intended to be used,
    exclusively:

    (A) for a dramatic presentation, such as a theatrical, film, or
    television production, or a historical re-enactment; or

    (B) for a costume worn, or intended to be worn, by a person under 18
    years of age.

    (f) Sentence.

    (1) A violation of paragraph (a)(8) is a petty offense subject to a
    fine of not less than \$5 nor more than \$100, and the person, firm,
    copartnership, or corporation commits an additional petty offense
    for each day he, she, or it continues to commit the violation. A
    violation of paragraph (c)(1) is a petty offense, and the company,
    association, or person commits an additional petty offense for each
    day he, she, or it continues to commit the violation. A violation of
    paragraph (a)(2.1) or subsection (e) is a petty offense for which
    the offender shall be fined at least \$100 and not more than \$200.

    (2) A violation of paragraph (a)(1), (a)(3), or

    (b)(7.5) is a Class C misdemeanor.

    (3) A violation of paragraph (a)(2), (a)(2.5),

    (a)(7), (b)(2), or (b)(7) or subsection (d) is a Class A
    misdemeanor. A second or subsequent violation of subsection (d) is a
    Class 3 felony.

    (4) A violation of paragraph (a)(4), (a)(5), (a)(6),

    (b)(1), (b)(2.3), (b)(2.7), (b)(3), (b)(8), or (b)(11) is a Class 4
    felony.

    (5) A violation of paragraph (b)(4), (b)(9), or

    (b)(12) is a Class 3 felony.

    (6) A violation of paragraph (b)(5) or (b)(10) is a Class 2 felony.

    (7) A violation of paragraph (b)(6) is a Class 1 felony.

    (g) A violation of subsection (a)(1) through (a)(7) or subsection
    (e) of this Section may be accomplished in person or by any means of
    communication, including but not limited to the use of an Internet
    website or any form of electronic communication.

    (Source: P.A. 99-143, eff. 7-27-15; 99-561, eff. 7-15-16; 100-201,
    eff. 8-18-17.)

    \hypertarget{ilcs-517-2.5}{%
    \subsection*{(720 ILCS 5/17-2.5)}\label{ilcs-517-2.5}}
    \addcontentsline{toc}{subsection}{(720 ILCS 5/17-2.5)}

    \hypertarget{sec.-17-2.5.-repealed.}{%
    \section*{Sec. 17-2.5. (Repealed).}\label{sec.-17-2.5.-repealed.}}
    \addcontentsline{toc}{section}{Sec. 17-2.5. (Repealed).}

    \markright{Sec. 17-2.5. (Repealed).}

    (Source: P.A. 93-239, eff. 7-22-03. Repealed by P.A. 96-1551, eff.
    7-1-11 .)

    \hypertarget{ilcs-517-3-from-ch.-38-par.-17-3}{%
    \subsection*{(720 ILCS 5/17-3) (from Ch. 38, par.
    17-3)}\label{ilcs-517-3-from-ch.-38-par.-17-3}}
    \addcontentsline{toc}{subsection}{(720 ILCS 5/17-3) (from Ch. 38,
    par. 17-3)}

    \hypertarget{sec.-17-3.-forgery.}{%
    \section*{Sec. 17-3. Forgery.}\label{sec.-17-3.-forgery.}}
    \addcontentsline{toc}{section}{Sec. 17-3. Forgery.}

    \markright{Sec. 17-3. Forgery.}

    (a) A person commits forgery when, with intent to defraud, he or she
    knowingly:

    (1) makes a false document or alters any document to make it false
    and that document is apparently capable of defrauding another; or

    (2) issues or delivers such document knowing it to have been thus
    made or altered; or

    (3) possesses, with intent to issue or deliver, any such document
    knowing it to have been thus made or altered; or

    (4) unlawfully uses the digital signature, as defined in the
    Financial Institutions Electronic Documents and Digital Signature
    Act, of another; or

    (5) unlawfully creates an electronic signature of another person, as
    that term is defined in the Uniform Electronic Transactions Act.

    (b) (Blank).

    (c) A document apparently capable of defrauding another includes,
    but is not limited to, one by which any right, obligation or power
    with reference to any person or property may be created,
    transferred, altered or terminated. A document includes any record
    or electronic record as those terms are defined in the Electronic
    Commerce Security Act. For purposes of this Section, a document also
    includes a Universal Price Code Label or coin.

    (c-5) For purposes of this Section, ``false document'' or ``document
    that is false'' includes, but is not limited to, a document whose
    contents are false in some material way, or that purports to have
    been made by another or at another time, or with different
    provisions, or by authority of one who did not give such authority.

    (d) Sentence.

    (1) Except as provided in paragraphs (2) and (3), forgery is a Class
    3 felony.

    (2) Forgery is a Class 4 felony when only one

    Universal Price Code Label is forged.

    (3) Forgery is a Class A misdemeanor when an academic degree or coin
    is forged.

    (e) It is not a violation of this Section if a false academic degree
    explicitly states ``for novelty purposes only''.

    (Source: P.A. 102-38, eff. 6-25-21.)

    \hypertarget{ilcs-517-3.5}{%
    \subsection*{(720 ILCS 5/17-3.5)}\label{ilcs-517-3.5}}
    \addcontentsline{toc}{subsection}{(720 ILCS 5/17-3.5)}

    \hypertarget{sec.-17-3.5.-deceptive-sale-of-gold-or-silver.}{%
    \section*{Sec. 17-3.5. Deceptive sale of gold or
    silver.}\label{sec.-17-3.5.-deceptive-sale-of-gold-or-silver.}}
    \addcontentsline{toc}{section}{Sec. 17-3.5. Deceptive sale of gold
    or silver.}

    \markright{Sec. 17-3.5. Deceptive sale of gold or silver.}

    (a) Whoever makes for sale, or sells, or offers to sell or dispose
    of, or has in his or her possession with intent to sell or dispose
    of, any article or articles construed in whole or in part, of gold
    or any alloy or imitation thereof, having thereon or on any box,
    package, cover, wrapper or other thing enclosing or encasing such
    article or articles for sale, any stamp, brand, engraving, printed
    label, trade mark, imprint or other mark, indicating or designed, or
    intended to indicate, that the gold, alloy or imitation thereof, in
    such article or articles, is different from or better than the
    actual kind and quality of such gold, alloy or imitation, shall be
    guilty of a petty offense and shall be fined in any sum not less
    than \$50 nor more than \$100.

    (b) Whoever makes for sale, sells or offers to sell or dispose of or
    has in his or her possession, with intent to sell or dispose of, any
    article or articles constructed in whole or in part of silver or any
    alloy or imitation thereof, having thereon--or on any box, package,
    cover, wrapper or other thing enclosing or encasing such article or
    articles for sale--any stamp, brand, engraving, printed label,
    trademark, imprint or other mark, containing the words ``sterling''
    or ``sterling silver,'' referring, or designed or intended to refer,
    to the silver, alloy or imitation thereof in such article or
    articles, when such silver, alloy or imitation thereof shall contain
    less than nine hundred and twenty-five one-thousandths thereof of
    pure silver, shall be guilty of a petty offense and shall be fined
    in any sum not less than \$50 nor more than \$100.

    (c) Whoever makes for sale, sells or offers to sell or dispose of or
    has in his or her possession, with intent to sell or dispose of, any
    article or articles constructed in whole or in part of silver or any
    alloy or imitation thereof, having thereon--or on any box, package,
    cover, wrapper or other thing enclosing or encasing such article or
    articles for sale--any stamp, brand, engraving, printed label,
    trademark, imprint, or other mark, containing the words ``coin'' or
    ``coin silver,'' referring to or designed or intended to refer to,
    the silver, alloy or imitation thereof, in such article or articles,
    when such silver, alloy or imitation shall contain less than
    nine-tenths thereof pure silver, shall be guilty of a petty offense
    and shall be fined in any sum not less than \$50 and not more than
    \$100.

    (Source: P.A. 96-1551, eff. 7-1-11 .)

    \hypertarget{ilcs-517-4-from-ch.-38-par.-17-4}{%
    \subsection*{(720 ILCS 5/17-4) (from Ch. 38, par.
    17-4)}\label{ilcs-517-4-from-ch.-38-par.-17-4}}
    \addcontentsline{toc}{subsection}{(720 ILCS 5/17-4) (from Ch. 38,
    par. 17-4)}

    \hypertarget{sec.-17-4.-repealed.}{%
    \section*{Sec. 17-4. (Repealed).}\label{sec.-17-4.-repealed.}}
    \addcontentsline{toc}{section}{Sec. 17-4. (Repealed).}

    \markright{Sec. 17-4. (Repealed).}

    (Source: P.A. 77-2638. Repealed by P.A. 96-1551, eff. 7-1-11 .)

    \hypertarget{ilcs-517-5-from-ch.-38-par.-17-5}{%
    \subsection*{(720 ILCS 5/17-5) (from Ch. 38, par.
    17-5)}\label{ilcs-517-5-from-ch.-38-par.-17-5}}
    \addcontentsline{toc}{subsection}{(720 ILCS 5/17-5) (from Ch. 38,
    par. 17-5)}

    \hypertarget{sec.-17-5.-deceptive-collection-practices.}{%
    \section*{Sec. 17-5. Deceptive collection
    practices.}\label{sec.-17-5.-deceptive-collection-practices.}}
    \addcontentsline{toc}{section}{Sec. 17-5. Deceptive collection
    practices.}

    \markright{Sec. 17-5. Deceptive collection practices.}

    A collection agency as defined in the Collection Agency Act or any
    employee of such collection agency commits a deceptive collection
    practice when, with the intent to collect a debt owed to an
    individual or a corporation or other entity, he, she, or it does any
    of the following:

    (a) Represents falsely that he or she is an attorney, a policeman, a
    sheriff or deputy sheriff, a bailiff, a county clerk or employee of
    a county clerk's office, or any other person who by statute is
    authorized to enforce the law or any order of a court.

    (b) While attempting to collect an alleged debt, misrepresents to
    the alleged debtor or to his or her immediate family the corporate,
    partnership or proprietary name or other trade or business name
    under which the debt collector is engaging in debt collections and
    which he, she, or it is legally authorized to use.

    (c) While attempting to collect an alleged debt, adds to the debt
    any service charge, interest or penalty which he, she, or it is not
    entitled by law to add.

    (d) Threatens to ruin, destroy, or otherwise adversely affect an
    alleged debtor's credit rating unless, at the same time, a
    disclosure is made in accordance with federal law that the alleged
    debtor has a right to inspect his or her credit rating.

    (e) Accepts from an alleged debtor a payment which he, she, or it
    knows is not owed.

    Sentence. The commission of a deceptive collection practice is a
    Business Offense punishable by a fine not to exceed \$3,000.

    (Source: P.A. 96-1551, eff. 7-1-11 .)

    \hypertarget{ilcs-517-5.5}{%
    \subsection*{(720 ILCS 5/17-5.5)}\label{ilcs-517-5.5}}
    \addcontentsline{toc}{subsection}{(720 ILCS 5/17-5.5)}

    \hypertarget{sec.-17-5.5.-unlawful-attempt-to-collect-compensated-debt-against-a-crime-victim.}{%
    \section*{Sec. 17-5.5. Unlawful attempt to collect compensated debt
    against a crime
    victim.}\label{sec.-17-5.5.-unlawful-attempt-to-collect-compensated-debt-against-a-crime-victim.}}
    \addcontentsline{toc}{section}{Sec. 17-5.5. Unlawful attempt to
    collect compensated debt against a crime victim.}

    \markright{Sec. 17-5.5. Unlawful attempt to collect compensated debt
    against a crime victim.}

    (a) A person or a vendor commits unlawful attempt to collect a
    compensated debt against a crime victim when, with intent to collect
    funds for a debt incurred by or on behalf of a crime victim, which
    debt has been approved for payment by the Court of Claims under the
    Crime Victims Compensation Act, but the funds are involuntarily
    withheld from the person or vendor by the Comptroller by virtue of
    an outstanding obligation owed by the person or vendor to the State
    under the Uncollected State Claims Act, the person or vendor:

    (1) communicates with, harasses, or intimidates the crime victim for
    payment;

    (2) contacts or distributes information to affect the compensated
    crime victim's credit rating as a result of the compensated debt; or

    (3) takes any other action adverse to the crime victim or his or her
    family on account of the compensated debt.

    (b) Sentence. Unlawful attempt to collect a compensated debt against
    a crime victim is a Class A misdemeanor.

    (c) Nothing in this Code prevents the attempt to collect an
    uncompensated debt or an uncompensated portion of a compensated debt
    incurred by or on behalf of a crime victim and not covered under the
    Crime Victims Compensation Act.

    (d) As used in this Section, ``crime victim'' means a victim of a
    violent crime or applicant as defined in the Crime Victims
    Compensation Act. ``Compensated debt'' means a debt incurred by or
    on behalf of a crime victim and approved for payment by the Court of
    Claims under the Crime Victims Compensation Act.

    (Source: P.A. 96-1551, eff. 7-1-11 .)

    \hypertarget{ilcs-517-5.7}{%
    \subsection*{(720 ILCS 5/17-5.7)}\label{ilcs-517-5.7}}
    \addcontentsline{toc}{subsection}{(720 ILCS 5/17-5.7)}

    \hypertarget{sec.-17-5.7.-deceptive-advertising.}{%
    \section*{Sec. 17-5.7. Deceptive
    advertising.}\label{sec.-17-5.7.-deceptive-advertising.}}
    \addcontentsline{toc}{section}{Sec. 17-5.7. Deceptive advertising.}

    \markright{Sec. 17-5.7. Deceptive advertising.}

    (a) Any person, firm, corporation or association or agent or
    employee thereof, who, with intent to sell, purchase, or in any wise
    dispose of, or to contract with reference to merchandise,
    securities, real estate, service, employment, money, credit or
    anything offered by such person, firm, corporation or association,
    or agent or employee thereof, directly or indirectly, to the public
    for sale, purchase, loan, distribution, or the hire of personal
    services, or with intent to increase the consumption of or to
    contract with reference to any merchandise, real estate, securities,
    money, credit, loan, service or employment, or to induce the public
    in any manner to enter into any obligation relating thereto, or to
    acquire title thereto, or an interest therein, or to make any loan,
    makes, publishes, disseminates, circulates, or places before the
    public, or causes, directly or indirectly, to be made, published,
    disseminated, circulated, or placed before the public, in this
    State, in a newspaper, magazine, or other publication, or in the
    form of a book, notice, handbill, poster, sign, bill, circular,
    pamphlet, letter, placard, card, label, or over any radio or
    television station, or in any other way similar or dissimilar to the
    foregoing, an advertisement, announcement, or statement of any sort
    regarding merchandise, securities, real estate, money, credit,
    service, employment, or anything so offered for use, purchase, loan
    or sale, or the interest, terms or conditions upon which such loan
    will be made to the public, which advertisement contains any
    assertion, representation or statement of fact which is untrue,
    misleading or deceptive, shall be guilty of a Class A misdemeanor.

    (b) Any person, firm or corporation offering for sale merchandise,
    commodities or service by making, publishing, disseminating,
    circulating or placing before the public within this State in any
    manner an advertisement of merchandise, commodities, or service,
    with the intent, design or purpose not to sell the merchandise,
    commodities, or service so advertised at the price stated therein,
    or otherwise communicated, or with intent not to sell the
    merchandise, commodities, or service so advertised, may be enjoined
    from such advertising upon application for injunctive relief by the
    State's Attorney or Attorney General, and shall also be guilty of a
    Class A misdemeanor.

    (c) Any person, firm or corporation who makes, publishes,
    disseminates, circulates or places before the public, or causes,
    directly or indirectly to be made, published, disseminated,
    circulated or placed before the public, in this State, in a
    newspaper, magazine or other publication published in this State, or
    in the form of a book, notice, handbill, poster, sign, bill,
    circular, pamphlet, letter, placard, card, or label distributed in
    this State, or over any radio or television station located in this
    State or in any other way in this State similar or dissimilar to the
    foregoing, an advertisement, announcement, statement or
    representation of any kind to the public relating to the sale,
    offering for sale, purchase, use or lease of any real estate in a
    subdivision located outside the State of Illinois may be enjoined
    from such activity upon application for injunctive relief by the
    State's Attorney or Attorney General and shall also be guilty of a
    Class A misdemeanor unless such advertisement, announcement,
    statement or representation contains or is accompanied by a clear,
    concise statement of the proximity of such real estate in common
    units of measurement to public schools, public highways, fresh water
    supply, public sewers, electric power, stores and shops, and
    telephone service or contains a statement that one or more of such
    facilities are not readily available, and name those not available.

    (d) Subsections (a), (b), and (c) do not apply to any medium for the
    printing, publishing, or disseminating of advertising, or any owner,
    agent or employee thereof, nor to any advertising agency or owner,
    agent or employee thereof, nor to any radio or television station,
    or owner, agent, or employee thereof, for printing, publishing, or
    disseminating, or causing to be printed, published, or disseminated,
    such advertisement in good faith and without knowledge of the
    deceptive character thereof.

    (e) No person, firm or corporation owning or operating a service
    station shall advertise or hold out or state to the public the per
    gallon price of gasoline, upon any sign on the premises of such
    station, unless such price includes all taxes, and unless the price,
    as so advertised, corresponds with the price appearing on the pump
    from which such gasoline is dispensed. Also, the identity of the
    product must be included with the price in any such advertisement,
    holding out or statement to the public. Any person who violates this
    subsection (e) shall be guilty of a petty offense.

    (Source: P.A. 96-1551, eff. 7-1-11 .)

    \hypertarget{ilcs-5art.-17-subdiv.-10-heading}{%
    \subsection*{(720 ILCS 5/Art. 17, Subdiv. 10
    heading)}\label{ilcs-5art.-17-subdiv.-10-heading}}
    \addcontentsline{toc}{subsection}{(720 ILCS 5/Art. 17, Subdiv. 10
    heading)}

    SUBDIVISION 10.

    FRAUD ON A GOVERNMENTAL ENTITY

    (Source: P.A. 96-1551, eff. 7-1-11.)

    \hypertarget{ilcs-517-6-from-ch.-38-par.-17-6}{%
    \subsection*{(720 ILCS 5/17-6) (from Ch. 38, par.
    17-6)}\label{ilcs-517-6-from-ch.-38-par.-17-6}}
    \addcontentsline{toc}{subsection}{(720 ILCS 5/17-6) (from Ch. 38,
    par. 17-6)}

    \hypertarget{sec.-17-6.-state-benefits-fraud.}{%
    \section*{Sec. 17-6. State benefits
    fraud.}\label{sec.-17-6.-state-benefits-fraud.}}
    \addcontentsline{toc}{section}{Sec. 17-6. State benefits fraud.}

    \markright{Sec. 17-6. State benefits fraud.}

    (a) A person commits State benefits fraud when he or she obtains or
    attempts to obtain money or benefits from the State of Illinois,
    from any political subdivision thereof, or from any program funded
    or administered in whole or in part by the State of Illinois or any
    political subdivision thereof through the knowing use of false
    identification documents or through the knowing misrepresentation of
    his or her age, place of residence, number of dependents, marital or
    family status, employment status, financial status, or any other
    material fact upon which his eligibility for or degree of
    participation in any benefit program might be based.

    (b) Notwithstanding any provision of State law to the contrary,
    every application or other document submitted to an agency or
    department of the State of Illinois or any political subdivision
    thereof to establish or determine eligibility for money or benefits
    from the State of Illinois or from any political subdivision
    thereof, or from any program funded or administered in whole or in
    part by the State of Illinois or any political subdivision thereof,
    shall be made available upon request to any law enforcement agency
    for use in the investigation or prosecution of State benefits fraud
    or for use in the investigation or prosecution of any other crime
    arising out of the same transaction or occurrence. Except as
    otherwise permitted by law, information disclosed pursuant to this
    subsection shall be used and disclosed only for the purposes
    provided herein. The provisions of this Section shall be operative
    only to the extent that they do not conflict with any federal law or
    regulation governing federal grants to this State.

    (c) Any employee of the State of Illinois or any agency or political
    subdivision thereof may seize as evidence any false or fraudulent
    document presented to him or her in connection with an application
    for or receipt of money or benefits from the State of Illinois, from
    any political subdivision thereof, or from any program funded or
    administered in whole or in part by the State of Illinois or any
    political subdivision thereof.

    (d) Sentence.

    (1) State benefits fraud is a Class 4 felony except when more than
    \$300 is obtained, in which case State benefits fraud is a Class 3
    felony.

    (2) If a person knowingly misrepresents oneself as a veteran or as a
    dependent of a veteran with the intent of obtaining benefits or
    privileges provided by the State or its political subdivisions to
    veterans or their dependents, then State benefits fraud is a Class 3
    felony when \$300 or less is obtained and a Class 2 felony when more
    than \$300 is obtained. For the purposes of this paragraph (2),
    benefits and privileges include, but are not limited to, those
    benefits and privileges available under the Veterans' Employment
    Act, the Viet Nam Veterans Compensation Act, the Prisoner of War
    Bonus Act, the War Bonus Extension Act, the Military Veterans
    Assistance Act, the Veterans' Employment Representative Act, the
    Veterans Preference Act, Service Member Employment and Reemployment
    Rights Act, the Service Member's Tenure Act, the Housing for
    Veterans with Disabilities Act, the Under Age Veterans Benefits Act,
    the Survivors Compensation Act, the Children of Deceased Veterans
    Act, the Veterans Burial Places Act, the Higher Education Student
    Assistance Act, or any other loans, assistance in employment,
    monetary payments, or tax exemptions offered by the State or its
    political subdivisions for veterans or their dependents.

    (Source: P.A. 99-143, eff. 7-27-15; 100-1101, eff. 1-1-19 .)

    \hypertarget{ilcs-517-6.3}{%
    \subsection*{(720 ILCS 5/17-6.3)}\label{ilcs-517-6.3}}
    \addcontentsline{toc}{subsection}{(720 ILCS 5/17-6.3)}

    \hypertarget{sec.-17-6.3.-wic-fraud.}{%
    \section*{Sec. 17-6.3. WIC fraud.}\label{sec.-17-6.3.-wic-fraud.}}
    \addcontentsline{toc}{section}{Sec. 17-6.3. WIC fraud.}

    \markright{Sec. 17-6.3. WIC fraud.}

    (a) For the purposes of this Section, the Special Supplemental Food
    Program for Women, Infants and Children administered by the Illinois
    Department of Public Health or Department of Human Services shall be
    referred to as ``WIC''.

    (b) A person commits WIC fraud if he or she knowingly (i) uses,
    acquires, possesses, or transfers WIC Food Instruments or
    authorizations to participate in WIC in any manner not authorized by
    law or the rules of the Illinois Department of Public Health or
    Department of Human Services or (ii) uses, acquires, possesses, or
    transfers altered WIC Food Instruments or authorizations to
    participate in WIC.

    (c) Administrative malfeasance.

    (1) A person commits administrative malfeasance if he or she
    knowingly or recklessly misappropriates, misuses, or unlawfully
    withholds or converts to his or her own use or to the use of another
    any public funds made available for WIC.

    (2) An official or employee of the State or a unit of local
    government who knowingly aids, abets, assists, or participates in a
    known violation of this Section is subject to disciplinary
    proceedings under the rules of the applicable State agency or unit
    of local government.

    (d) Unauthorized possession of identification document. A person
    commits unauthorized possession of an identification document if he
    or she knowingly possesses, with intent to commit a misdemeanor or
    felony, another person's identification document issued by the
    Illinois Department of Public Health or Department of Human
    Services. For purposes of this Section, ``identification document''
    includes, but is not limited to, an authorization to participate in
    WIC or a card or other document that identifies a person as being
    entitled to WIC benefits.

    (e) Penalties.

    (1) If an individual, firm, corporation, association, agency,
    institution, or other legal entity is found by a court to have
    engaged in an act, practice, or course of conduct declared unlawful
    under subsection (a), (b), or (c) of this Section and:

    (A) the total amount of money involved in the violation, including
    the monetary value of the WIC Food Instruments and the value of
    commodities, is less than \$150, the violation is a Class A
    misdemeanor; a second or subsequent violation is a Class 4 felony;

    (B) the total amount of money involved in the violation, including
    the monetary value of the WIC Food Instruments and the value of
    commodities, is \$150 or more but less than \$1,000, the violation
    is a Class 4 felony; a second or subsequent violation is a Class 3
    felony;

    (C) the total amount of money involved in the violation, including
    the monetary value of the WIC Food Instruments and the value of
    commodities, is \$1,000 or more but less than \$5,000, the violation
    is a Class 3 felony; a second or subsequent violation is a Class 2
    felony;

    (D) the total amount of money involved in the violation, including
    the monetary value of the WIC Food Instruments and the value of
    commodities, is \$5,000 or more but less than \$10,000, the
    violation is a Class 2 felony; a second or subsequent violation is a
    Class 1 felony; or

    (E) the total amount of money involved in the violation, including
    the monetary value of the WIC Food Instruments and the value of
    commodities, is \$10,000 or more, the violation is a Class 1 felony
    and the defendant shall be permanently ineligible to participate in
    WIC.

    (2) A violation of subsection (d) is a Class 4 felony.

    (3) The State's Attorney of the county in which the violation of
    this Section occurred or the Attorney General shall bring actions
    arising under this Section in the name of the People of the State of
    Illinois.

    (4) For purposes of determining the classification of an offense
    under this subsection (e), all of the money received as a result of
    the unlawful act, practice, or course of conduct, including the
    value of any WIC Food Instruments and the value of commodities,
    shall be aggregated.

    (f) Seizure and forfeiture of property.

    (1) A person who commits a felony violation of this

    Section is subject to the property forfeiture provisions set forth
    in Article 124B of the Code of Criminal Procedure of 1963.

    (2) Property subject to forfeiture under this subsection (f) may be
    seized by the Director of the Illinois State Police or any local law
    enforcement agency upon process or seizure warrant issued by any
    court having jurisdiction over the property. The Director or a local
    law enforcement agency may seize property under this subsection (f)
    without process under any of the following circumstances:

    (A) If the seizure is incident to inspection under an administrative
    inspection warrant.

    (B) If the property subject to seizure has been the subject of a
    prior judgment in favor of the State in a criminal proceeding or in
    an injunction or forfeiture proceeding under Article 124B of the
    Code of Criminal Procedure of 1963.

    (C) If there is probable cause to believe that the property is
    directly or indirectly dangerous to health or safety.

    (D) If there is probable cause to believe that the property is
    subject to forfeiture under this subsection (f) and Article 124B of
    the Code of Criminal Procedure of 1963 and the property is seized
    under circumstances in which a warrantless seizure or arrest would
    be reasonable.

    (E) In accordance with the Code of Criminal

    Procedure of 1963.

    (g) Future participation as WIC vendor. A person who has been
    convicted of a felony violation of this Section is prohibited from
    participating as a WIC vendor for a minimum period of 3 years
    following conviction and until the total amount of money involved in
    the violation, including the value of WIC Food Instruments and the
    value of commodities, is repaid to WIC. This prohibition shall
    extend to any person with management responsibility in a firm,
    corporation, association, agency, institution, or other legal entity
    that has been convicted of a violation of this Section and to an
    officer or person owning, directly or indirectly, 5\% or more of the
    shares of stock or other evidences of ownership in a corporate
    vendor.

    (Source: P.A. 102-538, eff. 8-20-21.)

    \hypertarget{ilcs-517-6.5}{%
    \subsection*{(720 ILCS 5/17-6.5)}\label{ilcs-517-6.5}}
    \addcontentsline{toc}{subsection}{(720 ILCS 5/17-6.5)}

    \hypertarget{sec.-17-6.5.-persons-under-deportation-order-ineligibility-for-benefits.}{%
    \section*{Sec. 17-6.5. Persons under deportation order;
    ineligibility for
    benefits.}\label{sec.-17-6.5.-persons-under-deportation-order-ineligibility-for-benefits.}}
    \addcontentsline{toc}{section}{Sec. 17-6.5. Persons under
    deportation order; ineligibility for benefits.}

    \markright{Sec. 17-6.5. Persons under deportation order;
    ineligibility for benefits.}

    (a) An individual against whom a United States Immigration Judge has
    issued an order of deportation which has been affirmed by the Board
    of Immigration Review, as well as an individual who appeals such an
    order pending appeal, under paragraph 19 of Section 241(a) of the
    Immigration and Nationality Act relating to persecution of others on
    account of race, religion, national origin or political opinion
    under the direction of or in association with the Nazi government of
    Germany or its allies, shall be ineligible for the following
    benefits authorized by State law:

    (1) The homestead exemptions and homestead improvement exemption
    under Sections 15-170, 15-175, 15-176, and 15-180 of the Property
    Tax Code.

    (2) Grants under the Senior Citizens and Persons with

    Disabilities Property Tax Relief Act.

    (3) The double income tax exemption conferred upon persons 65 years
    of age or older by Section 204 of the Illinois Income Tax Act.

    (4) Grants provided by the Department on Aging.

    (5) Reductions in vehicle registration fees under

    Section 3-806.3 of the Illinois Vehicle Code.

    (6) Free fishing and reduced fishing license fees under Sections
    20-5 and 20-40 of the Fish and Aquatic Life Code.

    (7) Tuition free courses for senior citizens under the Senior
    Citizen Courses Act.

    (8) Any benefits under the Illinois Public Aid Code.

    (b) If a person has been found by a court to have knowingly received
    benefits in violation of subsection (a) and:

    (1) the total monetary value of the benefits received is less than
    \$150, the person is guilty of a Class A misdemeanor; a second or
    subsequent violation is a Class 4 felony;

    (2) the total monetary value of the benefits received is \$150 or
    more but less than \$1,000, the person is guilty of a Class 4
    felony; a second or subsequent violation is a Class 3 felony;

    (3) the total monetary value of the benefits received is \$1,000 or
    more but less than \$5,000, the person is guilty of a Class 3
    felony; a second or subsequent violation is a Class 2 felony;

    (4) the total monetary value of the benefits received is \$5,000 or
    more but less than \$10,000, the person is guilty of a Class 2
    felony; a second or subsequent violation is a Class 1 felony; or

    (5) the total monetary value of the benefits received is \$10,000 or
    more, the person is guilty of a Class 1 felony.

    (c) For purposes of determining the classification of an offense
    under this Section, all of the monetary value of the benefits
    received as a result of the unlawful act, practice, or course of
    conduct may be accumulated.

    (d) Any grants awarded to persons described in subsection (a) may be
    recovered by the State of Illinois in a civil action commenced by
    the Attorney General in the circuit court of Sangamon County or the
    State's Attorney of the county of residence of the person described
    in subsection (a).

    (e) An individual described in subsection (a) who has been deported
    shall be restored to any benefits which that individual has been
    denied under State law pursuant to subsection (a) if (i) the
    Attorney General of the United States has issued an order cancelling
    deportation and has adjusted the status of the individual to that of
    a person lawfully admitted for permanent residence in the United
    States or (ii) the country to which the individual has been deported
    adjudicates or exonerates the individual in a judicial or
    administrative proceeding as not being guilty of the persecution of
    others on account of race, religion, national origin, or political
    opinion under the direction of or in association with the Nazi
    government of Germany or its allies.

    (Source: P.A. 102-1030, eff. 5-27-22.)

    \hypertarget{ilcs-517-7-from-ch.-38-par.-17-7}{%
    \subsection*{(720 ILCS 5/17-7) (from Ch. 38, par.
    17-7)}\label{ilcs-517-7-from-ch.-38-par.-17-7}}
    \addcontentsline{toc}{subsection}{(720 ILCS 5/17-7) (from Ch. 38,
    par. 17-7)}

    (This Section was renumbered as Section 17-60 by P.A. 96-1551.)

    \hypertarget{sec.-17-7.-renumbered.}{%
    \section*{Sec. 17-7. (Renumbered).}\label{sec.-17-7.-renumbered.}}
    \addcontentsline{toc}{section}{Sec. 17-7. (Renumbered).}

    \markright{Sec. 17-7. (Renumbered).}

    (Source: P.A. 83-808. Renumbered by P.A. 96-1551, eff. 7-1-11 .)

    \hypertarget{ilcs-517-8-from-ch.-38-par.-17-8}{%
    \subsection*{(720 ILCS 5/17-8) (from Ch. 38, par.
    17-8)}\label{ilcs-517-8-from-ch.-38-par.-17-8}}
    \addcontentsline{toc}{subsection}{(720 ILCS 5/17-8) (from Ch. 38,
    par. 17-8)}

    \hypertarget{sec.-17-8.-repealed.}{%
    \section*{Sec. 17-8. (Repealed).}\label{sec.-17-8.-repealed.}}
    \addcontentsline{toc}{section}{Sec. 17-8. (Repealed).}

    \markright{Sec. 17-8. (Repealed).}

    (Source: P.A. 84-418. Repealed by P.A. 96-1551, eff. 7-1-11 .)

    \hypertarget{ilcs-517-8.3}{%
    \subsection*{(720 ILCS 5/17-8.3)}\label{ilcs-517-8.3}}
    \addcontentsline{toc}{subsection}{(720 ILCS 5/17-8.3)}

    (was 720 ILCS 5/17-22)

    \hypertarget{sec.-17-8.3.-false-information-on-an-application-for-employment-with-certain-public-or-private-agencies-use-of-false-academic-degree.}{%
    \section*{Sec. 17-8.3. False information on an application for
    employment with certain public or private agencies; use of false
    academic
    degree.}\label{sec.-17-8.3.-false-information-on-an-application-for-employment-with-certain-public-or-private-agencies-use-of-false-academic-degree.}}
    \addcontentsline{toc}{section}{Sec. 17-8.3. False information on an
    application for employment with certain public or private agencies;
    use of false academic degree.}

    \markright{Sec. 17-8.3. False information on an application for
    employment with certain public or private agencies; use of false
    academic degree.}

    (a) It is unlawful for an applicant for employment with a public or
    private agency that provides State funded services to persons with
    mental illness or developmental disabilities to knowingly furnish
    false information regarding professional certification, licensing,
    criminal background, or employment history for the 5 years
    immediately preceding the date of application on an application for
    employment with the agency if the position of employment requires or
    provides opportunity for contact with persons with mental illness or
    developmental disabilities.

    (b) It is unlawful for a person to knowingly use a false academic
    degree for the purpose of obtaining employment or admission to an
    institution of higher learning or admission to an advanced degree
    program at an institution of higher learning or for the purpose of
    obtaining a promotion or higher compensation in employment.

    (c) Sentence. A violation of this Section is a Class A misdemeanor.

    (Source: P.A. 96-1551, eff. 7-1-11 .)

    \hypertarget{ilcs-517-8.5}{%
    \subsection*{(720 ILCS 5/17-8.5)}\label{ilcs-517-8.5}}
    \addcontentsline{toc}{subsection}{(720 ILCS 5/17-8.5)}

    \hypertarget{sec.-17-8.5.-fraud-on-a-governmental-entity.}{%
    \section*{Sec. 17-8.5. Fraud on a governmental
    entity.}\label{sec.-17-8.5.-fraud-on-a-governmental-entity.}}
    \addcontentsline{toc}{section}{Sec. 17-8.5. Fraud on a governmental
    entity.}

    \markright{Sec. 17-8.5. Fraud on a governmental entity.}

    (a) Fraud on a governmental entity. A person commits fraud on a
    governmental entity when he or she knowingly obtains, attempts to
    obtain, or causes to be obtained, by deception, control over the
    property of any governmental entity by the making of a false claim
    of bodily injury or of damage to or loss or theft of property or by
    causing a false claim of bodily injury or of damage to or loss or
    theft of property to be made against the governmental entity,
    intending to deprive the governmental entity permanently of the use
    and benefit of that property.

    (b) Aggravated fraud on a governmental entity. A person commits
    aggravated fraud on a governmental entity when he or she commits
    fraud on a governmental entity 3 or more times within an 18-month
    period arising out of separate incidents or transactions.

    (c) Conspiracy to commit fraud on a governmental entity. If
    aggravated fraud on a governmental entity forms the basis for a
    charge of conspiracy under Section 8-2 of this Code against a
    person, the person or persons with whom the accused is alleged to
    have agreed to commit the 3 or more violations of this Section need
    not be the same person or persons for each violation, as long as the
    accused was a part of the common scheme or plan to engage in each of
    the 3 or more alleged violations.

    (d) Organizer of an aggravated fraud on a governmental entity
    conspiracy. A person commits being an organizer of an aggravated
    fraud on a governmental entity conspiracy if aggravated fraud on a
    governmental entity forms the basis for a charge of conspiracy under
    Section 8-2 of this Code and the person occupies a position of
    organizer, supervisor, financer, or other position of management
    within the conspiracy.

    For the purposes of this Section, the person or persons with whom
    the accused is alleged to have agreed to commit the 3 or more
    violations of subdivision (a)(1) of Section 17-10.5 or subsection
    (a) of Section 17-8.5 of this Code need not be the same person or
    persons for each violation, as long as the accused occupied a
    position of organizer, supervisor, financer, or other position of
    management in each of the 3 or more alleged violations.

    Notwithstanding Section 8-5 of this Code, a person may be convicted
    and sentenced both for the offense of being an organizer of an
    aggravated fraud conspiracy and for any other offense that is the
    object of the conspiracy.

    (e) Sentence.

    (1) A violation of subsection (a) in which the value of the property
    obtained or attempted to be obtained is \$300 or less is a Class A
    misdemeanor.

    (2) A violation of subsection (a) in which the value of the property
    obtained or attempted to be obtained is more than \$300 but not more
    than \$10,000 is a Class 3 felony.

    (3) A violation of subsection (a) in which the value of the property
    obtained or attempted to be obtained is more than \$10,000 but not
    more than \$100,000 is a Class 2 felony.

    (4) A violation of subsection (a) in which the value of the property
    obtained or attempted to be obtained is more than \$100,000 is a
    Class 1 felony.

    (5) A violation of subsection (b) is a Class 1 felony, regardless of
    the value of the property obtained, attempted to be obtained, or
    caused to be obtained.

    (6) The offense of being an organizer of an aggravated fraud
    conspiracy is a Class X felony.

    (7) Notwithstanding Section 8-5 of this Code, a person may be
    convicted and sentenced both for the offense of conspiracy to commit
    fraud and for any other offense that is the object of the
    conspiracy.

    (f) Civil damages for fraud on a governmental entity. A person who
    knowingly obtains, attempts to obtain, or causes to be obtained, by
    deception, control over the property of a governmental entity by the
    making of a false claim of bodily injury or of damage to or loss or
    theft of property, intending to deprive the governmental entity
    permanently of the use and benefit of that property, shall be
    civilly liable to the governmental entity that paid the claim or
    against whom the claim was made or to the subrogee of the
    governmental entity in an amount equal to either 3 times the value
    of the property wrongfully obtained or, if property was not
    wrongfully obtained, twice the value of the property attempted to be
    obtained, whichever amount is greater, plus reasonable attorney's
    fees.

    (g) Determination of property value. For the purposes of this
    Section, if the exact value of the property attempted to be obtained
    is either not alleged by the claimant or not otherwise specifically
    set, the value of the property shall be the fair market replacement
    value of the property claimed to be lost, the reasonable costs of
    reimbursing a vendor or other claimant for services to be rendered,
    or both.

    (h) Actions by State licensing agencies.

    (1) All State licensing agencies, the Illinois State

    Police, and the Department of Financial and Professional Regulation
    shall coordinate enforcement efforts relating to acts of fraud on a
    governmental entity.

    (2) If a person who is licensed or registered under the laws of the
    State of Illinois to engage in a business or profession is convicted
    of or pleads guilty to engaging in an act of fraud on a governmental
    entity, the Illinois State Police must forward to each State agency
    by which the person is licensed or registered a copy of the
    conviction or plea and all supporting evidence.

    (3) Any agency that receives information under this

    Section shall, not later than 6 months after the date on which it
    receives the information, publicly report the final action taken
    against the convicted person, including but not limited to the
    revocation or suspension of the license or any other disciplinary
    action taken.

    (i) Definitions. For the purposes of this Section, ``obtain'',
    ``obtains control'', ``deception'', ``property'', and ``permanent
    deprivation'' have the meanings ascribed to those terms in Article
    15 of this Code.

    (Source: P.A. 96-1551, eff. 7-1-11 .)

    \hypertarget{ilcs-517-9-from-ch.-38-par.-17-9}{%
    \subsection*{(720 ILCS 5/17-9) (from Ch. 38, par.
    17-9)}\label{ilcs-517-9-from-ch.-38-par.-17-9}}
    \addcontentsline{toc}{subsection}{(720 ILCS 5/17-9) (from Ch. 38,
    par. 17-9)}

    \hypertarget{sec.-17-9.-public-aid-wire-and-mail-fraud.}{%
    \section*{Sec. 17-9. Public aid wire and mail
    fraud.}\label{sec.-17-9.-public-aid-wire-and-mail-fraud.}}
    \addcontentsline{toc}{section}{Sec. 17-9. Public aid wire and mail
    fraud.}

    \markright{Sec. 17-9. Public aid wire and mail fraud.}

    (a) Whoever knowingly (i) makes or transmits any communication by
    means of telephone, wire, radio, or television or (ii) places any
    communication with the United States Postal Service, or with any
    private or other mail, package, or delivery service or system, such
    communication being made, transmitted, placed, or received within
    the State of Illinois, intending that such communication be made,
    transmitted, or delivered in furtherance of any plan, scheme, or
    design to obtain, unlawfully, any benefit or payment under the
    Illinois Public Aid Code, commits public aid wire and mail fraud.

    (b) Whoever knowingly directs or causes any communication to be (i)
    made or transmitted by means of telephone, wire, radio, or
    television or (ii) placed with the United States Postal Service, or
    with any private or other mail, package, or delivery service or
    system, intending that such communication be made, transmitted, or
    delivered in furtherance of any plan, scheme, or design to obtain,
    unlawfully, any benefit or payment under the Illinois Public Aid
    Code, commits public aid wire and mail fraud.

    (c) Sentence. A violation of this Section is a Class 4 felony.

    (Source: P.A. 96-1551, eff. 7-1-11 .)

    \hypertarget{ilcs-517-10-from-ch.-38-par.-17-10}{%
    \subsection*{(720 ILCS 5/17-10) (from Ch. 38, par.
    17-10)}\label{ilcs-517-10-from-ch.-38-par.-17-10}}
    \addcontentsline{toc}{subsection}{(720 ILCS 5/17-10) (from Ch. 38,
    par. 17-10)}

    \hypertarget{sec.-17-10.-repealed.}{%
    \section*{Sec. 17-10. (Repealed).}\label{sec.-17-10.-repealed.}}
    \addcontentsline{toc}{section}{Sec. 17-10. (Repealed).}

    \markright{Sec. 17-10. (Repealed).}

    (Source: P.A. 84-1438. Repealed by P.A. 96-1551, eff. 7-1-11 .)

    \hypertarget{ilcs-517-10.2}{%
    \subsection*{(720 ILCS 5/17-10.2)}\label{ilcs-517-10.2}}
    \addcontentsline{toc}{subsection}{(720 ILCS 5/17-10.2)}

    (was 720 ILCS 5/17-29)

    \hypertarget{sec.-17-10.2.-businesses-owned-by-minorities-females-and-persons-with-disabilities-fraudulent-contracts-with-governmental-units.}{%
    \section*{Sec. 17-10.2. Businesses owned by minorities, females, and
    persons with disabilities; fraudulent contracts with governmental
    units.}\label{sec.-17-10.2.-businesses-owned-by-minorities-females-and-persons-with-disabilities-fraudulent-contracts-with-governmental-units.}}
    \addcontentsline{toc}{section}{Sec. 17-10.2. Businesses owned by
    minorities, females, and persons with disabilities; fraudulent
    contracts with governmental units.}

    \markright{Sec. 17-10.2. Businesses owned by minorities, females,
    and persons with disabilities; fraudulent contracts with
    governmental units.}

    (a) In this Section:

    ``Minority person'' means a person who is any of the following:

    (1) American Indian or Alaska Native (a person having origins in any
    of the original peoples of North and South America, including
    Central America, and who maintains tribal affiliation or community
    attachment).

    (2) Asian (a person having origins in any of the original peoples of
    the Far East, Southeast Asia, or the Indian subcontinent, including,
    but not limited to, Cambodia, China, India, Japan, Korea, Malaysia,
    Pakistan, the Philippine Islands, Thailand, and Vietnam).

    (3) Black or African American (a person having origins in any of the
    black racial groups of Africa).

    (4) Hispanic or Latino (a person of Cuban, Mexican,

    Puerto Rican, South or Central American, or other Spanish culture or
    origin, regardless of race).

    (5) Native Hawaiian or Other Pacific Islander (a person having
    origins in any of the original peoples of Hawaii, Guam, Samoa, or
    other Pacific Islands).

    ``Female'' means a person who is of the female gender.

    ``Person with a disability'' means a person who is a person
    qualifying as having a disability.

    ``Disability'' means a severe physical or mental disability that:
    (1) results from: amputation, arthritis, autism, blindness, burn
    injury, cancer, cerebral palsy, cystic fibrosis, deafness, head
    injury, heart disease, hemiplegia, hemophilia, respiratory or
    pulmonary dysfunction, an intellectual disability, mental illness,
    multiple sclerosis, muscular dystrophy, musculoskeletal disorders,
    neurological disorders, including stroke and epilepsy, paraplegia,
    quadriplegia and other spinal cord conditions, sickle cell anemia,
    specific learning disabilities, or end stage renal failure disease;
    and (2) substantially limits one or more of the person's major life
    activities.

    ``Minority owned business'' means a business concern that is at
    least 51\% owned by one or more minority persons, or in the case of
    a corporation, at least 51\% of the stock in which is owned by one
    or more minority persons; and the management and daily business
    operations of which are controlled by one or more of the minority
    individuals who own it.

    ``Female owned business'' means a business concern that is at least
    51\% owned by one or more females, or, in the case of a corporation,
    at least 51\% of the stock in which is owned by one or more females;
    and the management and daily business operations of which are
    controlled by one or more of the females who own it.

    ``Business owned by a person with a disability'' means a business
    concern that is at least 51\% owned by one or more persons with a
    disability and the management and daily business operations of which
    are controlled by one or more of the persons with disabilities who
    own it. A not-for-profit agency for persons with disabilities that
    is exempt from taxation under Section 501 of the Internal Revenue
    Code of 1986 is also considered a ``business owned by a person with
    a disability''.

    ``Governmental unit'' means the State, a unit of local government,
    or school district.

    (b) In addition to any other penalties imposed by law or by an
    ordinance or resolution of a unit of local government or school
    district, any individual or entity that knowingly obtains, or
    knowingly assists another to obtain, a contract with a governmental
    unit, or a subcontract or written commitment for a subcontract under
    a contract with a governmental unit, by falsely representing that
    the individual or entity, or the individual or entity assisted, is a
    minority owned business, female owned business, or business owned by
    a person with a disability is guilty of a Class 2 felony, regardless
    of whether the preference for awarding the contract to a minority
    owned business, female owned business, or business owned by a person
    with a disability was established by statute or by local ordinance
    or resolution.

    (c) In addition to any other penalties authorized by law, the court
    shall order that an individual or entity convicted of a violation of
    this Section must pay to the governmental unit that awarded the
    contract a penalty equal to one and one-half times the amount of the
    contract obtained because of the false representation.

    (Source: P.A. 102-465, eff. 1-1-22 .)

    \hypertarget{ilcs-517-10.3}{%
    \subsection*{(720 ILCS 5/17-10.3)}\label{ilcs-517-10.3}}
    \addcontentsline{toc}{subsection}{(720 ILCS 5/17-10.3)}

    \hypertarget{sec.-17-10.3.-deception-relating-to-certification-of-disadvantaged-business-enterprises.}{%
    \section*{Sec. 17-10.3. Deception relating to certification of
    disadvantaged business
    enterprises.}\label{sec.-17-10.3.-deception-relating-to-certification-of-disadvantaged-business-enterprises.}}
    \addcontentsline{toc}{section}{Sec. 17-10.3. Deception relating to
    certification of disadvantaged business enterprises.}

    \markright{Sec. 17-10.3. Deception relating to certification of
    disadvantaged business enterprises.}

    (a) Fraudulently obtaining or retaining certification. A person who,
    in the course of business, fraudulently obtains or retains
    certification as a minority-owned business, women-owned business,
    service-disabled veteran-owned small business, or veteran-owned
    small business commits a Class 2 felony.

    (b) Willfully making a false statement. A person who, in the course
    of business, willfully makes a false statement whether by affidavit,
    report or other representation, to an official or employee of a
    State agency or the Business Enterprise Council for Minorities,
    Women, and Persons with Disabilities for the purpose of influencing
    the certification or denial of certification of any business entity
    as a minority-owned business, women-owned business, service-disabled
    veteran-owned small business, or veteran-owned small business
    commits a Class 2 felony.

    (c) Willfully obstructing or impeding an official or employee of any
    agency in his or her investigation. Any person who, in the course of
    business, willfully obstructs or impedes an official or employee of
    any State agency or the Business Enterprise Council for Minorities,
    Women, and Persons with Disabilities who is investigating the
    qualifications of a business entity which has requested
    certification as a minority-owned business, women-owned business,
    service-disabled veteran-owned small business, or veteran-owned
    small business commits a Class 2 felony.

    (d) Fraudulently obtaining public moneys reserved for disadvantaged
    business enterprises. Any person who, in the course of business,
    fraudulently obtains public moneys reserved for, or allocated or
    available to, minority-owned businesses, women-owned businesses,
    service-disabled veteran-owned small businesses, or veteran-owned
    small businesses commits a Class 2 felony.

    (e) Definitions. As used in this Article, ``minority-owned
    business'', ``women-owned business'', ``State agency'' with respect
    to minority-owned businesses and women-owned businesses, and
    ``certification'' with respect to minority-owned businesses and
    women-owned businesses shall have the meanings ascribed to them in
    Section 2 of the Business Enterprise for Minorities, Women, and
    Persons with Disabilities Act. As used in this Article,
    ``service-disabled veteran-owned small business'', ``veteran-owned
    small business'', ``State agency'' with respect to service-disabled
    veteran-owned small businesses and veteran-owned small businesses,
    and ``certification'' with respect to service-disabled veteran-owned
    small businesses and veteran-owned small businesses have the same
    meanings as in Section 45-57 of the Illinois Procurement Code.

    (Source: P.A. 100-391, eff. 8-25-17; 101-170, eff. 1-1-20; 101-601,
    eff. 1-1-20.)

    \hypertarget{ilcs-5art.-17-subdiv.-15-heading}{%
    \subsection*{(720 ILCS 5/Art. 17, Subdiv. 15
    heading)}\label{ilcs-5art.-17-subdiv.-15-heading}}
    \addcontentsline{toc}{subsection}{(720 ILCS 5/Art. 17, Subdiv. 15
    heading)}

    SUBDIVISION 15.

    FRAUD ON A PRIVATE ENTITY

    (Source: P.A. 96-1551, eff. 7-1-11.)

    \hypertarget{ilcs-517-10.5}{%
    \subsection*{(720 ILCS 5/17-10.5)}\label{ilcs-517-10.5}}
    \addcontentsline{toc}{subsection}{(720 ILCS 5/17-10.5)}

    \hypertarget{sec.-17-10.5.-insurance-fraud.}{%
    \section*{Sec. 17-10.5. Insurance
    fraud.}\label{sec.-17-10.5.-insurance-fraud.}}
    \addcontentsline{toc}{section}{Sec. 17-10.5. Insurance fraud.}

    \markright{Sec. 17-10.5. Insurance fraud.}

    (a) Insurance fraud.

    (1) A person commits insurance fraud when he or she knowingly
    obtains, attempts to obtain, or causes to be obtained, by deception,
    control over the property of an insurance company or self-insured
    entity by the making of a false claim or by causing a false claim to
    be made on any policy of insurance issued by an insurance company or
    by the making of a false claim or by causing a false claim to be
    made to a self-insured entity, intending to deprive an insurance
    company or self-insured entity permanently of the use and benefit of
    that property.

    (2) A person commits health care benefits fraud against a provider,
    other than a governmental unit or agency, when he or she knowingly
    obtains or attempts to obtain, by deception, health care benefits
    and that obtaining or attempt to obtain health care benefits does
    not involve control over property of the provider.

    (b) Aggravated insurance fraud.

    (1) A person commits aggravated insurance fraud on a private entity
    when he or she commits insurance fraud 3 or more times within an
    18-month period arising out of separate incidents or transactions.

    (2) A person commits being an organizer of an aggravated insurance
    fraud on a private entity conspiracy if aggravated insurance fraud
    on a private entity forms the basis for a charge of conspiracy under
    Section 8-2 of this Code and the person occupies a position of
    organizer, supervisor, financer, or other position of management
    within the conspiracy.

    (c) Conspiracy to commit insurance fraud. If aggravated insurance
    fraud on a private entity forms the basis for charges of conspiracy
    under Section 8-2 of this Code, the person or persons with whom the
    accused is alleged to have agreed to commit the 3 or more violations
    of this Section need not be the same person or persons for each
    violation, as long as the accused was a part of the common scheme or
    plan to engage in each of the 3 or more alleged violations.

    If aggravated insurance fraud on a private entity forms the basis
    for a charge of conspiracy under Section 8-2 of this Code, and the
    accused occupies a position of organizer, supervisor, financer, or
    other position of management within the conspiracy, the person or
    persons with whom the accused is alleged to have agreed to commit
    the 3 or more violations of this Section need not be the same person
    or persons for each violation as long as the accused occupied a
    position of organizer, supervisor, financer, or other position of
    management in each of the 3 or more alleged violations.

    (d) Sentence.

    (1) A violation of paragraph (a)(1) in which the value of the
    property obtained, attempted to be obtained, or caused to be
    obtained is \$300 or less is a Class A misdemeanor.

    (2) A violation of paragraph (a)(1) in which the value of the
    property obtained, attempted to be obtained, or caused to be
    obtained is more than \$300 but not more than \$10,000 is a Class 3
    felony.

    (3) A violation of paragraph (a)(1) in which the value of the
    property obtained, attempted to be obtained, or caused to be
    obtained is more than \$10,000 but not more than \$100,000 is a
    Class 2 felony.

    (4) A violation of paragraph (a)(1) in which the value of the
    property obtained, attempted to be obtained, or caused to be
    obtained is more than \$100,000 is a Class 1 felony.

    (5) A violation of paragraph (a)(2) is a Class A misdemeanor.

    (6) A violation of paragraph (b)(1) is a Class 1 felony, regardless
    of the value of the property obtained, attempted to be obtained, or
    caused to be obtained.

    (7) A violation of paragraph (b)(2) is a Class X felony.

    (8) A person convicted of insurance fraud, vendor fraud, or a
    federal criminal violation associated with defrauding the Medicaid
    program shall be ordered to pay monetary restitution to the
    insurance company or self-insured entity or any other person for any
    financial loss sustained as a result of a violation of this Section,
    including any court costs and attorney's fees. An order of
    restitution shall include expenses incurred and paid by the State of
    Illinois or an insurance company or self-insured entity in
    connection with any medical evaluation or treatment services.

    (9) Notwithstanding Section 8-5 of this Code, a person may be
    convicted and sentenced both for the offense of conspiracy to commit
    insurance fraud or the offense of being an organizer of an
    aggravated insurance fraud conspiracy and for any other offense that
    is the object of the conspiracy.

    (e) Civil damages for insurance fraud.

    (1) A person who knowingly obtains, attempts to obtain, or causes to
    be obtained, by deception, control over the property of any
    insurance company by the making of a false claim or by causing a
    false claim to be made on a policy of insurance issued by an
    insurance company, or by the making of a false claim or by causing a
    false claim to be made to a self-insured entity, intending to
    deprive an insurance company or self-insured entity permanently of
    the use and benefit of that property, shall be civilly liable to the
    insurance company or self-insured entity that paid the claim or
    against whom the claim was made or to the subrogee of that insurance
    company or self-insured entity in an amount equal to either 3 times
    the value of the property wrongfully obtained or, if no property was
    wrongfully obtained, twice the value of the property attempted to be
    obtained, whichever amount is greater, plus reasonable attorney's
    fees.

    (2) An insurance company or self-insured entity that brings an
    action against a person under paragraph (1) of this subsection in
    bad faith shall be liable to that person for twice the value of the
    property claimed, plus reasonable attorney's fees. In determining
    whether an insurance company or self-insured entity acted in bad
    faith, the court shall relax the rules of evidence to allow for the
    introduction of any facts or other information on which the
    insurance company or self-insured entity may have relied in bringing
    an action under paragraph (1) of this subsection.

    (f) Determination of property value. For the purposes of this
    Section, if the exact value of the property attempted to be obtained
    is either not alleged by the claimant or not specifically set by the
    terms of a policy of insurance, the value of the property shall be
    the fair market replacement value of the property claimed to be
    lost, the reasonable costs of reimbursing a vendor or other claimant
    for services to be rendered, or both.

    (g) Actions by State licensing agencies.

    (1) All State licensing agencies, the Illinois State

    Police, and the Department of Financial and Professional Regulation
    shall coordinate enforcement efforts relating to acts of insurance
    fraud.

    (2) If a person who is licensed or registered under the laws of the
    State of Illinois to engage in a business or profession is convicted
    of or pleads guilty to engaging in an act of insurance fraud, the
    Illinois State Police must forward to each State agency by which the
    person is licensed or registered a copy of the conviction or plea
    and all supporting evidence.

    (3) Any agency that receives information under this

    Section shall, not later than 6 months after the date on which it
    receives the information, publicly report the final action taken
    against the convicted person, including but not limited to the
    revocation or suspension of the license or any other disciplinary
    action taken.

    (h) Definitions. For the purposes of this Section, ``obtain'',
    ``obtains control'', ``deception'', ``property'', and ``permanent
    deprivation'' have the meanings ascribed to those terms in Article
    15 of this Code.

    (Source: P.A. 96-1551, eff. 7-1-11; 97-1150, eff. 1-25-13.)

    \hypertarget{ilcs-517-10.6}{%
    \subsection*{(720 ILCS 5/17-10.6)}\label{ilcs-517-10.6}}
    \addcontentsline{toc}{subsection}{(720 ILCS 5/17-10.6)}

    \hypertarget{sec.-17-10.6.-financial-institution-fraud.}{%
    \section*{Sec. 17-10.6. Financial institution
    fraud.}\label{sec.-17-10.6.-financial-institution-fraud.}}
    \addcontentsline{toc}{section}{Sec. 17-10.6. Financial institution
    fraud.}

    \markright{Sec. 17-10.6. Financial institution fraud.}

    (a) Misappropriation of financial institution property. A person
    commits misappropriation of a financial institution's property
    whenever he or she knowingly obtains or exerts unauthorized control
    over any of the moneys, funds, credits, assets, securities, or other
    property owned by or under the custody or control of a financial
    institution, or under the custody or care of any agent, officer,
    director, or employee of such financial institution.

    (b) Commercial bribery of a financial institution.

    (1) A person commits commercial bribery of a financial institution
    when he or she knowingly confers or offers or agrees to confer any
    benefit upon any employee, agent, or fiduciary without the consent
    of the latter's employer or principal, with the intent to influence
    his or her conduct in relation to his or her employer's or
    principal's affairs.

    (2) An employee, agent, or fiduciary of a financial institution
    commits commercial bribery of a financial institution when, without
    the consent of his or her employer or principal, he or she knowingly
    solicits, accepts, or agrees to accept any benefit from another
    person upon an agreement or understanding that such benefit will
    influence his or her conduct in relation to his or her employer's or
    principal's affairs.

    (c) Financial institution fraud. A person commits financial
    institution fraud when he or she knowingly executes or attempts to
    execute a scheme or artifice:

    (1) to defraud a financial institution; or

    (2) to obtain any of the moneys, funds, credits, assets, securities,
    or other property owned by or under the custody or control of a
    financial institution, by means of pretenses, representations, or
    promises he or she knows to be false.

    (d) Loan fraud. A person commits loan fraud when he or she
    knowingly, with intent to defraud, makes any false statement or
    report, or overvalues any land, property, or security, with the
    intent to influence in any way the action of a financial institution
    to act upon any application, advance, discount, purchase, purchase
    agreement, repurchase agreement, commitment, or loan, or any change
    or extension of any of the same, by renewal, deferment of action, or
    otherwise, or the acceptance, release, or substitution of security.

    (e) Concealment of collateral. A person commits concealment of
    collateral when he or she, with intent to defraud, knowingly
    conceals, removes, disposes of, or converts to the person's own use
    or to that of another any property mortgaged or pledged to or held
    by a financial institution.

    (f) Financial institution robbery. A person commits robbery when he
    or she knowingly, by force or threat of force, or by intimidation,
    takes, or attempts to take, from the person or presence of another,
    or obtains or attempts to obtain by extortion, any property or money
    or any other thing of value belonging to, or in the care, custody,
    control, management, or possession of, a financial institution.

    (g) Conspiracy to commit a financial crime.

    (1) A person commits conspiracy to commit a financial crime when,
    with the intent that any violation of this Section be committed, he
    or she agrees with another person to the commission of that offense.

    (2) No person may be convicted of conspiracy to commit a financial
    crime unless an overt act or acts in furtherance of the agreement is
    alleged and proved to have been committed by that person or by a
    co-conspirator and the accused is a part of a common scheme or plan
    to engage in the unlawful activity.

    (3) It shall not be a defense to conspiracy to commit a financial
    crime that the person or persons with whom the accused is alleged to
    have conspired:

    (A) has not been prosecuted or convicted;

    (B) has been convicted of a different offense;

    (C) is not amenable to justice;

    (D) has been acquitted; or

    (E) lacked the capacity to commit the offense.

    (h) Continuing financial crimes enterprise. A person commits a
    continuing financial crimes enterprise when he or she knowingly,
    within an 18-month period, commits 3 or more separate offenses
    constituting any combination of the following:

    (1) an offense under this Section;

    (2) a felony offense in violation of Section 16A-3 or subsection (a)
    of Section 16-25 or paragraph (4) or (5) of subsection (a) of
    Section 16-1 of this Code for the purpose of reselling or otherwise
    re-entering the merchandise in commerce, including conveying the
    merchandise to a merchant in exchange for anything of value; or

    (3) if involving a financial institution, any other felony offense
    under this Code.

    (i) Organizer of a continuing financial crimes enterprise.

    (1) A person commits being an organizer of a continuing financial
    crimes enterprise when he or she:

    (A) with the intent to commit any offense, agrees with another
    person to the commission of any combination of the following
    offenses on 3 or more separate occasions within an 18-month period:

    (i) an offense under this Section;

    (ii) a felony offense in violation of Section

    16A-3 or subsection (a) of Section 16-25 or paragraph (4) or (5) of
    subsection (a) of Section 16-1 of this Code for the purpose of
    reselling or otherwise re-entering the merchandise in commerce,
    including conveying the merchandise to a merchant in exchange for
    anything of value; or

    (iii) if involving a financial institution, any other felony offense
    under this Code; and

    (B) with respect to the other persons within the conspiracy,
    occupies a position of organizer, supervisor, or financier or other
    position of management.

    (2) The person with whom the accused agreed to commit the 3 or more
    offenses under this Section, or, if involving a financial
    institution, any other felony offenses under this Code, need not be
    the same person or persons for each offense, as long as the accused
    was a part of the common scheme or plan to engage in each of the 3
    or more alleged offenses.

    (j) Sentence.

    (1) Except as otherwise provided in this subsection, a violation of
    this Section, the full value of which:

    (A) does not exceed \$500, is a Class A misdemeanor;

    (B) does not exceed \$500, and the person has been previously
    convicted of a financial crime or any type of theft, robbery, armed
    robbery, burglary, residential burglary, possession of burglary
    tools, or home invasion, is guilty of a Class 4 felony;

    (C) exceeds \$500 but does not exceed \$10,000, is a Class 3 felony;

    (D) exceeds \$10,000 but does not exceed \$100,000, is a Class 2
    felony;

    (E) exceeds \$100,000 but does not exceed

    \$500,000, is a Class 1 felony;

    (F) exceeds \$500,000 but does not exceed

    \$1,000,000, is a Class 1 non-probationable felony; when a charge of
    financial crime, the full value of which exceeds \$500,000 but does
    not exceed \$1,000,000, is brought, the value of the financial crime
    involved is an element of the offense to be resolved by the trier of
    fact as either exceeding or not exceeding \$500,000;

    (G) exceeds \$1,000,000, is a Class X felony; when a charge of
    financial crime, the full value of which exceeds \$1,000,000, is
    brought, the value of the financial crime involved is an element of
    the offense to be resolved by the trier of fact as either exceeding
    or not exceeding \$1,000,000.

    (2) A violation of subsection (f) is a Class 1 felony.

    (3) A violation of subsection (h) is a Class 1 felony.

    (4) A violation for subsection (i) is a Class X felony.

    (k) A ``financial crime'' means an offense described in this
    Section.

    (l) Period of limitations. The period of limitations for prosecution
    of any offense defined in this Section begins at the time when the
    last act in furtherance of the offense is committed.

    (m) Forfeiture. Any violation of subdivision (2) of subsection (h)
    or subdivision (i)(1)(A)(ii) shall be subject to the remedies,
    procedures, and forfeiture as set forth in Article 29B of this Code.

    Property seized or forfeited under this Section is subject to
    reporting under the Seizure and Forfeiture Reporting Act.

    (Source: P.A. 100-512, eff. 7-1-18; 100-699, eff. 8-3-18.)

    \hypertarget{ilcs-517-10.7}{%
    \subsection*{(720 ILCS 5/17-10.7)}\label{ilcs-517-10.7}}
    \addcontentsline{toc}{subsection}{(720 ILCS 5/17-10.7)}

    \hypertarget{sec.-17-10.7.-insurance-claims-for-excessive-charges.}{%
    \section*{Sec. 17-10.7. Insurance claims for excessive
    charges.}\label{sec.-17-10.7.-insurance-claims-for-excessive-charges.}}
    \addcontentsline{toc}{section}{Sec. 17-10.7. Insurance claims for
    excessive charges.}

    \markright{Sec. 17-10.7. Insurance claims for excessive charges.}

    (a) A person who sells goods or services commits insurance claims
    for excessive charges if:

    (1) the person knowingly advertises or promises to provide the goods
    or services and to pay:

    (A) all or part of any applicable insurance deductible; or

    (B) a rebate in an amount equal to all or part of any applicable
    insurance deductible;

    (2) the goods or services are paid for by the consumer from proceeds
    of a property or casualty insurance policy; and

    (3) the person knowingly charges an amount for the goods or services
    that exceeds the usual and customary charge by the person for the
    goods or services by an amount equal to or greater than all or part
    of the applicable insurance deductible paid by the person to an
    insurer on behalf of an insured or remitted to an insured by the
    person as a rebate.

    (b) A person who is insured under a property or casualty insurance
    policy commits insurance claims for excessive charges if the person
    knowingly:

    (1) submits a claim under the policy based on charges that are in
    violation of subsection (a) of this Section; or

    (2) knowingly allows a claim in violation of subsection (a) of this
    Section to be submitted, unless the person promptly notifies the
    insurer of the excessive charges.

    (c) Sentence. A violation of this Section is a Class A misdemeanor.

    (Source: P.A. 96-1551, eff. 7-1-11 .)

    \hypertarget{ilcs-5art.-17-subdiv.-20-heading}{%
    \subsection*{(720 ILCS 5/Art. 17, Subdiv. 20
    heading)}\label{ilcs-5art.-17-subdiv.-20-heading}}
    \addcontentsline{toc}{subsection}{(720 ILCS 5/Art. 17, Subdiv. 20
    heading)}

    SUBDIVISION 20.

    FRAUDULENT TAMPERING

    (Source: P.A. 96-1551, eff. 7-1-11.)

    \hypertarget{ilcs-517-11-from-ch.-38-par.-17-11}{%
    \subsection*{(720 ILCS 5/17-11) (from Ch. 38, par.
    17-11)}\label{ilcs-517-11-from-ch.-38-par.-17-11}}
    \addcontentsline{toc}{subsection}{(720 ILCS 5/17-11) (from Ch. 38,
    par. 17-11)}

    \hypertarget{sec.-17-11.-odometer-or-hour-meter-fraud.}{%
    \section*{Sec. 17-11. Odometer or hour meter
    fraud.}\label{sec.-17-11.-odometer-or-hour-meter-fraud.}}
    \addcontentsline{toc}{section}{Sec. 17-11. Odometer or hour meter
    fraud.}

    \markright{Sec. 17-11. Odometer or hour meter fraud.}

    A person commits odometer or hour meter fraud when he or she
    disconnects, resets, or alters, or causes to be disconnected, reset,
    or altered, the odometer of any used motor vehicle or the hour meter
    of any used farm implement to conceal or change the actual miles
    driven or hours of operation with the intent to defraud another. A
    violation of this Section is a Class A misdemeanor. A second or
    subsequent violation is a Class 4 felony. This Section does not
    apply to legitimate practices of automotive or implement parts
    recyclers who recycle used odometers or hour meters for resale.

    (Source: P.A. 96-1551, eff. 7-1-11 .)

    \hypertarget{ilcs-517-11.1}{%
    \subsection*{(720 ILCS 5/17-11.1)}\label{ilcs-517-11.1}}
    \addcontentsline{toc}{subsection}{(720 ILCS 5/17-11.1)}

    \hypertarget{sec.-17-11.1.-repealed.}{%
    \section*{Sec. 17-11.1. (Repealed).}\label{sec.-17-11.1.-repealed.}}
    \addcontentsline{toc}{section}{Sec. 17-11.1. (Repealed).}

    \markright{Sec. 17-11.1. (Repealed).}

    (Source: P.A. 89-626, eff. 8-9-96. Repealed by P.A. 96-1551, eff.
    7-1-11 .)

    \hypertarget{ilcs-517-11.2}{%
    \subsection*{(720 ILCS 5/17-11.2)}\label{ilcs-517-11.2}}
    \addcontentsline{toc}{subsection}{(720 ILCS 5/17-11.2)}

    \hypertarget{sec.-17-11.2.-installation-of-object-in-lieu-of-air-bag.}{%
    \section*{Sec. 17-11.2. Installation of object in lieu of air
    bag.}\label{sec.-17-11.2.-installation-of-object-in-lieu-of-air-bag.}}
    \addcontentsline{toc}{section}{Sec. 17-11.2. Installation of object
    in lieu of air bag.}

    \markright{Sec. 17-11.2. Installation of object in lieu of air bag.}

    A person commits installation of object in lieu of airbag when he or
    she, for consideration, knowingly installs or reinstalls in a
    vehicle any object in lieu of an air bag that was designed in
    accordance with federal safety regulations for the make, model, and
    year of the vehicle as part of a vehicle inflatable restraint
    system. A violation of this Section is a Class A misdemeanor.

    (Source: P.A. 96-1551, eff. 7-1-11 .)

    \hypertarget{ilcs-517-11.5}{%
    \subsection*{(720 ILCS 5/17-11.5)}\label{ilcs-517-11.5}}
    \addcontentsline{toc}{subsection}{(720 ILCS 5/17-11.5)}

    (was 720 ILCS 5/16-22)

    \hypertarget{sec.-17-11.5.-tampering-with-a-security-fire-or-life-safety-system.}{%
    \section*{Sec. 17-11.5. Tampering with a security, fire, or life
    safety
    system.}\label{sec.-17-11.5.-tampering-with-a-security-fire-or-life-safety-system.}}
    \addcontentsline{toc}{section}{Sec. 17-11.5. Tampering with a
    security, fire, or life safety system.}

    \markright{Sec. 17-11.5. Tampering with a security, fire, or life
    safety system.}

    (a) A person commits tampering with a security, fire, or life safety
    system when he or she knowingly damages, sabotages, destroys, or
    causes a permanent or temporary malfunction in any physical or
    electronic security, fire, or life safety system or any component
    part of any of those systems including, but not limited to, card
    readers, magnetic stripe readers, Wiegand card readers, smart card
    readers, proximity card readers, digital keypads, keypad access
    controls, digital locks, electromagnetic locks, electric strikes,
    electronic exit hardware, exit alarm systems, delayed egress
    systems, biometric access control equipment, intrusion detection
    systems and sensors, burglar alarm systems, wireless burglar alarms,
    silent alarms, duress alarms, hold-up alarms, glass break detectors,
    motion detectors, seismic detectors, glass shock sensors, magnetic
    contacts, closed circuit television (CCTV), security cameras,
    digital cameras, dome cameras, covert cameras, spy cameras, hidden
    cameras, wireless cameras, network cameras, IP addressable cameras,
    CCTV camera lenses, video cassette recorders, CCTV monitors, CCTV
    consoles, CCTV housings and enclosures, CCTV pan-and-tilt devices,
    CCTV transmission and signal equipment, wireless video transmitters,
    wireless video receivers, radio frequency (RF) or microwave
    components, or both, infrared illuminators, video motion detectors,
    video recorders, time lapse CCTV recorders, digital video recorders
    (DVRs), digital image storage systems, video converters, video
    distribution amplifiers, video time-date generators, multiplexers,
    switchers, splitters, fire alarms, smoke alarm systems, smoke
    detectors, flame detectors, fire detection systems and sensors, fire
    sprinklers, fire suppression systems, fire extinguishing systems,
    public address systems, intercoms, emergency telephones, emergency
    call boxes, emergency pull stations, telephone entry systems, video
    entry equipment, annunciators, sirens, lights, sounders, control
    panels and components, and all associated computer hardware,
    computer software, control panels, wires, cables, connectors,
    electromechanical components, electronic modules, fiber optics,
    filters, passive components, and power sources including batteries
    and back-up power supplies.

    (b) Sentence. A violation of this Section is a Class 4 felony.

    (Source: P.A. 96-1551, eff. 7-1-11 .)

    \hypertarget{ilcs-517-12}{%
    \subsection*{(720 ILCS 5/17-12)}\label{ilcs-517-12}}
    \addcontentsline{toc}{subsection}{(720 ILCS 5/17-12)}

    \hypertarget{sec.-17-12.-repealed.}{%
    \section*{Sec. 17-12. (Repealed).}\label{sec.-17-12.-repealed.}}
    \addcontentsline{toc}{section}{Sec. 17-12. (Repealed).}

    \markright{Sec. 17-12. (Repealed).}

    (Source: P.A. 93-967, eff. 1-1-05. Repealed by P.A. 96-1551, eff.
    7-1-11 .)

    \hypertarget{ilcs-517-13}{%
    \subsection*{(720 ILCS 5/17-13)}\label{ilcs-517-13}}
    \addcontentsline{toc}{subsection}{(720 ILCS 5/17-13)}

    \hypertarget{sec.-17-13.-fraud-in-transfers-of-real-and-personal-property.}{%
    \section*{Sec. 17-13. Fraud in transfers of real and personal
    property.}\label{sec.-17-13.-fraud-in-transfers-of-real-and-personal-property.}}
    \addcontentsline{toc}{section}{Sec. 17-13. Fraud in transfers of
    real and personal property.}

    \markright{Sec. 17-13. Fraud in transfers of real and personal
    property.}

    (a) Conditional sale; sale without consent of title holder. No
    person purchasing personal property under a conditional sales
    contract shall, during the existence of such conditional sales
    contract and before the conditions thereof have been fulfilled,
    knowingly sell, transfer, conceal, or in any manner dispose of such
    property, or cause or allow the same to be done, without the written
    consent of the holder of title.

    (b) Acknowledgment of fraudulent conveyance. No officer authorized
    to take the proof and acknowledgment of a conveyance of real or
    personal property or other instrument shall knowingly certify that
    the conveyance or other instrument was duly proven or acknowledged
    by a party to the conveyance or other instrument when no such
    acknowledgment or proof was made, or was not made at the time it was
    certified to have been made, with intent to injure or defraud or to
    enable any other person to injure or defraud.

    (c) Fraudulent land sales. No person, after once selling, bartering,
    or disposing of a tract or tracts of land or a town lot or lots, or
    executing a bond or agreement for the sale of lands or a town lot or
    lots, shall again knowingly and with intent to defraud sell, barter,
    or dispose of the same tract or tracts of land or town lot or lots,
    or any part of those tracts of land or town lot or lots, or
    knowingly and with intent to defraud execute a bond or agreement to
    sell, barter, or dispose of the same land or lot or lots, or any
    part of that land or lot or lots, to any other person for a valuable
    consideration.

    (d) Sentence. A violation of subsection (a) of this Section is a
    Class A misdemeanor. A violation of subsection (b) of this Section
    is a Class 4 felony. A violation of subsection (c) of this Section
    is a Class 3 felony.

    (Source: P.A. 96-1551, eff. 7-1-11 .)

    \hypertarget{ilcs-517-14}{%
    \subsection*{(720 ILCS 5/17-14)}\label{ilcs-517-14}}
    \addcontentsline{toc}{subsection}{(720 ILCS 5/17-14)}

    \hypertarget{sec.-17-14.-repealed.}{%
    \section*{Sec. 17-14. (Repealed).}\label{sec.-17-14.-repealed.}}
    \addcontentsline{toc}{section}{Sec. 17-14. (Repealed).}

    \markright{Sec. 17-14. (Repealed).}

    (Source: P.A. 89-234, eff. 1-1-96. Repealed by P.A. 96-1551, eff.
    7-1-11 .)

    \hypertarget{ilcs-517-15}{%
    \subsection*{(720 ILCS 5/17-15)}\label{ilcs-517-15}}
    \addcontentsline{toc}{subsection}{(720 ILCS 5/17-15)}

    \hypertarget{sec.-17-15.-repealed.}{%
    \section*{Sec. 17-15. (Repealed).}\label{sec.-17-15.-repealed.}}
    \addcontentsline{toc}{section}{Sec. 17-15. (Repealed).}

    \markright{Sec. 17-15. (Repealed).}

    (Source: P.A. 89-234, eff. 1-1-96. Repealed by P.A. 96-1551, eff.
    7-1-11 .)

    \hypertarget{ilcs-517-16}{%
    \subsection*{(720 ILCS 5/17-16)}\label{ilcs-517-16}}
    \addcontentsline{toc}{subsection}{(720 ILCS 5/17-16)}

    (This Section was renumbered as Section 17-58 by P.A. 96-1551.)

    \hypertarget{sec.-17-16.-renumbered.}{%
    \section*{Sec. 17-16. (Renumbered).}\label{sec.-17-16.-renumbered.}}
    \addcontentsline{toc}{section}{Sec. 17-16. (Renumbered).}

    \markright{Sec. 17-16. (Renumbered).}

    (Source: P.A. 89-234, eff. 1-1-96. Renumbered by P.A. 96-1551, eff.
    7-1-11 .)

    \hypertarget{ilcs-517-17}{%
    \subsection*{(720 ILCS 5/17-17)}\label{ilcs-517-17}}
    \addcontentsline{toc}{subsection}{(720 ILCS 5/17-17)}

    \hypertarget{sec.-17-17.-fraud-in-stock-transactions.}{%
    \section*{Sec. 17-17. Fraud in stock
    transactions.}\label{sec.-17-17.-fraud-in-stock-transactions.}}
    \addcontentsline{toc}{section}{Sec. 17-17. Fraud in stock
    transactions.}

    \markright{Sec. 17-17. Fraud in stock transactions.}

    (a) No officer, director, or agent of a bank, railroad, or other
    corporation, nor any other person, shall knowingly, and with intent
    to defraud, issue, sell, transfer, assign, or pledge, or cause or
    procure to be issued, sold, transferred, assigned, or pledged, any
    false, fraudulent, or simulated certificate or other evidence of
    ownership of a share or shares of the capital stock of a bank,
    railroad, or other corporation.

    (b) No officer, director, or agent of a bank, railroad, or other
    corporation shall knowingly sign, with intent to issue, sell,
    pledge, or cause to be issued, sold, or pledged, any false,
    fraudulent, or simulated certificate or other evidence of the
    ownership or transfer of a share or shares of the capital stock of
    that corporation, or an instrument purporting to be a certificate or
    other evidence of the ownership or transfer, the signing, issuing,
    selling, or pledging of which by the officer, director, or agent is
    not authorized by law.

    (c) Sentence. A violation of this Section is a Class 3 felony.

    (Source: P.A. 96-1551, eff. 7-1-11 .)

    \hypertarget{ilcs-517-18}{%
    \subsection*{(720 ILCS 5/17-18)}\label{ilcs-517-18}}
    \addcontentsline{toc}{subsection}{(720 ILCS 5/17-18)}

    \hypertarget{sec.-17-18.-repealed.}{%
    \section*{Sec. 17-18. (Repealed).}\label{sec.-17-18.-repealed.}}
    \addcontentsline{toc}{section}{Sec. 17-18. (Repealed).}

    \markright{Sec. 17-18. (Repealed).}

    (Source: P.A. 89-234, eff. 1-1-96. Repealed by P.A. 96-1551, eff.
    7-1-11 .)

    \hypertarget{ilcs-517-19}{%
    \subsection*{(720 ILCS 5/17-19)}\label{ilcs-517-19}}
    \addcontentsline{toc}{subsection}{(720 ILCS 5/17-19)}

    \hypertarget{sec.-17-19.-repealed.}{%
    \section*{Sec. 17-19. (Repealed).}\label{sec.-17-19.-repealed.}}
    \addcontentsline{toc}{section}{Sec. 17-19. (Repealed).}

    \markright{Sec. 17-19. (Repealed).}

    (Source: P.A. 89-234, eff. 1-1-96. Repealed by P.A. 96-1551, eff.
    7-1-11 .)

    \hypertarget{ilcs-517-20}{%
    \subsection*{(720 ILCS 5/17-20)}\label{ilcs-517-20}}
    \addcontentsline{toc}{subsection}{(720 ILCS 5/17-20)}

    \hypertarget{sec.-17-20.-obstructing-gas-water-or-electric-current-meters.}{%
    \section*{Sec. 17-20. Obstructing gas, water, or electric current
    meters.}\label{sec.-17-20.-obstructing-gas-water-or-electric-current-meters.}}
    \addcontentsline{toc}{section}{Sec. 17-20. Obstructing gas, water,
    or electric current meters.}

    \markright{Sec. 17-20. Obstructing gas, water, or electric current
    meters.}

    A person commits obstructing gas, water, or electric current meters
    when he or she knowingly, and with intent to injure or defraud a
    company, body corporate, copartnership, or individual, injures,
    alters, obstructs, or prevents the action of a meter provided for
    the purpose of measuring and registering the quantity of gas, water,
    or electric current consumed by or at a burner, orifice, or place,
    or supplied to a lamp, motor, machine, or appliance, or causes,
    procures, or aids the injuring or altering of any such meter or the
    obstruction or prevention of its action, or makes or causes to be
    made with a gas pipe, water pipe, or electrical conductor any
    connection so as to conduct or supply illumination or inflammable
    gas, water, or electric current to any burner, orifice, lamp, motor,
    or other machine or appliance from which the gas, water, or
    electricity may be consumed or utilized without passing through or
    being registered by a meter or without the consent or acquiescence
    of the company, municipal corporation, body corporate,
    copartnership, or individual furnishing or transmitting the gas,
    water, or electric current through the gas pipe, water pipe, or
    electrical conductor. A violation of this Section is a Class B
    misdemeanor.

    (Source: P.A. 96-1551, eff. 7-1-11 .)

    \hypertarget{ilcs-517-21}{%
    \subsection*{(720 ILCS 5/17-21)}\label{ilcs-517-21}}
    \addcontentsline{toc}{subsection}{(720 ILCS 5/17-21)}

    \hypertarget{sec.-17-21.-obstructing-service-meters.}{%
    \section*{Sec. 17-21. Obstructing service
    meters.}\label{sec.-17-21.-obstructing-service-meters.}}
    \addcontentsline{toc}{section}{Sec. 17-21. Obstructing service
    meters.}

    \markright{Sec. 17-21. Obstructing service meters.}

    A person commits obstructing service meters when he or she
    knowingly, and, with the intent to defraud, tampers with, alters,
    obstructs or prevents the action of a meter, register, or other
    counting device that is a part of a mechanical or electrical
    machine, equipment, or device that measures service, without the
    consent of the owner of the machine, equipment, or device. A
    violation of this Section is a Class B misdemeanor.

    (Source: P.A. 96-1551, eff. 7-1-11 .)

    \hypertarget{ilcs-517-22}{%
    \subsection*{(720 ILCS 5/17-22)}\label{ilcs-517-22}}
    \addcontentsline{toc}{subsection}{(720 ILCS 5/17-22)}

    (This Section renumbered as Section 17-8.3 by P.A. 96-1551.)

    \hypertarget{sec.-17-22.-renumbered.}{%
    \section*{Sec. 17-22. (Renumbered).}\label{sec.-17-22.-renumbered.}}
    \addcontentsline{toc}{section}{Sec. 17-22. (Renumbered).}

    \markright{Sec. 17-22. (Renumbered).}

    (Source: P.A. 90-390, eff. 1-1-98. Renumbered by P.A. 96-1551, eff.
    7-1-11 .)

    \hypertarget{ilcs-517-23}{%
    \subsection*{(720 ILCS 5/17-23)}\label{ilcs-517-23}}
    \addcontentsline{toc}{subsection}{(720 ILCS 5/17-23)}

    \hypertarget{sec.-17-23.-repealed.}{%
    \section*{Sec. 17-23. (Repealed).}\label{sec.-17-23.-repealed.}}
    \addcontentsline{toc}{section}{Sec. 17-23. (Repealed).}

    \markright{Sec. 17-23. (Repealed).}

    (Source: P.A. 92-16, eff. 6-28-01. Repealed by P.A. 96-1551, eff.
    7-1-11 .)

    \hypertarget{ilcs-517-24}{%
    \subsection*{(720 ILCS 5/17-24)}\label{ilcs-517-24}}
    \addcontentsline{toc}{subsection}{(720 ILCS 5/17-24)}

    \hypertarget{sec.-17-24.-mail-fraud-and-wire-fraud.}{%
    \section*{Sec. 17-24. Mail fraud and wire
    fraud.}\label{sec.-17-24.-mail-fraud-and-wire-fraud.}}
    \addcontentsline{toc}{section}{Sec. 17-24. Mail fraud and wire
    fraud.}

    \markright{Sec. 17-24. Mail fraud and wire fraud.}

    (a) Mail fraud. A person commits mail fraud when he or she:

    (1) devises or intends to devise any scheme or artifice to defraud,
    or to obtain money or property by means of false or fraudulent
    pretenses, representations, or promises, or to sell, dispose of,
    loan, exchange, alter, give away, distribute, supply, or furnish or
    procure for unlawful use any counterfeit obligation, security, or
    other article, or anything represented to be or intimated or held
    out to be such a counterfeit or spurious article; and

    (2) with the intent to execute such scheme or artifice or to attempt
    to do so, does any of the following:

    (A) Places in any post office or authorized depository for mail
    matter within this State any matter or thing to be delivered by the
    United States Postal Service, according to the direction on the
    matter or thing.

    (B) Deposits or causes to be deposited in this

    State any matter or thing to be sent or delivered by mail or by
    private or commercial carrier, according to the direction on the
    matter or thing.

    (C) Takes or receives from mail or from a private or commercial
    carrier any such matter or thing at the place at which it is
    directed to be delivered by the person to whom it is addressed.

    (D) Knowingly causes any such matter or thing to be delivered by
    mail or by private or commercial carrier, according to the direction
    on the matter or thing.

    (b) Wire fraud. A person commits wire fraud when he or she:

    (1) devises or intends to devise a scheme or artifice to defraud or
    to obtain money or property by means of false pretenses,
    representations, or promises; and

    (2) for the purpose of executing the scheme or artifice, transmits
    or causes to be transmitted any writings, signals, pictures, sounds,
    or electronic or electric impulses by means of wire, radio, or
    television communications:

    (A) from within this State; or

    (B) so that the transmission is received by a person within this
    State; or

    (C) so that the transmission may be accessed by a person within this
    State.

    (c) Jurisdiction.

    (1) Mail fraud using a government or private carrier occurs in the
    county in which mail or other matter is deposited with the United
    States Postal Service or a private commercial carrier for delivery,
    if deposited with the United States Postal Service or a private or
    commercial carrier within this State, and the county in which a
    person within this State receives the mail or other matter from the
    United States Postal Service or a private or commercial carrier.

    (2) Wire fraud occurs in the county from which a transmission is
    sent, if the transmission is sent from within this State, the county
    in which a person within this State receives the transmission, and
    the county in which a person who is within this State is located
    when the person accesses a transmission.

    (d) Sentence. A violation of this Section is a Class 3 felony.

    The period of limitations for prosecution of any offense defined in
    this Section begins at the time when the last act in furtherance of
    the scheme or artifice is committed.

    (Source: P.A. 96-1000, eff. 7-2-10; 96-1551, eff. 7-1-11 .)

    \hypertarget{ilcs-517-25}{%
    \subsection*{(720 ILCS 5/17-25)}\label{ilcs-517-25}}
    \addcontentsline{toc}{subsection}{(720 ILCS 5/17-25)}

    \hypertarget{sec.-17-25.-repealed.}{%
    \section*{Sec. 17-25. (Repealed).}\label{sec.-17-25.-repealed.}}
    \addcontentsline{toc}{section}{Sec. 17-25. (Repealed).}

    \markright{Sec. 17-25. (Repealed).}

    (Source: P.A. 92-818, eff. 8-21-02. Repealed by P.A. 97-597, eff.
    1-1-12.)

    \hypertarget{ilcs-517-26}{%
    \subsection*{(720 ILCS 5/17-26)}\label{ilcs-517-26}}
    \addcontentsline{toc}{subsection}{(720 ILCS 5/17-26)}

    \hypertarget{sec.-17-26.-misconduct-by-a-corporate-official.}{%
    \section*{Sec. 17-26. Misconduct by a corporate
    official.}\label{sec.-17-26.-misconduct-by-a-corporate-official.}}
    \addcontentsline{toc}{section}{Sec. 17-26. Misconduct by a corporate
    official.}

    \markright{Sec. 17-26. Misconduct by a corporate official.}

    (a) A person commits misconduct by a corporate official when:

    (1) being a director of a corporation, he or she knowingly, with the
    intent to defraud, concurs in any vote or act of the directors of
    the corporation, or any of them, which has the purpose of:

    (A) making a dividend except in the manner provided by law;

    (B) dividing, withdrawing or in any manner paying any stockholder
    any part of the capital stock of the corporation except in the
    manner provided by law;

    (C) discounting or receiving any note or other evidence of debt in
    payment of an installment of capital stock actually called in and
    required to be paid, or with purpose of providing the means of
    making such payment;

    (D) receiving or discounting any note or other evidence of debt with
    the purpose of enabling any stockholder to withdraw any part of the
    money paid in by him or her on his or her stock; or

    (E) applying any portion of the funds of such corporation, directly
    or indirectly, to the purchase of shares of its own stock, except in
    the manner provided by law; or

    (2) being a director or officer of a corporation, he or she, with
    the intent to defraud:

    (A) issues, participates in issuing, or concurs in a vote to issue
    any increase of its capital stock beyond the amount of the capital
    stock thereof, duly authorized by or in pursuance of law;

    (B) sells, or agrees to sell, or is directly interested in the sale
    of any share of stock of such corporation, or in any agreement to
    sell such stock, unless at the time of the sale or agreement he or
    she is an actual owner of such share, provided that the foregoing
    shall not apply to a sale by or on behalf of an underwriter or
    dealer in connection with a bona fide public offering of shares of
    stock of such corporation;

    (C) executes a scheme or attempts to execute a scheme to obtain any
    share of stock of such corporation by means of false representation;
    or

    (3) being a director or officer of a corporation, he or she with the
    intent to defraud or evade a financial disclosure reporting
    requirement of this State or of Section 13(A) or 15(D) of the
    Securities Exchange Act of 1934, as amended, 15 U. S. C. 78M(A) or
    78O(D):

    (A) causes or attempts to cause a corporation or accounting firm
    representing the corporation or any other individual or entity to
    fail to file a financial disclosure report as required by State or
    federal law; or

    (B) causes or attempts to cause a corporation or accounting firm
    representing the corporation or any other individual or entity to
    file a financial disclosure report, as required by State or federal
    law, that contains a material omission or misstatement of fact.

    (b) Sentence. If the benefit derived from a violation of this
    Section is \$500,000 or more, the violation is a Class 2 felony. If
    the benefit derived from a violation of this Section is less than
    \$500,000, the violation is a Class 3 felony.

    (Source: P.A. 96-1000, eff. 7-2-10; 96-1551, eff. 7-1-11 .)

    \hypertarget{ilcs-517-27}{%
    \subsection*{(720 ILCS 5/17-27)}\label{ilcs-517-27}}
    \addcontentsline{toc}{subsection}{(720 ILCS 5/17-27)}

    \hypertarget{sec.-17-27.-fraud-on-creditors.}{%
    \section*{Sec. 17-27. Fraud on
    creditors.}\label{sec.-17-27.-fraud-on-creditors.}}
    \addcontentsline{toc}{section}{Sec. 17-27. Fraud on creditors.}

    \markright{Sec. 17-27. Fraud on creditors.}

    (a) Fraud in insolvency. A person commits fraud in insolvency when,
    knowing that proceedings have or are about to be instituted for the
    appointment of a receiver or other person entitled to administer
    property for the benefit of creditors, or that any other composition
    or liquidation for the benefit of creditors has been or is about to
    be made, he or she:

    (1) destroys, removes, conceals, encumbers, transfers, or otherwise
    deals with any property or obtains any substantial part of or
    interest in the debtor's estate with the intent to defeat or
    obstruct the claim of any creditor, or otherwise to obstruct the
    operation of any law relating to administration of property for the
    benefit of creditors;

    (2) knowingly falsifies any writing or record relating to the
    property; or

    (3) knowingly misrepresents or refuses to disclose to a receiver or
    other person entitled to administer property for the benefit of
    creditors, the existence, amount, or location of the property, or
    any other information which the actor could be legally required to
    furnish in relation to such administration.

    Sentence. If the benefit derived from a violation of this subsection
    (a) is \$500,000 or more, the violation is a Class 2 felony. If the
    benefit derived from a violation of this subsection (a) is less than
    \$500,000, the violation is a Class 3 felony.

    (b) Fraud in property transfer. A person commits fraud in property
    transfer when he or she transfers or conveys any interest in
    property with the intent to defraud, defeat, hinder, or delay his or
    her creditors. A violation of this subsection (b) is a business
    offense subject to a fine not to exceed \$1,000.

    (Source: P.A. 96-1551, eff. 7-1-11 .)

    \hypertarget{ilcs-517-28}{%
    \subsection*{(720 ILCS 5/17-28)}\label{ilcs-517-28}}
    \addcontentsline{toc}{subsection}{(720 ILCS 5/17-28)}

    (This Section was renumbered as Section 17-57 by P.A. 96-1551.)

    \hypertarget{sec.-17-28.-renumbered.}{%
    \section*{Sec. 17-28. (Renumbered).}\label{sec.-17-28.-renumbered.}}
    \addcontentsline{toc}{section}{Sec. 17-28. (Renumbered).}

    \markright{Sec. 17-28. (Renumbered).}

    (Source: P.A. 93-691, eff. 7-9-04. Renumbered by P.A. 96-1551, eff.
    7-1-11 .)

    \hypertarget{ilcs-517-29}{%
    \subsection*{(720 ILCS 5/17-29)}\label{ilcs-517-29}}
    \addcontentsline{toc}{subsection}{(720 ILCS 5/17-29)}

    (This Section was renumbered as Section 17-10.2 by P.A. 96-1551.)

    \hypertarget{sec.-17-29.-renumbered.}{%
    \section*{Sec. 17-29. (Renumbered).}\label{sec.-17-29.-renumbered.}}
    \addcontentsline{toc}{section}{Sec. 17-29. (Renumbered).}

    \markright{Sec. 17-29. (Renumbered).}

    (Source: P.A. 97-396, eff. 1-1-12. Renumbered by P.A. 96-1551, eff.
    7-1-11.)

    \hypertarget{ilcs-517-30}{%
    \subsection*{(720 ILCS 5/17-30)}\label{ilcs-517-30}}
    \addcontentsline{toc}{subsection}{(720 ILCS 5/17-30)}

    (was 720 ILCS 5/16C-2)

    \hypertarget{sec.-17-30.-defaced-altered-or-removed-manufacturer-or-owner-identification-number.}{%
    \section*{Sec. 17-30. Defaced, altered, or removed manufacturer or
    owner identification
    number.}\label{sec.-17-30.-defaced-altered-or-removed-manufacturer-or-owner-identification-number.}}
    \addcontentsline{toc}{section}{Sec. 17-30. Defaced, altered, or
    removed manufacturer or owner identification number.}

    \markright{Sec. 17-30. Defaced, altered, or removed manufacturer or
    owner identification number.}

    (a) Unlawful sale of household appliances. A person commits unlawful
    sale of household appliances when he or she knowingly, with the
    intent to defraud or deceive another, keeps for sale, within any
    commercial context, any household appliance with a missing, defaced,
    obliterated, or otherwise altered manufacturer's identification
    number.

    (b) Construction equipment identification defacement. A person
    commits construction equipment identification defacement when he or
    she knowingly changes, alters, removes, mutilates, or obliterates a
    permanently affixed serial number, product identification number,
    part number, component identification number, owner-applied
    identification, or other mark of identification attached to or
    stamped, inscribed, molded, or etched into a machine or other
    equipment, whether stationary or mobile or self-propelled, or a part
    of such machine or equipment, used in the construction, maintenance,
    or demolition of buildings, structures, bridges, tunnels, sewers,
    utility pipes or lines, ditches or open cuts, roads, highways, dams,
    airports, or waterways or in material handling for such projects.

    The trier of fact may infer that the defendant has knowingly
    changed, altered, removed, or obliterated the serial number, product
    identification number, part number, component identification number,
    owner-applied identification number, or other mark of
    identification, if the defendant was in possession of any machine or
    other equipment or a part of such machine or equipment used in the
    construction, maintenance, or demolition of buildings, structures,
    bridges, tunnels, sewers, utility pipes or lines, ditches or open
    cuts, roads, highways, dams, airports, or waterways or in material
    handling for such projects upon which any such serial number,
    product identification number, part number, component identification
    number, owner-applied identification number, or other mark of
    identification has been changed, altered, removed, or obliterated.

    (c) Defacement of manufacturer's serial number or identification
    mark. A person commits defacement of a manufacturer's serial number
    or identification mark when he or she knowingly removes, alters,
    defaces, covers, or destroys the manufacturer's serial number or any
    other manufacturer's number or distinguishing identification mark
    upon any machine or other article of merchandise, other than a motor
    vehicle as defined in Section 1-146 of the Illinois Vehicle Code or
    a firearm as defined in the Firearm Owners Identification Card Act,
    with the intent of concealing or destroying the identity of such
    machine or other article of merchandise.

    (d) Sentence.

    (1) A violation of subsection (a) of this Section is a Class 4
    felony if the value of the appliance or appliances exceeds \$1,000
    and a Class B misdemeanor if the value of the appliance or
    appliances is \$1,000 or less.

    (2) A violation of subsection (b) of this Section is a Class A
    misdemeanor.

    (3) A violation of subsection (c) of this Section is a Class B
    misdemeanor.

    (e) No liability shall be imposed upon any person for the
    unintentional failure to comply with subsection (a).

    (f) Definitions. In this Section:

    ``Commercial context'' means a continuing business enterprise
    conducted for profit by any person whose primary business is the
    wholesale or retail marketing of household appliances, or a
    significant portion of whose business or inventory consists of
    household appliances kept or sold on a wholesale or retail basis.

    ``Household appliance'' means any gas or electric device or machine
    marketed for use as home entertainment or for facilitating or
    expediting household tasks or chores. The term shall include but not
    necessarily be limited to refrigerators, freezers, ranges, radios,
    television sets, vacuum cleaners, toasters, dishwashers, and other
    similar household items.

    ``Manufacturer's identification number'' means any serial number or
    other similar numerical or alphabetical designation imprinted upon
    or attached to or placed, stamped, or otherwise imprinted upon or
    attached to a household appliance or item by the manufacturer for
    purposes of identifying a particular appliance or item individually
    or by lot number.

    (Source: P.A. 96-1551, eff. 7-1-11 .)

    \hypertarget{ilcs-5art.-17-subdiv.-25-heading}{%
    \subsection*{(720 ILCS 5/Art. 17, Subdiv. 25
    heading)}\label{ilcs-5art.-17-subdiv.-25-heading}}
    \addcontentsline{toc}{subsection}{(720 ILCS 5/Art. 17, Subdiv. 25
    heading)}

    SUBDIVISION 25.

    CREDIT AND DEBIT CARD FRAUD

    (Source: P.A. 96-1551, eff. 7-1-11.)

    \hypertarget{ilcs-517-31}{%
    \subsection*{(720 ILCS 5/17-31)}\label{ilcs-517-31}}
    \addcontentsline{toc}{subsection}{(720 ILCS 5/17-31)}

    \hypertarget{sec.-17-31.-false-statement-to-procure-credit-or-debit-card.}{%
    \section*{Sec. 17-31. False statement to procure credit or debit
    card.}\label{sec.-17-31.-false-statement-to-procure-credit-or-debit-card.}}
    \addcontentsline{toc}{section}{Sec. 17-31. False statement to
    procure credit or debit card.}

    \markright{Sec. 17-31. False statement to procure credit or debit
    card.}

    A person commits false statement to procure credit or debit card
    when he or she makes or causes to be made, either directly or
    indirectly, any false statement in writing, knowing it to be false
    and with the intent that it be relied on, respecting his or her
    identity, his or her address, or his or her employment, or that of
    any other person, firm, or corporation, with the intent to procure
    the issuance of a credit card or debit card. A violation of this
    Section is a Class 4 felony.

    (Source: P.A. 96-1551, eff. 7-1-11 .)

    \hypertarget{ilcs-517-32}{%
    \subsection*{(720 ILCS 5/17-32)}\label{ilcs-517-32}}
    \addcontentsline{toc}{subsection}{(720 ILCS 5/17-32)}

    \hypertarget{sec.-17-32.-possession-of-anothers-credit-debit-or-identification-card.}{%
    \section*{Sec. 17-32. Possession of another's credit, debit, or
    identification
    card.}\label{sec.-17-32.-possession-of-anothers-credit-debit-or-identification-card.}}
    \addcontentsline{toc}{section}{Sec. 17-32. Possession of another's
    credit, debit, or identification card.}

    \markright{Sec. 17-32. Possession of another's credit, debit, or
    identification card.}

    (a) Possession of another's identification card. A person commits
    possession of another's identification card when he or she, with the
    intent to defraud, possesses any check guarantee card or key card or
    identification card for cash dispensing machines without the
    authority of the account holder or financial institution.

    (b) Possession of another's credit or debit card. A person commits
    possession of another's credit or debit card when he or she receives
    a credit card or debit card from the person, possession, custody, or
    control of another without the cardholder's consent or if he or she,
    with knowledge that it has been so acquired, receives the credit
    card or debit card with the intent to use it or to sell it, or to
    transfer it to a person other than the issuer or the cardholder. The
    trier of fact may infer that a person who has in his or her
    possession or under his or her control 2 or more such credit cards
    or debit cards each issued to a cardholder other than himself or
    herself has violated this Section.

    (c) Sentence.

    (1) A violation of subsection (a) of this Section is a Class A
    misdemeanor. A person who, within any 12-month period, violates
    subsection (a) of this Section at the same time or consecutively
    with respect to 3 or more cards, each the property of different
    account holders, is guilty of a Class 4 felony. A person convicted
    under subsection (a) of this Section, when the value of property so
    obtained, in a single transaction or in separate transactions within
    any 90-day period, exceeds \$150 is guilty of a Class 4 felony.

    (2) A violation of subsection (b) of this Section is a Class 4
    felony. A person who, in any 12-month period, violates subsection
    (b) of this Section with respect to 3 or more credit cards or debit
    cards each issued to a cardholder other than himself or herself is
    guilty of a Class 3 felony.

    (Source: P.A. 96-1551, eff. 7-1-11 .)

    \hypertarget{ilcs-517-33}{%
    \subsection*{(720 ILCS 5/17-33)}\label{ilcs-517-33}}
    \addcontentsline{toc}{subsection}{(720 ILCS 5/17-33)}

    \hypertarget{sec.-17-33.-possession-of-lost-or-mislaid-credit-or-debit-card.}{%
    \section*{Sec. 17-33. Possession of lost or mislaid credit or debit
    card.}\label{sec.-17-33.-possession-of-lost-or-mislaid-credit-or-debit-card.}}
    \addcontentsline{toc}{section}{Sec. 17-33. Possession of lost or
    mislaid credit or debit card.}

    \markright{Sec. 17-33. Possession of lost or mislaid credit or debit
    card.}

    A person who receives a credit card or debit card that he or she
    knows to have been lost or mislaid and who retains possession with
    intent to use it or to sell it or to transfer it to a person other
    than the issuer or the cardholder is guilty of a Class 4 felony.

    A person who, in a single transaction, violates this Section with
    respect to 3 or more credit cards or debit cards each issued to
    different cardholders other than himself or herself is guilty of a
    Class 3 felony.

    (Source: P.A. 96-1551, eff. 7-1-11 .)

    \hypertarget{ilcs-517-34}{%
    \subsection*{(720 ILCS 5/17-34)}\label{ilcs-517-34}}
    \addcontentsline{toc}{subsection}{(720 ILCS 5/17-34)}

    \hypertarget{sec.-17-34.-sale-of-credit-or-debit-card.}{%
    \section*{Sec. 17-34. Sale of credit or debit
    card.}\label{sec.-17-34.-sale-of-credit-or-debit-card.}}
    \addcontentsline{toc}{section}{Sec. 17-34. Sale of credit or debit
    card.}

    \markright{Sec. 17-34. Sale of credit or debit card.}

    A person other than the issuer who sells a credit card or debit
    card, without the consent of the issuer, is guilty of a Class 4
    felony.

    A person who knowingly purchases a credit card or debit card from a
    person other than the issuer, without the consent of the issuer, is
    guilty of a Class 4 felony.

    A person who, in a single transaction, makes a sale or purchase
    prohibited by this Section with respect to 3 or more credit cards or
    debit cards each issued to a cardholder other than himself or
    herself is guilty of a Class 3 felony.

    (Source: P.A. 96-1551, eff. 7-1-11 .)

    \hypertarget{ilcs-517-35}{%
    \subsection*{(720 ILCS 5/17-35)}\label{ilcs-517-35}}
    \addcontentsline{toc}{subsection}{(720 ILCS 5/17-35)}

    \hypertarget{sec.-17-35.-use-of-credit-or-debit-card-as-security-for-debt.}{%
    \section*{Sec. 17-35. Use of credit or debit card as security for
    debt.}\label{sec.-17-35.-use-of-credit-or-debit-card-as-security-for-debt.}}
    \addcontentsline{toc}{section}{Sec. 17-35. Use of credit or debit
    card as security for debt.}

    \markright{Sec. 17-35. Use of credit or debit card as security for
    debt.}

    A person who, with intent to defraud either the issuer, or a person
    providing an item or items of value, or any other person, obtains
    control over a credit card or debit card as security for debt or
    transfers, conveys, or gives control over a credit card or debit
    card as security for debt is guilty of a Class 4 felony.

    (Source: P.A. 96-1551, eff. 7-1-11 .)

    \hypertarget{ilcs-517-36}{%
    \subsection*{(720 ILCS 5/17-36)}\label{ilcs-517-36}}
    \addcontentsline{toc}{subsection}{(720 ILCS 5/17-36)}

    \hypertarget{sec.-17-36.-use-of-counterfeited-forged-expired-revoked-or-unissued-credit-or-debit-card.}{%
    \section*{Sec. 17-36. Use of counterfeited, forged, expired,
    revoked, or unissued credit or debit
    card.}\label{sec.-17-36.-use-of-counterfeited-forged-expired-revoked-or-unissued-credit-or-debit-card.}}
    \addcontentsline{toc}{section}{Sec. 17-36. Use of counterfeited,
    forged, expired, revoked, or unissued credit or debit card.}

    \markright{Sec. 17-36. Use of counterfeited, forged, expired,
    revoked, or unissued credit or debit card.}

    A person who, with intent to defraud either the issuer, or a person
    providing an item or items of value, or any other person, (i) uses,
    with the intent to obtain an item or items of value, a credit card
    or debit card obtained or retained in violation of this Subdivision
    25 or without the cardholder's consent, or a credit card or debit
    card which he or she knows is counterfeited, or forged, or expired,
    or revoked or (ii) obtains or attempts to obtain an item or items of
    value by representing without the consent of the cardholder that he
    or she is the holder of a specified card or by representing that he
    or she is the holder of a card and such card has not in fact been
    issued is guilty of a Class 4 felony if the value of all items of
    value obtained or sought in violation of this Section does not
    exceed \$300 in any 6-month period; and is guilty of a Class 3
    felony if the value exceeds \$300 in any 6-month period. The trier
    of fact may infer that knowledge of revocation has been received by
    a cardholder 4 days after it has been mailed to him or her at the
    address set forth on the credit card or debit card or at his or her
    last known address by registered or certified mail, return receipt
    requested, and, if the address is more than 500 miles from the place
    of mailing, by air mail. The trier of fact may infer that notice was
    received 10 days after mailing by registered or certified mail if
    the address is located outside the United States, Puerto Rico, the
    Virgin Islands, the Canal Zone, and Canada.

    (Source: P.A. 96-1551, eff. 7-1-11 .)

    \hypertarget{ilcs-517-37}{%
    \subsection*{(720 ILCS 5/17-37)}\label{ilcs-517-37}}
    \addcontentsline{toc}{subsection}{(720 ILCS 5/17-37)}

    \hypertarget{sec.-17-37.-use-of-credit-or-debit-card-with-intent-to-defraud.}{%
    \section*{Sec. 17-37. Use of credit or debit card with intent to
    defraud.}\label{sec.-17-37.-use-of-credit-or-debit-card-with-intent-to-defraud.}}
    \addcontentsline{toc}{section}{Sec. 17-37. Use of credit or debit
    card with intent to defraud.}

    \markright{Sec. 17-37. Use of credit or debit card with intent to
    defraud.}

    (a) A cardholder who uses a credit card or debit card issued to him
    or her, or allows another person to use a credit card or debit card
    issued to him or her, with intent to defraud the issuer, or a person
    providing an item or items of value, or any other person is guilty
    of a Class A misdemeanor if the value of all items of value does not
    exceed \$150 in any 6-month period; and is guilty of a Class 4
    felony if the value exceeds \$150 in any 6-month period.

    (b) Where an investigation into an intent to defraud under
    subsection (a) occurs, issuers shall consider a merchant's timely
    submission of compelling evidence under the applicable dispute
    management guidelines of the card association with whom the merchant
    maintains an agreement. A merchant shall comply with merchant
    responsibilities under any such agreement.

    (Source: P.A. 102-757, eff. 5-13-22.)

    \hypertarget{ilcs-517-38}{%
    \subsection*{(720 ILCS 5/17-38)}\label{ilcs-517-38}}
    \addcontentsline{toc}{subsection}{(720 ILCS 5/17-38)}

    \hypertarget{sec.-17-38.-use-of-account-number-or-code-with-intent-to-defraud-possession-of-record-of-charge-forms.}{%
    \section*{Sec. 17-38. Use of account number or code with intent to
    defraud; possession of record of charge
    forms.}\label{sec.-17-38.-use-of-account-number-or-code-with-intent-to-defraud-possession-of-record-of-charge-forms.}}
    \addcontentsline{toc}{section}{Sec. 17-38. Use of account number or
    code with intent to defraud; possession of record of charge forms.}

    \markright{Sec. 17-38. Use of account number or code with intent to
    defraud; possession of record of charge forms.}

    (a) A person who, with intent to defraud either an issuer, or a
    person providing an item or items of value, or any other person,
    utilizes an account number or code or enters information on a record
    of charge form with the intent to obtain an item or items of value
    is guilty of a Class 4 felony if the value of the item or items of
    value obtained does not exceed \$150 in any 6-month period; and is
    guilty of a Class 3 felony if the value exceeds \$150 in any 6-month
    period.

    (b) A person who, with intent to defraud either an issuer or a
    person providing an item or items of value, or any other person,
    possesses, without the consent of the issuer or purported issuer,
    record of charge forms bearing the printed impression of a credit
    card or debit card is guilty of a Class 4 felony. The trier of fact
    may infer intent to defraud from the possession of such record of
    charge forms by a person other than the issuer or a person
    authorized by the issuer to possess record of charge forms.

    (Source: P.A. 96-1551, eff. 7-1-11 .)

    \hypertarget{ilcs-517-39}{%
    \subsection*{(720 ILCS 5/17-39)}\label{ilcs-517-39}}
    \addcontentsline{toc}{subsection}{(720 ILCS 5/17-39)}

    \hypertarget{sec.-17-39.-receipt-of-goods-or-services.}{%
    \section*{Sec. 17-39. Receipt of goods or
    services.}\label{sec.-17-39.-receipt-of-goods-or-services.}}
    \addcontentsline{toc}{section}{Sec. 17-39. Receipt of goods or
    services.}

    \markright{Sec. 17-39. Receipt of goods or services.}

    A person who receives an item or items of value obtained in
    violation of this Subdivision 25, knowing that it was so obtained or
    under such circumstances as would reasonably induce him or her to
    believe that it was so obtained, is guilty of a Class A misdemeanor
    if the value of all items of value obtained does not exceed \$150 in
    any 6-month period; and is guilty of a Class 4 felony if the value
    exceeds \$150 in any 6-month period.

    (Source: P.A. 96-1551, eff. 7-1-11 .)

    \hypertarget{ilcs-517-40}{%
    \subsection*{(720 ILCS 5/17-40)}\label{ilcs-517-40}}
    \addcontentsline{toc}{subsection}{(720 ILCS 5/17-40)}

    \hypertarget{sec.-17-40.-signing-anothers-card-with-intent-to-defraud.}{%
    \section*{Sec. 17-40. Signing another's card with intent to
    defraud.}\label{sec.-17-40.-signing-anothers-card-with-intent-to-defraud.}}
    \addcontentsline{toc}{section}{Sec. 17-40. Signing another's card
    with intent to defraud.}

    \markright{Sec. 17-40. Signing another's card with intent to
    defraud.}

    A person other than the cardholder or a person authorized by him or
    her who, with intent to defraud either the issuer, or a person
    providing an item or items of value, or any other person, signs a
    credit card or debit card is guilty of a Class A misdemeanor.

    (Source: P.A. 96-1551, eff. 7-1-11 .)

    \hypertarget{ilcs-517-41}{%
    \subsection*{(720 ILCS 5/17-41)}\label{ilcs-517-41}}
    \addcontentsline{toc}{subsection}{(720 ILCS 5/17-41)}

    \hypertarget{sec.-17-41.-altered-or-counterfeited-card.}{%
    \section*{Sec. 17-41. Altered or counterfeited
    card.}\label{sec.-17-41.-altered-or-counterfeited-card.}}
    \addcontentsline{toc}{section}{Sec. 17-41. Altered or counterfeited
    card.}

    \markright{Sec. 17-41. Altered or counterfeited card.}

    (a) A person commits an offense under this Section when he or she,
    with intent to defraud either a purported issuer, or a person
    providing an item or items of value, or any other person, commits an
    offense under this Section if he or she: (i) alters a credit card or
    debit card or a purported credit card or debit card, or possesses a
    credit card or debit card or a purported credit card or debit card
    with knowledge that the same has been altered; or (ii) counterfeits
    a purported credit card or debit card, or possesses a purported
    credit card or debit card with knowledge that the card has been
    counterfeited.

    (b) Sentence. A violation of item (i) of subsection (a) is a Class 4
    felony. A violation of item (ii) of subsection (a) is a Class 3
    felony. The trier of fact may infer that possession of 2 or more
    credit cards or debit cards by a person other than the issuer in
    violation of subsection (a) is evidence that the person intended to
    defraud or that he or she knew the credit cards or debit cards to
    have been so altered or counterfeited.

    (Source: P.A. 96-1551, eff. 7-1-11 .)

    \hypertarget{ilcs-517-42}{%
    \subsection*{(720 ILCS 5/17-42)}\label{ilcs-517-42}}
    \addcontentsline{toc}{subsection}{(720 ILCS 5/17-42)}

    \hypertarget{sec.-17-42.-possession-of-incomplete-card.}{%
    \section*{Sec. 17-42. Possession of incomplete
    card.}\label{sec.-17-42.-possession-of-incomplete-card.}}
    \addcontentsline{toc}{section}{Sec. 17-42. Possession of incomplete
    card.}

    \markright{Sec. 17-42. Possession of incomplete card.}

    A person other than the cardholder possessing an incomplete credit
    card or debit card, with intent to complete it without the consent
    of the issuer or a person possessing, with knowledge of its
    character, machinery, plates, or any other contrivance designed to
    reproduce instruments purporting to be credit cards or debit cards
    of an issuer who has not consented to the preparation of such credit
    cards or debit cards is guilty of a Class 3 felony. The trier of
    fact may infer that a person other than the cardholder or issuer who
    possesses 2 or more incomplete credit cards or debit cards possesses
    those cards without the consent of the issuer.

    (Source: P.A. 96-1551, eff. 7-1-11 .)

    \hypertarget{ilcs-517-43}{%
    \subsection*{(720 ILCS 5/17-43)}\label{ilcs-517-43}}
    \addcontentsline{toc}{subsection}{(720 ILCS 5/17-43)}

    \hypertarget{sec.-17-43.-prohibited-deposits.}{%
    \section*{Sec. 17-43. Prohibited
    deposits.}\label{sec.-17-43.-prohibited-deposits.}}
    \addcontentsline{toc}{section}{Sec. 17-43. Prohibited deposits.}

    \markright{Sec. 17-43. Prohibited deposits.}

    (a) A person who, with intent to defraud the issuer of a credit card
    or debit card or any person providing an item or items of value, or
    any other person, deposits into his or her account or any account,
    via an electronic fund transfer terminal, a check, draft, money
    order, or other such document, knowing such document to be false,
    fictitious, forged, altered, counterfeit, or not his or her lawful
    or legal property, is guilty of a Class 4 felony.

    (b) A person who receives value as a result of a false, fictitious,
    forged, altered, or counterfeit check, draft, money order, or other
    such document having been deposited into an account via an
    electronic fund transfer terminal, knowing at the time of receipt of
    the value that the document so deposited was false, fictitious,
    forged, altered, counterfeit, or not his or her lawful or legal
    property, is guilty of a Class 4 felony.

    (Source: P.A. 96-1551, eff. 7-1-11 .)

    \hypertarget{ilcs-517-44}{%
    \subsection*{(720 ILCS 5/17-44)}\label{ilcs-517-44}}
    \addcontentsline{toc}{subsection}{(720 ILCS 5/17-44)}

    \hypertarget{sec.-17-44.-fraudulent-use-of-electronic-transmission.}{%
    \section*{Sec. 17-44. Fraudulent use of electronic
    transmission.}\label{sec.-17-44.-fraudulent-use-of-electronic-transmission.}}
    \addcontentsline{toc}{section}{Sec. 17-44. Fraudulent use of
    electronic transmission.}

    \markright{Sec. 17-44. Fraudulent use of electronic transmission.}

    (a) A person who, with intent to defraud the issuer of a credit card
    or debit card, the cardholder, or any other person, intercepts,
    taps, or alters electronic information between an electronic fund
    transfer terminal and the issuer, or originates electronic
    information to an electronic fund transfer terminal or to the
    issuer, via any line, wire, or other means of electronic
    transmission, at any junction, terminal, or device, or at any
    location within the EFT System, with the intent to obtain value, is
    guilty of a Class 4 felony.

    (b) Any person who, with intent to defraud the issuer of a credit
    card or debit card, the cardholder, or any other person, intercepts,
    taps, or alters electronic information between an electronic fund
    transfer terminal and the issuer, or originates electronic
    information to an electronic fund transfer terminal or to the
    issuer, via any line, wire, or other means of electronic
    transmission, at any junction, terminal, or device, or at any
    location within the EFT System, and thereby causes funds to be
    transferred from one account to any other account, is guilty of a
    Class 4 felony.

    (Source: P.A. 96-1551, eff. 7-1-11 .)

    \hypertarget{ilcs-517-45}{%
    \subsection*{(720 ILCS 5/17-45)}\label{ilcs-517-45}}
    \addcontentsline{toc}{subsection}{(720 ILCS 5/17-45)}

    \hypertarget{sec.-17-45.-payment-of-charges-without-furnishing-item-of-value.}{%
    \section*{Sec. 17-45. Payment of charges without furnishing item of
    value.}\label{sec.-17-45.-payment-of-charges-without-furnishing-item-of-value.}}
    \addcontentsline{toc}{section}{Sec. 17-45. Payment of charges
    without furnishing item of value.}

    \markright{Sec. 17-45. Payment of charges without furnishing item of
    value.}

    (a) No person shall process, deposit, negotiate, or obtain payment
    of a credit card charge through a retail seller's account with a
    financial institution or through a retail seller's agreement with a
    financial institution, card issuer, or organization of financial
    institutions or card issuers if that retail seller did not furnish
    or agree to furnish the item or items of value that are the subject
    of the credit card charge.

    (b) No retail seller shall permit any person to process, deposit,
    negotiate, or obtain payment of a credit card charge through the
    retail seller's account with a financial institution or the retail
    seller's agreement with a financial institution, card issuer, or
    organization of financial institutions or card issuers if that
    retail seller did not furnish or agree to furnish the item or items
    of value that are the subject of the credit card charge.

    (c) Subsections (a) and (b) do not apply to any of the following:

    (1) A person who furnishes goods or services on the business
    premises of a general merchandise retail seller and who processes,
    deposits, negotiates, or obtains payment of a credit card charge
    through that general merchandise retail seller's account or
    agreement.

    (2) A general merchandise retail seller who permits a person
    described in paragraph (1) to process, deposit, negotiate, or obtain
    payment of a credit card charge through that general merchandise
    retail seller's account or agreement.

    (3) A franchisee who furnishes the cardholder with an item or items
    of value that are provided in whole or in part by the franchisor and
    who processes, deposits, negotiates, or obtains payment of a credit
    card charge through that franchisor's account or agreement.

    (4) A franchisor who permits a franchisee described in paragraph (3)
    to process, deposit, negotiate, or obtain payment of a credit card
    charge through that franchisor's account or agreement.

    (5) The credit card issuer or a financial institution or a parent,
    subsidiary, or affiliate of the card issuer or a financial
    institution.

    (6) A person who processes, deposits, negotiates, or obtains payment
    of less than \$500 of credit card charges in any one-year period
    through a retail seller's account or agreement. The person has the
    burden of producing evidence that the person transacted less than
    \$500 in credit card charges during any one-year period.

    (7) A telecommunications carrier that includes charges of other
    parties in its billings to its subscribers and those other parties
    whose charges are included in the billings of the telecommunications
    carrier to its subscribers.

    (d) A person injured by a violation of this Section may bring an
    action for the recovery of damages, equitable relief, and reasonable
    attorney's fees and costs.

    (e) A person who violates this Section is guilty of a business
    offense and shall be fined \$10,000 for each offense. Each
    occurrence in which a person processes, deposits, negotiates, or
    otherwise seeks to obtain payment of a credit card charge in
    violation of subsection (a) constitutes a separate offense.

    (f) The penalties and remedies provided in this Section are in
    addition to any other remedies or penalties provided by law.

    (g) As used in this Section:

    ``Franchisor'' and ``franchisee'' have the same meanings as in
    Section 3 of the Franchise Disclosure Act of 1987.

    ``Retail seller'' has the same meaning as in Section 2.4 of the
    Retail Installment Sales Act.

    ``Telecommunications carrier'' has the same meaning as in Section
    13-202 of the Public Utilities Act.

    (Source: P.A. 96-1551, eff. 7-1-11 .)

    \hypertarget{ilcs-517-46}{%
    \subsection*{(720 ILCS 5/17-46)}\label{ilcs-517-46}}
    \addcontentsline{toc}{subsection}{(720 ILCS 5/17-46)}

    \hypertarget{sec.-17-46.-furnishing-items-of-value-with-intent-to-defraud.}{%
    \section*{Sec. 17-46. Furnishing items of value with intent to
    defraud.}\label{sec.-17-46.-furnishing-items-of-value-with-intent-to-defraud.}}
    \addcontentsline{toc}{section}{Sec. 17-46. Furnishing items of value
    with intent to defraud.}

    \markright{Sec. 17-46. Furnishing items of value with intent to
    defraud.}

    A person who is authorized by an issuer to furnish money, goods,
    property, services or anything else of value upon presentation of a
    credit card or debit card by the cardholder, or any agent or
    employee of such person, who, with intent to defraud the issuer or
    the cardholder, furnishes money, goods, property, services or
    anything else of value upon presentation of a credit card or debit
    card obtained or retained in violation of this Code or a credit card
    or debit card which he knows is counterfeited, or forged, or
    expired, or revoked is guilty of a Class A misdemeanor, if the value
    furnished in violation of this Section does not exceed \$150 in any
    6-month period; and is guilty of a Class 4 felony if such value
    exceeds \$150 in any 6-month period.

    (Source: P.A. 96-1551, eff. 7-1-11 .)

    \hypertarget{ilcs-517-47}{%
    \subsection*{(720 ILCS 5/17-47)}\label{ilcs-517-47}}
    \addcontentsline{toc}{subsection}{(720 ILCS 5/17-47)}

    \hypertarget{sec.-17-47.-failure-to-furnish-items-of-value.}{%
    \section*{Sec. 17-47. Failure to furnish items of
    value.}\label{sec.-17-47.-failure-to-furnish-items-of-value.}}
    \addcontentsline{toc}{section}{Sec. 17-47. Failure to furnish items
    of value.}

    \markright{Sec. 17-47. Failure to furnish items of value.}

    A person who is authorized by an issuer to furnish money, goods,
    property, services or anything else of value upon presentation of a
    credit card or debit card by the cardholder, or any agent or
    employee of such person, who, with intent to defraud the issuer or
    the cardholder, fails to furnish money, goods, property, services or
    anything else of value which he represents in writing to the issuer
    that he has furnished is guilty of a Class A misdemeanor if the
    difference between the value of all money, goods, property, services
    and anything else of value actually furnished and the value
    represented to the issuer to have been furnished does not exceed
    \$150 in any 6-month period; and is guilty of a Class 4 felony if
    such difference exceeds \$150 in any 6-month period.

    (Source: P.A. 96-1551, eff. 7-1-11 .)

    \hypertarget{ilcs-517-48}{%
    \subsection*{(720 ILCS 5/17-48)}\label{ilcs-517-48}}
    \addcontentsline{toc}{subsection}{(720 ILCS 5/17-48)}

    \hypertarget{sec.-17-48.-repeat-offenses.}{%
    \section*{Sec. 17-48. Repeat
    offenses.}\label{sec.-17-48.-repeat-offenses.}}
    \addcontentsline{toc}{section}{Sec. 17-48. Repeat offenses.}

    \markright{Sec. 17-48. Repeat offenses.}

    Any person convicted of a second or subsequent offense under this
    Subdivision 25 is guilty of a Class 3 felony.

    For purposes of this Section, an offense is considered a second or
    subsequent offense if, prior to his or her conviction of the
    offense, the offender has at any time been convicted under this
    Subdivision 25, or under any prior Act, or under any law of the
    United States or of any state relating to credit card or debit card
    offenses.

    (Source: P.A. 96-1551, eff. 7-1-11 .)

    \hypertarget{ilcs-517-49}{%
    \subsection*{(720 ILCS 5/17-49)}\label{ilcs-517-49}}
    \addcontentsline{toc}{subsection}{(720 ILCS 5/17-49)}

    \hypertarget{sec.-17-49.-severability.}{%
    \section*{Sec. 17-49.
    Severability.}\label{sec.-17-49.-severability.}}
    \addcontentsline{toc}{section}{Sec. 17-49. Severability.}

    \markright{Sec. 17-49. Severability.}

    If any provision of this Subdivision 25 or its application to any
    person or circumstances is held invalid, the invalidity shall not
    affect other provisions or applications of this Subdivision 25 which
    can be given effect without the invalid provision or application,
    and to this end the provisions of this Subdivision 25 are declared
    to be severable.

    (Source: P.A. 96-1551, eff. 7-1-11 .)

    \hypertarget{ilcs-517-49.5}{%
    \subsection*{(720 ILCS 5/17-49.5)}\label{ilcs-517-49.5}}
    \addcontentsline{toc}{subsection}{(720 ILCS 5/17-49.5)}

    \hypertarget{sec.-17-49.5.-telephone-charge-fraud-act-unaffected.}{%
    \section*{Sec. 17-49.5. Telephone Charge Fraud Act
    unaffected.}\label{sec.-17-49.5.-telephone-charge-fraud-act-unaffected.}}
    \addcontentsline{toc}{section}{Sec. 17-49.5. Telephone Charge Fraud
    Act unaffected.}

    \markright{Sec. 17-49.5. Telephone Charge Fraud Act unaffected.}

    Nothing contained in this Subdivision 25 shall be construed to
    repeal, amend, or otherwise affect the Telephone Charge Fraud Act.

    (Source: P.A. 96-1551, eff. 7-1-11 .)

    \hypertarget{ilcs-5art.-17-subdiv.-30-heading}{%
    \subsection*{(720 ILCS 5/Art. 17, Subdiv. 30
    heading)}\label{ilcs-5art.-17-subdiv.-30-heading}}
    \addcontentsline{toc}{subsection}{(720 ILCS 5/Art. 17, Subdiv. 30
    heading)}

    SUBDIVISION 30.

    COMPUTER FRAUD

    (Source: P.A. 96-1551, eff. 7-1-11.)

    \hypertarget{ilcs-517-50}{%
    \subsection*{(720 ILCS 5/17-50)}\label{ilcs-517-50}}
    \addcontentsline{toc}{subsection}{(720 ILCS 5/17-50)}

    (was 720 ILCS 5/16D-5 and 5/16D-6)

    \hypertarget{sec.-17-50.-computer-fraud.}{%
    \section*{Sec. 17-50. Computer
    fraud.}\label{sec.-17-50.-computer-fraud.}}
    \addcontentsline{toc}{section}{Sec. 17-50. Computer fraud.}

    \markright{Sec. 17-50. Computer fraud.}

    (a) A person commits computer fraud when he or she knowingly:

    (1) Accesses or causes to be accessed a computer or any part
    thereof, or a program or data, with the intent of devising or
    executing any scheme or artifice to defraud, or as part of a
    deception;

    (2) Obtains use of, damages, or destroys a computer or any part
    thereof, or alters, deletes, or removes any program or data
    contained therein, in connection with any scheme or artifice to
    defraud, or as part of a deception; or

    (3) Accesses or causes to be accessed a computer or any part
    thereof, or a program or data, and obtains money or control over any
    such money, property, or services of another in connection with any
    scheme or artifice to defraud, or as part of a deception.

    (b) Sentence.

    (1) A violation of subdivision (a)(1) of this Section is a Class 4
    felony.

    (2) A violation of subdivision (a)(2) of this Section is a Class 3
    felony.

    (3) A violation of subdivision (a)(3) of this Section:

    (i) is a Class 4 felony if the value of the money, property, or
    services is \$1,000 or less; or

    (ii) is a Class 3 felony if the value of the money, property, or
    services is more than \$1,000 but less than \$50,000; or

    (iii) is a Class 2 felony if the value of the money, property, or
    services is \$50,000 or more.

    (c) Forfeiture of property. Any person who commits computer fraud as
    set forth in subsection (a) is subject to the property forfeiture
    provisions set forth in Article 124B of the Code of Criminal
    Procedure of 1963.

    (Source: P.A. 96-712, eff. 1-1-10; 96-1551, eff. 7-1-11 .)

    \hypertarget{ilcs-517-51}{%
    \subsection*{(720 ILCS 5/17-51)}\label{ilcs-517-51}}
    \addcontentsline{toc}{subsection}{(720 ILCS 5/17-51)}

    (was 720 ILCS 5/16D-3)

    \hypertarget{sec.-17-51.-computer-tampering.}{%
    \section*{Sec. 17-51. Computer
    tampering.}\label{sec.-17-51.-computer-tampering.}}
    \addcontentsline{toc}{section}{Sec. 17-51. Computer tampering.}

    \markright{Sec. 17-51. Computer tampering.}

    (a) A person commits computer tampering when he or she knowingly and
    without the authorization of a computer's owner or in excess of the
    authority granted to him or her:

    (1) Accesses or causes to be accessed a computer or any part
    thereof, a computer network, or a program or data;

    (2) Accesses or causes to be accessed a computer or any part
    thereof, a computer network, or a program or data, and obtains data
    or services;

    (3) Accesses or causes to be accessed a computer or any part
    thereof, a computer network, or a program or data, and damages or
    destroys the computer or alters, deletes, or removes a computer
    program or data;

    (4) Inserts or attempts to insert a program into a computer or
    computer program knowing or having reason to know that such program
    contains information or commands that will or may:

    (A) damage or destroy that computer, or any other computer
    subsequently accessing or being accessed by that computer;

    (B) alter, delete, or remove a computer program or data from that
    computer, or any other computer program or data in a computer
    subsequently accessing or being accessed by that computer; or

    (C) cause loss to the users of that computer or the users of a
    computer which accesses or which is accessed by such program; or

    (5) Falsifies or forges electronic mail transmission information or
    other routing information in any manner in connection with the
    transmission of unsolicited bulk electronic mail through or into the
    computer network of an electronic mail service provider or its
    subscribers.

    (a-5) Distributing software to falsify routing information. It is
    unlawful for any person knowingly to sell, give, or otherwise
    distribute or possess with the intent to sell, give, or distribute
    software which:

    (1) is primarily designed or produced for the purpose of
    facilitating or enabling the falsification of electronic mail
    transmission information or other routing information;

    (2) has only a limited commercially significant purpose or use other
    than to facilitate or enable the falsification of electronic mail
    transmission information or other routing information; or

    (3) is marketed by that person or another acting in concert with
    that person with that person's knowledge for use in facilitating or
    enabling the falsification of electronic mail transmission
    information or other routing information.

    (a-10) For purposes of subsection (a), accessing a computer network
    is deemed to be with the authorization of a computer's owner if:

    (1) the owner authorizes patrons, customers, or guests to access the
    computer network and the person accessing the computer network is an
    authorized patron, customer, or guest and complies with all terms or
    conditions for use of the computer network that are imposed by the
    owner;

    (2) the owner authorizes the public to access the computer network
    and the person accessing the computer network complies with all
    terms or conditions for use of the computer network that are imposed
    by the owner; or

    (3) the person accesses the computer network in compliance with the
    Revised Uniform Fiduciary Access to Digital Assets Act (2015).

    (b) Sentence.

    (1) A person who commits computer tampering as set forth in
    subdivision (a)(1) or (a)(5) or subsection (a-5) of this Section is
    guilty of a Class B misdemeanor.

    (2) A person who commits computer tampering as set forth in
    subdivision (a)(2) of this Section is guilty of a Class A
    misdemeanor and a Class 4 felony for the second or subsequent
    offense.

    (3) A person who commits computer tampering as set forth in
    subdivision (a)(3) or (a)(4) of this Section is guilty of a Class 4
    felony and a Class 3 felony for the second or subsequent offense.

    (4) If an injury arises from the transmission of unsolicited bulk
    electronic mail, the injured person, other than an electronic mail
    service provider, may also recover attorney's fees and costs, and
    may elect, in lieu of actual damages, to recover the lesser of \$10
    for each unsolicited bulk electronic mail message transmitted in
    violation of this Section, or \$25,000 per day. The injured person
    shall not have a cause of action against the electronic mail service
    provider that merely transmits the unsolicited bulk electronic mail
    over its computer network.

    (5) If an injury arises from the transmission of unsolicited bulk
    electronic mail, an injured electronic mail service provider may
    also recover attorney's fees and costs, and may elect, in lieu of
    actual damages, to recover the greater of \$10 for each unsolicited
    electronic mail advertisement transmitted in violation of this
    Section, or \$25,000 per day.

    (6) The provisions of this Section shall not be construed to limit
    any person's right to pursue any additional civil remedy otherwise
    allowed by law.

    (c) Whoever suffers loss by reason of a violation of subdivision
    (a)(4) of this Section may, in a civil action against the violator,
    obtain appropriate relief. In a civil action under this Section, the
    court may award to the prevailing party reasonable attorney's fees
    and other litigation expenses.

    (Source: P.A. 99-775, eff. 8-12-16.)

    \hypertarget{ilcs-517-52}{%
    \subsection*{(720 ILCS 5/17-52)}\label{ilcs-517-52}}
    \addcontentsline{toc}{subsection}{(720 ILCS 5/17-52)}

    (was 720 ILCS 5/16D-4)

    \hypertarget{sec.-17-52.-aggravated-computer-tampering.}{%
    \section*{Sec. 17-52. Aggravated computer
    tampering.}\label{sec.-17-52.-aggravated-computer-tampering.}}
    \addcontentsline{toc}{section}{Sec. 17-52. Aggravated computer
    tampering.}

    \markright{Sec. 17-52. Aggravated computer tampering.}

    (a) A person commits aggravated computer tampering when he or she
    commits computer tampering as set forth in paragraph (a)(3) of
    Section 17-51 and he or she knowingly:

    (1) causes disruption of or interference with vital services or
    operations of State or local government or a public utility; or

    (2) creates a strong probability of death or great bodily harm to
    one or more individuals.

    (b) Sentence.

    (1) A person who commits aggravated computer tampering as set forth
    in paragraph (a)(1) of this Section is guilty of a Class 3 felony.

    (2) A person who commits aggravated computer tampering as set forth
    in paragraph (a)(2) of this Section is guilty of a Class 2 felony.

    (Source: P.A. 96-1551, eff. 7-1-11 .)

    \hypertarget{ilcs-517-52.5}{%
    \subsection*{(720 ILCS 5/17-52.5)}\label{ilcs-517-52.5}}
    \addcontentsline{toc}{subsection}{(720 ILCS 5/17-52.5)}

    (was 720 ILCS 5/16D-5.5)

    \hypertarget{sec.-17-52.5.-unlawful-use-of-encryption.}{%
    \section*{Sec. 17-52.5. Unlawful use of
    encryption.}\label{sec.-17-52.5.-unlawful-use-of-encryption.}}
    \addcontentsline{toc}{section}{Sec. 17-52.5. Unlawful use of
    encryption.}

    \markright{Sec. 17-52.5. Unlawful use of encryption.}

    (a) For the purpose of this Section:

    ``Computer'' has the meaning ascribed to the term in

    Section 17-0.5.

    ``Encryption'' means the use of any protective or disruptive
    measure, including, without limitation, cryptography, enciphering,
    encoding, or a computer contaminant, to: (1) prevent, impede, delay,
    or disrupt access to any data, information, image, program, signal,
    or sound; (2) cause or make any data, information, image, program,
    signal, or sound unintelligible or unusable; or (3) prevent, impede,
    delay, or disrupt the normal operation or use of any component,
    device, equipment, system, or network.

    ``Network'' means a set of related, remotely connected devices and
    facilities, including more than one system, with the capability to
    transmit data among any of the devices and facilities. The term
    includes, without limitation, a local, regional, or global computer
    network.

    ``Program'' means an ordered set of data representing coded
    instructions or statements which can be executed by a computer and
    cause the computer to perform one or more tasks.

    ``System'' means a set of related equipment, whether or not
    connected, which is used with or for a computer.

    (b) A person shall not knowingly use or attempt to use encryption,
    directly or indirectly, to:

    (1) commit, facilitate, further, or promote any criminal offense;

    (2) aid, assist, or encourage another person to commit any criminal
    offense;

    (3) conceal evidence of the commission of any criminal offense; or

    (4) conceal or protect the identity of a person who has committed
    any criminal offense.

    (c) Telecommunications carriers and information service providers
    are not liable under this Section, except for willful and wanton
    misconduct, for providing encryption services used by others in
    violation of this Section.

    (d) Sentence. A person who violates this Section is guilty of a
    Class A misdemeanor, unless the encryption was used or attempted to
    be used to commit an offense for which a greater penalty is provided
    by law. If the encryption was used or attempted to be used to commit
    an offense for which a greater penalty is provided by law, the
    person shall be punished as prescribed by law for that offense.

    (e) A person who violates this Section commits a criminal offense
    that is separate and distinct from any other criminal offense and
    may be prosecuted and convicted under this Section whether or not
    the person or any other person is or has been prosecuted or
    convicted for any other criminal offense arising out of the same
    facts as the violation of this Section.

    (Source: P.A. 101-87, eff. 1-1-20 .)

    \hypertarget{ilcs-517-54}{%
    \subsection*{(720 ILCS 5/17-54)}\label{ilcs-517-54}}
    \addcontentsline{toc}{subsection}{(720 ILCS 5/17-54)}

    (was 720 ILCS 5/16D-7)

    \hypertarget{sec.-17-54.-evidence-of-lack-of-authority.}{%
    \section*{Sec. 17-54. Evidence of lack of
    authority.}\label{sec.-17-54.-evidence-of-lack-of-authority.}}
    \addcontentsline{toc}{section}{Sec. 17-54. Evidence of lack of
    authority.}

    \markright{Sec. 17-54. Evidence of lack of authority.}

    For the purposes of Sections 17-50 through 17-52, the trier of fact
    may infer that a person accessed a computer without the
    authorization of its owner or in excess of the authority granted if
    the person accesses or causes to be accessed a computer, which
    access requires a confidential or proprietary code which has not
    been issued to or authorized for use by that person. This Section
    does not apply to a person who acquires access in compliance with
    the Revised Uniform Fiduciary Access to Digital Assets Act (2015).

    (Source: P.A. 99-775, eff. 8-12-16.)

    \hypertarget{ilcs-517-55}{%
    \subsection*{(720 ILCS 5/17-55)}\label{ilcs-517-55}}
    \addcontentsline{toc}{subsection}{(720 ILCS 5/17-55)}

    \hypertarget{sec.-17-55.-definitions.}{%
    \section*{Sec. 17-55. Definitions.}\label{sec.-17-55.-definitions.}}
    \addcontentsline{toc}{section}{Sec. 17-55. Definitions.}

    \markright{Sec. 17-55. Definitions.}

    For the purposes of this subdivision 30:

    In addition to its meaning as defined in Section 15-1 of this Code,
    ``property'' means: (1) electronic impulses; (2) electronically
    produced data; (3) confidential, copyrighted, or proprietary
    information; (4) private identification codes or numbers which
    permit access to a computer by authorized computer users or generate
    billings to consumers for purchase of goods and services, including
    but not limited to credit card transactions and telecommunications
    services or permit electronic fund transfers; (5) software or
    programs in either machine or human readable form; or (6) any other
    tangible or intangible item relating to a computer or any part
    thereof.

    ``Access'' means to use, instruct, communicate with, store data in,
    retrieve or intercept data from, or otherwise utilize any services
    of, a computer, a network, or data.

    ``Services'' includes but is not limited to computer time, data
    manipulation, or storage functions.

    ``Vital services or operations'' means those services or operations
    required to provide, operate, maintain, and repair network cabling,
    transmission, distribution, or computer facilities necessary to
    ensure or protect the public health, safety, or welfare. Those
    services or operations include, but are not limited to, services
    provided by medical personnel or institutions, fire departments,
    emergency services agencies, national defense contractors, armed
    forces or militia personnel, private and public utility companies,
    or law enforcement agencies.

    (Source: P.A. 101-87, eff. 1-1-20 .)

    \hypertarget{ilcs-5art.-17-subdiv.-35-heading}{%
    \subsection*{(720 ILCS 5/Art. 17, Subdiv. 35
    heading)}\label{ilcs-5art.-17-subdiv.-35-heading}}
    \addcontentsline{toc}{subsection}{(720 ILCS 5/Art. 17, Subdiv. 35
    heading)}

    SUBDIVISION 35.

    MISCELLANEOUS SPECIAL FRAUD

    (Source: P.A. 96-1551, eff. 7-1-11.)

    \hypertarget{ilcs-517-56}{%
    \subsection*{(720 ILCS 5/17-56)}\label{ilcs-517-56}}
    \addcontentsline{toc}{subsection}{(720 ILCS 5/17-56)}

    (was 720 ILCS 5/16-1.3)

    \hypertarget{sec.-17-56.-financial-exploitation-of-an-elderly-person-or-a-person-with-a-disability.}{%
    \section*{Sec. 17-56. Financial exploitation of an elderly person or
    a person with a
    disability.}\label{sec.-17-56.-financial-exploitation-of-an-elderly-person-or-a-person-with-a-disability.}}
    \addcontentsline{toc}{section}{Sec. 17-56. Financial exploitation of
    an elderly person or a person with a disability.}

    \markright{Sec. 17-56. Financial exploitation of an elderly person
    or a person with a disability.}

    (a) A person commits financial exploitation of an elderly person or
    a person with a disability when he or she stands in a position of
    trust or confidence with the elderly person or a person with a
    disability and he or she knowingly:

    (1) by deception or intimidation obtains control over the property
    of an elderly person or a person with a disability; or

    (2) illegally uses the assets or resources of an elderly person or a
    person with a disability.

    (b) Sentence. Financial exploitation of an elderly person or a
    person with a disability is: (1) a Class 4 felony if the value of
    the property is \$300 or less, (2) a Class 3 felony if the value of
    the property is more than \$300 but less than \$5,000, (3) a Class 2
    felony if the value of the property is \$5,000 or more but less than
    \$50,000, and (4) a Class 1 felony if the value of the property is
    \$50,000 or more or if the elderly person is over 70 years of age
    and the value of the property is \$15,000 or more or if the elderly
    person is 80 years of age or older and the value of the property is
    \$5,000 or more.

    (c) For purposes of this Section:

    (1) ``Elderly person'' means a person 60 years of age or older.

    (2) ``Person with a disability'' means a person who suffers from a
    physical or mental impairment resulting from disease, injury,
    functional disorder or congenital condition that impairs the
    individual's mental or physical ability to independently manage his
    or her property or financial resources, or both.

    (3) ``Intimidation'' means the communication to an elderly person or
    a person with a disability that he or she shall be deprived of food
    and nutrition, shelter, prescribed medication or medical care and
    treatment or conduct as provided in Section 12-6 of this Code.

    (4) ``Deception'' means, in addition to its meaning as defined in
    Section 15-4 of this Code, a misrepresentation or concealment of
    material fact relating to the terms of a contract or agreement
    entered into with the elderly person or person with a disability or
    to the existing or pre-existing condition of any of the property
    involved in such contract or agreement; or the use or employment of
    any misrepresentation, false pretense or false promise in order to
    induce, encourage or solicit the elderly person or person with a
    disability to enter into a contract or agreement.

    The illegal use of the assets or resources of an elderly person or a
    person with a disability includes, but is not limited to, the
    misappropriation of those assets or resources by undue influence,
    breach of a fiduciary relationship, fraud, deception, extortion, or
    use of the assets or resources contrary to law.

    A person stands in a position of trust and confidence with an
    elderly person or person with a disability when he (i) is a parent,
    spouse, adult child or other relative by blood or marriage of the
    elderly person or person with a disability, (ii) is a joint tenant
    or tenant in common with the elderly person or person with a
    disability, (iii) has a legal or fiduciary relationship with the
    elderly person or person with a disability, (iv) is a financial
    planning or investment professional, (v) is a paid or unpaid
    caregiver for the elderly person or person with a disability, or
    (vi) is a friend or acquaintance in a position of trust.

    (d) Limitations. Nothing in this Section shall be construed to limit
    the remedies available to the victim under the Illinois Domestic
    Violence Act of 1986.

    (e) Good faith efforts. Nothing in this Section shall be construed
    to impose criminal liability on a person who has made a good faith
    effort to assist the elderly person or person with a disability in
    the management of his or her property, but through no fault of his
    or her own has been unable to provide such assistance.

    (f) Not a defense. It shall not be a defense to financial
    exploitation of an elderly person or person with a disability that
    the accused reasonably believed that the victim was not an elderly
    person or person with a disability. Consent is not a defense to
    financial exploitation of an elderly person or a person with a
    disability if the accused knew or had reason to know that the
    elderly person or a person with a disability lacked capacity to
    consent.

    (g) Civil Liability. A civil cause of action exists for financial
    exploitation of an elderly person or a person with a disability as
    described in subsection (a) of this Section. A person against whom a
    civil judgment has been entered for financial exploitation of an
    elderly person or person with a disability shall be liable to the
    victim or to the estate of the victim in damages of treble the
    amount of the value of the property obtained, plus reasonable
    attorney fees and court costs. In a civil action under this
    subsection, the burden of proof that the defendant committed
    financial exploitation of an elderly person or a person with a
    disability as described in subsection (a) of this Section shall be
    by a preponderance of the evidence. This subsection shall be
    operative whether or not the defendant has been charged or convicted
    of the criminal offense as described in subsection (a) of this
    Section. This subsection (g) shall not limit or affect the right of
    any person to bring any cause of action or seek any remedy available
    under the common law, or other applicable law, arising out of the
    financial exploitation of an elderly person or a person with a
    disability.

    (h) If a person is charged with financial exploitation of an elderly
    person or a person with a disability that involves the taking or
    loss of property valued at more than \$5,000, a prosecuting attorney
    may file a petition with the circuit court of the county in which
    the defendant has been charged to freeze the assets of the defendant
    in an amount equal to but not greater than the alleged value of lost
    or stolen property in the defendant's pending criminal proceeding
    for purposes of restitution to the victim. The burden of proof
    required to freeze the defendant's assets shall be by a
    preponderance of the evidence.

    (Source: P.A. 101-394, eff. 1-1-20; 102-244, eff. 1-1-22 .)

    \hypertarget{ilcs-517-57}{%
    \subsection*{(720 ILCS 5/17-57)}\label{ilcs-517-57}}
    \addcontentsline{toc}{subsection}{(720 ILCS 5/17-57)}

    (was 720 ILCS 5/17-28)

    \hypertarget{sec.-17-57.-defrauding-drug-and-alcohol-screening-tests.}{%
    \section*{Sec. 17-57. Defrauding drug and alcohol screening
    tests.}\label{sec.-17-57.-defrauding-drug-and-alcohol-screening-tests.}}
    \addcontentsline{toc}{section}{Sec. 17-57. Defrauding drug and
    alcohol screening tests.}

    \markright{Sec. 17-57. Defrauding drug and alcohol screening tests.}

    (a) It is unlawful for a person to:

    (1) manufacture, sell, give away, distribute, or market synthetic or
    human substances or other products in this State or transport urine
    into this State with the intent of using the synthetic or human
    substances or other products to defraud a drug or alcohol screening
    test;

    (2) substitute or spike a sample or advertise a sample substitution
    or other spiking device or measure, with the intent of attempting to
    foil or defeat a drug or alcohol screening test;

    (3) adulterate synthetic or human substances with the intent to
    defraud a drug or alcohol screening test; or

    (4) manufacture, sell, or possess adulterants that are intended to
    be used to adulterate synthetic or human substances with the intent
    of defrauding a drug or alcohol screening test.

    (b) The trier of fact may infer intent to violate this Section if a
    heating element or any other device used to thwart a drug or alcohol
    screening test accompanies the sale, giving, distribution, or
    marketing of synthetic or human substances or other products or
    instructions that provide a method for thwarting a drug or alcohol
    screening test accompany the sale, giving, distribution, or
    marketing of synthetic or human substances or other products.

    (c) Sentence. A violation of this Section is a Class 4 felony for
    which the court shall impose a minimum fine of \$1,000.

    (d) For the purposes of this Section, ``drug or alcohol screening
    test'' includes, but is not limited to, urine testing, hair follicle
    testing, perspiration testing, saliva testing, blood testing,
    fingernail testing, and eye drug testing.

    (Source: P.A. 96-1551, eff. 7-1-11 .)

    \hypertarget{ilcs-517-58}{%
    \subsection*{(720 ILCS 5/17-58)}\label{ilcs-517-58}}
    \addcontentsline{toc}{subsection}{(720 ILCS 5/17-58)}

    (was 720 ILCS 5/17-16)

    \hypertarget{sec.-17-58.-fraudulent-production-of-infant.}{%
    \section*{Sec. 17-58. Fraudulent production of
    infant.}\label{sec.-17-58.-fraudulent-production-of-infant.}}
    \addcontentsline{toc}{section}{Sec. 17-58. Fraudulent production of
    infant.}

    \markright{Sec. 17-58. Fraudulent production of infant.}

    A person who fraudulently produces an infant, falsely pretending it
    to have been born of parents whose child would be entitled to a
    share of a personal estate, or to inherit real estate, with the
    intent of intercepting the inheritance of the real estate, or the
    distribution of the personal property from a person lawfully
    entitled to the personal property, is guilty of a Class 3 felony.

    (Source: P.A. 96-1551, eff. 7-1-11 .)

    \hypertarget{ilcs-517-59}{%
    \subsection*{(720 ILCS 5/17-59)}\label{ilcs-517-59}}
    \addcontentsline{toc}{subsection}{(720 ILCS 5/17-59)}

    (was 720 ILCS 5/39-1)

    \hypertarget{sec.-17-59.-criminal-usury.}{%
    \section*{Sec. 17-59. Criminal
    usury.}\label{sec.-17-59.-criminal-usury.}}
    \addcontentsline{toc}{section}{Sec. 17-59. Criminal usury.}

    \markright{Sec. 17-59. Criminal usury.}

    (a) A person commits criminal usury when, in exchange for either a
    loan of money or other property or forbearance from the collection
    of such a loan, he or she knowingly contracts for or receives from
    an individual, directly or indirectly, interest, discount, or other
    consideration at a rate greater than 20\% per annum either before or
    after the maturity of the loan.

    (b) When a person has in his or her personal or constructive
    possession records, memoranda, or other documentary record of
    usurious loans, the trier of fact may infer that he or she has
    violated subsection (a) of this Section.

    (c) Sentence. Criminal usury is a Class 4 felony.

    (d) Non-application to licensed persons. This Section does not apply
    to any loan authorized to be made by any person licensed under the
    Consumer Installment Loan Act or to any loan permitted by Sections
    4, 4.2 and 4a of the Interest Act or by any other law of this State.

    (Source: P.A. 96-1551, eff. 7-1-11 .)

    \hypertarget{ilcs-517-60}{%
    \subsection*{(720 ILCS 5/17-60)}\label{ilcs-517-60}}
    \addcontentsline{toc}{subsection}{(720 ILCS 5/17-60)}

    (was 720 ILCS 5/17-7)

    \hypertarget{sec.-17-60.-promotion-of-pyramid-sales-schemes.}{%
    \section*{Sec. 17-60. Promotion of pyramid sales
    schemes.}\label{sec.-17-60.-promotion-of-pyramid-sales-schemes.}}
    \addcontentsline{toc}{section}{Sec. 17-60. Promotion of pyramid
    sales schemes.}

    \markright{Sec. 17-60. Promotion of pyramid sales schemes.}

    (a) A person who knowingly sells, offers to sell, or attempts to
    sell the right to participate in a pyramid sales scheme commits a
    Class A misdemeanor.

    (b) The term ``pyramid sales scheme'' means any plan or operation
    whereby a person, in exchange for money or other thing of value,
    acquires the opportunity to receive a benefit or thing of value,
    which is primarily based upon the inducement of additional persons,
    by himself or others, regardless of number, to participate in the
    same plan or operation and is not primarily contingent on the volume
    or quantity of goods, services, or other property sold or
    distributed or to be sold or distributed to persons for purposes of
    resale to consumers. For purposes of this subsection, ``money or
    other thing of value'' shall not include payments made for sales
    demonstration equipment and materials furnished on a nonprofit basis
    for use in making sales and not for resale.

    (Source: P.A. 96-1551, eff. 7-1-11 .)

    \hypertarget{ilcs-517-61}{%
    \subsection*{(720 ILCS 5/17-61)}\label{ilcs-517-61}}
    \addcontentsline{toc}{subsection}{(720 ILCS 5/17-61)}

    \hypertarget{sec.-17-61.-unauthorized-use-of-university-stationery.}{%
    \section*{Sec. 17-61. Unauthorized use of university
    stationery.}\label{sec.-17-61.-unauthorized-use-of-university-stationery.}}
    \addcontentsline{toc}{section}{Sec. 17-61. Unauthorized use of
    university stationery.}

    \markright{Sec. 17-61. Unauthorized use of university stationery.}

    (a) No person, firm or corporation shall use the official stationery
    or seal or a facsimile thereof, of any State supported university,
    college or other institution of higher education or any organization
    thereof unless approved in writing in advance by the university,
    college or institution of higher education affected, for any private
    promotional scheme wherein it is made to appear that the
    organization or university, college or other institution of higher
    education is endorsing the private promotional scheme.

    (b) A violation of this Section is a petty offense.

    (Source: P.A. 96-1551, eff. 7-1-11 .)

    \hypertarget{ilcs-517-62}{%
    \subsection*{(720 ILCS 5/17-62)}\label{ilcs-517-62}}
    \addcontentsline{toc}{subsection}{(720 ILCS 5/17-62)}

    \hypertarget{sec.-17-62.-unlawful-possession-of-device-for-manufacturing-a-false-universal-price-code-label.}{%
    \section*{Sec. 17-62. Unlawful possession of device for
    manufacturing a false universal price code
    label.}\label{sec.-17-62.-unlawful-possession-of-device-for-manufacturing-a-false-universal-price-code-label.}}
    \addcontentsline{toc}{section}{Sec. 17-62. Unlawful possession of
    device for manufacturing a false universal price code label.}

    \markright{Sec. 17-62. Unlawful possession of device for
    manufacturing a false universal price code label.}

    It is unlawful for a person to knowingly possess a device the
    purpose of which is to manufacture a false, counterfeit, altered, or
    simulated universal price code label. A violation of this Section is
    a Class 3 felony.

    (Source: P.A. 96-1551, eff. 7-1-11 .)
  \end{enumerate}
\end{enumerate}

\bookmarksetup{startatroot}

\hypertarget{article-17a.-disqualification-for-state-benefits}{%
\chapter*{Article 17a. Disqualification For State
Benefits}\label{article-17a.-disqualification-for-state-benefits}}
\addcontentsline{toc}{chapter}{Article 17a. Disqualification For State
Benefits}

\markboth{Article 17a. Disqualification For State Benefits}{Article 17a.
Disqualification For State Benefits}

(Repealed)

(Article repealed by P.A. 96-1551, eff. 7-1-11)

\bookmarksetup{startatroot}

\hypertarget{article-17b.-wic-fraud}{%
\chapter*{Article 17b. Wic Fraud}\label{article-17b.-wic-fraud}}
\addcontentsline{toc}{chapter}{Article 17b. Wic Fraud}

\markboth{Article 17b. Wic Fraud}{Article 17b. Wic Fraud}

(Source: Repealed by P.A. 97-1150, eff. 1-25-13.)

\bookmarksetup{startatroot}

\hypertarget{article-18.-robbery}{%
\chapter*{Article 18. Robbery}\label{article-18.-robbery}}
\addcontentsline{toc}{chapter}{Article 18. Robbery}

\markboth{Article 18. Robbery}{Article 18. Robbery}

\hypertarget{ilcs-518-1-from-ch.-38-par.-18-1}{%
\subsection*{(720 ILCS 5/18-1) (from Ch. 38, par.
18-1)}\label{ilcs-518-1-from-ch.-38-par.-18-1}}
\addcontentsline{toc}{subsection}{(720 ILCS 5/18-1) (from Ch. 38, par.
18-1)}

\hypertarget{sec.-18-1.-robbery-aggravated-robbery.}{%
\section*{Sec. 18-1. Robbery; aggravated
robbery.}\label{sec.-18-1.-robbery-aggravated-robbery.}}
\addcontentsline{toc}{section}{Sec. 18-1. Robbery; aggravated robbery.}

\markright{Sec. 18-1. Robbery; aggravated robbery.}

(a) Robbery. A person commits robbery when he or she knowingly takes
property, except a motor vehicle covered by Section 18-3 or 18-4, from
the person or presence of another by the use of force or by threatening
the imminent use of force.

(b) Aggravated robbery.

(1) A person commits aggravated robbery when he or she violates
subsection (a) while indicating verbally or by his or her actions to the
victim that he or she is presently armed with a firearm or other
dangerous weapon, including a knife, club, ax, or bludgeon. This offense
shall be applicable even though it is later determined that he or she
had no firearm or other dangerous weapon, including a knife, club, ax,
or bludgeon, in his or her possession when he or she committed the
robbery.

(2) A person commits aggravated robbery when he or she knowingly takes
property from the person or presence of another by delivering (by
injection, inhalation, ingestion, transfer of possession, or any other
means) to the victim without his or her consent, or by threat or
deception, and for other than medical purposes, any controlled
substance.

(c) Sentence.

Robbery is a Class 2 felony, unless the victim is 60 years of age or
over or is a person with a physical disability, or the robbery is
committed in a school, day care center, day care home, group day care
home, or part day child care facility, or place of worship, in which
case robbery is a Class 1 felony. Aggravated robbery is a Class 1
felony.

(d) Regarding penalties prescribed in subsection (c) for violations
committed in a day care center, day care home, group day care home, or
part day child care facility, the time of day, time of year, and whether
children under 18 years of age were present in the day care center, day
care home, group day care home, or part day child care facility are
irrelevant.

(Source: P.A. 99-143, eff. 7-27-15.)

\hypertarget{ilcs-518-2-from-ch.-38-par.-18-2}{%
\subsection*{(720 ILCS 5/18-2) (from Ch. 38, par.
18-2)}\label{ilcs-518-2-from-ch.-38-par.-18-2}}
\addcontentsline{toc}{subsection}{(720 ILCS 5/18-2) (from Ch. 38, par.
18-2)}

\hypertarget{sec.-18-2.-armed-robbery.}{%
\section*{Sec. 18-2. Armed robbery.}\label{sec.-18-2.-armed-robbery.}}
\addcontentsline{toc}{section}{Sec. 18-2. Armed robbery.}

\markright{Sec. 18-2. Armed robbery.}

(a) A person commits armed robbery when he or she violates Section 18-1;
and

(1) he or she carries on or about his or her person or is otherwise
armed with a dangerous weapon other than a firearm; or

(2) he or she carries on or about his or her person or is otherwise
armed with a firearm; or

(3) he or she, during the commission of the offense, personally
discharges a firearm; or

(4) he or she, during the commission of the offense, personally
discharges a firearm that proximately causes great bodily harm,
permanent disability, permanent disfigurement, or death to another
person.

(b) Sentence.

Armed robbery in violation of subsection (a)(1) is a Class X felony. A
violation of subsection (a)(2) is a Class X felony for which 15 years
shall be added to the term of imprisonment imposed by the court. A
violation of subsection (a)(3) is a Class X felony for which 20 years
shall be added to the term of imprisonment imposed by the court. A
violation of subsection (a)(4) is a Class X felony for which 25 years or
up to a term of natural life shall be added to the term of imprisonment
imposed by the court.

(Source: P.A. 91-404, eff. 1-1-00 .)

\hypertarget{ilcs-518-3}{%
\subsection*{(720 ILCS 5/18-3)}\label{ilcs-518-3}}
\addcontentsline{toc}{subsection}{(720 ILCS 5/18-3)}

\hypertarget{sec.-18-3.-vehicular-hijacking.}{%
\section*{Sec. 18-3. Vehicular
hijacking.}\label{sec.-18-3.-vehicular-hijacking.}}
\addcontentsline{toc}{section}{Sec. 18-3. Vehicular hijacking.}

\markright{Sec. 18-3. Vehicular hijacking.}

(a) A person commits vehicular hijacking when he or she knowingly takes
a motor vehicle from the person or the immediate presence of another by
the use of force or by threatening the imminent use of force.

(b) Sentence. Vehicular hijacking is a Class 1 felony.

(Source: P.A. 97-1108, eff. 1-1-13.)

\hypertarget{ilcs-518-4}{%
\subsection*{(720 ILCS 5/18-4)}\label{ilcs-518-4}}
\addcontentsline{toc}{subsection}{(720 ILCS 5/18-4)}

\hypertarget{sec.-18-4.-aggravated-vehicular-hijacking.}{%
\section*{Sec. 18-4. Aggravated vehicular
hijacking.}\label{sec.-18-4.-aggravated-vehicular-hijacking.}}
\addcontentsline{toc}{section}{Sec. 18-4. Aggravated vehicular
hijacking.}

\markright{Sec. 18-4. Aggravated vehicular hijacking.}

(a) A person commits aggravated vehicular hijacking when he or she
violates Section 18-3; and

(1) the person from whose immediate presence the motor vehicle is taken
is a person with a physical disability or a person 60 years of age or
over; or

(2) a person under 16 years of age is a passenger in the motor vehicle
at the time of the offense; or

(3) he or she carries on or about his or her person, or is otherwise
armed with a dangerous weapon, other than a firearm; or

(4) he or she carries on or about his or her person or is otherwise
armed with a firearm; or

(5) he or she, during the commission of the offense, personally
discharges a firearm; or

(6) he or she, during the commission of the offense, personally
discharges a firearm that proximately causes great bodily harm,
permanent disability, permanent disfigurement, or death to another
person.

(b) Sentence. Aggravated vehicular hijacking in violation of subsections
(a)(1) or (a)(2) is a Class X felony. A violation of subsection (a)(3)
is a Class X felony for which a term of imprisonment of not less than 7
years shall be imposed. A violation of subsection (a)(4) is a Class X
felony for which 15 years shall be added to the term of imprisonment
imposed by the court. A violation of subsection (a)(5) is a Class X
felony for which 20 years shall be added to the term of imprisonment
imposed by the court. A violation of subsection (a)(6) is a Class X
felony for which 25 years or up to a term of natural life shall be added
to the term of imprisonment imposed by the court.

(Source:

P.A. 99-143, eff. 7-27-15.)

\hypertarget{ilcs-518-5}{%
\subsection*{(720 ILCS 5/18-5)}\label{ilcs-518-5}}
\addcontentsline{toc}{subsection}{(720 ILCS 5/18-5)}

\hypertarget{sec.-18-5.-repealed.}{%
\section*{Sec. 18-5. (Repealed).}\label{sec.-18-5.-repealed.}}
\addcontentsline{toc}{section}{Sec. 18-5. (Repealed).}

\markright{Sec. 18-5. (Repealed).}

(Source: P.A. 91-357, eff. 7-29-99. Repealed by P.A. 97-1108, eff.
1-1-13.)

\hypertarget{ilcs-518-6}{%
\subsection*{(720 ILCS 5/18-6)}\label{ilcs-518-6}}
\addcontentsline{toc}{subsection}{(720 ILCS 5/18-6)}

(was 720 ILCS 5/12-11.1)

\hypertarget{sec.-18-6.-vehicular-invasion.}{%
\section*{Sec. 18-6. Vehicular
invasion.}\label{sec.-18-6.-vehicular-invasion.}}
\addcontentsline{toc}{section}{Sec. 18-6. Vehicular invasion.}

\markright{Sec. 18-6. Vehicular invasion.}

(a) A person commits vehicular invasion when he or she knowingly, by
force and without lawful justification, enters or reaches into the
interior of a motor vehicle while the motor vehicle is occupied by
another person or persons, with the intent to commit therein a theft or
felony.

(b) Sentence. Vehicular invasion is a Class 1 felony.

(Source: P.A. 97-1108, eff. 1-1-13.)

\bookmarksetup{startatroot}

\hypertarget{article-19.-burglary}{%
\chapter*{Article 19. Burglary}\label{article-19.-burglary}}
\addcontentsline{toc}{chapter}{Article 19. Burglary}

\markboth{Article 19. Burglary}{Article 19. Burglary}

\hypertarget{ilcs-519-1-from-ch.-38-par.-19-1}{%
\subsection*{(720 ILCS 5/19-1) (from Ch. 38, par.
19-1)}\label{ilcs-519-1-from-ch.-38-par.-19-1}}
\addcontentsline{toc}{subsection}{(720 ILCS 5/19-1) (from Ch. 38, par.
19-1)}

\hypertarget{sec.-19-1.-burglary.}{%
\section*{Sec. 19-1. Burglary.}\label{sec.-19-1.-burglary.}}
\addcontentsline{toc}{section}{Sec. 19-1. Burglary.}

\markright{Sec. 19-1. Burglary.}

(a) A person commits burglary when without authority he or she knowingly
enters or without authority remains within a building, housetrailer,
watercraft, aircraft, motor vehicle, railroad car, freight container, or
any part thereof, with intent to commit therein a felony or theft. This
offense shall not include the offenses set out in Section 4-102 of the
Illinois Vehicle Code.

(b) Sentence.

Burglary committed in, and without causing damage to, a watercraft,
aircraft, motor vehicle, railroad car, freight container, or any part
thereof is a Class 3 felony. Burglary committed in a building,
housetrailer, or any part thereof or while causing damage to a
watercraft, aircraft, motor vehicle, railroad car, freight container, or
any part thereof is a Class 2 felony. A burglary committed in a school,
day care center, day care home, group day care home, or part day child
care facility, or place of worship is a Class 1 felony, except that this
provision does not apply to a day care center, day care home, group day
care home, or part day child care facility operated in a private
residence used as a dwelling.

(c) Regarding penalties prescribed in subsection (b) for violations
committed in a day care center, day care home, group day care home, or
part day child care facility, the time of day, time of year, and whether
children under 18 years of age were present in the day care center, day
care home, group day care home, or part day child care facility are
irrelevant.

(Source: P.A. 102-546, eff. 1-1-22 .)

\hypertarget{ilcs-519-2-from-ch.-38-par.-19-2}{%
\subsection*{(720 ILCS 5/19-2) (from Ch. 38, par.
19-2)}\label{ilcs-519-2-from-ch.-38-par.-19-2}}
\addcontentsline{toc}{subsection}{(720 ILCS 5/19-2) (from Ch. 38, par.
19-2)}

\hypertarget{sec.-19-2.-possession-of-burglary-tools.}{%
\section*{Sec. 19-2. Possession of burglary
tools.}\label{sec.-19-2.-possession-of-burglary-tools.}}
\addcontentsline{toc}{section}{Sec. 19-2. Possession of burglary tools.}

\markright{Sec. 19-2. Possession of burglary tools.}

(a) A person commits possession of burglary tools when he or she
possesses any key, tool, instrument, device, or any explosive, suitable
for use in breaking into a building, housetrailer, watercraft, aircraft,
motor vehicle, railroad car, or any depository designed for the
safekeeping of property, or any part thereof, with intent to enter that
place and with intent to commit therein a felony or theft. The trier of
fact may infer from the possession of a key designed for lock bumping an
intent to commit a felony or theft; however, this inference does not
apply to any peace officer or other employee of a law enforcement
agency, or to any person or agency licensed under the Private Detective,
Private Alarm, Private Security, Fingerprint Vendor, and Locksmith Act
of 2004. For the purposes of this Section, ``lock bumping'' means a lock
picking technique for opening a pin tumbler lock using a
specially-crafted bumpkey.

(a-5) A person also commits possession of burglary tools when he or she,
knowingly and with the intent to enter the motor vehicle and with the
intent to commit therein a felony or theft, possesses a device designed
to:

(1) unlock or start a motor vehicle without the use or possession of the
key to the motor vehicle; or

(2) capture or duplicate a signal from the key fob of a motor vehicle to
unlock or start the motor vehicle without the use or possession of the
key to the motor vehicle.

(b) Sentence. Possession of burglary tools is a Class 4 felony.

(Source: P.A. 102-903, eff. 1-1-23 .)

\hypertarget{ilcs-519-2.5}{%
\subsection*{(720 ILCS 5/19-2.5)}\label{ilcs-519-2.5}}
\addcontentsline{toc}{subsection}{(720 ILCS 5/19-2.5)}

\hypertarget{sec.-19-2.5.-unlawful-sale-of-burglary-tools.}{%
\section*{Sec. 19-2.5. Unlawful sale of burglary
tools.}\label{sec.-19-2.5.-unlawful-sale-of-burglary-tools.}}
\addcontentsline{toc}{section}{Sec. 19-2.5. Unlawful sale of burglary
tools.}

\markright{Sec. 19-2.5. Unlawful sale of burglary tools.}

(a) For the purposes of this Section:

``Lock bumping'' means a lock picking technique for opening a pin
tumbler lock using a specially-crafted bumpkey.

``Motor vehicle'' has the meaning ascribed to it in the

Illinois Vehicle Code.

(b) A person commits the offense of unlawful sale of burglary tools when
he or she knowingly sells or transfers any key, including a key designed
for lock bumping, or a lock pick specifically manufactured or altered
for use in breaking into a building, housetrailer, watercraft, aircraft,
motor vehicle, railroad car, or any depository designed for the
safekeeping of property, or any part of that property.

(c) This Section does not apply to the sale or transfer of any item
described in subsection (b) to any peace officer or other employee of a
law enforcement agency, or to any person or agency licensed as a
locksmith under the Private Detective, Private Alarm, Private Security,
Fingerprint Vendor, and Locksmith Act of 2004, or to any person engaged
in the business of towing vehicles, or to any person engaged in the
business of lawful repossession of property who possesses a valid
Repossessor-ICC Authorization Card.

(d) Sentence. Unlawful sale of burglary tools is a Class 4 felony.

(Source: P.A. 96-1307, eff. 1-1-11.)

\hypertarget{ilcs-519-3-from-ch.-38-par.-19-3}{%
\subsection*{(720 ILCS 5/19-3) (from Ch. 38, par.
19-3)}\label{ilcs-519-3-from-ch.-38-par.-19-3}}
\addcontentsline{toc}{subsection}{(720 ILCS 5/19-3) (from Ch. 38, par.
19-3)}

\hypertarget{sec.-19-3.-residential-burglary.}{%
\section*{Sec. 19-3. Residential
burglary.}\label{sec.-19-3.-residential-burglary.}}
\addcontentsline{toc}{section}{Sec. 19-3. Residential burglary.}

\markright{Sec. 19-3. Residential burglary.}

(a) A person commits residential burglary when he or she knowingly and
without authority enters or knowingly and without authority remains
within the dwelling place of another, or any part thereof, with the
intent to commit therein a felony or theft. This offense includes the
offense of burglary as defined in Section 19-1.

(a-5) A person commits residential burglary when he or she falsely
represents himself or herself, including but not limited to falsely
representing himself or herself to be a representative of any unit of
government or a construction, telecommunications, or utility company,
for the purpose of gaining entry to the dwelling place of another, with
the intent to commit therein a felony or theft or to facilitate the
commission therein of a felony or theft by another.

(b) Sentence. Residential burglary is a Class 1 felony.

(Source: P.A. 96-1113, eff. 1-1-11; 97-1108, eff. 1-1-13.)

\hypertarget{ilcs-519-4-from-ch.-38-par.-19-4}{%
\subsection*{(720 ILCS 5/19-4) (from Ch. 38, par.
19-4)}\label{ilcs-519-4-from-ch.-38-par.-19-4}}
\addcontentsline{toc}{subsection}{(720 ILCS 5/19-4) (from Ch. 38, par.
19-4)}

\hypertarget{sec.-19-4.-criminal-trespass-to-a-residence.}{%
\section*{Sec. 19-4. Criminal trespass to a
residence.}\label{sec.-19-4.-criminal-trespass-to-a-residence.}}
\addcontentsline{toc}{section}{Sec. 19-4. Criminal trespass to a
residence.}

\markright{Sec. 19-4. Criminal trespass to a residence.}

(a) (1) A person commits criminal trespass to a residence when, without
authority, he or she knowingly enters or remains within any residence,
including a house trailer that is the dwelling place of another.

(2) A person commits criminal trespass to a residence when, without
authority, he or she knowingly enters the residence of another and knows
or has reason to know that one or more persons is present or he or she
knowingly enters the residence of another and remains in the residence
after he or she knows or has reason to know that one or more persons is
present.

(a-5) For purposes of this Section, in the case of a multi-unit
residential building or complex, ``residence'' shall only include the
portion of the building or complex which is the actual dwelling place of
any person and shall not include such places as common recreational
areas or lobbies.

(b) Sentence.

(1) Criminal trespass to a residence under paragraph

(1) of subsection (a) is a Class A misdemeanor.

(2) Criminal trespass to a residence under paragraph

(2) of subsection (a) is a Class 4 felony.

(Source: P.A. 97-1108, eff. 1-1-13; 98-756, eff. 7-16-14.)

\hypertarget{ilcs-519-5-from-ch.-38-par.-19-5}{%
\subsection*{(720 ILCS 5/19-5) (from Ch. 38, par.
19-5)}\label{ilcs-519-5-from-ch.-38-par.-19-5}}
\addcontentsline{toc}{subsection}{(720 ILCS 5/19-5) (from Ch. 38, par.
19-5)}

\hypertarget{sec.-19-5.-criminal-fortification-of-a-residence-or-building.}{%
\section*{Sec. 19-5. Criminal fortification of a residence or
building.}\label{sec.-19-5.-criminal-fortification-of-a-residence-or-building.}}
\addcontentsline{toc}{section}{Sec. 19-5. Criminal fortification of a
residence or building.}

\markright{Sec. 19-5. Criminal fortification of a residence or
building.}

(a) A person commits criminal fortification of a residence or building
when, with the intent to prevent the lawful entry of a law enforcement
officer or another, he or she maintains a residence or building in a
fortified condition, knowing that the residence or building is used for
the unlawful manufacture, storage with intent to deliver or manufacture,
delivery, or trafficking of cannabis, controlled substances, or
methamphetamine as defined in the Cannabis Control Act, the Illinois
Controlled Substances Act, or the Methamphetamine Control and Community
Protection Act.

(b) ``Fortified condition'' means preventing or impeding entry through
the use of steel doors, wooden planking, crossbars, alarm systems, dogs,
video surveillance, motion sensing devices, booby traps, or other
similar means. If video surveillance is the sole component of the
fortified condition, the video surveillance must be with the intent to
alert an occupant to the presence of a law enforcement officer for the
purpose of interfering with the official duties of a law enforcement
officer, allowing removal or destruction of evidence, or facilitating
the infliction of harm to a law enforcement officer. For the purposes of
this Section, ``booby trap'' means any device, including but not limited
to any explosive device, designed to cause physical injury or the
destruction of evidence, when triggered by an act of a person
approaching, entering, or moving through a structure.

(c) Sentence. Criminal fortification of a residence or building is a
Class 3 felony.

(d) This Section does not apply to the fortification of a residence or
building used in the manufacture of methamphetamine as described in
Sections 10 and 15 of the Methamphetamine Control and Community
Protection Act.

(Source: P.A. 98-897, eff. 1-1-15 .)

\hypertarget{ilcs-519-6}{%
\subsection*{(720 ILCS 5/19-6)}\label{ilcs-519-6}}
\addcontentsline{toc}{subsection}{(720 ILCS 5/19-6)}

(was 720 ILCS 5/12-11)

\hypertarget{sec.-19-6.-home-invasion.}{%
\section*{Sec. 19-6. Home Invasion.}\label{sec.-19-6.-home-invasion.}}
\addcontentsline{toc}{section}{Sec. 19-6. Home Invasion.}

\markright{Sec. 19-6. Home Invasion.}

(a) A person who is not a peace officer acting in the line of duty
commits home invasion when without authority he or she knowingly enters
the dwelling place of another when he or she knows or has reason to know
that one or more persons is present or he or she knowingly enters the
dwelling place of another and remains in the dwelling place until he or
she knows or has reason to know that one or more persons is present or
who falsely represents himself or herself, including but not limited to,
falsely representing himself or herself to be a representative of any
unit of government or a construction, telecommunications, or utility
company, for the purpose of gaining entry to the dwelling place of
another when he or she knows or has reason to know that one or more
persons are present and

(1) While armed with a dangerous weapon, other than a firearm, uses
force or threatens the imminent use of force upon any person or persons
within the dwelling place whether or not injury occurs, or

(2) Intentionally causes any injury, except as provided in subsection
(a)(5), to any person or persons within the dwelling place, or

(3) While armed with a firearm uses force or threatens the imminent use
of force upon any person or persons within the dwelling place whether or
not injury occurs, or

(4) Uses force or threatens the imminent use of force upon any person or
persons within the dwelling place whether or not injury occurs and
during the commission of the offense personally discharges a firearm, or

(5) Personally discharges a firearm that proximately causes great bodily
harm, permanent disability, permanent disfigurement, or death to another
person within the dwelling place, or

(6) Commits, against any person or persons within that dwelling place, a
violation of Section 11-1.20, 11-1.30, 11-1.40, 11-1.50, or 11-1.60 of
this Code.

(b) It is an affirmative defense to a charge of home invasion that the
accused who knowingly enters the dwelling place of another and remains
in the dwelling place until he or she knows or has reason to know that
one or more persons is present either immediately leaves the premises or
surrenders to the person or persons lawfully present therein without
either attempting to cause or causing serious bodily injury to any
person present therein.

(c) Sentence. Home invasion in violation of subsection (a)(1), (a)(2) or
(a)(6) is a Class X felony. A violation of subsection (a)(3) is a Class
X felony for which 15 years shall be added to the term of imprisonment
imposed by the court. A violation of subsection (a)(4) is a Class X
felony for which 20 years shall be added to the term of imprisonment
imposed by the court. A violation of subsection (a)(5) is a Class X
felony for which 25 years or up to a term of natural life shall be added
to the term of imprisonment imposed by the court.

(d) For purposes of this Section, ``dwelling place of another'' includes
a dwelling place where the defendant maintains a tenancy interest but
from which the defendant has been barred by a divorce decree, judgment
of dissolution of marriage, order of protection, or other court order.

(Source: P.A. 96-1113, eff. 1-1-11; 96-1551, eff. 7-1-11; 97-1108, eff.
1-1-13; 97-1150, eff. 1-25-13.)

\bookmarksetup{startatroot}

\hypertarget{article-20.-arson}{%
\chapter*{Article 20. Arson}\label{article-20.-arson}}
\addcontentsline{toc}{chapter}{Article 20. Arson}

\markboth{Article 20. Arson}{Article 20. Arson}

\hypertarget{ilcs-520-1-from-ch.-38-par.-20-1}{%
\subsection*{(720 ILCS 5/20-1) (from Ch. 38, par.
20-1)}\label{ilcs-520-1-from-ch.-38-par.-20-1}}
\addcontentsline{toc}{subsection}{(720 ILCS 5/20-1) (from Ch. 38, par.
20-1)}

\hypertarget{sec.-20-1.-arson-residential-arson-place-of-worship-arson.}{%
\section*{Sec. 20-1. Arson; residential arson; place of worship
arson.}\label{sec.-20-1.-arson-residential-arson-place-of-worship-arson.}}
\addcontentsline{toc}{section}{Sec. 20-1. Arson; residential arson;
place of worship arson.}

\markright{Sec. 20-1. Arson; residential arson; place of worship arson.}

(a) A person commits arson when, by means of fire or explosive, he or
she knowingly:

(1) Damages any real property, or any personal property having a value
of \$150 or more, of another without his or her consent; or

(2) With intent to defraud an insurer, damages any property or any
personal property having a value of \$150 or more.

Property ``of another'' means a building or other property, whether real
or personal, in which a person other than the offender has an interest
which the offender has no authority to defeat or impair, even though the
offender may also have an interest in the building or property.

(b) A person commits residential arson when he or she, in the course of
committing arson, knowingly damages, partially or totally, any building
or structure that is the dwelling place of another.

(b-5) A person commits place of worship arson when he or she, in the
course of committing arson, knowingly damages, partially or totally, any
place of worship.

(c) Sentence.

Arson is a Class 2 felony. Residential arson or place of worship arson
is a Class 1 felony.

(Source: P.A. 97-1108, eff. 1-1-13.)

\hypertarget{ilcs-520-1.1-from-ch.-38-par.-20-1.1}{%
\subsection*{(720 ILCS 5/20-1.1) (from Ch. 38, par.
20-1.1)}\label{ilcs-520-1.1-from-ch.-38-par.-20-1.1}}
\addcontentsline{toc}{subsection}{(720 ILCS 5/20-1.1) (from Ch. 38, par.
20-1.1)}

\hypertarget{sec.-20-1.1.-aggravated-arson.}{%
\section*{Sec. 20-1.1. Aggravated
Arson.}\label{sec.-20-1.1.-aggravated-arson.}}
\addcontentsline{toc}{section}{Sec. 20-1.1. Aggravated Arson.}

\markright{Sec. 20-1.1. Aggravated Arson.}

(a) A person commits aggravated arson when in the course of committing
arson he or she knowingly damages, partially or totally, any building or
structure, including any adjacent building or structure, including all
or any part of a school building, house trailer, watercraft, motor
vehicle, or railroad car, and (1) he knows or reasonably should know
that one or more persons are present therein or (2) any person suffers
great bodily harm, or permanent disability or disfigurement as a result
of the fire or explosion or (3) a fireman, policeman, or correctional
officer who is present at the scene acting in the line of duty is
injured as a result of the fire or explosion. For purposes of this
Section, property ``of another'' means a building or other property,
whether real or personal, in which a person other than the offender has
an interest that the offender has no authority to defeat or impair, even
though the offender may also have an interest in the building or
property; and ``school building'' means any public or private preschool,
elementary or secondary school, community college, college, or
university.

(b) Sentence. Aggravated arson is a Class X felony.

(Source: P.A. 93-335, eff. 7-24-03; 94-127, eff. 7-7-05; 94-393, eff.
8-1-05.)

\hypertarget{ilcs-520-1.2}{%
\subsection*{(720 ILCS 5/20-1.2)}\label{ilcs-520-1.2}}
\addcontentsline{toc}{subsection}{(720 ILCS 5/20-1.2)}

\hypertarget{sec.-20-1.2.-repealed.}{%
\section*{Sec. 20-1.2. (Repealed).}\label{sec.-20-1.2.-repealed.}}
\addcontentsline{toc}{section}{Sec. 20-1.2. (Repealed).}

\markright{Sec. 20-1.2. (Repealed).}

(Source: P.A. 90-787, eff. 8-14-98. Repealed by P.A. 97-1108, eff.
1-1-13.)

\hypertarget{ilcs-520-1.3}{%
\subsection*{(720 ILCS 5/20-1.3)}\label{ilcs-520-1.3}}
\addcontentsline{toc}{subsection}{(720 ILCS 5/20-1.3)}

\hypertarget{sec.-20-1.3.-repealed.}{%
\section*{Sec. 20-1.3. (Repealed).}\label{sec.-20-1.3.-repealed.}}
\addcontentsline{toc}{section}{Sec. 20-1.3. (Repealed).}

\markright{Sec. 20-1.3. (Repealed).}

(Source: P.A. 93-169, eff. 7-10-03. Repealed by P.A. 97-1108, eff.
1-1-13.)

\hypertarget{ilcs-520-1.4}{%
\subsection*{(720 ILCS 5/20-1.4)}\label{ilcs-520-1.4}}
\addcontentsline{toc}{subsection}{(720 ILCS 5/20-1.4)}

\hypertarget{sec.-20-1.4.-repealed.}{%
\section*{Sec. 20-1.4. (Repealed).}\label{sec.-20-1.4.-repealed.}}
\addcontentsline{toc}{section}{Sec. 20-1.4. (Repealed).}

\markright{Sec. 20-1.4. (Repealed).}

(Source: P.A. 93-969, eff. 1-1-05. Repealed by P.A. 94-556, eff.
9-11-2005 .)

\hypertarget{ilcs-520-1.5}{%
\subsection*{(720 ILCS 5/20-1.5)}\label{ilcs-520-1.5}}
\addcontentsline{toc}{subsection}{(720 ILCS 5/20-1.5)}

\hypertarget{sec.-20-1.5.-repealed.}{%
\section*{Sec. 20-1.5. (Repealed).}\label{sec.-20-1.5.-repealed.}}
\addcontentsline{toc}{section}{Sec. 20-1.5. (Repealed).}

\markright{Sec. 20-1.5. (Repealed).}

(Source: P.A. 93-969, eff. 1-1-05. Repealed by P.A. 94-556, eff.
9-11-2005.)

\hypertarget{ilcs-520-2-from-ch.-38-par.-20-2}{%
\subsection*{(720 ILCS 5/20-2) (from Ch. 38, par.
20-2)}\label{ilcs-520-2-from-ch.-38-par.-20-2}}
\addcontentsline{toc}{subsection}{(720 ILCS 5/20-2) (from Ch. 38, par.
20-2)}

\hypertarget{sec.-20-2.-possession-of-explosives-or-explosive-or-incendiary-devices.}{%
\section*{Sec. 20-2. Possession of explosives or explosive or incendiary
devices.}\label{sec.-20-2.-possession-of-explosives-or-explosive-or-incendiary-devices.}}
\addcontentsline{toc}{section}{Sec. 20-2. Possession of explosives or
explosive or incendiary devices.}

\markright{Sec. 20-2. Possession of explosives or explosive or
incendiary devices.}

(a) A person commits possession of explosives or explosive or incendiary
devices in violation of this Section when he or she possesses,
manufactures or transports any explosive compound, timing or detonating
device for use with any explosive compound or incendiary device and
either intends to use the explosive or device to commit any offense or
knows that another intends to use the explosive or device to commit a
felony.

(b) Sentence.

Possession of explosives or explosive or incendiary devices is a Class 1
felony for which a person, if sentenced to a term of imprisonment, shall
be sentenced to not less than 4 years and not more than 30 years.

(c) (Blank).

(Source: P.A. 97-1108, eff. 1-1-13.)

\bookmarksetup{startatroot}

\hypertarget{article-20.5.-causing-a-catastrophe-deadly-substances}{%
\chapter*{Article 20.5. Causing A Catastrophe; Deadly
Substances}\label{article-20.5.-causing-a-catastrophe-deadly-substances}}
\addcontentsline{toc}{chapter}{Article 20.5. Causing A Catastrophe;
Deadly Substances}

\markboth{Article 20.5. Causing A Catastrophe; Deadly
Substances}{Article 20.5. Causing A Catastrophe; Deadly Substances}

\hypertarget{ilcs-520.5-5}{%
\subsection*{(720 ILCS 5/20.5-5)}\label{ilcs-520.5-5}}
\addcontentsline{toc}{subsection}{(720 ILCS 5/20.5-5)}

\hypertarget{sec.-20.5-5.-renumbered.}{%
\section*{Sec. 20.5-5. (Renumbered).}\label{sec.-20.5-5.-renumbered.}}
\addcontentsline{toc}{section}{Sec. 20.5-5. (Renumbered).}

\markright{Sec. 20.5-5. (Renumbered).}

(Source: Renumbered by P.A. 96-710, eff. 1-1-10.)

\hypertarget{ilcs-520.5-6}{%
\subsection*{(720 ILCS 5/20.5-6)}\label{ilcs-520.5-6}}
\addcontentsline{toc}{subsection}{(720 ILCS 5/20.5-6)}

\hypertarget{sec.-20.5-6.-renumbered.}{%
\section*{Sec. 20.5-6. (Renumbered).}\label{sec.-20.5-6.-renumbered.}}
\addcontentsline{toc}{section}{Sec. 20.5-6. (Renumbered).}

\markright{Sec. 20.5-6. (Renumbered).}

(Source: Renumbered by P.A. 96-710, eff. 1-1-10.)

\bookmarksetup{startatroot}

\hypertarget{article-21.-damage-and-trespass-to-property}{%
\chapter*{Article 21. Damage And Trespass To
Property}\label{article-21.-damage-and-trespass-to-property}}
\addcontentsline{toc}{chapter}{Article 21. Damage And Trespass To
Property}

\markboth{Article 21. Damage And Trespass To Property}{Article 21.
Damage And Trespass To Property}

\hypertarget{ilcs-5art.-21-subdiv.-1-heading}{%
\subsection*{(720 ILCS 5/Art. 21, Subdiv. 1
heading)}\label{ilcs-5art.-21-subdiv.-1-heading}}
\addcontentsline{toc}{subsection}{(720 ILCS 5/Art. 21, Subdiv. 1
heading)}

SUBDIVISION 1.

DAMAGE TO PROPERTY

(Source: P.A. 97-1108, eff. 1-1-13.)

\hypertarget{ilcs-521-1-from-ch.-38-par.-21-1}{%
\subsection*{(720 ILCS 5/21-1) (from Ch. 38, par.
21-1)}\label{ilcs-521-1-from-ch.-38-par.-21-1}}
\addcontentsline{toc}{subsection}{(720 ILCS 5/21-1) (from Ch. 38, par.
21-1)}

\hypertarget{sec.-21-1.-criminal-damage-to-property.}{%
\section*{Sec. 21-1. Criminal damage to
property.}\label{sec.-21-1.-criminal-damage-to-property.}}
\addcontentsline{toc}{section}{Sec. 21-1. Criminal damage to property.}

\markright{Sec. 21-1. Criminal damage to property.}

(a) A person commits criminal damage to property when he or she:

(1) knowingly damages any property of another;

(2) recklessly by means of fire or explosive damages property of
another;

(3) knowingly starts a fire on the land of another;

(4) knowingly injures a domestic animal of another without his or her
consent;

(5) knowingly deposits on the land or in the building of another any
stink bomb or any offensive smelling compound and thereby intends to
interfere with the use by another of the land or building;

(6) knowingly damages any property, other than as described in paragraph
(2) of subsection (a) of Section 20-1, with intent to defraud an
insurer;

(7) knowingly shoots a firearm at any portion of a railroad train;

(8) knowingly, without proper authorization, cuts, injures, damages,
defaces, destroys, or tampers with any fire hydrant or any public or
private fire fighting equipment, or any apparatus appertaining to fire
fighting equipment; or

(9) intentionally, without proper authorization, opens any fire hydrant.

(b) When the charge of criminal damage to property exceeding a specified
value is brought, the extent of the damage is an element of the offense
to be resolved by the trier of fact as either exceeding or not exceeding
the specified value.

(c) It is an affirmative defense to a violation of paragraph (1), (3),
or (5) of subsection (a) of this Section that the owner of the property
or land damaged consented to the damage.

(d) Sentence.

(1) A violation of subsection (a) shall have the following penalties:

(A) A violation of paragraph (8) or (9) is a Class B misdemeanor.

(B) A violation of paragraph (1), (2), (3), (5), or (6) is a Class A
misdemeanor when the damage to property does not exceed \$500.

(C) A violation of paragraph (1), (2), (3), (5), or (6) is a Class 4
felony when the damage to property does not exceed \$500 and the damage
occurs to property of a school or place of worship or to farm equipment
or immovable items of agricultural production, including but not limited
to grain elevators, grain bins, and barns or property which memorializes
or honors an individual or group of police officers, fire fighters,
members of the United States Armed Forces, National Guard, or veterans.

(D) A violation of paragraph (4) is a Class 4 felony when the damage to
property does not exceed \$10,000.

(E) A violation of paragraph (7) is a Class 4 felony.

(F) A violation of paragraph (1), (2), (3), (5) or (6) is a Class 4
felony when the damage to property exceeds \$500 but does not exceed
\$10,000.

(G) A violation of paragraphs (1) through (6) is a Class 3 felony when
the damage to property exceeds \$500 but does not exceed \$10,000 and
the damage occurs to property of a school or place of worship or to farm
equipment or immovable items of agricultural production, including but
not limited to grain elevators, grain bins, and barns or property which
memorializes or honors an individual or group of police officers, fire
fighters, members of the United States Armed Forces, National Guard, or
veterans.

(H) A violation of paragraphs (1) through (6) is a Class 3 felony when
the damage to property exceeds \$10,000 but does not exceed \$100,000.

(I) A violation of paragraphs (1) through (6) is a Class 2 felony when
the damage to property exceeds \$10,000 but does not exceed \$100,000
and the damage occurs to property of a school or place of worship or to
farm equipment or immovable items of agricultural production, including
but not limited to grain elevators, grain bins, and barns or property
which memorializes or honors an individual or group of police officers,
fire fighters, members of the United States Armed Forces, National
Guard, or veterans.

(J) A violation of paragraphs (1) through (6) is a Class 2 felony when
the damage to property exceeds \$100,000. A violation of paragraphs (1)
through (6) is a Class 1 felony when the damage to property exceeds
\$100,000 and the damage occurs to property of a school or place of
worship or to farm equipment or immovable items of agricultural
production, including but not limited to grain elevators, grain bins,
and barns or property which memorializes or honors an individual or
group of police officers, fire fighters, members of the United States
Armed Forces, National Guard, or veterans.

(2) When the damage to property exceeds \$10,000, the court shall impose
upon the offender a fine equal to the value of the damages to the
property.

(3) In addition to any other sentence that may be imposed, a court shall
order any person convicted of criminal damage to property to perform
community service for not less than 30 and not more than 120 hours, if
community service is available in the jurisdiction and is funded and
approved by the county board of the county where the offense was
committed. In addition, whenever any person is placed on supervision for
an alleged offense under this Section, the supervision shall be
conditioned upon the performance of the community service.

The community service requirement does not apply when the court imposes
a sentence of incarceration.

(4) In addition to any criminal penalties imposed for a violation of
this Section, if a person is convicted of or placed on supervision for
knowingly damaging or destroying crops of another, including crops
intended for personal, commercial, research, or developmental purposes,
the person is liable in a civil action to the owner of any crops damaged
or destroyed for money damages up to twice the market value of the crops
damaged or destroyed.

(5) For the purposes of this subsection (d), ``farm equipment'' means
machinery or other equipment used in farming.

(Source: P.A. 98-315, eff. 1-1-14; 99-631, eff. 1-1-17 .)

\hypertarget{ilcs-521-1.01}{%
\subsection*{(720 ILCS 5/21-1.01)}\label{ilcs-521-1.01}}
\addcontentsline{toc}{subsection}{(720 ILCS 5/21-1.01)}

(was 720 ILCS 5/21-4)

\hypertarget{sec.-21-1.01.-criminal-damage-to-government-supported-property.}{%
\section*{Sec. 21-1.01. Criminal Damage to Government Supported
Property.}\label{sec.-21-1.01.-criminal-damage-to-government-supported-property.}}
\addcontentsline{toc}{section}{Sec. 21-1.01. Criminal Damage to
Government Supported Property.}

\markright{Sec. 21-1.01. Criminal Damage to Government Supported
Property.}

(a) A person commits criminal damage to government supported property
when he or she knowingly:

(1) damages any government supported property without the consent of the
State;

(2) by means of fire or explosive damages government supported property;

(3) starts a fire on government supported property without the consent
of the State; or

(4) deposits on government supported land or in a government supported
building, without the consent of the State, any stink bomb or any
offensive smelling compound and thereby intends to interfere with the
use by another of the land or building.

(b) For the purposes of this Section, ``government supported'' means any
property supported in whole or in part with State funds, funds of a unit
of local government or school district, or federal funds administered or
granted through State agencies.

(c) Sentence. A violation of this Section is a Class 4 felony when the
damage to property is \$500 or less; a Class 3 felony when the damage to
property exceeds \$500 but does not exceed \$10,000; a Class 2 felony
when the damage to property exceeds \$10,000 but does not exceed
\$100,000; and a Class 1 felony when the damage to property exceeds
\$100,000. When the damage to property exceeds \$10,000, the court shall
impose upon the offender a fine equal to the value of the damages to the
property.

(Source: P.A. 97-1108, eff. 1-1-13.)

\hypertarget{ilcs-521-1.1-from-ch.-38-par.-21-1.1}{%
\subsection*{(720 ILCS 5/21-1.1) (from Ch. 38, par.
21-1.1)}\label{ilcs-521-1.1-from-ch.-38-par.-21-1.1}}
\addcontentsline{toc}{subsection}{(720 ILCS 5/21-1.1) (from Ch. 38, par.
21-1.1)}

\hypertarget{sec.-21-1.1.-repealed.}{%
\section*{Sec. 21-1.1. (Repealed).}\label{sec.-21-1.1.-repealed.}}
\addcontentsline{toc}{section}{Sec. 21-1.1. (Repealed).}

\markright{Sec. 21-1.1. (Repealed).}

(Source: P.A. 78-255. Repealed by P.A. 97-1108, eff. 1-1-13.)

\hypertarget{ilcs-521-1.2-from-ch.-38-par.-21-1.2}{%
\subsection*{(720 ILCS 5/21-1.2) (from Ch. 38, par.
21-1.2)}\label{ilcs-521-1.2-from-ch.-38-par.-21-1.2}}
\addcontentsline{toc}{subsection}{(720 ILCS 5/21-1.2) (from Ch. 38, par.
21-1.2)}

\hypertarget{sec.-21-1.2.-institutional-vandalism.}{%
\section*{Sec. 21-1.2. Institutional
vandalism.}\label{sec.-21-1.2.-institutional-vandalism.}}
\addcontentsline{toc}{section}{Sec. 21-1.2. Institutional vandalism.}

\markright{Sec. 21-1.2. Institutional vandalism.}

(a) A person commits institutional vandalism when, by reason of the
actual or perceived race, color, creed, religion, ancestry, gender,
sexual orientation, physical or mental disability, or national origin of
another individual or group of individuals, regardless of the existence
of any other motivating factor or factors, he or she knowingly and
without consent inflicts damage to any of the following properties:

(1) A church, synagogue, mosque, or other building, structure or place
used for religious worship or other religious purpose;

(2) A cemetery, mortuary, or other facility used for the purpose of
burial or memorializing the dead;

(3) A school, educational facility or community center;

(4) The grounds adjacent to, and owned or rented by, any institution,
facility, building, structure or place described in paragraphs (1), (2)
or (3) of this subsection (a); or

(5) Any personal property contained in any institution, facility,
building, structure or place described in paragraphs (1), (2) or (3) of
this subsection (a).

(b) Sentence.

(1) Institutional vandalism is a Class 3 felony when the damage to the
property does not exceed \$500. Institutional vandalism is a Class 2
felony when the damage to the property exceeds \$500. Institutional
vandalism is a Class 2 felony for any second or subsequent offense.

(2) Upon imposition of any sentence, the trial court shall also either
order restitution paid to the victim or impose a fine up to \$1,000. In
addition, any order of probation or conditional discharge entered
following a conviction or an adjudication of delinquency shall include a
condition that the offender perform public or community service of no
less than 200 hours if that service is established in the county where
the offender was convicted of institutional vandalism. The court may
also impose any other condition of probation or conditional discharge
under this Section.

(c) Independent of any criminal prosecution or the result of that
prosecution, a person suffering damage to property or injury to his or
her person as a result of institutional vandalism may bring a civil
action for damages, injunction or other appropriate relief. The court
may award actual damages, including damages for emotional distress, or
punitive damages. A judgment may include attorney's fees and costs. The
parents or legal guardians of an unemancipated minor, other than
guardians appointed under the Juvenile Court Act or the Juvenile Court
Act of 1987, shall be liable for the amount of any judgment for actual
damages rendered against the minor under this subsection in an amount
not exceeding the amount provided under Section 5 of the Parental
Responsibility Law.

(d) As used in this Section, ``sexual orientation'' has the meaning
ascribed to it in paragraph (O-1) of Section 1-103 of the Illinois Human
Rights Act.

(Source: P.A. 99-77, eff. 1-1-16; 99-631, eff. 1-1-17 .)

\hypertarget{ilcs-521-1.3}{%
\subsection*{(720 ILCS 5/21-1.3)}\label{ilcs-521-1.3}}
\addcontentsline{toc}{subsection}{(720 ILCS 5/21-1.3)}

\hypertarget{sec.-21-1.3.-criminal-defacement-of-property.}{%
\section*{Sec. 21-1.3. Criminal defacement of
property.}\label{sec.-21-1.3.-criminal-defacement-of-property.}}
\addcontentsline{toc}{section}{Sec. 21-1.3. Criminal defacement of
property.}

\markright{Sec. 21-1.3. Criminal defacement of property.}

(a) A person commits criminal defacement of property when the person
knowingly damages the property of another by defacing, deforming, or
otherwise damaging the property by the use of paint or any other similar
substance, or by the use of a writing instrument, etching tool, or any
other similar device. It is an affirmative defense to a violation of
this Section that the owner of the property damaged consented to such
damage.

(b) Sentence.

(1) Criminal defacement of property is a Class A misdemeanor for a first
offense when the aggregate value of the damage to the property does not
exceed \$500. Criminal defacement of property is a Class 4 felony when
the aggregate value of the damage to property does not exceed \$500 and
the property damaged is a school building or place of worship or
property which memorializes or honors an individual or group of police
officers, fire fighters, members of the United States Armed Forces or
National Guard, or veterans. Criminal defacement of property is a Class
4 felony for a second or subsequent conviction or when the aggregate
value of the damage to the property exceeds \$500. Criminal defacement
of property is a Class 3 felony when the aggregate value of the damage
to property exceeds \$500 and the property damaged is a school building
or place of worship or property which memorializes or honors an
individual or group of police officers, fire fighters, members of the
United States Armed Forces or National Guard, or veterans.

(2) In addition to any other sentence that may be imposed for a
violation of this Section, a person convicted of criminal defacement of
property shall:

(A) pay the actual costs incurred by the property owner or the unit of
government to abate, remediate, repair, or remove the effect of the
damage to the property. To the extent permitted by law, reimbursement
for the costs of abatement, remediation, repair, or removal shall be
payable to the person who incurred the costs; and

(B) if convicted of criminal defacement of property that is chargeable
as a Class 3 or Class 4 felony, pay a mandatory minimum fine of \$500.

(3) In addition to any other sentence that may be imposed, a court shall
order any person convicted of criminal defacement of property to perform
community service for not less than 30 and not more than 120 hours, if
community service is available in the jurisdiction. The community
service shall include, but need not be limited to, the cleanup and
repair of the damage to property that was caused by the offense, or
similar damage to property located in the municipality or county in
which the offense occurred. When the property damaged is a school
building, the community service may include cleanup, removal, or
painting over the defacement. In addition, whenever any person is placed
on supervision for an alleged offense under this Section, the
supervision shall be conditioned upon the performance of the community
service.

(4) For the purposes of this subsection (b), aggregate value shall be
determined by adding the value of the damage to one or more properties
if the offenses were committed as part of a single course of conduct.

(Source: P.A. 98-315, eff. 1-1-14; 98-466, eff. 8-16-13; 98-756, eff.
7-16-14; 99-631, eff. 1-1-17 .)

\hypertarget{ilcs-521-1.4}{%
\subsection*{(720 ILCS 5/21-1.4)}\label{ilcs-521-1.4}}
\addcontentsline{toc}{subsection}{(720 ILCS 5/21-1.4)}

\hypertarget{sec.-21-1.4.-jackrocks-violation.}{%
\section*{Sec. 21-1.4. Jackrocks
violation.}\label{sec.-21-1.4.-jackrocks-violation.}}
\addcontentsline{toc}{section}{Sec. 21-1.4. Jackrocks violation.}

\markright{Sec. 21-1.4. Jackrocks violation.}

(a) A person commits a jackrocks violation when he or she knowingly:

(1) sells, gives away, manufactures, purchases, or possesses a jackrock;
or

(2) places, tosses, or throws a jackrock on public or private property.

(b) As used in this Section, ``jackrock'' means a caltrop or other
object manufactured with one or more rounded or sharpened points, which
when placed or thrown present at least one point at such an angle that
it is peculiar to and designed for use in puncturing or damaging vehicle
tires. It does not include a device designed to puncture or damage the
tires of a vehicle driven over it in a particular direction, if a
conspicuous and clearly visible warning is posted at the device's
location, alerting persons to its presence.

(c) This Section does not apply to the possession, transfer, or use of
jackrocks by any law enforcement officer in the course of his or her
official duties.

(d) Sentence. A jackrocks violation is a Class A misdemeanor.

(Source: P.A. 97-1108, eff. 1-1-13.)

\hypertarget{ilcs-521-1.5}{%
\subsection*{(720 ILCS 5/21-1.5)}\label{ilcs-521-1.5}}
\addcontentsline{toc}{subsection}{(720 ILCS 5/21-1.5)}

\hypertarget{sec.-21-1.5.-repealed.}{%
\section*{Sec. 21-1.5. (Repealed).}\label{sec.-21-1.5.-repealed.}}
\addcontentsline{toc}{section}{Sec. 21-1.5. (Repealed).}

\markright{Sec. 21-1.5. (Repealed).}

(Source: P.A. 93-596, eff. 8-26-03. Repealed by P.A. 94-556, eff.
9-11-05.)

\hypertarget{ilcs-5art.-21-subdiv.-5-heading}{%
\subsection*{(720 ILCS 5/Art. 21, Subdiv. 5
heading)}\label{ilcs-5art.-21-subdiv.-5-heading}}
\addcontentsline{toc}{subsection}{(720 ILCS 5/Art. 21, Subdiv. 5
heading)}

SUBDIVISION 5.

TRESPASS

(Source: P.A. 97-1108, eff. 1-1-13.)

\hypertarget{ilcs-521-2-from-ch.-38-par.-21-2}{%
\subsection*{(720 ILCS 5/21-2) (from Ch. 38, par.
21-2)}\label{ilcs-521-2-from-ch.-38-par.-21-2}}
\addcontentsline{toc}{subsection}{(720 ILCS 5/21-2) (from Ch. 38, par.
21-2)}

\hypertarget{sec.-21-2.-criminal-trespass-to-vehicles.}{%
\section*{Sec. 21-2. Criminal trespass to
vehicles.}\label{sec.-21-2.-criminal-trespass-to-vehicles.}}
\addcontentsline{toc}{section}{Sec. 21-2. Criminal trespass to
vehicles.}

\markright{Sec. 21-2. Criminal trespass to vehicles.}

(a) A person commits criminal trespass to vehicles when he or she
knowingly and without authority enters any part of or operates any
vehicle, aircraft, watercraft or snowmobile.

(b) Sentence. Criminal trespass to vehicles is a Class A misdemeanor.

(Source: P.A. 97-1108, eff. 1-1-13.)

\hypertarget{ilcs-521-2.5}{%
\subsection*{(720 ILCS 5/21-2.5)}\label{ilcs-521-2.5}}
\addcontentsline{toc}{subsection}{(720 ILCS 5/21-2.5)}

\hypertarget{sec.-21-2.5.-electronic-tracking-devices-prohibited.}{%
\section*{Sec. 21-2.5. Electronic tracking devices
prohibited.}\label{sec.-21-2.5.-electronic-tracking-devices-prohibited.}}
\addcontentsline{toc}{section}{Sec. 21-2.5. Electronic tracking devices
prohibited.}

\markright{Sec. 21-2.5. Electronic tracking devices prohibited.}

(a) As used in this Section:

``Electronic tracking device'' means any device attached to a vehicle
that reveals its location or movement by the transmission of electronic
signals.

``State agency'' means all departments, officers, commissions, boards,
institutions, and bodies politic and corporate of the State. The term,
however, does not mean the judicial branch, including, without
limitation, the several courts of the State, the offices of the clerk of
the supreme court and the clerks of the appellate court, and the
Administrative Office of the Illinois Courts, nor does it mean the
legislature or its committees or commissions.

``Telematics'' includes, but is not limited to, automatic airbag
deployment and crash notification, remote diagnostics, navigation,
stolen vehicle location, remote door unlock, transmitting emergency and
vehicle location information to public safety answering points, and any
other service integrating vehicle location technology and wireless
communications.

``Vehicle'' has the meaning ascribed to it in Section

1-217 of the Illinois Vehicle Code.

(b) A person or entity in this State may not use an electronic tracking
device to determine the location or movement of a person.

(c) This Section does not apply:

(1) when the registered owner, lessor, or lessee of a vehicle has
consented to the use of the electronic tracking device with respect to
that vehicle;

(2) to the lawful use of an electronic tracking device by a law
enforcement agency;

(3) when the vehicle is owned or leased by a business that is authorized
to transact business in this State and the tracking device is used by
the business for the purpose of tracking vehicles driven by employees of
that business, its affiliates, or contractors of that business or its
affiliates;

(4) when the vehicle is under the control of a State agency and the
electronic tracking device is used by the agency, or the Inspector
General appointed under the State Officials and Employees Ethics Act who
has jurisdiction over that State agency, for the purpose of tracking
vehicles driven by employees or contractors of that State agency; or

(5) telematic services that were installed by the manufacturer, or
installed by or with the consent of the owner or lessee of the vehicle
and to which the owner or lessee has subscribed. Consent by the owner or
lessee of the vehicle constitutes consent for any other driver or
passenger of that vehicle.

(d) Sentence. A violation of this Section is a Class A misdemeanor.

(Source: P.A. 98-381, eff. 1-1-14.)

\hypertarget{ilcs-521-3-from-ch.-38-par.-21-3}{%
\subsection*{(720 ILCS 5/21-3) (from Ch. 38, par.
21-3)}\label{ilcs-521-3-from-ch.-38-par.-21-3}}
\addcontentsline{toc}{subsection}{(720 ILCS 5/21-3) (from Ch. 38, par.
21-3)}

\hypertarget{sec.-21-3.-criminal-trespass-to-real-property.}{%
\section*{Sec. 21-3. Criminal trespass to real
property.}\label{sec.-21-3.-criminal-trespass-to-real-property.}}
\addcontentsline{toc}{section}{Sec. 21-3. Criminal trespass to real
property.}

\markright{Sec. 21-3. Criminal trespass to real property.}

(a) A person commits criminal trespass to real property when he or she:

(1) knowingly and without lawful authority enters or remains within or
on a building;

(2) enters upon the land of another, after receiving, prior to the
entry, notice from the owner or occupant that the entry is forbidden;

(3) remains upon the land of another, after receiving notice from the
owner or occupant to depart;

(3.5) presents false documents or falsely represents his or her identity
orally to the owner or occupant of a building or land in order to obtain
permission from the owner or occupant to enter or remain in the building
or on the land;

(3.7) intentionally removes a notice posted on residential real estate
as required by subsection (l) of Section 15-1505.8 of Article XV of the
Code of Civil Procedure before the date and time set forth in the
notice; or

(4) enters a field used or capable of being used for growing crops, an
enclosed area containing livestock, an agricultural building containing
livestock, or an orchard in or on a motor vehicle (including an off-road
vehicle, motorcycle, moped, or any other powered two-wheel vehicle)
after receiving, prior to the entry, notice from the owner or occupant
that the entry is forbidden or remains upon or in the area after
receiving notice from the owner or occupant to depart.

For purposes of item (1) of this subsection, this Section shall not
apply to being in a building which is open to the public while the
building is open to the public during its normal hours of operation; nor
shall this Section apply to a person who enters a public building under
the reasonable belief that the building is still open to the public.

(b) A person has received notice from the owner or occupant within the
meaning of Subsection (a) if he or she has been notified personally,
either orally or in writing including a valid court order as defined by
subsection (7) of Section 112A-3 of the Code of Criminal Procedure of
1963 granting remedy (2) of subsection (b) of Section 112A-14 of that
Code, or if a printed or written notice forbidding such entry has been
conspicuously posted or exhibited at the main entrance to the land or
the forbidden part thereof.

(b-5) Subject to the provisions of subsection (b-10), as an alternative
to the posting of real property as set forth in subsection (b), the
owner or lessee of any real property may post the property by placing
identifying purple marks on trees or posts around the area to be posted.
Each purple mark shall be:

(1) A vertical line of at least 8 inches in length and the bottom of the
mark shall be no less than 3 feet nor more than 5 feet high. Such marks
shall be placed no more than 100 feet apart and shall be readily visible
to any person approaching the property; or

(2) A post capped or otherwise marked on at least its top 2 inches. The
bottom of the cap or mark shall be not less than 3 feet but not more
than 5 feet 6 inches high. Posts so marked shall be placed not more than
36 feet apart and shall be readily visible to any person approaching the
property. Prior to applying a cap or mark which is visible from both
sides of a fence shared by different property owners or lessees, all
such owners or lessees shall concur in the decision to post their own
property.

Nothing in this subsection (b-5) shall be construed to authorize the
owner or lessee of any real property to place any purple marks on any
tree or post or to install any post or fence if doing so would violate
any applicable law, rule, ordinance, order, covenant, bylaw,
declaration, regulation, restriction, contract, or instrument.

(b-10) Any owner or lessee who marks his or her real property using the
method described in subsection (b-5) must also provide notice as
described in subsection (b) of this Section. The public of this State
shall be informed of the provisions of subsection (b-5) of this Section
by the Illinois Department of Agriculture and the Illinois Department of
Natural Resources. These Departments shall conduct an information
campaign for the general public concerning the interpretation and
implementation of subsection (b-5). The information shall inform the
public about the marking requirements and the applicability of
subsection (b-5) including information regarding the size requirements
of the markings as well as the manner in which the markings shall be
displayed. The Departments shall also include information regarding the
requirement that, until the date this subsection becomes inoperative,
any owner or lessee who chooses to mark his or her property using paint,
must also comply with one of the notice requirements listed in
subsection (b). The Departments may prepare a brochure or may
disseminate the information through agency websites. Non-governmental
organizations including, but not limited to, the Illinois Forestry
Association, Illinois Tree Farm and the Walnut Council may help to
disseminate the information regarding the requirements and applicability
of subsection (b-5) based on materials provided by the Departments. This
subsection (b-10) is inoperative on and after January 1, 2013.

(b-15) Subsections (b-5) and (b-10) do not apply to real property
located in a municipality of over 2,000,000 inhabitants.

(c) This Section does not apply to any person, whether a migrant worker
or otherwise, living on the land with permission of the owner or of his
or her agent having apparent authority to hire workers on this land and
assign them living quarters or a place of accommodations for living
thereon, nor to anyone living on the land at the request of, or by
occupancy, leasing or other agreement or arrangement with the owner or
his or her agent, nor to anyone invited by the migrant worker or other
person so living on the land to visit him or her at the place he is so
living upon the land.

(d) A person shall be exempt from prosecution under this Section if he
or she beautifies unoccupied and abandoned residential and industrial
properties located within any municipality. For the purpose of this
subsection, ``unoccupied and abandoned residential and industrial
property'' means any real estate (1) in which the taxes have not been
paid for a period of at least 2 years; and (2) which has been left
unoccupied and abandoned for a period of at least one year; and
``beautifies'' means to landscape, clean up litter, or to repair
dilapidated conditions on or to board up windows and doors.

(e) No person shall be liable in any civil action for money damages to
the owner of unoccupied and abandoned residential and industrial
property which that person beautifies pursuant to subsection (d) of this
Section.

(e-5) Mortgagee or agent of the mortgagee exceptions.

(1) A mortgagee or agent of the mortgagee shall be exempt from
prosecution for criminal trespass for entering, securing, or maintaining
an abandoned residential property.

(2) No mortgagee or agent of the mortgagee shall be liable to the
mortgagor or other owner of an abandoned residential property in any
civil action for negligence or civil trespass in connection with
entering, securing, or maintaining the abandoned residential property.

(3) For the purpose of this subsection (e-5) only,

``abandoned residential property'' means mortgaged real estate that the
mortgagee or agent of the mortgagee determines in good faith meets the
definition of abandoned residential property set forth in Section
15-1200.5 of Article XV of the Code of Civil Procedure.

(f) This Section does not prohibit a person from entering a building or
upon the land of another for emergency purposes. For purposes of this
subsection (f), ``emergency'' means a condition or circumstance in which
an individual is or is reasonably believed by the person to be in
imminent danger of serious bodily harm or in which property is or is
reasonably believed to be in imminent danger of damage or destruction.

(g) Paragraph (3.5) of subsection (a) does not apply to a peace officer
or other official of a unit of government who enters a building or land
in the performance of his or her official duties.

(h) Sentence. A violation of subdivision (a)(1), (a)(2), (a)(3), or
(a)(3.5) is a Class B misdemeanor. A violation of subdivision (a)(4) is
a Class A misdemeanor.

(i) Civil liability. A person may be liable in any civil action for
money damages to the owner of the land he or she entered upon with a
motor vehicle as prohibited under paragraph (4) of subsection (a) of
this Section. A person may also be liable to the owner for court costs
and reasonable attorney's fees. The measure of damages shall be: (i) the
actual damages, but not less than \$250, if the vehicle is operated in a
nature preserve or registered area as defined in Sections 3.11 and 3.14
of the Illinois Natural Areas Preservation Act; (ii) twice the actual
damages if the owner has previously notified the person to cease
trespassing; or (iii) in any other case, the actual damages, but not
less than \$50. If the person operating the vehicle is under the age of
16, the owner of the vehicle and the parent or legal guardian of the
minor are jointly and severally liable. For the purposes of this
subsection (i):

``Land'' includes, but is not limited to, land used for crop land,
fallow land, orchard, pasture, feed lot, timber land, prairie land, mine
spoil nature preserves and registered areas. ``Land'' does not include
driveways or private roadways upon which the owner allows the public to
drive.

``Owner'' means the person who has the right to possession of the land,
including the owner, operator or tenant.

``Vehicle'' has the same meaning as provided under

Section 1-217 of the Illinois Vehicle Code.

(j) This Section does not apply to the following persons while serving
process:

(1) a person authorized to serve process under

Section 2-202 of the Code of Civil Procedure; or

(2) a special process server appointed by the circuit court.

(Source: P.A. 97-184, eff. 7-22-11; 97-477, eff. 8-22-11; 97-813, eff.
7-13-12; 97-1108, eff. 1-1-13; 97-1164, eff. 6-1-13 .)

\hypertarget{ilcs-521-4-from-ch.-38-par.-21-4}{%
\subsection*{(720 ILCS 5/21-4) (from Ch. 38, par.
21-4)}\label{ilcs-521-4-from-ch.-38-par.-21-4}}
\addcontentsline{toc}{subsection}{(720 ILCS 5/21-4) (from Ch. 38, par.
21-4)}

(This Section was renumbered as Section 21-1.01 by P.A. 97-1108.)

\hypertarget{sec.-21-4.-renumbered.}{%
\section*{Sec. 21-4. (Renumbered).}\label{sec.-21-4.-renumbered.}}
\addcontentsline{toc}{section}{Sec. 21-4. (Renumbered).}

\markright{Sec. 21-4. (Renumbered).}

(Source: P.A. 89-30, eff. 1-1-96. Renumbered by P.A. 97-1108, eff.
1-1-13.)

\hypertarget{ilcs-521-5-from-ch.-38-par.-21-5}{%
\subsection*{(720 ILCS 5/21-5) (from Ch. 38, par.
21-5)}\label{ilcs-521-5-from-ch.-38-par.-21-5}}
\addcontentsline{toc}{subsection}{(720 ILCS 5/21-5) (from Ch. 38, par.
21-5)}

\hypertarget{sec.-21-5.-criminal-trespass-to-state-supported-land.}{%
\section*{Sec. 21-5. Criminal trespass to State supported
land.}\label{sec.-21-5.-criminal-trespass-to-state-supported-land.}}
\addcontentsline{toc}{section}{Sec. 21-5. Criminal trespass to State
supported land.}

\markright{Sec. 21-5. Criminal trespass to State supported land.}

(a) A person commits criminal trespass to State supported land when he
or she enters upon land supported in whole or in part with State funds,
or federal funds administered or granted through State agencies or any
building on the land, after receiving, prior to the entry, notice from
the State or its representative that the entry is forbidden, or remains
upon the land or in the building after receiving notice from the State
or its representative to depart, and who thereby interferes with another
person's lawful use or enjoyment of the building or land.

A person has received notice from the State within the meaning of this
subsection if he or she has been notified personally, either orally or
in writing, or if a printed or written notice forbidding entry to him or
her or a group of which he or she is a part, has been conspicuously
posted or exhibited at the main entrance to the land or the forbidden
part thereof.

(a-5) A person commits criminal trespass to State supported land when he
or she enters upon a right of way, including facilities and improvements
thereon, owned, leased, or otherwise used by a public body or district
organized under the Metropolitan Transit Authority Act, the Local Mass
Transit District Act, or the Regional Transportation Authority Act,
after receiving, prior to the entry, notice from the public body or
district, or its representative, that the entry is forbidden, or the
person remains upon the right of way after receiving notice from the
public body or district, or its representative, to depart, and in either
of these instances intends to compromise public safety by causing a
delay in transit service lasting more than 15 minutes or destroying
property.

A person has received notice from the public body or district within the
meaning of this subsection if he or she has been notified personally,
either orally or in writing, or if a printed or written notice
forbidding entry to him or her has been conspicuously posted or
exhibited at any point of entrance to the right of way or the forbidden
part of the right of way.

As used in this subsection (a-5), ``right of way'' has the meaning
ascribed to it in Section 18c-7502 of the Illinois Vehicle Code.

(b) A person commits criminal trespass to State supported land when he
or she enters upon land supported in whole or in part with State funds,
or federal funds administered or granted through State agencies or any
building on the land by presenting false documents or falsely
representing his or her identity orally to the State or its
representative in order to obtain permission from the State or its
representative to enter the building or land; or remains upon the land
or in the building by presenting false documents or falsely representing
his or her identity orally to the State or its representative in order
to remain upon the land or in the building, and who thereby interferes
with another person's lawful use or enjoyment of the building or land.

This subsection does not apply to a peace officer or other official of a
unit of government who enters upon land supported in whole or in part
with State funds, or federal funds administered or granted through State
agencies or any building on the land in the performance of his or her
official duties.

(c) Sentence. Criminal trespass to State supported land is a Class A
misdemeanor, except a violation of subsection (a-5) of this Section is a
Class A misdemeanor for a first violation and a Class 4 felony for a
second or subsequent violation.

(Source: P.A. 97-1108, eff. 1-1-13; 98-748, eff. 1-1-15 .)

\hypertarget{ilcs-521-5.5}{%
\subsection*{(720 ILCS 5/21-5.5)}\label{ilcs-521-5.5}}
\addcontentsline{toc}{subsection}{(720 ILCS 5/21-5.5)}

\hypertarget{sec.-21-5.5.-criminal-trespass-to-a-safe-school-zone.}{%
\section*{Sec. 21-5.5. Criminal trespass to a safe school
zone.}\label{sec.-21-5.5.-criminal-trespass-to-a-safe-school-zone.}}
\addcontentsline{toc}{section}{Sec. 21-5.5. Criminal trespass to a safe
school zone.}

\markright{Sec. 21-5.5. Criminal trespass to a safe school zone.}

(a) As used in this Section:

``Employee'' means a person employed by a school whose relationship with
that agency constitutes an employer-employee relationship under the
usual common law rules, and who is not an independent contractor.
``Employee'' includes, but is not limited to, a teacher, student
teacher, aide, secretary, custodial engineer, coach, or his or her
designee.

``School administrator'' means the school's principal, or his or her
designee.

``Safe school zone'' means an area that encompasses any of the following
places during regular school hours or within 60 minutes before or after
the school day or 60 minutes before or after a school-sponsored
activity. This shall include any school property, ground, or street,
sidewalk, or public way immediately adjacent thereto and any public
right-of-way situated immediately adjacent to school property. The safe
school zone shall not include any portion of the highway not actually on
school property.

``School activity'' means and includes any school session, any
extracurricular activity or event sponsored by or participated in by the
school, and the 60-minute periods immediately preceding and following
any session, activity, or event.

``Student'' means any person enrolled or previously enrolled in a
school.

(b) A person commits the offense of criminal trespass to a safe school
zone when he or she knowingly:

(1) enters or remains in a safe school zone without lawful business,
when as a student or employee, who has been suspended, expelled, or
dismissed for disrupting the orderly operation of the school, and as a
condition of the suspension or dismissal, has been denied access to the
safe school zone for the period of the suspension or in the case of
dismissal for a period not to exceed the term of expulsion, and has been
served in person or by registered or certified mail, at the last address
given by that person, with a written notice of the suspension or
dismissal and condition; or

(2) enters or remains in a safe school zone without lawful business,
once being served either in person or by registered or certified mail
that his or her presence has been withdrawn by the school administrator,
or his or her designee, and whose presence or acts interfere with, or
whenever there is reasonable suspicion to believe, such person will
disrupt the orderly operation, or the safety, or peaceful conduct of the
school or school activities. This clause (b)(2) has no application to
conduct protected by the Illinois Educational Labor Relations Act or any
other law applicable to labor relations. This clause (b)(2) has no
application to conduct protected by the First Amendment to the
Constitution of the United States or Article I of the Illinois
Constitution, including the exercise of free speech, free expression,
and the free exercise of religion or expression of religiously based
views.

(c) Sentence. Criminal trespass to a safe school zone is a Class A
misdemeanor.

(Source: P.A. 97-547, eff. 1-1-12.)

\hypertarget{ilcs-521-6-from-ch.-38-par.-21-6}{%
\subsection*{(720 ILCS 5/21-6) (from Ch. 38, par.
21-6)}\label{ilcs-521-6-from-ch.-38-par.-21-6}}
\addcontentsline{toc}{subsection}{(720 ILCS 5/21-6) (from Ch. 38, par.
21-6)}

\hypertarget{sec.-21-6.-unauthorized-possession-or-storage-of-weapons.}{%
\section*{Sec. 21-6. Unauthorized Possession or Storage of
Weapons.}\label{sec.-21-6.-unauthorized-possession-or-storage-of-weapons.}}
\addcontentsline{toc}{section}{Sec. 21-6. Unauthorized Possession or
Storage of Weapons.}

\markright{Sec. 21-6. Unauthorized Possession or Storage of Weapons.}

(a) Whoever possesses or stores any weapon enumerated in Section 33A-1
in any building or on land supported in whole or in part with public
funds or in any building on such land without prior written permission
from the chief security officer for such land or building commits a
Class A misdemeanor.

(b) The chief security officer must grant any reasonable request for
permission under paragraph (a).

(Source: P.A. 89-685, eff. 6-1-97.)

\hypertarget{ilcs-521-7-from-ch.-38-par.-21-7}{%
\subsection*{(720 ILCS 5/21-7) (from Ch. 38, par.
21-7)}\label{ilcs-521-7-from-ch.-38-par.-21-7}}
\addcontentsline{toc}{subsection}{(720 ILCS 5/21-7) (from Ch. 38, par.
21-7)}

\hypertarget{sec.-21-7.-criminal-trespass-to-restricted-areas-and-restricted-landing-areas-at-airports-aggravated-criminal-trespass-to-restricted-areas-and-restricted-landing-areas-at-airports.}{%
\section*{Sec. 21-7. Criminal trespass to restricted areas and
restricted landing areas at airports; aggravated criminal trespass to
restricted areas and restricted landing areas at
airports.}\label{sec.-21-7.-criminal-trespass-to-restricted-areas-and-restricted-landing-areas-at-airports-aggravated-criminal-trespass-to-restricted-areas-and-restricted-landing-areas-at-airports.}}
\addcontentsline{toc}{section}{Sec. 21-7. Criminal trespass to
restricted areas and restricted landing areas at airports; aggravated
criminal trespass to restricted areas and restricted landing areas at
airports.}

\markright{Sec. 21-7. Criminal trespass to restricted areas and
restricted landing areas at airports; aggravated criminal trespass to
restricted areas and restricted landing areas at airports.}

(a) A person commits criminal trespass to restricted areas and
restricted landing areas at airports when he or she enters upon, or
remains in, any:

(1) restricted area or restricted landing area used in connection with
an airport facility, or part thereof, in this State, after the person
has received notice from the airport authority that the entry is
forbidden;

(2) restricted area or restricted landing area used in connection with
an airport facility, or part thereof, in this State by presenting false
documents or falsely representing his or her identity orally to the
airport authority;

(3) restricted area or restricted landing area as prohibited in
paragraph (1) of this subsection, while dressed in the uniform of,
improperly wearing the identification of, presenting false credentials
of, or otherwise physically impersonating an airman, employee of an
airline, employee of an airport, or contractor at an airport.

(b) A person commits aggravated criminal trespass to restricted areas
and restricted landing areas at airports when he or she enters upon, or
remains in, any restricted area or restricted landing area used in
connection with an airport facility, or part thereof, in this State,
while in possession of a weapon, replica of a weapon, or ammunition,
after the person has received notice from the airport authority that the
entry is forbidden.

(c) Notice that the area is ``restricted'' and entry thereto
``forbidden'', for purposes of this Section, means that the person or
persons have been notified personally, either orally or in writing, or
by a printed or written notice forbidding the entry to him or her or a
group or an organization of which he or she is a member, which has been
conspicuously posted or exhibited at every usable entrance to the area
or the forbidden part thereof.

(d) (Blank).

(e) (Blank).

(f) The terms ``Restricted area'' or ``Restricted landing area'' in this
Section are defined to incorporate the meaning ascribed to those terms
in Section 8 of the ``Illinois Aeronautics Act'', approved July 24,
1945, as amended, and also include any other area of the airport that
has been designated such by the airport authority.

The terms ``airman'' and ``airport'' in this Section are defined to
incorporate the meaning ascribed to those terms in Sections 6 and 12 of
the Illinois Aeronautics Act.

(g) Paragraph (2) of subsection (a) does not apply to a peace officer or
other official of a unit of government who enters a restricted area or a
restricted landing area used in connection with an airport facility, or
part thereof, in the performance of his or her official duties.

(h) Sentence.

(1) A violation of paragraph (2) of subsection (a) is a Class A
misdemeanor.

(2) A violation of paragraph (1) or (3) of subsection (a) is a Class 4
felony.

(3) A violation of subsection (b) is a Class 3 felony.

(Source: P.A. 97-1108, eff. 1-1-13.)

\hypertarget{ilcs-521-8}{%
\subsection*{(720 ILCS 5/21-8)}\label{ilcs-521-8}}
\addcontentsline{toc}{subsection}{(720 ILCS 5/21-8)}

\hypertarget{sec.-21-8.-criminal-trespass-to-a-nuclear-facility.}{%
\section*{Sec. 21-8. Criminal trespass to a nuclear
facility.}\label{sec.-21-8.-criminal-trespass-to-a-nuclear-facility.}}
\addcontentsline{toc}{section}{Sec. 21-8. Criminal trespass to a nuclear
facility.}

\markright{Sec. 21-8. Criminal trespass to a nuclear facility.}

(a) A person commits criminal trespass to a nuclear facility when he or
she knowingly and without lawful authority:

(1) enters or remains within a nuclear facility or on the grounds of a
nuclear facility, after receiving notice before entry that entry to the
nuclear facility is forbidden;

(2) remains within the facility or on the grounds of the facility after
receiving notice from the owner or manager of the facility or other
person authorized by the owner or manager of the facility to give that
notice to depart from the facility or grounds of the facility; or

(3) enters or remains within a nuclear facility or on the grounds of a
nuclear facility, by presenting false documents or falsely representing
his or her identity orally to the owner or manager of the facility. This
paragraph (3) does not apply to a peace officer or other official of a
unit of government who enters or remains in the facility in the
performance of his or her official duties.

(b) A person has received notice from the owner or manager of the
facility or other person authorized by the owner or manager of the
facility within the meaning of paragraphs (1) and (2) of subsection (a)
if he or she has been notified personally, either orally or in writing,
or if a printed or written notice forbidding the entry has been
conspicuously posted or exhibited at the main entrance to the facility
or grounds of the facility or the forbidden part of the facility.

(c) In this Section, ``nuclear facility'' has the meaning ascribed to it
in Section 3 of the Illinois Nuclear Safety Preparedness Act.

(d) Sentence. Criminal trespass to a nuclear facility is a Class 4
felony.

(Source: P.A. 97-1108, eff. 1-1-13.)

\hypertarget{ilcs-521-9}{%
\subsection*{(720 ILCS 5/21-9)}\label{ilcs-521-9}}
\addcontentsline{toc}{subsection}{(720 ILCS 5/21-9)}

\hypertarget{sec.-21-9.-criminal-trespass-to-a-place-of-public-amusement.}{%
\section*{Sec. 21-9. Criminal trespass to a place of public
amusement.}\label{sec.-21-9.-criminal-trespass-to-a-place-of-public-amusement.}}
\addcontentsline{toc}{section}{Sec. 21-9. Criminal trespass to a place
of public amusement.}

\markright{Sec. 21-9. Criminal trespass to a place of public amusement.}

(a) A person commits criminal trespass to a place of public amusement
when he or she knowingly and without lawful authority enters or remains
on any portion of a place of public amusement after having received
notice that the general public is restricted from access to that portion
of the place of public amusement. These areas may include, but are not
limited to: a playing field, an athletic surface, a stage, a locker
room, or a dressing room located at the place of public amusement.

(a-5) A person commits the offense of criminal trespass to a place of
public amusement when he or she knowingly and without lawful authority
gains access to or remains on any portion of a place of public amusement
by presenting false documents or falsely representing his or her
identity orally to the property owner, a lessee, an agent of either the
owner or lessee, or a performer or participant. This subsection (a-5)
does not apply to a peace officer or other official of a unit of
government who enters or remains in the place of public amusement in the
performance of his or her official duties.

(b) A property owner, a lessee, an agent of either the owner or lessee,
or a performer or participant may use reasonable force to restrain a
trespasser and remove him or her from the restricted area; however, any
use of force beyond reasonable force may subject that person to any
applicable criminal penalty.

(c) A person has received notice within the meaning of subsection (a) if
he or she has been notified personally, either orally or in writing, or
if a printed or written notice forbidding such entry has been
conspicuously posted or exhibited at the entrance to the portion of the
place of public amusement that is restricted or an oral warning has been
broadcast over the public address system of the place of public
amusement.

(d) In this Section, ``place of public amusement'' means a stadium, a
theater, or any other facility of any kind, whether licensed or not,
where a live performance, a sporting event, or any other activity takes
place for other entertainment and where access to the facility is made
available to the public, regardless of whether admission is charged.

(e) Sentence. Criminal trespass to a place of public amusement is a
Class 4 felony. Upon imposition of any sentence, the court shall also
impose a fine of not less than \$1,000. In addition, any order of
probation or conditional discharge entered following a conviction shall
include a condition that the offender perform public or community
service of not less than 30 and not more than 120 hours, if community
service is available in the jurisdiction and is funded and approved by
the county board of the county where the offender was convicted. The
court may also impose any other condition of probation or conditional
discharge under this Section.

(Source: P.A. 97-1108, eff. 1-1-13.)

\hypertarget{ilcs-5art.-21-subdiv.-10-heading}{%
\subsection*{(720 ILCS 5/Art. 21, Subdiv. 10
heading)}\label{ilcs-5art.-21-subdiv.-10-heading}}
\addcontentsline{toc}{subsection}{(720 ILCS 5/Art. 21, Subdiv. 10
heading)}

SUBDIVISION 10.

MISCELLANEOUS OFFENSES

(Source: P.A. 97-1108, eff. 1-1-13.)

\hypertarget{ilcs-521-10}{%
\subsection*{(720 ILCS 5/21-10)}\label{ilcs-521-10}}
\addcontentsline{toc}{subsection}{(720 ILCS 5/21-10)}

\hypertarget{sec.-21-10.-criminal-use-of-a-motion-picture-exhibition-facility.}{%
\section*{Sec. 21-10. Criminal use of a motion picture exhibition
facility.}\label{sec.-21-10.-criminal-use-of-a-motion-picture-exhibition-facility.}}
\addcontentsline{toc}{section}{Sec. 21-10. Criminal use of a motion
picture exhibition facility.}

\markright{Sec. 21-10. Criminal use of a motion picture exhibition
facility.}

(a) A person commits criminal use of a motion picture exhibition
facility, when he or she, where a motion picture is being exhibited,
knowingly operates an audiovisual recording function of a device without
the consent of the owner or lessee of that exhibition facility and of
the licensor of the motion picture being exhibited.

(b) Sentence. Criminal use of a motion picture exhibition facility is a
Class 4 felony.

(c) The owner or lessee of a facility where a motion picture is being
exhibited, the authorized agent or employee of that owner or lessee, or
the licensor of the motion picture being exhibited or his or her agent
or employee, who alerts law enforcement authorities of an alleged
violation of this Section is not liable in any civil action arising out
of measures taken by that owner, lessee, licensor, agent, or employee in
the course of subsequently detaining a person that the owner, lessee,
licensor, agent, or employee, in good faith believed to have violated
this Section while awaiting the arrival of law enforcement authorities,
unless the plaintiff in such an action shows by clear and convincing
evidence that such measures were manifestly unreasonable or the period
of detention was unreasonably long.

(d) This Section does not prevent any lawfully authorized investigative,
law enforcement, protective, or intelligence gathering employee or agent
of the State or federal government from operating any audiovisual
recording device in any facility where a motion picture is being
exhibited as part of lawfully authorized investigative, protective, law
enforcement, or intelligence gathering activities.

(e) This Section does not apply to a person who operates an audiovisual
recording function of a device in a retail establishment solely to
demonstrate the use of that device for sales and display purposes.

(f) Nothing in this Section prevents the prosecution for conduct that
constitutes a violation of this Section under any other provision of law
providing for a greater penalty.

(g) In this Section, ``audiovisual recording function'' means the
capability of a device to record or transmit a motion picture or any
part of a motion picture by means of any technology now known or later
developed and ``facility'' does not include a personal residence.

(Source: P.A. 97-1108, eff. 1-1-13.)

\hypertarget{ilcs-521-11}{%
\subsection*{(720 ILCS 5/21-11)}\label{ilcs-521-11}}
\addcontentsline{toc}{subsection}{(720 ILCS 5/21-11)}

\hypertarget{sec.-21-11.-distributing-or-delivering-written-or-printed-solicitation-on-school-property.}{%
\section*{Sec. 21-11. Distributing or delivering written or printed
solicitation on school
property.}\label{sec.-21-11.-distributing-or-delivering-written-or-printed-solicitation-on-school-property.}}
\addcontentsline{toc}{section}{Sec. 21-11. Distributing or delivering
written or printed solicitation on school property.}

\markright{Sec. 21-11. Distributing or delivering written or printed
solicitation on school property.}

(a) Distributing or delivering written or printed solicitation on school
property or within 1,000 feet of school property, for the purpose of
inviting students to any event when a significant purpose of the event
is to commit illegal acts or to solicit attendees to commit illegal
acts, or to be held in or around abandoned buildings, is prohibited.

(b) For the purposes of this Section, ``school property'' is defined as
the buildings or grounds of any public or private elementary or
secondary school.

(c) Sentence. A violation of this Section is a Class C misdemeanor.

(Source: P.A. 97-1108, eff. 1-1-13.)

\bookmarksetup{startatroot}

\hypertarget{article-21.1.-residential-picketing}{%
\chapter*{Article 21.1. Residential
Picketing}\label{article-21.1.-residential-picketing}}
\addcontentsline{toc}{chapter}{Article 21.1. Residential Picketing}

\markboth{Article 21.1. Residential Picketing}{Article 21.1. Residential
Picketing}

\hypertarget{ilcs-521.1-1-from-ch.-38-par.-21.1-1}{%
\subsection*{(720 ILCS 5/21.1-1) (from Ch. 38, par.
21.1-1)}\label{ilcs-521.1-1-from-ch.-38-par.-21.1-1}}
\addcontentsline{toc}{subsection}{(720 ILCS 5/21.1-1) (from Ch. 38, par.
21.1-1)}

\hypertarget{sec.-21.1-1.-legislative-finding-and-declaration.}{%
\section*{Sec. 21.1-1. Legislative finding and
declaration.}\label{sec.-21.1-1.-legislative-finding-and-declaration.}}
\addcontentsline{toc}{section}{Sec. 21.1-1. Legislative finding and
declaration.}

\markright{Sec. 21.1-1. Legislative finding and declaration.}

The Legislature finds and declares that men in a free society have the
right to quiet enjoyment of their homes; that the stability of community
and family life cannot be maintained unless the right to privacy and a
sense of security and peace in the home are respected and encouraged;
that residential picketing, however just the cause inspiring it,
disrupts home, family and communal life; that residential picketing is
inappropriate in our society, where the jealously guarded rights of free
speech and assembly have always been associated with respect for the
rights of others. For these reasons the Legislature finds and declares
this Article to be necessary.

(Source: Laws 1967, p.~940.)

\hypertarget{ilcs-521.1-2-from-ch.-38-par.-21.1-2}{%
\subsection*{(720 ILCS 5/21.1-2) (from Ch. 38, par.
21.1-2)}\label{ilcs-521.1-2-from-ch.-38-par.-21.1-2}}
\addcontentsline{toc}{subsection}{(720 ILCS 5/21.1-2) (from Ch. 38, par.
21.1-2)}

\hypertarget{sec.-21.1-2.-residential-picketing.}{%
\section*{Sec. 21.1-2. Residential
picketing.}\label{sec.-21.1-2.-residential-picketing.}}
\addcontentsline{toc}{section}{Sec. 21.1-2. Residential picketing.}

\markright{Sec. 21.1-2. Residential picketing.}

A person commits residential picketing when he or she pickets before or
about the residence or dwelling of any person, except when the residence
or dwelling is used as a place of business. This Article does not apply
to a person peacefully picketing his own residence or dwelling and does
not prohibit the peaceful picketing of the place of holding a meeting or
assembly on premises commonly used to discuss subjects of general public
interest.

(Source: P.A. 97-1108, eff. 1-1-13.)

\hypertarget{ilcs-521.1-3-from-ch.-38-par.-21.1-3}{%
\subsection*{(720 ILCS 5/21.1-3) (from Ch. 38, par.
21.1-3)}\label{ilcs-521.1-3-from-ch.-38-par.-21.1-3}}
\addcontentsline{toc}{subsection}{(720 ILCS 5/21.1-3) (from Ch. 38, par.
21.1-3)}

\hypertarget{sec.-21.1-3.-sentence.}{%
\section*{Sec. 21.1-3. Sentence.}\label{sec.-21.1-3.-sentence.}}
\addcontentsline{toc}{section}{Sec. 21.1-3. Sentence.}

\markright{Sec. 21.1-3. Sentence.}

Violation of Section 21.1-2 is a Class B misdemeanor.

(Source: P.A. 77-2638.)

\bookmarksetup{startatroot}

\hypertarget{article-21.2.-interference-with-a-public-institution-of-education}{%
\chapter*{Article 21.2. Interference With A Public Institution of
Education}\label{article-21.2.-interference-with-a-public-institution-of-education}}
\addcontentsline{toc}{chapter}{Article 21.2. Interference With A Public
Institution of Education}

\markboth{Article 21.2. Interference With A Public Institution of
Education}{Article 21.2. Interference With A Public Institution of
Education}

(Source: P.A. 96-807, eff. 1-1-10.)

\hypertarget{ilcs-521.2-1-from-ch.-38-par.-21.2-1}{%
\subsection*{(720 ILCS 5/21.2-1) (from Ch. 38, par.
21.2-1)}\label{ilcs-521.2-1-from-ch.-38-par.-21.2-1}}
\addcontentsline{toc}{subsection}{(720 ILCS 5/21.2-1) (from Ch. 38, par.
21.2-1)}

\hypertarget{sec.-21.2-1.}{%
\section*{Sec. 21.2-1.}\label{sec.-21.2-1.}}
\addcontentsline{toc}{section}{Sec. 21.2-1.}

\markright{Sec. 21.2-1.}

The General Assembly, in recognition of unlawful campus and school
disorders across the nation which are disruptive of the educational
process, dangerous to the health and safety of persons, damaging to
public and private property, and which divert the use of institutional
facilities from the primary function of education, establishes by this
Act criminal penalties for conduct declared in this Article to be
unlawful. However, this Article does not modify or supersede any other
law relating to damage to persons or property, nor does it prevent a
public institution of education from establishing restrictions upon the
availability or use of any building or other facility owned, operated or
controlled by the institution to preserve their dedication to education,
nor from establishing standards of scholastic and behavioral conduct
reasonably relevant to the missions, processes and functions of the
institution, nor from invoking appropriate discipline or expulsion for
violations of such standards.

(Source: P.A. 96-807, eff. 1-1-10.)

\hypertarget{ilcs-521.2-2-from-ch.-38-par.-21.2-2}{%
\subsection*{(720 ILCS 5/21.2-2) (from Ch. 38, par.
21.2-2)}\label{ilcs-521.2-2-from-ch.-38-par.-21.2-2}}
\addcontentsline{toc}{subsection}{(720 ILCS 5/21.2-2) (from Ch. 38, par.
21.2-2)}

\hypertarget{sec.-21.2-2.-interference-with-a-public-institution-of-education.}{%
\section*{Sec. 21.2-2. Interference with a public institution of
education.}\label{sec.-21.2-2.-interference-with-a-public-institution-of-education.}}
\addcontentsline{toc}{section}{Sec. 21.2-2. Interference with a public
institution of education.}

\markright{Sec. 21.2-2. Interference with a public institution of
education.}

A person commits interference with a public institution of education
when he or she, on the campus of a public institution of education, or
at or in any building or other facility owned, operated or controlled by
the institution, without authority from the institution he or she,
through force or violence, actual or threatened:

(1) knowingly denies to a trustee, school board member, superintendent,
principal, employee, student or invitee of the institution:

(A) Freedom of movement at that place; or

(B) Use of the property or facilities of the institution; or

(C) The right of ingress or egress to the property or facilities of the
institution; or

(2) knowingly impedes, obstructs, interferes with or disrupts:

(A) the performance of institutional duties by a trustee, school board
member, superintendent, principal, or employee of the institution; or

(B) the pursuit of educational activities, as determined or prescribed
by the institution, by a trustee, school board member, superintendent,
principal, employee, student or invitee of the institution; or

(3) knowingly occupies or remains in or at any building, property or
other facility owned, operated or controlled by the institution after
due notice to depart.

(Source: P.A. 96-807, eff. 1-1-10; 97-1108, eff. 1-1-13.)

\hypertarget{ilcs-521.2-3-from-ch.-38-par.-21.2-3}{%
\subsection*{(720 ILCS 5/21.2-3) (from Ch. 38, par.
21.2-3)}\label{ilcs-521.2-3-from-ch.-38-par.-21.2-3}}
\addcontentsline{toc}{subsection}{(720 ILCS 5/21.2-3) (from Ch. 38, par.
21.2-3)}

\hypertarget{sec.-21.2-3.}{%
\section*{Sec. 21.2-3.}\label{sec.-21.2-3.}}
\addcontentsline{toc}{section}{Sec. 21.2-3.}

\markright{Sec. 21.2-3.}

Nothing in this Article prevents lawful assembly of the trustees, school
board members, superintendent, principal, employees, students or
invitees of a public institution of education, or prevents orderly
petition for redress of grievances.

(Source: P.A. 96-807, eff. 1-1-10.)

\hypertarget{ilcs-521.2-4-from-ch.-38-par.-21.2-4}{%
\subsection*{(720 ILCS 5/21.2-4) (from Ch. 38, par.
21.2-4)}\label{ilcs-521.2-4-from-ch.-38-par.-21.2-4}}
\addcontentsline{toc}{subsection}{(720 ILCS 5/21.2-4) (from Ch. 38, par.
21.2-4)}

\hypertarget{sec.-21.2-4.-sentence.}{%
\section*{Sec. 21.2-4. Sentence.}\label{sec.-21.2-4.-sentence.}}
\addcontentsline{toc}{section}{Sec. 21.2-4. Sentence.}

\markright{Sec. 21.2-4. Sentence.}

A person convicted of violation of this Article commits a Class C
misdemeanor for the first offense and for a second or subsequent offense
commits a Class B misdemeanor. If the interference with the public
institution of education is accompanied by a threat of personal injury
or property damage, the person commits a Class 3 felony and may be
sentenced to a term of imprisonment of not less than 2 years and not
more than 10 years and may be prosecuted for intimidation in accordance
with Section 12-6 of this Code.

(Source: P.A. 96-807, eff. 1-1-10.)

\hypertarget{ilcs-521.2-5-from-ch.-38-par.-21.2-5}{%
\subsection*{(720 ILCS 5/21.2-5) (from Ch. 38, par.
21.2-5)}\label{ilcs-521.2-5-from-ch.-38-par.-21.2-5}}
\addcontentsline{toc}{subsection}{(720 ILCS 5/21.2-5) (from Ch. 38, par.
21.2-5)}

\hypertarget{sec.-21.2-5.}{%
\section*{Sec. 21.2-5.}\label{sec.-21.2-5.}}
\addcontentsline{toc}{section}{Sec. 21.2-5.}

\markright{Sec. 21.2-5.}

For the purposes of this Article the words and phrases described in this
Section have the meanings designated in this Section, except when a
particular context clearly requires a different meaning.

``Public institution of education'' means an educational organization
located in this State which provides an organized elementary, secondary,
or post-high school educational program, and which is supported in whole
or in part by appropriations of the General Assembly, a unit of local
government or school district.

A person has received ``due notice'' if he, or the group of which he is
a part, has been given oral or written notice from an authorized
representative of the public institution of education in a manner
reasonably designated to inform him, or the group of which he is a part,
that he or they should cease such action or depart from such premises.
The notice may also be given by a printed or written notice forbidding
entry conspicuously posted or exhibited at the main entrance of the
building or other facility, or the forbidden part thereof.

``Force or violence'' includes, but is not limited to, use of one's
person, individually or in concert with others, to impede access to or
movement within or otherwise to interfere with the conduct of the
authorized activities of the public institution of education, its
trustees, school board members, superintendent, principal, employees,
students or invitees.

(Source: P.A. 96-807, eff. 1-1-10.)

\hypertarget{ilcs-521.2-6-from-ch.-38-par.-21.2-6}{%
\subsection*{(720 ILCS 5/21.2-6) (from Ch. 38, par.
21.2-6)}\label{ilcs-521.2-6-from-ch.-38-par.-21.2-6}}
\addcontentsline{toc}{subsection}{(720 ILCS 5/21.2-6) (from Ch. 38, par.
21.2-6)}

\hypertarget{sec.-21.2-6.}{%
\section*{Sec. 21.2-6.}\label{sec.-21.2-6.}}
\addcontentsline{toc}{section}{Sec. 21.2-6.}

\markright{Sec. 21.2-6.}

If any provision of this Act or the application thereof to any person or
circumstances is held invalid, such invalidity shall not affect other
provisions or applications of the Act which can be given effect without
the invalid provision or application, and to this end the provisions of
this Act are declared severable.

(Source: P.A. 76-1582 .)

\bookmarksetup{startatroot}

\hypertarget{article-21.3.-solicitation-on-school-property}{%
\chapter*{Article 21.3. Solicitation On School
Property}\label{article-21.3.-solicitation-on-school-property}}
\addcontentsline{toc}{chapter}{Article 21.3. Solicitation On School
Property}

\markboth{Article 21.3. Solicitation On School Property}{Article 21.3.
Solicitation On School Property}

(Repealed)

Source: P.A. 88-357. Repealed by P.A. 97-1108, eff. 1-1-13.

\hypertarget{ilcs-5tit.-iii-pt.-d-heading}{%
\subsection*{(720 ILCS 5/Tit. III Pt. D
heading)}\label{ilcs-5tit.-iii-pt.-d-heading}}
\addcontentsline{toc}{subsection}{(720 ILCS 5/Tit. III Pt. D heading)}

PART D.

OFFENSES AFFECTING PUBLIC HEALTH, SAFETY AND DECENCY

\bookmarksetup{startatroot}

\hypertarget{article-24.-deadly-weapons}{%
\chapter*{Article 24. Deadly Weapons}\label{article-24.-deadly-weapons}}
\addcontentsline{toc}{chapter}{Article 24. Deadly Weapons}

\markboth{Article 24. Deadly Weapons}{Article 24. Deadly Weapons}

\hypertarget{ilcs-524-1-from-ch.-38-par.-24-1}{%
\subsection*{(720 ILCS 5/24-1) (from Ch. 38, par.
24-1)}\label{ilcs-524-1-from-ch.-38-par.-24-1}}
\addcontentsline{toc}{subsection}{(720 ILCS 5/24-1) (from Ch. 38, par.
24-1)}

\hypertarget{sec.-24-1.-unlawful-use-of-weapons.}{%
\section*{Sec. 24-1. Unlawful use of
weapons.}\label{sec.-24-1.-unlawful-use-of-weapons.}}
\addcontentsline{toc}{section}{Sec. 24-1. Unlawful use of weapons.}

\markright{Sec. 24-1. Unlawful use of weapons.}

(a) A person commits the offense of unlawful use of weapons when he
knowingly:

(1) Sells, manufactures, purchases, possesses or carries any bludgeon,
black-jack, slung-shot, sand-club, sand-bag, metal knuckles or other
knuckle weapon regardless of its composition, throwing star, or any
knife, commonly referred to as a switchblade knife, which has a blade
that opens automatically by hand pressure applied to a button, spring or
other device in the handle of the knife, or a ballistic knife, which is
a device that propels a knifelike blade as a projectile by means of a
coil spring, elastic material or compressed gas; or

(2) Carries or possesses with intent to use the same unlawfully against
another, a dagger, dirk, billy, dangerous knife, razor, stiletto, broken
bottle or other piece of glass, stun gun or taser or any other dangerous
or deadly weapon or instrument of like character; or

(2.5) Carries or possesses with intent to use the same unlawfully
against another, any firearm in a church, synagogue, mosque, or other
building, structure, or place used for religious worship; or

(3) Carries on or about his person or in any vehicle, a tear gas gun
projector or bomb or any object containing noxious liquid gas or
substance, other than an object containing a non-lethal noxious liquid
gas or substance designed solely for personal defense carried by a
person 18 years of age or older; or

(4) Carries or possesses in any vehicle or concealed on or about his
person except when on his land or in his own abode, legal dwelling, or
fixed place of business, or on the land or in the legal dwelling of
another person as an invitee with that person's permission, any pistol,
revolver, stun gun or taser or other firearm, except that this
subsection (a)(4) does not apply to or affect transportation of weapons
that meet one of the following conditions:

(i) are broken down in a non-functioning state; or

(ii) are not immediately accessible; or

(iii) are unloaded and enclosed in a case, firearm carrying box,
shipping box, or other container by a person who has been issued a
currently valid Firearm Owner's Identification Card; or

(iv) are carried or possessed in accordance with the Firearm Concealed
Carry Act by a person who has been issued a currently valid license
under the Firearm Concealed Carry Act; or

(5) Sets a spring gun; or

(6) Possesses any device or attachment of any kind designed, used or
intended for use in silencing the report of any firearm; or

(7) Sells, manufactures, purchases, possesses or carries:

(i) a machine gun, which shall be defined for the purposes of this
subsection as any weapon, which shoots, is designed to shoot, or can be
readily restored to shoot, automatically more than one shot without
manually reloading by a single function of the trigger, including the
frame or receiver of any such weapon, or sells, manufactures, purchases,
possesses, or carries any combination of parts designed or intended for
use in converting any weapon into a machine gun, or any combination or
parts from which a machine gun can be assembled if such parts are in the
possession or under the control of a person;

(ii) any rifle having one or more barrels less than 16 inches in length
or a shotgun having one or more barrels less than 18 inches in length or
any weapon made from a rifle or shotgun, whether by alteration,
modification, or otherwise, if such a weapon as modified has an overall
length of less than 26 inches; or

(iii) any bomb, bomb-shell, grenade, bottle or other container
containing an explosive substance of over one-quarter ounce for like
purposes, such as, but not limited to, black powder bombs and Molotov
cocktails or artillery projectiles; or

(8) Carries or possesses any firearm, stun gun or taser or other deadly
weapon in any place which is licensed to sell intoxicating beverages, or
at any public gathering held pursuant to a license issued by any
governmental body or any public gathering at which an admission is
charged, excluding a place where a showing, demonstration or lecture
involving the exhibition of unloaded firearms is conducted.

This subsection (a)(8) does not apply to any auction or raffle of a
firearm held pursuant to a license or permit issued by a governmental
body, nor does it apply to persons engaged in firearm safety training
courses; or

(9) Carries or possesses in a vehicle or on or about his or her person
any pistol, revolver, stun gun or taser or firearm or ballistic knife,
when he or she is hooded, robed or masked in such manner as to conceal
his or her identity; or

(10) Carries or possesses on or about his or her person, upon any public
street, alley, or other public lands within the corporate limits of a
city, village, or incorporated town, except when an invitee thereon or
therein, for the purpose of the display of such weapon or the lawful
commerce in weapons, or except when on his land or in his or her own
abode, legal dwelling, or fixed place of business, or on the land or in
the legal dwelling of another person as an invitee with that person's
permission, any pistol, revolver, stun gun, or taser or other firearm,
except that this subsection (a)(10) does not apply to or affect
transportation of weapons that meet one of the following conditions:

(i) are broken down in a non-functioning state; or

(ii) are not immediately accessible; or

(iii) are unloaded and enclosed in a case, firearm carrying box,
shipping box, or other container by a person who has been issued a
currently valid Firearm Owner's Identification Card; or

(iv) are carried or possessed in accordance with the Firearm Concealed
Carry Act by a person who has been issued a currently valid license
under the Firearm Concealed Carry Act.

A ``stun gun or taser'', as used in this paragraph (a) means (i) any
device which is powered by electrical charging units, such as,
batteries, and which fires one or several barbs attached to a length of
wire and which, upon hitting a human, can send out a current capable of
disrupting the person's nervous system in such a manner as to render him
incapable of normal functioning or (ii) any device which is powered by
electrical charging units, such as batteries, and which, upon contact
with a human or clothing worn by a human, can send out current capable
of disrupting the person's nervous system in such a manner as to render
him incapable of normal functioning; or

(11) Sells, manufactures, delivers, imports, possesses, or purchases any
assault weapon attachment or .50 caliber cartridge in violation of
Section 24-1.9 or any explosive bullet. For purposes of this paragraph
(a) ``explosive bullet'' means the projectile portion of an ammunition
cartridge which contains or carries an explosive charge which will
explode upon contact with the flesh of a human or an animal.
``Cartridge'' means a tubular metal case having a projectile affixed at
the front thereof and a cap or primer at the rear end thereof, with the
propellant contained in such tube between the projectile and the cap; or

(12) (Blank); or

(13) Carries or possesses on or about his or her person while in a
building occupied by a unit of government, a billy club, other weapon of
like character, or other instrument of like character intended for use
as a weapon. For the purposes of this Section, ``billy club'' means a
short stick or club commonly carried by police officers which is either
telescopic or constructed of a solid piece of wood or other man-made
material; or

(14) Manufactures, possesses, sells, or offers to sell, purchase,
manufacture, import, transfer, or use any device, part, kit, tool,
accessory, or combination of parts that is designed to and functions to
increase the rate of fire of a semiautomatic firearm above the standard
rate of fire for semiautomatic firearms that is not equipped with that
device, part, or combination of parts; or

(15) Carries or possesses any assault weapon or .50 caliber rifle in
violation of Section 24-1.9; or

(16) Manufactures, sells, delivers, imports, or purchases any assault
weapon or .50 caliber rifle in violation of Section 24-1.9.

(b) Sentence. A person convicted of a violation of subsection 24-1(a)(1)
through (5), subsection 24-1(a)(10), subsection 24-1(a)(11), subsection
24-1(a)(13), or 24-1(a)(15) commits a Class A misdemeanor. A person
convicted of a violation of subsection 24-1(a)(8) or 24-1(a)(9) commits
a Class 4 felony; a person convicted of a violation of subsection
24-1(a)(6), 24-1(a)(7)(ii), 24-1(a)(7)(iii), or 24-1(a)(16) commits a
Class 3 felony. A person convicted of a violation of subsection
24-1(a)(7)(i) commits a Class 2 felony and shall be sentenced to a term
of imprisonment of not less than 3 years and not more than 7 years,
unless the weapon is possessed in the passenger compartment of a motor
vehicle as defined in Section 1-146 of the Illinois Vehicle Code, or on
the person, while the weapon is loaded, in which case it shall be a
Class X felony. A person convicted of a second or subsequent violation
of subsection 24-1(a)(4), 24-1(a)(8), 24-1(a)(9), 24-1(a)(10), or
24-1(a)(15) commits a Class 3 felony. A person convicted of a violation
of subsection 24-1(a)(2.5) or 24-1(a)(14) commits a Class 2 felony. The
possession of each weapon or device in violation of this Section
constitutes a single and separate violation.

(c) Violations in specific places.

(1) A person who violates subsection 24-1(a)(6) or

24-1(a)(7) in any school, regardless of the time of day or the time of
year, in residential property owned, operated or managed by a public
housing agency or leased by a public housing agency as part of a
scattered site or mixed-income development, in a public park, in a
courthouse, on the real property comprising any school, regardless of
the time of day or the time of year, on residential property owned,
operated or managed by a public housing agency or leased by a public
housing agency as part of a scattered site or mixed-income development,
on the real property comprising any public park, on the real property
comprising any courthouse, in any conveyance owned, leased or contracted
by a school to transport students to or from school or a school related
activity, in any conveyance owned, leased, or contracted by a public
transportation agency, or on any public way within 1,000 feet of the
real property comprising any school, public park, courthouse, public
transportation facility, or residential property owned, operated, or
managed by a public housing agency or leased by a public housing agency
as part of a scattered site or mixed-income development commits a Class
2 felony and shall be sentenced to a term of imprisonment of not less
than 3 years and not more than 7 years.

(1.5) A person who violates subsection 24-1(a)(4),

24-1(a)(9), or 24-1(a)(10) in any school, regardless of the time of day
or the time of year, in residential property owned, operated, or managed
by a public housing agency or leased by a public housing agency as part
of a scattered site or mixed-income development, in a public park, in a
courthouse, on the real property comprising any school, regardless of
the time of day or the time of year, on residential property owned,
operated, or managed by a public housing agency or leased by a public
housing agency as part of a scattered site or mixed-income development,
on the real property comprising any public park, on the real property
comprising any courthouse, in any conveyance owned, leased, or
contracted by a school to transport students to or from school or a
school related activity, in any conveyance owned, leased, or contracted
by a public transportation agency, or on any public way within 1,000
feet of the real property comprising any school, public park,
courthouse, public transportation facility, or residential property
owned, operated, or managed by a public housing agency or leased by a
public housing agency as part of a scattered site or mixed-income
development commits a Class 3 felony.

(2) A person who violates subsection 24-1(a)(1),

24-1(a)(2), or 24-1(a)(3) in any school, regardless of the time of day
or the time of year, in residential property owned, operated or managed
by a public housing agency or leased by a public housing agency as part
of a scattered site or mixed-income development, in a public park, in a
courthouse, on the real property comprising any school, regardless of
the time of day or the time of year, on residential property owned,
operated or managed by a public housing agency or leased by a public
housing agency as part of a scattered site or mixed-income development,
on the real property comprising any public park, on the real property
comprising any courthouse, in any conveyance owned, leased or contracted
by a school to transport students to or from school or a school related
activity, in any conveyance owned, leased, or contracted by a public
transportation agency, or on any public way within 1,000 feet of the
real property comprising any school, public park, courthouse, public
transportation facility, or residential property owned, operated, or
managed by a public housing agency or leased by a public housing agency
as part of a scattered site or mixed-income development commits a Class
4 felony. ``Courthouse'' means any building that is used by the Circuit,
Appellate, or Supreme Court of this State for the conduct of official
business.

(3) Paragraphs (1), (1.5), and (2) of this subsection

(c) shall not apply to law enforcement officers or security officers of
such school, college, or university or to students carrying or
possessing firearms for use in training courses, parades, hunting,
target shooting on school ranges, or otherwise with the consent of
school authorities and which firearms are transported unloaded enclosed
in a suitable case, box, or transportation package.

(4) For the purposes of this subsection (c), ``school'' means any public
or private elementary or secondary school, community college, college,
or university.

(5) For the purposes of this subsection (c),

``public transportation agency'' means a public or private agency that
provides for the transportation or conveyance of persons by means
available to the general public, except for transportation by
automobiles not used for conveyance of the general public as passengers;
and ``public transportation facility'' means a terminal or other place
where one may obtain public transportation.

(d) The presence in an automobile other than a public omnibus of any
weapon, instrument or substance referred to in subsection (a)(7) is
prima facie evidence that it is in the possession of, and is being
carried by, all persons occupying such automobile at the time such
weapon, instrument or substance is found, except under the following
circumstances: (i) if such weapon, instrument or instrumentality is
found upon the person of one of the occupants therein; or (ii) if such
weapon, instrument or substance is found in an automobile operated for
hire by a duly licensed driver in the due, lawful and proper pursuit of
his or her trade, then such presumption shall not apply to the driver.

(e) Exemptions.

(1) Crossbows, Common or Compound bows and

Underwater Spearguns are exempted from the definition of ballistic knife
as defined in paragraph (1) of subsection (a) of this Section.

(2) The provision of paragraph (1) of subsection (a) of this Section
prohibiting the sale, manufacture, purchase, possession, or carrying of
any knife, commonly referred to as a switchblade knife, which has a
blade that opens automatically by hand pressure applied to a button,
spring or other device in the handle of the knife, does not apply to a
person who possesses a currently valid Firearm Owner's Identification
Card previously issued in his or her name by the Illinois State Police
or to a person or an entity engaged in the business of selling or
manufacturing switchblade knives.

(Source: P.A. 101-223, eff. 1-1-20; 102-538, eff. 8-20-21; 102-1116,
eff. 1-10-23.)

\hypertarget{ilcs-524-1.1-from-ch.-38-par.-24-1.1}{%
\subsection*{(720 ILCS 5/24-1.1) (from Ch. 38, par.
24-1.1)}\label{ilcs-524-1.1-from-ch.-38-par.-24-1.1}}
\addcontentsline{toc}{subsection}{(720 ILCS 5/24-1.1) (from Ch. 38, par.
24-1.1)}

\hypertarget{sec.-24-1.1.-unlawful-use-or-possession-of-weapons-by-felons-or-persons-in-the-custody-of-the-department-of-corrections-facilities.}{%
\section*{Sec. 24-1.1. Unlawful use or possession of weapons by felons
or persons in the custody of the Department of Corrections
facilities.}\label{sec.-24-1.1.-unlawful-use-or-possession-of-weapons-by-felons-or-persons-in-the-custody-of-the-department-of-corrections-facilities.}}
\addcontentsline{toc}{section}{Sec. 24-1.1. Unlawful use or possession
of weapons by felons or persons in the custody of the Department of
Corrections facilities.}

\markright{Sec. 24-1.1. Unlawful use or possession of weapons by felons
or persons in the custody of the Department of Corrections facilities.}

(a) It is unlawful for a person to knowingly possess on or about his
person or on his land or in his own abode or fixed place of business any
weapon prohibited under Section 24-1 of this Act or any firearm or any
firearm ammunition if the person has been convicted of a felony under
the laws of this State or any other jurisdiction. This Section shall not
apply if the person has been granted relief by the Director of the
Illinois State Police under Section 10 of the Firearm Owners
Identification Card Act.

(b) It is unlawful for any person confined in a penal institution, which
is a facility of the Illinois Department of Corrections, to possess any
weapon prohibited under Section 24-1 of this Code or any firearm or
firearm ammunition, regardless of the intent with which he possesses it.

(c) It shall be an affirmative defense to a violation of subsection (b),
that such possession was specifically authorized by rule, regulation, or
directive of the Illinois Department of Corrections or order issued
pursuant thereto.

(d) The defense of necessity is not available to a person who is charged
with a violation of subsection (b) of this Section.

(e) Sentence. Violation of this Section by a person not confined in a
penal institution shall be a Class 3 felony for which the person shall
be sentenced to no less than 2 years and no more than 10 years. A second
or subsequent violation of this Section shall be a Class 2 felony for
which the person shall be sentenced to a term of imprisonment of not
less than 3 years and not more than 14 years, except as provided for in
Section 5-4.5-110 of the Unified Code of Corrections. Violation of this
Section by a person not confined in a penal institution who has been
convicted of a forcible felony, a felony violation of Article 24 of this
Code or of the Firearm Owners Identification Card Act, stalking or
aggravated stalking, or a Class 2 or greater felony under the Illinois
Controlled Substances Act, the Cannabis Control Act, or the
Methamphetamine Control and Community Protection Act is a Class 2 felony
for which the person shall be sentenced to not less than 3 years and not
more than 14 years, except as provided for in Section 5-4.5-110 of the
Unified Code of Corrections. Violation of this Section by a person who
is on parole or mandatory supervised release is a Class 2 felony for
which the person shall be sentenced to not less than 3 years and not
more than 14 years, except as provided for in Section 5-4.5-110 of the
Unified Code of Corrections. Violation of this Section by a person not
confined in a penal institution is a Class X felony when the firearm
possessed is a machine gun. Any person who violates this Section while
confined in a penal institution, which is a facility of the Illinois
Department of Corrections, is guilty of a Class 1 felony, if he
possesses any weapon prohibited under Section 24-1 of this Code
regardless of the intent with which he possesses it, a Class X felony if
he possesses any firearm, firearm ammunition or explosive, and a Class X
felony for which the offender shall be sentenced to not less than 12
years and not more than 50 years when the firearm possessed is a machine
gun. A violation of this Section while wearing or in possession of body
armor as defined in Section 33F-1 is a Class X felony punishable by a
term of imprisonment of not less than 10 years and not more than 40
years. The possession of each firearm or firearm ammunition in violation
of this Section constitutes a single and separate violation.

(Source: P.A. 102-538, eff. 8-20-21.)

\hypertarget{ilcs-524-1.2-from-ch.-38-par.-24-1.2}{%
\subsection*{(720 ILCS 5/24-1.2) (from Ch. 38, par.
24-1.2)}\label{ilcs-524-1.2-from-ch.-38-par.-24-1.2}}
\addcontentsline{toc}{subsection}{(720 ILCS 5/24-1.2) (from Ch. 38, par.
24-1.2)}

\hypertarget{sec.-24-1.2.-aggravated-discharge-of-a-firearm.}{%
\section*{Sec. 24-1.2. Aggravated discharge of a
firearm.}\label{sec.-24-1.2.-aggravated-discharge-of-a-firearm.}}
\addcontentsline{toc}{section}{Sec. 24-1.2. Aggravated discharge of a
firearm.}

\markright{Sec. 24-1.2. Aggravated discharge of a firearm.}

(a) A person commits aggravated discharge of a firearm when he or she
knowingly or intentionally:

(1) Discharges a firearm at or into a building he or she knows or
reasonably should know to be occupied and the firearm is discharged from
a place or position outside that building;

(2) Discharges a firearm in the direction of another person or in the
direction of a vehicle he or she knows or reasonably should know to be
occupied by a person;

(3) Discharges a firearm in the direction of a person he or she knows to
be a peace officer, a community policing volunteer, a correctional
institution employee, or a fireman while the officer, volunteer,
employee or fireman is engaged in the execution of any of his or her
official duties, or to prevent the officer, volunteer, employee or
fireman from performing his or her official duties, or in retaliation
for the officer, volunteer, employee or fireman performing his or her
official duties;

(4) Discharges a firearm in the direction of a vehicle he or she knows
to be occupied by a peace officer, a person summoned or directed by a
peace officer, a correctional institution employee or a fireman while
the officer, employee or fireman is engaged in the execution of any of
his or her official duties, or to prevent the officer, employee or
fireman from performing his or her official duties, or in retaliation
for the officer, employee or fireman performing his or her official
duties;

(5) Discharges a firearm in the direction of a person he or she knows to
be emergency medical services personnel who is engaged in the execution
of any of his or her official duties, or to prevent the emergency
medical services personnel from performing his or her official duties,
or in retaliation for the emergency medical services personnel
performing his or her official duties;

(6) Discharges a firearm in the direction of a vehicle he or she knows
to be occupied by emergency medical services personnel while the
emergency medical services personnel is engaged in the execution of any
of his or her official duties, or to prevent the emergency medical
services personnel from performing his or her official duties, or in
retaliation for the emergency medical services personnel performing his
or her official duties;

(7) Discharges a firearm in the direction of a person he or she knows to
be a teacher or other person employed in any school and the teacher or
other employee is upon the grounds of a school or grounds adjacent to a
school, or is in any part of a building used for school purposes;

(8) Discharges a firearm in the direction of a person he or she knows to
be an emergency management worker while the emergency management worker
is engaged in the execution of any of his or her official duties, or to
prevent the emergency management worker from performing his or her
official duties, or in retaliation for the emergency management worker
performing his or her official duties; or

(9) Discharges a firearm in the direction of a vehicle he or she knows
to be occupied by an emergency management worker while the emergency
management worker is engaged in the execution of any of his or her
official duties, or to prevent the emergency management worker from
performing his or her official duties, or in retaliation for the
emergency management worker performing his or her official duties.

(b) A violation of subsection (a)(1) or subsection (a)(2) of this
Section is a Class 1 felony. A violation of subsection (a)(1) or (a)(2)
of this Section committed in a school, on the real property comprising a
school, within 1,000 feet of the real property comprising a school, at a
school related activity or on or within 1,000 feet of any conveyance
owned, leased, or contracted by a school to transport students to or
from school or a school related activity, regardless of the time of day
or time of year that the offense was committed is a Class X felony. A
violation of subsection (a)(3), (a)(4), (a)(5), (a)(6), (a)(7), (a)(8),
or (a)(9) of this Section is a Class X felony for which the sentence
shall be a term of imprisonment of no less than 10 years and not more
than 45 years.

(c) For purposes of this Section:

``Emergency medical services personnel'' has the meaning specified in
Section 3.5 of the Emergency Medical Services (EMS) Systems Act and
shall include all ambulance crew members, including drivers or pilots.

``School'' means a public or private elementary or secondary school,
community college, college, or university.

``School related activity'' means any sporting, social, academic, or
other activity for which students' attendance or participation is
sponsored, organized, or funded in whole or in part by a school or
school district.

(Source: P.A. 99-816, eff. 8-15-16.)

\hypertarget{ilcs-524-1.2-5}{%
\subsection*{(720 ILCS 5/24-1.2-5)}\label{ilcs-524-1.2-5}}
\addcontentsline{toc}{subsection}{(720 ILCS 5/24-1.2-5)}

\hypertarget{sec.-24-1.2-5.-aggravated-discharge-of-a-machine-gun-or-a-firearm-equipped-with-a-device-designed-or-used-for-silencing-the-report-of-a-firearm.}{%
\section*{Sec. 24-1.2-5. Aggravated discharge of a machine gun or a
firearm equipped with a device designed or used for silencing the report
of a
firearm.}\label{sec.-24-1.2-5.-aggravated-discharge-of-a-machine-gun-or-a-firearm-equipped-with-a-device-designed-or-used-for-silencing-the-report-of-a-firearm.}}
\addcontentsline{toc}{section}{Sec. 24-1.2-5. Aggravated discharge of a
machine gun or a firearm equipped with a device designed or used for
silencing the report of a firearm.}

\markright{Sec. 24-1.2-5. Aggravated discharge of a machine gun or a
firearm equipped with a device designed or used for silencing the report
of a firearm.}

(a) A person commits aggravated discharge of a machine gun or a firearm
equipped with a device designed or used for silencing the report of a
firearm when he or she knowingly or intentionally:

(1) Discharges a machine gun or a firearm equipped with a device
designed or used for silencing the report of a firearm at or into a
building he or she knows to be occupied and the machine gun or the
firearm equipped with a device designed or used for silencing the report
of a firearm is discharged from a place or position outside that
building;

(2) Discharges a machine gun or a firearm equipped with a device
designed or used for silencing the report of a firearm in the direction
of another person or in the direction of a vehicle he or she knows to be
occupied;

(3) Discharges a machine gun or a firearm equipped with a device
designed or used for silencing the report of a firearm in the direction
of a person he or she knows to be a peace officer, a person summoned or
directed by a peace officer, a correctional institution employee, or a
fireman while the officer, employee or fireman is engaged in the
execution of any of his or her official duties, or to prevent the
officer, employee or fireman from performing his or her official duties,
or in retaliation for the officer, employee or fireman performing his or
her official duties;

(4) Discharges a machine gun or a firearm equipped with a device
designed or used for silencing the report of a firearm in the direction
of a vehicle he or she knows to be occupied by a peace officer, a person
summoned or directed by a peace officer, a correctional institution
employee or a fireman while the officer, employee or fireman is engaged
in the execution of any of his or her official duties, or to prevent the
officer, employee or fireman from performing his or her official duties,
or in retaliation for the officer, employee or fireman performing his or
her official duties;

(5) Discharges a machine gun or a firearm equipped with a device
designed or used for silencing the report of a firearm in the direction
of a person he or she knows to be emergency medical services personnel
while the emergency medical services personnel is engaged in the
execution of any of his or her official duties, or to prevent the
emergency medical services personnel from performing his or her official
duties, or in retaliation for the emergency medical services personnel
performing his or her official duties;

(6) Discharges a machine gun or a firearm equipped with a device
designed or used for silencing the report of a firearm in the direction
of a vehicle he or she knows to be occupied by emergency medical
services personnel, while the emergency medical services personnel is
engaged in the execution of any of his or her official duties, or to
prevent the emergency medical services personnel from performing his or
her official duties, or in retaliation for the emergency medical
services personnel performing his or her official duties;

(7) Discharges a machine gun or a firearm equipped with a device
designed or used for silencing the report of a firearm in the direction
of a person he or she knows to be an emergency management worker while
the emergency management worker is engaged in the execution of any of
his or her official duties, or to prevent the emergency management
worker from performing his or her official duties, or in retaliation for
the emergency management worker performing his or her official duties;
or

(8) Discharges a machine gun or a firearm equipped with a device
designed or used for silencing the report of a firearm in the direction
of a vehicle he or she knows to be occupied by an emergency management
worker while the emergency management worker is engaged in the execution
of any of his or her official duties, or to prevent the emergency
management worker from performing his or her official duties, or in
retaliation for the emergency management worker performing his or her
official duties.

(b) A violation of subsection (a) (1) or subsection (a) (2) of this
Section is a Class X felony. A violation of subsection (a) (3), (a) (4),
(a) (5), (a) (6), (a) (7), or (a) (8) of this Section is a Class X
felony for which the sentence shall be a term of imprisonment of no less
than 12 years and no more than 50 years.

(c) For the purpose of this Section:

``Emergency medical services personnel'' has the meaning specified in
Section 3.5 of the Emergency Medical Services (EMS) Systems Act and
shall include all ambulance crew members, including drivers or pilots.

``Machine gun'' has the meaning ascribed to it in clause (i) of
paragraph (7) of subsection (a) of Section 24-1 of this Code.

(d) This Section does not apply to a peace officer while serving as a
member of a tactical response team or special operations team. A peace
officer may not personally own or apply for ownership of a device or
attachment of any kind designed, used, or intended for use in silencing
the report of any firearm. These devices shall be owned and maintained
by lawfully recognized units of government whose duties include the
investigation of criminal acts.

(Source: P.A. 99-816, eff. 8-15-16.)

\hypertarget{ilcs-524-1.5}{%
\subsection*{(720 ILCS 5/24-1.5)}\label{ilcs-524-1.5}}
\addcontentsline{toc}{subsection}{(720 ILCS 5/24-1.5)}

\hypertarget{sec.-24-1.5.-reckless-discharge-of-a-firearm.}{%
\section*{Sec. 24-1.5. Reckless discharge of a
firearm.}\label{sec.-24-1.5.-reckless-discharge-of-a-firearm.}}
\addcontentsline{toc}{section}{Sec. 24-1.5. Reckless discharge of a
firearm.}

\markright{Sec. 24-1.5. Reckless discharge of a firearm.}

(a) A person commits reckless discharge of a firearm by discharging a
firearm in a reckless manner which endangers the bodily safety of an
individual.

(b) If the conduct described in subsection (a) is committed by a
passenger of a moving motor vehicle with the knowledge and consent of
the driver of the motor vehicle the driver is accountable for such
conduct.

(c) Reckless discharge of a firearm is a Class 4 felony.

(d) This Section does not apply to a peace officer while in the
performance of his or her official duties.

(Source: P.A. 88-217.)

\hypertarget{ilcs-524-1.6}{%
\subsection*{(720 ILCS 5/24-1.6)}\label{ilcs-524-1.6}}
\addcontentsline{toc}{subsection}{(720 ILCS 5/24-1.6)}

\hypertarget{sec.-24-1.6.-aggravated-unlawful-use-of-a-weapon.}{%
\section*{Sec. 24-1.6. Aggravated unlawful use of a
weapon.}\label{sec.-24-1.6.-aggravated-unlawful-use-of-a-weapon.}}
\addcontentsline{toc}{section}{Sec. 24-1.6. Aggravated unlawful use of a
weapon.}

\markright{Sec. 24-1.6. Aggravated unlawful use of a weapon.}

(a) A person commits the offense of aggravated unlawful use of a weapon
when he or she knowingly:

(1) Carries on or about his or her person or in any vehicle or concealed
on or about his or her person except when on his or her land or in his
or her abode, legal dwelling, or fixed place of business, or on the land
or in the legal dwelling of another person as an invitee with that
person's permission, any pistol, revolver, stun gun or taser or other
firearm; or

(2) Carries or possesses on or about his or her person, upon any public
street, alley, or other public lands within the corporate limits of a
city, village or incorporated town, except when an invitee thereon or
therein, for the purpose of the display of such weapon or the lawful
commerce in weapons, or except when on his or her own land or in his or
her own abode, legal dwelling, or fixed place of business, or on the
land or in the legal dwelling of another person as an invitee with that
person's permission, any pistol, revolver, stun gun or taser or other
firearm; and

(3) One of the following factors is present:

(A) the firearm, other than a pistol, revolver, or handgun, possessed
was uncased, loaded, and immediately accessible at the time of the
offense; or

(A-5) the pistol, revolver, or handgun possessed was uncased, loaded,
and immediately accessible at the time of the offense and the person
possessing the pistol, revolver, or handgun has not been issued a
currently valid license under the Firearm Concealed Carry Act; or

(B) the firearm, other than a pistol, revolver, or handgun, possessed
was uncased, unloaded, and the ammunition for the weapon was immediately
accessible at the time of the offense; or

(B-5) the pistol, revolver, or handgun possessed was uncased, unloaded,
and the ammunition for the weapon was immediately accessible at the time
of the offense and the person possessing the pistol, revolver, or
handgun has not been issued a currently valid license under the Firearm
Concealed Carry Act; or

(C) the person possessing the firearm has not been issued a currently
valid Firearm Owner's Identification Card; or

(D) the person possessing the weapon was previously adjudicated a
delinquent minor under the Juvenile Court Act of 1987 for an act that if
committed by an adult would be a felony; or

(E) the person possessing the weapon was engaged in a misdemeanor
violation of the Cannabis Control Act, in a misdemeanor violation of the
Illinois Controlled Substances Act, or in a misdemeanor violation of the
Methamphetamine Control and Community Protection Act; or

(F) (blank); or

(G) the person possessing the weapon had an order of protection issued
against him or her within the previous 2 years; or

(H) the person possessing the weapon was engaged in the commission or
attempted commission of a misdemeanor involving the use or threat of
violence against the person or property of another; or

(I) the person possessing the weapon was under 21 years of age and in
possession of a handgun, unless the person under 21 is engaged in lawful
activities under the Wildlife Code or described in subsection
24-2(b)(1), (b)(3), or 24-2(f).

(a-5) ``Handgun'' as used in this Section has the meaning given to it in
Section 5 of the Firearm Concealed Carry Act.

(b) ``Stun gun or taser'' as used in this Section has the same
definition given to it in Section 24-1 of this Code.

(c) This Section does not apply to or affect the transportation or
possession of weapons that:

(i) are broken down in a non-functioning state; or

(ii) are not immediately accessible; or

(iii) are unloaded and enclosed in a case, firearm carrying box,
shipping box, or other container by a person who has been issued a
currently valid Firearm Owner's Identification Card.

(d) Sentence.

(1) Aggravated unlawful use of a weapon is a Class 4 felony; a second or
subsequent offense is a Class 2 felony for which the person shall be
sentenced to a term of imprisonment of not less than 3 years and not
more than 7 years, except as provided for in Section 5-4.5-110 of the
Unified Code of Corrections.

(2) Except as otherwise provided in paragraphs (3) and (4) of this
subsection (d), a first offense of aggravated unlawful use of a weapon
committed with a firearm by a person 18 years of age or older where the
factors listed in both items (A) and (C) or both items (A-5) and (C) of
paragraph (3) of subsection (a) are present is a Class 4 felony, for
which the person shall be sentenced to a term of imprisonment of not
less than one year and not more than 3 years.

(3) Aggravated unlawful use of a weapon by a person who has been
previously convicted of a felony in this State or another jurisdiction
is a Class 2 felony for which the person shall be sentenced to a term of
imprisonment of not less than 3 years and not more than 7 years, except
as provided for in Section 5-4.5-110 of the Unified Code of Corrections.

(4) Aggravated unlawful use of a weapon while wearing or in possession
of body armor as defined in Section 33F-1 by a person who has not been
issued a valid Firearms Owner's Identification Card in accordance with
Section 5 of the Firearm Owners Identification Card Act is a Class X
felony.

(e) The possession of each firearm in violation of this Section
constitutes a single and separate violation.

(Source: P.A. 100-3, eff. 1-1-18; 100-201, eff. 8-18-17 .)

\hypertarget{ilcs-524-1.7}{%
\subsection*{(720 ILCS 5/24-1.7)}\label{ilcs-524-1.7}}
\addcontentsline{toc}{subsection}{(720 ILCS 5/24-1.7)}

\hypertarget{sec.-24-1.7.-armed-habitual-criminal.}{%
\section*{Sec. 24-1.7. Armed habitual
criminal.}\label{sec.-24-1.7.-armed-habitual-criminal.}}
\addcontentsline{toc}{section}{Sec. 24-1.7. Armed habitual criminal.}

\markright{Sec. 24-1.7. Armed habitual criminal.}

(a) A person commits the offense of being an armed habitual criminal if
he or she receives, sells, possesses, or transfers any firearm after
having been convicted a total of 2 or more times of any combination of
the following offenses:

(1) a forcible felony as defined in Section 2-8 of this Code;

(2) unlawful use of a weapon by a felon; aggravated unlawful use of a
weapon; aggravated discharge of a firearm; vehicular hijacking;
aggravated vehicular hijacking; aggravated battery of a child as
described in Section 12-4.3 or subdivision (b)(1) of Section 12-3.05;
intimidation; aggravated intimidation; gunrunning; home invasion; or
aggravated battery with a firearm as described in Section 12-4.2 or
subdivision (e)(1), (e)(2), (e)(3), or (e)(4) of Section 12-3.05; or

(3) any violation of the Illinois Controlled

Substances Act or the Cannabis Control Act that is punishable as a Class
3 felony or higher.

(b) Sentence. Being an armed habitual criminal is a Class X felony.

(Source: P.A. 96-1551, eff. 7-1-11 .)

\hypertarget{ilcs-524-1.8}{%
\subsection*{(720 ILCS 5/24-1.8)}\label{ilcs-524-1.8}}
\addcontentsline{toc}{subsection}{(720 ILCS 5/24-1.8)}

\hypertarget{sec.-24-1.8.-unlawful-possession-of-a-firearm-by-a-street-gang-member.}{%
\section*{Sec. 24-1.8. Unlawful possession of a firearm by a street gang
member.}\label{sec.-24-1.8.-unlawful-possession-of-a-firearm-by-a-street-gang-member.}}
\addcontentsline{toc}{section}{Sec. 24-1.8. Unlawful possession of a
firearm by a street gang member.}

\markright{Sec. 24-1.8. Unlawful possession of a firearm by a street
gang member.}

(a) A person commits unlawful possession of a firearm by a street gang
member when he or she knowingly:

(1) possesses, carries, or conceals on or about his or her person a
firearm and firearm ammunition while on any street, road, alley,
gangway, sidewalk, or any other lands, except when inside his or her own
abode or inside his or her fixed place of business, and has not been
issued a currently valid Firearm Owner's Identification Card and is a
member of a street gang; or

(2) possesses or carries in any vehicle a firearm and firearm ammunition
which are both immediately accessible at the time of the offense while
on any street, road, alley, or any other lands, except when inside his
or her own abode or garage, and has not been issued a currently valid
Firearm Owner's Identification Card and is a member of a street gang.

(b) Unlawful possession of a firearm by a street gang member is a Class
2 felony for which the person, if sentenced to a term of imprisonment,
shall be sentenced to no less than 3 years and no more than 10 years. A
period of probation, a term of periodic imprisonment or conditional
discharge shall not be imposed for the offense of unlawful possession of
a firearm by a street gang member when the firearm was loaded or
contained firearm ammunition and the court shall sentence the offender
to not less than the minimum term of imprisonment authorized for the
Class 2 felony.

(c) For purposes of this Section:

``Street gang'' or ``gang'' has the meaning ascribed to it in Section 10
of the Illinois Streetgang Terrorism Omnibus Prevention Act.

``Street gang member'' or ``gang member'' has the meaning ascribed to it
in Section 10 of the Illinois Streetgang Terrorism Omnibus Prevention
Act.

(Source: P.A. 96-829, eff. 12-3-09.)

\hypertarget{ilcs-524-1.9}{%
\subsection*{(720 ILCS 5/24-1.9)}\label{ilcs-524-1.9}}
\addcontentsline{toc}{subsection}{(720 ILCS 5/24-1.9)}

\hypertarget{sec.-24-1.9.-manufacture-possession-delivery-sale-and-purchase-of-assault-weapons-.50-caliber-rifles-and-.50-caliber-cartridges.}{%
\section*{Sec. 24-1.9. Manufacture, possession, delivery, sale, and
purchase of assault weapons, .50 caliber rifles, and .50 caliber
cartridges.}\label{sec.-24-1.9.-manufacture-possession-delivery-sale-and-purchase-of-assault-weapons-.50-caliber-rifles-and-.50-caliber-cartridges.}}
\addcontentsline{toc}{section}{Sec. 24-1.9. Manufacture, possession,
delivery, sale, and purchase of assault weapons, .50 caliber rifles, and
.50 caliber cartridges.}

\markright{Sec. 24-1.9. Manufacture, possession, delivery, sale, and
purchase of assault weapons, .50 caliber rifles, and .50 caliber
cartridges.}

(a) Definitions. In this Section:

(1) ``Assault weapon'' means any of the following, except as provided in
subdivision (2) of this subsection:

(A) A semiautomatic rifle that has the capacity to accept a detachable
magazine or that may be readily modified to accept a detachable
magazine, if the firearm has one or more of the following:

(i) a pistol grip or thumbhole stock;

(ii) any feature capable of functioning as a protruding grip that can be
held by the non-trigger hand;

(iii) a folding, telescoping, thumbhole, or detachable stock, or a stock
that is otherwise foldable or adjustable in a manner that operates to
reduce the length, size, or any other dimension, or otherwise enhances
the concealability of, the weapon;

(iv) a flash suppressor;

(v) a grenade launcher;

(vi) a shroud attached to the barrel or that partially or completely
encircles the barrel, allowing the bearer to hold the firearm with the
non-trigger hand without being burned, but excluding a slide that
encloses the barrel.

(B) A semiautomatic rifle that has a fixed magazine with the capacity to
accept more than 10 rounds, except for an attached tubular device
designed to accept, and capable of operating only with, .22 caliber
rimfire ammunition.

(C) A semiautomatic pistol that has the capacity to accept a detachable
magazine or that may be readily modified to accept a detachable
magazine, if the firearm has one or more of the following:

(i) a threaded barrel;

(ii) a second pistol grip or another feature capable of functioning as a
protruding grip that can be held by the non-trigger hand;

(iii) a shroud attached to the barrel or that partially or completely
encircles the barrel, allowing the bearer to hold the firearm with the
non-trigger hand without being burned, but excluding a slide that
encloses the barrel;

(iv) a flash suppressor;

(v) the capacity to accept a detachable magazine at some location
outside of the pistol grip; or

(vi) a buffer tube, arm brace, or other part that protrudes horizontally
behind the pistol grip and is designed or redesigned to allow or
facilitate a firearm to be fired from the shoulder.

(D) A semiautomatic pistol that has a fixed magazine with the capacity
to accept more than 15 rounds.

(E) Any shotgun with a revolving cylinder.

(F) A semiautomatic shotgun that has one or more of the following:

(i) a pistol grip or thumbhole stock;

(ii) any feature capable of functioning as a protruding grip that can be
held by the non-trigger hand;

(iii) a folding or thumbhole stock;

(iv) a grenade launcher;

(v) a fixed magazine with the capacity of more than 5 rounds; or

(vi) the capacity to accept a detachable magazine.

(G) Any semiautomatic firearm that has the capacity to accept a belt
ammunition feeding device.

(H) Any firearm that has been modified to be operable as an assault
weapon as defined in this Section.

(I) Any part or combination of parts designed or intended to convert a
firearm into an assault weapon, including any combination of parts from
which an assault weapon may be readily assembled if those parts are in
the possession or under the control of the same person.

(J) All of the following rifles, copies, duplicates, variants, or
altered facsimiles with the capability of any such weapon:

(i) All AK types, including the following:

(I) AK, AK47, AK47S, AK-74, AKM, AKS, ARM, MAK90, MISR, NHM90, NHM91,
SA85, SA93, Vector Arms AK-47, VEPR, WASR-10, and WUM.

(II) IZHMASH Saiga AK.

(III) MAADI AK47 and ARM.

(IV) Norinco 56S, 56S2, 84S, and 86S.

(V) Poly Technologies AK47 and AKS.

(VI) SKS with a detachable magazine.

(ii) all AR types, including the following:

(I) AR-10.

(II) AR-15.

(III) Alexander Arms Overmatch Plus 16.

(IV) Armalite M15 22LR Carbine.

(V) Armalite M15-T.

(VI) Barrett REC7.

(VII) Beretta AR-70.

(VIII) Black Rain Ordnance Recon Scout.

(IX) Bushmaster ACR.

(X) Bushmaster Carbon 15.

(XI) Bushmaster MOE series.

(XII) Bushmaster XM15.

(XIII) Chiappa Firearms MFour rifles.

(XIV) Colt Match Target rifles.

(XV) CORE Rifle Systems CORE15 rifles.

(XVI) Daniel Defense M4A1 rifles.

(XVII) Devil Dog Arms 15 Series rifles.

(XVIII) Diamondback DB15 rifles.

(XIX) DoubleStar AR rifles.

(XX) DPMS Tactical rifles.

(XXI) DSA Inc.~ZM-4 Carbine.

(XXII) Heckler \& Koch MR556.

(XXIII) High Standard HSA-15 rifles.

(XXIV) Jesse James Nomad AR-15 rifle.

(XXV) Knight's Armament SR-15.

(XXVI) Lancer L15 rifles.

(XXVII) MGI Hydra Series rifles.

(XXVIII) Mossberg MMR Tactical rifles.

(XXIX) Noreen Firearms BN 36 rifle.

(XXX) Olympic Arms.

(XXXI) POF USA P415.

(XXXII) Precision Firearms AR rifles.

(XXXIII) Remington R-15 rifles.

(XXXIV) Rhino Arms AR rifles.

(XXXV) Rock River Arms LAR-15 or Rock River Arms LAR-47.

(XXXVI) Sig Sauer SIG516 rifles and MCX rifles.

(XXXVII) Smith \& Wesson M\&P15 rifles.

(XXXVIII) Stag Arms AR rifles.

(XXXIX) Sturm, Ruger \& Co.~SR556 and AR-556 rifles.

(XL) Uselton Arms Air-Lite M-4 rifles.

(XLI) Windham Weaponry AR rifles.

(XLII) WMD Guns Big Beast.

(XLIII) Yankee Hill Machine Company, Inc.~YHM-15 rifles.

(iii) Barrett M107A1.

(iv) Barrett M82A1.

(v) Beretta CX4 Storm.

(vi) Calico Liberty Series.

(vii) CETME Sporter.

(viii) Daewoo K-1, K-2, Max 1, Max 2, AR 100, andb AR 110C.

(ix) Fabrique Nationale/FN Herstal FAL, LAR, 22 FNC, 308 Match, L1A1
Sporter, PS90, SCAR, and FS2000.

(x) Feather Industries AT-9.

(xi) Galil Model AR and Model ARM.

(xii) Hi-Point Carbine.

(xiii) HK-91, HK-93, HK-94, HK-PSG-1, and HK USC.

(xiv) IWI TAVOR, Galil ACE rifle.

(xv) Kel-Tec Sub-2000, SU-16, and RFB.

(xvi) SIG AMT, SIG PE-57, Sig Sauer SG 550, Sig

Sauer SG 551, and SIG MCX.

(xvii) Springfield Armory SAR-48.

(xviii) Steyr AUG.

(xix) Sturm, Ruger \& Co.~Mini-14 Tactical Rifle

M-14/20CF.

(xx) All Thompson rifles, including the following:

(I) Thompson M1SB.

(II) Thompson T1100D.

(III) Thompson T150D.

\begin{enumerate}
\def\labelenumi{(\Roman{enumi})}
\setcounter{enumi}{3}
\tightlist
\item
  Thompson T1B.

  (V) Thompson T1B100D.

  \begin{enumerate}
  \def\labelenumii{(\Roman{enumii})}
  \setcounter{enumii}{5}
  \tightlist
  \item
    Thompson T1B50D.

    (VII) Thompson T1BSB.

    (VIII) Thompson T1-C.

    (IX) Thompson T1D.

    (X) Thompson T1SB.

    (XI) Thompson T5.

    (XII) Thompson T5100D.

    (XIII) Thompson TM1.

    (XIV) Thompson TM1C.

    (xxi) UMAREX UZI rifle.

    (xxii) UZI Mini Carbine, UZI Model A Carbine, and UZI Model B
    Carbine.

    (xxiii) Valmet M62S, M71S, and M78.

    (xxiv) Vector Arms UZI Type.

    (xxv) Weaver Arms Nighthawk.

    (xxvi) Wilkinson Arms Linda Carbine.

    (K) All of the following pistols, copies, duplicates, variants, or
    altered facsimiles with the capability of any such weapon thereof:

    (i) All AK types, including the following:

    (I) Centurion 39 AK pistol.

    (II) CZ Scorpion pistol.

    (III) Draco AK-47 pistol.

    (IV) HCR AK-47 pistol.

    (V) IO Inc.~Hellpup AK-47 pistol.

    (VI) Krinkov pistol.

    (VII) Mini Draco AK-47 pistol.

    (VIII) PAP M92 pistol.

    (IX) Yugo Krebs Krink pistol.

    (ii) All AR types, including the following:

    (I) American Spirit AR-15 pistol.

    (II) Bushmaster Carbon 15 pistol.

    (III) Chiappa Firearms M4 Pistol GEN II.

    (IV) CORE Rifle Systems CORE15 Roscoe pistol.

    (V) Daniel Defense MK18 pistol.

    (VI) DoubleStar Corporation AR pistol.

    (VII) DPMS AR-15 pistol.

    (VIII) Jesse James Nomad AR-15 pistol.

    (IX) Olympic Arms AR-15 pistol.

    (X) Osprey Armament MK-18 pistol.

    \begin{enumerate}
    \def\labelenumiii{(\Roman{enumiii})}
    \setcounter{enumiii}{10}
    \tightlist
    \item
      POF USA AR pistols.

      (XII) Rock River Arms LAR 15 pistol.

      (XIII) Uselton Arms Air-Lite M-4 pistol.

      (iii) Calico pistols.

      (iv) DSA SA58 PKP FAL pistol.

      (v) Encom MP-9 and MP-45.

      (vi) Heckler \& Koch model SP-89 pistol.

      (vii) Intratec AB-10, TEC-22 Scorpion, TEC-9, and

      TEC-DC9.

      (viii) IWI Galil Ace pistol, UZI PRO pistol.

      (ix) Kel-Tec PLR 16 pistol.

      (x) All MAC types, including the following:

      (I) MAC-10.

      (II) MAC-11.

      (III) Masterpiece Arms MPA A930 Mini Pistol, MPA460 Pistol, MPA
      Tactical Pistol, and MPA Mini Tactical Pistol.

      (IV) Military Armament Corp.~Ingram M-11.

      (V) Velocity Arms VMAC.

      (xi) Sig Sauer P556 pistol.

      (xii) Sites Spectre.

      (xiii) All Thompson types, including the following:

      (I) Thompson TA510D.

      (II) Thompson TA5.

      (xiv) All UZI types, including Micro-UZI.

      (L) All of the following shotguns, copies, duplicates, variants,
      or altered facsimiles with the capability of any such weapon
      thereof:

      (i) DERYA Anakon MC-1980, Anakon SD12.

      (ii) Doruk Lethal shotguns.

      (iii) Franchi LAW-12 and SPAS 12.

      (iv) All IZHMASH Saiga 12 types, including the following:

      (I) IZHMASH Saiga 12.

      (II) IZHMASH Saiga 12S.

      (III) IZHMASH Saiga 12S EXP-01.

      (IV) IZHMASH Saiga 12K.

      (V) IZHMASH Saiga 12K-030.

      (VI) IZHMASH Saiga 12K-040 Taktika.

      (v) Streetsweeper.

      (vi) Striker 12.

      (2) ``Assault weapon'' does not include:

      (A) Any firearm that is an unserviceable firearm or has been made
      permanently inoperable.

      (B) An antique firearm or a replica of an antique firearm.

      (C) A firearm that is manually operated by bolt, pump, lever or
      slide action, unless the firearm is a shotgun with a revolving
      cylinder.

      (D) Any air rifle as defined in Section 24.8-0.1 of this Code.

      (E) Any handgun, as defined under the Firearm Concealed Carry Act,
      unless otherwise listed in this Section.

      (3) ``Assault weapon attachment'' means any device capable of
      being attached to a firearm that is specifically designed for
      making or converting a firearm into any of the firearms listed in
      paragraph (1) of this subsection (a).

      (4) ``Antique firearm'' has the meaning ascribed to it in 18
      U.S.C. 921(a)(16).

      (5) ``.50 caliber rifle'' means a centerfire rifle capable of
      firing a .50 caliber cartridge. The term does not include any
      antique firearm, any shotgun including a shotgun that has a rifle
      barrel, or any muzzle-loader which uses black powder for hunting
      or historical reenactments.

      (6) ``.50 caliber cartridge'' means a cartridge in .50 BMG
      caliber, either by designation or actual measurement, that is
      capable of being fired from a centerfire rifle. The term ``.50
      caliber cartridge'' does not include any memorabilia or display
      item that is filled with a permanent inert substance or that is
      otherwise permanently altered in a manner that prevents ready
      modification for use as live ammunition or shotgun ammunition with
      a caliber measurement that is equal to or greater than .50
      caliber.

      (7) ``Detachable magazine'' means an ammunition feeding device
      that may be removed from a firearm without disassembly of the
      firearm action, including an ammunition feeding device that may be
      readily removed from a firearm with the use of a bullet,
      cartridge, accessory, or other tool, or any other object that
      functions as a tool, including a bullet or cartridge.

      (8) ``Fixed magazine'' means an ammunition feeding device that is
      permanently attached to a firearm, or contained in and not
      removable from a firearm, or that is otherwise not a detachable
      magazine, but does not include an attached tubular device designed
      to accept, and capable of operating only with, .22 caliber rimfire
      ammunition.

      (b) Except as provided in subsections (c), (d), and (e), on or
      after the effective date of this amendatory Act of the 102nd
      General Assembly, it is unlawful for any person within this State
      to knowingly manufacture, deliver, sell, import, or purchase or
      cause to be manufactured, delivered, sold, imported, or purchased
      by another, an assault weapon, assault weapon attachment, .50
      caliber rifle, or .50 caliber cartridge.

      (c) Except as otherwise provided in subsection (d), beginning
      January 1, 2024, it is unlawful for any person within this State
      to knowingly possess an assault weapon, assault weapon attachment,
      .50 caliber rifle, or .50 caliber cartridge.

      (d) This Section does not apply to a person's possession of an
      assault weapon, assault weapon attachment, .50 caliber rifle, or
      .50 caliber cartridge device if the person lawfully possessed that
      assault weapon, assault weapon attachment, .50 caliber rifle, or
      .50 caliber cartridge prohibited by subsection (c) of this
      Section, if the person has provided in an endorsement affidavit,
      prior to January 1, 2024, under oath or affirmation and in the
      form and manner prescribed by the Illinois State Police, no later
      than October 1, 2023:

      (1) the affiant's Firearm Owner's Identification Card number;

      (2) an affirmation that the affiant: (i) possessed an assault
      weapon, assault weapon attachment, .50 caliber rifle, or .50
      caliber cartridge before the effective date of this amendatory Act
      of the 102nd General Assembly; or (ii) inherited the assault
      weapon, assault weapon attachment, .50 caliber rifle, or .50
      caliber cartridge from a person with an endorsement under this
      Section or from a person authorized under subdivisions (1) through
      (5) of subsection (e) to possess the assault weapon, assault
      weapon attachment, .50 caliber rifle, or .50 caliber cartridge;
      and

      (3) the make, model, caliber, and serial number of the .50 caliber
      rifle or assault weapon or assault weapons listed in paragraphs
      (J), (K), and (L) of subdivision (1) of subsection (a) of this
      Section possessed by the affiant prior to the effective date of
      this amendatory Act of the 102nd General Assembly and any assault
      weapons identified and published by the Illinois State Police
      pursuant to this subdivision (3). No later than October 1, 2023,
      and every October 1 thereafter, the Illinois State Police shall,
      via rulemaking, identify, publish, and make available on its
      website, the list of assault weapons subject to an endorsement
      affidavit under this subsection (d). The list shall identify, but
      is not limited to, the copies, duplicates, variants, and altered
      facsimiles of the assault weapons identified in paragraphs (J),
      (K), and (L) of subdivision (1) of subsection (a) of this Section
      and shall be consistent with the definition of ``assault weapon''
      identified in this Section. The Illinois State Police may adopt
      emergency rulemaking in accordance with Section 5-45 of the
      Illinois Administrative Procedure Act. The adoption of emergency
      rules authorized by Section 5-45 of the Illinois Administrative
      Procedure Act and this paragraph is deemed to be necessary for the
      public interest, safety, and welfare.

      The affidavit form shall include the following statement printed
      in bold type: ``Warning: Entering false information on this form
      is punishable as perjury under Section 32-2 of the Criminal Code
      of 2012. Entering false information on this form is a violation of
      the Firearm Owners Identification Card Act.''

      In any administrative, civil, or criminal proceeding in this
      State, a completed endorsement affidavit submitted to the Illinois
      State Police by a person under this Section creates a rebuttable
      presumption that the person is entitled to possess and transport
      the assault weapon, assault weapon attachment, .50 caliber rifle,
      or .50 caliber cartridge.

      Beginning 90 days after the effective date of this amendatory Act
      of the 102nd General Assembly, a person authorized under this
      Section to possess an assault weapon, assault weapon attachment,
      .50 caliber rifle, or .50 caliber cartridge shall possess such
      items only:

      (1) on private property owned or immediately controlled by the
      person;

      (2) on private property that is not open to the public with the
      express permission of the person who owns or immediately controls
      such property;

      (3) while on the premises of a licensed firearms dealer or
      gunsmith for the purpose of lawful repair;

      (4) while engaged in the legal use of the assault weapon, assault
      weapon attachment, .50 caliber rifle, or .50 caliber cartridge at
      a properly licensed firing range or sport shooting competition
      venue; or

      (5) while traveling to or from these locations, provided that the
      assault weapon, assault weapon attachment, or .50 caliber rifle is
      unloaded and the assault weapon, assault weapon attachment, .50
      caliber rifle, or .50 caliber cartridge is enclosed in a case,
      firearm carrying box, shipping box, or other container.

      Beginning on January 1, 2024, the person with the endorsement for
      an assault weapon, assault weapon attachment, .50 caliber rifle,
      or .50 caliber cartridge or a person authorized under subdivisions
      (1) through (5) of subsection (e) to possess an assault weapon,
      assault weapon attachment, .50 caliber rifle, or .50 caliber
      cartridge may transfer the assault weapon, assault weapon
      attachment, .50 caliber rifle, or .50 caliber cartridge only to an
      heir, an individual residing in another state maintaining it in
      another state, or a dealer licensed as a federal firearms dealer
      under Section 923 of the federal Gun Control Act of 1968. Within
      10 days after transfer of the weapon except to an heir, the person
      shall notify the Illinois State Police of the name and address of
      the transferee and comply with the requirements of subsection (b)
      of Section 3 of the Firearm Owners Identification Card Act. The
      person to whom the weapon or ammunition is transferred shall,
      within 60 days of the transfer, complete an affidavit required
      under this Section. A person to whom the weapon is transferred may
      transfer it only as provided in this subsection.

      Except as provided in subsection (e) and beginning on January 1,
      2024, any person who moves into this State in possession of an
      assault weapon, assault weapon attachment, .50 caliber rifle, or
      .50 caliber cartridge shall, within 60 days, apply for a Firearm
      Owners Identification Card and complete an endorsement application
      as outlined in subsection (d).

      Notwithstanding any other law, information contained in the
      endorsement affidavit shall be confidential, is exempt from
      disclosure under the Freedom of Information Act, and shall not be
      disclosed, except to law enforcement agencies acting in the
      performance of their duties.

      (e) The provisions of this Section regarding the purchase or
      possession of assault weapons, assault weapon attachments, .50
      caliber rifles, and .50 cartridges, as well as the provisions of
      this Section that prohibit causing those items to be purchased or
      possessed, do not apply to:

      (1) Peace officers, as defined in Section 2-13 of this Code.

      (2) Qualified law enforcement officers and qualified retired law
      enforcement officers as defined in the Law Enforcement Officers
      Safety Act of 2004 (18 U.S.C. 926B and 926C) and as recognized
      under Illinois law.

      (3) Acquisition and possession by a federal, State, or local law
      enforcement agency for the purpose of equipping the agency's peace
      officers as defined in paragraph (1) or (2) of this subsection
      (e).

      (4) Wardens, superintendents, and keepers of prisons,
      penitentiaries, jails, and other institutions for the detention of
      persons accused or convicted of an offense.

      (5) Members of the Armed Services or Reserve Forces of the United
      States or the Illinois National Guard, while performing their
      official duties or while traveling to or from their places of
      duty.

      (6) Any company that employs armed security officers in this State
      at a nuclear energy, storage, weapons, or development site or
      facility regulated by the federal Nuclear Regulatory Commission
      and any person employed as an armed security force member at a
      nuclear energy, storage, weapons, or development site or facility
      regulated by the federal Nuclear Regulatory Commission who has
      completed the background screening and training mandated by the
      rules and regulations of the federal Nuclear Regulatory Commission
      and while performing official duties.

      (7) Any private security contractor agency licensed under the
      Private Detective, Private Alarm, Private Security, Fingerprint
      Vendor, and Locksmith Act of 2004 that employs private security
      contractors and any private security contractor who is licensed
      and has been issued a firearm control card under the Private
      Detective, Private Alarm, Private Security, Fingerprint Vendor,
      and Locksmith Act of 2004 while performing official duties.

      The provisions of this Section do not apply to the manufacture,
      delivery, sale, import, purchase, or possession of an assault
      weapon, assault weapon attachment, .50 caliber rifle, or .50
      caliber cartridge or causing the manufacture, delivery, sale,
      importation, purchase, or possession of those items:

      (A) for sale or transfer to persons authorized under subdivisions
      (1) through (7) of this subsection (e) to possess those items;

      (B) for sale or transfer to the United States or any department or
      agency thereof; or

      (C) for sale or transfer in another state or for export.

      This Section does not apply to or affect any of the following:

      (i) Possession of any firearm if that firearm is sanctioned by the
      International Olympic Committee and by USA Shooting, the national
      governing body for international shooting competition in the
      United States, but only when the firearm is in the actual
      possession of an Olympic target shooting competitor or target
      shooting coach for the purpose of storage, transporting to and
      from Olympic target shooting practice or events if the firearm is
      broken down in a nonfunctioning state, is not immediately
      accessible, or is unloaded and enclosed in a firearm case,
      carrying box, shipping box, or other similar portable container
      designed for the safe transportation of firearms, and when the
      Olympic target shooting competitor or target shooting coach is
      engaging in those practices or events. For the purposes of this
      paragraph (8), ``firearm'' has the meaning provided in Section 1.1
      of the Firearm Owners Identification Card Act.

      (ii) Any nonresident who transports, within 24 hours, a weapon for
      any lawful purpose from any place where the nonresident may
      lawfully possess and carry that weapon to any other place where
      the nonresident may lawfully possess and carry that weapon if,
      during the transportation, the weapon is unloaded, and neither the
      weapon nor any ammunition being transported is readily accessible
      or is directly accessible from the passenger compartment of the
      transporting vehicle. In the case of a vehicle without a
      compartment separate from the driver's compartment, the weapon or
      ammunition shall be contained in a locked container other than the
      glove compartment or console.

      (iii) Possession of a weapon at an event taking place at the World
      Shooting and Recreational Complex at Sparta, only while engaged in
      the legal use of the weapon, or while traveling to or from that
      location if the weapon is broken down in a nonfunctioning state,
      is not immediately accessible, or is unloaded and enclosed in a
      firearm case, carrying box, shipping box, or other similar
      portable container designed for the safe transportation of
      firearms.

      (iv) Possession of a weapon only for hunting use expressly
      permitted under the Wildlife Code, or while traveling to or from a
      location authorized for this hunting use under the Wildlife Code
      if the weapon is broken down in a nonfunctioning state, is not
      immediately accessible, or is unloaded and enclosed in a firearm
      case, carrying box, shipping box, or other similar portable
      container designed for the safe transportation of firearms. By
      October 1, 2023, the Illinois State Police, in consultation with
      the Department of Natural Resources, shall adopt rules concerning
      the list of applicable weapons approved under this subparagraph
      (iv). The Illinois State Police may adopt emergency rules in
      accordance with Section 5-45 of the Illinois Administrative
      Procedure Act. The adoption of emergency rules authorized by
      Section 5-45 of the Illinois Administrative Procedure Act and this
      paragraph is deemed to be necessary for the public interest,
      safety, and welfare.

      (v) The manufacture, transportation, possession, sale, or rental
      of blank-firing assault weapons and .50 caliber rifles, or the
      weapon's respective attachments, to persons authorized or
      permitted, or both authorized and permitted, to acquire and
      possess these weapons or attachments for the purpose of rental for
      use solely as props for a motion picture, television, or video
      production or entertainment event.

      Any person not subject to this Section may submit an endorsement
      affidavit if the person chooses.

      (f) Any sale or transfer with a background check initiated to the
      Illinois State Police on or before the effective date of this
      amendatory Act of the 102nd General Assembly is allowed to be
      completed after the effective date of this amendatory Act once an
      approval is issued by the Illinois State Police and any applicable
      waiting period under Section 24-3 has expired.

      (g) The Illinois State Police shall take all steps necessary to
      carry out the requirements of this Section within by October 1,
      2023.

      (h) The Department of the State Police shall also develop and
      implement a public notice and public outreach campaign to promote
      awareness about the provisions of this amendatory Act of the 102nd
      General Assembly and to increase compliance with this Section.

      (Source: P.A. 102-1116, eff. 1-10-23.)

      \hypertarget{ilcs-524-1.10}{%
      \subsection*{(720 ILCS 5/24-1.10)}\label{ilcs-524-1.10}}
      \addcontentsline{toc}{subsection}{(720 ILCS 5/24-1.10)}

      \hypertarget{sec.-24-1.10.-manufacture-delivery-sale-and-possession-of-large-capacity-ammunition-feeding-devices.}{%
      \section*{Sec. 24-1.10. Manufacture, delivery, sale, and
      possession of large capacity ammunition feeding
      devices.}\label{sec.-24-1.10.-manufacture-delivery-sale-and-possession-of-large-capacity-ammunition-feeding-devices.}}
      \addcontentsline{toc}{section}{Sec. 24-1.10. Manufacture,
      delivery, sale, and possession of large capacity ammunition
      feeding devices.}

      \markright{Sec. 24-1.10. Manufacture, delivery, sale, and
      possession of large capacity ammunition feeding devices.}

      (a) In this Section:

      ``Handgun'' has the meaning ascribed to it in the Firearm
      Concealed Carry Act.

      ``Long gun'' means a rifle or shotgun.

      ``Large capacity ammunition feeding device'' means:

      (1) a magazine, belt, drum, feed strip, or similar device that has
      a capacity of, or that can be readily restored or converted to
      accept, more than 10 rounds of ammunition for long guns and more
      than 15 rounds of ammunition for handguns; or

      (2) any combination of parts from which a device described in
      paragraph (1) can be assembled.

      ``Large capacity ammunition feeding device'' does not include an
      attached tubular device designed to accept, and capable of
      operating only with, .22 caliber rimfire ammunition. ``Large
      capacity ammunition feeding device'' does not include a tubular
      magazine that is contained in a lever-action firearm or any device
      that has been made permanently inoperable.

      (b) Except as provided in subsections (e) and (f), it is unlawful
      for any person within this State to knowingly manufacture,
      deliver, sell, purchase, or cause to be manufactured, delivered,
      sold, or purchased a large capacity ammunition feeding device.

      (c) Except as provided in subsections (d), (e), and (f), and
      beginning 90 days after the effective date of this amendatory Act
      of the 102nd General Assembly, it is unlawful to knowingly possess
      a large capacity ammunition feeding device.

      (d) Subsection (c) does not apply to a person's possession of a
      large capacity ammunition feeding device if the person lawfully
      possessed that large capacity ammunition feeding device before the
      effective date of this amendatory Act of the 102nd General
      Assembly, provided that the person shall possess such device only:

      (1) on private property owned or immediately controlled by the
      person;

      (2) on private property that is not open to the public with the
      express permission of the person who owns or immediately controls
      such property;

      (3) while on the premises of a licensed firearms dealer or
      gunsmith for the purpose of lawful repair;

      (4) while engaged in the legal use of the large capacity
      ammunition feeding device at a properly licensed firing range or
      sport shooting competition venue; or

      (5) while traveling to or from these locations, provided that the
      large capacity ammunition feeding device is stored unloaded and
      enclosed in a case, firearm carrying box, shipping box, or other
      container.

      A person authorized under this Section to possess a large capacity
      ammunition feeding device may transfer the large capacity
      ammunition feeding device only to an heir, an individual residing
      in another state maintaining it in another state, or a dealer
      licensed as a federal firearms dealer under Section 923 of the
      federal Gun Control Act of 1968. Within 10 days after transfer of
      the large capacity ammunition feeding device except to an heir,
      the person shall notify the Illinois State Police of the name and
      address of the transferee and comply with the requirements of
      subsection (b) of Section 3 of the Firearm Owners Identification
      Card Act. The person to whom the large capacity ammunition feeding
      device is transferred shall, within 60 days of the transfer,
      notify the Illinois State Police of the person's acquisition and
      comply with the requirements of subsection (b) of Section 3 of the
      Firearm Owners Identification Card Act. A person to whom the large
      capacity ammunition feeding device is transferred may transfer it
      only as provided in this subsection.

      Except as provided in subsections (e) and (f) and beginning 90
      days after the effective date of this amendatory Act of the 102nd
      General Assembly, any person who moves into this State in
      possession of a large capacity ammunition feeding device shall,
      within 60 days, apply for a Firearm Owners Identification Card.

      (e) The provisions of this Section regarding the purchase or
      possession of large capacity ammunition feeding devices, as well
      as the provisions of this Section that prohibit causing those
      items to be purchased or possessed, do not apply to:

      (1) Peace officers as defined in Section 2-13 of this

      Code.

      (2) Qualified law enforcement officers and qualified retired law
      enforcement officers as defined in the Law Enforcement Officers
      Safety Act of 2004 (18 U.S.C. 926B and 926C) and as recognized
      under Illinois law.

      (3) A federal, State, or local law enforcement agency for the
      purpose of equipping the agency's peace officers as defined in
      paragraph (1) or (2) of this subsection (e).

      (4) Wardens, superintendents, and keepers of prisons,
      penitentiaries, jails, and other institutions for the detention of
      persons accused or convicted of an offense.

      (5) Members of the Armed Services or Reserve Forces of the United
      States or the Illinois National Guard, while their official duties
      or while traveling to or from their places of duty.

      (6) Any company that employs armed security officers in this State
      at a nuclear energy, storage, weapons, or development site or
      facility regulated by the federal Nuclear Regulatory Commission
      and any person employed as an armed security force member at a
      nuclear energy, storage, weapons, or development site or facility
      regulated by the federal Nuclear Regulatory Commission who has
      completed the background screening and training mandated by the
      rules and regulations of the federal Nuclear Regulatory Commission
      and while performing official duties.

      (7) Any private security contractor agency licensed under the
      Private Detective, Private Alarm, Private Security, Fingerprint
      Vendor, and Locksmith Act of 2004 that employs private security
      contractors and any private security contractor who is licensed
      and has been issued a firearm control card under the Private
      Detective, Private Alarm, Private Security, Fingerprint Vendor,
      and Locksmith Act of 2004 while performing official duties.

      (f) This Section does not apply to or affect any of the following:

      (1) Manufacture, delivery, sale, importation, purchase, or
      possession or causing to be manufactured, delivered, sold,
      imported, purchased, or possessed a large capacity ammunition
      feeding device:

      (A) for sale or transfer to persons authorized under subdivisions
      (1) through (7) of subsection (e) to possess those items;

      (B) for sale or transfer to the United States or any department or
      agency thereof; or

      (C) for sale or transfer in another state or for export.

      (2) Sale or rental of large capacity ammunition feeding devices
      for blank-firing assault weapons and .50 caliber rifles, to
      persons authorized or permitted, or both authorized and permitted,
      to acquire these devices for the purpose of rental for use solely
      as props for a motion picture, television, or video production or
      entertainment event.

      (g) Sentence. A person who knowingly manufactures, delivers,
      sells, purchases, possesses, or causes to be manufactured,
      delivered, sold, possessed, or purchased in violation of this
      Section a large capacity ammunition feeding device capable of
      holding more than 10 rounds of ammunition for long guns or more
      than 15 rounds of ammunition for handguns commits a petty offense
      with a fine of \$1,000 for each violation.

      (h) The Department of the State Police shall also develop and
      implement a public notice and public outreach campaign to promote
      awareness about the provisions of this amendatory Act of the 102nd
      General Assembly and to increase compliance with this Section.

      (Source: P.A. 102-1116, eff. 1-10-23.)

      \hypertarget{ilcs-524-2-text-of-section-from-p.a.-102-779}{%
      \subsection*{(720 ILCS 5/24-2) (Text of Section from P.A.
      102-779)}\label{ilcs-524-2-text-of-section-from-p.a.-102-779}}
      \addcontentsline{toc}{subsection}{(720 ILCS 5/24-2) (Text of
      Section from P.A. 102-779)}

      \hypertarget{sec.-24-2.-exemptions.}{%
      \section*{Sec. 24-2. Exemptions.}\label{sec.-24-2.-exemptions.}}
      \addcontentsline{toc}{section}{Sec. 24-2. Exemptions.}

      \markright{Sec. 24-2. Exemptions.}

      (a) Subsections 24-1(a)(3), 24-1(a)(4), 24-1(a)(10), and
      24-1(a)(13) and Section 24-1.6 do not apply to or affect any of
      the following:

      (1) Peace officers, and any person summoned by a peace officer to
      assist in making arrests or preserving the peace, while actually
      engaged in assisting such officer.

      (2) Wardens, superintendents and keepers of prisons,
      penitentiaries, jails and other institutions for the detention of
      persons accused or convicted of an offense, while in the
      performance of their official duty, or while commuting between
      their homes and places of employment.

      (3) Members of the Armed Services or Reserve Forces of the United
      States or the Illinois National Guard or the Reserve Officers
      Training Corps, while in the performance of their official duty.

      (4) Special agents employed by a railroad or a public utility to
      perform police functions, and guards of armored car companies,
      while actually engaged in the performance of the duties of their
      employment or commuting between their homes and places of
      employment; and watchmen while actually engaged in the performance
      of the duties of their employment.

      (5) Persons licensed as private security contractors, private
      detectives, or private alarm contractors, or employed by a private
      security contractor, private detective, or private alarm
      contractor agency licensed by the Department of Financial and
      Professional Regulation, if their duties include the carrying of a
      weapon under the provisions of the Private Detective, Private
      Alarm, Private Security, Fingerprint Vendor, and Locksmith Act of
      2004, while actually engaged in the performance of the duties of
      their employment or commuting between their homes and places of
      employment. A person shall be considered eligible for this
      exemption if he or she has completed the required 20 hours of
      training for a private security contractor, private detective, or
      private alarm contractor, or employee of a licensed private
      security contractor, private detective, or private alarm
      contractor agency and 28 hours of required firearm training, and
      has been issued a firearm control card by the Department of
      Financial and Professional Regulation. Conditions for the renewal
      of firearm control cards issued under the provisions of this
      Section shall be the same as for those cards issued under the
      provisions of the Private Detective, Private Alarm, Private
      Security, Fingerprint Vendor, and Locksmith Act of 2004. The
      firearm control card shall be carried by the private security
      contractor, private detective, or private alarm contractor, or
      employee of the licensed private security contractor, private
      detective, or private alarm contractor agency at all times when he
      or she is in possession of a concealable weapon permitted by his
      or her firearm control card.

      (6) Any person regularly employed in a commercial or industrial
      operation as a security guard for the protection of persons
      employed and private property related to such commercial or
      industrial operation, while actually engaged in the performance of
      his or her duty or traveling between sites or properties belonging
      to the employer, and who, as a security guard, is a member of a
      security force registered with the Department of Financial and
      Professional Regulation; provided that such security guard has
      successfully completed a course of study, approved by and
      supervised by the Department of Financial and Professional
      Regulation, consisting of not less than 48 hours of training that
      includes the theory of law enforcement, liability for acts, and
      the handling of weapons. A person shall be considered eligible for
      this exemption if he or she has completed the required 20 hours of
      training for a security officer and 28 hours of required firearm
      training, and has been issued a firearm control card by the
      Department of Financial and Professional Regulation. Conditions
      for the renewal of firearm control cards issued under the
      provisions of this Section shall be the same as for those cards
      issued under the provisions of the Private Detective, Private
      Alarm, Private Security, Fingerprint Vendor, and Locksmith Act of
      2004. The firearm control card shall be carried by the security
      guard at all times when he or she is in possession of a
      concealable weapon permitted by his or her firearm control card.

      (7) Agents and investigators of the Illinois

      Legislative Investigating Commission authorized by the Commission
      to carry the weapons specified in subsections 24-1(a)(3) and
      24-1(a)(4), while on duty in the course of any investigation for
      the Commission.

      (8) Persons employed by a financial institution as a security
      guard for the protection of other employees and property related
      to such financial institution, while actually engaged in the
      performance of their duties, commuting between their homes and
      places of employment, or traveling between sites or properties
      owned or operated by such financial institution, and who, as a
      security guard, is a member of a security force registered with
      the Department; provided that any person so employed has
      successfully completed a course of study, approved by and
      supervised by the Department of Financial and Professional
      Regulation, consisting of not less than 48 hours of training which
      includes theory of law enforcement, liability for acts, and the
      handling of weapons. A person shall be considered to be eligible
      for this exemption if he or she has completed the required 20
      hours of training for a security officer and 28 hours of required
      firearm training, and has been issued a firearm control card by
      the Department of Financial and Professional Regulation.
      Conditions for renewal of firearm control cards issued under the
      provisions of this Section shall be the same as for those issued
      under the provisions of the Private Detective, Private Alarm,
      Private Security, Fingerprint Vendor, and Locksmith Act of 2004.
      The firearm control card shall be carried by the security guard at
      all times when he or she is in possession of a concealable weapon
      permitted by his or her firearm control card. For purposes of this
      subsection, ``financial institution'' means a bank, savings and
      loan association, credit union or company providing armored car
      services.

      (9) Any person employed by an armored car company to drive an
      armored car, while actually engaged in the performance of his
      duties.

      (10) Persons who have been classified as peace officers pursuant
      to the Peace Officer Fire Investigation Act.

      (11) Investigators of the Office of the State's

      Attorneys Appellate Prosecutor authorized by the board of
      governors of the Office of the State's Attorneys Appellate
      Prosecutor to carry weapons pursuant to Section 7.06 of the
      State's Attorneys Appellate Prosecutor's Act.

      (12) Special investigators appointed by a State's

      Attorney under Section 3-9005 of the Counties Code.

      (12.5) Probation officers while in the performance of their
      duties, or while commuting between their homes, places of
      employment or specific locations that are part of their assigned
      duties, with the consent of the chief judge of the circuit for
      which they are employed, if they have received weapons training
      according to requirements of the Peace Officer and Probation
      Officer Firearm Training Act.

      (13) Court Security Officers while in the performance of their
      official duties, or while commuting between their homes and places
      of employment, with the consent of the Sheriff.

      (13.5) A person employed as an armed security guard at a nuclear
      energy, storage, weapons or development site or facility regulated
      by the Nuclear Regulatory Commission who has completed the
      background screening and training mandated by the rules and
      regulations of the Nuclear Regulatory Commission.

      (14) Manufacture, transportation, or sale of weapons to persons
      authorized under subdivisions (1) through (13.5) of this
      subsection to possess those weapons.

      (a-5) Subsections 24-1(a)(4) and 24-1(a)(10) do not apply to or
      affect any person carrying a concealed pistol, revolver, or
      handgun and the person has been issued a currently valid license
      under the Firearm Concealed Carry Act at the time of the
      commission of the offense.

      (a-6) Subsections 24-1(a)(4) and 24-1(a)(10) do not apply to or
      affect a qualified current or retired law enforcement officer or a
      current or retired deputy, county correctional officer, or
      correctional officer of the Department of Corrections qualified
      under the laws of this State or under the federal Law Enforcement
      Officers Safety Act.

      (b) Subsections 24-1(a)(4) and 24-1(a)(10) and Section 24-1.6 do
      not apply to or affect any of the following:

      (1) Members of any club or organization organized for the purpose
      of practicing shooting at targets upon established target ranges,
      whether public or private, and patrons of such ranges, while such
      members or patrons are using their firearms on those target
      ranges.

      (2) Duly authorized military or civil organizations while
      parading, with the special permission of the Governor.

      (3) Hunters, trappers or fishermen with a license or permit while
      engaged in hunting, trapping or fishing.

      (4) Transportation of weapons that are broken down in a
      non-functioning state or are not immediately accessible.

      (5) Carrying or possessing any pistol, revolver, stun gun or taser
      or other firearm on the land or in the legal dwelling of another
      person as an invitee with that person's permission.

      (c) Subsection 24-1(a)(7) does not apply to or affect any of the
      following:

      (1) Peace officers while in performance of their official duties.

      (2) Wardens, superintendents and keepers of prisons,
      penitentiaries, jails and other institutions for the detention of
      persons accused or convicted of an offense.

      (3) Members of the Armed Services or Reserve Forces of the United
      States or the Illinois National Guard, while in the performance of
      their official duty.

      (4) Manufacture, transportation, or sale of machine guns to
      persons authorized under subdivisions (1) through (3) of this
      subsection to possess machine guns, if the machine guns are broken
      down in a non-functioning state or are not immediately accessible.

      (5) Persons licensed under federal law to manufacture any weapon
      from which 8 or more shots or bullets can be discharged by a
      single function of the firing device, or ammunition for such
      weapons, and actually engaged in the business of manufacturing
      such weapons or ammunition, but only with respect to activities
      which are within the lawful scope of such business, such as the
      manufacture, transportation, or testing of such weapons or
      ammunition. This exemption does not authorize the general private
      possession of any weapon from which 8 or more shots or bullets can
      be discharged by a single function of the firing device, but only
      such possession and activities as are within the lawful scope of a
      licensed manufacturing business described in this paragraph.

      During transportation, such weapons shall be broken down in a
      non-functioning state or not immediately accessible.

      (6) The manufacture, transport, testing, delivery, transfer or
      sale, and all lawful commercial or experimental activities
      necessary thereto, of rifles, shotguns, and weapons made from
      rifles or shotguns, or ammunition for such rifles, shotguns or
      weapons, where engaged in by a person operating as a contractor or
      subcontractor pursuant to a contract or subcontract for the
      development and supply of such rifles, shotguns, weapons or
      ammunition to the United States government or any branch of the
      Armed Forces of the United States, when such activities are
      necessary and incident to fulfilling the terms of such contract.

      The exemption granted under this subdivision (c)(6) shall also
      apply to any authorized agent of any such contractor or
      subcontractor who is operating within the scope of his employment,
      where such activities involving such weapon, weapons or ammunition
      are necessary and incident to fulfilling the terms of such
      contract.

      (7) A person possessing a rifle with a barrel or barrels less than
      16 inches in length if: (A) the person has been issued a Curios
      and Relics license from the U.S. Bureau of Alcohol, Tobacco,
      Firearms and Explosives; or (B) the person is an active member of
      a bona fide, nationally recognized military re-enacting group and
      the modification is required and necessary to accurately portray
      the weapon for historical re-enactment purposes; the re-enactor is
      in possession of a valid and current re-enacting group membership
      credential; and the overall length of the weapon as modified is
      not less than 26 inches.

      (d) Subsection 24-1(a)(1) does not apply to the purchase,
      possession or carrying of a black-jack or slung-shot by a peace
      officer.

      (e) Subsection 24-1(a)(8) does not apply to any owner, manager or
      authorized employee of any place specified in that subsection nor
      to any law enforcement officer.

      (f) Subsection 24-1(a)(4) and subsection 24-1(a)(10) and Section
      24-1.6 do not apply to members of any club or organization
      organized for the purpose of practicing shooting at targets upon
      established target ranges, whether public or private, while using
      their firearms on those target ranges.

      (g) Subsections 24-1(a)(11) and 24-3.1(a)(6) do not apply to:

      (1) Members of the Armed Services or Reserve Forces of the United
      States or the Illinois National Guard, while in the performance of
      their official duty.

      (2) Bonafide collectors of antique or surplus military ordnance.

      (3) Laboratories having a department of forensic ballistics, or
      specializing in the development of ammunition or explosive
      ordnance.

      (4) Commerce, preparation, assembly or possession of explosive
      bullets by manufacturers of ammunition licensed by the federal
      government, in connection with the supply of those organizations
      and persons exempted by subdivision (g)(1) of this Section, or
      like organizations and persons outside this State, or the
      transportation of explosive bullets to any organization or person
      exempted in this Section by a common carrier or by a vehicle owned
      or leased by an exempted manufacturer.

      (g-5) Subsection 24-1(a)(6) does not apply to or affect persons
      licensed under federal law to manufacture any device or attachment
      of any kind designed, used, or intended for use in silencing the
      report of any firearm, firearms, or ammunition for those firearms
      equipped with those devices, and actually engaged in the business
      of manufacturing those devices, firearms, or ammunition, but only
      with respect to activities that are within the lawful scope of
      that business, such as the manufacture, transportation, or testing
      of those devices, firearms, or ammunition. This exemption does not
      authorize the general private possession of any device or
      attachment of any kind designed, used, or intended for use in
      silencing the report of any firearm, but only such possession and
      activities as are within the lawful scope of a licensed
      manufacturing business described in this subsection (g-5). During
      transportation, these devices shall be detached from any weapon or
      not immediately accessible.

      (g-6) Subsections 24-1(a)(4) and 24-1(a)(10) and Section 24-1.6 do
      not apply to or affect any parole agent or parole supervisor who
      meets the qualifications and conditions prescribed in Section
      3-14-1.5 of the Unified Code of Corrections.

      (g-7) Subsection 24-1(a)(6) does not apply to a peace officer
      while serving as a member of a tactical response team or special
      operations team. A peace officer may not personally own or apply
      for ownership of a device or attachment of any kind designed,
      used, or intended for use in silencing the report of any firearm.
      These devices shall be owned and maintained by lawfully recognized
      units of government whose duties include the investigation of
      criminal acts.

      (g-10) (Blank).

      (h) An information or indictment based upon a violation of any
      subsection of this Article need not negative any exemptions
      contained in this Article. The defendant shall have the burden of
      proving such an exemption.

      (i) Nothing in this Article shall prohibit, apply to, or affect
      the transportation, carrying, or possession, of any pistol or
      revolver, stun gun, taser, or other firearm consigned to a common
      carrier operating under license of the State of Illinois or the
      federal government, where such transportation, carrying, or
      possession is incident to the lawful transportation in which such
      common carrier is engaged; and nothing in this Article shall
      prohibit, apply to, or affect the transportation, carrying, or
      possession of any pistol, revolver, stun gun, taser, or other
      firearm, not the subject of and regulated by subsection 24-1(a)(7)
      or subsection 24-2(c) of this Article, which is unloaded and
      enclosed in a case, firearm carrying box, shipping box, or other
      container, by the possessor of a valid Firearm Owners
      Identification Card.

      (Source: P.A. 101-80, eff. 7-12-19; 102-152, eff. 1-1-22; 102-779,
      eff. 1-1-23.)

      \hypertarget{text-of-section-from-p.a.-102-837}{%
      \subsection*{(Text of Section from P.A.
      102-837)}\label{text-of-section-from-p.a.-102-837}}
      \addcontentsline{toc}{subsection}{(Text of Section from P.A.
      102-837)}

      \hypertarget{sec.-24-2.-exemptions.-1}{%
      \section*{Sec. 24-2. Exemptions.}\label{sec.-24-2.-exemptions.-1}}
      \addcontentsline{toc}{section}{Sec. 24-2. Exemptions.}

      \markright{Sec. 24-2. Exemptions.}

      (a) Subsections 24-1(a)(3), 24-1(a)(4), 24-1(a)(10), and
      24-1(a)(13) and Section 24-1.6 do not apply to or affect any of
      the following:

      (1) Peace officers, and any person summoned by a peace officer to
      assist in making arrests or preserving the peace, while actually
      engaged in assisting such officer.

      (2) Wardens, superintendents and keepers of prisons,
      penitentiaries, jails and other institutions for the detention of
      persons accused or convicted of an offense, while in the
      performance of their official duty, or while commuting between
      their homes and places of employment.

      (3) Members of the Armed Services or Reserve Forces of the United
      States or the Illinois National Guard or the Reserve Officers
      Training Corps, while in the performance of their official duty.

      (4) Special agents employed by a railroad or a public utility to
      perform police functions, and guards of armored car companies,
      while actually engaged in the performance of the duties of their
      employment or commuting between their homes and places of
      employment; and watchmen while actually engaged in the performance
      of the duties of their employment.

      (5) Persons licensed as private security contractors, private
      detectives, or private alarm contractors, or employed by a private
      security contractor, private detective, or private alarm
      contractor agency licensed by the Department of Financial and
      Professional Regulation, if their duties include the carrying of a
      weapon under the provisions of the Private Detective, Private
      Alarm, Private Security, Fingerprint Vendor, and Locksmith Act of
      2004, while actually engaged in the performance of the duties of
      their employment or commuting between their homes and places of
      employment. A person shall be considered eligible for this
      exemption if he or she has completed the required 20 hours of
      training for a private security contractor, private detective, or
      private alarm contractor, or employee of a licensed private
      security contractor, private detective, or private alarm
      contractor agency and 28 hours of required firearm training, and
      has been issued a firearm control card by the Department of
      Financial and Professional Regulation. Conditions for the renewal
      of firearm control cards issued under the provisions of this
      Section shall be the same as for those cards issued under the
      provisions of the Private Detective, Private Alarm, Private
      Security, Fingerprint Vendor, and Locksmith Act of 2004. The
      firearm control card shall be carried by the private security
      contractor, private detective, or private alarm contractor, or
      employee of the licensed private security contractor, private
      detective, or private alarm contractor agency at all times when he
      or she is in possession of a concealable weapon permitted by his
      or her firearm control card.

      (6) Any person regularly employed in a commercial or industrial
      operation as a security guard for the protection of persons
      employed and private property related to such commercial or
      industrial operation, while actually engaged in the performance of
      his or her duty or traveling between sites or properties belonging
      to the employer, and who, as a security guard, is a member of a
      security force registered with the Department of Financial and
      Professional Regulation; provided that such security guard has
      successfully completed a course of study, approved by and
      supervised by the Department of Financial and Professional
      Regulation, consisting of not less than 48 hours of training that
      includes the theory of law enforcement, liability for acts, and
      the handling of weapons. A person shall be considered eligible for
      this exemption if he or she has completed the required 20 hours of
      training for a security officer and 28 hours of required firearm
      training, and has been issued a firearm control card by the
      Department of Financial and Professional Regulation. Conditions
      for the renewal of firearm control cards issued under the
      provisions of this Section shall be the same as for those cards
      issued under the provisions of the Private Detective, Private
      Alarm, Private Security, Fingerprint Vendor, and Locksmith Act of
      2004. The firearm control card shall be carried by the security
      guard at all times when he or she is in possession of a
      concealable weapon permitted by his or her firearm control card.

      (7) Agents and investigators of the Illinois

      Legislative Investigating Commission authorized by the Commission
      to carry the weapons specified in subsections 24-1(a)(3) and
      24-1(a)(4), while on duty in the course of any investigation for
      the Commission.

      (8) Persons employed by a financial institution as a security
      guard for the protection of other employees and property related
      to such financial institution, while actually engaged in the
      performance of their duties, commuting between their homes and
      places of employment, or traveling between sites or properties
      owned or operated by such financial institution, and who, as a
      security guard, is a member of a security force registered with
      the Department; provided that any person so employed has
      successfully completed a course of study, approved by and
      supervised by the Department of Financial and Professional
      Regulation, consisting of not less than 48 hours of training which
      includes theory of law enforcement, liability for acts, and the
      handling of weapons. A person shall be considered to be eligible
      for this exemption if he or she has completed the required 20
      hours of training for a security officer and 28 hours of required
      firearm training, and has been issued a firearm control card by
      the Department of Financial and Professional Regulation.
      Conditions for renewal of firearm control cards issued under the
      provisions of this Section shall be the same as for those issued
      under the provisions of the Private Detective, Private Alarm,
      Private Security, Fingerprint Vendor, and Locksmith Act of 2004.
      The firearm control card shall be carried by the security guard at
      all times when he or she is in possession of a concealable weapon
      permitted by his or her firearm control card. For purposes of this
      subsection, ``financial institution'' means a bank, savings and
      loan association, credit union or company providing armored car
      services.

      (9) Any person employed by an armored car company to drive an
      armored car, while actually engaged in the performance of his
      duties.

      (10) Persons who have been classified as peace officers pursuant
      to the Peace Officer Fire Investigation Act.

      (11) Investigators of the Office of the State's

      Attorneys Appellate Prosecutor authorized by the board of
      governors of the Office of the State's Attorneys Appellate
      Prosecutor to carry weapons pursuant to Section 7.06 of the
      State's Attorneys Appellate Prosecutor's Act.

      (12) Special investigators appointed by a State's

      Attorney under Section 3-9005 of the Counties Code.

      (12.5) Probation officers while in the performance of their
      duties, or while commuting between their homes, places of
      employment or specific locations that are part of their assigned
      duties, with the consent of the chief judge of the circuit for
      which they are employed, if they have received weapons training
      according to requirements of the Peace Officer and Probation
      Officer Firearm Training Act.

      (13) Court Security Officers while in the performance of their
      official duties, or while commuting between their homes and places
      of employment, with the consent of the Sheriff.

      (13.5) A person employed as an armed security guard at a nuclear
      energy, storage, weapons or development site or facility regulated
      by the Nuclear Regulatory Commission who has completed the
      background screening and training mandated by the rules and
      regulations of the Nuclear Regulatory Commission.

      (14) Manufacture, transportation, or sale of weapons to persons
      authorized under subdivisions (1) through (13.5) of this
      subsection to possess those weapons.

      (a-5) Subsections 24-1(a)(4) and 24-1(a)(10) do not apply to or
      affect any person carrying a concealed pistol, revolver, or
      handgun and the person has been issued a currently valid license
      under the Firearm Concealed Carry Act at the time of the
      commission of the offense.

      (a-6) Subsections 24-1(a)(4) and 24-1(a)(10) do not apply to or
      affect a qualified current or retired law enforcement officer
      qualified under the laws of this State or under the federal Law
      Enforcement Officers Safety Act.

      (b) Subsections 24-1(a)(4) and 24-1(a)(10) and Section 24-1.6 do
      not apply to or affect any of the following:

      (1) Members of any club or organization organized for the purpose
      of practicing shooting at targets upon established target ranges,
      whether public or private, and patrons of such ranges, while such
      members or patrons are using their firearms on those target
      ranges.

      (2) Duly authorized military or civil organizations while
      parading, with the special permission of the Governor.

      (3) Hunters, trappers, or fishermen while engaged in lawful
      hunting, trapping, or fishing under the provisions of the Wildlife
      Code or the Fish and Aquatic Life Code.

      (4) Transportation of weapons that are broken down in a
      non-functioning state or are not immediately accessible.

      (5) Carrying or possessing any pistol, revolver, stun gun or taser
      or other firearm on the land or in the legal dwelling of another
      person as an invitee with that person's permission.

      (c) Subsection 24-1(a)(7) does not apply to or affect any of the
      following:

      (1) Peace officers while in performance of their official duties.

      (2) Wardens, superintendents and keepers of prisons,
      penitentiaries, jails and other institutions for the detention of
      persons accused or convicted of an offense.

      (3) Members of the Armed Services or Reserve Forces of the United
      States or the Illinois National Guard, while in the performance of
      their official duty.

      (4) Manufacture, transportation, or sale of machine guns to
      persons authorized under subdivisions (1) through (3) of this
      subsection to possess machine guns, if the machine guns are broken
      down in a non-functioning state or are not immediately accessible.

      (5) Persons licensed under federal law to manufacture any weapon
      from which 8 or more shots or bullets can be discharged by a
      single function of the firing device, or ammunition for such
      weapons, and actually engaged in the business of manufacturing
      such weapons or ammunition, but only with respect to activities
      which are within the lawful scope of such business, such as the
      manufacture, transportation, or testing of such weapons or
      ammunition. This exemption does not authorize the general private
      possession of any weapon from which 8 or more shots or bullets can
      be discharged by a single function of the firing device, but only
      such possession and activities as are within the lawful scope of a
      licensed manufacturing business described in this paragraph.

      During transportation, such weapons shall be broken down in a
      non-functioning state or not immediately accessible.

      (6) The manufacture, transport, testing, delivery, transfer or
      sale, and all lawful commercial or experimental activities
      necessary thereto, of rifles, shotguns, and weapons made from
      rifles or shotguns, or ammunition for such rifles, shotguns or
      weapons, where engaged in by a person operating as a contractor or
      subcontractor pursuant to a contract or subcontract for the
      development and supply of such rifles, shotguns, weapons or
      ammunition to the United States government or any branch of the
      Armed Forces of the United States, when such activities are
      necessary and incident to fulfilling the terms of such contract.

      The exemption granted under this subdivision (c)(6) shall also
      apply to any authorized agent of any such contractor or
      subcontractor who is operating within the scope of his employment,
      where such activities involving such weapon, weapons or ammunition
      are necessary and incident to fulfilling the terms of such
      contract.

      (7) A person possessing a rifle with a barrel or barrels less than
      16 inches in length if: (A) the person has been issued a Curios
      and Relics license from the U.S. Bureau of Alcohol, Tobacco,
      Firearms and Explosives; or (B) the person is an active member of
      a bona fide, nationally recognized military re-enacting group and
      the modification is required and necessary to accurately portray
      the weapon for historical re-enactment purposes; the re-enactor is
      in possession of a valid and current re-enacting group membership
      credential; and the overall length of the weapon as modified is
      not less than 26 inches.

      (d) Subsection 24-1(a)(1) does not apply to the purchase,
      possession or carrying of a black-jack or slung-shot by a peace
      officer.

      (e) Subsection 24-1(a)(8) does not apply to any owner, manager or
      authorized employee of any place specified in that subsection nor
      to any law enforcement officer.

      (f) Subsection 24-1(a)(4) and subsection 24-1(a)(10) and Section
      24-1.6 do not apply to members of any club or organization
      organized for the purpose of practicing shooting at targets upon
      established target ranges, whether public or private, while using
      their firearms on those target ranges.

      (g) Subsections 24-1(a)(11) and 24-3.1(a)(6) do not apply to:

      (1) Members of the Armed Services or Reserve Forces of the United
      States or the Illinois National Guard, while in the performance of
      their official duty.

      (2) Bonafide collectors of antique or surplus military ordnance.

      (3) Laboratories having a department of forensic ballistics, or
      specializing in the development of ammunition or explosive
      ordnance.

      (4) Commerce, preparation, assembly or possession of explosive
      bullets by manufacturers of ammunition licensed by the federal
      government, in connection with the supply of those organizations
      and persons exempted by subdivision (g)(1) of this Section, or
      like organizations and persons outside this State, or the
      transportation of explosive bullets to any organization or person
      exempted in this Section by a common carrier or by a vehicle owned
      or leased by an exempted manufacturer.

      (g-5) Subsection 24-1(a)(6) does not apply to or affect persons
      licensed under federal law to manufacture any device or attachment
      of any kind designed, used, or intended for use in silencing the
      report of any firearm, firearms, or ammunition for those firearms
      equipped with those devices, and actually engaged in the business
      of manufacturing those devices, firearms, or ammunition, but only
      with respect to activities that are within the lawful scope of
      that business, such as the manufacture, transportation, or testing
      of those devices, firearms, or ammunition. This exemption does not
      authorize the general private possession of any device or
      attachment of any kind designed, used, or intended for use in
      silencing the report of any firearm, but only such possession and
      activities as are within the lawful scope of a licensed
      manufacturing business described in this subsection (g-5). During
      transportation, these devices shall be detached from any weapon or
      not immediately accessible.

      (g-6) Subsections 24-1(a)(4) and 24-1(a)(10) and Section 24-1.6 do
      not apply to or affect any parole agent or parole supervisor who
      meets the qualifications and conditions prescribed in Section
      3-14-1.5 of the Unified Code of Corrections.

      (g-7) Subsection 24-1(a)(6) does not apply to a peace officer
      while serving as a member of a tactical response team or special
      operations team. A peace officer may not personally own or apply
      for ownership of a device or attachment of any kind designed,
      used, or intended for use in silencing the report of any firearm.
      These devices shall be owned and maintained by lawfully recognized
      units of government whose duties include the investigation of
      criminal acts.

      (g-10) (Blank).

      (h) An information or indictment based upon a violation of any
      subsection of this Article need not negative any exemptions
      contained in this Article. The defendant shall have the burden of
      proving such an exemption.

      (i) Nothing in this Article shall prohibit, apply to, or affect
      the transportation, carrying, or possession, of any pistol or
      revolver, stun gun, taser, or other firearm consigned to a common
      carrier operating under license of the State of Illinois or the
      federal government, where such transportation, carrying, or
      possession is incident to the lawful transportation in which such
      common carrier is engaged; and nothing in this Article shall
      prohibit, apply to, or affect the transportation, carrying, or
      possession of any pistol, revolver, stun gun, taser, or other
      firearm, not the subject of and regulated by subsection 24-1(a)(7)
      or subsection 24-2(c) of this Article, which is unloaded and
      enclosed in a case, firearm carrying box, shipping box, or other
      container, by the possessor of a valid Firearm Owners
      Identification Card.

      (Source: P.A. 101-80, eff. 7-12-19; 102-152, eff. 1-1-22; 102-837,
      eff. 5-13-22.)

      \hypertarget{ilcs-524-2.1-from-ch.-38-par.-24-2.1}{%
      \subsection*{(720 ILCS 5/24-2.1) (from Ch. 38, par.
      24-2.1)}\label{ilcs-524-2.1-from-ch.-38-par.-24-2.1}}
      \addcontentsline{toc}{subsection}{(720 ILCS 5/24-2.1) (from Ch.
      38, par. 24-2.1)}

      \hypertarget{sec.-24-2.1.-unlawful-use-of-firearm-projectiles.}{%
      \section*{Sec. 24-2.1. Unlawful use of firearm
      projectiles.}\label{sec.-24-2.1.-unlawful-use-of-firearm-projectiles.}}
      \addcontentsline{toc}{section}{Sec. 24-2.1. Unlawful use of
      firearm projectiles.}

      \markright{Sec. 24-2.1. Unlawful use of firearm projectiles.}

      (a) A person commits the offense of unlawful use of firearm
      projectiles when he or she knowingly manufactures, sells,
      purchases, possesses, or carries any armor piercing bullet,
      dragon's breath shotgun shell, bolo shell, or flechette shell.

      For the purposes of this Section:

      ``Armor piercing bullet'' means any handgun bullet or handgun
      ammunition with projectiles or projectile cores constructed
      entirely (excluding the presence of traces of other substances)
      from tungsten alloys, steel, iron, brass, bronze, beryllium copper
      or depleted uranium, or fully jacketed bullets larger than 22
      caliber designed and intended for use in a handgun and whose
      jacket has a weight of more than 25\% of the total weight of the
      projectile, and excluding those handgun projectiles whose cores
      are composed of soft materials such as lead or lead alloys, zinc
      or zinc alloys, frangible projectiles designed primarily for
      sporting purposes, and any other projectiles or projectile cores
      that the U. S. Secretary of the Treasury finds to be primarily
      intended to be used for sporting purposes or industrial purposes
      or that otherwise does not constitute ``armor piercing
      ammunition'' as that term is defined by federal law.

      The definition contained herein shall not be construed to include
      shotgun shells.

      ``Dragon's breath shotgun shell'' means any shotgun shell that
      contains exothermic pyrophoric mesh metal as the projectile and is
      designed for the purpose of throwing or spewing a flame or
      fireball to simulate a flame-thrower.

      ``Bolo shell'' means any shell that can be fired in a firearm and
      expels as projectiles 2 or more metal balls connected by solid
      metal wire.

      ``Flechette shell'' means any shell that can be fired in a firearm
      and expels 2 or more pieces of fin-stabilized solid metal wire or
      2 or more solid dart-type projectiles.

      (b) Exemptions. This Section does not apply to or affect any of
      the following:

      (1) Peace officers.

      (2) Wardens, superintendents and keepers of prisons,
      penitentiaries, jails and other institutions for the detention of
      persons accused or convicted of an offense.

      (3) Members of the Armed Services or Reserve Forces of the United
      States or the Illinois National Guard while in the performance of
      their official duties.

      (4) Federal officials required to carry firearms, while engaged in
      the performance of their official duties.

      (5) United States Marshals, while engaged in the performance of
      their official duties.

      (6) Persons licensed under federal law to manufacture, import, or
      sell firearms and firearm ammunition, and actually engaged in any
      such business, but only with respect to activities which are
      within the lawful scope of such business, such as the manufacture,
      transportation, or testing of such bullets or ammunition.

      This exemption does not authorize the general private possession
      of any armor piercing bullet, dragon's breath shotgun shell, bolo
      shell, or flechette shell, but only such possession and activities
      which are within the lawful scope of a licensed business described
      in this paragraph.

      (7) Laboratories having a department of forensic ballistics or
      specializing in the development of ammunition or explosive
      ordnance.

      (8) Manufacture, transportation, or sale of armor piercing
      bullets, dragon's breath shotgun shells, bolo shells, or flechette
      shells to persons specifically authorized under paragraphs (1)
      through (7) of this subsection to possess such bullets or shells.

      (c) An information or indictment based upon a violation of this
      Section need not negate any exemption herein contained. The
      defendant shall have the burden of proving such an exemption.

      (d) Sentence. A person convicted of unlawful use of armor piercing
      bullets shall be guilty of a Class 3 felony.

      (Source: P.A. 92-423, eff. 1-1-02.)

      \hypertarget{ilcs-524-2.2-from-ch.-38-par.-24-2.2}{%
      \subsection*{(720 ILCS 5/24-2.2) (from Ch. 38, par.
      24-2.2)}\label{ilcs-524-2.2-from-ch.-38-par.-24-2.2}}
      \addcontentsline{toc}{subsection}{(720 ILCS 5/24-2.2) (from Ch.
      38, par. 24-2.2)}

      \hypertarget{sec.-24-2.2.-manufacture-sale-or-transfer-of-bullets-or-shells-represented-to-be-armor-piercing-bullets-dragons-breath-shotgun-shells-bolo-shells-or-flechette-shells.}{%
      \section*{Sec. 24-2.2. Manufacture, sale or transfer of bullets or
      shells represented to be armor piercing bullets, dragon's breath
      shotgun shells, bolo shells, or flechette
      shells.}\label{sec.-24-2.2.-manufacture-sale-or-transfer-of-bullets-or-shells-represented-to-be-armor-piercing-bullets-dragons-breath-shotgun-shells-bolo-shells-or-flechette-shells.}}
      \addcontentsline{toc}{section}{Sec. 24-2.2. Manufacture, sale or
      transfer of bullets or shells represented to be armor piercing
      bullets, dragon's breath shotgun shells, bolo shells, or flechette
      shells.}

      \markright{Sec. 24-2.2. Manufacture, sale or transfer of bullets
      or shells represented to be armor piercing bullets, dragon's
      breath shotgun shells, bolo shells, or flechette shells.}

      (a) Except as provided in subsection (b) of this Section, it is
      unlawful for any person to knowingly manufacture, sell, offer to
      sell, or transfer any bullet or shell which is represented to be
      an armor piercing bullet, a dragon's breath shotgun shell, a bolo
      shell, or a flechette shell as defined in Section 24-2.1 of this
      Code.

      (b) Exemptions. This Section does not apply to or affect any
      person authorized under Section 24-2.1 to manufacture, sell,
      purchase, possess, or carry any armor piercing bullet or any
      dragon's breath shotgun shell, bolo shell, or flechette shell with
      respect to activities which are within the lawful scope of the
      exemption therein granted.

      (c) An information or indictment based upon a violation of this
      Section need not negate any exemption herein contained. The
      defendant shall have the burden of proving such an exemption and
      that the activities forming the basis of any criminal charge
      brought pursuant to this Section were within the lawful scope of
      such exemption.

      (d) Sentence. A violation of this Section is a Class 4 felony.

      (Source: P.A. 92-423, eff. 1-1-02 .)

      \hypertarget{ilcs-524-3-from-ch.-38-par.-24-3}{%
      \subsection*{(720 ILCS 5/24-3) (from Ch. 38, par.
      24-3)}\label{ilcs-524-3-from-ch.-38-par.-24-3}}
      \addcontentsline{toc}{subsection}{(720 ILCS 5/24-3) (from Ch. 38,
      par. 24-3)}

      \hypertarget{sec.-24-3.-unlawful-sale-or-delivery-of-firearms.}{%
      \section*{Sec. 24-3. Unlawful sale or delivery of
      firearms.}\label{sec.-24-3.-unlawful-sale-or-delivery-of-firearms.}}
      \addcontentsline{toc}{section}{Sec. 24-3. Unlawful sale or
      delivery of firearms.}

      \markright{Sec. 24-3. Unlawful sale or delivery of firearms.}

      (A) A person commits the offense of unlawful sale or delivery of
      firearms when he or she knowingly does any of the following:

      (a) Sells or gives any firearm of a size which may be concealed
      upon the person to any person under 18 years of age.

      (b) Sells or gives any firearm to a person under 21 years of age
      who has been convicted of a misdemeanor other than a traffic
      offense or adjudged delinquent.

      (c) Sells or gives any firearm to any narcotic addict.

      (d) Sells or gives any firearm to any person who has been
      convicted of a felony under the laws of this or any other
      jurisdiction.

      (e) Sells or gives any firearm to any person who has been a
      patient in a mental institution within the past 5 years. In this
      subsection (e):

      ``Mental institution'' means any hospital, institution, clinic,
      evaluation facility, mental health center, or part thereof, which
      is used primarily for the care or treatment of persons with mental
      illness.

      ``Patient in a mental institution'' means the person was admitted,
      either voluntarily or involuntarily, to a mental institution for
      mental health treatment, unless the treatment was voluntary and
      solely for an alcohol abuse disorder and no other secondary
      substance abuse disorder or mental illness.

      (f) Sells or gives any firearms to any person who is a person with
      an intellectual disability.

      (g) Delivers any firearm, incidental to a sale, without
      withholding delivery of the firearm for at least 72 hours after
      application for its purchase has been made, or delivers a stun gun
      or taser, incidental to a sale, without withholding delivery of
      the stun gun or taser for at least 24 hours after application for
      its purchase has been made. However, this paragraph (g) does not
      apply to: (1) the sale of a firearm to a law enforcement officer
      if the seller of the firearm knows that the person to whom he or
      she is selling the firearm is a law enforcement officer or the
      sale of a firearm to a person who desires to purchase a firearm
      for use in promoting the public interest incident to his or her
      employment as a bank guard, armed truck guard, or other similar
      employment; (2) a mail order sale of a firearm from a federally
      licensed firearms dealer to a nonresident of Illinois under which
      the firearm is mailed to a federally licensed firearms dealer
      outside the boundaries of Illinois; (3) (blank); (4) the sale of a
      firearm to a dealer licensed as a federal firearms dealer under
      Section 923 of the federal Gun Control Act of 1968 (18 U.S.C.
      923); or (5) the transfer or sale of any rifle, shotgun, or other
      long gun to a resident registered competitor or attendee or
      non-resident registered competitor or attendee by any dealer
      licensed as a federal firearms dealer under Section 923 of the
      federal Gun Control Act of 1968 at competitive shooting events
      held at the World Shooting Complex sanctioned by a national
      governing body. For purposes of transfers or sales under
      subparagraph (5) of this paragraph (g), the Department of Natural
      Resources shall give notice to the Illinois State Police at least
      30 calendar days prior to any competitive shooting events at the
      World Shooting Complex sanctioned by a national governing body.
      The notification shall be made on a form prescribed by the
      Illinois State Police. The sanctioning body shall provide a list
      of all registered competitors and attendees at least 24 hours
      before the events to the Illinois State Police. Any changes to the
      list of registered competitors and attendees shall be forwarded to
      the Illinois State Police as soon as practicable. The Illinois
      State Police must destroy the list of registered competitors and
      attendees no later than 30 days after the date of the event.
      Nothing in this paragraph (g) relieves a federally licensed
      firearm dealer from the requirements of conducting a NICS
      background check through the Illinois Point of Contact under 18
      U.S.C. 922(t). For purposes of this paragraph (g), ``application''
      means when the buyer and seller reach an agreement to purchase a
      firearm. For purposes of this paragraph (g), ``national governing
      body'' means a group of persons who adopt rules and formulate
      policy on behalf of a national firearm sporting organization.

      (h) While holding any license as a dealer, importer, manufacturer
      or pawnbroker under the federal Gun Control Act of 1968,
      manufactures, sells or delivers to any unlicensed person a handgun
      having a barrel, slide, frame or receiver which is a die casting
      of zinc alloy or any other nonhomogeneous metal which will melt or
      deform at a temperature of less than 800 degrees Fahrenheit. For
      purposes of this paragraph, (1) ``firearm'' is defined as in the
      Firearm Owners Identification Card Act; and (2) ``handgun'' is
      defined as a firearm designed to be held and fired by the use of a
      single hand, and includes a combination of parts from which such a
      firearm can be assembled.

      (i) Sells or gives a firearm of any size to any person under 18
      years of age who does not possess a valid Firearm Owner's
      Identification Card.

      (j) Sells or gives a firearm while engaged in the business of
      selling firearms at wholesale or retail without being licensed as
      a federal firearms dealer under Section 923 of the federal Gun
      Control Act of 1968 (18 U.S.C. 923). In this paragraph (j):

      A person ``engaged in the business'' means a person who devotes
      time, attention, and labor to engaging in the activity as a
      regular course of trade or business with the principal objective
      of livelihood and profit, but does not include a person who makes
      occasional repairs of firearms or who occasionally fits special
      barrels, stocks, or trigger mechanisms to firearms.

      ``With the principal objective of livelihood and profit'' means
      that the intent underlying the sale or disposition of firearms is
      predominantly one of obtaining livelihood and pecuniary gain, as
      opposed to other intents, such as improving or liquidating a
      personal firearms collection; however, proof of profit shall not
      be required as to a person who engages in the regular and
      repetitive purchase and disposition of firearms for criminal
      purposes or terrorism.

      (k) Sells or transfers ownership of a firearm to a person who does
      not display to the seller or transferor of the firearm either: (1)
      a currently valid Firearm Owner's Identification Card that has
      previously been issued in the transferee's name by the Illinois
      State Police under the provisions of the Firearm Owners
      Identification Card Act; or (2) a currently valid license to carry
      a concealed firearm that has previously been issued in the
      transferee's name by the Illinois State Police under the Firearm
      Concealed Carry Act. This paragraph (k) does not apply to the
      transfer of a firearm to a person who is exempt from the
      requirement of possessing a Firearm Owner's Identification Card
      under Section 2 of the Firearm Owners Identification Card Act. For
      the purposes of this Section, a currently valid Firearm Owner's
      Identification Card or license to carry a concealed firearm means
      receipt of an approval number issued in accordance with subsection
      (a-10) of Section 3 or Section 3.1 of the Firearm Owners
      Identification Card Act.

      (1) In addition to the other requirements of this paragraph (k),
      all persons who are not federally licensed firearms dealers must
      also have complied with subsection (a-10) of Section 3 of the
      Firearm Owners Identification Card Act by determining the validity
      of a purchaser's Firearm Owner's Identification Card.

      (2) All sellers or transferors who have complied with the
      requirements of subparagraph (1) of this paragraph (k) shall not
      be liable for damages in any civil action arising from the use or
      misuse by the transferee of the firearm transferred, except for
      willful or wanton misconduct on the part of the seller or
      transferor.

      (l) Not being entitled to the possession of a firearm, delivers
      the firearm, knowing it to have been stolen or converted. It may
      be inferred that a person who possesses a firearm with knowledge
      that its serial number has been removed or altered has knowledge
      that the firearm is stolen or converted.

      (B) Paragraph (h) of subsection (A) does not include firearms sold
      within 6 months after enactment of Public Act 78-355 (approved
      August 21, 1973, effective October 1, 1973), nor is any firearm
      legally owned or possessed by any citizen or purchased by any
      citizen within 6 months after the enactment of Public Act 78-355
      subject to confiscation or seizure under the provisions of that
      Public Act. Nothing in Public Act 78-355 shall be construed to
      prohibit the gift or trade of any firearm if that firearm was
      legally held or acquired within 6 months after the enactment of
      that Public Act.

      (C) Sentence.

      (1) Any person convicted of unlawful sale or delivery of firearms
      in violation of paragraph (c), (e), (f), (g), or (h) of subsection
      (A) commits a Class 4 felony.

      (2) Any person convicted of unlawful sale or delivery of firearms
      in violation of paragraph (b) or (i) of subsection (A) commits a
      Class 3 felony.

      (3) Any person convicted of unlawful sale or delivery of firearms
      in violation of paragraph (a) of subsection (A) commits a Class 2
      felony.

      (4) Any person convicted of unlawful sale or delivery of firearms
      in violation of paragraph (a), (b), or (i) of subsection (A) in
      any school, on the real property comprising a school, within 1,000
      feet of the real property comprising a school, at a school related
      activity, or on or within 1,000 feet of any conveyance owned,
      leased, or contracted by a school or school district to transport
      students to or from school or a school related activity,
      regardless of the time of day or time of year at which the offense
      was committed, commits a Class 1 felony. Any person convicted of a
      second or subsequent violation of unlawful sale or delivery of
      firearms in violation of paragraph (a), (b), or (i) of subsection
      (A) in any school, on the real property comprising a school,
      within 1,000 feet of the real property comprising a school, at a
      school related activity, or on or within 1,000 feet of any
      conveyance owned, leased, or contracted by a school or school
      district to transport students to or from school or a school
      related activity, regardless of the time of day or time of year at
      which the offense was committed, commits a Class 1 felony for
      which the sentence shall be a term of imprisonment of no less than
      5 years and no more than 15 years.

      (5) Any person convicted of unlawful sale or delivery of firearms
      in violation of paragraph (a) or (i) of subsection (A) in
      residential property owned, operated, or managed by a public
      housing agency or leased by a public housing agency as part of a
      scattered site or mixed-income development, in a public park, in a
      courthouse, on residential property owned, operated, or managed by
      a public housing agency or leased by a public housing agency as
      part of a scattered site or mixed-income development, on the real
      property comprising any public park, on the real property
      comprising any courthouse, or on any public way within 1,000 feet
      of the real property comprising any public park, courthouse, or
      residential property owned, operated, or managed by a public
      housing agency or leased by a public housing agency as part of a
      scattered site or mixed-income development commits a Class 2
      felony.

      (6) Any person convicted of unlawful sale or delivery of firearms
      in violation of paragraph (j) of subsection (A) commits a Class A
      misdemeanor. A second or subsequent violation is a Class 4 felony.

      (7) Any person convicted of unlawful sale or delivery of firearms
      in violation of paragraph (k) of subsection (A) commits a Class 4
      felony, except that a violation of subparagraph (1) of paragraph
      (k) of subsection (A) shall not be punishable as a crime or petty
      offense. A third or subsequent conviction for a violation of
      paragraph (k) of subsection (A) is a Class 1 felony.

      (8) A person 18 years of age or older convicted of unlawful sale
      or delivery of firearms in violation of paragraph (a) or (i) of
      subsection (A), when the firearm that was sold or given to another
      person under 18 years of age was used in the commission of or
      attempt to commit a forcible felony, shall be fined or imprisoned,
      or both, not to exceed the maximum provided for the most serious
      forcible felony so committed or attempted by the person under 18
      years of age who was sold or given the firearm.

      (9) Any person convicted of unlawful sale or delivery of firearms
      in violation of paragraph (d) of subsection (A) commits a Class 3
      felony.

      (10) Any person convicted of unlawful sale or delivery of firearms
      in violation of paragraph (l) of subsection (A) commits a Class 2
      felony if the delivery is of one firearm. Any person convicted of
      unlawful sale or delivery of firearms in violation of paragraph
      (l) of subsection (A) commits a Class 1 felony if the delivery is
      of not less than 2 and not more than 5 firearms at the same time
      or within a one-year period. Any person convicted of unlawful sale
      or delivery of firearms in violation of paragraph (l) of
      subsection (A) commits a Class X felony for which he or she shall
      be sentenced to a term of imprisonment of not less than 6 years
      and not more than 30 years if the delivery is of not less than 6
      and not more than 10 firearms at the same time or within a 2-year
      period. Any person convicted of unlawful sale or delivery of
      firearms in violation of paragraph (l) of subsection (A) commits a
      Class X felony for which he or she shall be sentenced to a term of
      imprisonment of not less than 6 years and not more than 40 years
      if the delivery is of not less than 11 and not more than 20
      firearms at the same time or within a 3-year period. Any person
      convicted of unlawful sale or delivery of firearms in violation of
      paragraph (l) of subsection (A) commits a Class X felony for which
      he or she shall be sentenced to a term of imprisonment of not less
      than 6 years and not more than 50 years if the delivery is of not
      less than 21 and not more than 30 firearms at the same time or
      within a 4-year period. Any person convicted of unlawful sale or
      delivery of firearms in violation of paragraph (l) of subsection
      (A) commits a Class X felony for which he or she shall be
      sentenced to a term of imprisonment of not less than 6 years and
      not more than 60 years if the delivery is of 31 or more firearms
      at the same time or within a 5-year period.

      (D) For purposes of this Section:

      ``School'' means a public or private elementary or secondary
      school, community college, college, or university.

      ``School related activity'' means any sporting, social, academic,
      or other activity for which students' attendance or participation
      is sponsored, organized, or funded in whole or in part by a school
      or school district.

      (E) A prosecution for a violation of paragraph (k) of subsection
      (A) of this Section may be commenced within 6 years after the
      commission of the offense. A prosecution for a violation of this
      Section other than paragraph (g) of subsection (A) of this Section
      may be commenced within 5 years after the commission of the
      offense defined in the particular paragraph.

      (Source: P.A. 102-237, eff. 1-1-22; 102-538, eff. 8-20-21;
      102-813, eff. 5-13-22.)

      \hypertarget{ilcs-524-3a}{%
      \subsection*{(720 ILCS 5/24-3A)}\label{ilcs-524-3a}}
      \addcontentsline{toc}{subsection}{(720 ILCS 5/24-3A)}

      \hypertarget{sec.-24-3a.-gunrunning.}{%
      \section*{Sec. 24-3A. Gunrunning.}\label{sec.-24-3a.-gunrunning.}}
      \addcontentsline{toc}{section}{Sec. 24-3A. Gunrunning.}

      \markright{Sec. 24-3A. Gunrunning.}

      (a) A person commits gunrunning when he or she transfers 3 or more
      firearms in violation of any of the paragraphs of Section 24-3 of
      this Code.

      (b) Sentence. A person who commits gunrunning:

      (1) is guilty of a Class 1 felony;

      (2) is guilty of a Class X felony for which the sentence shall be
      a term of imprisonment of not less than 8 years and not more than
      40 years if the transfer is of not less than 11 firearms and not
      more than 20 firearms;

      (3) is guilty of a Class X felony for which the sentence shall be
      a term of imprisonment of not less than 10 years and not more than
      50 years if the transfer is of more than 20 firearms.

      A person who commits gunrunning by transferring firearms to a
      person who, at the time of the commission of the offense, is under
      18 years of age is guilty of a Class X felony.

      (Source: P.A. 93-906, eff. 8-11-04.)

      \hypertarget{ilcs-524-3b}{%
      \subsection*{(720 ILCS 5/24-3B)}\label{ilcs-524-3b}}
      \addcontentsline{toc}{subsection}{(720 ILCS 5/24-3B)}

      \hypertarget{sec.-24-3b.-firearms-trafficking.}{%
      \section*{Sec. 24-3B. Firearms
      trafficking.}\label{sec.-24-3b.-firearms-trafficking.}}
      \addcontentsline{toc}{section}{Sec. 24-3B. Firearms trafficking.}

      \markright{Sec. 24-3B. Firearms trafficking.}

      (a) A person commits firearms trafficking when he or she has not
      been issued a currently valid Firearm Owner's Identification Card
      and knowingly:

      (1) brings, or causes to be brought, into this State, a firearm or
      firearm ammunition for the purpose of sale, delivery, or transfer
      to any other person or with the intent to sell, deliver, or
      transfer the firearm or firearm ammunition to any other person; or

      (2) brings, or causes to be brought, into this State, a firearm
      and firearm ammunition for the purpose of sale, delivery, or
      transfer to any other person or with the intent to sell, deliver,
      or transfer the firearm and firearm ammunition to any other
      person.

      (a-5) This Section does not apply to:

      (1) a person exempt under Section 2 of the Firearm

      Owners Identification Card Act from the requirement of having
      possession of a Firearm Owner's Identification Card previously
      issued in his or her name by the Illinois State Police in order to
      acquire or possess a firearm or firearm ammunition;

      (2) a common carrier under subsection (i) of Section

      24-2 of this Code; or

      (3) a non-resident who may lawfully possess a firearm in his or
      her resident state.

      (b) Sentence.

      (1) Firearms trafficking is a Class 1 felony for which the person,
      if sentenced to a term of imprisonment, shall be sentenced to not
      less than 4 years and not more than 20 years.

      (2) Firearms trafficking by a person who has been previously
      convicted of firearms trafficking, gunrunning, or a felony offense
      for the unlawful sale, delivery, or transfer of a firearm or
      firearm ammunition in this State or another jurisdiction is a
      Class X felony.

      (Source: P.A. 102-538, eff. 8-20-21.)

      \hypertarget{ilcs-524-3.1-from-ch.-38-par.-24-3.1}{%
      \subsection*{(720 ILCS 5/24-3.1) (from Ch. 38, par.
      24-3.1)}\label{ilcs-524-3.1-from-ch.-38-par.-24-3.1}}
      \addcontentsline{toc}{subsection}{(720 ILCS 5/24-3.1) (from Ch.
      38, par. 24-3.1)}

      \hypertarget{sec.-24-3.1.-unlawful-possession-of-firearms-and-firearm-ammunition.}{%
      \section*{Sec. 24-3.1. Unlawful possession of firearms and firearm
      ammunition.}\label{sec.-24-3.1.-unlawful-possession-of-firearms-and-firearm-ammunition.}}
      \addcontentsline{toc}{section}{Sec. 24-3.1. Unlawful possession of
      firearms and firearm ammunition.}

      \markright{Sec. 24-3.1. Unlawful possession of firearms and
      firearm ammunition.}

      (a) A person commits the offense of unlawful possession of
      firearms or firearm ammunition when:

      (1) He is under 18 years of age and has in his possession any
      firearm of a size which may be concealed upon the person; or

      (2) He is under 21 years of age, has been convicted of a
      misdemeanor other than a traffic offense or adjudged delinquent
      and has any firearms or firearm ammunition in his possession; or

      (3) He is a narcotic addict and has any firearms or firearm
      ammunition in his possession; or

      (4) He has been a patient in a mental institution within the past
      5 years and has any firearms or firearm ammunition in his
      possession. For purposes of this paragraph (4):

      ``Mental institution'' means any hospital, institution, clinic,
      evaluation facility, mental health center, or part thereof, which
      is used primarily for the care or treatment of persons with mental
      illness.

      ``Patient in a mental institution'' means the person was admitted,
      either voluntarily or involuntarily, to a mental institution for
      mental health treatment, unless the treatment was voluntary and
      solely for an alcohol abuse disorder and no other secondary
      substance abuse disorder or mental illness; or

      (5) He is a person with an intellectual disability and has any
      firearms or firearm ammunition in his possession; or

      (6) He has in his possession any explosive bullet.

      For purposes of this paragraph ``explosive bullet'' means the
      projectile portion of an ammunition cartridge which contains or
      carries an explosive charge which will explode upon contact with
      the flesh of a human or an animal. ``Cartridge'' means a tubular
      metal case having a projectile affixed at the front thereof and a
      cap or primer at the rear end thereof, with the propellant
      contained in such tube between the projectile and the cap.

      (b) Sentence.

      Unlawful possession of firearms, other than handguns, and firearm
      ammunition is a Class A misdemeanor. Unlawful possession of
      handguns is a Class 4 felony. The possession of each firearm or
      firearm ammunition in violation of this Section constitutes a
      single and separate violation.

      (c) Nothing in paragraph (1) of subsection (a) of this Section
      prohibits a person under 18 years of age from participating in any
      lawful recreational activity with a firearm such as, but not
      limited to, practice shooting at targets upon established public
      or private target ranges or hunting, trapping, or fishing in
      accordance with the Wildlife Code or the Fish and Aquatic Life
      Code.

      (Source: P.A. 99-143, eff. 7-27-15.)

      \hypertarget{ilcs-524-3.2-from-ch.-38-par.-24-3.2}{%
      \subsection*{(720 ILCS 5/24-3.2) (from Ch. 38, par.
      24-3.2)}\label{ilcs-524-3.2-from-ch.-38-par.-24-3.2}}
      \addcontentsline{toc}{subsection}{(720 ILCS 5/24-3.2) (from Ch.
      38, par. 24-3.2)}

      \hypertarget{sec.-24-3.2.-unlawful-discharge-of-firearm-projectiles.}{%
      \section*{Sec. 24-3.2. Unlawful discharge of firearm
      projectiles.}\label{sec.-24-3.2.-unlawful-discharge-of-firearm-projectiles.}}
      \addcontentsline{toc}{section}{Sec. 24-3.2. Unlawful discharge of
      firearm projectiles.}

      \markright{Sec. 24-3.2. Unlawful discharge of firearm
      projectiles.}

      (a) A person commits the offense of unlawful discharge of firearm
      projectiles when he or she knowingly or recklessly uses an armor
      piercing bullet, dragon's breath shotgun shell, bolo shell, or
      flechette shell in violation of this Section.

      For purposes of this Section:

      ``Armor piercing bullet'' means any handgun bullet or handgun
      ammunition with projectiles or projectile cores constructed
      entirely (excluding the presence of traces of other substances)
      from tungsten alloys, steel, iron, brass, bronze, beryllium copper
      or depleted uranium, or fully jacketed bullets larger than 22
      caliber whose jacket has a weight of more than 25\% of the total
      weight of the projectile, and excluding those handgun projectiles
      whose cores are composed of soft materials such as lead or lead
      alloys, zinc or zinc alloys, frangible projectiles designed
      primarily for sporting purposes, and any other projectiles or
      projectile cores that the U. S. Secretary of the Treasury finds to
      be primarily intended to be used for sporting purposes or
      industrial purposes or that otherwise does not constitute ``armor
      piercing ammunition'' as that term is defined by federal law.

      ``Dragon's breath shotgun shell'' means any shotgun shell that
      contains exothermic pyrophoric mesh metal as the projectile and is
      designed for the purpose of throwing or spewing a flame or
      fireball to simulate a flame-thrower.

      ``Bolo shell'' means any shell that can be fired in a firearm and
      expels as projectiles 2 or more metal balls connected by solid
      metal wire.

      ``Flechette shell'' means any shell that can be fired in a firearm
      and expels 2 or more pieces of fin-stabilized solid metal wire or
      2 or more solid dart-type projectiles.

      (b) A person commits a Class X felony when he or she, knowing that
      a firearm, as defined in Section 1.1 of the Firearm Owners
      Identification Card Act, is loaded with an armor piercing bullet,
      dragon's breath shotgun shell, bolo shell, or flechette shell,
      intentionally or recklessly discharges such firearm and such
      bullet or shell strikes any other person.

      (c) Any person who possesses, concealed on or about his or her
      person, an armor piercing bullet, dragon's breath shotgun shell,
      bolo shell, or flechette shell and a firearm suitable for the
      discharge thereof is guilty of a Class 2 felony.

      (d) This Section does not apply to or affect any of the following:

      (1) Peace officers;

      (2) Wardens, superintendents and keepers of prisons,
      penitentiaries, jails and other institutions for the detention of
      persons accused or convicted of an offense;

      (3) Members of the Armed Services or Reserve Forces of the United
      States or the Illinois National Guard while in the performance of
      their official duties;

      (4) Federal officials required to carry firearms, while engaged in
      the performance of their official duties;

      (5) United States Marshals, while engaged in the performance of
      their official duties.

      (Source: P.A. 92-423, eff. 1-1-02.)

      \hypertarget{ilcs-524-3.3-from-ch.-38-par.-24-3.3}{%
      \subsection*{(720 ILCS 5/24-3.3) (from Ch. 38, par.
      24-3.3)}\label{ilcs-524-3.3-from-ch.-38-par.-24-3.3}}
      \addcontentsline{toc}{subsection}{(720 ILCS 5/24-3.3) (from Ch.
      38, par. 24-3.3)}

      \hypertarget{sec.-24-3.3.-unlawful-sale-or-delivery-of-firearms-on-the-premises-of-any-school-regardless-of-the-time-of-day-or-the-time-of-year-or-any-conveyance-owned-leased-or-contracted-by-a-school-to-transport-students-to-or-from-school-or-a-school-related-activity-or-residential-property-owned-operated-or-managed-by-a-public-housing-agency.}{%
      \section*{Sec. 24-3.3. Unlawful Sale or Delivery of Firearms on
      the Premises of Any School, regardless of the time of day or the
      time of year, or any conveyance owned, leased or contracted by a
      school to transport students to or from school or a school related
      activity, or residential property owned, operated or managed by a
      public housing
      agency.}\label{sec.-24-3.3.-unlawful-sale-or-delivery-of-firearms-on-the-premises-of-any-school-regardless-of-the-time-of-day-or-the-time-of-year-or-any-conveyance-owned-leased-or-contracted-by-a-school-to-transport-students-to-or-from-school-or-a-school-related-activity-or-residential-property-owned-operated-or-managed-by-a-public-housing-agency.}}
      \addcontentsline{toc}{section}{Sec. 24-3.3. Unlawful Sale or
      Delivery of Firearms on the Premises of Any School, regardless of
      the time of day or the time of year, or any conveyance owned,
      leased or contracted by a school to transport students to or from
      school or a school related activity, or residential property
      owned, operated or managed by a public housing agency.}

      \markright{Sec. 24-3.3. Unlawful Sale or Delivery of Firearms on
      the Premises of Any School, regardless of the time of day or the
      time of year, or any conveyance owned, leased or contracted by a
      school to transport students to or from school or a school related
      activity, or residential property owned, operated or managed by a
      public housing agency.}

      Any person 18 years of age or older who sells, gives or delivers
      any firearm to any person under 18 years of age in any school,
      regardless of the time of day or the time of year or residential
      property owned, operated or managed by a public housing agency or
      leased by a public housing agency as part of a scattered site or
      mixed-income development, on the real property comprising any
      school, regardless of the time of day or the time of year or
      residential property owned, operated or managed by a public
      housing agency or leased by a public housing agency as part of a
      scattered site or mixed-income development commits a Class 3
      felony. School is defined, for the purposes of this Section, as
      any public or private elementary or secondary school, community
      college, college or university. This does not apply to peace
      officers or to students carrying or possessing firearms for use in
      school training courses, parades, target shooting on school
      ranges, or otherwise with the consent of school authorities and
      which firearms are transported unloaded and enclosed in a suitable
      case, box or transportation package.

      (Source: P.A. 91-673, eff. 12-22-99 .)

      \hypertarget{ilcs-524-3.4-from-ch.-38-par.-24-3.4}{%
      \subsection*{(720 ILCS 5/24-3.4) (from Ch. 38, par.
      24-3.4)}\label{ilcs-524-3.4-from-ch.-38-par.-24-3.4}}
      \addcontentsline{toc}{subsection}{(720 ILCS 5/24-3.4) (from Ch.
      38, par. 24-3.4)}

      \hypertarget{sec.-24-3.4.-unlawful-sale-of-firearms-by-liquor-licensee.}{%
      \section*{Sec. 24-3.4. Unlawful sale of firearms by liquor
      licensee.}\label{sec.-24-3.4.-unlawful-sale-of-firearms-by-liquor-licensee.}}
      \addcontentsline{toc}{section}{Sec. 24-3.4. Unlawful sale of
      firearms by liquor licensee.}

      \markright{Sec. 24-3.4. Unlawful sale of firearms by liquor
      licensee.}

      (a) It shall be unlawful for any person who holds a license to
      sell at retail any alcoholic liquor issued by the Illinois Liquor
      Control Commission or local liquor control commissioner under the
      Liquor Control Act of 1934 or an agent or employee of the licensee
      to sell or deliver to any other person a firearm in or on the real
      property of the establishment where the licensee is licensed to
      sell alcoholic liquors unless the sale or delivery of the firearm
      is otherwise lawful under this Article and under the Firearm
      Owners Identification Card Act.

      (b) Sentence. A violation of subsection (a) of this Section is a
      Class 4 felony.

      (Source: P.A. 87-591.)

      \hypertarget{ilcs-524-3.5}{%
      \subsection*{(720 ILCS 5/24-3.5)}\label{ilcs-524-3.5}}
      \addcontentsline{toc}{subsection}{(720 ILCS 5/24-3.5)}

      \hypertarget{sec.-24-3.5.-unlawful-purchase-of-a-firearm.}{%
      \section*{Sec. 24-3.5. Unlawful purchase of a
      firearm.}\label{sec.-24-3.5.-unlawful-purchase-of-a-firearm.}}
      \addcontentsline{toc}{section}{Sec. 24-3.5. Unlawful purchase of a
      firearm.}

      \markright{Sec. 24-3.5. Unlawful purchase of a firearm.}

      (a) For purposes of this Section, ``firearms transaction record
      form'' means a form:

      (1) executed by a transferee of a firearm stating:

      (i) the transferee's name and address (including county or similar
      political subdivision); (ii) whether the transferee is a citizen
      of the United States; (iii) the transferee's State of residence;
      and (iv) the date and place of birth, height, weight, and race of
      the transferee; and

      (2) on which the transferee certifies that he or she is not
      prohibited by federal law from transporting or shipping a firearm
      in interstate or foreign commerce or receiving a firearm that has
      been shipped or transported in interstate or foreign commerce or
      possessing a firearm in or affecting commerce.

      (b) A person commits the offense of unlawful purchase of a firearm
      who knowingly purchases or attempts to purchase a firearm with the
      intent to deliver that firearm to another person who is prohibited
      by federal or State law from possessing a firearm.

      (c) A person commits the offense of unlawful purchase of a firearm
      when he or she, in purchasing or attempting to purchase a firearm,
      intentionally provides false or misleading information on a United
      States Department of the Treasury, Bureau of Alcohol, Tobacco and
      Firearms firearms transaction record form.

      (d) Exemption. It is not a violation of subsection (b) of this
      Section for a person to make a gift or loan of a firearm to a
      person who is not prohibited by federal or State law from
      possessing a firearm if the transfer of the firearm is made in
      accordance with Section 3 of the Firearm Owners Identification
      Card Act.

      (e) Sentence.

      (1) A person who commits the offense of unlawful purchase of a
      firearm:

      (A) is guilty of a Class 2 felony for purchasing or attempting to
      purchase one firearm;

      (B) is guilty of a Class 1 felony for purchasing or attempting to
      purchase not less than 2 firearms and not more than 5 firearms at
      the same time or within a one year period;

      (C) is guilty of a Class X felony for which the offender shall be
      sentenced to a term of imprisonment of not less than 9 years and
      not more than 40 years for purchasing or attempting to purchase
      not less than 6 firearms at the same time or within a 2 year
      period.

      (2) In addition to any other penalty that may be imposed for a
      violation of this Section, the court may sentence a person
      convicted of a violation of subsection (c) of this Section to a
      fine not to exceed \$250,000 for each violation.

      (f) A prosecution for unlawful purchase of a firearm may be
      commenced within 6 years after the commission of the offense.

      (Source: P.A. 95-882, eff. 1-1-09.)

      \hypertarget{ilcs-524-3.6}{%
      \subsection*{(720 ILCS 5/24-3.6)}\label{ilcs-524-3.6}}
      \addcontentsline{toc}{subsection}{(720 ILCS 5/24-3.6)}

      \hypertarget{sec.-24-3.6.-unlawful-use-of-a-firearm-in-the-shape-of-a-wireless-telephone.}{%
      \section*{Sec. 24-3.6. Unlawful use of a firearm in the shape of a
      wireless
      telephone.}\label{sec.-24-3.6.-unlawful-use-of-a-firearm-in-the-shape-of-a-wireless-telephone.}}
      \addcontentsline{toc}{section}{Sec. 24-3.6. Unlawful use of a
      firearm in the shape of a wireless telephone.}

      \markright{Sec. 24-3.6. Unlawful use of a firearm in the shape of
      a wireless telephone.}

      (a) For the purposes of this Section, ``wireless telephone'' means
      a device that is capable of transmitting or receiving telephonic
      communications without a wire connecting the device to the
      telephone network.

      (b) A person commits the offense of unlawful use of a firearm in
      the shape of a wireless telephone when he or she manufactures,
      sells, transfers, purchases, possesses, or carries a firearm
      shaped or designed to appear as a wireless telephone.

      (c) This Section does not apply to or affect the sale to or
      possession of a firearm in the shape of a wireless telephone by a
      peace officer.

      (d) Sentence. Unlawful use of a firearm in the shape of a wireless
      telephone is a Class 4 felony.

      (Source: P.A. 92-155, eff. 1-1-02.)

      \hypertarget{ilcs-524-3.7}{%
      \subsection*{(720 ILCS 5/24-3.7)}\label{ilcs-524-3.7}}
      \addcontentsline{toc}{subsection}{(720 ILCS 5/24-3.7)}

      \hypertarget{sec.-24-3.7.-use-of-a-stolen-firearm-in-the-commission-of-an-offense.}{%
      \section*{Sec. 24-3.7. Use of a stolen firearm in the commission
      of an
      offense.}\label{sec.-24-3.7.-use-of-a-stolen-firearm-in-the-commission-of-an-offense.}}
      \addcontentsline{toc}{section}{Sec. 24-3.7. Use of a stolen
      firearm in the commission of an offense.}

      \markright{Sec. 24-3.7. Use of a stolen firearm in the commission
      of an offense.}

      (a) A person commits the offense of use of a stolen firearm in the
      commission of an offense when he or she knowingly uses a stolen
      firearm in the commission of any offense and the person knows that
      the firearm was stolen.

      (b) Sentence. Use of a stolen firearm in the commission of an
      offense is a Class 2 felony.

      (Source: P.A. 96-190, eff. 1-1-10.)

      \hypertarget{ilcs-524-3.8}{%
      \subsection*{(720 ILCS 5/24-3.8)}\label{ilcs-524-3.8}}
      \addcontentsline{toc}{subsection}{(720 ILCS 5/24-3.8)}

      \hypertarget{sec.-24-3.8.-possession-of-a-stolen-firearm.}{%
      \section*{Sec. 24-3.8. Possession of a stolen
      firearm.}\label{sec.-24-3.8.-possession-of-a-stolen-firearm.}}
      \addcontentsline{toc}{section}{Sec. 24-3.8. Possession of a stolen
      firearm.}

      \markright{Sec. 24-3.8. Possession of a stolen firearm.}

      (a) A person commits possession of a stolen firearm when he or
      she, not being entitled to the possession of a firearm, possesses
      the firearm, knowing it to have been stolen or converted. The
      trier of fact may infer that a person who possesses a firearm with
      knowledge that its serial number has been removed or altered has
      knowledge that the firearm is stolen or converted.

      (b) Possession of a stolen firearm is a Class 2 felony.

      (Source: P.A. 97-597, eff. 1-1-12; incorporates 97-347, eff.
      1-1-12; 97-1109, eff. 1-1-13.)

      \hypertarget{ilcs-524-3.9}{%
      \subsection*{(720 ILCS 5/24-3.9)}\label{ilcs-524-3.9}}
      \addcontentsline{toc}{subsection}{(720 ILCS 5/24-3.9)}

      \hypertarget{sec.-24-3.9.-aggravated-possession-of-a-stolen-firearm.}{%
      \section*{Sec. 24-3.9. Aggravated possession of a stolen
      firearm.}\label{sec.-24-3.9.-aggravated-possession-of-a-stolen-firearm.}}
      \addcontentsline{toc}{section}{Sec. 24-3.9. Aggravated possession
      of a stolen firearm.}

      \markright{Sec. 24-3.9. Aggravated possession of a stolen
      firearm.}

      (a) A person commits aggravated possession of a stolen firearm
      when he or she:

      (1) Not being entitled to the possession of not less than 2 and
      not more than 5 firearms, possesses those firearms at the same
      time or within a one-year period, knowing the firearms to have
      been stolen or converted.

      (2) Not being entitled to the possession of not less than 6 and
      not more than 10 firearms, possesses those firearms at the same
      time or within a 2-year period, knowing the firearms to have been
      stolen or converted.

      (3) Not being entitled to the possession of not less than 11 and
      not more than 20 firearms, possesses those firearms at the same
      time or within a 3-year period, knowing the firearms to have been
      stolen or converted.

      (4) Not being entitled to the possession of not less than 21 and
      not more than 30 firearms, possesses those firearms at the same
      time or within a 4-year period, knowing the firearms to have been
      stolen or converted.

      (5) Not being entitled to the possession of more than

      30 firearms, possesses those firearms at the same time or within a
      5-year period, knowing the firearms to have been stolen or
      converted.

      (b) The trier of fact may infer that a person who possesses a
      firearm with knowledge that its serial number has been removed or
      altered has knowledge that the firearm is stolen or converted.

      (c) Sentence.

      (1) A person who violates paragraph (1) of subsection

      (a) of this Section commits a Class 1 felony.

      (2) A person who violates paragraph (2) of subsection

      (a) of this Section commits a Class X felony for which he or she
      shall be sentenced to a term of imprisonment of not less than 6
      years and not more than 30 years.

      (3) A person who violates paragraph (3) of subsection

      (a) of this Section commits a Class X felony for which he or she
      shall be sentenced to a term of imprisonment of not less than 6
      years and not more than 40 years.

      (4) A person who violates paragraph (4) of subsection

      (a) of this Section commits a Class X felony for which he or she
      shall be sentenced to a term of imprisonment of not less than 6
      years and not more than 50 years.

      (5) A person who violates paragraph (5) of subsection

      (a) of this Section commits a Class X felony for which he or she
      shall be sentenced to a term of imprisonment of not less than 6
      years and not more than 60 years.

      (Source: P.A. 97-597, eff. 1-1-12; incorporates 97-347, eff.
      1-1-12; 97-1109, eff. 1-1-13.)

      \hypertarget{ilcs-524-4-from-ch.-38-par.-24-4}{%
      \subsection*{(720 ILCS 5/24-4) (from Ch. 38, par.
      24-4)}\label{ilcs-524-4-from-ch.-38-par.-24-4}}
      \addcontentsline{toc}{subsection}{(720 ILCS 5/24-4) (from Ch. 38,
      par. 24-4)}

      \hypertarget{sec.-24-4.-register-of-sales-by-dealer.}{%
      \section*{Sec. 24-4. Register of sales by
      dealer.}\label{sec.-24-4.-register-of-sales-by-dealer.}}
      \addcontentsline{toc}{section}{Sec. 24-4. Register of sales by
      dealer.}

      \markright{Sec. 24-4. Register of sales by dealer.}

      (a) Any seller of firearms of a size which may be concealed upon
      the person, other than a manufacturer selling to a bona fide
      wholesaler or retailer or a wholesaler selling to a bona fide
      retailer, shall keep a register of all firearms sold or given
      away.

      (b) Such register shall contain the date of the sale or gift, the
      name, address, age and occupation of the person to whom the weapon
      is sold or given, the price of the weapon, the kind, description
      and number of the weapon, and the purpose for which it is
      purchased and obtained.

      (c) Such seller on demand of a peace officer shall produce for
      inspection the register and allow such peace officer to inspect
      such register and all stock on hand.

      (d) Sentence.

      Violation of this Section is a Class B misdemeanor.

      (Source: P.A. 77-2638.)

      \hypertarget{ilcs-524-4.1}{%
      \subsection*{(720 ILCS 5/24-4.1)}\label{ilcs-524-4.1}}
      \addcontentsline{toc}{subsection}{(720 ILCS 5/24-4.1)}

      \hypertarget{sec.-24-4.1.-report-of-lost-or-stolen-firearms.}{%
      \section*{Sec. 24-4.1. Report of lost or stolen
      firearms.}\label{sec.-24-4.1.-report-of-lost-or-stolen-firearms.}}
      \addcontentsline{toc}{section}{Sec. 24-4.1. Report of lost or
      stolen firearms.}

      \markright{Sec. 24-4.1. Report of lost or stolen firearms.}

      (a) If a person who possesses a valid Firearm Owner's
      Identification Card and who possesses or acquires a firearm
      thereafter loses the firearm, or if the firearm is stolen from the
      person, the person must report the loss or theft to the local law
      enforcement agency within 72 hours after obtaining knowledge of
      the loss or theft.

      (b) A law enforcement agency having jurisdiction shall take a
      written report and shall, as soon as practical, enter the
      firearm's serial number as stolen into the Law Enforcement
      Agencies Data System (LEADS).

      (c) A person shall not be in violation of this Section if:

      (1) the failure to report is due to an act of God, act of war, or
      inability of a law enforcement agency to receive the report;

      (2) the person is hospitalized, in a coma, or is otherwise
      seriously physically or mentally impaired as to prevent the person
      from reporting; or

      (3) the person's designee makes a report if the person is unable
      to make the report.

      (d) Sentence. A person who violates this Section is guilty of a
      petty offense for a first violation. A second or subsequent
      violation of this Section is a Class A misdemeanor.

      (Source: P.A. 98-508, eff. 8-19-13.)

      \hypertarget{ilcs-524-5-from-ch.-38-par.-24-5}{%
      \subsection*{(720 ILCS 5/24-5) (from Ch. 38, par.
      24-5)}\label{ilcs-524-5-from-ch.-38-par.-24-5}}
      \addcontentsline{toc}{subsection}{(720 ILCS 5/24-5) (from Ch. 38,
      par. 24-5)}

      \hypertarget{sec.-24-5.-defacing-identification-marks-of-firearms.}{%
      \section*{Sec. 24-5. Defacing identification marks of
      firearms.}\label{sec.-24-5.-defacing-identification-marks-of-firearms.}}
      \addcontentsline{toc}{section}{Sec. 24-5. Defacing identification
      marks of firearms.}

      \markright{Sec. 24-5. Defacing identification marks of firearms.}

      (a) Any person who shall knowingly or intentionally change, alter,
      remove or obliterate the name of the importer's or manufacturer's
      serial number of any firearm commits a Class 2 felony.

      (b) A person who possesses any firearm upon which any such
      importer's or manufacturer's serial number has been changed,
      altered, removed or obliterated commits a Class 3 felony.

      (c) Nothing in this Section shall prevent a person from making
      repairs, replacement of parts, or other changes to a firearm if
      those repairs, replacement of parts, or changes cause the removal
      of the name of the maker, model, or other marks of identification
      other than the serial number on the firearm's frame or receiver.

      (d) A prosecution for a violation of this Section may be commenced
      within 6 years after the commission of the offense.

      (Source: P.A. 93-906, eff. 8-11-04.)

      \hypertarget{ilcs-524-5.1}{%
      \subsection*{(720 ILCS 5/24-5.1)}\label{ilcs-524-5.1}}
      \addcontentsline{toc}{subsection}{(720 ILCS 5/24-5.1)}

      \hypertarget{sec.-24-5.1.-serialization-of-unfinished-frames-or-receivers-prohibition-on-unserialized-firearms-exceptions-penalties.}{%
      \section*{Sec. 24-5.1. Serialization of unfinished frames or
      receivers; prohibition on unserialized firearms; exceptions;
      penalties.}\label{sec.-24-5.1.-serialization-of-unfinished-frames-or-receivers-prohibition-on-unserialized-firearms-exceptions-penalties.}}
      \addcontentsline{toc}{section}{Sec. 24-5.1. Serialization of
      unfinished frames or receivers; prohibition on unserialized
      firearms; exceptions; penalties.}

      \markright{Sec. 24-5.1. Serialization of unfinished frames or
      receivers; prohibition on unserialized firearms; exceptions;
      penalties.}

      (a) In this Section:

      ``Bona fide supplier'' means an established business entity
      engaged in the development and sale of firearms parts to one or
      more federal firearms manufacturers or federal firearms importers.

      ``Federal firearms dealer'' means a licensed manufacturer pursuant
      to 18 U.S.C. 921(a)(11).

      ``Federal firearms importer'' means a licensed importer pursuant
      to 18 U.S.C. 921(a)(9).

      ``Federal firearms manufacturer'' means a licensed manufacturer
      pursuant to 18 U.S.C. 921(a)(10).

      ``Frame or receiver'' means a part of a firearm that, when the
      complete weapon is assembled, is visible from the exterior and
      provides housing or a structure designed to hold or integrate one
      or more fire control components, even if pins or other attachments
      are required to connect those components to the housing or
      structure. For models of firearms in which multiple parts provide
      such housing or structure, the part or parts that the Director of
      the federal Bureau of Alcohol, Tobacco, Firearms and Explosives
      has determined are a frame or receiver constitute the frame or
      receiver. For purposes of this definition, ``fire control
      component'' means a component necessary for the firearm to
      initiate, complete, or continue the firing sequence, including any
      of the following: hammer, bolt, bolt carrier, breechblock,
      cylinder, trigger mechanism, firing pin, striker, or slide rails.

      ``Security exemplar'' means an object to be fabricated at the
      direction of the United States Attorney General that is (1)
      constructed of 3.7 ounces of material type 17-4 PH stainless steel
      in a shape resembling a handgun and (2) suitable for testing and
      calibrating metal detectors.

      ``Three-dimensional printer'' means a computer or computer-drive
      machine capable of producing a three-dimensional object from a
      digital model.

      ``Undetectable firearm'' means (1) a firearm constructed entirely
      of non-metal substances; (2) a firearm that, after removal of all
      parts but the major components of the firearm, is not detectable
      by walk-through metal detectors calibrated and operated to detect
      the security exemplar; or (3) a firearm that includes a major
      component of a firearm, which, if subject to the types of
      detection devices commonly used at airports for security
      screening, would not generate an image that accurately depicts the
      shape of the component. ``Undetectable firearm'' does not include
      a firearm subject to the provisions of 18 U.S.C. 922(p)(3) through
      (6).

      ``Unfinished frame or receiver'' means any forging, casting,
      printing, extrusion, machined body, or similar article that:

      (1) has reached a stage in manufacture where it may readily be
      completed, assembled, or converted to be a functional firearm; or

      (2) is marketed or sold to the public to become or be used as the
      frame or receiver of a functional firearm once completed,
      assembled, or converted.

      ``Unserialized'' means lacking a serial number imprinted by:

      (1) a federal firearms manufacturer, federal firearms importer,
      federal firearms dealer, or other federal licensee authorized to
      provide marking services, pursuant to a requirement under federal
      law; or

      (2) a federal firearms dealer or other federal licensee authorized
      to provide marking services pursuant to subsection (f) of this
      Section.

      (b) It is unlawful for any person to knowingly sell, offer to
      sell, or transfer an unserialized unfinished frame or receiver or
      unserialized firearm, including those produced using a
      three-dimensional printer, unless the party purchasing or
      receiving the unfinished frame or receiver or unserialized firearm
      is a federal firearms importer, federal firearms manufacturer, or
      federal firearms dealer.

      (c) Beginning 180 days after the effective date of this amendatory
      Act of the 102nd General Assembly, it is unlawful for any person
      to knowingly possess, transport, or receive an unfinished frame or
      receiver, unless:

      (1) the party possessing or receiving the unfinished frame or
      receiver is a federal firearms importer or federal firearms
      manufacturer;

      (2) the unfinished frame or receiver is possessed or transported
      by a person for transfer to a federal firearms importer or federal
      firearms manufacturer; or

      (3) the unfinished frame or receiver has been imprinted with a
      serial number issued by a federal firearms importer or federal
      firearms manufacturer in compliance with subsection (f) of this
      Section.

      (d) Beginning 180 days after the effective date of this amendatory
      Act of the 102nd General Assembly, unless the party receiving the
      firearm is a federal firearms importer or federal firearms
      manufacturer, it is unlawful for any person to knowingly possess,
      purchase, transport, or receive a firearm that is not imprinted
      with a serial number by (1) a federal firearms importer or federal
      firearms manufacturer in compliance with all federal laws and
      regulations regulating the manufacture and import of firearms or
      (2) a federal firearms manufacturer, federal firearms dealer, or
      other federal licensee authorized to provide marking services in
      compliance with the unserialized firearm serialization process
      under subsection (f) of this Section.

      (e) Any firearm or unfinished frame or receiver manufactured using
      a three-dimensional printer must also be serialized in accordance
      with the requirements of subsection (f) within 30 days after the
      effective date of this amendatory Act of the 102nd General
      Assembly, or prior to reaching a stage of manufacture where it may
      be readily completed, assembled, or converted to be a functional
      firearm.

      (f) Unserialized unfinished frames or receivers and unserialized
      firearms serialized pursuant to this Section shall be serialized
      in compliance with all of the following:

      (1) An unserialized unfinished frame or receiver and unserialized
      firearm shall be serialized by a federally licensed firearms
      dealer or other federal licensee authorized to provide marking
      services with the licensee's abbreviated federal firearms license
      number as a prefix (which is the first 3 and last 5 digits)
      followed by a hyphen, and then followed by a number as a suffix,
      such as 12345678-(number). The serial number or numbers must be
      placed in a manner that accords with the requirements under
      federal law for affixing serial numbers to firearms, including the
      requirements that the serial number or numbers be at the minimum
      size and depth, and not susceptible to being readily obliterated,
      altered, or removed, and the licensee must retain records that
      accord with the requirements under federal law in the case of the
      sale of a firearm. The imprinting of any serial number upon a
      undetectable firearm must be done on a steel plaque in compliance
      with 18 U.S.C. 922(p).

      (2) Every federally licensed firearms dealer or other federal
      licensee that engraves, casts, stamps, or otherwise conspicuously
      and permanently places a unique serial number pursuant to this
      Section shall maintain a record of such indefinitely. Licensees
      subject to the Firearm Dealer License Certification Act shall make
      all records accessible for inspection upon the request of the
      Illinois State Police or a law enforcement agency in accordance
      with Section 5-35 of the Firearm Dealer License Certification Act.

      (3) Every federally licensed firearms dealer or other federal
      licensee that engraves, casts, stamps, or otherwise conspicuously
      and permanently places a unique serial number pursuant to this
      Section shall record it at the time of every transaction involving
      the transfer of a firearm, rifle, shotgun, finished frame or
      receiver, or unfinished frame or receiver that has been so marked
      in compliance with the federal guidelines set forth in 27 CFR
      478.124.

      (4) Every federally licensed firearms dealer or other federal
      licensee that engraves, casts, stamps, or otherwise conspicuously
      and permanently places a unique serial number pursuant to this
      Section shall review and confirm the validity of the owner's
      Firearm Owner's Identification Card issued under the Firearm
      Owners Identification Card Act prior to returning the firearm to
      the owner.

      (g) Within 30 days after the effective date of this amendatory Act
      of the 102nd General Assembly, the Director of the Illinois State
      Police shall issue a public notice regarding the provisions of
      this Section. The notice shall include posting on the Illinois
      State Police website and may include written notification or any
      other means of communication statewide to all Illinois-based
      federal firearms manufacturers, federal firearms dealers, or other
      federal licensees authorized to provide marking services in
      compliance with the serialization process in subsection (f) in
      order to educate the public.

      (h) Exceptions. This Section does not apply to an unserialized
      unfinished frame or receiver or an unserialized firearm that:

      (1) has been rendered permanently inoperable;

      (2) is an antique firearm, as defined in 18 U.S.C.

      921(a)(16);

      (3) was manufactured prior to October 22, 1968;

      (4) is an unfinished frame or receiver and is possessed by a bona
      fide supplier exclusively for transfer to a federal firearms
      manufacturer or federal firearms importer, or is possessed by a
      federal firearms manufacturer or federal firearms importer in
      compliance with all federal laws and regulations regulating the
      manufacture and import of firearms; except this exemption does not
      apply if an unfinished frame or receiver is possessed for transfer
      or is transferred to a person other than a federal firearms
      manufacturer or federal firearms importer; or

      (5) is possessed by a person who received the unserialized
      unfinished frame or receiver or unserialized firearm through
      inheritance, and is not otherwise prohibited from possessing the
      unserialized unfinished frame or receiver or unserialized firearm,
      for a period not exceeding 30 days after inheriting the
      unserialized unfinished frame or receiver or unserialized firearm.

      (i) Penalties.

      (1) A person who violates subsection (c) or (d) is guilty of a
      Class A misdemeanor for a first violation and is guilty of a Class
      3 felony for a second or subsequent violation.

      (2) A person who violates subsection (b) is guilty of a Class 4
      felony for a first violation and is guilty of a Class 2 felony for
      a second or subsequent violation.

      (Source: P.A. 102-889, eff. 5-18-22.)

      \hypertarget{ilcs-524-6-from-ch.-38-par.-24-6}{%
      \subsection*{(720 ILCS 5/24-6) (from Ch. 38, par.
      24-6)}\label{ilcs-524-6-from-ch.-38-par.-24-6}}
      \addcontentsline{toc}{subsection}{(720 ILCS 5/24-6) (from Ch. 38,
      par. 24-6)}

      \hypertarget{sec.-24-6.-confiscation-and-disposition-of-weapons.}{%
      \section*{Sec. 24-6. Confiscation and disposition of
      weapons.}\label{sec.-24-6.-confiscation-and-disposition-of-weapons.}}
      \addcontentsline{toc}{section}{Sec. 24-6. Confiscation and
      disposition of weapons.}

      \markright{Sec. 24-6. Confiscation and disposition of weapons.}

      (a) Upon conviction of an offense in which a weapon was used or
      possessed by the offender, any weapon seized shall be confiscated
      by the trial court.

      (b) Any stolen weapon so confiscated, when no longer needed for
      evidentiary purposes, shall be returned to the person entitled to
      possession, if known. After the disposition of a criminal case or
      in any criminal case where a final judgment in the case was not
      entered due to the death of the defendant, and when a confiscated
      weapon is no longer needed for evidentiary purposes, and when in
      due course no legitimate claim has been made for the weapon, the
      court may transfer the weapon to the sheriff of the county who may
      proceed to destroy it, or may in its discretion order the weapon
      preserved as property of the governmental body whose police agency
      seized the weapon, or may in its discretion order the weapon to be
      transferred to the Illinois State Police for use by the crime
      laboratory system, for training purposes, or for any other
      application as deemed appropriate by the Department. If, after the
      disposition of a criminal case, a need still exists for the use of
      the confiscated weapon for evidentiary purposes, the court may
      transfer the weapon to the custody of the State Department of
      Corrections for preservation. The court may not order the transfer
      of the weapon to any private individual or private organization
      other than to return a stolen weapon to its rightful owner.

      The provisions of this Section shall not apply to violations of
      the Fish and Aquatic Life Code or the Wildlife Code. Confiscation
      of weapons for Fish and Aquatic Life Code and Wildlife Code
      violations shall be only as provided in those Codes.

      (c) Any mental hospital that admits a person as an inpatient
      pursuant to any of the provisions of the Mental Health and
      Developmental Disabilities Code shall confiscate any firearms in
      the possession of that person at the time of admission, or at any
      time the firearms are discovered in the person's possession during
      the course of hospitalization. The hospital shall, as soon as
      possible following confiscation, transfer custody of the firearms
      to the appropriate law enforcement agency. The hospital shall give
      written notice to the person from whom the firearm was confiscated
      of the identity and address of the law enforcement agency to which
      it has given the firearm.

      The law enforcement agency shall maintain possession of any
      firearm it obtains pursuant to this subsection for a minimum of 90
      days. Thereafter, the firearm may be disposed of pursuant to the
      provisions of subsection (b) of this Section.

      (Source: P.A. 102-538, eff. 8-20-21.)

      \hypertarget{ilcs-524-7}{%
      \subsection*{(720 ILCS 5/24-7)}\label{ilcs-524-7}}
      \addcontentsline{toc}{subsection}{(720 ILCS 5/24-7)}

      \hypertarget{sec.-24-7.-weapons-offenses-community-service.}{%
      \section*{Sec. 24-7. Weapons offenses; community
      service.}\label{sec.-24-7.-weapons-offenses-community-service.}}
      \addcontentsline{toc}{section}{Sec. 24-7. Weapons offenses;
      community service.}

      \markright{Sec. 24-7. Weapons offenses; community service.}

      In addition to any other sentence that may be imposed, a court
      shall order any person convicted of a violation of this Article to
      perform community service for not less than 30 and not more than
      120 hours, if community service is available in the jurisdiction
      and is funded and approved by the county board of the county where
      the offense was committed. In addition, whenever any person is
      placed on supervision for an alleged offense under this Article,
      the supervision shall be conditioned upon the performance of the
      community service.

      This Section does not apply when the court imposes a sentence of
      incarceration.

      (Source: P.A. 88-558, eff. 1-1-95; 89-8, eff. 3-21-95.)

      \hypertarget{ilcs-524-8}{%
      \subsection*{(720 ILCS 5/24-8)}\label{ilcs-524-8}}
      \addcontentsline{toc}{subsection}{(720 ILCS 5/24-8)}

      \hypertarget{sec.-24-8.-firearm-evidence.}{%
      \section*{Sec. 24-8. Firearm
      evidence.}\label{sec.-24-8.-firearm-evidence.}}
      \addcontentsline{toc}{section}{Sec. 24-8. Firearm evidence.}

      \markright{Sec. 24-8. Firearm evidence.}

      (a) Upon recovering a firearm from the possession of anyone who is
      not permitted by federal or State law to possess a firearm, a law
      enforcement agency shall use the best available information,
      including a firearms trace when necessary, to determine how and
      from whom the person gained possession of the firearm. Upon
      recovering a firearm that was used in the commission of any
      offense classified as a felony or upon recovering a firearm that
      appears to have been lost, mislaid, stolen, or otherwise
      unclaimed, a law enforcement agency shall use the best available
      information, including a firearms trace, to determine prior
      ownership of the firearm.

      (b) Law enforcement shall, when appropriate, use the National
      Tracing Center of the Federal Bureau of Alcohol, Tobacco and
      Firearms and the National Crime Information Center of the Federal
      Bureau of Investigation in complying with subsection (a) of this
      Section.

      (c) Law enforcement agencies shall use the Illinois State Police
      Law Enforcement Agencies Data System (LEADS) Gun File to enter all
      stolen, seized, or recovered firearms as prescribed by LEADS
      regulations and policies.

      (d) Whenever a law enforcement agency recovers a fired cartridge
      case at a crime scene or has reason to believe that the recovered
      fired cartridge case is related to or associated with the
      commission of a crime, the law enforcement agency shall submit the
      evidence to the National Integrated Ballistics Information Network
      (NIBIN) or an Illinois State Police laboratory for NIBIN
      processing. Whenever a law enforcement agency seizes or recovers a
      semiautomatic firearm that is deemed suitable to be entered into
      the NIBIN that was: (i) unlawfully possessed, (ii) used for any
      unlawful purpose, (iii) recovered from the scene of a crime, (iv)
      is reasonably believed to have been used or associated with the
      commission of a crime, or (v) is acquired by the law enforcement
      agency as an abandoned or discarded firearm, the law enforcement
      agency shall submit the evidence to the NIBIN or an Illinois State
      Police laboratory for NIBIN processing. When practicable, all
      NIBIN-suitable evidence and NIBIN-suitable test fires from
      recovered firearms shall be entered into the NIBIN within 2
      business days of submission to Illinois State Police laboratories
      that have NIBIN access or another NIBIN site. Exceptions to this
      may occur if the evidence in question requires analysis by other
      forensic disciplines. The Illinois State Police laboratory,
      submitting agency, and relevant court representatives shall
      determine whether the request for additional analysis outweighs
      the 2 business-day requirement. Illinois State Police laboratories
      that do not have NIBIN access shall submit NIBIN-suitable evidence
      and test fires to an Illinois State Police laboratory with NIBIN
      access. Upon receipt at the laboratory with NIBIN access, when
      practicable, the evidence and test fires shall be entered into the
      NIBIN within 2 business days. Exceptions to this 2 business-day
      requirement may occur if the evidence in question requires
      analysis by other forensic disciplines. The Illinois State Police
      laboratory, submitting agency, and relevant court representatives
      shall determine whether the request for additional analysis
      outweighs the 2 business-day requirement. Nothing in this Section
      shall be interpreted to conflict with standards and policies for
      NIBIN sites as promulgated by the federal Bureau of Alcohol,
      Tobacco, Firearms and Explosives or successor agencies.

      (Source: P.A. 102-237, eff. 1-1-22; 102-538, eff. 8-20-21;
      102-813, eff. 5-13-22.)

      \hypertarget{ilcs-524-9}{%
      \subsection*{(720 ILCS 5/24-9)}\label{ilcs-524-9}}
      \addcontentsline{toc}{subsection}{(720 ILCS 5/24-9)}

      \hypertarget{sec.-24-9.-firearms-child-protection.}{%
      \section*{Sec. 24-9. Firearms; Child
      Protection.}\label{sec.-24-9.-firearms-child-protection.}}
      \addcontentsline{toc}{section}{Sec. 24-9. Firearms; Child
      Protection.}

      \markright{Sec. 24-9. Firearms; Child Protection.}

      (a) Except as provided in subsection (c), it is unlawful for any
      person to store or leave, within premises under his or her
      control, a firearm if the person knows or has reason to believe
      that a minor under the age of 14 years who does not have a Firearm
      Owners Identification Card is likely to gain access to the firearm
      without the lawful permission of the minor's parent, guardian, or
      person having charge of the minor, and the minor causes death or
      great bodily harm with the firearm, unless the firearm is:

      (1) secured by a device or mechanism, other than the firearm
      safety, designed to render a firearm temporarily inoperable; or

      (2) placed in a securely locked box or container; or

      (3) placed in some other location that a reasonable person would
      believe to be secure from a minor under the age of 14 years.

      (b) Sentence. A person who violates this Section is guilty of a
      Class C misdemeanor and shall be fined not less than \$1,000. A
      second or subsequent violation of this Section is a Class A
      misdemeanor.

      (c) Subsection (a) does not apply:

      (1) if the minor under 14 years of age gains access to a firearm
      and uses it in a lawful act of self-defense or defense of another;
      or

      (2) to any firearm obtained by a minor under the age of 14 because
      of an unlawful entry of the premises by the minor or another
      person.

      (d) For the purposes of this Section, ``firearm'' has the meaning
      ascribed to it in Section 1.1 of the Firearm Owners Identification
      Card Act.

      (Source: P.A. 91-18, eff. 1-1-00.)

      \hypertarget{ilcs-524-9.5}{%
      \subsection*{(720 ILCS 5/24-9.5)}\label{ilcs-524-9.5}}
      \addcontentsline{toc}{subsection}{(720 ILCS 5/24-9.5)}

      \hypertarget{sec.-24-9.5.-handgun-safety-devices.}{%
      \section*{Sec. 24-9.5. Handgun safety
      devices.}\label{sec.-24-9.5.-handgun-safety-devices.}}
      \addcontentsline{toc}{section}{Sec. 24-9.5. Handgun safety
      devices.}

      \markright{Sec. 24-9.5. Handgun safety devices.}

      (a) It is unlawful for a person licensed as a federal firearms
      dealer under Section 923 of the federal Gun Control Act of 1968
      (18 U.S.C. 923) to offer for sale, sell, or transfer a handgun to
      a person not licensed under that Act, unless he or she sells or
      includes with the handgun a device or mechanism, other than the
      firearm safety, designed to render the handgun temporarily
      inoperable or inaccessible. This includes but is not limited to:

      (1) An external device that is:

      (i) attached to the handgun with a key or combination lock; and

      (ii) designed to prevent the handgun from being discharged unless
      the device has been deactivated.

      (2) An integrated mechanical safety, disabling, or locking device
      that is:

      (i) built into the handgun; and

      (ii) designed to prevent the handgun from being discharged unless
      the device has been deactivated.

      (b) Sentence. A person who violates this Section is guilty of a
      Class C misdemeanor and shall be fined not less than \$1,000. A
      second or subsequent violation of this Section is a Class A
      misdemeanor.

      (c) For the purposes of this Section, ``handgun'' has the meaning
      ascribed to it in clause (h)(2) of subsection (A) of Section 24-3
      of this Code.

      (d) This Section does not apply to:

      (1) the purchase, sale, or transportation of a handgun to or by a
      federally licensed firearms dealer or manufacturer that provides
      or services a handgun for:

      (i) personnel of any unit of the federal government;

      (ii) members of the armed forces of the United

      States or the National Guard;

      (iii) law enforcement personnel of the State or any local law
      enforcement agency in the State while acting within the scope of
      their official duties; and

      (iv) an organization that is required by federal law governing its
      specific business or activity to maintain handguns and applicable
      ammunition;

      (2) a firearm modified to be permanently inoperative;

      (3) the sale or transfer of a handgun by a federally licensed
      firearms dealer or manufacturer described in item (1) of this
      subsection (d);

      (4) the sale or transfer of a handgun by a federally licensed
      firearms dealer or manufacturer to a lawful customer outside the
      State; or

      (5) an antique firearm.

      (Source: P.A. 94-390, eff. 1-1-06.)

      \hypertarget{ilcs-524-10}{%
      \subsection*{(720 ILCS 5/24-10)}\label{ilcs-524-10}}
      \addcontentsline{toc}{subsection}{(720 ILCS 5/24-10)}

      \hypertarget{sec.-24-10.-municipal-ordinance-regulating-firearms-affirmative-defense-to-a-violation.}{%
      \section*{Sec. 24-10. Municipal ordinance regulating firearms;
      affirmative defense to a
      violation.}\label{sec.-24-10.-municipal-ordinance-regulating-firearms-affirmative-defense-to-a-violation.}}
      \addcontentsline{toc}{section}{Sec. 24-10. Municipal ordinance
      regulating firearms; affirmative defense to a violation.}

      \markright{Sec. 24-10. Municipal ordinance regulating firearms;
      affirmative defense to a violation.}

      It is an affirmative defense to a violation of a municipal
      ordinance that prohibits, regulates, or restricts the private
      ownership of firearms if the individual who is charged with the
      violation used the firearm in an act of self-defense or defense of
      another as defined in Sections 7-1 and 7-2 of this Code when on
      his or her land or in his or her abode or fixed place of business.

      (Source: P.A. 93-1048, eff. 11-16-04.)
    \end{enumerate}
  \end{enumerate}
\end{enumerate}

\bookmarksetup{startatroot}

\hypertarget{article-24.5.-nitrous-oxide}{%
\chapter*{Article 24.5. Nitrous
Oxide}\label{article-24.5.-nitrous-oxide}}
\addcontentsline{toc}{chapter}{Article 24.5. Nitrous Oxide}

\markboth{Article 24.5. Nitrous Oxide}{Article 24.5. Nitrous Oxide}

\hypertarget{ilcs-524.5-5}{%
\subsection*{(720 ILCS 5/24.5-5)}\label{ilcs-524.5-5}}
\addcontentsline{toc}{subsection}{(720 ILCS 5/24.5-5)}

\hypertarget{sec.-24.5-5.-unlawful-possession.}{%
\section*{Sec. 24.5-5. Unlawful
possession.}\label{sec.-24.5-5.-unlawful-possession.}}
\addcontentsline{toc}{section}{Sec. 24.5-5. Unlawful possession.}

\markright{Sec. 24.5-5. Unlawful possession.}

Any person who possesses nitrous oxide or any substance containing
nitrous oxide, with the intent to breathe, inhale, or ingest for the
purpose of causing a condition of intoxication, elation, euphoria,
dizziness, stupefaction, or dulling of the senses or for the purpose of,
in any manner, changing, distorting, or disturbing the audio, visual, or
mental processes, or who knowingly and with the intent to do so is under
the influence of nitrous oxide or any material containing nitrous oxide
is guilty of a Class A misdemeanor. A person who commits a second or
subsequent violation of this Section is guilty of a Class 4 felony. This
Section shall not apply to any person who is under the influence of
nitrous oxide or any material containing nitrous oxide pursuant to an
administration for the purpose of medical, surgical, or dental care by a
person duly licensed to administer such an agent.

(Source: P.A. 91-366, eff. 1-1-00.)

\hypertarget{ilcs-524.5-10}{%
\subsection*{(720 ILCS 5/24.5-10)}\label{ilcs-524.5-10}}
\addcontentsline{toc}{subsection}{(720 ILCS 5/24.5-10)}

\hypertarget{sec.-24.5-10.-unlawful-manufacture-or-delivery.}{%
\section*{Sec. 24.5-10. Unlawful manufacture or
delivery.}\label{sec.-24.5-10.-unlawful-manufacture-or-delivery.}}
\addcontentsline{toc}{section}{Sec. 24.5-10. Unlawful manufacture or
delivery.}

\markright{Sec. 24.5-10. Unlawful manufacture or delivery.}

Any person, firm, corporation, co-partnership, limited liability
company, or association that intentionally manufactures, delivers, or
possesses with intent to manufacture or deliver nitrous oxide for any
purpose prohibited under Section 24.5-5 is guilty of a Class 3 felony.

(Source: P.A. 91-366, eff. 1-1-00.)

\bookmarksetup{startatroot}

\hypertarget{article-24.6.-lasers-and-laser-pointers}{%
\chapter*{Article 24.6. Lasers And Laser
Pointers}\label{article-24.6.-lasers-and-laser-pointers}}
\addcontentsline{toc}{chapter}{Article 24.6. Lasers And Laser Pointers}

\markboth{Article 24.6. Lasers And Laser Pointers}{Article 24.6. Lasers
And Laser Pointers}

(Repealed)

(Source: P.A. 97-813, eff. 7-13-12. Repealed by P.A. 97-1108, eff.
1-1-13.)

\bookmarksetup{startatroot}

\hypertarget{article-24.8.-air-rifles}{%
\chapter*{Article 24.8. Air Rifles}\label{article-24.8.-air-rifles}}
\addcontentsline{toc}{chapter}{Article 24.8. Air Rifles}

\markboth{Article 24.8. Air Rifles}{Article 24.8. Air Rifles}

(Source: P.A. 97-1108, eff. 1-1-13.)

\hypertarget{ilcs-524.8-0.1}{%
\subsection*{(720 ILCS 5/24.8-0.1)}\label{ilcs-524.8-0.1}}
\addcontentsline{toc}{subsection}{(720 ILCS 5/24.8-0.1)}

\hypertarget{sec.-24.8-0.1.-definitions.}{%
\section*{Sec. 24.8-0.1.
Definitions.}\label{sec.-24.8-0.1.-definitions.}}
\addcontentsline{toc}{section}{Sec. 24.8-0.1. Definitions.}

\markright{Sec. 24.8-0.1. Definitions.}

As used in this Article:

``Air rifle'' means and includes any air gun, air pistol, spring gun,
spring pistol, B-B gun, paint ball gun, pellet gun or any implement that
is not a firearm which impels a breakable paint ball containing washable
marking colors or, a pellet constructed of hard plastic, steel, lead or
other hard materials with a force that reasonably is expected to cause
bodily harm.

``Dealer'' means any person, copartnership, association or corporation
engaged in the business of selling at retail or renting any of the
articles included in the definition of ``air rifle''.

``Municipalities'' include cities, villages, incorporated towns and
townships.

(Source: P.A. 97-1108, eff. 1-1-13.)

\hypertarget{ilcs-524.8-1}{%
\subsection*{(720 ILCS 5/24.8-1)}\label{ilcs-524.8-1}}
\addcontentsline{toc}{subsection}{(720 ILCS 5/24.8-1)}

\hypertarget{sec.-24.8-1.-selling-renting-or-transferring-air-rifles-to-children.}{%
\section*{Sec. 24.8-1. Selling, renting, or transferring air rifles to
children.}\label{sec.-24.8-1.-selling-renting-or-transferring-air-rifles-to-children.}}
\addcontentsline{toc}{section}{Sec. 24.8-1. Selling, renting, or
transferring air rifles to children.}

\markright{Sec. 24.8-1. Selling, renting, or transferring air rifles to
children.}

(a) A dealer commits selling, renting, or transferring air rifles to
children when he or she sells, lends, rents, gives or otherwise
transfers an air rifle to any person under the age of 13 years where the
dealer knows or has cause to believe the person to be under 13 years of
age or where the dealer has failed to make reasonable inquiry relative
to the age of the person and the person is under 13 years of age.

(b) A person commits selling, renting, or transferring air rifles to
children when he or she sells, gives, lends, or otherwise transfers any
air rifle to any person under 13 years of age except where the
relationship of parent and child, guardian and ward or adult instructor
and pupil, exists between this person and the person under 13 years of
age, or where the person stands in loco parentis to the person under 13
years of age.

(Source: P.A. 97-1108, eff. 1-1-13.)

\hypertarget{ilcs-524.8-2}{%
\subsection*{(720 ILCS 5/24.8-2)}\label{ilcs-524.8-2}}
\addcontentsline{toc}{subsection}{(720 ILCS 5/24.8-2)}

\hypertarget{sec.-24.8-2.-carrying-or-discharging-air-rifles-on-public-streets.}{%
\section*{Sec. 24.8-2. Carrying or discharging air rifles on public
streets.}\label{sec.-24.8-2.-carrying-or-discharging-air-rifles-on-public-streets.}}
\addcontentsline{toc}{section}{Sec. 24.8-2. Carrying or discharging air
rifles on public streets.}

\markright{Sec. 24.8-2. Carrying or discharging air rifles on public
streets.}

(a) A person under 13 years of age commits carrying or discharging air
rifles on public streets when he or she carries any air rifle on the
public streets, roads, highways or public lands within this State,
unless the person under 13 years of age carries the air rifle unloaded.

(b) A person commits carrying or discharging air rifles on public
streets when he or she discharges any air rifle from or across any
street, sidewalk, road, highway or public land or any public place
except on a safely constructed target range.

(Source: P.A. 97-1108, eff. 1-1-13.)

\hypertarget{ilcs-524.8-3}{%
\subsection*{(720 ILCS 5/24.8-3)}\label{ilcs-524.8-3}}
\addcontentsline{toc}{subsection}{(720 ILCS 5/24.8-3)}

\hypertarget{sec.-24.8-3.-permissive-possession-of-an-air-rifle-by-a-person-under-13-years-of-age.}{%
\section*{Sec. 24.8-3. Permissive possession of an air rifle by a person
under 13 years of
age.}\label{sec.-24.8-3.-permissive-possession-of-an-air-rifle-by-a-person-under-13-years-of-age.}}
\addcontentsline{toc}{section}{Sec. 24.8-3. Permissive possession of an
air rifle by a person under 13 years of age.}

\markright{Sec. 24.8-3. Permissive possession of an air rifle by a
person under 13 years of age.}

Notwithstanding any provision of this Article, it is lawful for any
person under 13 years of age to have in his or her possession any air
rifle if it is:

(1) Kept within his or her house of residence or other private
enclosure;

(2) Used by the person and he or she is a duly enrolled member of any
club, team or society organized for educational purposes and maintaining
as part of its facilities or having written permission to use an indoor
or outdoor rifle range under the supervision guidance and instruction of
a responsible adult and then only if the air rifle is actually being
used in connection with the activities of the club team or society under
the supervision of a responsible adult; or

(3) Used in or on any private grounds or residence under circumstances
when the air rifle is fired, discharged or operated in a manner as not
to endanger persons or property and then only if it is used in a manner
as to prevent the projectile from passing over any grounds or space
outside the limits of the grounds or residence.

(Source: P.A. 97-1108, eff. 1-1-13.)

\hypertarget{ilcs-524.8-4}{%
\subsection*{(720 ILCS 5/24.8-4)}\label{ilcs-524.8-4}}
\addcontentsline{toc}{subsection}{(720 ILCS 5/24.8-4)}

\hypertarget{sec.-24.8-4.-permissive-sales.}{%
\section*{Sec. 24.8-4. Permissive
sales.}\label{sec.-24.8-4.-permissive-sales.}}
\addcontentsline{toc}{section}{Sec. 24.8-4. Permissive sales.}

\markright{Sec. 24.8-4. Permissive sales.}

The provisions of this Article do not prohibit sales of air rifles:

(1) By wholesale dealers or jobbers;

(2) To be shipped out of the State; or

(3) To be used at a target range operated in accordance with Section
24.8-3 of this Article or by members of the Armed Services of the United
States or Veterans' organizations.

(Source: P.A. 97-1108, eff. 1-1-13.)

\hypertarget{ilcs-524.8-5}{%
\subsection*{(720 ILCS 5/24.8-5)}\label{ilcs-524.8-5}}
\addcontentsline{toc}{subsection}{(720 ILCS 5/24.8-5)}

\hypertarget{sec.-24.8-5.-sentence.}{%
\section*{Sec. 24.8-5. Sentence.}\label{sec.-24.8-5.-sentence.}}
\addcontentsline{toc}{section}{Sec. 24.8-5. Sentence.}

\markright{Sec. 24.8-5. Sentence.}

A violation of this Article is a petty offense. The Illinois State
Police or any sheriff or police officer shall seize, take, remove or
cause to be removed at the expense of the owner, any air rifle sold or
used in any manner in violation of this Article.

(Source: P.A. 102-538, eff. 8-20-21.)

\hypertarget{ilcs-524.8-6}{%
\subsection*{(720 ILCS 5/24.8-6)}\label{ilcs-524.8-6}}
\addcontentsline{toc}{subsection}{(720 ILCS 5/24.8-6)}

\hypertarget{sec.-24.8-6.-municipal-regulation.}{%
\section*{Sec. 24.8-6. Municipal
regulation.}\label{sec.-24.8-6.-municipal-regulation.}}
\addcontentsline{toc}{section}{Sec. 24.8-6. Municipal regulation.}

\markright{Sec. 24.8-6. Municipal regulation.}

The provisions of any ordinance enacted by any municipality which impose
greater restrictions or limitations in respect to the sale and purchase,
use or possession of air rifles as herein defined than are imposed by
this Article, are not invalidated nor affected by this Article.

(Source: P.A. 97-1108, eff. 1-1-13.)

\bookmarksetup{startatroot}

\hypertarget{article-25.-mob-action-and-related-offenses}{%
\chapter*{Article 25. Mob Action And Related
Offenses}\label{article-25.-mob-action-and-related-offenses}}
\addcontentsline{toc}{chapter}{Article 25. Mob Action And Related
Offenses}

\markboth{Article 25. Mob Action And Related Offenses}{Article 25. Mob
Action And Related Offenses}

\hypertarget{ilcs-525-1-from-ch.-38-par.-25-1}{%
\subsection*{(720 ILCS 5/25-1) (from Ch. 38, par.
25-1)}\label{ilcs-525-1-from-ch.-38-par.-25-1}}
\addcontentsline{toc}{subsection}{(720 ILCS 5/25-1) (from Ch. 38, par.
25-1)}

\hypertarget{sec.-25-1.-mob-action.}{%
\section*{Sec. 25-1. Mob action.}\label{sec.-25-1.-mob-action.}}
\addcontentsline{toc}{section}{Sec. 25-1. Mob action.}

\markright{Sec. 25-1. Mob action.}

(a) A person commits mob action when he or she engages in any of the
following:

(1) the knowing or reckless use of force or violence disturbing the
public peace by 2 or more persons acting together and without authority
of law;

(2) the knowing assembly of 2 or more persons with the intent to commit
or facilitate the commission of a felony or misdemeanor; or

(3) the knowing assembly of 2 or more persons, without authority of law,
for the purpose of doing violence to the person or property of anyone
supposed to have been guilty of a violation of the law, or for the
purpose of exercising correctional powers or regulative powers over any
person by violence.

(b) Sentence.

(1) Mob action in violation of paragraph (1) of subsection (a) is a
Class 4 felony.

(2) Mob action in violation of paragraphs (2) and (3) of subsection (a)
is a Class C misdemeanor.

(3) A participant in a mob action that by violence inflicts injury to
the person or property of another commits a Class 4 felony.

(4) A participant in a mob action who does not withdraw when commanded
to do so by a peace officer commits a Class A misdemeanor.

(5) In addition to any other sentence that may be imposed, a court shall
order any person convicted of mob action to perform community service
for not less than 30 and not more than 120 hours, if community service
is available in the jurisdiction and is funded and approved by the
county board of the county where the offense was committed. In addition,
whenever any person is placed on supervision for an alleged offense
under this Section, the supervision shall be conditioned upon the
performance of the community service. This paragraph does not apply when
the court imposes a sentence of incarceration.

(Source: P.A. 96-710, eff. 1-1-10; 97-1108, eff. 1-1-13.)

\hypertarget{ilcs-525-1.1}{%
\subsection*{(720 ILCS 5/25-1.1)}\label{ilcs-525-1.1}}
\addcontentsline{toc}{subsection}{(720 ILCS 5/25-1.1)}

\hypertarget{sec.-25-1.1.-renumbered.}{%
\section*{Sec. 25-1.1. (Renumbered).}\label{sec.-25-1.1.-renumbered.}}
\addcontentsline{toc}{section}{Sec. 25-1.1. (Renumbered).}

\markright{Sec. 25-1.1. (Renumbered).}

(Source: Renumbered by P.A. 96-710, eff. 1-1-10.)

\hypertarget{ilcs-525-2-from-ch.-38-par.-25-2}{%
\subsection*{(720 ILCS 5/25-2) (from Ch. 38, par.
25-2)}\label{ilcs-525-2-from-ch.-38-par.-25-2}}
\addcontentsline{toc}{subsection}{(720 ILCS 5/25-2) (from Ch. 38, par.
25-2)}

\hypertarget{sec.-25-2.-renumbered.}{%
\section*{Sec. 25-2. (Renumbered).}\label{sec.-25-2.-renumbered.}}
\addcontentsline{toc}{section}{Sec. 25-2. (Renumbered).}

\markright{Sec. 25-2. (Renumbered).}

(Source: Renumbered by P.A. 96-710, eff. 1-1-10.)

\hypertarget{ilcs-525-4}{%
\subsection*{(720 ILCS 5/25-4)}\label{ilcs-525-4}}
\addcontentsline{toc}{subsection}{(720 ILCS 5/25-4)}

\hypertarget{sec.-25-4.-looting-by-individuals.}{%
\section*{Sec. 25-4. Looting by
individuals.}\label{sec.-25-4.-looting-by-individuals.}}
\addcontentsline{toc}{section}{Sec. 25-4. Looting by individuals.}

\markright{Sec. 25-4. Looting by individuals.}

(a) A person commits looting when he or she knowingly without authority
of law or the owner enters any home or dwelling or upon any premises of
another, or enters any commercial, mercantile, business, or industrial
building, plant, or establishment, in which normal security of property
is not present by virtue of a hurricane, fire, or vis major of any kind
or by virtue of a riot, mob, or other human agency, and obtains or
exerts control over property of the owner.

(b) Sentence. Looting is a Class 4 felony. In addition to any other
penalty imposed, the court shall impose a sentence of at least 100 hours
of community service as determined by the court and shall require the
defendant to make restitution to the owner of the property looted
pursuant to Section 5-5-6 of the Unified Code of Corrections.

(Source: P.A. 96-710, eff. 1-1-10; 97-1108, eff. 1-1-13.)

\hypertarget{ilcs-525-5}{%
\subsection*{(720 ILCS 5/25-5)}\label{ilcs-525-5}}
\addcontentsline{toc}{subsection}{(720 ILCS 5/25-5)}

(was 720 ILCS 5/25-1.1)

\hypertarget{sec.-25-5.-unlawful-participation-in-streetgang-related-activity.}{%
\section*{Sec. 25-5. Unlawful participation in streetgang related
activity.}\label{sec.-25-5.-unlawful-participation-in-streetgang-related-activity.}}
\addcontentsline{toc}{section}{Sec. 25-5. Unlawful participation in
streetgang related activity.}

\markright{Sec. 25-5. Unlawful participation in streetgang related
activity.}

(a) A person commits unlawful participation in streetgang related
activity when he or she knowingly commits any act in furtherance of
streetgang related activity as defined in Section 10 of the Illinois
Streetgang Terrorism Omnibus Prevention Act after having been:

(1) sentenced to probation, conditional discharge, or supervision for a
criminal offense with a condition of that sentence being to refrain from
direct or indirect contact with a streetgang member or members;

(2) released on bond for any criminal offense with a condition of that
bond being to refrain from direct or indirect contact with a streetgang
member or members;

(3) ordered by a judge in any non-criminal proceeding to refrain from
direct or indirect contact with a streetgang member or members; or

(4) released from the Illinois Department of

Corrections on a condition of parole or mandatory supervised release
that he or she refrain from direct or indirect contact with a streetgang
member or members.

(b) Unlawful participation in streetgang related activity is a Class A
misdemeanor.

(c) (Blank).

(Source: P.A. 100-279, eff. 1-1-18 .)

\hypertarget{ilcs-525-6}{%
\subsection*{(720 ILCS 5/25-6)}\label{ilcs-525-6}}
\addcontentsline{toc}{subsection}{(720 ILCS 5/25-6)}

(was 720 ILCS 5/25-2)

\hypertarget{sec.-25-6.-removal-of-chief-of-police-or-sheriff-for-allowing-a-person-in-his-or-her-custody-to-be-lynched.}{%
\section*{Sec. 25-6. Removal of chief of police or sheriff for allowing
a person in his or her custody to be
lynched.}\label{sec.-25-6.-removal-of-chief-of-police-or-sheriff-for-allowing-a-person-in-his-or-her-custody-to-be-lynched.}}
\addcontentsline{toc}{section}{Sec. 25-6. Removal of chief of police or
sheriff for allowing a person in his or her custody to be lynched.}

\markright{Sec. 25-6. Removal of chief of police or sheriff for allowing
a person in his or her custody to be lynched.}

(a) If a prisoner is taken from the custody of any policeman or chief of
police of any municipality and lynched, it shall be prima facie evidence
of wrong-doing on the part of that chief of police and he or she shall
be suspended. The mayor or chief executive of the municipality shall
appoint an acting chief of police until he or she has ascertained
whether the suspended chief of police had done all in his or her power
to protect the life of the prisoner. If, upon hearing all evidence and
argument, the mayor or chief executive finds that the chief of police
had done his or her utmost to protect the prisoner, he or she may
reinstate the chief of police; but, if he or she finds the chief of
police guilty of not properly protecting the prisoner, a new chief of
police shall be appointed. Any chief of police replaced is not be
eligible to serve again in that office.

(b) If a prisoner is taken from the custody of any sheriff or his or her
deputy and lynched, it is prima facie evidence of wrong-doing on the
part of that sheriff and he or she shall be suspended. The Governor
shall appoint an acting sheriff until he or she has ascertained whether
the suspended sheriff had done all in his or her power to protect the
life of the prisoner. If, upon hearing all evidence and argument, the
Governor finds that the sheriff had done his or her utmost to protect
the prisoner, he or she shall reinstate the sheriff; but, if he or she
finds the sheriff guilty of not properly protecting the prisoner, a new
sheriff shall be duly elected or appointed, pursuant to the existing law
provided for the filling of vacancies in that office. Any sheriff
replaced is not eligible to serve again in that office.

(Source: P.A. 96-710, eff. 1-1-10.)

\bookmarksetup{startatroot}

\hypertarget{article-26.-disorderly-conduct}{%
\chapter*{Article 26. Disorderly
Conduct}\label{article-26.-disorderly-conduct}}
\addcontentsline{toc}{chapter}{Article 26. Disorderly Conduct}

\markboth{Article 26. Disorderly Conduct}{Article 26. Disorderly
Conduct}

\hypertarget{ilcs-526-1-from-ch.-38-par.-26-1}{%
\subsection*{(720 ILCS 5/26-1) (from Ch. 38, par.
26-1)}\label{ilcs-526-1-from-ch.-38-par.-26-1}}
\addcontentsline{toc}{subsection}{(720 ILCS 5/26-1) (from Ch. 38, par.
26-1)}

\hypertarget{sec.-26-1.-disorderly-conduct.}{%
\section*{Sec. 26-1. Disorderly
conduct.}\label{sec.-26-1.-disorderly-conduct.}}
\addcontentsline{toc}{section}{Sec. 26-1. Disorderly conduct.}

\markright{Sec. 26-1. Disorderly conduct.}

(a) A person commits disorderly conduct when he or she knowingly:

(1) Does any act in such unreasonable manner as to alarm or disturb
another and to provoke a breach of the peace;

(2) Transmits or causes to be transmitted in any manner to the fire
department of any city, town, village or fire protection district a
false alarm of fire, knowing at the time of the transmission that there
is no reasonable ground for believing that the fire exists;

(3) Transmits or causes to be transmitted in any manner to another a
false alarm to the effect that a bomb or other explosive of any nature
or a container holding poison gas, a deadly biological or chemical
contaminant, or radioactive substance is concealed in a place where its
explosion or release would endanger human life, knowing at the time of
the transmission that there is no reasonable ground for believing that
the bomb, explosive or a container holding poison gas, a deadly
biological or chemical contaminant, or radioactive substance is
concealed in the place;

(3.5) Transmits or causes to be transmitted in any manner a threat of
destruction of a school building or school property, or a threat of
violence, death, or bodily harm directed against persons at a school,
school function, or school event, whether or not school is in session;

(4) Transmits or causes to be transmitted in any manner to any peace
officer, public officer or public employee a report to the effect that
an offense will be committed, is being committed, or has been committed,
knowing at the time of the transmission that there is no reasonable
ground for believing that the offense will be committed, is being
committed, or has been committed;

(5) Transmits or causes to be transmitted in any manner a false report
to any public safety agency without the reasonable grounds necessary to
believe that transmitting the report is necessary for the safety and
welfare of the public; or

(6) Calls the number ``911'' or transmits or causes to be transmitted in
any manner to a public safety agency for the purpose of making or
transmitting a false alarm or complaint and reporting information when,
at the time the call or transmission is made, the person knows there is
no reasonable ground for making the call or transmission and further
knows that the call or transmission could result in the emergency
response of any public safety agency;

(7) Transmits or causes to be transmitted in any manner a false report
to the Department of Children and Family Services under Section 4 of the
Abused and Neglected Child Reporting Act;

(8) Transmits or causes to be transmitted in any manner a false report
to the Department of Public Health under the Nursing Home Care Act, the
Specialized Mental Health Rehabilitation Act of 2013, the ID/DD
Community Care Act, or the MC/DD Act;

(9) Transmits or causes to be transmitted in any manner to the police
department or fire department of any municipality or fire protection
district, or any privately owned and operated ambulance service, a false
request for an ambulance, emergency medical technician-ambulance or
emergency medical technician-paramedic knowing at the time there is no
reasonable ground for believing that the assistance is required;

(10) Transmits or causes to be transmitted in any manner a false report
under Article II of Public Act 83-1432;

(11) Enters upon the property of another and for a lewd or unlawful
purpose deliberately looks into a dwelling on the property through any
window or other opening in it; or

(12) While acting as a collection agency as defined in the Collection
Agency Act or as an employee of the collection agency, and while
attempting to collect an alleged debt, makes a telephone call to the
alleged debtor which is designed to harass, annoy or intimidate the
alleged debtor.

(b) Sentence. A violation of subsection (a)(1) of this Section is a
Class C misdemeanor. A violation of subsection (a)(5) or (a)(11) of this
Section is a Class A misdemeanor. A violation of subsection (a)(8) or
(a)(10) of this Section is a Class B misdemeanor. A violation of
subsection (a)(2), (a)(3.5), (a)(4), (a)(6), (a)(7), or (a)(9) of this
Section is a Class 4 felony. A violation of subsection (a)(3) of this
Section is a Class 3 felony, for which a fine of not less than \$3,000
and no more than \$10,000 shall be assessed in addition to any other
penalty imposed.

A violation of subsection (a)(12) of this Section is a Business Offense
and shall be punished by a fine not to exceed \$3,000. A second or
subsequent violation of subsection (a)(7) or (a)(5) of this Section is a
Class 4 felony. A third or subsequent violation of subsection (a)(11) of
this Section is a Class 4 felony.

(c) In addition to any other sentence that may be imposed, a court shall
order any person convicted of disorderly conduct to perform community
service for not less than 30 and not more than 120 hours, if community
service is available in the jurisdiction and is funded and approved by
the county board of the county where the offense was committed. In
addition, whenever any person is placed on supervision for an alleged
offense under this Section, the supervision shall be conditioned upon
the performance of the community service.

This subsection does not apply when the court imposes a sentence of
incarceration.

(d) In addition to any other sentence that may be imposed, the court
shall order any person convicted of disorderly conduct under paragraph
(3) of subsection (a) involving a false alarm of a threat that a bomb or
explosive device has been placed in a school that requires an emergency
response to reimburse the unit of government that employs the emergency
response officer or officers that were dispatched to the school for the
cost of the response. If the court determines that the person convicted
of disorderly conduct that requires an emergency response to a school is
indigent, the provisions of this subsection (d) do not apply.

(e) In addition to any other sentence that may be imposed, the court
shall order any person convicted of disorderly conduct under paragraph
(3.5) or (6) of subsection (a) to reimburse the public agency for the
reasonable costs of the emergency response by the public agency up to
\$10,000. If the court determines that the person convicted of
disorderly conduct under paragraph (3.5) or (6) of subsection (a) is
indigent, the provisions of this subsection (e) do not apply.

(f) For the purposes of this Section, ``emergency response'' means any
condition that results in, or could result in, the response of a public
official in an authorized emergency vehicle, any condition that
jeopardizes or could jeopardize public safety and results in, or could
result in, the evacuation of any area, building, structure, vehicle, or
of any other place that any person may enter, or any incident requiring
a response by a police officer, a firefighter, a State Fire Marshal
employee, or an ambulance.

(Source: P.A. 101-238, eff. 1-1-20 .)

\hypertarget{ilcs-526-1.1}{%
\subsection*{(720 ILCS 5/26-1.1)}\label{ilcs-526-1.1}}
\addcontentsline{toc}{subsection}{(720 ILCS 5/26-1.1)}

\hypertarget{sec.-26-1.1.-false-report-of-theft-and-other-losses.}{%
\section*{Sec. 26-1.1. False report of theft and other
losses.}\label{sec.-26-1.1.-false-report-of-theft-and-other-losses.}}
\addcontentsline{toc}{section}{Sec. 26-1.1. False report of theft and
other losses.}

\markright{Sec. 26-1.1. False report of theft and other losses.}

(a) A person who knowingly makes a false report of a theft, destruction,
damage or conversion of any property to a law enforcement agency or
other governmental agency with the intent to defraud an insurer is
guilty of a Class A misdemeanor.

(b) A person convicted of a violation of this Section a second or
subsequent time is guilty of a Class 4 felony.

(Source: P.A. 97-597, eff. 1-1-12.)

\hypertarget{ilcs-526-2-from-ch.-38-par.-26-2}{%
\subsection*{(720 ILCS 5/26-2) (from Ch. 38, par.
26-2)}\label{ilcs-526-2-from-ch.-38-par.-26-2}}
\addcontentsline{toc}{subsection}{(720 ILCS 5/26-2) (from Ch. 38, par.
26-2)}

\hypertarget{sec.-26-2.-interference-with-emergency-communication.}{%
\section*{Sec. 26-2. Interference with emergency
communication.}\label{sec.-26-2.-interference-with-emergency-communication.}}
\addcontentsline{toc}{section}{Sec. 26-2. Interference with emergency
communication.}

\markright{Sec. 26-2. Interference with emergency communication.}

(a) A person commits interference with emergency communication when he
or she knowingly, intentionally and without lawful justification
interrupts, disrupts, impedes, or otherwise interferes with the
transmission of a communication over a citizens band radio channel, the
purpose of which communication is to inform or inquire about an
emergency.

(b) For the purpose of this Section, ``emergency'' means a condition or
circumstance in which an individual is or is reasonably believed by the
person transmitting the communication to be in imminent danger of
serious bodily injury or in which property is or is reasonably believed
by the person transmitting the communication to be in imminent danger of
damage or destruction.

(c) Sentence.

(1) Interference with emergency communication is a Class B misdemeanor,
except as otherwise provided in paragraph (2).

(2) Interference with emergency communication, where serious bodily
injury or property loss in excess of \$1,000 results, is a Class A
misdemeanor.

(Source: P.A. 97-1108, eff. 1-1-13.)

\hypertarget{ilcs-526-3-from-ch.-38-par.-26-3}{%
\subsection*{(720 ILCS 5/26-3) (from Ch. 38, par.
26-3)}\label{ilcs-526-3-from-ch.-38-par.-26-3}}
\addcontentsline{toc}{subsection}{(720 ILCS 5/26-3) (from Ch. 38, par.
26-3)}

\hypertarget{sec.-26-3.-use-of-a-facsimile-machine-in-unsolicited-advertising-or-fund-raising.}{%
\section*{Sec. 26-3. Use of a facsimile machine in unsolicited
advertising or
fund-raising.}\label{sec.-26-3.-use-of-a-facsimile-machine-in-unsolicited-advertising-or-fund-raising.}}
\addcontentsline{toc}{section}{Sec. 26-3. Use of a facsimile machine in
unsolicited advertising or fund-raising.}

\markright{Sec. 26-3. Use of a facsimile machine in unsolicited
advertising or fund-raising.}

(a) Definitions:

(1) ``Facsimile machine'' means a device which is capable of sending or
receiving facsimiles of documents through connection with a
telecommunications network.

(2) ``Person'' means an individual, public or private corporation, unit
of government, partnership or unincorporated association.

(b) A person commits use of a facsimile machine in unsolicited
advertising or fund-raising when he or she knowingly uses a facsimile
machine to send or cause to be sent to another person a facsimile of a
document containing unsolicited advertising or fund-raising material,
except to a person which the sender knows or under all of the
circumstances reasonably believes has given the sender permission,
either on a case by case or continuing basis, for the sending of the
material.

(c) Sentence. Any person who violates subsection (b) is guilty of a
petty offense and shall be fined an amount not to exceed \$500.

(Source: P.A. 97-1108, eff. 1-1-13.)

\hypertarget{ilcs-526-4-from-ch.-38-par.-26-4}{%
\subsection*{(720 ILCS 5/26-4) (from Ch. 38, par.
26-4)}\label{ilcs-526-4-from-ch.-38-par.-26-4}}
\addcontentsline{toc}{subsection}{(720 ILCS 5/26-4) (from Ch. 38, par.
26-4)}

\hypertarget{sec.-26-4.-unauthorized-video-recording-and-live-video-transmission.}{%
\section*{Sec. 26-4. Unauthorized video recording and live video
transmission.}\label{sec.-26-4.-unauthorized-video-recording-and-live-video-transmission.}}
\addcontentsline{toc}{section}{Sec. 26-4. Unauthorized video recording
and live video transmission.}

\markright{Sec. 26-4. Unauthorized video recording and live video
transmission.}

(a) It is unlawful for any person to knowingly make a video record or
transmit live video of another person without that person's consent in a
restroom, tanning bed, tanning salon, locker room, changing room, or
hotel bedroom.

(a-5) It is unlawful for any person to knowingly make a video record or
transmit live video of another person in that other person's residence
without that person's consent.

(a-6) It is unlawful for any person to knowingly make a video record or
transmit live video of another person in that other person's residence
without that person's consent when the recording or transmission is made
outside that person's residence by use of an audio or video device that
records or transmits from a remote location.

(a-10) It is unlawful for any person to knowingly make a video record or
transmit live video of another person's intimate parts for the purpose
of viewing the body of or the undergarments worn by that other person
without that person's consent. For the purposes of this subsection
(a-10), ``intimate parts'' means the fully unclothed, partially
unclothed, or transparently clothed genitals, pubic area, anus, or if
the person is female, a partially or fully exposed nipple, including
exposure through transparent clothing.

(a-15) It is unlawful for any person to place or cause to be placed a
device that makes a video record or transmits a live video in a
restroom, tanning bed, tanning salon, locker room, changing room, or
hotel bedroom with the intent to make a video record or transmit live
video of another person without that person's consent.

(a-20) It is unlawful for any person to place or cause to be placed a
device that makes a video record or transmits a live video with the
intent to make a video record or transmit live video of another person
in that other person's residence without that person's consent.

(a-25) It is unlawful for any person to, by any means, knowingly
disseminate, or permit to be disseminated, a video record or live video
that he or she knows to have been made or transmitted in violation of
(a), (a-5), (a-6), (a-10), (a-15), or (a-20).

(b) Exemptions. The following activities shall be exempt from the
provisions of this Section:

(1) The making of a video record or transmission of live video by law
enforcement officers pursuant to a criminal investigation, which is
otherwise lawful;

(2) The making of a video record or transmission of live video by
correctional officials for security reasons or for investigation of
alleged misconduct involving a person committed to the Department of
Corrections; and

(3) The making of a video record or transmission of live video in a
locker room by a reporter or news medium, as those terms are defined in
Section 8-902 of the Code of Civil Procedure, where the reporter or news
medium has been granted access to the locker room by an appropriate
authority for the purpose of conducting interviews.

(c) The provisions of this Section do not apply to any sound recording
or transmission of an oral conversation made as the result of the making
of a video record or transmission of live video, and to which Article 14
of this Code applies.

(d) Sentence.

(1) A violation of subsection (a-15) or (a-20) is a Class A misdemeanor.

(2) A violation of subsection (a), (a-5), (a-6), or

(a-10) is a Class 4 felony.

(3) A violation of subsection (a-25) is a Class 3 felony.

(4) A violation of subsection (a), (a-5), (a-6),

(a-10), (a-15) or (a-20) is a Class 3 felony if the victim is a person
under 18 years of age or if the violation is committed by an individual
who is required to register as a sex offender under the Sex Offender
Registration Act.

(5) A violation of subsection (a-25) is a Class 2 felony if the victim
is a person under 18 years of age or if the violation is committed by an
individual who is required to register as a sex offender under the Sex
Offender Registration Act.

(e) For purposes of this Section:

(1) ``Residence'' includes a rental dwelling, but does not include
stairwells, corridors, laundry facilities, or additional areas in which
the general public has access.

(2) ``Video record'' means and includes any videotape, photograph, film,
or other electronic or digital recording of a still or moving visual
image; and ``live video'' means and includes any real-time or
contemporaneous electronic or digital transmission of a still or moving
visual image.

(Source: P.A. 102-567, eff. 1-1-22 .)

\hypertarget{ilcs-526-4.5}{%
\subsection*{(720 ILCS 5/26-4.5)}\label{ilcs-526-4.5}}
\addcontentsline{toc}{subsection}{(720 ILCS 5/26-4.5)}

\hypertarget{sec.-26-4.5.-consumer-communications-privacy.}{%
\section*{Sec. 26-4.5. Consumer communications
privacy.}\label{sec.-26-4.5.-consumer-communications-privacy.}}
\addcontentsline{toc}{section}{Sec. 26-4.5. Consumer communications
privacy.}

\markright{Sec. 26-4.5. Consumer communications privacy.}

(a) For purposes of this Section, ``communications company'' means any
person or organization which owns, controls, operates or manages any
company which provides information or entertainment electronically to a
household, including but not limited to a cable or community antenna
television system.

(b) It shall be unlawful for a communications company to:

(1) install and use any equipment which would allow a communications
company to visually observe or listen to what is occurring in an
individual subscriber's household without the knowledge or permission of
the subscriber;

(2) provide any person or public or private organization with a list
containing the name of a subscriber, unless the communications company
gives notice thereof to the subscriber;

(3) disclose the television viewing habits of any individual subscriber
without the subscriber's consent; or

(4) install or maintain a home-protection scanning device in a dwelling
as part of a communication service without the express written consent
of the occupant.

(c) Sentence. A violation of this Section is a business offense,
punishable by a fine not to exceed \$10,000 for each violation.

(d) Civil liability. Any person who has been injured by a violation of
this Section may commence an action in the circuit court for damages
against any communications company which has committed a violation. If
the court awards damages, the plaintiff shall be awarded costs.

(Source: P.A. 97-1108, eff. 1-1-13.)

\hypertarget{ilcs-526-5}{%
\subsection*{(720 ILCS 5/26-5)}\label{ilcs-526-5}}
\addcontentsline{toc}{subsection}{(720 ILCS 5/26-5)}

(This Section was renumbered as Section 48-1 by P.A. 97-1108.)

\hypertarget{sec.-26-5.-renumbered.}{%
\section*{Sec. 26-5. (Renumbered).}\label{sec.-26-5.-renumbered.}}
\addcontentsline{toc}{section}{Sec. 26-5. (Renumbered).}

\markright{Sec. 26-5. (Renumbered).}

(Source: P.A. 96-226, eff. 8-11-09; 96-712, eff. 1-1-10; 96-1000, eff.
7-2-10; 96-1091, eff. 1-1-11. Renumbered by P.A. 97-1108, eff. 1-1-13.)

\hypertarget{ilcs-526-6}{%
\subsection*{(720 ILCS 5/26-6)}\label{ilcs-526-6}}
\addcontentsline{toc}{subsection}{(720 ILCS 5/26-6)}

\hypertarget{sec.-26-6.-disorderly-conduct-at-a-funeral-or-memorial-service.}{%
\section*{Sec. 26-6. Disorderly conduct at a funeral or memorial
service.}\label{sec.-26-6.-disorderly-conduct-at-a-funeral-or-memorial-service.}}
\addcontentsline{toc}{section}{Sec. 26-6. Disorderly conduct at a
funeral or memorial service.}

\markright{Sec. 26-6. Disorderly conduct at a funeral or memorial
service.}

(a) The General Assembly finds and declares that due to the unique
nature of funeral and memorial services and the heightened opportunity
for extreme emotional distress on such occasions, the purpose of this
Section is to protect the privacy and ability to mourn of grieving
families directly before, during, and after a funeral or memorial
service.

(b) For purposes of this Section:

(1) ``Funeral'' means the ceremonies, rituals, processions, and memorial
services held at a funeral site in connection with the burial,
cremation, or memorial of a deceased person.

(2) ``Funeral site'' means a church, synagogue, mosque, funeral home,
mortuary, cemetery, gravesite, mausoleum, or other place at which a
funeral is conducted or is scheduled to be conducted within the next 30
minutes or has been conducted within the last 30 minutes.

(c) A person commits the offense of disorderly conduct at a funeral or
memorial service when he or she:

(1) engages, with knowledge of the existence of a funeral site, in any
loud singing, playing of music, chanting, whistling, yelling, or
noisemaking with, or without, noise amplification including, but not
limited to, bullhorns, auto horns, and microphones within 300 feet of
any ingress or egress of that funeral site, where the volume of such
singing, music, chanting, whistling, yelling, or noisemaking is likely
to be audible at and disturbing to the funeral site;

(2) displays, with knowledge of the existence of a funeral site and
within 300 feet of any ingress or egress of that funeral site, any
visual images that convey fighting words or actual or veiled threats
against any other person; or

(3) with knowledge of the existence of a funeral site, knowingly
obstructs, hinders, impedes, or blocks another person's entry to or exit
from that funeral site or a facility containing that funeral site,
except that the owner or occupant of property may take lawful actions to
exclude others from that property.

(d) Disorderly conduct at a funeral or memorial service is a Class C
misdemeanor. A second or subsequent violation is a Class 4 felony.

(e) If any clause, sentence, section, provision, or part of this Section
or the application thereof to any person or circumstance is adjudged to
be unconstitutional, the remainder of this Section or its application to
persons or circumstances other than those to which it is held invalid,
is not affected thereby.

(Source: P.A. 97-359, eff. 8-15-11.)

\hypertarget{ilcs-526-7}{%
\subsection*{(720 ILCS 5/26-7)}\label{ilcs-526-7}}
\addcontentsline{toc}{subsection}{(720 ILCS 5/26-7)}

\hypertarget{sec.-26-7.-disorderly-conduct-with-a-laser-or-laser-pointer.}{%
\section*{Sec. 26-7. Disorderly conduct with a laser or laser
pointer.}\label{sec.-26-7.-disorderly-conduct-with-a-laser-or-laser-pointer.}}
\addcontentsline{toc}{section}{Sec. 26-7. Disorderly conduct with a
laser or laser pointer.}

\markright{Sec. 26-7. Disorderly conduct with a laser or laser pointer.}

(a) Definitions. For the purposes of this Section:

``Aircraft'' means any contrivance now known or hereafter invented,
used, or designed for navigation of or flight in the air, but excluding
parachutes.

``Laser'' means both of the following:

(1) any device that utilizes the natural oscillations of atoms or
molecules between energy levels for generating coherent electromagnetic
radiation in the ultraviolet, visible, or infrared region of the
spectrum and when discharged exceeds one milliwatt continuous wave;

(2) any device designed or used to amplify electromagnetic radiation by
simulated emission that is visible to the human eye.

``Laser pointer'' means a hand-held device that emits light amplified by
the stimulated emission of radiation that is visible to the human eye.

``Laser sight'' means a laser pointer that can be attached to a firearm
and can be used to improve the accuracy of the firearm.

(b) A person commits disorderly conduct with a laser or laser pointer
when he or she intentionally or knowingly:

(1) aims an operating laser pointer at a person he or she knows or
reasonably should know to be a peace officer; or

(2) aims and discharges a laser or other device that creates visible
light into the cockpit of an aircraft that is in the process of taking
off, landing, or is in flight.

(c) Paragraph (2) of subsection (b) does not apply to the following
individuals who aim and discharge a laser or other device at an
aircraft:

(1) an authorized individual in the conduct of research and development
or flight test operations conducted by an aircraft manufacturer, the
Federal Aviation Administration, or any other person authorized by the
Federal Aviation Administration to conduct this research and development
or flight test operations; or

(2) members or elements of the Department of Defense or Department of
Homeland Security acting in an official capacity for the purpose of
research, development, operations, testing, or training.

(d) Sentence. Disorderly conduct with a laser or laser pointer is a
Class A misdemeanor.

(Source: P.A. 97-1108, eff. 1-1-13.)

\bookmarksetup{startatroot}

\hypertarget{article-26.5.-harassing-and-obscene-communications}{%
\chapter*{Article 26.5. Harassing And Obscene
Communications}\label{article-26.5.-harassing-and-obscene-communications}}
\addcontentsline{toc}{chapter}{Article 26.5. Harassing And Obscene
Communications}

\markboth{Article 26.5. Harassing And Obscene Communications}{Article
26.5. Harassing And Obscene Communications}

(Source: P.A. 97-1108, eff. 1-1-13.)

\hypertarget{ilcs-526.5-0.1}{%
\subsection*{(720 ILCS 5/26.5-0.1)}\label{ilcs-526.5-0.1}}
\addcontentsline{toc}{subsection}{(720 ILCS 5/26.5-0.1)}

\hypertarget{sec.-26.5-0.1.-definitions.}{%
\section*{Sec. 26.5-0.1.
Definitions.}\label{sec.-26.5-0.1.-definitions.}}
\addcontentsline{toc}{section}{Sec. 26.5-0.1. Definitions.}

\markright{Sec. 26.5-0.1. Definitions.}

As used in this Article:

``Electronic communication'' means any transfer of signs, signals,
writings, images, sounds, data or intelligence of any nature transmitted
in whole or in part by a wire, radio, electromagnetic, photoelectric or
photo-optical system. ``Electronic communication'' includes
transmissions through an electronic device including, but not limited
to, a telephone, cellular phone, computer, or pager, which communication
includes, but is not limited to, e-mail, instant message, text message,
or voice mail.

``Family or household member'' includes spouses, former spouses,
parents, children, stepchildren and other persons related by blood or by
present or prior marriage, persons who share or formerly shared a common
dwelling, persons who have or allegedly share a blood relationship
through a child, persons who have or have had a dating or engagement
relationship, and persons with disabilities and their personal
assistants. For purposes of this Article, neither a casual
acquaintanceship nor ordinary fraternization between 2 individuals in
business or social contexts shall be deemed to constitute a dating
relationship.

``Harass'' or ``harassing'' means knowing conduct which is not necessary
to accomplish a purpose that is reasonable under the circumstances, that
would cause a reasonable person emotional distress and does cause
emotional distress to another.

(Source: P.A. 97-1108, eff. 1-1-13.)

\hypertarget{ilcs-526.5-1}{%
\subsection*{(720 ILCS 5/26.5-1)}\label{ilcs-526.5-1}}
\addcontentsline{toc}{subsection}{(720 ILCS 5/26.5-1)}

\hypertarget{sec.-26.5-1.-transmission-of-obscene-messages.}{%
\section*{Sec. 26.5-1. Transmission of obscene
messages.}\label{sec.-26.5-1.-transmission-of-obscene-messages.}}
\addcontentsline{toc}{section}{Sec. 26.5-1. Transmission of obscene
messages.}

\markright{Sec. 26.5-1. Transmission of obscene messages.}

(a) A person commits transmission of obscene messages when he or she
sends messages or uses language or terms which are obscene, lewd or
immoral with the intent to offend by means of or while using a telephone
or telegraph facilities, equipment or wires of any person, firm or
corporation engaged in the transmission of news or messages between
states or within the State of Illinois.

(b) The trier of fact may infer intent to offend from the use of
language or terms which are obscene, lewd or immoral.

(Source: P.A. 97-1108, eff. 1-1-13.)

\hypertarget{ilcs-526.5-2}{%
\subsection*{(720 ILCS 5/26.5-2)}\label{ilcs-526.5-2}}
\addcontentsline{toc}{subsection}{(720 ILCS 5/26.5-2)}

\hypertarget{sec.-26.5-2.-harassment-by-telephone.}{%
\section*{Sec. 26.5-2. Harassment by
telephone.}\label{sec.-26.5-2.-harassment-by-telephone.}}
\addcontentsline{toc}{section}{Sec. 26.5-2. Harassment by telephone.}

\markright{Sec. 26.5-2. Harassment by telephone.}

(a) A person commits harassment by telephone when he or she uses
telephone communication for any of the following purposes:

(1) Making any comment, request, suggestion or proposal which is
obscene, lewd, lascivious, filthy or indecent with an intent to offend;

(2) Making a telephone call, whether or not conversation ensues, with
intent to abuse, threaten or harass any person at the called number;

(3) Making or causing the telephone of another repeatedly to ring, with
intent to harass any person at the called number;

(4) Making repeated telephone calls, during which conversation ensues,
solely to harass any person at the called number;

(5) Making a telephone call or knowingly inducing a person to make a
telephone call for the purpose of harassing another person who is under
13 years of age, regardless of whether the person under 13 years of age
consents to the harassment, if the defendant is at least 16 years of age
at the time of the commission of the offense; or

(6) Knowingly permitting any telephone under one's control to be used
for any of the purposes mentioned herein.

(b) Every telephone directory published for distribution to members of
the general public shall contain a notice setting forth a summary of the
provisions of this Section. The notice shall be printed in type which is
no smaller than any other type on the same page and shall be preceded by
the word ``WARNING''. All telephone companies in this State shall
cooperate with law enforcement agencies in using their facilities and
personnel to detect and prevent violations of this Article.

(Source: P.A. 97-1108, eff. 1-1-13.)

\hypertarget{ilcs-526.5-3}{%
\subsection*{(720 ILCS 5/26.5-3)}\label{ilcs-526.5-3}}
\addcontentsline{toc}{subsection}{(720 ILCS 5/26.5-3)}

\hypertarget{sec.-26.5-3.-harassment-through-electronic-communications.}{%
\section*{Sec. 26.5-3. Harassment through electronic
communications.}\label{sec.-26.5-3.-harassment-through-electronic-communications.}}
\addcontentsline{toc}{section}{Sec. 26.5-3. Harassment through
electronic communications.}

\markright{Sec. 26.5-3. Harassment through electronic communications.}

(a) A person commits harassment through electronic communications when
he or she uses electronic communication for any of the following
purposes:

(1) Making any comment, request, suggestion or proposal which is obscene
with an intent to offend;

(2) Interrupting, with the intent to harass, the telephone service or
the electronic communication service of any person;

(3) Transmitting to any person, with the intent to harass and regardless
of whether the communication is read in its entirety or at all, any
file, document, or other communication which prevents that person from
using his or her telephone service or electronic communications device;

(4) Transmitting an electronic communication or knowingly inducing a
person to transmit an electronic communication for the purpose of
harassing another person who is under 13 years of age, regardless of
whether the person under 13 years of age consents to the harassment, if
the defendant is at least 16 years of age at the time of the commission
of the offense;

(5) Threatening injury to the person or to the property of the person to
whom an electronic communication is directed or to any of his or her
family or household members; or

(6) Knowingly permitting any electronic communications device to be used
for any of the purposes mentioned in this subsection (a).

(b) Telecommunications carriers, commercial mobile service providers,
and providers of information services, including, but not limited to,
Internet service providers and hosting service providers, are not liable
under this Section, except for willful and wanton misconduct, by virtue
of the transmission, storage, or caching of electronic communications or
messages of others or by virtue of the provision of other related
telecommunications, commercial mobile services, or information services
used by others in violation of this Section.

(Source: P.A. 97-1108, eff. 1-1-13.)

\hypertarget{ilcs-526.5-4}{%
\subsection*{(720 ILCS 5/26.5-4)}\label{ilcs-526.5-4}}
\addcontentsline{toc}{subsection}{(720 ILCS 5/26.5-4)}

\hypertarget{sec.-26.5-4.-evidence-inference.}{%
\section*{Sec. 26.5-4. Evidence
inference.}\label{sec.-26.5-4.-evidence-inference.}}
\addcontentsline{toc}{section}{Sec. 26.5-4. Evidence inference.}

\markright{Sec. 26.5-4. Evidence inference.}

Evidence that a defendant made additional telephone calls or engaged in
additional electronic communications after having been requested by a
named complainant or by a family or household member of the complainant
to stop may be considered as evidence of an intent to harass unless
disproved by evidence to the contrary.

(Source: P.A. 97-1108, eff. 1-1-13.)

\hypertarget{ilcs-526.5-5}{%
\subsection*{(720 ILCS 5/26.5-5)}\label{ilcs-526.5-5}}
\addcontentsline{toc}{subsection}{(720 ILCS 5/26.5-5)}

\hypertarget{sec.-26.5-5.-sentence.}{%
\section*{Sec. 26.5-5. Sentence.}\label{sec.-26.5-5.-sentence.}}
\addcontentsline{toc}{section}{Sec. 26.5-5. Sentence.}

\markright{Sec. 26.5-5. Sentence.}

(a) Except as provided in subsection (b), a person who violates any of
the provisions of Section 26.5-1, 26.5-2, or 26.5-3 of this Article is
guilty of a Class B misdemeanor. Except as provided in subsection (b), a
second or subsequent violation of Section 26.5-1, 26.5-2, or 26.5-3 of
this Article is a Class A misdemeanor, for which the court shall impose
a minimum of 14 days in jail or, if public or community service is
established in the county in which the offender was convicted, 240 hours
of public or community service.

(b) In any of the following circumstances, a person who violates Section
26.5-1, 26.5-2, or 26.5-3 of this Article shall be guilty of a Class 4
felony:

(1) The person has 3 or more prior violations in the last 10 years of
harassment by telephone, harassment through electronic communications,
or any similar offense of any other state;

(2) The person has previously violated the harassment by telephone
provisions, or the harassment through electronic communications
provisions, or committed any similar offense in any other state with the
same victim or a member of the victim's family or household;

(3) At the time of the offense, the offender was under conditions of
pretrial release, probation, conditional discharge, mandatory supervised
release or was the subject of an order of protection, in this or any
other state, prohibiting contact with the victim or any member of the
victim's family or household;

(4) In the course of the offense, the offender threatened to kill the
victim or any member of the victim's family or household;

(5) The person has been convicted in the last 10 years of a forcible
felony as defined in Section 2-8 of the Criminal Code of 1961 or the
Criminal Code of 2012;

(6) The person violates paragraph (5) of Section

26.5-2 or paragraph (4) of Section 26.5-3; or

(7) The person was at least 18 years of age at the time of the
commission of the offense and the victim was under 18 years of age at
the time of the commission of the offense.

(c) The court may order any person convicted under this Article to
submit to a psychiatric examination.

(Source: P.A. 101-652, eff. 1-1-23 .)

\bookmarksetup{startatroot}

\hypertarget{article-28.-gambling-and-related-offenses}{%
\chapter*{Article 28. Gambling And Related
Offenses}\label{article-28.-gambling-and-related-offenses}}
\addcontentsline{toc}{chapter}{Article 28. Gambling And Related
Offenses}

\markboth{Article 28. Gambling And Related Offenses}{Article 28.
Gambling And Related Offenses}

\hypertarget{ilcs-528-1-from-ch.-38-par.-28-1}{%
\subsection*{(720 ILCS 5/28-1) (from Ch. 38, par.
28-1)}\label{ilcs-528-1-from-ch.-38-par.-28-1}}
\addcontentsline{toc}{subsection}{(720 ILCS 5/28-1) (from Ch. 38, par.
28-1)}

\hypertarget{sec.-28-1.-gambling.}{%
\section*{Sec. 28-1. Gambling.}\label{sec.-28-1.-gambling.}}
\addcontentsline{toc}{section}{Sec. 28-1. Gambling.}

\markright{Sec. 28-1. Gambling.}

(a) A person commits gambling when he or she:

(1) knowingly plays a game of chance or skill for money or other thing
of value, unless excepted in subsection (b) of this Section;

(2) knowingly makes a wager upon the result of any game, contest, or any
political nomination, appointment or election;

(3) knowingly operates, keeps, owns, uses, purchases, exhibits, rents,
sells, bargains for the sale or lease of, manufactures or distributes
any gambling device;

(4) contracts to have or give himself or herself or another the option
to buy or sell, or contracts to buy or sell, at a future time, any grain
or other commodity whatsoever, or any stock or security of any company,
where it is at the time of making such contract intended by both parties
thereto that the contract to buy or sell, or the option, whenever
exercised, or the contract resulting therefrom, shall be settled, not by
the receipt or delivery of such property, but by the payment only of
differences in prices thereof; however, the issuance, purchase, sale,
exercise, endorsement or guarantee, by or through a person registered
with the Secretary of State pursuant to Section 8 of the Illinois
Securities Law of 1953, or by or through a person exempt from such
registration under said Section 8, of a put, call, or other option to
buy or sell securities which have been registered with the Secretary of
State or which are exempt from such registration under Section 3 of the
Illinois Securities Law of 1953 is not gambling within the meaning of
this paragraph (4);

(5) knowingly owns or possesses any book, instrument or apparatus by
means of which bets or wagers have been, or are, recorded or registered,
or knowingly possesses any money which he has received in the course of
a bet or wager;

(6) knowingly sells pools upon the result of any game or contest of
skill or chance, political nomination, appointment or election;

(7) knowingly sets up or promotes any lottery or sells, offers to sell
or transfers any ticket or share for any lottery;

(8) knowingly sets up or promotes any policy game or sells, offers to
sell or knowingly possesses or transfers any policy ticket, slip,
record, document or other similar device;

(9) knowingly drafts, prints or publishes any lottery ticket or share,
or any policy ticket, slip, record, document or similar device, except
for such activity related to lotteries, bingo games and raffles
authorized by and conducted in accordance with the laws of Illinois or
any other state or foreign government;

(10) knowingly advertises any lottery or policy game, except for such
activity related to lotteries, bingo games and raffles authorized by and
conducted in accordance with the laws of Illinois or any other state;

(11) knowingly transmits information as to wagers, betting odds, or
changes in betting odds by telephone, telegraph, radio, semaphore or
similar means; or knowingly installs or maintains equipment for the
transmission or receipt of such information; except that nothing in this
subdivision (11) prohibits transmission or receipt of such information
for use in news reporting of sporting events or contests; or

(12) knowingly establishes, maintains, or operates an

Internet site that permits a person to play a game of chance or skill
for money or other thing of value by means of the Internet or to make a
wager upon the result of any game, contest, political nomination,
appointment, or election by means of the Internet. This item (12) does
not apply to activities referenced in items (6), (6.1), (8), (8.1), and
(15) of subsection (b) of this Section.

(b) Participants in any of the following activities shall not be
convicted of gambling:

(1) Agreements to compensate for loss caused by the happening of chance
including without limitation contracts of indemnity or guaranty and life
or health or accident insurance.

(2) Offers of prizes, award or compensation to the actual contestants in
any bona fide contest for the determination of skill, speed, strength or
endurance or to the owners of animals or vehicles entered in such
contest.

(3) Pari-mutuel betting as authorized by the law of this State.

(4) Manufacture of gambling devices, including the acquisition of
essential parts therefor and the assembly thereof, for transportation in
interstate or foreign commerce to any place outside this State when such
transportation is not prohibited by any applicable Federal law; or the
manufacture, distribution, or possession of video gaming terminals, as
defined in the Video Gaming Act, by manufacturers, distributors, and
terminal operators licensed to do so under the Video Gaming Act.

(5) The game commonly known as ``bingo'', when conducted in accordance
with the Bingo License and Tax Act.

(6) Lotteries when conducted by the State of Illinois in accordance with
the Illinois Lottery Law. This exemption includes any activity conducted
by the Department of Revenue to sell lottery tickets pursuant to the
provisions of the Illinois Lottery Law and its rules.

(6.1) The purchase of lottery tickets through the

Internet for a lottery conducted by the State of Illinois under the
program established in Section 7.12 of the Illinois Lottery Law.

(7) Possession of an antique slot machine that is neither used nor
intended to be used in the operation or promotion of any unlawful
gambling activity or enterprise. For the purpose of this subparagraph
(b)(7), an antique slot machine is one manufactured 25 years ago or
earlier.

(8) Raffles and poker runs when conducted in accordance with the Raffles
and Poker Runs Act.

(8.1) The purchase of raffle chances for a raffle conducted in
accordance with the Raffles and Poker Runs Act.

(9) Charitable games when conducted in accordance with the Charitable
Games Act.

(10) Pull tabs and jar games when conducted under the

Illinois Pull Tabs and Jar Games Act.

(11) Gambling games when authorized by the Illinois

Gambling Act.

(12) Video gaming terminal games at a licensed establishment, licensed
truck stop establishment, licensed large truck stop establishment,
licensed fraternal establishment, or licensed veterans establishment
when conducted in accordance with the Video Gaming Act.

(13) Games of skill or chance where money or other things of value can
be won but no payment or purchase is required to participate.

(14) Savings promotion raffles authorized under

Section 5g of the Illinois Banking Act, Section 7008 of the Savings Bank
Act, Section 42.7 of the Illinois Credit Union Act, Section 5136B of the
National Bank Act (12 U.S.C. 25a), or Section 4 of the Home Owners' Loan
Act (12 U.S.C. 1463).

(15) Sports wagering when conducted in accordance with the Sports
Wagering Act.

(c) Sentence.

Gambling is a Class A misdemeanor. A second or subsequent conviction
under subsections (a)(3) through (a)(12), is a Class 4 felony.

(d) Circumstantial evidence.

In prosecutions under this Section circumstantial evidence shall have
the same validity and weight as in any criminal prosecution.

(Source: P.A. 101-31, Article 25, Section 25-915, eff. 6-28-19; 101-31,
Article 35, Section 35-80, eff. 6-28-19; 101-109, eff. 7-19-19; 102-558,
eff. 8-20-21.)

\hypertarget{ilcs-528-1.1-from-ch.-38-par.-28-1.1}{%
\subsection*{(720 ILCS 5/28-1.1) (from Ch. 38, par.
28-1.1)}\label{ilcs-528-1.1-from-ch.-38-par.-28-1.1}}
\addcontentsline{toc}{subsection}{(720 ILCS 5/28-1.1) (from Ch. 38, par.
28-1.1)}

\hypertarget{sec.-28-1.1.-syndicated-gambling.}{%
\section*{Sec. 28-1.1. Syndicated
gambling.}\label{sec.-28-1.1.-syndicated-gambling.}}
\addcontentsline{toc}{section}{Sec. 28-1.1. Syndicated gambling.}

\markright{Sec. 28-1.1. Syndicated gambling.}

(a) Declaration of Purpose. Recognizing the close relationship between
professional gambling and other organized crime, it is declared to be
the policy of the legislature to restrain persons from engaging in the
business of gambling for profit in this State. This Section shall be
liberally construed and administered with a view to carrying out this
policy.

(b) A person commits syndicated gambling when he or she operates a
``policy game'' or engages in the business of bookmaking.

(c) A person ``operates a policy game'' when he or she knowingly uses
any premises or property for the purpose of receiving or knowingly does
receive from what is commonly called ``policy'':

(1) money from a person other than the bettor or player whose bets or
plays are represented by the money; or

(2) written ``policy game'' records, made or used over any period of
time, from a person other than the bettor or player whose bets or plays
are represented by the written record.

(d) A person engages in bookmaking when he or she knowingly receives or
accepts more than five bets or wagers upon the result of any trials or
contests of skill, speed or power of endurance or upon any lot, chance,
casualty, unknown or contingent event whatsoever, which bets or wagers
shall be of such size that the total of the amounts of money paid or
promised to be paid to the bookmaker on account thereof shall exceed
\$2,000. Bookmaking is the receiving or accepting of bets or wagers
regardless of the form or manner in which the bookmaker records them.

(e) Participants in any of the following activities shall not be
convicted of syndicated gambling:

(1) Agreements to compensate for loss caused by the happening of chance
including without limitation contracts of indemnity or guaranty and life
or health or accident insurance;

(2) Offers of prizes, award or compensation to the actual contestants in
any bona fide contest for the determination of skill, speed, strength or
endurance or to the owners of animals or vehicles entered in the
contest;

(3) Pari-mutuel betting as authorized by law of this

State;

(4) Manufacture of gambling devices, including the acquisition of
essential parts therefor and the assembly thereof, for transportation in
interstate or foreign commerce to any place outside this State when the
transportation is not prohibited by any applicable Federal law;

(5) Raffles and poker runs when conducted in accordance with the Raffles
and Poker Runs Act;

(6) Gambling games conducted on riverboats, in casinos, or at
organization gaming facilities when authorized by the Illinois Gambling
Act;

(7) Video gaming terminal games at a licensed establishment, licensed
truck stop establishment, licensed large truck stop establishment,
licensed fraternal establishment, or licensed veterans establishment
when conducted in accordance with the Video Gaming Act; and

(8) Savings promotion raffles authorized under

Section 5g of the Illinois Banking Act, Section 7008 of the Savings Bank
Act, Section 42.7 of the Illinois Credit Union Act, Section 5136B of the
National Bank Act (12 U.S.C. 25a), or Section 4 of the Home Owners' Loan
Act (12 U.S.C. 1463).

(f) Sentence. Syndicated gambling is a Class 3 felony.

(Source: P.A. 101-31, eff. 6-28-19.)

\hypertarget{ilcs-528-2-from-ch.-38-par.-28-2}{%
\subsection*{(720 ILCS 5/28-2) (from Ch. 38, par.
28-2)}\label{ilcs-528-2-from-ch.-38-par.-28-2}}
\addcontentsline{toc}{subsection}{(720 ILCS 5/28-2) (from Ch. 38, par.
28-2)}

\hypertarget{sec.-28-2.-definitions.}{%
\section*{Sec. 28-2. Definitions.}\label{sec.-28-2.-definitions.}}
\addcontentsline{toc}{section}{Sec. 28-2. Definitions.}

\markright{Sec. 28-2. Definitions.}

(a) A ``gambling device'' is any clock, tape machine, slot machine or
other machines or device for the reception of money or other thing of
value on chance or skill or upon the action of which money or other
thing of value is staked, hazarded, bet, won, or lost; or any mechanism,
furniture, fixture, equipment, or other device designed primarily for
use in a gambling place. A ``gambling device'' does not include:

(1) A coin-in-the-slot operated mechanical device played for amusement
which rewards the player with the right to replay such mechanical
device, which device is so constructed or devised as to make such result
of the operation thereof depend in part upon the skill of the player and
which returns to the player thereof no money, property, or right to
receive money or property.

(2) Vending machines by which full and adequate return is made for the
money invested and in which there is no element of chance or hazard.

(3) A crane game. For the purposes of this paragraph

(3), a ``crane game'' is an amusement device involving skill, if it
rewards the player exclusively with merchandise contained within the
amusement device proper and limited to toys, novelties, and prizes other
than currency, each having a wholesale value which is not more than
\$25.

(4) A redemption machine. For the purposes of this paragraph (4), a
``redemption machine'' is a single-player or multi-player amusement
device involving a game, the object of which is throwing, rolling,
bowling, shooting, placing, or propelling a ball or other object that is
either physical or computer generated on a display or with lights into,
upon, or against a hole or other target that is either physical or
computer generated on a display or with lights, or stopping, by
physical, mechanical, or electronic means, a moving object that is
either physical or computer generated on a display or with lights into,
upon, or against a hole or other target that is either physical or
computer generated on a display or with lights, provided that all of the
following conditions are met:

(A) The outcome of the game is predominantly determined by the skill of
the player.

(B) The award of the prize is based solely upon the player's achieving
the object of the game or otherwise upon the player's score.

(C) Only merchandise prizes are awarded.

(D) The wholesale value of prizes awarded in lieu of tickets or tokens
for single play of the device does not exceed \$25.

(E) The redemption value of tickets, tokens, and other representations
of value, which may be accumulated by players to redeem prizes of
greater value, for a single play of the device does not exceed \$25.

(5) Video gaming terminals at a licensed establishment, licensed truck
stop establishment, licensed large truck stop establishment, licensed
fraternal establishment, or licensed veterans establishment licensed in
accordance with the Video Gaming Act.

(a-5) ``Internet'' means an interactive computer service or system or an
information service, system, or access software provider that provides
or enables computer access by multiple users to a computer server, and
includes, but is not limited to, an information service, system, or
access software provider that provides access to a network system
commonly known as the Internet, or any comparable system or service and
also includes, but is not limited to, a World Wide Web page, newsgroup,
message board, mailing list, or chat area on any interactive computer
service or system or other online service.

(a-6) ``Access'' has the meaning ascribed to the term in Section 17-55.

(a-7) ``Computer'' has the meaning ascribed to the term in Section
17-0.5.

(b) A ``lottery'' is any scheme or procedure whereby one or more prizes
are distributed by chance among persons who have paid or promised
consideration for a chance to win such prizes, whether such scheme or
procedure is called a lottery, raffle, gift, sale, or some other name,
excluding savings promotion raffles authorized under Section 5g of the
Illinois Banking Act, Section 7008 of the Savings Bank Act, Section 42.7
of the Illinois Credit Union Act, Section 5136B of the National Bank Act
(12 U.S.C. 25a), or Section 4 of the Home Owners' Loan Act (12 U.S.C.
1463).

(c) A ``policy game'' is any scheme or procedure whereby a person
promises or guarantees by any instrument, bill, certificate, writing,
token, or other device that any particular number, character, ticket, or
certificate shall in the event of any contingency in the nature of a
lottery entitle the purchaser or holder to receive money, property, or
evidence of debt.

(Source: P.A. 101-31, eff. 6-28-19; 101-87, eff. 1-1-20; 102-558, eff.
8-20-21.)

\hypertarget{ilcs-528-3-from-ch.-38-par.-28-3}{%
\subsection*{(720 ILCS 5/28-3) (from Ch. 38, par.
28-3)}\label{ilcs-528-3-from-ch.-38-par.-28-3}}
\addcontentsline{toc}{subsection}{(720 ILCS 5/28-3) (from Ch. 38, par.
28-3)}

\hypertarget{sec.-28-3.-keeping-a-gambling-place.}{%
\section*{Sec. 28-3. Keeping a gambling
place.}\label{sec.-28-3.-keeping-a-gambling-place.}}
\addcontentsline{toc}{section}{Sec. 28-3. Keeping a gambling place.}

\markright{Sec. 28-3. Keeping a gambling place.}

A ``gambling place'' is any real estate, vehicle, boat, or any other
property whatsoever used for the purposes of gambling other than
gambling conducted in the manner authorized by the Illinois Gambling
Act, the Sports Wagering Act, or the Video Gaming Act. Any person who
knowingly permits any premises or property owned or occupied by him or
under his control to be used as a gambling place commits a Class A
misdemeanor. Each subsequent offense is a Class 4 felony. When any
premises is determined by the circuit court to be a gambling place:

(a) Such premises is a public nuisance and may be proceeded against as
such, and

(b) All licenses, permits or certificates issued by the State of
Illinois or any subdivision or public agency thereof authorizing the
serving of food or liquor on such premises shall be void; and no
license, permit or certificate so cancelled shall be reissued for such
premises for a period of 60 days thereafter; nor shall any person
convicted of keeping a gambling place be reissued such license for one
year from his conviction and, after a second conviction of keeping a
gambling place, any such person shall not be reissued such license, and

(c) Such premises of any person who knowingly permits thereon a
violation of any Section of this Article shall be held liable for, and
may be sold to pay any unsatisfied judgment that may be recovered and
any unsatisfied fine that may be levied under any Section of this
Article.

(Source: P.A. 101-31, Article 25, Section 25-915, eff. 6-28-19; 101-31,
Article 35, Section 35-80, eff. 6-28-19; 102-558, eff. 8-20-21.)

\hypertarget{ilcs-528-4-from-ch.-38-par.-28-4}{%
\subsection*{(720 ILCS 5/28-4) (from Ch. 38, par.
28-4)}\label{ilcs-528-4-from-ch.-38-par.-28-4}}
\addcontentsline{toc}{subsection}{(720 ILCS 5/28-4) (from Ch. 38, par.
28-4)}

\hypertarget{sec.-28-4.-registration-of-federal-gambling-stamps.}{%
\section*{Sec. 28-4. Registration of Federal Gambling
Stamps.}\label{sec.-28-4.-registration-of-federal-gambling-stamps.}}
\addcontentsline{toc}{section}{Sec. 28-4. Registration of Federal
Gambling Stamps.}

\markright{Sec. 28-4. Registration of Federal Gambling Stamps.}

(a) Every person who has purchased a Federal Wagering Occupational Tax
Stamp, as required by the United States under the applicable provisions
of the Internal Revenue Code, or a Federal Gaming Device Tax Stamp, as
required by the United States under the applicable provisions of the
Internal Revenue Code, shall register forthwith such stamp or stamps
with the county clerk's office in which he resides and the county
clerk's office of each and every county in which he conducts any
business. A violation of this Section is a Class B misdemeanor. A
subsequent violation is a Class A misdemeanor.

(b) To register a stamp as required by this Section, each individual
stamp purchaser and each member of a firm or association which is a
stamp purchaser and, if such purchaser is corporate, the registered
agent of the purchasing corporation shall deliver the stamp to the
county clerk for inspection and shall under oath or affirmation complete
and sign a registration form which shall state the full name and
residence and business address of each purchaser and of each member of a
purchasing firm or association and of each person employed or engaged in
gambling on behalf of such purchaser, shall state the registered agent
and registered address of a corporate purchaser, shall state each place
where gambling is to be performed by or on behalf of the purchaser, and
shall state the duration of validity of the stamp and the federal
registration number and tax return number thereof. Any false statement
in the registration form is material and is evidence of perjury.

(c) Within 3 days after such registration the county clerk shall by
registered mail forward notice of such registration and a duplicate copy
of each registration form to the Attorney General of this State, to the
Chairman of the Illinois Liquor Control Commission, to the State's
Attorney and Sheriff of each county wherein the stamp is registered, and
to the principal official of the department of police of each city,
village and incorporated town in this State wherein the stamp is
registered or wherein the registrant maintains a business address.

(Source: P.A. 77-2638.)

\hypertarget{ilcs-528-5-from-ch.-38-par.-28-5}{%
\subsection*{(720 ILCS 5/28-5) (from Ch. 38, par.
28-5)}\label{ilcs-528-5-from-ch.-38-par.-28-5}}
\addcontentsline{toc}{subsection}{(720 ILCS 5/28-5) (from Ch. 38, par.
28-5)}

\hypertarget{sec.-28-5.-seizure-of-gambling-devices-and-gambling-funds.}{%
\section*{Sec. 28-5. Seizure of gambling devices and gambling
funds.}\label{sec.-28-5.-seizure-of-gambling-devices-and-gambling-funds.}}
\addcontentsline{toc}{section}{Sec. 28-5. Seizure of gambling devices
and gambling funds.}

\markright{Sec. 28-5. Seizure of gambling devices and gambling funds.}

(a) Every device designed for gambling which is incapable of lawful use
or every device used unlawfully for gambling shall be considered a
``gambling device'', and shall be subject to seizure, confiscation and
destruction by the Illinois State Police or by any municipal, or other
local authority, within whose jurisdiction the same may be found. As
used in this Section, a ``gambling device'' includes any slot machine,
and includes any machine or device constructed for the reception of
money or other thing of value and so constructed as to return, or to
cause someone to return, on chance to the player thereof money, property
or a right to receive money or property. With the exception of any
device designed for gambling which is incapable of lawful use, no
gambling device shall be forfeited or destroyed unless an individual
with a property interest in said device knows of the unlawful use of the
device.

(b) Every gambling device shall be seized and forfeited to the county
wherein such seizure occurs. Any money or other thing of value
integrally related to acts of gambling shall be seized and forfeited to
the county wherein such seizure occurs.

(c) If, within 60 days after any seizure pursuant to subparagraph (b) of
this Section, a person having any property interest in the seized
property is charged with an offense, the court which renders judgment
upon such charge shall, within 30 days after such judgment, conduct a
forfeiture hearing to determine whether such property was a gambling
device at the time of seizure. Such hearing shall be commenced by a
written petition by the State, including material allegations of fact,
the name and address of every person determined by the State to have any
property interest in the seized property, a representation that written
notice of the date, time and place of such hearing has been mailed to
every such person by certified mail at least 10 days before such date,
and a request for forfeiture. Every such person may appear as a party
and present evidence at such hearing. The quantum of proof required
shall be a preponderance of the evidence, and the burden of proof shall
be on the State. If the court determines that the seized property was a
gambling device at the time of seizure, an order of forfeiture and
disposition of the seized property shall be entered: a gambling device
shall be received by the State's Attorney, who shall effect its
destruction, except that valuable parts thereof may be liquidated and
the resultant money shall be deposited in the general fund of the county
wherein such seizure occurred; money and other things of value shall be
received by the State's Attorney and, upon liquidation, shall be
deposited in the general fund of the county wherein such seizure
occurred. However, in the event that a defendant raises the defense that
the seized slot machine is an antique slot machine described in
subparagraph (b) (7) of Section 28-1 of this Code and therefore he is
exempt from the charge of a gambling activity participant, the seized
antique slot machine shall not be destroyed or otherwise altered until a
final determination is made by the Court as to whether it is such an
antique slot machine. Upon a final determination by the Court of this
question in favor of the defendant, such slot machine shall be
immediately returned to the defendant. Such order of forfeiture and
disposition shall, for the purposes of appeal, be a final order and
judgment in a civil proceeding.

(d) If a seizure pursuant to subparagraph (b) of this Section is not
followed by a charge pursuant to subparagraph (c) of this Section, or if
the prosecution of such charge is permanently terminated or indefinitely
discontinued without any judgment of conviction or acquittal (1) the
State's Attorney shall commence an in rem proceeding for the forfeiture
and destruction of a gambling device, or for the forfeiture and deposit
in the general fund of the county of any seized money or other things of
value, or both, in the circuit court and (2) any person having any
property interest in such seized gambling device, money or other thing
of value may commence separate civil proceedings in the manner provided
by law.

(e) Any gambling device displayed for sale to a riverboat gambling
operation, casino gambling operation, or organization gaming facility or
used to train occupational licensees of a riverboat gambling operation,
casino gambling operation, or organization gaming facility as authorized
under the Illinois Gambling Act is exempt from seizure under this
Section.

(f) Any gambling equipment, devices, and supplies provided by a licensed
supplier in accordance with the Illinois Gambling Act which are removed
from a riverboat, casino, or organization gaming facility for repair are
exempt from seizure under this Section.

(g) The following video gaming terminals are exempt from seizure under
this Section:

(1) Video gaming terminals for sale to a licensed distributor or
operator under the Video Gaming Act.

(2) Video gaming terminals used to train licensed technicians or
licensed terminal handlers.

(3) Video gaming terminals that are removed from a licensed
establishment, licensed truck stop establishment, licensed large truck
stop establishment, licensed fraternal establishment, or licensed
veterans establishment for repair.

(h) Property seized or forfeited under this Section is subject to
reporting under the Seizure and Forfeiture Reporting Act.

(i) Any sports lottery terminals provided by a central system provider
that are removed from a lottery retailer for repair under the Sports
Wagering Act are exempt from seizure under this Section.

(Source: P.A. 101-31, Article 25, Section 25-915, eff. 6-28-19; 101-31,
Article 35, Section 35-80, eff. 6-28-19; 102-538, eff. 8-20-21; 102-558,
eff. 8-20-21.)

\hypertarget{ilcs-528-7-from-ch.-38-par.-28-7}{%
\subsection*{(720 ILCS 5/28-7) (from Ch. 38, par.
28-7)}\label{ilcs-528-7-from-ch.-38-par.-28-7}}
\addcontentsline{toc}{subsection}{(720 ILCS 5/28-7) (from Ch. 38, par.
28-7)}

\hypertarget{sec.-28-7.-gambling-contracts-void.}{%
\section*{Sec. 28-7. Gambling contracts
void.}\label{sec.-28-7.-gambling-contracts-void.}}
\addcontentsline{toc}{section}{Sec. 28-7. Gambling contracts void.}

\markright{Sec. 28-7. Gambling contracts void.}

(a) All promises, notes, bills, bonds, covenants, contracts, agreements,
judgments, mortgages, or other securities or conveyances made, given,
granted, drawn, or entered into, or executed by any person whatsoever,
where the whole or any part of the consideration thereof is for any
money or thing of value, won or obtained in violation of any Section of
this Article are null and void.

(b) Any obligation void under this Section may be set aside and vacated
by any court of competent jurisdiction, upon a complaint filed for that
purpose, by the person so granting, giving, entering into, or executing
the same, or by his executors or administrators, or by any creditor,
heir, legatee, purchaser or other person interested therein; or if a
judgment, the same may be set aside on motion of any person stated
above, on due notice thereof given.

(c) No assignment of any obligation void under this Section may in any
manner affect the defense of the person giving, granting, drawing,
entering into or executing such obligation, or the remedies of any
person interested therein.

(d) This Section shall not prevent a licensed owner of a riverboat
gambling operation, a casino gambling operation, or an organization
gaming licensee under the Illinois Gambling Act and the Illinois Horse
Racing Act of 1975 from instituting a cause of action to collect any
amount due and owing under an extension of credit to a gambling patron
as authorized under Section 11.1 of the Illinois Gambling Act.

(Source: P.A. 101-31, eff. 6-28-19.)

\hypertarget{ilcs-528-8-from-ch.-38-par.-28-8}{%
\subsection*{(720 ILCS 5/28-8) (from Ch. 38, par.
28-8)}\label{ilcs-528-8-from-ch.-38-par.-28-8}}
\addcontentsline{toc}{subsection}{(720 ILCS 5/28-8) (from Ch. 38, par.
28-8)}

\hypertarget{sec.-28-8.-gambling-losses-recoverable.}{%
\section*{Sec. 28-8. Gambling losses
recoverable.}\label{sec.-28-8.-gambling-losses-recoverable.}}
\addcontentsline{toc}{section}{Sec. 28-8. Gambling losses recoverable.}

\markright{Sec. 28-8. Gambling losses recoverable.}

(a) Any person who by gambling shall lose to any other person, any sum
of money or thing of value, amounting to the sum of \$50 or more and
shall pay or deliver the same or any part thereof, may sue for and
recover the money or other thing of value, so lost and paid or
delivered, in a civil action against the winner thereof, with costs, in
the circuit court. No person who accepts from another person for
transmission, and transmits, either in his own name or in the name of
such other person, any order for any transaction to be made upon, or who
executes any order given to him by another person, or who executes any
transaction for his own account on, any regular board of trade or
commercial, commodity or stock exchange, shall, under any circumstances,
be deemed a ``winner'' of any moneys lost by such other person in or
through any such transactions.

(b) If within 6 months, such person who under the terms of Subsection
28-8(a) is entitled to initiate action to recover his losses does not in
fact pursue his remedy, any person may initiate a civil action against
the winner. The court or the jury, as the case may be, shall determine
the amount of the loss. After such determination, the court shall enter
a judgment of triple the amount so determined.

(c) Gambling losses as a result of gambling conducted on a video gaming
terminal licensed under the Video Gaming Act are not recoverable under
this Section.

(Source: P.A. 98-31, eff. 6-24-13.)

\hypertarget{ilcs-528-9-from-ch.-38-par.-28-9}{%
\subsection*{(720 ILCS 5/28-9) (from Ch. 38, par.
28-9)}\label{ilcs-528-9-from-ch.-38-par.-28-9}}
\addcontentsline{toc}{subsection}{(720 ILCS 5/28-9) (from Ch. 38, par.
28-9)}

\hypertarget{sec.-28-9.-at-the-option-of-the-prosecuting-attorney-any-prosecution-under-this-article-may-be-commenced-by-an-information-as-defined-in-section-102-12-of-the-code-of-criminal-procedure-of-1963.}{%
\section*{Sec. 28-9. At the option of the prosecuting attorney any
prosecution under this Article may be commenced by an information as
defined in Section 102-12 of the Code of Criminal Procedure of
1963.}\label{sec.-28-9.-at-the-option-of-the-prosecuting-attorney-any-prosecution-under-this-article-may-be-commenced-by-an-information-as-defined-in-section-102-12-of-the-code-of-criminal-procedure-of-1963.}}
\addcontentsline{toc}{section}{Sec. 28-9. At the option of the
prosecuting attorney any prosecution under this Article may be commenced
by an information as defined in Section 102-12 of the Code of Criminal
Procedure of 1963.}

\markright{Sec. 28-9. At the option of the prosecuting attorney any
prosecution under this Article may be commenced by an information as
defined in Section 102-12 of the Code of Criminal Procedure of 1963.}

(Source: P.A. 76-1131 .)

\bookmarksetup{startatroot}

\hypertarget{article-29.-bribery-in-contests}{%
\chapter*{Article 29. Bribery In
Contests}\label{article-29.-bribery-in-contests}}
\addcontentsline{toc}{chapter}{Article 29. Bribery In Contests}

\markboth{Article 29. Bribery In Contests}{Article 29. Bribery In
Contests}

\hypertarget{ilcs-529-1-from-ch.-38-par.-29-1}{%
\subsection*{(720 ILCS 5/29-1) (from Ch. 38, par.
29-1)}\label{ilcs-529-1-from-ch.-38-par.-29-1}}
\addcontentsline{toc}{subsection}{(720 ILCS 5/29-1) (from Ch. 38, par.
29-1)}

\hypertarget{sec.-29-1.-offering-a-bribe.}{%
\section*{Sec. 29-1. Offering a
bribe.}\label{sec.-29-1.-offering-a-bribe.}}
\addcontentsline{toc}{section}{Sec. 29-1. Offering a bribe.}

\markright{Sec. 29-1. Offering a bribe.}

(a) Any person who, with intent to influence any person participating
in, officiating or connected with any professional or amateur athletic
contest, sporting event or exhibition, gives, offers or promises any
money, bribe or other thing of value or advantage to induce such
participant, official or other person not to use his best efforts in
connection with such contest, event or exhibition commits a Class 4
felony.

(b) Any person who, with the intent to influence the decision of any
individual, offers or promises any money, bribe or other thing of value
or advantage to induce such individual to attend, refrain from attending
or continue to attend a particular public or private institution of
secondary education or higher education for the purpose of participating
or not participating in interscholastic athletic competition for such
institution commits a Class A misdemeanor. This Section does not apply
to the: (1) offering or awarding to an individual any type of
scholarship, grant or other bona fide financial aid or employment; (2)
offering of any type of financial assistance by such individual's
family; or (3) offering of any item of de minimis value by such
institution's authorities if such item is of the nature of an item that
is commonly provided to any or all students or prospective students.

(c) Any person who gives any money, goods or other thing of value to an
individual enrolled in an institution of higher education who
participates in interscholastic competition and represents or attempts
to represent such individual in future negotiations for employment with
any professional sports team commits a Class A misdemeanor.

(Source: P.A. 85-665 .)

\hypertarget{ilcs-529-2-from-ch.-38-par.-29-2}{%
\subsection*{(720 ILCS 5/29-2) (from Ch. 38, par.
29-2)}\label{ilcs-529-2-from-ch.-38-par.-29-2}}
\addcontentsline{toc}{subsection}{(720 ILCS 5/29-2) (from Ch. 38, par.
29-2)}

\hypertarget{sec.-29-2.-accepting-a-bribe.}{%
\section*{Sec. 29-2. Accepting a
bribe.}\label{sec.-29-2.-accepting-a-bribe.}}
\addcontentsline{toc}{section}{Sec. 29-2. Accepting a bribe.}

\markright{Sec. 29-2. Accepting a bribe.}

Any person participating in, officiating or connected with any
professional or amateur athletic contest, sporting event or exhibition
who accepts or agrees to accept any money, bribe or other thing of value
or advantage with the intent, understanding or agreement that he will
not use his best efforts in connection with such contest, event or
exhibition commits a Class 4 felony.

(Source: P.A. 77-2638.)

\hypertarget{ilcs-529-3-from-ch.-38-par.-29-3}{%
\subsection*{(720 ILCS 5/29-3) (from Ch. 38, par.
29-3)}\label{ilcs-529-3-from-ch.-38-par.-29-3}}
\addcontentsline{toc}{subsection}{(720 ILCS 5/29-3) (from Ch. 38, par.
29-3)}

\hypertarget{sec.-29-3.-failure-to-report-offer-of-bribe.}{%
\section*{Sec. 29-3. Failure to report offer of
bribe.}\label{sec.-29-3.-failure-to-report-offer-of-bribe.}}
\addcontentsline{toc}{section}{Sec. 29-3. Failure to report offer of
bribe.}

\markright{Sec. 29-3. Failure to report offer of bribe.}

Any person participating, officiating or connected with any professional
or amateur athletic contest, sporting event or exhibition who fails to
report forthwith to his employer, the promoter of such contest, event or
exhibition, a peace officer, or the local State's Attorney any offer or
promise made to him in violation of Section 29-1 commits a Class A
misdemeanor.

(Source: P.A. 77-2638.)

\bookmarksetup{startatroot}

\hypertarget{article-29a.-commercial-bribery}{%
\chapter*{Article 29a. Commercial
Bribery}\label{article-29a.-commercial-bribery}}
\addcontentsline{toc}{chapter}{Article 29a. Commercial Bribery}

\markboth{Article 29a. Commercial Bribery}{Article 29a. Commercial
Bribery}

\hypertarget{ilcs-529a-1-from-ch.-38-par.-29a-1}{%
\subsection*{(720 ILCS 5/29A-1) (from Ch. 38, par.
29A-1)}\label{ilcs-529a-1-from-ch.-38-par.-29a-1}}
\addcontentsline{toc}{subsection}{(720 ILCS 5/29A-1) (from Ch. 38, par.
29A-1)}

\hypertarget{sec.-29a-1.-a-person-commits-commercial-bribery-when-he-confers-or-offers-or-agrees-to-confer-any-benefit-upon-any-employee-agent-or-fiduciary-without-the-consent-of-the-latters-employer-or-principal-with-intent-to-influence-his-conduct-in-relation-to-his-employers-or-principals-affairs.}{%
\section*{Sec. 29A-1. A person commits commercial bribery when he
confers, or offers or agrees to confer, any benefit upon any employee,
agent or fiduciary without the consent of the latter's employer or
principal, with intent to influence his conduct in relation to his
employer's or principal's
affairs.}\label{sec.-29a-1.-a-person-commits-commercial-bribery-when-he-confers-or-offers-or-agrees-to-confer-any-benefit-upon-any-employee-agent-or-fiduciary-without-the-consent-of-the-latters-employer-or-principal-with-intent-to-influence-his-conduct-in-relation-to-his-employers-or-principals-affairs.}}
\addcontentsline{toc}{section}{Sec. 29A-1. A person commits commercial
bribery when he confers, or offers or agrees to confer, any benefit upon
any employee, agent or fiduciary without the consent of the latter's
employer or principal, with intent to influence his conduct in relation
to his employer's or principal's affairs.}

\markright{Sec. 29A-1. A person commits commercial bribery when he
confers, or offers or agrees to confer, any benefit upon any employee,
agent or fiduciary without the consent of the latter's employer or
principal, with intent to influence his conduct in relation to his
employer's or principal's affairs.}

(Source: P.A. 76-1129 .)

\hypertarget{ilcs-529a-2-from-ch.-38-par.-29a-2}{%
\subsection*{(720 ILCS 5/29A-2) (from Ch. 38, par.
29A-2)}\label{ilcs-529a-2-from-ch.-38-par.-29a-2}}
\addcontentsline{toc}{subsection}{(720 ILCS 5/29A-2) (from Ch. 38, par.
29A-2)}

\hypertarget{sec.-29a-2.-an-employee-agent-or-fiduciary-commits-commercial-bribe-receiving-when-without-consent-of-his-employer-or-principal-he-solicits-accepts-or-agrees-to-accept-any-benefit-from-another-person-upon-an-agreement-or-understanding-that-such-benefit-will-influence-his-conduct-in-relation-to-his-employers-or-principals-affairs.}{%
\section*{Sec. 29A-2. An employee, agent or fiduciary commits commercial
bribe receiving when, without consent of his employer or principal, he
solicits, accepts or agrees to accept any benefit from another person
upon an agreement or understanding that such benefit will influence his
conduct in relation to his employer's or principal's
affairs.}\label{sec.-29a-2.-an-employee-agent-or-fiduciary-commits-commercial-bribe-receiving-when-without-consent-of-his-employer-or-principal-he-solicits-accepts-or-agrees-to-accept-any-benefit-from-another-person-upon-an-agreement-or-understanding-that-such-benefit-will-influence-his-conduct-in-relation-to-his-employers-or-principals-affairs.}}
\addcontentsline{toc}{section}{Sec. 29A-2. An employee, agent or
fiduciary commits commercial bribe receiving when, without consent of
his employer or principal, he solicits, accepts or agrees to accept any
benefit from another person upon an agreement or understanding that such
benefit will influence his conduct in relation to his employer's or
principal's affairs.}

\markright{Sec. 29A-2. An employee, agent or fiduciary commits
commercial bribe receiving when, without consent of his employer or
principal, he solicits, accepts or agrees to accept any benefit from
another person upon an agreement or understanding that such benefit will
influence his conduct in relation to his employer's or principal's
affairs.}

(Source: P.A. 76-1129 .)

\hypertarget{ilcs-529a-3-from-ch.-38-par.-29a-3}{%
\subsection*{(720 ILCS 5/29A-3) (from Ch. 38, par.
29A-3)}\label{ilcs-529a-3-from-ch.-38-par.-29a-3}}
\addcontentsline{toc}{subsection}{(720 ILCS 5/29A-3) (from Ch. 38, par.
29A-3)}

\hypertarget{sec.-29a-3.-sentence.}{%
\section*{Sec. 29A-3. Sentence.}\label{sec.-29a-3.-sentence.}}
\addcontentsline{toc}{section}{Sec. 29A-3. Sentence.}

\markright{Sec. 29A-3. Sentence.}

(a) If the benefit offered, conferred, or agreed to be conferred,
solicited, accepted or agreed to be accepted is less than \$500,000,
commercial bribery or commercial bribe receiving is a Class A
misdemeanor and the sentence shall include, but not be limited to, a
fine not to exceed \$5,000.

(b) If the benefit offered, conferred, or agreed to be conferred,
solicited, accepted, or agreed to be accepted in violation of this
Article is \$500,000 or more, the offender is guilty of a Class 3
felony.

(Source: P.A. 93-496, eff. 1-1-04.)

\hypertarget{ilcs-529a-4}{%
\subsection*{(720 ILCS 5/29A-4)}\label{ilcs-529a-4}}
\addcontentsline{toc}{subsection}{(720 ILCS 5/29A-4)}

\hypertarget{sec.-29a-4.-corporate-crime-fund.}{%
\section*{Sec. 29A-4. Corporate Crime
Fund.}\label{sec.-29a-4.-corporate-crime-fund.}}
\addcontentsline{toc}{section}{Sec. 29A-4. Corporate Crime Fund.}

\markright{Sec. 29A-4. Corporate Crime Fund.}

(a) In addition to any fines, penalties, and assessments otherwise
authorized under this Code, any person convicted of a violation of this
Article or Section 17-26 or 17-27 of this Code shall be assessed a
penalty of not more than 3 times the value of all property involved in
the criminal activity.

(b) The penalties assessed under subsection (a) shall be deposited into
the Corporate Crime Fund, a special fund hereby created in the State
treasury. Moneys in the Fund shall be used to make restitution to a
person who has suffered property loss as a result of violations of this
Article. The court may determine the reasonable amount, terms, and
conditions of the restitution. In determining the amount and method of
payment of restitution, the court shall take into account all financial
resources of the defendant.

(Source: P.A. 93-496, eff. 1-1-04.)

\bookmarksetup{startatroot}

\hypertarget{article-29b.-money-laundering}{%
\chapter*{Article 29b. Money
Laundering}\label{article-29b.-money-laundering}}
\addcontentsline{toc}{chapter}{Article 29b. Money Laundering}

\markboth{Article 29b. Money Laundering}{Article 29b. Money Laundering}

\hypertarget{ilcs-529b-0.5}{%
\subsection*{(720 ILCS 5/29B-0.5)}\label{ilcs-529b-0.5}}
\addcontentsline{toc}{subsection}{(720 ILCS 5/29B-0.5)}

\hypertarget{sec.-29b-0.5.-definitions.}{%
\section*{Sec. 29B-0.5. Definitions.}\label{sec.-29b-0.5.-definitions.}}
\addcontentsline{toc}{section}{Sec. 29B-0.5. Definitions.}

\markright{Sec. 29B-0.5. Definitions.}

In this Article:

``Conduct'' or ``conducts'' includes, in addition to its ordinary
meaning, initiating, concluding, or participating in initiating or
concluding a transaction.

``Criminally derived property'' means: (1) any property, real or
personal, constituting or derived from proceeds obtained, directly or
indirectly, from activity that constitutes a felony under State,
federal, or foreign law; or (2) any property represented to be property
constituting or derived from proceeds obtained, directly or indirectly,
from activity that constitutes a felony under State, federal, or foreign
law.

``Director'' means the Director of the Illinois State Police or his or
her designated agents.

``Financial institution'' means any bank; savings and loan association;
trust company; agency or branch of a foreign bank in the United States;
currency exchange; credit union; mortgage banking institution;
pawnbroker; loan or finance company; operator of a credit card system;
issuer, redeemer, or cashier of travelers checks, checks, or money
orders; dealer in precious metals, stones, or jewels; broker or dealer
in securities or commodities; investment banker; or investment company.

``Financial transaction'' means a purchase, sale, loan, pledge, gift,
transfer, delivery, or other disposition utilizing criminally derived
property, and with respect to financial institutions, includes a
deposit, withdrawal, transfer between accounts, exchange of currency,
loan, extension of credit, purchase or sale of any stock, bond,
certificate of deposit or other monetary instrument, use of safe deposit
box, or any other payment, transfer or delivery by, through, or to a
financial institution. ``Financial transaction'' also means a
transaction which without regard to whether the funds, monetary
instruments, or real or personal property involved in the transaction
are criminally derived, any transaction which in any way or degree: (1)
involves the movement of funds by wire or any other means; (2) involves
one or more monetary instruments; or (3) the transfer of title to any
real or personal property. The receipt by an attorney of bona fide fees
for the purpose of legal representation is not a financial transaction
for purposes of this Article.

``Form 4-64'' means the Illinois State Police Notice/Inventory of Seized
Property (Form 4-64).

``Knowing that the property involved in a financial transaction
represents the proceeds of some form of unlawful activity'' means that
the person knew the property involved in the transaction represented
proceeds from some form, though not necessarily which form, of activity
that constitutes a felony under State, federal, or foreign law.

``Monetary instrument'' means United States coins and currency; coins
and currency of a foreign country; travelers checks; personal checks,
bank checks, and money orders; investment securities; bearer negotiable
instruments; bearer investment securities; or bearer securities and
certificates of stock in a form that title passes upon delivery.

``Specified criminal activity'' means any violation of Section 29D-15.1
and any violation of Article 29D of this Code.

``Transaction reporting requirement under State law'' means any
violation as defined under the Currency Reporting Act.

(Source: P.A. 102-538, eff. 8-20-21.)

\hypertarget{ilcs-529b-1-from-ch.-38-par.-29b-1}{%
\subsection*{(720 ILCS 5/29B-1) (from Ch. 38, par.
29B-1)}\label{ilcs-529b-1-from-ch.-38-par.-29b-1}}
\addcontentsline{toc}{subsection}{(720 ILCS 5/29B-1) (from Ch. 38, par.
29B-1)}

\hypertarget{sec.-29b-1.-money-laundering.}{%
\section*{Sec. 29B-1. Money
laundering.}\label{sec.-29b-1.-money-laundering.}}
\addcontentsline{toc}{section}{Sec. 29B-1. Money laundering.}

\markright{Sec. 29B-1. Money laundering.}

(a) A person commits the offense of money laundering:

(1) when, knowing that the property involved in a financial transaction
represents the proceeds of some form of unlawful activity, he or she
conducts or attempts to conduct the financial transaction which in fact
involves criminally derived property:

(A) with the intent to promote the carrying on of the unlawful activity
from which the criminally derived property was obtained; or

(B) where he or she knows or reasonably should know that the financial
transaction is designed in whole or in part:

(i) to conceal or disguise the nature, the location, the source, the
ownership or the control of the criminally derived property; or

(ii) to avoid a transaction reporting requirement under State law; or

(1.5) when he or she transports, transmits, or transfers, or attempts to
transport, transmit, or transfer a monetary instrument:

(A) with the intent to promote the carrying on of the unlawful activity
from which the criminally derived property was obtained; or

(B) knowing, or having reason to know, that the financial transaction is
designed in whole or in part:

(i) to conceal or disguise the nature, the location, the source, the
ownership or the control of the criminally derived property; or

(ii) to avoid a transaction reporting requirement under State law; or

(2) when, with the intent to:

(A) promote the carrying on of a specified criminal activity as defined
in this Article; or

(B) conceal or disguise the nature, location, source, ownership, or
control of property believed to be the proceeds of a specified criminal
activity as defined in this Article; or

he or she conducts or attempts to conduct a financial transaction
involving property he or she believes to be the proceeds of specified
criminal activity or property used to conduct or facilitate specified
criminal activity as defined in this Article.

(b) (Blank).

(c) Sentence.

(1) Laundering of criminally derived property of a value not exceeding
\$10,000 is a Class 3 felony;

(2) Laundering of criminally derived property of a value exceeding
\$10,000 but not exceeding \$100,000 is a Class 2 felony;

(3) Laundering of criminally derived property of a value exceeding
\$100,000 but not exceeding \$500,000 is a Class 1 felony;

(4) Money laundering in violation of subsection

(a)(2) of this Section is a Class X felony;

(5) Laundering of criminally derived property of a value exceeding
\$500,000 is a Class 1 non-probationable felony;

(6) In a prosecution under clause (a)(1.5)(B)(ii) of this Section, the
sentences are as follows:

(A) Laundering of property of a value not exceeding \$10,000 is a Class
3 felony;

(B) Laundering of property of a value exceeding

\$10,000 but not exceeding \$100,000 is a Class 2 felony;

(C) Laundering of property of a value exceeding

\$100,000 but not exceeding \$500,000 is a Class 1 felony;

(D) Laundering of property of a value exceeding

\$500,000 is a Class 1 non-probationable felony.

(Source: P.A. 99-480, eff. 9-9-15; 100-512, eff. 7-1-18; 100-699, eff.
8-3-18; 100-759, eff. 1-1-19; 100-1163, eff. 12-20-18.)

\hypertarget{ilcs-529b-2}{%
\subsection*{(720 ILCS 5/29B-2)}\label{ilcs-529b-2}}
\addcontentsline{toc}{subsection}{(720 ILCS 5/29B-2)}

\hypertarget{sec.-29b-2.-evidence-in-money-laundering-prosecutions.}{%
\section*{Sec. 29B-2. Evidence in money laundering
prosecutions.}\label{sec.-29b-2.-evidence-in-money-laundering-prosecutions.}}
\addcontentsline{toc}{section}{Sec. 29B-2. Evidence in money laundering
prosecutions.}

\markright{Sec. 29B-2. Evidence in money laundering prosecutions.}

In a prosecution under this Article, either party may introduce the
following evidence pertaining to the issue of whether the property or
proceeds were known to be some form of criminally derived property or
from some form of unlawful activity:

(1) a financial transaction was conducted or structured or attempted in
violation of the reporting requirements of any State or federal law;

(2) a financial transaction was conducted or attempted with the use of a
false or fictitious name or a forged instrument;

(3) a falsely altered or completed written instrument or a written
instrument that contains any materially false personal identifying
information was made, used, offered, or presented, whether accepted or
not, in connection with a financial transaction;

(4) a financial transaction was structured or attempted to be structured
so as to falsely report the actual consideration or value of the
transaction;

(5) a money transmitter, a person engaged in a trade or business, or any
employee of a money transmitter or a person engaged in a trade or
business, knows or reasonably should know that false personal
identifying information has been presented and incorporates the false
personal identifying information into any report or record;

(6) the criminally derived property is transported or possessed in a
fashion inconsistent with the ordinary or usual means of transportation
or possession of the property and where the property is discovered in
the absence of any documentation or other indicia of legitimate origin
or right to the property;

(7) a person pays or receives substantially less than face value for one
or more monetary instruments; or

(8) a person engages in a transaction involving one or more monetary
instruments, where the physical condition or form of the monetary
instrument or instruments makes it apparent that they are not the
product of bona fide business or financial transactions.

(Source: P.A. 100-699, eff. 8-3-18; 100-1163, eff. 12-20-18.)

\hypertarget{ilcs-529b-3}{%
\subsection*{(720 ILCS 5/29B-3)}\label{ilcs-529b-3}}
\addcontentsline{toc}{subsection}{(720 ILCS 5/29B-3)}

\hypertarget{sec.-29b-3.-duty-to-enforce-this-article.}{%
\section*{Sec. 29B-3. Duty to enforce this
Article.}\label{sec.-29b-3.-duty-to-enforce-this-article.}}
\addcontentsline{toc}{section}{Sec. 29B-3. Duty to enforce this
Article.}

\markright{Sec. 29B-3. Duty to enforce this Article.}

(a) It is the duty of the Illinois State Police, and its agents,
officers, and investigators, to enforce this Article, except those
provisions otherwise specifically delegated, and to cooperate with all
agencies charged with the enforcement of the laws of the United States,
or of any state, relating to money laundering. Only an agent, officer,
or investigator designated by the Director may be authorized in
accordance with this Section to serve seizure notices, warrants,
subpoenas, and summonses under the authority of this State.

(b) An agent, officer, investigator, or peace officer designated by the
Director may: (1) make seizure of property under this Article; and (2)
perform other law enforcement duties as the Director designates. It is
the duty of all State's Attorneys to prosecute violations of this
Article and institute legal proceedings as authorized under this
Article.

(Source: P.A. 102-538, eff. 8-20-21.)

\hypertarget{ilcs-529b-4}{%
\subsection*{(720 ILCS 5/29B-4)}\label{ilcs-529b-4}}
\addcontentsline{toc}{subsection}{(720 ILCS 5/29B-4)}

\hypertarget{sec.-29b-4.-protective-orders-and-warrants-for-forfeiture-purposes.}{%
\section*{Sec. 29B-4. Protective orders and warrants for forfeiture
purposes.}\label{sec.-29b-4.-protective-orders-and-warrants-for-forfeiture-purposes.}}
\addcontentsline{toc}{section}{Sec. 29B-4. Protective orders and
warrants for forfeiture purposes.}

\markright{Sec. 29B-4. Protective orders and warrants for forfeiture
purposes.}

(a) Upon application of the State, the court may enter a restraining
order or injunction, require the execution of a satisfactory performance
bond, or take any other action to preserve the availability of property
described in Section 29B-5 of this Article for forfeiture under this
Article:

(1) upon the filing of an indictment, information, or complaint charging
a violation of this Article for which forfeiture may be ordered under
this Article and alleging that the property with respect to which the
order is sought would be subject to forfeiture under this Article; or

(2) prior to the filing of the indictment, information, or complaint,
if, after notice to persons appearing to have an interest in the
property and opportunity for a hearing, the court determines that:

(A) there is probable cause to believe that the

State will prevail on the issue of forfeiture and that failure to enter
the order will result in the property being destroyed, removed from the
jurisdiction of the court, or otherwise made unavailable for forfeiture;
and

(B) the need to preserve the availability of the property through the
entry of the requested order outweighs the hardship on any party against
whom the order is to be entered.

Provided, however, that an order entered under paragraph (2) of this
Section shall be effective for not more than 90 days, unless extended by
the court for good cause shown or unless an indictment, information,
complaint, or administrative notice has been filed.

(b) A temporary restraining order under this subsection (b) may be
entered upon application of the State without notice or opportunity for
a hearing when an indictment, information, complaint, or administrative
notice has not yet been filed with respect to the property, if the State
demonstrates that there is probable cause to believe that the property
with respect to which the order is sought would be subject to forfeiture
under this Article and that provision of notice will jeopardize the
availability of the property for forfeiture. The temporary order shall
expire not more than 30 days after the date on which it is entered,
unless extended for good cause shown or unless the party against whom it
is entered consents to an extension for a longer period. A hearing
requested concerning an order entered under this subsection (b) shall be
held at the earliest possible time and prior to the expiration of the
temporary order.

(c) The court may receive and consider, at a hearing held under this
Section, evidence and information that would be inadmissible under the
Illinois rules of evidence.

(d) Under its authority to enter a pretrial restraining order under this
Section, the court may order a defendant to repatriate any property that
may be seized and forfeited and to deposit that property pending trial
with the Illinois State Police or another law enforcement agency
designated by the Illinois State Police. Failure to comply with an order
under this Section is punishable as a civil or criminal contempt of
court.

(e) The State may request the issuance of a warrant authorizing the
seizure of property described in Section 29B-5 of this Article in the
same manner as provided for a search warrant. If the court determines
that there is probable cause to believe that the property to be seized
would be subject to forfeiture, the court shall issue a warrant
authorizing the seizure of that property.

(Source: P.A. 102-538, eff. 8-20-21.)

\hypertarget{ilcs-529b-5}{%
\subsection*{(720 ILCS 5/29B-5)}\label{ilcs-529b-5}}
\addcontentsline{toc}{subsection}{(720 ILCS 5/29B-5)}

\hypertarget{sec.-29b-5.-property-subject-to-forfeiture.}{%
\section*{Sec. 29B-5. Property subject to
forfeiture.}\label{sec.-29b-5.-property-subject-to-forfeiture.}}
\addcontentsline{toc}{section}{Sec. 29B-5. Property subject to
forfeiture.}

\markright{Sec. 29B-5. Property subject to forfeiture.}

The following are subject to forfeiture:

(1) any property, real or personal, constituting, derived from, or
traceable to any proceeds the person obtained, directly or indirectly,
as a result of a violation of this Article;

(2) any of the person's property used, or intended to be used, in any
manner or part, to commit, or to facilitate the commission of, a
violation of this Article;

(3) all conveyances, including aircraft, vehicles, or vessels, which are
used, or intended for use, to transport, or in any manner to facilitate
the transportation, sale, receipt, possession, or concealment of
property described in paragraphs (1) and (2) of this Section, but:

(A) no conveyance used by any person as a common carrier in the
transaction of business as a common carrier is subject to forfeiture
under this Section unless it appears that the owner or other person in
charge of the conveyance is a consenting party or privy to a violation
of this Article;

(B) no conveyance is subject to forfeiture under this Article by reason
of any act or omission which the owner proves to have been committed or
omitted without his or her knowledge or consent;

(C) a forfeiture of a conveyance encumbered by a bona fide security
interest is subject to the interest of the secured party if he or she
neither had knowledge of nor consented to the act or omission;

(4) all real property, including any right, title, and interest,
including, but not limited to, any leasehold interest or the beneficial
interest in a land trust, in the whole of any lot or tract of land and
any appurtenances or improvements, which is used or intended to be used,
in any manner or part, to commit, or in any manner to facilitate the
commission of, any violation of this Article or that is the proceeds of
any violation or act that constitutes a violation of this Article.

(Source: P.A. 100-699, eff. 8-3-18; 100-1163, eff. 12-20-18.)

\hypertarget{ilcs-529b-6}{%
\subsection*{(720 ILCS 5/29B-6)}\label{ilcs-529b-6}}
\addcontentsline{toc}{subsection}{(720 ILCS 5/29B-6)}

\hypertarget{sec.-29b-6.-seizure.}{%
\section*{Sec. 29B-6. Seizure.}\label{sec.-29b-6.-seizure.}}
\addcontentsline{toc}{section}{Sec. 29B-6. Seizure.}

\markright{Sec. 29B-6. Seizure.}

(a) Property subject to forfeiture under this Article may be seized by
the Director or any peace officer upon process or seizure warrant issued
by any court having jurisdiction over the property. Seizure by the
Director or any peace officer without process may be made:

(1) if the seizure is incident to a seizure warrant;

(2) if the property subject to seizure has been the subject of a prior
judgment in favor of the State in a criminal proceeding, or in an
injunction or forfeiture proceeding based upon this Article;

(3) if there is probable cause to believe that the property is directly
or indirectly dangerous to health or safety;

(4) if there is probable cause to believe that the property is subject
to forfeiture under this Article and the property is seized under
circumstances in which a warrantless seizure or arrest would be
reasonable; or

(5) in accordance with the Code of Criminal Procedure of 1963.

(b) In the event of seizure under subsection (a) of this Section,
forfeiture proceedings shall be instituted in accordance with this
Article.

(c) Actual physical seizure of real property subject to forfeiture
requires the issuance of a seizure warrant. Nothing in this Article
prohibits the constructive seizure of real property through the filing
of a complaint for forfeiture in circuit court and the recording of a
lis pendens against the real property that is subject to forfeiture
without any hearing, warrant application, or judicial approval.

(Source: P.A. 100-699, eff. 8-3-18.)

\hypertarget{ilcs-529b-7}{%
\subsection*{(720 ILCS 5/29B-7)}\label{ilcs-529b-7}}
\addcontentsline{toc}{subsection}{(720 ILCS 5/29B-7)}

\hypertarget{sec.-29b-7.-safekeeping-of-seized-property-pending-disposition.}{%
\section*{Sec. 29B-7. Safekeeping of seized property pending
disposition.}\label{sec.-29b-7.-safekeeping-of-seized-property-pending-disposition.}}
\addcontentsline{toc}{section}{Sec. 29B-7. Safekeeping of seized
property pending disposition.}

\markright{Sec. 29B-7. Safekeeping of seized property pending
disposition.}

(a) If property is seized under this Article, the seizing agency shall
promptly conduct an inventory of the seized property and estimate the
property's value and shall forward a copy of the inventory of seized
property and the estimate of the property's value to the Director. Upon
receiving notice of seizure, the Director may:

(1) place the property under seal;

(2) remove the property to a place designated by the

Director;

(3) keep the property in the possession of the seizing agency;

(4) remove the property to a storage area for safekeeping or, if the
property is a negotiable instrument or money and is not needed for
evidentiary purposes, deposit it in an interest bearing account;

(5) place the property under constructive seizure by posting notice of
pending forfeiture on it, by giving notice of pending forfeiture to its
owners and interest holders, or by filing notice of pending forfeiture
in any appropriate public record relating to the property; or

(6) provide for another agency or custodian, including an owner, secured
party, or lienholder, to take custody of the property upon the terms and
conditions set by the Director.

(b) When property is forfeited under this Article, the Director shall
sell all the property unless the property is required by law to be
destroyed or is harmful to the public and shall distribute the proceeds
of the sale, together with any moneys forfeited or seized, under Section
29B-26 of this Article.

(Source: P.A. 100-699, eff. 8-3-18; 100-1163, eff. 12-20-18.)

\hypertarget{ilcs-529b-8}{%
\subsection*{(720 ILCS 5/29B-8)}\label{ilcs-529b-8}}
\addcontentsline{toc}{subsection}{(720 ILCS 5/29B-8)}

\hypertarget{sec.-29b-8.-notice-to-states-attorney.}{%
\section*{Sec. 29B-8. Notice to State's
Attorney.}\label{sec.-29b-8.-notice-to-states-attorney.}}
\addcontentsline{toc}{section}{Sec. 29B-8. Notice to State's Attorney.}

\markright{Sec. 29B-8. Notice to State's Attorney.}

The law enforcement agency seizing property for forfeiture under this
Article shall, within 60 days after seizure, notify the State's Attorney
for the county, either where an act or omission giving rise to the
forfeiture occurred or where the property was seized, of the seizure of
the property and the facts and circumstances giving rise to the seizure
and shall provide the State's Attorney with the inventory of the
property and its estimated value. If the property seized for forfeiture
is a vehicle, the law enforcement agency seizing the property shall
immediately notify the Secretary of State that forfeiture proceedings
are pending regarding the vehicle. This notice shall be by Form 4-64.

(Source: P.A. 100-699, eff. 8-3-18.)

\hypertarget{ilcs-529b-9}{%
\subsection*{(720 ILCS 5/29B-9)}\label{ilcs-529b-9}}
\addcontentsline{toc}{subsection}{(720 ILCS 5/29B-9)}

\hypertarget{sec.-29b-9.-preliminary-review.}{%
\section*{Sec. 29B-9. Preliminary
review.}\label{sec.-29b-9.-preliminary-review.}}
\addcontentsline{toc}{section}{Sec. 29B-9. Preliminary review.}

\markright{Sec. 29B-9. Preliminary review.}

(a) Within 28 days of the seizure, the State shall seek a preliminary
determination from the circuit court as to whether there is probable
cause that the property may be subject to forfeiture.

(b) The rules of evidence shall not apply to any proceeding conducted
under this Section.

(c) The court may conduct the review under subsection (a) of this
Section simultaneously with a proceeding under Section 109-1 of the Code
of Criminal Procedure of 1963 for a related criminal offense if a
prosecution is commenced by information or complaint.

(d) The court may accept a finding of probable cause at a preliminary
hearing following the filing of an information or complaint charging a
related criminal offense or following the return of indictment by a
grand jury charging the related offense as sufficient evidence of
probable cause as required under subsection (a) of this Section.

(e) Upon a finding of probable cause as required under this Section, the
circuit court shall order the property subject to the applicable
forfeiture Act held until the conclusion of any forfeiture proceeding.

(Source: P.A. 100-699, eff. 8-3-18.)

\hypertarget{ilcs-529b-10}{%
\subsection*{(720 ILCS 5/29B-10)}\label{ilcs-529b-10}}
\addcontentsline{toc}{subsection}{(720 ILCS 5/29B-10)}

\hypertarget{sec.-29b-10.-notice-to-owner-or-interest-holder.}{%
\section*{Sec. 29B-10. Notice to owner or interest
holder.}\label{sec.-29b-10.-notice-to-owner-or-interest-holder.}}
\addcontentsline{toc}{section}{Sec. 29B-10. Notice to owner or interest
holder.}

\markright{Sec. 29B-10. Notice to owner or interest holder.}

(a) The first attempted service of notice shall be commenced within 28
days of the latter of filing of the verified claim or the receipt of the
notice from the seizing agency by Form 4-64. A complaint for forfeiture
or a notice of pending forfeiture shall be served on a claimant if the
owner's or interest holder's name and current address are known, then by
either: (1) personal service; or (2) mailing a copy of the notice by
certified mail, return receipt requested, and first class mail to that
address.

(b) If no signed return receipt is received by the State's Attorney
within 28 days of mailing or no communication from the owner or interest
holder is received by the State's Attorney documenting actual notice by
the parties, the State's Attorney shall, within a reasonable period of
time, mail a second copy of the notice by certified mail, return receipt
requested, and first class mail to that address. If no signed return
receipt is received by the State's Attorney within 28 days of the second
mailing, or no communication from the owner or interest holder is
received by the State's Attorney documenting actual notice by the
parties, the State's Attorney shall have 60 days to attempt to serve the
notice by personal service, including substitute service by leaving a
copy at the usual place of abode with some person of the family or a
person residing there, of the age of 13 years or upwards. If, after 3
attempts at service in this manner, no service of the notice is
accomplished, the notice shall be posted in a conspicuous manner at the
address and service shall be made by the posting. The attempts at
service and the posting, if required, shall be documented by the person
attempting service which shall be made part of a return of service
returned to the State's Attorney. The State's Attorney may utilize any
Sheriff or Deputy Sheriff, a peace officer, a private process server or
investigator, or an employee, agent, or investigator of the State's
Attorney's Office to attempt service without seeking leave of court.

(c) After the procedures listed are followed, service shall be effective
on the owner or interest holder on the date of receipt by the State's
Attorney of a return receipt, or on the date of receipt of a
communication from an owner or interest holder documenting actual
notice, whichever is first in time, or on the date of the last act
performed by the State's Attorney in attempting personal service. For
purposes of notice under this Section, if a person has been arrested for
the conduct giving rise to the forfeiture, the address provided to the
arresting agency at the time of arrest shall be deemed to be that
person's known address. Provided, however, if an owner or interest
holder's address changes prior to the effective date of the notice of
pending forfeiture, the owner or interest holder shall promptly notify
the seizing agency of the change in address or, if the owner or interest
holder's address changes subsequent to the effective date of the notice
of pending forfeiture, the owner or interest holder shall promptly
notify the State's Attorney of the change in address. If the property
seized is a conveyance, notice shall also be directed to the address
reflected in the office of the agency or official in which title to or
interest in the conveyance is required by law to be recorded.

(d) If the owner's or interest holder's address is not known, and is not
on record as provided in this Section, service by publication for 3
successive weeks in a newspaper of general circulation in the county in
which the seizure occurred shall suffice for service requirements.

(e) Notice to any business entity, corporation, limited liability
company, limited liability partnership, or partnership shall be
completed by a single mailing of a copy of the notice by certified mail,
return receipt requested, and first class mail to that address. This
notice is complete regardless of the return of a signed return receipt.

(f) Notice to a person whose address is not within the State shall be
completed by a single mailing of a copy of the notice by certified mail,
return receipt requested, and first class mail to that address. This
notice is complete regardless of the return of a signed return receipt.

(g) Notice to a person whose address is not within the United States
shall be completed by a single mailing of a copy of the notice by
certified mail, return receipt requested, and first class mail to that
address. This notice is complete regardless of the return of a signed
return receipt. If certified mail is not available in the foreign
country where the person has an address, notice shall proceed by
publication requirements under subsection (d) of this Section.

(h) Notice to a person whom the State's Attorney reasonably should know
is incarcerated within this State shall also include mailing a copy of
the notice by certified mail, return receipt requested, and first class
mail to the address of the detention facility with the inmate's name
clearly marked on the envelope.

(i) After a claimant files a verified claim with the State's Attorney
and provides an address at which the claimant will accept service, the
complaint shall be served and notice shall be complete upon the mailing
of the complaint to the claimant at the address the claimant provided
via certified mail, return receipt requested, and first class mail. No
return receipt need be received, or any other attempts at service need
be made to comply with service and notice requirements under this
Section. This certified mailing, return receipt requested, shall be
proof of service of the complaint on the claimant. If notice is to be
shown by actual notice from communication with a claimant, then the
State's Attorney shall file an affidavit as proof of service, providing
details of the communication, which shall be accepted as proof of
service by the court.

(j) If the property seized is a conveyance, notice shall also be
directed to the address reflected in the office of the agency or
official in which title to or interest in the conveyance is required by
law to be recorded by mailing a copy of the notice by certified mail,
return receipt requested, to that address.

(k) Notice served under this Article is effective upon personal service,
the last date of publication, or the mailing of written notice,
whichever is earlier.

(Source: P.A. 100-699, eff. 8-3-18; 100-1163, eff. 12-20-18.)

\hypertarget{ilcs-529b-11}{%
\subsection*{(720 ILCS 5/29B-11)}\label{ilcs-529b-11}}
\addcontentsline{toc}{subsection}{(720 ILCS 5/29B-11)}

\hypertarget{sec.-29b-11.-replevin-prohibited.}{%
\section*{Sec. 29B-11. Replevin
prohibited.}\label{sec.-29b-11.-replevin-prohibited.}}
\addcontentsline{toc}{section}{Sec. 29B-11. Replevin prohibited.}

\markright{Sec. 29B-11. Replevin prohibited.}

Property taken or detained under this Article shall not be subject to
replevin, but is deemed to be in the custody of the Director subject
only to the order and judgments of the circuit court having jurisdiction
over the forfeiture proceedings and the decisions of the State's
Attorney under this Article.

(Source: P.A. 100-699, eff. 8-3-18.)

\hypertarget{ilcs-529b-12}{%
\subsection*{(720 ILCS 5/29B-12)}\label{ilcs-529b-12}}
\addcontentsline{toc}{subsection}{(720 ILCS 5/29B-12)}

\hypertarget{sec.-29b-12.-non-judicial-forfeiture.}{%
\section*{Sec. 29B-12. Non-judicial
forfeiture.}\label{sec.-29b-12.-non-judicial-forfeiture.}}
\addcontentsline{toc}{section}{Sec. 29B-12. Non-judicial forfeiture.}

\markright{Sec. 29B-12. Non-judicial forfeiture.}

If non-real property that exceeds \$20,000 in value excluding the value
of any conveyance, or if real property is seized under the provisions of
this Article, the State's Attorney shall institute judicial in rem
forfeiture proceedings as described in Section 29B-13 of this Article
within 28 days from receipt of notice of seizure from the seizing agency
under Section 29B-8 of this Article. However, if non-real property that
does not exceed \$20,000 in value excluding the value of any conveyance
is seized, the following procedure shall be used:

(1) If, after review of the facts surrounding the seizure, the State's
Attorney is of the opinion that the seized property is subject to
forfeiture, then, within 28 days after the receipt of notice of seizure
from the seizing agency, the State's Attorney shall cause notice of
pending forfeiture to be given to the owner of the property and all
known interest holders of the property in accordance with Section 29B-10
of this Article.

(2) The notice of pending forfeiture shall include a description of the
property, the estimated value of the property, the date and place of
seizure, the conduct giving rise to forfeiture or the violation of law
alleged, and a summary of procedures and procedural rights applicable to
the forfeiture action.

(3)(A) Any person claiming an interest in property that is the subject
of notice under paragraph (1) of this Section, must, in order to
preserve any rights or claims to the property, within 45 days after the
effective date of notice as described in Section 29B-10 of this Article,
file a verified claim with the State's Attorney expressing his or her
interest in the property. The claim shall set forth:

(i) the caption of the proceedings as set forth on the notice of pending
forfeiture and the name of the claimant;

(ii) the address at which the claimant will accept mail;

(iii) the nature and extent of the claimant's interest in the property;

(iv) the date, identity of the transferor, and circumstances of the
claimant's acquisition of the interest in the property;

(v) the names and addresses of all other persons known to have an
interest in the property;

(vi) the specific provision of law relied on in asserting the property
is not subject to forfeiture;

(vii) all essential facts supporting each assertion; and

(viii) the relief sought.

(B) If a claimant files the claim, then the State's

Attorney shall institute judicial in rem forfeiture proceedings with the
clerk of the court as described in Section 29B-13 of this Article within
28 days after receipt of the claim.

(4) If no claim is filed within the 28-day period as described in
paragraph (3) of this Section, the State's Attorney shall declare the
property forfeited and shall promptly notify the owner and all known
interest holders of the property and the Director of the Illinois State
Police of the declaration of forfeiture and the Director shall dispose
of the property in accordance with law.

(Source: P.A. 102-538, eff. 8-20-21.)

\hypertarget{ilcs-529b-13}{%
\subsection*{(720 ILCS 5/29B-13)}\label{ilcs-529b-13}}
\addcontentsline{toc}{subsection}{(720 ILCS 5/29B-13)}

\hypertarget{sec.-29b-13.-judicial-in-rem-procedures.}{%
\section*{Sec. 29B-13. Judicial in rem
procedures.}\label{sec.-29b-13.-judicial-in-rem-procedures.}}
\addcontentsline{toc}{section}{Sec. 29B-13. Judicial in rem procedures.}

\markright{Sec. 29B-13. Judicial in rem procedures.}

If property seized under this Article is non-real property that exceeds
\$20,000 in value excluding the value of any conveyance, or is real
property, or a claimant has filed a claim under paragraph (3) of Section
29B-12 of this Article, the following judicial in rem procedures shall
apply:

(1) If, after a review of the facts surrounding the seizure, the State's
Attorney is of the opinion that the seized property is subject to
forfeiture, then, within 28 days of the receipt of notice of seizure by
the seizing agency or the filing of the claim, whichever is later, the
State's Attorney shall institute judicial forfeiture proceedings by
filing a verified complaint for forfeiture. If authorized by law, a
forfeiture shall be ordered by a court on an action in rem brought by a
State's Attorney under a verified complaint for forfeiture.

(2) A complaint of forfeiture shall include:

(A) a description of the property seized;

(B) the date and place of seizure of the property;

(C) the name and address of the law enforcement agency making the
seizure; and

(D) the specific statutory and factual grounds for the seizure.

(3) The complaint shall be served upon the person from whom the property
was seized and all persons known or reasonably believed by the State to
claim an interest in the property, as provided in Section 29B-10 of this
Article. The complaint shall be accompanied by the following written
notice:

``This is a civil court proceeding subject to the Code of Civil
Procedure. You received this Complaint of Forfeiture because the State's
Attorney's office has brought a legal action seeking forfeiture of your
seized property. This complaint starts the court process where the State
seeks to prove that your property should be forfeited and not returned
to you. This process is also your opportunity to try to prove to a judge
that you should get your property back. The complaint lists the date,
time, and location of your first court date. You must appear in court on
that day, or you may lose the case automatically. You must also file an
appearance and answer. If you are unable to pay the appearance fee, you
may qualify to have the fee waived. If there is a criminal case related
to the seizure of your property, your case may be set for trial after
the criminal case has been resolved. Before trial, the judge may allow
discovery, where the State can ask you to respond in writing to
questions and give them certain documents, and you can make similar
requests of the State. The trial is your opportunity to explain what
happened when your property was seized and why you should get the
property back.''

(4) Forfeiture proceedings under this Article shall be subject to the
Code of Civil Procedure and the rules of evidence relating to civil
actions shall apply to proceedings under this Article with the following
exception. The parties shall be allowed to use, and the court shall
receive and consider, all relevant hearsay evidence that relates to
evidentiary foundation, chain of custody, business records, recordings,
laboratory analysis, laboratory reports, and relevant hearsay related to
the use of technology in the investigation that resulted in the seizure
of property that is subject to the forfeiture action.

(5) Only an owner of or interest holder in the property may file an
answer asserting a claim against the property in the action in rem. For
purposes of this Section, the owner or interest holder shall be referred
to as claimant. Upon motion of the State, the court shall first hold a
hearing, in which a claimant shall establish by a preponderance of the
evidence, that he or she has a lawful, legitimate ownership interest in
the property and that it was obtained through a lawful source.

(6) The answer must be signed by the owner or interest holder under
penalty of perjury and shall set forth:

(A) the caption of the proceedings as set forth on the notice of pending
forfeiture and the name of the claimant;

(B) the address at which the claimant will accept mail;

(C) the nature and extent of the claimant's interest in the property;

(D) the date, identity of transferor, and circumstances of the
claimant's acquisition of the interest in the property;

(E) the names and addresses of all other persons known to have an
interest in the property;

(F) all essential facts supporting each assertion;

(G) the precise relief sought; and

(H) in a forfeiture action involving currency or its equivalent, a
claimant shall provide the State with notice of his or her intent to
allege that the currency or its equivalent is not related to the alleged
factual basis for the forfeiture, and why.

The answer shall follow the rules under the Code of

Civil Procedure.

(7) The answer shall be filed with the court within

45 days after service of the civil in rem complaint.

(8) The hearing shall be held within 60 days after filing of the answer
unless continued for good cause.

(9) At the judicial in rem proceeding, in the State's case in chief, the
State shall show by a preponderance of the evidence that the property is
subject to forfeiture. If the State makes such a showing, the claimant
shall have the burden of production to set forth evidence that the
property is not related to the alleged factual basis of the forfeiture.
After this production of evidence, the State shall maintain the burden
of proof to overcome this assertion. A claimant shall provide the State
notice of its intent to allege that the currency or its equivalent is
not related to the alleged factual basis of the forfeiture and why. As
to conveyances, at the judicial in rem proceeding, in its case in chief,
the State shall show by a preponderance of the evidence:

(A) that the property is subject to forfeiture; and

(B) at least one of the following:

(i) that the claimant was legally accountable for the conduct giving
rise to the forfeiture;

(ii) that the claimant knew or reasonably should have known of the
conduct giving rise to the forfeiture;

(iii) that the claimant knew or reasonably should have known that the
conduct giving rise to the forfeiture was likely to occur;

(iv) that the claimant held the property for the benefit of, or as
nominee for, any person whose conduct gave rise to its forfeiture;

(v) that if the claimant acquired the interest through any person
engaging in any of the conduct described above or conduct giving rise to
the forfeiture:

(a) the claimant did not acquire it as a bona fide purchaser for value;
or

(b) the claimant acquired the interest under the circumstances that the
claimant reasonably should have known the property was derived from, or
used in, the conduct giving rise to the forfeiture; or

(vi) that the claimant is not the true owner of the property that is
subject to forfeiture.

(10) If the State does not meet its burden to show that the property is
subject to forfeiture, the court shall order the interest in the
property returned or conveyed to the claimant and shall order all other
property forfeited to the State. If the State does meet its burden to
show that the property is subject to forfeiture, the court shall order
all property forfeited to the State.

(11) A defendant convicted in any criminal proceeding is precluded from
later denying the essential allegations of the criminal offense of which
the defendant was convicted in any proceeding under this Article
regardless of the pendency of an appeal from that conviction. However,
evidence of the pendency of an appeal is admissible.

(12) On a motion by the parties, the court may stay civil forfeiture
proceedings during the criminal trial for a related criminal indictment
or information alleging a money laundering violation. Such a stay shall
not be available pending an appeal. Property subject to forfeiture under
this Article shall not be subject to return or release by a court
exercising jurisdiction over a criminal case involving the seizure of
the property unless the return or release is consented to by the State's
Attorney.

(Source: P.A. 100-699, eff. 8-3-18; 100-1163, eff. 12-20-18.)

\hypertarget{ilcs-529b-14}{%
\subsection*{(720 ILCS 5/29B-14)}\label{ilcs-529b-14}}
\addcontentsline{toc}{subsection}{(720 ILCS 5/29B-14)}

\hypertarget{sec.-29b-14.-innocent-owner-hearing.}{%
\section*{Sec. 29B-14. Innocent owner
hearing.}\label{sec.-29b-14.-innocent-owner-hearing.}}
\addcontentsline{toc}{section}{Sec. 29B-14. Innocent owner hearing.}

\markright{Sec. 29B-14. Innocent owner hearing.}

(a) After a complaint for forfeiture has been filed and all claimants
have appeared and answered, a claimant may file a motion with the court
for an innocent owner hearing prior to trial. This motion shall be made
and supported by sworn affidavit and shall assert the following along
with specific facts that support each assertion:

(1) that the claimant filing the motion is the true owner of the
conveyance as interpreted by case law;

(2) that the claimant was not legally accountable for the conduct giving
rise to the forfeiture or acquiesced in the conduct;

(3) that the claimant did not solicit, conspire, or attempt to commit
the conduct giving rise to the forfeiture;

(4) that the claimant did not know or did not have reason to know that
the conduct giving rise to the forfeiture was likely to occur; and

(5) that the claimant did not hold the property for the benefit of, or
as nominee for, any person whose conduct gave rise to its forfeiture, or
if the claimant acquired the interest through any person, the claimant
acquired it as a bona fide purchaser for value or acquired the interest
without knowledge of the seizure of the property for forfeiture.

(b) The claimant's motion shall include specific facts supporting these
assertions.

(c) Upon this filing, a hearing may only be conducted after the parties
have been given the opportunity to conduct limited discovery as to the
ownership and control of the property, the claimant's knowledge, or any
matter relevant to the issues raised or facts alleged in the claimant's
motion. Discovery shall be limited to the People's requests in these
areas but may proceed by any means allowed in the Code of Civil
Procedure.

(1) After discovery is complete and the court has allowed for sufficient
time to review and investigate the discovery responses, the court shall
conduct a hearing. At the hearing, the fact that the conveyance is
subject to forfeiture shall not be at issue. The court shall only hear
evidence relating to the issue of innocent ownership.

(2) At the hearing on the motion, it shall be the burden of the claimant
to prove each of the assertions listed in subsection (a) of this Section
by a preponderance of the evidence.

(3) If a claimant meets his or her burden of proof, the court shall
grant the motion and order the property returned to the claimant. If the
claimant fails to meet his or her burden of proof, then the court shall
deny the motion and the forfeiture case shall proceed according to the
Code of Civil Procedure.

(Source: P.A. 100-699, eff. 8-3-18; 100-1163, eff. 12-20-18.)

\hypertarget{ilcs-529b-15}{%
\subsection*{(720 ILCS 5/29B-15)}\label{ilcs-529b-15}}
\addcontentsline{toc}{subsection}{(720 ILCS 5/29B-15)}

\hypertarget{sec.-29b-15.-burden-and-commencement-of-forfeiture-action.}{%
\section*{Sec. 29B-15. Burden and commencement of forfeiture
action.}\label{sec.-29b-15.-burden-and-commencement-of-forfeiture-action.}}
\addcontentsline{toc}{section}{Sec. 29B-15. Burden and commencement of
forfeiture action.}

\markright{Sec. 29B-15. Burden and commencement of forfeiture action.}

(a) Notwithstanding any other provision of this Article, the State's
burden of proof at the trial of the forfeiture action shall be by clear
and convincing evidence if:

(1) a finding of not guilty is entered as to all counts and all
defendants in a criminal proceeding relating to the conduct giving rise
to the forfeiture action; or

(2) the State receives an adverse finding at a preliminary hearing and
fails to secure an indictment in a criminal proceeding relating to the
factual allegations of the forfeiture action.

(b) All property declared forfeited under this Article vests in the
State on the commission of the conduct giving rise to forfeiture
together with the proceeds of the property after that time. Except as
otherwise provided in this Article, title to any property or proceeds
subject to forfeiture subsequently transferred to any person remain
subject to forfeiture and thereafter shall be ordered forfeited unless
the person to whom the property was transferred makes an appropriate
claim and has his or her claim adjudicated at the judicial in rem
hearing.

(c) A civil action under this Article shall be commenced within 5 years
after the last conduct giving rise to forfeiture became known or should
have become known or 5 years after the forfeitable property is
discovered, whichever is later, excluding any time during which either
the property or claimant is out of the State or in confinement or during
which criminal proceedings relating to the same conduct are in progress.

(Source: P.A. 100-699, eff. 8-3-18.)

\hypertarget{ilcs-529b-16}{%
\subsection*{(720 ILCS 5/29B-16)}\label{ilcs-529b-16}}
\addcontentsline{toc}{subsection}{(720 ILCS 5/29B-16)}

\hypertarget{sec.-29b-16.-joint-tenancy-or-tenancy-in-common.}{%
\section*{Sec. 29B-16. Joint tenancy or tenancy in
common.}\label{sec.-29b-16.-joint-tenancy-or-tenancy-in-common.}}
\addcontentsline{toc}{section}{Sec. 29B-16. Joint tenancy or tenancy in
common.}

\markright{Sec. 29B-16. Joint tenancy or tenancy in common.}

If property is ordered forfeited under this Section from a claimant who
held title to the property in joint tenancy or tenancy in common with
another claimant, the court shall determine the amount of each owner's
interest in the property according to principles of property law.

(Source: P.A. 100-699, eff. 8-3-18.)

\hypertarget{ilcs-529b-17}{%
\subsection*{(720 ILCS 5/29B-17)}\label{ilcs-529b-17}}
\addcontentsline{toc}{subsection}{(720 ILCS 5/29B-17)}

\hypertarget{sec.-29b-17.-exception-for-bona-fide-purchasers.}{%
\section*{Sec. 29B-17. Exception for bona fide
purchasers.}\label{sec.-29b-17.-exception-for-bona-fide-purchasers.}}
\addcontentsline{toc}{section}{Sec. 29B-17. Exception for bona fide
purchasers.}

\markright{Sec. 29B-17. Exception for bona fide purchasers.}

No property shall be forfeited under this Article from a person who,
without actual or constructive notice that the property was the subject
of forfeiture proceedings, obtained possession of the property as a bona
fide purchaser for value. A person who purports to effect transfer of
property after receiving actual or constructive notice that the property
is subject to seizure or forfeiture is guilty of contempt of court and
shall be liable to the State for a penalty in the amount of the fair
market value of the property.

(Source: P.A. 100-699, eff. 8-3-18; 100-1163, eff. 12-20-18.)

\hypertarget{ilcs-529b-18}{%
\subsection*{(720 ILCS 5/29B-18)}\label{ilcs-529b-18}}
\addcontentsline{toc}{subsection}{(720 ILCS 5/29B-18)}

\hypertarget{sec.-29b-18.-proportionality.}{%
\section*{Sec. 29B-18.
Proportionality.}\label{sec.-29b-18.-proportionality.}}
\addcontentsline{toc}{section}{Sec. 29B-18. Proportionality.}

\markright{Sec. 29B-18. Proportionality.}

Property that is forfeited shall be subject to an 8th Amendment to the
United States Constitution disproportionate penalties analysis and the
property forfeiture may be denied in whole or in part if the court finds
that the forfeiture would constitute an excessive fine in violation of
the 8th Amendment as interpreted by case law.

(Source: P.A. 100-699, eff. 8-3-18.)

\hypertarget{ilcs-529b-19}{%
\subsection*{(720 ILCS 5/29B-19)}\label{ilcs-529b-19}}
\addcontentsline{toc}{subsection}{(720 ILCS 5/29B-19)}

\hypertarget{sec.-29b-19.-stay-of-time-periods.}{%
\section*{Sec. 29B-19. Stay of time
periods.}\label{sec.-29b-19.-stay-of-time-periods.}}
\addcontentsline{toc}{section}{Sec. 29B-19. Stay of time periods.}

\markright{Sec. 29B-19. Stay of time periods.}

If property is seized for evidence and for forfeiture, the time periods
for instituting judicial and non-judicial forfeiture proceedings shall
not begin until the property is no longer necessary for evidence.

(Source: P.A. 100-699, eff. 8-3-18.)

\hypertarget{ilcs-529b-20}{%
\subsection*{(720 ILCS 5/29B-20)}\label{ilcs-529b-20}}
\addcontentsline{toc}{subsection}{(720 ILCS 5/29B-20)}

\hypertarget{sec.-29b-20.-settlement-of-claims.}{%
\section*{Sec. 29B-20. Settlement of
claims.}\label{sec.-29b-20.-settlement-of-claims.}}
\addcontentsline{toc}{section}{Sec. 29B-20. Settlement of claims.}

\markright{Sec. 29B-20. Settlement of claims.}

Notwithstanding other provisions of this Article, the State's Attorney
and a claimant of seized property may enter into an agreed-upon
settlement concerning the seized property in such an amount and upon
such terms as are set out in writing in a settlement agreement. All
proceeds from a settlement agreement shall be tendered to the Illinois
State Police and distributed under Section 29B-26 of this Article.

(Source: P.A. 102-538, eff. 8-20-21.)

\hypertarget{ilcs-529b-21}{%
\subsection*{(720 ILCS 5/29B-21)}\label{ilcs-529b-21}}
\addcontentsline{toc}{subsection}{(720 ILCS 5/29B-21)}

\hypertarget{sec.-29b-21.-attorneys-fees.}{%
\section*{Sec. 29B-21. Attorney's
fees.}\label{sec.-29b-21.-attorneys-fees.}}
\addcontentsline{toc}{section}{Sec. 29B-21. Attorney's fees.}

\markright{Sec. 29B-21. Attorney's fees.}

Nothing in this Article applies to property that constitutes reasonable
bona fide attorney's fees paid to an attorney for services rendered or
to be rendered in the forfeiture proceeding or criminal proceeding
relating directly thereto if the property was paid before its seizure
and before the issuance of any seizure warrant or court order
prohibiting transfer of the property and if the attorney, at the time he
or she received the property, did not know that it was property subject
to forfeiture under this Article.

(Source: P.A. 102-558, eff. 8-20-21.)

\hypertarget{ilcs-529b-22}{%
\subsection*{(720 ILCS 5/29B-22)}\label{ilcs-529b-22}}
\addcontentsline{toc}{subsection}{(720 ILCS 5/29B-22)}

\hypertarget{sec.-29b-22.-construction.}{%
\section*{Sec. 29B-22. Construction.}\label{sec.-29b-22.-construction.}}
\addcontentsline{toc}{section}{Sec. 29B-22. Construction.}

\markright{Sec. 29B-22. Construction.}

(a) It is the intent of the General Assembly that the forfeiture
provisions of this Article be liberally construed so as to effect their
remedial purpose. The forfeiture of property and other remedies under
this Article shall be considered to be in addition to, and not exclusive
of, any sentence or other remedy provided by law.

(b) The changes made to this Article by Public Act 100-512 and Public
Act 100-699 are subject to Section 2 of the Statute on Statutes.

(Source: P.A. 100-699, eff. 8-3-18; 100-1163, eff. 12-20-18.)

\hypertarget{ilcs-529b-23}{%
\subsection*{(720 ILCS 5/29B-23)}\label{ilcs-529b-23}}
\addcontentsline{toc}{subsection}{(720 ILCS 5/29B-23)}

\hypertarget{sec.-29b-23.-judicial-review.}{%
\section*{Sec. 29B-23. Judicial
review.}\label{sec.-29b-23.-judicial-review.}}
\addcontentsline{toc}{section}{Sec. 29B-23. Judicial review.}

\markright{Sec. 29B-23. Judicial review.}

If property has been declared forfeited under Section 29B-12 of this
Article, any person who has an interest in the property declared
forfeited may, within 30 days after the effective date of the notice of
the declaration of forfeiture, file a claim as described in paragraph
(3) of Section 29B-12 of this Article. If a claim is filed under this
Section, then the procedures described in Section of 29B-13 of this
Article apply.

(Source: P.A. 100-699, eff. 8-3-18.)

\hypertarget{ilcs-529b-24}{%
\subsection*{(720 ILCS 5/29B-24)}\label{ilcs-529b-24}}
\addcontentsline{toc}{subsection}{(720 ILCS 5/29B-24)}

\hypertarget{sec.-29b-24.-review-of-administrative-decisions.}{%
\section*{Sec. 29B-24. Review of administrative
decisions.}\label{sec.-29b-24.-review-of-administrative-decisions.}}
\addcontentsline{toc}{section}{Sec. 29B-24. Review of administrative
decisions.}

\markright{Sec. 29B-24. Review of administrative decisions.}

All administrative findings, rulings, final determinations, findings,
and conclusions of the State's Attorney's Office under this Article are
final and conclusive decisions of the matters involved. Any person
aggrieved by the decision may obtain review of the decision under the
provisions of the Administrative Review Law and the rules adopted under
that Law. Pending final decision on such review, the administrative
acts, orders, and rulings of the State's Attorney's Office remain in
full force and effect unless modified or suspended by order of court
pending final judicial decision. Pending final decision on such review,
the acts, orders, and rulings of the State's Attorney's Office remain in
full force and effect, unless stayed by order of court. However, no stay
of any decision of the administrative agency shall issue unless the
person aggrieved by the decision establishes by a preponderance of the
evidence that good cause exists for the stay. In determining good cause,
the court shall find that the aggrieved party has established a
substantial likelihood of prevailing on the merits and that granting the
stay will not have an injurious effect on the general public.

(Source: P.A. 100-699, eff. 8-3-18.)

\hypertarget{ilcs-529b-25}{%
\subsection*{(720 ILCS 5/29B-25)}\label{ilcs-529b-25}}
\addcontentsline{toc}{subsection}{(720 ILCS 5/29B-25)}

\hypertarget{sec.-29b-25.-return-of-property-damages-and-costs.}{%
\section*{Sec. 29B-25. Return of property, damages, and
costs.}\label{sec.-29b-25.-return-of-property-damages-and-costs.}}
\addcontentsline{toc}{section}{Sec. 29B-25. Return of property, damages,
and costs.}

\markright{Sec. 29B-25. Return of property, damages, and costs.}

(a) The law enforcement agency that holds custody of property seized for
forfeiture shall deliver property ordered by the court to be returned or
conveyed to the claimant within a reasonable time not to exceed 7 days,
unless the order is stayed by the trial court or a reviewing court
pending an appeal, motion to reconsider, or other reason.

(b) The law enforcement agency that holds custody of property is
responsible for any damages, storage fees, and related costs applicable
to property returned. The claimant shall not be subject to any charges
by the State for storage of the property or expenses incurred in the
preservation of the property. Charges for the towing of a conveyance
shall be borne by the claimant unless the conveyance was towed for the
sole reason of seizure for forfeiture. This Section does not prohibit
the imposition of any fees or costs by a home rule unit of local
government related to the impoundment of a conveyance under an ordinance
enacted by the unit of government.

(c) A law enforcement agency shall not retain forfeited property for its
own use or transfer the property to any person or entity, except as
provided under this Section. A law enforcement agency may apply in
writing to the Director of the Illinois State Police to request that
forfeited property be awarded to the agency for a specifically
articulated official law enforcement use in an investigation. The
Director shall provide a written justification in each instance
detailing the reasons why the forfeited property was placed into
official use and the justification shall be retained for a period of not
less than 3 years.

(d) A claimant or a party interested in personal property contained
within a seized conveyance may file a request with the State's Attorney
in a non-judicial forfeiture action, or a motion with the court in a
judicial forfeiture action for the return of any personal property
contained within a conveyance that is seized under this Article. The
return of personal property shall not be unreasonably withheld if the
personal property is not mechanically or electrically coupled to the
conveyance, needed for evidentiary purposes, or otherwise contraband.
Any law enforcement agency that returns property under a court order
under this Section shall not be liable to any person who claims
ownership to the property if it is returned to an improper party.

(Source: P.A. 102-538, eff. 8-20-21.)

\hypertarget{ilcs-529b-26}{%
\subsection*{(720 ILCS 5/29B-26)}\label{ilcs-529b-26}}
\addcontentsline{toc}{subsection}{(720 ILCS 5/29B-26)}

\hypertarget{sec.-29b-26.-distribution-of-proceeds.}{%
\section*{Sec. 29B-26. Distribution of
proceeds.}\label{sec.-29b-26.-distribution-of-proceeds.}}
\addcontentsline{toc}{section}{Sec. 29B-26. Distribution of proceeds.}

\markright{Sec. 29B-26. Distribution of proceeds.}

All moneys and the sale proceeds of all other property forfeited and
seized under this Article shall be distributed as follows:

(1) 65\% shall be distributed to the metropolitan enforcement group,
local, municipal, county, or State law enforcement agency or agencies
that conducted or participated in the investigation resulting in the
forfeiture. The distribution shall bear a reasonable relationship to the
degree of direct participation of the law enforcement agency in the
effort resulting in the forfeiture, taking into account the total value
of the property forfeited and the total law enforcement effort with
respect to the violation of the law upon which the forfeiture is based.
Amounts distributed to the agency or agencies shall be used for the
enforcement of laws.

(2)(i) 12.5\% shall be distributed to the Office of the State's Attorney
of the county in which the prosecution resulting in the forfeiture was
instituted, deposited in a special fund in the county treasury and
appropriated to the State's Attorney for use in the enforcement of laws.
In counties over 3,000,000 population, 25\% shall be distributed to the
Office of the State's Attorney for use in the enforcement of laws. If
the prosecution is undertaken solely by the Attorney General, the
portion provided under this subparagraph (i) shall be distributed to the
Attorney General for use in the enforcement of laws.

(ii) 12.5\% shall be distributed to the Office of the

State's Attorneys Appellate Prosecutor and deposited in the Narcotics
Profit Forfeiture Fund of that office to be used for additional expenses
incurred in the investigation, prosecution, and appeal of cases arising
under laws. The Office of the State's Attorneys Appellate Prosecutor
shall not receive distribution from cases brought in counties with over
3,000,000 population.

(3) 10\% shall be retained by the Illinois State

Police for expenses related to the administration and sale of seized and
forfeited property.

Moneys and the sale proceeds distributed to the Illinois State Police
under this Article shall be deposited in the Money Laundering Asset
Recovery Fund created in the State treasury and shall be used by the
Illinois State Police for State law enforcement purposes. All moneys and
sale proceeds of property forfeited and seized under this Article and
distributed according to this Section may also be used to purchase
opioid antagonists as defined in Section 5-23 of the Substance Use
Disorder Act.

(Source: P.A. 102-538, eff. 8-20-21.)

\hypertarget{ilcs-529b-27}{%
\subsection*{(720 ILCS 5/29B-27)}\label{ilcs-529b-27}}
\addcontentsline{toc}{subsection}{(720 ILCS 5/29B-27)}

\hypertarget{sec.-29b-27.-applicability-savings-clause.}{%
\section*{Sec. 29B-27. Applicability; savings
clause.}\label{sec.-29b-27.-applicability-savings-clause.}}
\addcontentsline{toc}{section}{Sec. 29B-27. Applicability; savings
clause.}

\markright{Sec. 29B-27. Applicability; savings clause.}

(a) The changes made to this Article by Public Act 100-512 and Public
Act 100-699 only apply to property seized on and after July 1, 2018.

(b) The changes made to this Article by Public Act 100-699 are subject
to Section 4 of the Statute on Statutes.

(Source: P.A. 100-699, eff. 8-3-18; 100-1163, eff. 12-20-18.)

\bookmarksetup{startatroot}

\hypertarget{article-29c.-international-terrorism}{%
\chapter*{Article 29c. International
Terrorism}\label{article-29c.-international-terrorism}}
\addcontentsline{toc}{chapter}{Article 29c. International Terrorism}

\markboth{Article 29c. International Terrorism}{Article 29c.
International Terrorism}

(Repealed by P.A. 92-854, eff. 12-5-02)

\hypertarget{ilcs-529c-5}{%
\subsection*{(720 ILCS 5/29C-5)}\label{ilcs-529c-5}}
\addcontentsline{toc}{subsection}{(720 ILCS 5/29C-5)}

\hypertarget{sec.-29c-5.-repealed.}{%
\section*{Sec. 29C-5. (Repealed).}\label{sec.-29c-5.-repealed.}}
\addcontentsline{toc}{section}{Sec. 29C-5. (Repealed).}

\markright{Sec. 29C-5. (Repealed).}

(Source: P.A. 89-515, eff. 1-1-97. Repealed by P.A. 92-854, eff.
12-5-02.)

\hypertarget{ilcs-529c-10}{%
\subsection*{(720 ILCS 5/29C-10)}\label{ilcs-529c-10}}
\addcontentsline{toc}{subsection}{(720 ILCS 5/29C-10)}

\hypertarget{sec.-29c-10.-repealed.}{%
\section*{Sec. 29C-10. (Repealed).}\label{sec.-29c-10.-repealed.}}
\addcontentsline{toc}{section}{Sec. 29C-10. (Repealed).}

\markright{Sec. 29C-10. (Repealed).}

(Source: P.A. 89-515, eff. 1-1-97. Repealed by P.A. 92-854, eff.
12-5-02.)

\hypertarget{ilcs-529c-15}{%
\subsection*{(720 ILCS 5/29C-15)}\label{ilcs-529c-15}}
\addcontentsline{toc}{subsection}{(720 ILCS 5/29C-15)}

\hypertarget{sec.-29c-15.-repealed.}{%
\section*{Sec. 29C-15. (Repealed).}\label{sec.-29c-15.-repealed.}}
\addcontentsline{toc}{section}{Sec. 29C-15. (Repealed).}

\markright{Sec. 29C-15. (Repealed).}

(Source: P.A. 89-515, eff. 1-1-97. Repealed by P.A. 92-854, eff.
12-5-02.)

\bookmarksetup{startatroot}

\hypertarget{article-29d.-terrorism}{%
\chapter*{Article 29d. Terrorism}\label{article-29d.-terrorism}}
\addcontentsline{toc}{chapter}{Article 29d. Terrorism}

\markboth{Article 29d. Terrorism}{Article 29d. Terrorism}

\hypertarget{ilcs-529d-5}{%
\subsection*{(720 ILCS 5/29D-5)}\label{ilcs-529d-5}}
\addcontentsline{toc}{subsection}{(720 ILCS 5/29D-5)}

\hypertarget{sec.-29d-5.-legislative-findings.}{%
\section*{Sec. 29D-5. Legislative
findings.}\label{sec.-29d-5.-legislative-findings.}}
\addcontentsline{toc}{section}{Sec. 29D-5. Legislative findings.}

\markright{Sec. 29D-5. Legislative findings.}

The devastating consequences of the barbaric attacks on the World Trade
Center and the Pentagon on September 11, 2001 underscore the compelling
need for legislation that is specifically designed to combat the evils
of terrorism. Terrorism is inconsistent with civilized society and
cannot be tolerated.

A comprehensive State law is urgently needed to complement federal laws
in the fight against terrorism and to better protect all citizens
against terrorist acts. Accordingly, the legislature finds that our laws
must be strengthened to ensure that terrorists, as well as those who
solicit or provide financial and other support to terrorists, are
prosecuted and punished in State courts with appropriate severity. The
legislature further finds that due to the grave nature and global reach
of terrorism that a comprehensive law encompassing State criminal
statutes and strong civil remedies is needed.

An investigation may not be initiated or continued for activities
protected by the First Amendment to the United States Constitution,
including expressions of support or the provision of financial support
for the nonviolent political, religious, philosophical, or ideological
goals or beliefs of any person or group.

(Source: P.A. 92-854, eff. 12-5-02.)

\hypertarget{ilcs-529d-10}{%
\subsection*{(720 ILCS 5/29D-10)}\label{ilcs-529d-10}}
\addcontentsline{toc}{subsection}{(720 ILCS 5/29D-10)}

\hypertarget{sec.-29d-10.-definitions.}{%
\section*{Sec. 29D-10. Definitions.}\label{sec.-29d-10.-definitions.}}
\addcontentsline{toc}{section}{Sec. 29D-10. Definitions.}

\markright{Sec. 29D-10. Definitions.}

As used in this Article, where not otherwise distinctly expressed or
manifestly incompatible with the intent of this Article:

(a) ``Computer network'' means a set of related, remotely connected
devices and any communications facilities including more than one
computer with the capability to transmit data among them through
communication facilities.

(b) ``Computer'' means a device that accepts, processes, stores,
retrieves, or outputs data, and includes, but is not limited to,
auxiliary storage and telecommunications devices.

(c) ``Computer program'' means a series of coded instruction or
statements in a form acceptable to a computer which causes the computer
to process data and supply the results of data processing.

(d) ``Data'' means representations of information, knowledge, facts,
concepts or instructions, including program documentation, that are
prepared in a formalized manner and are stored or processed in or
transmitted by a computer. Data may be in any form, including but not
limited to magnetic or optical storage media, punch cards, or data
stored internally in the memory of a computer.

(e) ``Biological products used in or in connection with agricultural
production'' includes, but is not limited to, seeds, plants, and DNA of
plants or animals altered for use in crop or livestock breeding or
production or which are sold, intended, designed, or produced for use in
crop production or livestock breeding or production.

(f) ``Agricultural products'' means crops and livestock.

(g) ``Agricultural production'' means the breeding and growing of
livestock and crops.

(g-5) ``Animal feed'' means an article that is intended for use for food
for animals other than humans and that is intended for use as a
substantial source of nutrients in the diet of the animal, and is not
limited to a mixture intended to be the sole ration of the animal.

(g-10) ``Contagious or infectious disease'' means a specific disease
designated by the Illinois Department of Agriculture as contagious or
infectious under rules pertaining to the Illinois Diseased Animals Act.

(g-15) ``Processed food'' means any food other than a raw agricultural
commodity and includes any raw agricultural commodity that has been
subject to processing, such as canning, cooking, freezing, dehydration,
or milling.

(g-20) ``Raw agricultural commodity'' means any food in its raw or
natural state, including all fruits that are washed, colored, or
otherwise treated in their unpeeled natural form prior to marketing and
honey that is in the comb or that is removed from the comb and in an
unadulterated condition.

(g-25) ``Endangering the food supply'' means to knowingly:

(1) bring into this State any domestic animal that is affected with any
contagious or infectious disease or any animal that has been exposed to
any contagious or infectious disease;

(2) expose any animal in this State to any contagious or infectious
disease;

(3) deliver any poultry that is infected with any contagious or
infectious disease to any poultry producer pursuant to a production
contract;

(4) except as permitted under the Insect Pest and

Plant Disease Act, bring or release into this State any insect pest or
expose any plant to an insect pest; or

(5) expose any raw agricultural commodity, animal feed, or processed
food to any contaminant or contagious or infectious disease.

``Endangering the food supply'' does not include bona fide experiments
and actions related to those experiments carried on by commonly
recognized research facilities or actions by agricultural producers and
animal health professionals who may inadvertently contribute to the
spread of detrimental biological agents while employing generally
acceptable management practices.

(g-30) ``Endangering the water supply'' means to knowingly contaminate a
public or private water well or water reservoir or any water supply of a
public utility or tamper with the production of bottled or packaged
water or tamper with bottled or packaged water at a retail or wholesale
mercantile establishment. ``Endangering the water supply'' does not
include contamination of a public or private well or water reservoir or
any water supply of a public utility that may occur inadvertently as
part of the operation of a public utility or electrical generating
station.

(h) ``Livestock'' means animals bred or raised for human consumption.

(i) ``Crops'' means plants raised for: (1) human consumption, (2) fruits
that are intended for human consumption, (3) consumption by livestock,
and (4) fruits that are intended for consumption by livestock.

(j) ``Communications systems'' means any works, property, or material of
any radio, telegraph, telephone, microwave, or cable line, station, or
system.

(k) ``Substantial damage'' means monetary damage greater than \$100,000.

(l) ``Terrorist act'' or ``act of terrorism'' means: (1) any act that is
intended to cause or create a risk and does cause or create a risk of
death or great bodily harm to one or more persons; (2) any act that
disables or destroys the usefulness or operation of any communications
system; (3) any act or any series of 2 or more acts committed in
furtherance of a single intention, scheme, or design that disables or
destroys the usefulness or operation of a computer network, computers,
computer programs, or data used by any industry, by any class of
business, or by 5 or more businesses or by the federal government, State
government, any unit of local government, a public utility, a
manufacturer of pharmaceuticals, a national defense contractor, or a
manufacturer of chemical or biological products used in or in connection
with agricultural production; (4) any act that disables or causes
substantial damage to or destruction of any structure or facility used
in or used in connection with ground, air, or water transportation; the
production or distribution of electricity, gas, oil, or other fuel
(except for acts that occur inadvertently and as the result of operation
of the facility that produces or distributes electricity, gas, oil, or
other fuel); the treatment of sewage or the treatment or distribution of
water; or controlling the flow of any body of water; (5) any act that
causes substantial damage to or destruction of livestock or to crops or
a series of 2 or more acts committed in furtherance of a single
intention, scheme, or design which, in the aggregate, causes substantial
damage to or destruction of livestock or crops; (6) any act that causes
substantial damage to or destruction of any hospital or any building or
facility used by the federal government, State government, any unit of
local government or by a national defense contractor or by a public
utility, a manufacturer of pharmaceuticals, a manufacturer of chemical
or biological products used in or in connection with agricultural
production or the storage or processing of agricultural products or the
preparation of agricultural products for food or food products intended
for resale or for feed for livestock; (7) any act that causes
substantial damage to any building containing 5 or more businesses of
any type or to any building in which 10 or more people reside; (8)
endangering the food supply; or (9) endangering the water supply.

(m) ``Terrorist'' and ``terrorist organization'' means any person who
engages or is about to engage in a terrorist act with the intent to
intimidate or coerce a significant portion of a civilian population.

(n) ``Material support or resources'' means currency or other financial
securities, financial services, lodging, training, safe houses, false
documentation or identification, communications equipment, facilities,
weapons, lethal substances, explosives, personnel, transportation, any
other kind of physical assets or intangible property, and expert
services or expert assistance.

(o) ``Person'' has the meaning given in Section 2-15 of this Code and,
in addition to that meaning, includes, without limitation, any
charitable organization, whether incorporated or unincorporated, any
professional fund raiser, professional solicitor, limited liability
company, association, joint stock company, association, trust, trustee,
or any group of people formally or informally affiliated or associated
for a common purpose, and any officer, director, partner, member, or
agent of any person.

(p) ``Render criminal assistance'' means to do any of the following with
the intent to prevent, hinder, or delay the discovery or apprehension
of, or the lodging of a criminal charge against, a person who he or she
knows or believes has committed an offense under this Article or is
being sought by law enforcement officials for the commission of an
offense under this Article, or with the intent to assist a person in
profiting or benefiting from the commission of an offense under this
Article:

(1) harbor or conceal the person;

(2) warn the person of impending discovery or apprehension;

(3) provide the person with money, transportation, a weapon, a disguise,
false identification documents, or any other means of avoiding discovery
or apprehension;

(4) prevent or obstruct, by means of force, intimidation, or deception,
anyone from performing an act that might aid in the discovery or
apprehension of the person or in the lodging of a criminal charge
against the person;

(5) suppress, by any act of concealment, alteration, or destruction, any
physical evidence that might aid in the discovery or apprehension of the
person or in the lodging of a criminal charge against the person;

(6) aid the person to protect or expeditiously profit from an advantage
derived from the crime; or

(7) provide expert services or expert assistance to the person.
Providing expert services or expert assistance shall not be construed to
apply to: (1) a licensed attorney who discusses with a client the legal
consequences of a proposed course of conduct or advises a client of
legal or constitutional rights and (2) a licensed medical doctor who
provides emergency medical treatment to a person whom he or she believes
has committed an offense under this Article if, as soon as reasonably
practicable either before or after providing such treatment, he or she
notifies a law enforcement agency.

(Source: P.A. 96-1028, eff. 1-1-11.)

\hypertarget{ilcs-529d-14.9}{%
\subsection*{(720 ILCS 5/29D-14.9)}\label{ilcs-529d-14.9}}
\addcontentsline{toc}{subsection}{(720 ILCS 5/29D-14.9)}

(was 720 ILCS 5/29D-30)

\hypertarget{sec.-29d-14.9.-terrorism.}{%
\section*{Sec. 29D-14.9. Terrorism.}\label{sec.-29d-14.9.-terrorism.}}
\addcontentsline{toc}{section}{Sec. 29D-14.9. Terrorism.}

\markright{Sec. 29D-14.9. Terrorism.}

(a) A person commits the offense of terrorism when, with the intent to
intimidate or coerce a significant portion of a civilian population:

(1) he or she knowingly commits a terrorist act as defined in Section
29D-10(1) of this Code within this State; or

(2) he or she, while outside this State, knowingly commits a terrorist
act as defined in Section 29D-10(1) of this Code that takes effect
within this State or produces substantial detrimental effects within
this State.

(b) Sentence. Terrorism is a Class X felony. If no deaths are caused by
the terrorist act, the sentence shall be a term of 20 years to natural
life imprisonment; if the terrorist act caused the death of one or more
persons, however, a mandatory term of natural life imprisonment shall be
the sentence if the death penalty is not imposed and the person has
attained the age of 18 years at the time of the commission of the
offense. An offender under the age of 18 years at the time of the
commission of the offense shall be sentenced under Section 5-4.5-105 of
the Unified Code of Corrections.

(Source: P.A. 99-69, eff. 1-1-16 .)

\hypertarget{ilcs-529d-15}{%
\subsection*{(720 ILCS 5/29D-15)}\label{ilcs-529d-15}}
\addcontentsline{toc}{subsection}{(720 ILCS 5/29D-15)}

\hypertarget{sec.-29d-15.-renumbered.}{%
\section*{Sec. 29D-15. (Renumbered).}\label{sec.-29d-15.-renumbered.}}
\addcontentsline{toc}{section}{Sec. 29D-15. (Renumbered).}

\markright{Sec. 29D-15. (Renumbered).}

(Source: Renumbered by P.A. 96-710, eff. 1-1-10.)

\hypertarget{ilcs-529d-15.1}{%
\subsection*{(720 ILCS 5/29D-15.1)}\label{ilcs-529d-15.1}}
\addcontentsline{toc}{subsection}{(720 ILCS 5/29D-15.1)}

(was 720 ILCS 5/20.5-5)

\hypertarget{sec.-29d-15.1.-causing-a-catastrophe.}{%
\section*{Sec. 29D-15.1. Causing a
catastrophe.}\label{sec.-29d-15.1.-causing-a-catastrophe.}}
\addcontentsline{toc}{section}{Sec. 29D-15.1. Causing a catastrophe.}

\markright{Sec. 29D-15.1. Causing a catastrophe.}

(a) A person commits the offense of causing a catastrophe if he or she
knowingly causes a catastrophe by explosion, fire, flood, collapse of a
building, or release of poison, radioactive material, bacteria, virus,
or other dangerous and difficult to confine force or substance.

(b) As used in this Section, ``catastrophe'' means serious physical
injury to 5 or more persons, substantial damage to 5 or more buildings
or inhabitable structures, or substantial damage to a vital public
facility that seriously impairs its usefulness or operation; and ``vital
public facility'' means a facility that is necessary to ensure or
protect the public health, safety, or welfare, including, but not
limited to, a hospital, a law enforcement agency, a fire department, a
private or public utility company, a national defense contractor, a
facility of the armed forces, or an emergency services agency.

(c) Sentence. Causing a catastrophe is a Class X felony.

(Source: P.A. 96-710, eff. 1-1-10.)

\hypertarget{ilcs-529d-15.2}{%
\subsection*{(720 ILCS 5/29D-15.2)}\label{ilcs-529d-15.2}}
\addcontentsline{toc}{subsection}{(720 ILCS 5/29D-15.2)}

(was 720 ILCS 5/20.5-6)

\hypertarget{sec.-29d-15.2.-possession-of-a-deadly-substance.}{%
\section*{Sec. 29D-15.2. Possession of a deadly
substance.}\label{sec.-29d-15.2.-possession-of-a-deadly-substance.}}
\addcontentsline{toc}{section}{Sec. 29D-15.2. Possession of a deadly
substance.}

\markright{Sec. 29D-15.2. Possession of a deadly substance.}

(a) A person commits the offense of possession of a deadly substance
when he or she possesses, manufactures, or transports any poisonous gas,
deadly biological or chemical contaminant or agent, or radioactive
substance either with the intent to use that gas, biological or chemical
contaminant or agent, or radioactive substance to commit a felony or
with the knowledge that another person intends to use that gas,
biological or chemical contaminant or agent, or radioactive substance to
commit a felony.

(b) Sentence. Possession of a deadly substance is a Class 1 felony for
which a person, if sentenced to a term of imprisonment, shall be
sentenced to a term of not less than 4 years and not more than 30 years.

(Source: P.A. 96-710, eff. 1-1-10.)

\hypertarget{ilcs-529d-20}{%
\subsection*{(720 ILCS 5/29D-20)}\label{ilcs-529d-20}}
\addcontentsline{toc}{subsection}{(720 ILCS 5/29D-20)}

\hypertarget{sec.-29d-20.-making-a-terrorist-threat.}{%
\section*{Sec. 29D-20. Making a terrorist
threat.}\label{sec.-29d-20.-making-a-terrorist-threat.}}
\addcontentsline{toc}{section}{Sec. 29D-20. Making a terrorist threat.}

\markright{Sec. 29D-20. Making a terrorist threat.}

(a) A person is guilty of making a terrorist threat when, with the
intent to intimidate or coerce a significant portion of a civilian
population, he or she in any manner knowingly threatens to commit or
threatens to cause the commission of a terrorist act as defined in
Section 29D-10(1) and thereby causes a reasonable expectation or fear of
the imminent commission of a terrorist act as defined in Section
29D-10(1) or of another terrorist act as defined in Section 29D-10(1).

(b) It is not a defense to a prosecution under this Section that at the
time the defendant made the terrorist threat, unknown to the defendant,
it was impossible to carry out the threat, nor is it a defense that the
threat was not made to a person who was a subject or intended victim of
the threatened act.

(c) Sentence. Making a terrorist threat is a Class X felony.

(d) In addition to any other sentence that may be imposed, the court
shall order any person convicted of making a terrorist threat involving
a threat that a bomb or explosive device has been placed in a school to
reimburse the unit of government that employs the emergency response
officer or officers that were dispatched to the school for the cost of
the search for a bomb or explosive device. For the purposes of this
Section, ``emergency response'' means any incident requiring a response
by a police officer, a firefighter, a State Fire Marshal employee, or an
ambulance.

(Source: P.A. 96-413, eff. 8-13-09.)

\hypertarget{ilcs-529d-25}{%
\subsection*{(720 ILCS 5/29D-25)}\label{ilcs-529d-25}}
\addcontentsline{toc}{subsection}{(720 ILCS 5/29D-25)}

\hypertarget{sec.-29d-25.-falsely-making-a-terrorist-threat.}{%
\section*{Sec. 29D-25. Falsely making a terrorist
threat.}\label{sec.-29d-25.-falsely-making-a-terrorist-threat.}}
\addcontentsline{toc}{section}{Sec. 29D-25. Falsely making a terrorist
threat.}

\markright{Sec. 29D-25. Falsely making a terrorist threat.}

\hypertarget{a-a-person-commits-the-offense-of-falsely-making-a-terrorist-threat-when-in-any-manner-he-or-she-knowingly-makes-a-threat-to-commit-or-cause-to-be-committed-a-terrorist-act-as-defined-in-section-29d-101-or-otherwise-knowingly-creates-the-impression-or-belief-that-a-terrorist-act-is-about-to-be-or-has-been-committed-or-in-any-manner-knowingly-makes-a-threat-to-commit-or-cause-to-be-committed-a-catastrophe-as-defined-in-section-29d-15.1-720-ilcs-529d-15.1-of-this-code-that-he-or-she-knows-is-false.}{%
\subsection*{(a) A person commits the offense of falsely making a
terrorist threat when in any manner he or she knowingly makes a threat
to commit or cause to be committed a terrorist act as defined in Section
29D-10(1) or otherwise knowingly creates the impression or belief that a
terrorist act is about to be or has been committed, or in any manner
knowingly makes a threat to commit or cause to be committed a
catastrophe as defined in Section 29D-15.1 (720 ILCS 5/29D-15.1) of this
Code that he or she knows is
false.}\label{a-a-person-commits-the-offense-of-falsely-making-a-terrorist-threat-when-in-any-manner-he-or-she-knowingly-makes-a-threat-to-commit-or-cause-to-be-committed-a-terrorist-act-as-defined-in-section-29d-101-or-otherwise-knowingly-creates-the-impression-or-belief-that-a-terrorist-act-is-about-to-be-or-has-been-committed-or-in-any-manner-knowingly-makes-a-threat-to-commit-or-cause-to-be-committed-a-catastrophe-as-defined-in-section-29d-15.1-720-ilcs-529d-15.1-of-this-code-that-he-or-she-knows-is-false.}}
\addcontentsline{toc}{subsection}{(a) A person commits the offense of
falsely making a terrorist threat when in any manner he or she knowingly
makes a threat to commit or cause to be committed a terrorist act as
defined in Section 29D-10(1) or otherwise knowingly creates the
impression or belief that a terrorist act is about to be or has been
committed, or in any manner knowingly makes a threat to commit or cause
to be committed a catastrophe as defined in Section 29D-15.1 (720 ILCS
5/29D-15.1) of this Code that he or she knows is false.}

(b) Sentence. Falsely making a terrorist threat is a Class 1 felony.

(c) In addition to any other sentence that may be imposed, the court
shall order any person convicted of falsely making a terrorist threat,
involving a threat that a bomb or explosive device has been placed in a
school in which the offender knows that such bomb or explosive device
was not placed in the school, to reimburse the unit of government that
employs the emergency response officer or officers that were dispatched
to the school for the cost of the search for a bomb or explosive device.
For the purposes of this Section, ``emergency response'' means any
incident requiring a response by a police officer, a firefighter, a
State Fire Marshal employee, or an ambulance.

(Source: P.A. 96-413, eff. 8-13-09; 96-710, eff. 1-1-10; 96-1000, eff.
7-2-10.)

\hypertarget{ilcs-529d-29.9}{%
\subsection*{(720 ILCS 5/29D-29.9)}\label{ilcs-529d-29.9}}
\addcontentsline{toc}{subsection}{(720 ILCS 5/29D-29.9)}

(was 720 ILCS 5/29D-15)

\hypertarget{sec.-29d-29.9.-material-support-for-terrorism.}{%
\section*{Sec. 29D-29.9. Material support for
terrorism.}\label{sec.-29d-29.9.-material-support-for-terrorism.}}
\addcontentsline{toc}{section}{Sec. 29D-29.9. Material support for
terrorism.}

\markright{Sec. 29D-29.9. Material support for terrorism.}

\hypertarget{a-a-person-commits-the-offense-of-soliciting-or-providing-material-support-for-terrorism-if-he-or-she-knowingly-raises-solicits-collects-or-provides-material-support-or-resources-knowing-that-the-material-support-or-resources-will-be-used-in-whole-or-in-part-to-plan-prepare-carry-out-facilitate-or-avoid-apprehension-for-committing-terrorism-as-defined-in-section-29d-14.9-720-ilcs-529d-14.9-or-causing-a-catastrophe-as-defined-in-section-29d-15.1-720-ilcs-529d-15.1-of-this-code-or-who-knows-and-intends-that-the-material-support-or-resources-so-raised-solicited-collected-or-provided-will-be-used-in-the-commission-of-a-terrorist-act-as-defined-in-section-29d-101-of-this-code-by-an-organization-designated-under-8-u.s.c.-1189-as-amended.-it-is-not-an-element-of-the-offense-that-the-defendant-actually-knows-that-an-organization-has-been-designated-under-8-u.s.c.-1189-as-amended.}{%
\subsection*{(a) A person commits the offense of soliciting or providing
material support for terrorism if he or she knowingly raises, solicits,
collects, or provides material support or resources knowing that the
material support or resources will be used, in whole or in part, to
plan, prepare, carry out, facilitate, or avoid apprehension for
committing terrorism as defined in Section 29D-14.9 (720 ILCS
5/29D-14.9) or causing a catastrophe as defined in Section 29D-15.1 (720
ILCS 5/29D-15.1) of this Code, or who knows and intends that the
material support or resources so raised, solicited, collected, or
provided will be used in the commission of a terrorist act as defined in
Section 29D-10(1) of this Code by an organization designated under 8
U.S.C. 1189, as amended. It is not an element of the offense that the
defendant actually knows that an organization has been designated under
8 U.S.C. 1189, as
amended.}\label{a-a-person-commits-the-offense-of-soliciting-or-providing-material-support-for-terrorism-if-he-or-she-knowingly-raises-solicits-collects-or-provides-material-support-or-resources-knowing-that-the-material-support-or-resources-will-be-used-in-whole-or-in-part-to-plan-prepare-carry-out-facilitate-or-avoid-apprehension-for-committing-terrorism-as-defined-in-section-29d-14.9-720-ilcs-529d-14.9-or-causing-a-catastrophe-as-defined-in-section-29d-15.1-720-ilcs-529d-15.1-of-this-code-or-who-knows-and-intends-that-the-material-support-or-resources-so-raised-solicited-collected-or-provided-will-be-used-in-the-commission-of-a-terrorist-act-as-defined-in-section-29d-101-of-this-code-by-an-organization-designated-under-8-u.s.c.-1189-as-amended.-it-is-not-an-element-of-the-offense-that-the-defendant-actually-knows-that-an-organization-has-been-designated-under-8-u.s.c.-1189-as-amended.}}
\addcontentsline{toc}{subsection}{(a) A person commits the offense of
soliciting or providing material support for terrorism if he or she
knowingly raises, solicits, collects, or provides material support or
resources knowing that the material support or resources will be used,
in whole or in part, to plan, prepare, carry out, facilitate, or avoid
apprehension for committing terrorism as defined in Section 29D-14.9
(720 ILCS 5/29D-14.9) or causing a catastrophe as defined in Section
29D-15.1 (720 ILCS 5/29D-15.1) of this Code, or who knows and intends
that the material support or resources so raised, solicited, collected,
or provided will be used in the commission of a terrorist act as defined
in Section 29D-10(1) of this Code by an organization designated under 8
U.S.C. 1189, as amended. It is not an element of the offense that the
defendant actually knows that an organization has been designated under
8 U.S.C. 1189, as amended.}

(b) Sentence. Soliciting or providing material support for terrorism is
a Class X felony for which the sentence shall be a term of imprisonment
of no less than 9 years and no more than 40 years.

(Source: P.A. 96-710, eff. 1-1-10.)

\hypertarget{ilcs-529d-30}{%
\subsection*{(720 ILCS 5/29D-30)}\label{ilcs-529d-30}}
\addcontentsline{toc}{subsection}{(720 ILCS 5/29D-30)}

\hypertarget{sec.-29d-30.-renumbered.}{%
\section*{Sec. 29D-30. (Renumbered).}\label{sec.-29d-30.-renumbered.}}
\addcontentsline{toc}{section}{Sec. 29D-30. (Renumbered).}

\markright{Sec. 29D-30. (Renumbered).}

(Source: Renumbered by P.A. 96-710, eff. 1-1-10.)

\hypertarget{ilcs-529d-35}{%
\subsection*{(720 ILCS 5/29D-35)}\label{ilcs-529d-35}}
\addcontentsline{toc}{subsection}{(720 ILCS 5/29D-35)}

\hypertarget{sec.-29d-35.-hindering-prosecution-of-terrorism.}{%
\section*{Sec. 29D-35. Hindering prosecution of
terrorism.}\label{sec.-29d-35.-hindering-prosecution-of-terrorism.}}
\addcontentsline{toc}{section}{Sec. 29D-35. Hindering prosecution of
terrorism.}

\markright{Sec. 29D-35. Hindering prosecution of terrorism.}

(a) A person commits the offense of hindering prosecution of terrorism
when he or she renders criminal assistance to a person who has committed
terrorism as defined in Section 29D-14.9 or caused a catastrophe as
defined in Section 29D-15.1 of this Code when he or she knows that the
person to whom he or she rendered criminal assistance engaged in an act
of terrorism or caused a catastrophe.

(b) Hindering prosecution of terrorism is a Class X felony, the sentence
for which shall be a term of 20 years to natural life imprisonment if no
death was caused by the act of terrorism committed by the person to whom
the defendant rendered criminal assistance and a mandatory term of
natural life imprisonment if death was caused by the act of terrorism
committed by the person to whom the defendant rendered criminal
assistance. An offender under the age of 18 years at the time of the
commission of the offense shall be sentenced under Section 5-4.5-105 of
the Unified Code of Corrections.

(Source: P.A. 99-69, eff. 1-1-16 .)

\hypertarget{ilcs-529d-35.1}{%
\subsection*{(720 ILCS 5/29D-35.1)}\label{ilcs-529d-35.1}}
\addcontentsline{toc}{subsection}{(720 ILCS 5/29D-35.1)}

\hypertarget{sec.-29d-35.1.-boarding-or-attempting-to-board-an-aircraft-with-weapon.}{%
\section*{Sec. 29D-35.1. Boarding or attempting to board an aircraft
with
weapon.}\label{sec.-29d-35.1.-boarding-or-attempting-to-board-an-aircraft-with-weapon.}}
\addcontentsline{toc}{section}{Sec. 29D-35.1. Boarding or attempting to
board an aircraft with weapon.}

\markright{Sec. 29D-35.1. Boarding or attempting to board an aircraft
with weapon.}

(a) It is unlawful for any person to board or attempt to board any
commercial or charter aircraft, knowingly having in his or her
possession any firearm, explosive of any type, or other lethal or
dangerous weapon.

(b) This Section does not apply to any person authorized by either the
federal government or any state government to carry firearms, but the
person so exempted from the provisions of this Section shall notify the
commander of any aircraft he or she is about to board that he or she
does possess a firearm and show identification satisfactory to the
aircraft commander that he or she is authorized to carry that firearm.

(c) Any person purchasing a ticket to board any commercial or charter
aircraft shall by that purchase consent to a search of his or her person
or personal belongings by the company selling the ticket to him or her.
The person may refuse to submit to a search of his or her person or
personal belongings by the aircraft company, but the person refusing may
be denied the right to board the commercial or charter aircraft at the
discretion of the carrier. Such a refusal creates no inference of
unlawful conduct.

(d) Any evidence of criminal activity found during a search made
pursuant to this Section shall be admissible in legal proceedings for
the sole purpose of supporting a charge of violation of this Section and
is inadmissible as evidence in any legal proceeding for any other
purpose, except in the prosecution of offenses related to weapons as set
out in Article 24 of this Code.

(e) No action may be brought against any commercial or charter airline
company operating in this State for the refusal of that company to
permit a person to board any aircraft if that person refused to be
searched as set out in subsection (c) of this Section.

(f) Violation of this Section is a Class 4 felony.

(Source: P.A. 96-710, eff. 1-1-10.)

\hypertarget{ilcs-529d-40}{%
\subsection*{(720 ILCS 5/29D-40)}\label{ilcs-529d-40}}
\addcontentsline{toc}{subsection}{(720 ILCS 5/29D-40)}

\hypertarget{sec.-29d-40.-restitution.}{%
\section*{Sec. 29D-40. Restitution.}\label{sec.-29d-40.-restitution.}}
\addcontentsline{toc}{section}{Sec. 29D-40. Restitution.}

\markright{Sec. 29D-40. Restitution.}

In addition to any other penalty that may be imposed, a court shall
sentence any person convicted of any violation of this Article to pay
all expenses incurred by the federal government, State government, or
any unit of local government in responding to any violation and cleaning
up following any violation.

(Source: P.A. 92-854, eff. 12-5-02.)

\hypertarget{ilcs-529d-45}{%
\subsection*{(720 ILCS 5/29D-45)}\label{ilcs-529d-45}}
\addcontentsline{toc}{subsection}{(720 ILCS 5/29D-45)}

\hypertarget{sec.-29d-45.-limitations.}{%
\section*{Sec. 29D-45. Limitations.}\label{sec.-29d-45.-limitations.}}
\addcontentsline{toc}{section}{Sec. 29D-45. Limitations.}

\markright{Sec. 29D-45. Limitations.}

A prosecution for any offense in this Article may be commenced at any
time.

(Source: P.A. 92-854, eff. 12-5-02.)

\hypertarget{ilcs-529d-60}{%
\subsection*{(720 ILCS 5/29D-60)}\label{ilcs-529d-60}}
\addcontentsline{toc}{subsection}{(720 ILCS 5/29D-60)}

\hypertarget{sec.-29d-60.-injunctive-relief.}{%
\section*{Sec. 29D-60. Injunctive
relief.}\label{sec.-29d-60.-injunctive-relief.}}
\addcontentsline{toc}{section}{Sec. 29D-60. Injunctive relief.}

\markright{Sec. 29D-60. Injunctive relief.}

Whenever it appears to the Attorney General or any State's Attorney that
any person is engaged in, or is about to engage in, any act that
constitutes or would constitute a violation of this Article, the
Attorney General or any State's Attorney may initiate a civil action in
the circuit court to enjoin the violation.

(Source: P.A. 92-854, eff. 12-5-02.)

\hypertarget{ilcs-529d-65}{%
\subsection*{(720 ILCS 5/29D-65)}\label{ilcs-529d-65}}
\addcontentsline{toc}{subsection}{(720 ILCS 5/29D-65)}

\hypertarget{sec.-29d-65.-forfeiture-of-property-acquired-in-connection-with-a-violation-of-this-article-property-freeze-or-seizure.}{%
\section*{Sec. 29D-65. Forfeiture of property acquired in connection
with a violation of this Article; property freeze or
seizure.}\label{sec.-29d-65.-forfeiture-of-property-acquired-in-connection-with-a-violation-of-this-article-property-freeze-or-seizure.}}
\addcontentsline{toc}{section}{Sec. 29D-65. Forfeiture of property
acquired in connection with a violation of this Article; property freeze
or seizure.}

\markright{Sec. 29D-65. Forfeiture of property acquired in connection
with a violation of this Article; property freeze or seizure.}

(a) If there is probable cause to believe that a person used, is using,
is about to use, or is intending to use property in a way that would
violate this Article, then that person's assets may be frozen or seized
pursuant to Part 800 of Article 124B of the Code of Criminal Procedure
of 1963.

(b) Any person who commits any offense under this Article is subject to
the property forfeiture provisions set forth in Article 124B of the Code
of Criminal Procedure of 1963. Forfeiture under this subsection may be
pursued in addition to or in lieu of proceeding under Section 124B-805
(property freeze or seizure; ex parte proceeding) of the Code of
Criminal Procedure of 1963.

(Source: P.A. 96-712, eff. 1-1-10.)

\hypertarget{ilcs-529d-70}{%
\subsection*{(720 ILCS 5/29D-70)}\label{ilcs-529d-70}}
\addcontentsline{toc}{subsection}{(720 ILCS 5/29D-70)}

\hypertarget{sec.-29d-70.-severability.}{%
\section*{Sec. 29D-70. Severability.}\label{sec.-29d-70.-severability.}}
\addcontentsline{toc}{section}{Sec. 29D-70. Severability.}

\markright{Sec. 29D-70. Severability.}

If any clause, sentence, Section, provision, or part of this Article or
the application thereof to any person or circumstance shall be adjudged
to be unconstitutional, the remainder of this Article or its application
to persons or circumstances other than those to which it is held
invalid, shall not be affected thereby.

(Source: P.A. 92-854, eff. 12-5-02.)

\hypertarget{ilcs-5tit.-iii-pt.-e-heading}{%
\subsection*{(720 ILCS 5/Tit. III Pt. E
heading)}\label{ilcs-5tit.-iii-pt.-e-heading}}
\addcontentsline{toc}{subsection}{(720 ILCS 5/Tit. III Pt. E heading)}

PART E.

OFFENSES AFFECTING GOVERNMENTAL FUNCTIONS

\bookmarksetup{startatroot}

\hypertarget{article-30.-treason-and-related-offenses}{%
\chapter*{Article 30. Treason And Related
Offenses}\label{article-30.-treason-and-related-offenses}}
\addcontentsline{toc}{chapter}{Article 30. Treason And Related Offenses}

\markboth{Article 30. Treason And Related Offenses}{Article 30. Treason
And Related Offenses}

\hypertarget{ilcs-530-1-from-ch.-38-par.-30-1}{%
\subsection*{(720 ILCS 5/30-1) (from Ch. 38, par.
30-1)}\label{ilcs-530-1-from-ch.-38-par.-30-1}}
\addcontentsline{toc}{subsection}{(720 ILCS 5/30-1) (from Ch. 38, par.
30-1)}

\hypertarget{sec.-30-1.-treason.}{%
\section*{Sec. 30-1. Treason.}\label{sec.-30-1.-treason.}}
\addcontentsline{toc}{section}{Sec. 30-1. Treason.}

\markright{Sec. 30-1. Treason.}

(a) A person owing allegiance to this State commits treason when he or
she knowingly:

(1) levies war against this State; or

(2) adheres to the enemies of this State, giving them aid or comfort.

(b) No person may be convicted of treason except on the testimony of 2
witnesses to the same overt act, or on his confession in open court.

(c) Sentence. Treason is a Class X felony for which an offender may be
sentenced to death under Section 5-5-3 of the Unified Code of
Corrections.

(Source: P.A. 80-1099 .)

\hypertarget{ilcs-530-2-from-ch.-38-par.-30-2}{%
\subsection*{(720 ILCS 5/30-2) (from Ch. 38, par.
30-2)}\label{ilcs-530-2-from-ch.-38-par.-30-2}}
\addcontentsline{toc}{subsection}{(720 ILCS 5/30-2) (from Ch. 38, par.
30-2)}

\hypertarget{sec.-30-2.-misprision-of-treason.}{%
\section*{Sec. 30-2. Misprision of
treason.}\label{sec.-30-2.-misprision-of-treason.}}
\addcontentsline{toc}{section}{Sec. 30-2. Misprision of treason.}

\markright{Sec. 30-2. Misprision of treason.}

(a) A person owing allegiance to this State commits misprision of
treason when he or she knowingly conceals or withholds his or her
knowledge that another has committed treason against this State.

(b) Sentence.

Misprision of treason is a Class 4 felony.

(Source: P.A. 97-1108, eff. 1-1-13.)

\hypertarget{ilcs-530-3-from-ch.-38-par.-30-3}{%
\subsection*{(720 ILCS 5/30-3) (from Ch. 38, par.
30-3)}\label{ilcs-530-3-from-ch.-38-par.-30-3}}
\addcontentsline{toc}{subsection}{(720 ILCS 5/30-3) (from Ch. 38, par.
30-3)}

\hypertarget{sec.-30-3.-advocating-overthrow-of-government.}{%
\section*{Sec. 30-3. Advocating overthrow of
Government.}\label{sec.-30-3.-advocating-overthrow-of-government.}}
\addcontentsline{toc}{section}{Sec. 30-3. Advocating overthrow of
Government.}

\markright{Sec. 30-3. Advocating overthrow of Government.}

A person who advocates, or with knowledge of its contents knowingly
publishes, sells or distributes any document which advocates or with
knowledge of its purpose, knowingly becomes a member of any organization
which advocates the overthrow or reformation of the existing form of
government of this State by violence or unlawful means commits a Class 3
felony.

(Source: P.A. 77-2638.)

\bookmarksetup{startatroot}

\hypertarget{article-31.-interference-with-public-officers}{%
\chapter*{Article 31. Interference With Public
Officers}\label{article-31.-interference-with-public-officers}}
\addcontentsline{toc}{chapter}{Article 31. Interference With Public
Officers}

\markboth{Article 31. Interference With Public Officers}{Article 31.
Interference With Public Officers}

\hypertarget{ilcs-531-1-from-ch.-38-par.-31-1}{%
\subsection*{(720 ILCS 5/31-1) (from Ch. 38, par.
31-1)}\label{ilcs-531-1-from-ch.-38-par.-31-1}}
\addcontentsline{toc}{subsection}{(720 ILCS 5/31-1) (from Ch. 38, par.
31-1)}

\hypertarget{sec.-31-1.-resisting-or-obstructing-a-peace-officer-firefighter-or-correctional-institution-employee.}{%
\section*{Sec. 31-1. Resisting or obstructing a peace officer,
firefighter, or correctional institution
employee.}\label{sec.-31-1.-resisting-or-obstructing-a-peace-officer-firefighter-or-correctional-institution-employee.}}
\addcontentsline{toc}{section}{Sec. 31-1. Resisting or obstructing a
peace officer, firefighter, or correctional institution employee.}

\markright{Sec. 31-1. Resisting or obstructing a peace officer,
firefighter, or correctional institution employee.}

(a) A person who knowingly:

(1) resists arrest, or

(2) obstructs the performance by one known to the person to be a peace
officer, firefighter, or correctional institution employee of any
authorized act within his or her official capacity commits a Class A
misdemeanor.

(a-5) In addition to any other sentence that may be imposed, a court
shall order any person convicted of resisting or obstructing a peace
officer, firefighter, or correctional institution employee to be
sentenced to a minimum of 48 consecutive hours of imprisonment or
ordered to perform community service for not less than 100 hours as may
be determined by the court. The person shall not be eligible for
probation in order to reduce the sentence of imprisonment or community
service.

(a-7) A person convicted for a violation of this Section whose violation
was the proximate cause of an injury to a peace officer, firefighter, or
correctional institution employee is guilty of a Class 4 felony.

(b) For purposes of this Section, ``correctional institution employee''
means any person employed to supervise and control inmates incarcerated
in a penitentiary, State farm, reformatory, prison, jail, house of
correction, police detention area, half-way house, or other institution
or place for the incarceration or custody of persons under sentence for
offenses or awaiting trial or sentence for offenses, under arrest for an
offense, a violation of probation, a violation of parole, a violation of
aftercare release, a violation of mandatory supervised release, or
awaiting a hearing or preliminary hearing on setting the conditions of
pretrial release, or who are sexually dangerous persons or who are
sexually violent persons; and ``firefighter'' means any individual,
either as an employee or volunteer, of a regularly constituted fire
department of a municipality or fire protection district who performs
fire fighting duties, including, but not limited to, the fire chief,
assistant fire chief, captain, engineer, driver, ladder person, hose
person, pipe person, and any other member of a regularly constituted
fire department. ``Firefighter'' also means a person employed by the
Office of the State Fire Marshal to conduct arson investigations.

(c) It is an affirmative defense to a violation of this Section if a
person resists or obstructs the performance of one known by the person
to be a firefighter by returning to or remaining in a dwelling,
residence, building, or other structure to rescue or to attempt to
rescue any person.

(d) A person shall not be subject to arrest for resisting arrest under
this Section unless there is an underlying offense for which the person
was initially subject to arrest.

(Source: P.A. 101-652, eff. 1-1-23; 102-28, eff. 6-25-21 .)

\hypertarget{ilcs-531-1a-from-ch.-38-par.-31-1a}{%
\subsection*{(720 ILCS 5/31-1a) (from Ch. 38, par.
31-1a)}\label{ilcs-531-1a-from-ch.-38-par.-31-1a}}
\addcontentsline{toc}{subsection}{(720 ILCS 5/31-1a) (from Ch. 38, par.
31-1a)}

\hypertarget{sec.-31-1a.-disarming-a-peace-officer-or-correctional-institution-employee.}{%
\section*{Sec. 31-1a. Disarming a peace officer or correctional
institution
employee.}\label{sec.-31-1a.-disarming-a-peace-officer-or-correctional-institution-employee.}}
\addcontentsline{toc}{section}{Sec. 31-1a. Disarming a peace officer or
correctional institution employee.}

\markright{Sec. 31-1a. Disarming a peace officer or correctional
institution employee.}

(a) A person who, without the consent of a peace officer or correctional
institution employee as defined in subsection (b) of Section 31-1, takes
a weapon from a person known to him or her to be a peace officer or
correctional institution employee, while the peace officer or
correctional institution employee is engaged in the performance of his
or her official duties or from an area within the peace officer's or
correctional institution employee's immediate presence is guilty of a
Class 1 felony.

(b) A person who, without the consent of a peace officer or correctional
institution employee as defined in subsection (b) of Section 31-1,
attempts to take a weapon from a person known to him or her to be a
peace officer or correctional institution employee, while the peace
officer or correctional institution employee is engaged in the
performance of his or her official duties or from an area within the
peace officer's or correctional institution employee's immediate
presence is guilty of a Class 2 felony.

(Source: P.A. 96-348, eff. 8-12-09.)

\hypertarget{ilcs-531-3-from-ch.-38-par.-31-3}{%
\subsection*{(720 ILCS 5/31-3) (from Ch. 38, par.
31-3)}\label{ilcs-531-3-from-ch.-38-par.-31-3}}
\addcontentsline{toc}{subsection}{(720 ILCS 5/31-3) (from Ch. 38, par.
31-3)}

\hypertarget{sec.-31-3.-obstructing-service-of-process.}{%
\section*{Sec. 31-3. Obstructing service of
process.}\label{sec.-31-3.-obstructing-service-of-process.}}
\addcontentsline{toc}{section}{Sec. 31-3. Obstructing service of
process.}

\markright{Sec. 31-3. Obstructing service of process.}

Whoever knowingly resists or obstructs the authorized service or
execution of any civil or criminal process or order of any court commits
a Class B misdemeanor.

(Source: P.A. 77-2638.)

\hypertarget{ilcs-531-4-from-ch.-38-par.-31-4}{%
\subsection*{(720 ILCS 5/31-4) (from Ch. 38, par.
31-4)}\label{ilcs-531-4-from-ch.-38-par.-31-4}}
\addcontentsline{toc}{subsection}{(720 ILCS 5/31-4) (from Ch. 38, par.
31-4)}

\hypertarget{sec.-31-4.-obstructing-justice.}{%
\section*{Sec. 31-4. Obstructing
justice.}\label{sec.-31-4.-obstructing-justice.}}
\addcontentsline{toc}{section}{Sec. 31-4. Obstructing justice.}

\markright{Sec. 31-4. Obstructing justice.}

(a) A person obstructs justice when, with intent to prevent the
apprehension or obstruct the prosecution or defense of any person, he or
she knowingly commits any of the following acts:

(1) Destroys, alters, conceals or disguises physical evidence, plants
false evidence, furnishes false information; or

(2) Induces a witness having knowledge material to the subject at issue
to leave the State or conceal himself or herself; or

(3) Possessing knowledge material to the subject at issue, he or she
leaves the State or conceals himself; or

(4) If a parent, legal guardian, or caretaker of a child under 13 years
of age reports materially false information to a law enforcement agency,
medical examiner, coroner, State's Attorney, or other governmental
agency during an investigation of the disappearance or death of a child
under circumstances described in subsection (a) or (b) of Section 10-10
of this Code.

(b) Sentence.

(1) Obstructing justice is a Class 4 felony, except as provided in
paragraph (2) of this subsection (b).

(2) Obstructing justice in furtherance of streetgang related or
gang-related activity, as defined in Section 10 of the Illinois
Streetgang Terrorism Omnibus Prevention Act, is a Class 3 felony.

(Source: P.A. 97-1079, eff. 1-1-13.)

\hypertarget{ilcs-531-4.5}{%
\subsection*{(720 ILCS 5/31-4.5)}\label{ilcs-531-4.5}}
\addcontentsline{toc}{subsection}{(720 ILCS 5/31-4.5)}

\hypertarget{sec.-31-4.5.-obstructing-identification.}{%
\section*{Sec. 31-4.5. Obstructing
identification.}\label{sec.-31-4.5.-obstructing-identification.}}
\addcontentsline{toc}{section}{Sec. 31-4.5. Obstructing identification.}

\markright{Sec. 31-4.5. Obstructing identification.}

(a) A person commits the offense of obstructing identification when he
or she intentionally or knowingly furnishes a false or fictitious name,
residence address, or date of birth to a peace officer who has:

(1) lawfully arrested the person;

(2) lawfully detained the person; or

(3) requested the information from a person that the peace officer has
good cause to believe is a witness to a criminal offense.

(b) Sentence. Obstructing identification is a Class A misdemeanor.

(Source: P.A. 96-335, eff. 1-1-10.)

\hypertarget{ilcs-531-5-from-ch.-38-par.-31-5}{%
\subsection*{(720 ILCS 5/31-5) (from Ch. 38, par.
31-5)}\label{ilcs-531-5-from-ch.-38-par.-31-5}}
\addcontentsline{toc}{subsection}{(720 ILCS 5/31-5) (from Ch. 38, par.
31-5)}

\hypertarget{sec.-31-5.-concealing-or-aiding-a-fugitive.}{%
\section*{Sec. 31-5. Concealing or aiding a
fugitive.}\label{sec.-31-5.-concealing-or-aiding-a-fugitive.}}
\addcontentsline{toc}{section}{Sec. 31-5. Concealing or aiding a
fugitive.}

\markright{Sec. 31-5. Concealing or aiding a fugitive.}

(a) Every person not standing in the relation of husband, wife, parent,
child, brother or sister to the offender, who, with intent to prevent
the apprehension of the offender, conceals his knowledge that an offense
has been committed or harbors, aids or conceals the offender, commits a
Class 4 felony.

(b) Every person, 18 years of age or older, who, with intent to prevent
the apprehension of the offender, aids or assists the offender, by some
volitional act, in fleeing the municipality, county, State, country, or
other defined jurisdiction in which the offender is to be arrested,
charged, or prosecuted, commits a Class 4 felony.

(Source: P.A. 97-741, eff. 1-1-13.)

\hypertarget{ilcs-531-6-from-ch.-38-par.-31-6}{%
\subsection*{(720 ILCS 5/31-6) (from Ch. 38, par.
31-6)}\label{ilcs-531-6-from-ch.-38-par.-31-6}}
\addcontentsline{toc}{subsection}{(720 ILCS 5/31-6) (from Ch. 38, par.
31-6)}

\hypertarget{sec.-31-6.-escape-failure-to-report-to-a-penal-institution-or-to-report-for-periodic-imprisonment.}{%
\section*{Sec. 31-6. Escape; failure to report to a penal institution or
to report for periodic
imprisonment.}\label{sec.-31-6.-escape-failure-to-report-to-a-penal-institution-or-to-report-for-periodic-imprisonment.}}
\addcontentsline{toc}{section}{Sec. 31-6. Escape; failure to report to a
penal institution or to report for periodic imprisonment.}

\markright{Sec. 31-6. Escape; failure to report to a penal institution
or to report for periodic imprisonment.}

(a) A person convicted of a felony or charged with the commission of a
felony, or charged with or adjudicated delinquent for an act which, if
committed by an adult, would constitute a felony, who intentionally
escapes from any penal institution or from the custody of an employee of
that institution commits a Class 2 felony; however, a person convicted
of a felony, or adjudicated delinquent for an act which, if committed by
an adult, would constitute a felony, who knowingly fails to report to a
penal institution or to report for periodic imprisonment at any time or
knowingly fails to return from furlough or from work and day release or
who knowingly fails to abide by the terms of home confinement is guilty
of a Class 3 felony.

(b) A person convicted of a misdemeanor or charged with the commission
of a misdemeanor, or charged with or adjudicated delinquent for an act
which, if committed by an adult, would constitute a misdemeanor, who
intentionally escapes from any penal institution or from the custody of
an employee of that institution commits a Class A misdemeanor; however,
a person convicted of a misdemeanor, or adjudicated delinquent for an
act which, if committed by an adult, would constitute a misdemeanor, who
knowingly fails to report to a penal institution or to report for
periodic imprisonment at any time or knowingly fails to return from
furlough or from work and day release or who knowingly fails to abide by
the terms of home confinement is guilty of a Class B misdemeanor.

(b-1) A person in the custody of the Department of Human Services under
the provisions of the Sexually Violent Persons Commitment Act under a
detention order, commitment order, conditional release order, or other
court order who intentionally escapes from any secure residential
facility or from a Department employee or any of its agents commits a
Class 2 felony.

(c) A person in the lawful custody of a peace officer for the alleged
commission of a felony offense or an act which, if committed by an
adult, would constitute a felony, and who intentionally escapes from
custody commits a Class 2 felony; however, a person in the lawful
custody of a peace officer for the alleged commission of a misdemeanor
offense or an act which, if committed by an adult, would constitute a
misdemeanor, who intentionally escapes from custody commits a Class A
misdemeanor.

(c-5) A person in the lawful custody of a peace officer for an alleged
violation of a term or condition of probation, conditional discharge,
parole, aftercare release, or mandatory supervised release for a felony
or an act which, if committed by an adult, would constitute a felony,
who intentionally escapes from custody is guilty of a Class 2 felony.

(c-6) A person in the lawful custody of a peace officer for an alleged
violation of a term or condition of supervision, probation, or
conditional discharge for a misdemeanor or an act which, if committed by
an adult, would constitute a misdemeanor, who intentionally escapes from
custody is guilty of a Class A misdemeanor.

(d) A person who violates this Section while armed with a dangerous
weapon commits a Class 1 felony.

(Source: P.A. 98-558, eff. 1-1-14; 98-770, eff. 1-1-15 .)

\hypertarget{ilcs-531-7-from-ch.-38-par.-31-7}{%
\subsection*{(720 ILCS 5/31-7) (from Ch. 38, par.
31-7)}\label{ilcs-531-7-from-ch.-38-par.-31-7}}
\addcontentsline{toc}{subsection}{(720 ILCS 5/31-7) (from Ch. 38, par.
31-7)}

\hypertarget{sec.-31-7.-aiding-escape.}{%
\section*{Sec. 31-7. Aiding escape.}\label{sec.-31-7.-aiding-escape.}}
\addcontentsline{toc}{section}{Sec. 31-7. Aiding escape.}

\markright{Sec. 31-7. Aiding escape.}

(a) Whoever, with intent to aid any prisoner in escaping from any penal
institution, conveys into the institution or transfers to the prisoner
anything for use in escaping commits a Class A misdemeanor.

(b) Whoever knowingly aids a person convicted of a felony or charged
with the commission of a felony, or charged with or adjudicated
delinquent for an act which, if committed by an adult, would constitute
a felony, in escaping from any penal institution or from the custody of
any employee of that institution commits a Class 2 felony; however,
whoever knowingly aids a person convicted of a felony or charged with
the commission of a felony, or charged with or adjudicated delinquent
for an act which, if committed by an adult, would constitute a felony,
in failing to return from furlough or from work and day release is
guilty of a Class 3 felony.

(c) Whoever knowingly aids a person convicted of a misdemeanor or
charged with the commission of a misdemeanor, or charged with or
adjudicated delinquent for an act which, if committed by an adult, would
constitute a misdemeanor, in escaping from any penal institution or from
the custody of an employee of that institution commits a Class A
misdemeanor; however, whoever knowingly aids a person convicted of a
misdemeanor or charged with the commission of a misdemeanor, or charged
with or adjudicated delinquent for an act which, if committed by an
adult, would constitute a misdemeanor, in failing to return from
furlough or from work and day release is guilty of a Class B
misdemeanor.

(d) Whoever knowingly aids a person in escaping from any public
institution, other than a penal institution, in which he is lawfully
detained, or from the custody of an employee of that institution,
commits a Class A misdemeanor.

(e) Whoever knowingly aids a person in the lawful custody of a peace
officer for the alleged commission of a felony offense or an act which,
if committed by an adult, would constitute a felony, in escaping from
custody commits a Class 2 felony; however, whoever knowingly aids a
person in the lawful custody of a peace officer for the alleged
commission of a misdemeanor offense or an act which, if committed by an
adult, would constitute a misdemeanor, in escaping from custody commits
a Class A misdemeanor.

(f) An officer or employee of any penal institution who recklessly
permits any prisoner in his custody to escape commits a Class A
misdemeanor.

(f-5) With respect to a person in the lawful custody of a peace officer
for an alleged violation of a term or condition of probation,
conditional discharge, parole, aftercare release, or mandatory
supervised release for a felony, whoever intentionally aids that person
to escape from that custody is guilty of a Class 2 felony.

(f-6) With respect to a person who is in the lawful custody of a peace
officer for an alleged violation of a term or condition of supervision,
probation, or conditional discharge for a misdemeanor, whoever
intentionally aids that person to escape from that custody is guilty of
a Class A misdemeanor.

(g) A person who violates this Section while armed with a dangerous
weapon commits a Class 2 felony.

(Source: P.A. 98-558, eff. 1-1-14.)

\hypertarget{ilcs-531-8-from-ch.-38-par.-31-8}{%
\subsection*{(720 ILCS 5/31-8) (from Ch. 38, par.
31-8)}\label{ilcs-531-8-from-ch.-38-par.-31-8}}
\addcontentsline{toc}{subsection}{(720 ILCS 5/31-8) (from Ch. 38, par.
31-8)}

\hypertarget{sec.-31-8.-refusing-to-aid-an-officer.}{%
\section*{Sec. 31-8. Refusing to aid an
officer.}\label{sec.-31-8.-refusing-to-aid-an-officer.}}
\addcontentsline{toc}{section}{Sec. 31-8. Refusing to aid an officer.}

\markright{Sec. 31-8. Refusing to aid an officer.}

Whoever upon command refuses or knowingly fails reasonably to aid a
person known by him to be a peace officer in:

(a) Apprehending a person whom the officer is authorized to apprehend;
or

(b) Preventing the commission by another of any offense, commits a petty
offense.

(Source: P.A. 77-2638.)

\hypertarget{ilcs-531-9}{%
\subsection*{(720 ILCS 5/31-9)}\label{ilcs-531-9}}
\addcontentsline{toc}{subsection}{(720 ILCS 5/31-9)}

\hypertarget{sec.-31-9.-obstructing-an-emergency-management-worker.-a-person-who-knowingly-obstructs-the-performance-by-one-known-to-the-person-to-be-an-emergency-management-worker-of-any-authorized-act-within-his-or-her-official-capacity-commits-a-class-a-misdemeanor.}{%
\section*{Sec. 31-9. Obstructing an emergency management worker. A
person who knowingly obstructs the performance by one known to the
person to be an emergency management worker of any authorized act within
his or her official capacity commits a Class A
misdemeanor.}\label{sec.-31-9.-obstructing-an-emergency-management-worker.-a-person-who-knowingly-obstructs-the-performance-by-one-known-to-the-person-to-be-an-emergency-management-worker-of-any-authorized-act-within-his-or-her-official-capacity-commits-a-class-a-misdemeanor.}}
\addcontentsline{toc}{section}{Sec. 31-9. Obstructing an emergency
management worker. A person who knowingly obstructs the performance by
one known to the person to be an emergency management worker of any
authorized act within his or her official capacity commits a Class A
misdemeanor.}

\markright{Sec. 31-9. Obstructing an emergency management worker. A
person who knowingly obstructs the performance by one known to the
person to be an emergency management worker of any authorized act within
his or her official capacity commits a Class A misdemeanor.}

(Source: P.A. 94-243, eff. 1-1-06.)

\bookmarksetup{startatroot}

\hypertarget{article-31a.-interference-with-penal-institution}{%
\chapter*{Article 31a. Interference With Penal
Institution}\label{article-31a.-interference-with-penal-institution}}
\addcontentsline{toc}{chapter}{Article 31a. Interference With Penal
Institution}

\markboth{Article 31a. Interference With Penal Institution}{Article 31a.
Interference With Penal Institution}

\hypertarget{ilcs-531a-0.1}{%
\subsection*{(720 ILCS 5/31A-0.1)}\label{ilcs-531a-0.1}}
\addcontentsline{toc}{subsection}{(720 ILCS 5/31A-0.1)}

\hypertarget{sec.-31a-0.1.-definitions.}{%
\section*{Sec. 31A-0.1. Definitions.}\label{sec.-31a-0.1.-definitions.}}
\addcontentsline{toc}{section}{Sec. 31A-0.1. Definitions.}

\markright{Sec. 31A-0.1. Definitions.}

For the purposes of this Article:

``Deliver'' or ``delivery'' means the actual, constructive or attempted
transfer of possession of an item of contraband, with or without
consideration, whether or not there is an agency relationship.

``Employee'' means any elected or appointed officer, trustee or employee
of a penal institution or of the governing authority of the penal
institution, or any person who performs services for the penal
institution pursuant to contract with the penal institution or its
governing authority.

``Item of contraband'' means any of the following:

(i) ``Alcoholic liquor'' as that term is defined in Section 1-3.05 of
the Liquor Control Act of 1934.

(ii) ``Cannabis'' as that term is defined in subsection (a) of Section 3
of the Cannabis Control Act.

(iii) ``Controlled substance'' as that term is defined in the Illinois
Controlled Substances Act.

(iii-a) ``Methamphetamine'' as that term is defined in the Illinois
Controlled Substances Act or the Methamphetamine Control and Community
Protection Act.

(iv) ``Hypodermic syringe'' or hypodermic needle, or any instrument
adapted for use of controlled substances or cannabis by subcutaneous
injection.

(v) ``Weapon'' means any knife, dagger, dirk, billy, razor, stiletto,
broken bottle, or other piece of glass which could be used as a
dangerous weapon. This term includes any of the devices or implements
designated in subsections (a)(1), (a)(3) and (a)(6) of Section 24-1 of
this Code, or any other dangerous weapon or instrument of like
character.

(vi) ``Firearm'' means any device, by whatever name known, which is
designed to expel a projectile or projectiles by the action of an
explosion, expansion of gas or escape of gas, including but not limited
to:

(A) any pneumatic gun, spring gun, or B-B gun which expels a single
globular projectile not exceeding .18 inch in diameter; or

(B) any device used exclusively for signaling or safety and required as
recommended by the United States Coast Guard or the Interstate Commerce
Commission; or

(C) any device used exclusively for the firing of stud cartridges,
explosive rivets or industrial ammunition; or

(D) any device which is powered by electrical charging units, such as
batteries, and which fires one or several barbs attached to a length of
wire and which, upon hitting a human, can send out current capable of
disrupting the person's nervous system in such a manner as to render him
or her incapable of normal functioning, commonly referred to as a stun
gun or taser.

(vii) ``Firearm ammunition'' means any self-contained cartridge or
shotgun shell, by whatever name known, which is designed to be used or
adaptable to use in a firearm, including but not limited to:

(A) any ammunition exclusively designed for use with a device used
exclusively for signaling or safety and required or recommended by the
United States Coast Guard or the Interstate Commerce Commission; or

(B) any ammunition designed exclusively for use with a stud or rivet
driver or other similar industrial ammunition.

(viii) ``Explosive'' means, but is not limited to, bomb, bombshell,
grenade, bottle or other container containing an explosive substance of
over one-quarter ounce for like purposes such as black powder bombs and
Molotov cocktails or artillery projectiles.

(ix) ``Tool to defeat security mechanisms'' means, but is not limited
to, handcuff or security restraint key, tool designed to pick locks,
popper, or any device or instrument used to or capable of unlocking or
preventing from locking any handcuff or security restraints, doors to
cells, rooms, gates or other areas of the penal institution.

(x) ``Cutting tool'' means, but is not limited to, hacksaw blade,
wirecutter, or device, instrument or file capable of cutting through
metal.

(xi) ``Electronic contraband'' for the purposes of Section 31A-1.1 of
this Article means, but is not limited to, any electronic, video
recording device, computer, or cellular communications equipment,
including, but not limited to, cellular telephones, cellular telephone
batteries, videotape recorders, pagers, computers, and computer
peripheral equipment brought into or possessed in a penal institution
without the written authorization of the Chief Administrative Officer.
``Electronic contraband'' for the purposes of Section 31A-1.2 of this
Article, means, but is not limited to, any electronic, video recording
device, computer, or cellular communications equipment, including, but
not limited to, cellular telephones, cellular telephone batteries,
videotape recorders, pagers, computers, and computer peripheral
equipment.

``Penal institution'' means any penitentiary, State farm, reformatory,
prison, jail, house of correction, police detention area, half-way house
or other institution or place for the incarceration or custody of
persons under sentence for offenses awaiting trial or sentence for
offenses, under arrest for an offense, a violation of probation, a
violation of parole, a violation of aftercare release, or a violation of
mandatory supervised release, or awaiting a hearing on the setting of
conditions of pretrial release or preliminary hearing; provided that
where the place for incarceration or custody is housed within another
public building this Article shall not apply to that part of the
building unrelated to the incarceration or custody of persons.

(Source: P.A. 101-652, eff. 1-1-23 .)

\hypertarget{ilcs-531a-1.1-from-ch.-38-par.-31a-1.1}{%
\subsection*{(720 ILCS 5/31A-1.1) (from Ch. 38, par.
31A-1.1)}\label{ilcs-531a-1.1-from-ch.-38-par.-31a-1.1}}
\addcontentsline{toc}{subsection}{(720 ILCS 5/31A-1.1) (from Ch. 38,
par. 31A-1.1)}

\hypertarget{sec.-31a-1.1.-bringing-contraband-into-a-penal-institution-possessing-contraband-in-a-penal-institution.}{%
\section*{Sec. 31A-1.1. Bringing Contraband into a Penal Institution;
Possessing Contraband in a Penal
Institution.}\label{sec.-31a-1.1.-bringing-contraband-into-a-penal-institution-possessing-contraband-in-a-penal-institution.}}
\addcontentsline{toc}{section}{Sec. 31A-1.1. Bringing Contraband into a
Penal Institution; Possessing Contraband in a Penal Institution.}

\markright{Sec. 31A-1.1. Bringing Contraband into a Penal Institution;
Possessing Contraband in a Penal Institution.}

(a) A person commits bringing contraband into a penal institution when
he or she knowingly and without authority of any person designated or
authorized to grant this authority (1) brings an item of contraband into
a penal institution or (2) causes another to bring an item of contraband
into a penal institution or (3) places an item of contraband in such
proximity to a penal institution as to give an inmate access to the
contraband.

(b) A person commits possessing contraband in a penal institution when
he or she knowingly possesses contraband in a penal institution,
regardless of the intent with which he or she possesses it.

(c) (Blank).

(d) Sentence.

(1) Bringing into or possessing alcoholic liquor in a penal institution
is a Class 4 felony.

(2) Bringing into or possessing cannabis in a penal institution is a
Class 3 felony.

(3) Bringing into or possessing any amount of a controlled substance
classified in Schedules III, IV or V of Article II of the Illinois
Controlled Substances Act in a penal institution is a Class 2 felony.

(4) Bringing into or possessing any amount of a controlled substance
classified in Schedules I or II of Article II of the Illinois Controlled
Substances Act in a penal institution is a Class 1 felony.

(5) Bringing into or possessing a hypodermic syringe in a penal
institution is a Class 1 felony.

(6) Bringing into or possessing a weapon, tool to defeat security
mechanisms, cutting tool, or electronic contraband in a penal
institution is a Class 1 felony.

(7) Bringing into or possessing a firearm, firearm ammunition, or
explosive in a penal institution is a Class X felony.

(e) It shall be an affirmative defense to subsection (b), that the
possession was specifically authorized by rule, regulation, or directive
of the governing authority of the penal institution or order issued
under it.

(f) It shall be an affirmative defense to subsection (a)(1) and
subsection (b) that the person bringing into or possessing contraband in
a penal institution had been arrested, and that person possessed the
contraband at the time of his or her arrest, and that the contraband was
brought into or possessed in the penal institution by that person as a
direct and immediate result of his or her arrest.

(g) Items confiscated may be retained for use by the Department of
Corrections or disposed of as deemed appropriate by the Chief
Administrative Officer in accordance with Department rules or disposed
of as required by law.

(Source: P.A. 97-1108, eff. 1-1-13; 98-756, eff. 7-16-14.)

\hypertarget{ilcs-531a-1.2-from-ch.-38-par.-31a-1.2}{%
\subsection*{(720 ILCS 5/31A-1.2) (from Ch. 38, par.
31A-1.2)}\label{ilcs-531a-1.2-from-ch.-38-par.-31a-1.2}}
\addcontentsline{toc}{subsection}{(720 ILCS 5/31A-1.2) (from Ch. 38,
par. 31A-1.2)}

\hypertarget{sec.-31a-1.2.-unauthorized-bringing-of-contraband-into-a-penal-institution-by-an-employee-unauthorized-possessing-of-contraband-in-a-penal-institution-by-an-employee-unauthorized-delivery-of-contraband-in-a-penal-institution-by-an-employee.}{%
\section*{Sec. 31A-1.2. Unauthorized bringing of contraband into a penal
institution by an employee; unauthorized possessing of contraband in a
penal institution by an employee; unauthorized delivery of contraband in
a penal institution by an
employee.}\label{sec.-31a-1.2.-unauthorized-bringing-of-contraband-into-a-penal-institution-by-an-employee-unauthorized-possessing-of-contraband-in-a-penal-institution-by-an-employee-unauthorized-delivery-of-contraband-in-a-penal-institution-by-an-employee.}}
\addcontentsline{toc}{section}{Sec. 31A-1.2. Unauthorized bringing of
contraband into a penal institution by an employee; unauthorized
possessing of contraband in a penal institution by an employee;
unauthorized delivery of contraband in a penal institution by an
employee.}

\markright{Sec. 31A-1.2. Unauthorized bringing of contraband into a
penal institution by an employee; unauthorized possessing of contraband
in a penal institution by an employee; unauthorized delivery of
contraband in a penal institution by an employee.}

(a) A person commits unauthorized bringing of contraband into a penal
institution by an employee when a person who is an employee knowingly
and without authority of any person designated or authorized to grant
this authority:

(1) brings or attempts to bring an item of contraband into a penal
institution, or

(2) causes or permits another to bring an item of contraband into a
penal institution.

(b) A person commits unauthorized possession of contraband in a penal
institution by an employee when a person who is an employee knowingly
and without authority of any person designated or authorized to grant
this authority possesses an item of contraband in a penal institution,
regardless of the intent with which he or she possesses it.

(c) A person commits unauthorized delivery of contraband in a penal
institution by an employee when a person who is an employee knowingly
and without authority of any person designated or authorized to grant
this authority:

(1) delivers or possesses with intent to deliver an item of contraband
to any inmate of a penal institution, or

(2) conspires to deliver or solicits the delivery of an item of
contraband to any inmate of a penal institution, or

(3) causes or permits the delivery of an item of contraband to any
inmate of a penal institution, or

(4) permits another person to attempt to deliver an item of contraband
to any inmate of a penal institution.

(d) For a violation of subsection (a) or (b) involving a cellular
telephone or cellular telephone battery, the defendant must intend to
provide the cellular telephone or cellular telephone battery to any
inmate in a penal institution, or to use the cellular telephone or
cellular telephone battery at the direction of an inmate or for the
benefit of any inmate of a penal institution.

(e) Sentence.

(1) A violation of paragraphs (a) or (b) of this

Section involving alcohol is a Class 4 felony. A violation of paragraph
(a) or (b) of this Section involving cannabis is a Class 2 felony. A
violation of paragraph (a) or (b) involving any amount of a controlled
substance classified in Schedules III, IV or V of Article II of the
Illinois Controlled Substances Act is a Class 1 felony. A violation of
paragraph (a) or (b) of this Section involving any amount of a
controlled substance classified in Schedules I or II of Article II of
the Illinois Controlled Substances Act is a Class X felony. A violation
of paragraph (a) or (b) involving a hypodermic syringe is a Class X
felony. A violation of paragraph (a) or (b) involving a weapon, tool to
defeat security mechanisms, cutting tool, or electronic contraband is a
Class 1 felony. A violation of paragraph (a) or (b) involving a firearm,
firearm ammunition, or explosive is a Class X felony.

(2) A violation of paragraph (c) of this Section involving alcoholic
liquor is a Class 3 felony. A violation of paragraph (c) involving
cannabis is a Class 1 felony. A violation of paragraph (c) involving any
amount of a controlled substance classified in Schedules III, IV or V of
Article II of the Illinois Controlled Substances Act is a Class X
felony. A violation of paragraph (c) involving any amount of a
controlled substance classified in Schedules I or II of Article II of
the Illinois Controlled Substances Act is a Class X felony for which the
minimum term of imprisonment shall be 8 years. A violation of paragraph
(c) involving a hypodermic syringe is a Class X felony for which the
minimum term of imprisonment shall be 8 years. A violation of paragraph
(c) involving a weapon, tool to defeat security mechanisms, cutting
tool, or electronic contraband is a Class X felony for which the minimum
term of imprisonment shall be 10 years. A violation of paragraph (c)
involving a firearm, firearm ammunition, or explosive is a Class X
felony for which the minimum term of imprisonment shall be 12 years.

(f) Items confiscated may be retained for use by the Department of
Corrections or disposed of as deemed appropriate by the Chief
Administrative Officer in accordance with Department rules or disposed
of as required by law.

(g) For a violation of subsection (a) or (b) involving alcoholic liquor,
a weapon, firearm, firearm ammunition, tool to defeat security
mechanisms, cutting tool, or electronic contraband, the items shall not
be considered to be in a penal institution when they are secured in an
employee's locked, private motor vehicle parked on the grounds of a
penal institution.

(Source: P.A. 96-328, eff. 8-11-09; 96-1112, eff. 1-1-11; 96-1325, eff.
7-27-10; 97-333, eff. 8-12-11; 97-1108, eff. 1-1-13.)

\bookmarksetup{startatroot}

\hypertarget{article-32.-interference-with-judicial-procedure}{%
\chapter*{Article 32. Interference With Judicial
Procedure}\label{article-32.-interference-with-judicial-procedure}}
\addcontentsline{toc}{chapter}{Article 32. Interference With Judicial
Procedure}

\markboth{Article 32. Interference With Judicial Procedure}{Article 32.
Interference With Judicial Procedure}

\hypertarget{ilcs-532-1-from-ch.-38-par.-32-1}{%
\subsection*{(720 ILCS 5/32-1) (from Ch. 38, par.
32-1)}\label{ilcs-532-1-from-ch.-38-par.-32-1}}
\addcontentsline{toc}{subsection}{(720 ILCS 5/32-1) (from Ch. 38, par.
32-1)}

\hypertarget{sec.-32-1.-compounding-a-crime.}{%
\section*{Sec. 32-1. Compounding a
crime.}\label{sec.-32-1.-compounding-a-crime.}}
\addcontentsline{toc}{section}{Sec. 32-1. Compounding a crime.}

\markright{Sec. 32-1. Compounding a crime.}

(a) A person commits compounding a crime when he or she knowingly
receives or offers to another any consideration for a promise not to
prosecute or aid in the prosecution of an offender.

(b) Sentence. Compounding a crime is a petty offense.

(Source: P.A. 97-1108, eff. 1-1-13.)

\hypertarget{ilcs-532-2-from-ch.-38-par.-32-2}{%
\subsection*{(720 ILCS 5/32-2) (from Ch. 38, par.
32-2)}\label{ilcs-532-2-from-ch.-38-par.-32-2}}
\addcontentsline{toc}{subsection}{(720 ILCS 5/32-2) (from Ch. 38, par.
32-2)}

\hypertarget{sec.-32-2.-perjury.}{%
\section*{Sec. 32-2. Perjury.}\label{sec.-32-2.-perjury.}}
\addcontentsline{toc}{section}{Sec. 32-2. Perjury.}

\markright{Sec. 32-2. Perjury.}

(a) A person commits perjury when, under oath or affirmation, in a
proceeding or in any other matter where by law the oath or affirmation
is required, he or she makes a false statement, material to the issue or
point in question, knowing the statement is false.

(b) Proof of Falsity.

An indictment or information for perjury alleging that the offender,
under oath, has knowingly made contradictory statements, material to the
issue or point in question, in the same or in different proceedings,
where the oath or affirmation is required, need not specify which
statement is false. At the trial, the prosecution need not establish
which statement is false.

(c) Admission of Falsity.

Where the contradictory statements are made in the same continuous
trial, an admission by the offender in that same continuous trial of the
falsity of a contradictory statement shall bar prosecution therefor
under any provisions of this Code.

(d) A person shall be exempt from prosecution under subsection (a) of
this Section if he or she is a peace officer who uses a false or
fictitious name in the enforcement of the criminal laws, and this use is
approved in writing as provided in Section 10-1 of ``The Liquor Control
Act of 1934'', as amended, Section 5 of ``An Act in relation to the use
of an assumed name in the conduct or transaction of business in this
State'', approved July 17, 1941, as amended, or Section 2605-200 of the
Illinois State Police Law. However, this exemption shall not apply to
testimony in judicial proceedings where the identity of the peace
officer is material to the issue, and he or she is ordered by the court
to disclose his or her identity.

(e) Sentence.

Perjury is a Class 3 felony.

(Source: P.A. 102-538, eff. 8-20-21.)

\hypertarget{ilcs-532-3-from-ch.-38-par.-32-3}{%
\subsection*{(720 ILCS 5/32-3) (from Ch. 38, par.
32-3)}\label{ilcs-532-3-from-ch.-38-par.-32-3}}
\addcontentsline{toc}{subsection}{(720 ILCS 5/32-3) (from Ch. 38, par.
32-3)}

\hypertarget{sec.-32-3.-subornation-of-perjury.}{%
\section*{Sec. 32-3. Subornation of
perjury.}\label{sec.-32-3.-subornation-of-perjury.}}
\addcontentsline{toc}{section}{Sec. 32-3. Subornation of perjury.}

\markright{Sec. 32-3. Subornation of perjury.}

(a) A person commits subornation of perjury when he or she knowingly
procures or induces another to make a statement in violation of Section
32-2 which the person knows to be false.

(b) Sentence.

Subornation of perjury is a Class 4 felony.

(Source: P.A. 97-1108, eff. 1-1-13.)

\hypertarget{ilcs-532-4-from-ch.-38-par.-32-4}{%
\subsection*{(720 ILCS 5/32-4) (from Ch. 38, par.
32-4)}\label{ilcs-532-4-from-ch.-38-par.-32-4}}
\addcontentsline{toc}{subsection}{(720 ILCS 5/32-4) (from Ch. 38, par.
32-4)}

\hypertarget{sec.-32-4.-communicating-with-jurors-and-witnesses.}{%
\section*{Sec. 32-4. Communicating with jurors and
witnesses.}\label{sec.-32-4.-communicating-with-jurors-and-witnesses.}}
\addcontentsline{toc}{section}{Sec. 32-4. Communicating with jurors and
witnesses.}

\markright{Sec. 32-4. Communicating with jurors and witnesses.}

(a) A person who, with intent to influence any person whom he believes
has been summoned as a juror, regarding any matter which is or may be
brought before such juror, communicates, directly or indirectly, with
such juror otherwise than as authorized by law commits a Class 4 felony.

(b) A person who, with intent to deter any party or witness from
testifying freely, fully and truthfully to any matter pending in any
court, or before a Grand Jury, Administrative agency or any other State
or local governmental unit, forcibly detains such party or witness, or
communicates, directly or indirectly, to such party or witness any
knowingly false information or a threat of injury or damage to the
property or person of any individual or offers or delivers or threatens
to withhold money or another thing of value to any individual commits a
Class 3 felony.

(c) A person who violates the Juror Protection Act commits a Class 4
felony.

(Source: P.A. 94-186, eff. 1-1-06.)

\hypertarget{ilcs-532-4a-from-ch.-38-par.-32-4a}{%
\subsection*{(720 ILCS 5/32-4a) (from Ch. 38, par.
32-4a)}\label{ilcs-532-4a-from-ch.-38-par.-32-4a}}
\addcontentsline{toc}{subsection}{(720 ILCS 5/32-4a) (from Ch. 38, par.
32-4a)}

\hypertarget{sec.-32-4a.-harassment-of-representatives-for-the-child-jurors-witnesses-and-others.}{%
\section*{Sec. 32-4a. Harassment of representatives for the child,
jurors, witnesses and
others.}\label{sec.-32-4a.-harassment-of-representatives-for-the-child-jurors-witnesses-and-others.}}
\addcontentsline{toc}{section}{Sec. 32-4a. Harassment of representatives
for the child, jurors, witnesses and others.}

\markright{Sec. 32-4a. Harassment of representatives for the child,
jurors, witnesses and others.}

(a) A person who, with intent to harass or annoy one who has served or
is serving or who is a family member of a person who has served or is
serving (1) as a juror because of the verdict returned by the jury in a
pending legal proceeding or the participation of the juror in the
verdict or (2) as a witness, or who may be expected to serve as a
witness in a pending legal proceeding, or who was expected to serve as a
witness but who did not serve as a witness because the charges against
the defendant were dismissed or because the defendant pleaded guilty to
the charges against him or her, because of the testimony or potential
testimony of the witness or person who may be expected or may have been
expected to serve as a witness, communicates directly or indirectly with
the juror, witness or person who may be expected or may have been
expected to serve as a witness, or family member of a juror or witness
or person who may be expected or may have been expected to serve as a
witness in such manner as to produce mental anguish or emotional
distress or who conveys a threat of injury or damage to the property or
person of any juror, witness or person who may be expected or may have
been expected to serve as a witness, or family member of the juror or
witness or person who may be expected or may have been expected to serve
as a witness commits a Class 2 felony.

(b) A person who, with intent to harass or annoy one who has served or
is serving or who is a family member of a person who has served or is
serving as a representative for the child, appointed under Section 506
of the Illinois Marriage and Dissolution of Marriage Act or Section
2-502 of the Code of Civil Procedure, because of the representative
service of that capacity, communicates directly or indirectly with the
representative or a family member of the representative in such manner
as to produce mental anguish or emotional distress or who conveys a
threat of injury or damage to the property or person of any
representative or a family member of the representative commits a Class
A misdemeanor.

(c) For purposes of this Section, ``family member'' means a spouse,
parent, child, stepchild or other person related by blood or by present
marriage, a person who has, or allegedly has a child in common, and a
person who shares or allegedly shares a blood relationship through a
child.

(Source: P.A. 93-108, eff. 1-1-04; 93-818, eff. 7-27-04.)

\hypertarget{ilcs-532-4b-from-ch.-38-par.-32-4b}{%
\subsection*{(720 ILCS 5/32-4b) (from Ch. 38, par.
32-4b)}\label{ilcs-532-4b-from-ch.-38-par.-32-4b}}
\addcontentsline{toc}{subsection}{(720 ILCS 5/32-4b) (from Ch. 38, par.
32-4b)}

\hypertarget{sec.-32-4b.-bribery-for-excuse-from-jury-duty.}{%
\section*{Sec. 32-4b. Bribery for excuse from jury
duty.}\label{sec.-32-4b.-bribery-for-excuse-from-jury-duty.}}
\addcontentsline{toc}{section}{Sec. 32-4b. Bribery for excuse from jury
duty.}

\markright{Sec. 32-4b. Bribery for excuse from jury duty.}

(a) A jury commissioner or any other person acting on behalf of a jury
commissioner commits bribery for excuse from jury duty, when he or she
knowingly requests, solicits, suggests, or accepts financial
compensation or any other form of consideration in exchange for a
promise to excuse or for excusing any person from jury duty.

(b) Sentence. Bribery for excuse from jury duty is a Class 3 felony. In
addition to any other penalty provided by law, a jury commissioner
convicted under this Section shall forfeit the performance bond required
by Section 1 of ``An Act in relation to jury commissioners and
authorizing judges to appoint such commissioners and to make rules
concerning their powers and duties'', approved June 15, 1887, as
amended, and shall be excluded from further service as a jury
commissioner.

(Source: P.A. 97-1108, eff. 1-1-13.)

\hypertarget{ilcs-532-4c}{%
\subsection*{(720 ILCS 5/32-4c)}\label{ilcs-532-4c}}
\addcontentsline{toc}{subsection}{(720 ILCS 5/32-4c)}

\hypertarget{sec.-32-4c.-witnesses-prohibition-on-accepting-payments-before-judgment-or-verdict.}{%
\section*{Sec. 32-4c. Witnesses; prohibition on accepting payments
before judgment or
verdict.}\label{sec.-32-4c.-witnesses-prohibition-on-accepting-payments-before-judgment-or-verdict.}}
\addcontentsline{toc}{section}{Sec. 32-4c. Witnesses; prohibition on
accepting payments before judgment or verdict.}

\markright{Sec. 32-4c. Witnesses; prohibition on accepting payments
before judgment or verdict.}

(a) A person who, after the commencement of a criminal prosecution, has
been identified in the criminal discovery process as a person who may be
called as a witness in a criminal proceeding shall not knowingly accept
or receive, directly or indirectly, any payment or benefit in
consideration for providing information obtained as a result of
witnessing an event or occurrence or having personal knowledge of
certain facts in relation to the criminal proceeding.

(b) Sentence. A violation of this Section is a Class B misdemeanor for
which the court may impose a fine not to exceed 3 times the amount of
compensation requested, accepted, or received.

(c) This Section remains applicable until the judgment of the court in
the action if the defendant is tried by the court without a jury or the
rendering of the verdict by the jury if the defendant is tried by jury
in the action.

(d) This Section does not apply to any of the following circumstances:

(1) Lawful compensation paid to expert witnesses, investigators,
employees, or agents by a prosecutor, law enforcement agency, or an
attorney employed to represent a person in a criminal matter.

(2) Lawful compensation or benefits provided to an informant by a
prosecutor or law enforcement agency.

(2.5) Lawful compensation or benefits, or both, provided to an informant
under a local anti-crime program, such as Crime Stoppers, We-Tip, and
similar programs designed to solve crimes or that foster the detection
of crime and encourage persons through the programs and otherwise to
come forward with information about criminal activity.

(2.6) Lawful compensation or benefits, or both, provided by a private
individual to another private individual as a reward for information
leading to the arrest and conviction of specified offenders.

(3) Lawful compensation paid to a publisher, editor, reporter, writer,
or other person connected with or employed by a newspaper, magazine,
television or radio station or any other publishing or media outlet for
disclosing information obtained from another person relating to an
offense.

(e) For purposes of this Section, ``publishing or media outlet'' means a
news gathering organization that sells or distributes news to
newspapers, television, or radio stations, or a cable or broadcast
television or radio network that disseminates news and information.

(f) The person identified as a witness may receive written notice from
counsel for either the prosecution or defense of the fact that he or she
has been identified as a witness who may be called in a criminal
proceeding and his or her responsibilities and possible penalties under
this Section. This Section shall be applicable only if the witness
received the written notice referred to in this subsection.

(Source: P.A. 97-1108, eff. 1-1-13.)

\hypertarget{ilcs-532-4d}{%
\subsection*{(720 ILCS 5/32-4d)}\label{ilcs-532-4d}}
\addcontentsline{toc}{subsection}{(720 ILCS 5/32-4d)}

\hypertarget{sec.-32-4d.-payment-of-jurors-by-parties-prohibited.}{%
\section*{Sec. 32-4d. Payment of jurors by parties
prohibited.}\label{sec.-32-4d.-payment-of-jurors-by-parties-prohibited.}}
\addcontentsline{toc}{section}{Sec. 32-4d. Payment of jurors by parties
prohibited.}

\markright{Sec. 32-4d. Payment of jurors by parties prohibited.}

(a) After a verdict has been rendered in a civil or criminal case, a
person who was a plaintiff or defendant in the case may not knowingly
offer or pay an award or other fee to a juror who was a member of the
jury that rendered the verdict in the case.

(b) After a verdict has been rendered in a civil or criminal case, a
member of the jury that rendered the verdict may not knowingly accept an
award or fee from the plaintiff or defendant in that case.

(c) Sentence. A violation of this Section is a Class A misdemeanor.

(d) This Section does not apply to the payment of a fee or award to a
person who was a juror for purposes unrelated to the jury's verdict or
to the outcome of the case.

(Source: P.A. 97-1108, eff. 1-1-13.)

\hypertarget{ilcs-532-4e}{%
\subsection*{(720 ILCS 5/32-4e)}\label{ilcs-532-4e}}
\addcontentsline{toc}{subsection}{(720 ILCS 5/32-4e)}

\hypertarget{sec.-32-4e.-interfering-with-the-duties-of-a-judicial-officer.}{%
\section*{Sec. 32-4e. Interfering with the duties of a judicial
officer.}\label{sec.-32-4e.-interfering-with-the-duties-of-a-judicial-officer.}}
\addcontentsline{toc}{section}{Sec. 32-4e. Interfering with the duties
of a judicial officer.}

\markright{Sec. 32-4e. Interfering with the duties of a judicial
officer.}

(a) A person may not give or offer to give benefits, promises, pecuniary
compensation, or any other form of compensation, either directly or
indirectly, to a judicial officer or a member of the judicial officer's
immediate family with the intent to:

(1) induce such judicial officer to do, or fail to do, any act in
violation of the lawful execution of his or her official duties; or

(2) induce such judicial officer to commit or aid in the commission of
any fraud, or to collude in, allow, or make available the opportunity
for the commission of any fraud on the State of Illinois.

(b) A person may not give or offer to give benefits, promises, pecuniary
compensation, or any other form of compensation, either directly or
indirectly, to court employees and staff with the intent to interfere
with the administration of the judicial process.

(c) Sentence. A person who violates this Section commits a Class 2
felony.

(d) Definitions. For purposes of this Section:

``Judicial officer'' means a justice, judge, associate judge, or
magistrate of a court of the United States of America or the State of
Illinois.

``Immediate family'' means a judicial officer's spouse or children.

(Source: P.A. 95-1035, eff. 6-1-09 .)

\hypertarget{ilcs-532-4f}{%
\subsection*{(720 ILCS 5/32-4f)}\label{ilcs-532-4f}}
\addcontentsline{toc}{subsection}{(720 ILCS 5/32-4f)}

\hypertarget{sec.-32-4f.-retaliating-against-a-judge-by-false-claim-slander-of-title-or-malicious-recording-of-fictitious-liens.}{%
\section*{Sec. 32-4f. Retaliating against a Judge by false claim,
slander of title, or malicious recording of fictitious
liens.}\label{sec.-32-4f.-retaliating-against-a-judge-by-false-claim-slander-of-title-or-malicious-recording-of-fictitious-liens.}}
\addcontentsline{toc}{section}{Sec. 32-4f. Retaliating against a Judge
by false claim, slander of title, or malicious recording of fictitious
liens.}

\markright{Sec. 32-4f. Retaliating against a Judge by false claim,
slander of title, or malicious recording of fictitious liens.}

A person who files or causes to be filed, in any public record or in any
private record that is generally available to the public, any false lien
or encumbrance against the real or personal property of a Supreme,
Appellate, Circuit, or Associate Judge of the State of Illinois with
knowledge that such lien or encumbrance is false or contains any
materially false, fictitious, or fraudulent statement or representation,
and with the intent of retaliating against that Judge for the
performance or non-performance of an official judicial duty, is guilty
of a violation of this Section. A person is guilty of a Class A
misdemeanor for a first offense and a Class 4 felony for a second or
subsequent offense.

(Source: P.A. 95-1035, eff. 6-1-09 .)

\hypertarget{ilcs-532-5-from-ch.-38-par.-32-5}{%
\subsection*{(720 ILCS 5/32-5) (from Ch. 38, par.
32-5)}\label{ilcs-532-5-from-ch.-38-par.-32-5}}
\addcontentsline{toc}{subsection}{(720 ILCS 5/32-5) (from Ch. 38, par.
32-5)}

\hypertarget{sec.-32-5.-repealed.}{%
\section*{Sec. 32-5. (Repealed).}\label{sec.-32-5.-repealed.}}
\addcontentsline{toc}{section}{Sec. 32-5. (Repealed).}

\markright{Sec. 32-5. (Repealed).}

(Source: P.A. 97-219, eff. 1-1-12. Repealed by P.A. 96-1551, eff.
7-1-11.)

\hypertarget{ilcs-532-5.1-from-ch.-38-par.-32-5.1}{%
\subsection*{(720 ILCS 5/32-5.1) (from Ch. 38, par.
32-5.1)}\label{ilcs-532-5.1-from-ch.-38-par.-32-5.1}}
\addcontentsline{toc}{subsection}{(720 ILCS 5/32-5.1) (from Ch. 38, par.
32-5.1)}

\hypertarget{sec.-32-5.1.-repealed.}{%
\section*{Sec. 32-5.1. (Repealed).}\label{sec.-32-5.1.-repealed.}}
\addcontentsline{toc}{section}{Sec. 32-5.1. (Repealed).}

\markright{Sec. 32-5.1. (Repealed).}

(Source: P.A. 94-730, eff. 4-17-06. Repealed by P.A. 96-1551, eff.
7-1-11 .)

\hypertarget{ilcs-532-5.1-1}{%
\subsection*{(720 ILCS 5/32-5.1-1)}\label{ilcs-532-5.1-1}}
\addcontentsline{toc}{subsection}{(720 ILCS 5/32-5.1-1)}

\hypertarget{sec.-32-5.1-1.-repealed.}{%
\section*{Sec. 32-5.1-1. (Repealed).}\label{sec.-32-5.1-1.-repealed.}}
\addcontentsline{toc}{section}{Sec. 32-5.1-1. (Repealed).}

\markright{Sec. 32-5.1-1. (Repealed).}

(Source: P.A. 94-730, eff. 4-17-06. Repealed by P.A. 96-1551, eff.
7-1-11 .)

\hypertarget{ilcs-532-5.2-from-ch.-38-par.-32-5.2}{%
\subsection*{(720 ILCS 5/32-5.2) (from Ch. 38, par.
32-5.2)}\label{ilcs-532-5.2-from-ch.-38-par.-32-5.2}}
\addcontentsline{toc}{subsection}{(720 ILCS 5/32-5.2) (from Ch. 38, par.
32-5.2)}

\hypertarget{sec.-32-5.2.-repealed.}{%
\section*{Sec. 32-5.2. (Repealed).}\label{sec.-32-5.2.-repealed.}}
\addcontentsline{toc}{section}{Sec. 32-5.2. (Repealed).}

\markright{Sec. 32-5.2. (Repealed).}

(Source: P.A. 95-331, eff. 8-21-07. Repealed by P.A. 96-1551, eff.
7-1-11 .)

\hypertarget{ilcs-532-5.2-5}{%
\subsection*{(720 ILCS 5/32-5.2-5)}\label{ilcs-532-5.2-5}}
\addcontentsline{toc}{subsection}{(720 ILCS 5/32-5.2-5)}

\hypertarget{sec.-32-5.2-5.-repealed.}{%
\section*{Sec. 32-5.2-5. (Repealed).}\label{sec.-32-5.2-5.-repealed.}}
\addcontentsline{toc}{section}{Sec. 32-5.2-5. (Repealed).}

\markright{Sec. 32-5.2-5. (Repealed).}

(Source: P.A. 94-341, eff. 1-1-06. Repealed by P.A. 96-1551, eff. 7-1-11
.)

\hypertarget{ilcs-532-5.3}{%
\subsection*{(720 ILCS 5/32-5.3)}\label{ilcs-532-5.3}}
\addcontentsline{toc}{subsection}{(720 ILCS 5/32-5.3)}

\hypertarget{sec.-32-5.3.-repealed.}{%
\section*{Sec. 32-5.3. (Repealed).}\label{sec.-32-5.3.-repealed.}}
\addcontentsline{toc}{section}{Sec. 32-5.3. (Repealed).}

\markright{Sec. 32-5.3. (Repealed).}

(Source: P.A. 88-677, eff. 12-15-94. Repealed by P.A. 96-1551, eff.
7-1-11 .)

\hypertarget{ilcs-532-5.4}{%
\subsection*{(720 ILCS 5/32-5.4)}\label{ilcs-532-5.4}}
\addcontentsline{toc}{subsection}{(720 ILCS 5/32-5.4)}

\hypertarget{sec.-32-5.4.-repealed.}{%
\section*{Sec. 32-5.4. (Repealed).}\label{sec.-32-5.4.-repealed.}}
\addcontentsline{toc}{section}{Sec. 32-5.4. (Repealed).}

\markright{Sec. 32-5.4. (Repealed).}

(Source: P.A. 94-323, eff. 1-1-06. Repealed by P.A. 96-1551, eff. 7-1-11
.)

\hypertarget{ilcs-532-5.4-1}{%
\subsection*{(720 ILCS 5/32-5.4-1)}\label{ilcs-532-5.4-1}}
\addcontentsline{toc}{subsection}{(720 ILCS 5/32-5.4-1)}

\hypertarget{sec.-32-5.4-1.-repealed.}{%
\section*{Sec. 32-5.4-1. (Repealed).}\label{sec.-32-5.4-1.-repealed.}}
\addcontentsline{toc}{section}{Sec. 32-5.4-1. (Repealed).}

\markright{Sec. 32-5.4-1. (Repealed).}

(Source: P.A. 94-730, eff. 4-17-06. Repealed by P.A. 96-1551, eff.
7-1-11 .)

\hypertarget{ilcs-532-5.5}{%
\subsection*{(720 ILCS 5/32-5.5)}\label{ilcs-532-5.5}}
\addcontentsline{toc}{subsection}{(720 ILCS 5/32-5.5)}

\hypertarget{sec.-32-5.5.-repealed.}{%
\section*{Sec. 32-5.5. (Repealed).}\label{sec.-32-5.5.-repealed.}}
\addcontentsline{toc}{section}{Sec. 32-5.5. (Repealed).}

\markright{Sec. 32-5.5. (Repealed).}

(Source: P.A. 94-730, eff. 4-17-06. Repealed by P.A. 96-1551, eff.
7-1-11 .)

\hypertarget{ilcs-532-5.6}{%
\subsection*{(720 ILCS 5/32-5.6)}\label{ilcs-532-5.6}}
\addcontentsline{toc}{subsection}{(720 ILCS 5/32-5.6)}

\hypertarget{sec.-32-5.6.-repealed.}{%
\section*{Sec. 32-5.6. (Repealed).}\label{sec.-32-5.6.-repealed.}}
\addcontentsline{toc}{section}{Sec. 32-5.6. (Repealed).}

\markright{Sec. 32-5.6. (Repealed).}

(Source: P.A. 94-323, eff. 1-1-06. Repealed by P.A. 96-1551, eff. 7-1-11
.)

\hypertarget{ilcs-532-5.7}{%
\subsection*{(720 ILCS 5/32-5.7)}\label{ilcs-532-5.7}}
\addcontentsline{toc}{subsection}{(720 ILCS 5/32-5.7)}

\hypertarget{sec.-32-5.7.-repealed.}{%
\section*{Sec. 32-5.7. (Repealed).}\label{sec.-32-5.7.-repealed.}}
\addcontentsline{toc}{section}{Sec. 32-5.7. (Repealed).}

\markright{Sec. 32-5.7. (Repealed).}

(Source: P.A. 94-323, eff. 1-1-06. Repealed by P.A. 96-1551, eff. 7-1-11
.)

\hypertarget{ilcs-532-6-from-ch.-38-par.-32-6}{%
\subsection*{(720 ILCS 5/32-6) (from Ch. 38, par.
32-6)}\label{ilcs-532-6-from-ch.-38-par.-32-6}}
\addcontentsline{toc}{subsection}{(720 ILCS 5/32-6) (from Ch. 38, par.
32-6)}

\hypertarget{sec.-32-6.-performance-of-unauthorized-acts.}{%
\section*{Sec. 32-6. Performance of unauthorized
acts.}\label{sec.-32-6.-performance-of-unauthorized-acts.}}
\addcontentsline{toc}{section}{Sec. 32-6. Performance of unauthorized
acts.}

\markright{Sec. 32-6. Performance of unauthorized acts.}

A person who performs any of the following acts, knowing that his
performance is not authorized by law, commits a Class 4 felony:

(a) Conducts a marriage ceremony; or

(b) Acknowledges the execution of any document which by law may be
recorded; or

(c) Becomes a surety for any party in any civil or criminal proceeding,
before any court or public officer authorized to accept such surety.

(Source: P.A. 77-2638)

\hypertarget{ilcs-532-7-from-ch.-38-par.-32-7}{%
\subsection*{(720 ILCS 5/32-7) (from Ch. 38, par.
32-7)}\label{ilcs-532-7-from-ch.-38-par.-32-7}}
\addcontentsline{toc}{subsection}{(720 ILCS 5/32-7) (from Ch. 38, par.
32-7)}

\hypertarget{sec.-32-7.-simulating-legal-process.}{%
\section*{Sec. 32-7. Simulating legal
process.}\label{sec.-32-7.-simulating-legal-process.}}
\addcontentsline{toc}{section}{Sec. 32-7. Simulating legal process.}

\markright{Sec. 32-7. Simulating legal process.}

(a) A person commits simulating legal process when he or she issues or
delivers any document which he or she knows falsely purports to be or
simulates any civil or criminal process.

(b) Sentence. Simulating legal process is a Class B misdemeanor.

(Source: P.A. 97-1108, eff. 1-1-13.)

\hypertarget{ilcs-532-8-from-ch.-38-par.-32-8}{%
\subsection*{(720 ILCS 5/32-8) (from Ch. 38, par.
32-8)}\label{ilcs-532-8-from-ch.-38-par.-32-8}}
\addcontentsline{toc}{subsection}{(720 ILCS 5/32-8) (from Ch. 38, par.
32-8)}

\hypertarget{sec.-32-8.-tampering-with-public-records.}{%
\section*{Sec. 32-8. Tampering with public
records.}\label{sec.-32-8.-tampering-with-public-records.}}
\addcontentsline{toc}{section}{Sec. 32-8. Tampering with public
records.}

\markright{Sec. 32-8. Tampering with public records.}

(a) A person commits tampering with public records when he or she
knowingly, without lawful authority, and with the intent to defraud any
party, public officer or entity, alters, destroys, defaces, removes or
conceals any public record.

(b) (Blank).

(c) A judge, circuit clerk or clerk of court, public official or
employee, court reporter, or other person commits tampering with public
records when he or she knowingly, without lawful authority, and with the
intent to defraud any party, public officer or entity, alters, destroys,
defaces, removes, or conceals any public record received or held by any
judge or by a clerk of any court.

(c-5) ``Public record'' expressly includes, but is not limited to, court
records, or documents, evidence, or exhibits filed with the clerk of the
court and which have become a part of the official court record,
pertaining to any civil or criminal proceeding in any court.

(d) Sentence. A violation of subsection (a) is a Class 4 felony. A
violation of subsection (c) is a Class 3 felony. Any person convicted
under subsection (c) who at the time of the violation was responsible
for making, keeping, storing, or reporting the record for which the
tampering occurred:

(1) shall forfeit his or her public office or public employment, if any,
and shall thereafter be ineligible for both State and local public
office and public employment in this State for a period of 5 years after
completion of any term of probation, conditional discharge, or
incarceration in a penitentiary including the period of mandatory
supervised release;

(2) shall forfeit all retirement, pension, and other benefits arising
out of public office or public employment as may be determined by the
court in accordance with the applicable provisions of the Illinois
Pension Code;

(3) shall be subject to termination of any professional licensure or
registration in this State as may be determined by the court in
accordance with the provisions of the applicable professional licensing
or registration laws;

(4) may be ordered by the court, after a hearing in accordance with
applicable law and in addition to any other penalty or fine imposed by
the court, to forfeit to the State an amount equal to any financial gain
or the value of any advantage realized by the person as a result of the
offense; and

(5) may be ordered by the court, after a hearing in accordance with
applicable law and in addition to any other penalty or fine imposed by
the court, to pay restitution to the victim in an amount equal to any
financial loss or the value of any advantage lost by the victim as a
result of the offense.

For the purposes of this subsection (d), an offense under subsection (c)
committed by a person holding public office or public employment shall
be rebuttably presumed to relate to or arise out of or in connection
with that public office or public employment.

(e) Any party litigant who believes a violation of this Section has
occurred may seek the restoration of the court record as provided in the
Court Records Restoration Act. Any order of the court denying the
restoration of the court record may be appealed as any other civil
judgment.

(f) When the sheriff or local law enforcement agency having jurisdiction
declines to investigate, or inadequately investigates, the court or any
interested party, shall notify the Illinois State Police of a suspected
violation of subsection (a) or (c), who shall have the authority to
investigate, and may investigate, the same, without regard to whether
the local law enforcement agency has requested the Illinois State Police
to do so.

(g) If the State's Attorney having jurisdiction declines to prosecute a
violation of subsection (a) or (c), the court or interested party shall
notify the Attorney General of the refusal. The Attorney General shall,
thereafter, have the authority to prosecute, and may prosecute, the
violation, without a referral from the State's Attorney.

(h) Prosecution of a violation of subsection (c) shall be commenced
within 3 years after the act constituting the violation is discovered or
reasonably should have been discovered.

(Source: P.A. 102-538, eff. 8-20-21.)

\hypertarget{ilcs-532-8.1}{%
\subsection*{(720 ILCS 5/32-8.1)}\label{ilcs-532-8.1}}
\addcontentsline{toc}{subsection}{(720 ILCS 5/32-8.1)}

\hypertarget{sec.-32-8.1.-tampering-with-a-certification-by-a-public-official.}{%
\section*{Sec. 32-8.1. Tampering with a certification by a public
official.}\label{sec.-32-8.1.-tampering-with-a-certification-by-a-public-official.}}
\addcontentsline{toc}{section}{Sec. 32-8.1. Tampering with a
certification by a public official.}

\markright{Sec. 32-8.1. Tampering with a certification by a public
official.}

(a) A person commits tampering with a certification by a public official
when he or she knowingly, without lawful authority, and with the intent
to defraud any individual, entity, public officer, or governmental unit,
uses a certification or part of a certification by a public official,
including but not limited to an apostille, the ``great seal of the State
of Illinois'', or other similar certification, in connection with any
document he or she knows or reasonably should know is not the original
document for which the public official originally issued the
certification.

(b) Sentence. Tampering with a certification by a public official is a
Class A misdemeanor for a first offense and a Class 4 felony for a
second or subsequent offense.

(Source: P.A. 98-170, eff. 8-5-13.)

\hypertarget{ilcs-532-9-from-ch.-38-par.-32-9}{%
\subsection*{(720 ILCS 5/32-9) (from Ch. 38, par.
32-9)}\label{ilcs-532-9-from-ch.-38-par.-32-9}}
\addcontentsline{toc}{subsection}{(720 ILCS 5/32-9) (from Ch. 38, par.
32-9)}

\hypertarget{sec.-32-9.-tampering-with-public-notice.}{%
\section*{Sec. 32-9. Tampering with public
notice.}\label{sec.-32-9.-tampering-with-public-notice.}}
\addcontentsline{toc}{section}{Sec. 32-9. Tampering with public notice.}

\markright{Sec. 32-9. Tampering with public notice.}

(a) A person commits tampering with public notice when he or she
knowingly and without lawful authority alters, destroys, defaces,
removes or conceals any public notice, posted according to law, during
the time for which the notice was to remain posted.

(b) Sentence. Tampering with public notice is a petty offense.

(Source: P.A. 97-1108, eff. 1-1-13.)

\hypertarget{ilcs-532-10-from-ch.-38-par.-32-10}{%
\subsection*{(720 ILCS 5/32-10) (from Ch. 38, par.
32-10)}\label{ilcs-532-10-from-ch.-38-par.-32-10}}
\addcontentsline{toc}{subsection}{(720 ILCS 5/32-10) (from Ch. 38, par.
32-10)}

\hypertarget{sec.-32-10.-violation-of-conditions-of-pretrial-release.}{%
\section*{Sec. 32-10. Violation of conditions of pretrial
release.}\label{sec.-32-10.-violation-of-conditions-of-pretrial-release.}}
\addcontentsline{toc}{section}{Sec. 32-10. Violation of conditions of
pretrial release.}

\markright{Sec. 32-10. Violation of conditions of pretrial release.}

(a) (Blank).

(a-5) Any person who knowingly violates a condition of pretrial release
by possessing a firearm in violation of his or her conditions of
pretrial release commits a Class 4 felony for a first violation and a
Class 3 felony for a second or subsequent violation.

(b) Whoever, having been released pretrial under conditions for
appearance before any court of this State, while charged with a criminal
offense in which the victim is a family or household member as defined
in Article 112A of the Code of Criminal Procedure of 1963, knowingly
violates a condition of that release as set forth in Section 110-10,
subsection (d) of the Code of Criminal Procedure of 1963, commits a
Class A misdemeanor.

(c) Whoever, having been released pretrial for appearance before any
court of this State for a felony, Class A misdemeanor or a criminal
offense in which the victim is a family or household member as defined
in Article 112A of the Code of Criminal Procedure of 1963, is charged
with any other felony, Class A misdemeanor, or a criminal offense in
which the victim is a family or household member as defined in Article
112A of the Code of Criminal Procedure of 1963 while on this release,
must appear before the court and may not be released by law enforcement
under 109-1 of the Code of Criminal Procedure of 1963 prior to the court
appearance.

(d) Nothing in this Section shall interfere with or prevent the exercise
by any court of its power to punish for contempt. Any sentence imposed
for violation of this Section may be served consecutive to the sentence
imposed for the charge for which pretrial release had been granted and
with respect to which the defendant has been convicted.

(Source: P.A. 101-652, eff. 1-1-23; 102-1104, eff. 1-1-23.)

\hypertarget{ilcs-532-11}{%
\subsection*{(720 ILCS 5/32-11)}\label{ilcs-532-11}}
\addcontentsline{toc}{subsection}{(720 ILCS 5/32-11)}

\hypertarget{sec.-32-11.-barratry.}{%
\section*{Sec. 32-11. Barratry.}\label{sec.-32-11.-barratry.}}
\addcontentsline{toc}{section}{Sec. 32-11. Barratry.}

\markright{Sec. 32-11. Barratry.}

If a person wickedly and willfully excites and stirs up actions or
quarrels between the people of this State with a view to promote strife
and contention, he or she is guilty of the petty offense of common
barratry; and if he or she is an attorney at law, he or she shall be
suspended from the practice of his or her profession, for any time not
exceeding 6 months.

(Source: P.A. 89-234, eff. 1-1-96.)

\hypertarget{ilcs-532-12}{%
\subsection*{(720 ILCS 5/32-12)}\label{ilcs-532-12}}
\addcontentsline{toc}{subsection}{(720 ILCS 5/32-12)}

\hypertarget{sec.-32-12.-maintenance.}{%
\section*{Sec. 32-12. Maintenance.}\label{sec.-32-12.-maintenance.}}
\addcontentsline{toc}{section}{Sec. 32-12. Maintenance.}

\markright{Sec. 32-12. Maintenance.}

If a person officiously intermeddles in an action that in no way belongs
to or concerns that person, by maintaining or assisting either party,
with money or otherwise, to prosecute or defend the action, with a view
to promote litigation, he or she is guilty of maintenance and upon
conviction shall be fined and punished as in cases of common barratry.
It is not maintenance for a person to maintain the action of his or her
relative or servant, or a poor person out of charity.

(Source: P.A. 89-234, eff. 1-1-96.)

\hypertarget{ilcs-532-13}{%
\subsection*{(720 ILCS 5/32-13)}\label{ilcs-532-13}}
\addcontentsline{toc}{subsection}{(720 ILCS 5/32-13)}

\hypertarget{sec.-32-13.-unlawful-clouding-of-title.}{%
\section*{Sec. 32-13. Unlawful clouding of
title.}\label{sec.-32-13.-unlawful-clouding-of-title.}}
\addcontentsline{toc}{section}{Sec. 32-13. Unlawful clouding of title.}

\markright{Sec. 32-13. Unlawful clouding of title.}

(a) Any person who intentionally records or files or causes to be
recorded or filed any document in the office of the recorder or
registrar of titles of any county of this State that is a cloud on the
title of land in this State, knowing that the theory upon which the
purported cloud on title is based is not recognized as a legitimate
legal theory by the courts of this State or of the United States,
commits the offense of unlawful clouding of title.

(b) Unlawful clouding of title is a Class A misdemeanor for a first
offense if the cloud on the title has a value that does not exceed
\$10,000. Unlawful clouding of title is a Class 4 felony if the cloud on
the title has a value that exceeds \$10,000, or for a second or
subsequent offense.

(c) In addition to any other sentence that may be imposed, the court
shall order any person convicted of a violation of this Section, or
placed on supervision for a violation of this Section, to execute a
release of the purported cloud on title as may be requested by or on
behalf of any person whose property is encumbered or potentially
encumbered by the document filed. Irrespective of whether or not a
person charged under this Section is convicted of the offense of
unlawful clouding of title, when the evidence demonstrates that, as a
matter of law, the cloud on title is not a type of cloud recognized or
authorized by the courts of this State or the United States, the court
shall forthwith direct the recorder or registrar of titles to expunge
the cloud.

(c-5) This Section does not apply to an attorney licensed to practice
law in this State who in good faith files a lien on behalf of his or her
client and who in good faith believes that the validity of the lien is
supported by statutory law, by a decision of a court of law, or by a
good faith argument for an extension, modification, or reversal of
existing court decisions relating to the validity of the lien.

(d) For purposes of this Section, ``cloud on title'' or ``cloud on the
title'' means an outstanding claim or encumbrance that, if valid, would
affect or impair the title of the owner of an estate in land and on its
face has that effect, but can be shown by extrinsic proof to be invalid
or inapplicable to that estate.

(Source: P.A. 98-98, eff. 1-1-14.)

\hypertarget{ilcs-532-14}{%
\subsection*{(720 ILCS 5/32-14)}\label{ilcs-532-14}}
\addcontentsline{toc}{subsection}{(720 ILCS 5/32-14)}

\hypertarget{sec.-32-14.-unlawful-manipulation-of-a-judicial-sale.}{%
\section*{Sec. 32-14. Unlawful manipulation of a judicial
sale.}\label{sec.-32-14.-unlawful-manipulation-of-a-judicial-sale.}}
\addcontentsline{toc}{section}{Sec. 32-14. Unlawful manipulation of a
judicial sale.}

\markright{Sec. 32-14. Unlawful manipulation of a judicial sale.}

(a) A person commits the offense of unlawful manipulation of a judicial
sale when he or she knowingly and by any means makes any contract with
or engages in any combination or conspiracy with any other person who
is, or but for a prior agreement is, a competitor of such person for the
purpose of or with the effect of fixing, controlling, limiting, or
otherwise manipulating (1) the participation of any person in, or (2)
the making of bids, at any judicial sale.

(b) Penalties. Unlawful manipulation of a judicial sale is a Class 3
felony. A mandatory fine shall be imposed for a violation, not to exceed
\$1,000,000 if the violator is a corporation, or, if the violator is any
other person, \$100,000. A second or subsequent violation is a Class 2
felony.

(c) Injunctive and other relief. The State's Attorney shall bring suit
in the circuit court to prevent and restrain violations of subsection
(a). In such a proceeding, the court shall determine whether a violation
has been committed, and shall enter such judgment as it considers
necessary to remove the effects of any violation which it finds, and to
prevent such violation from continuing or from being renewed in the
future. The court, in its discretion, may exercise all powers necessary
for this purpose, including, but not limited to, injunction and
divestiture of property.

(d) Private right of action. Any person who has been injured by a
violation of subsection (a) may maintain an action in the Circuit Court
for damages, or for an injunction, or both, against any person who has
committed such violation. If, in an action for an injunction, the court
issues an injunction, the plaintiff shall be awarded costs and
reasonable attorney's fees. In an action for damages, the person injured
shall be awarded 3 times the amount of actual damages. This State,
counties, municipalities, townships, and any political subdivision
organized under the authority of this State, and the United States, are
considered a person having standing to bring an action under this
subsection. Any action for damages under this subsection is forever
barred unless commenced within 4 years after the cause of action
accrued. In any action for damages under this subsection, the court may,
in its discretion, award reasonable fees to the prevailing defendant
upon a finding that the plaintiff acted in bad faith, vexatiously,
wantonly, or for oppressive reasons.

(e) Exclusion from subsequent judicial sales. Any person convicted of a
violation of subsection (a) or any similar offense of any state or the
United States shall be barred for 5 years from the date of conviction
from participating as a bidding entity in any judicial sale. No
corporation shall be barred from participating in a judicial sale as a
result of a conviction under subsection (a) of any employee or agent of
such corporation if the employee so convicted is no longer employed by
the corporation and: (1) it has been finally adjudicated not guilty or
(2) it demonstrates to the circuit court conducting such judicial sale
and the court so finds that the commission of the offense was neither
authorized, requested, commanded, nor performed by a director, officer
or a high managerial agent in behalf of the corporation as provided in
paragraph (2) of subsection (a) of Section 5-4 of this Code.

(f) Definitions. As used in this Section, unless the context otherwise
requires:

``Judicial sale'' means any sale of real or personal property in
accordance with a court order, including, but not limited to, judicial
sales conducted pursuant to Section 15-1507 of the Code of Civil
Procedure, sales ordered to satisfy judgments under Article XII of the
Code of Civil Procedure, and enforcements of delinquent property taxes
under Article 21 of the Property Tax Code.

``Person'' means any natural person, or any corporation, partnership, or
association of persons.

(Source: P.A. 100-201, eff. 8-18-17.)

\hypertarget{ilcs-532-15}{%
\subsection*{(720 ILCS 5/32-15)}\label{ilcs-532-15}}
\addcontentsline{toc}{subsection}{(720 ILCS 5/32-15)}

\hypertarget{sec.-32-15.-repealed.}{%
\section*{Sec. 32-15. (Repealed).}\label{sec.-32-15.-repealed.}}
\addcontentsline{toc}{section}{Sec. 32-15. (Repealed).}

\markright{Sec. 32-15. (Repealed).}

(Source: P.A. 101-652, eff. 1-1-23. Repealed by P.A. 102-1104, eff.
1-1-23.)

\bookmarksetup{startatroot}

\hypertarget{article-33.-official-misconduct}{%
\chapter*{Article 33. Official
Misconduct}\label{article-33.-official-misconduct}}
\addcontentsline{toc}{chapter}{Article 33. Official Misconduct}

\markboth{Article 33. Official Misconduct}{Article 33. Official
Misconduct}

\hypertarget{ilcs-533-1-from-ch.-38-par.-33-1}{%
\subsection*{(720 ILCS 5/33-1) (from Ch. 38, par.
33-1)}\label{ilcs-533-1-from-ch.-38-par.-33-1}}
\addcontentsline{toc}{subsection}{(720 ILCS 5/33-1) (from Ch. 38, par.
33-1)}

\hypertarget{sec.-33-1.-bribery.}{%
\section*{Sec. 33-1. Bribery.}\label{sec.-33-1.-bribery.}}
\addcontentsline{toc}{section}{Sec. 33-1. Bribery.}

\markright{Sec. 33-1. Bribery.}

A person commits bribery when:

(a) With intent to influence the performance of any act related to the
employment or function of any public officer, public employee, juror or
witness, he or she promises or tenders to that person any property or
personal advantage which he or she is not authorized by law to accept;
or

(b) With intent to influence the performance of any act related to the
employment or function of any public officer, public employee, juror or
witness, he or she promises or tenders to one whom he or she believes to
be a public officer, public employee, juror or witness, any property or
personal advantage which a public officer, public employee, juror or
witness would not be authorized by law to accept; or

(c) With intent to cause any person to influence the performance of any
act related to the employment or function of any public officer, public
employee, juror or witness, he or she promises or tenders to that person
any property or personal advantage which he or she is not authorized by
law to accept; or

(d) He or she receives, retains or agrees to accept any property or
personal advantage which he or she is not authorized by law to accept
knowing that the property or personal advantage was promised or tendered
with intent to cause him or her to influence the performance of any act
related to the employment or function of any public officer, public
employee, juror or witness; or

(e) He or she solicits, receives, retains, or agrees to accept any
property or personal advantage pursuant to an understanding that he or
she shall improperly influence or attempt to influence the performance
of any act related to the employment or function of any public officer,
public employee, juror or witness.

As used in this Section, ``tenders'' means any delivery or proffer made
with the requisite intent.

Sentence. Bribery is a Class 2 felony.

(Source: P.A. 97-1108, eff. 1-1-13; 98-756, eff. 7-16-14.)

\hypertarget{ilcs-533-2-from-ch.-38-par.-33-2}{%
\subsection*{(720 ILCS 5/33-2) (from Ch. 38, par.
33-2)}\label{ilcs-533-2-from-ch.-38-par.-33-2}}
\addcontentsline{toc}{subsection}{(720 ILCS 5/33-2) (from Ch. 38, par.
33-2)}

\hypertarget{sec.-33-2.-failure-to-report-a-bribe.}{%
\section*{Sec. 33-2. Failure to report a
bribe.}\label{sec.-33-2.-failure-to-report-a-bribe.}}
\addcontentsline{toc}{section}{Sec. 33-2. Failure to report a bribe.}

\markright{Sec. 33-2. Failure to report a bribe.}

Any public officer, public employee or juror who fails to report
forthwith to the local State's Attorney, or in the case of a State
employee to the Illinois State Police, any offer made to him in
violation of Section 33-1 commits a Class A misdemeanor.

In the case of a State employee, the making of such report to the
Illinois State Police shall discharge such employee from any further
duty under this Section. Upon receiving any such report, the Illinois
State Police shall forthwith transmit a copy thereof to the appropriate
State's Attorney.

(Source: P.A. 102-538, eff. 8-20-21.)

\hypertarget{ilcs-533-3-from-ch.-38-par.-33-3}{%
\subsection*{(720 ILCS 5/33-3) (from Ch. 38, par.
33-3)}\label{ilcs-533-3-from-ch.-38-par.-33-3}}
\addcontentsline{toc}{subsection}{(720 ILCS 5/33-3) (from Ch. 38, par.
33-3)}

\hypertarget{sec.-33-3.-official-misconduct.}{%
\section*{Sec. 33-3. Official
misconduct.}\label{sec.-33-3.-official-misconduct.}}
\addcontentsline{toc}{section}{Sec. 33-3. Official misconduct.}

\markright{Sec. 33-3. Official misconduct.}

(a) A public officer or employee or special government agent commits
misconduct when, in his official capacity or capacity as a special
government agent, he or she commits any of the following acts:

(1) Intentionally or recklessly fails to perform any mandatory duty as
required by law; or

(2) Knowingly performs an act which he knows he is forbidden by law to
perform; or

(3) With intent to obtain a personal advantage for himself or another,
he performs an act in excess of his lawful authority; or

(4) Solicits or knowingly accepts for the performance of any act a fee
or reward which he knows is not authorized by law.

(b) An employee of a law enforcement agency commits misconduct when he
or she knowingly uses or communicates, directly or indirectly,
information acquired in the course of employment, with the intent to
obstruct, impede, or prevent the investigation, apprehension, or
prosecution of any criminal offense or person. Nothing in this
subsection (b) shall be construed to impose liability for communicating
to a confidential resource, who is participating or aiding law
enforcement, in an ongoing investigation.

(c) A public officer or employee or special government agent convicted
of violating any provision of this Section forfeits his or her office or
employment or position as a special government agent. In addition, he or
she commits a Class 3 felony.

(d) For purposes of this Section:

``Special government agent'' has the meaning ascribed to it in
subsection (l) of Section 4A-101 of the Illinois Governmental Ethics
Act.

(Source: P.A. 101-652, eff. 7-1-21 .)

\hypertarget{ilcs-533-3.1}{%
\subsection*{(720 ILCS 5/33-3.1)}\label{ilcs-533-3.1}}
\addcontentsline{toc}{subsection}{(720 ILCS 5/33-3.1)}

\hypertarget{sec.-33-3.1.-solicitation-misconduct-state-government.}{%
\section*{Sec. 33-3.1. Solicitation misconduct (State
government).}\label{sec.-33-3.1.-solicitation-misconduct-state-government.}}
\addcontentsline{toc}{section}{Sec. 33-3.1. Solicitation misconduct
(State government).}

\markright{Sec. 33-3.1. Solicitation misconduct (State government).}

(a) An employee of an executive branch constitutional officer commits
solicitation misconduct (State government) when, at any time, he or she
knowingly solicits or receives contributions, as that term is defined in
Section 9-1.4 of the Election Code, from a person engaged in a business
or activity over which the person has regulatory authority.

(b) For the purpose of this Section, ``employee of an executive branch
constitutional officer'' means a full-time or part-time salaried
employee, full-time or part-time salaried appointee, or any contractual
employee of any office, board, commission, agency, department,
authority, administrative unit, or corporate outgrowth under the
jurisdiction of an executive branch constitutional officer; and
``regulatory authority'' means having the responsibility to investigate,
inspect, license, or enforce regulatory measures necessary to the
requirements of any State or federal statute or regulation relating to
the business or activity.

(c) An employee of an executive branch constitutional officer, including
one who does not have regulatory authority, commits a violation of this
Section if that employee knowingly acts in concert with an employee of
an executive branch constitutional officer who does have regulatory
authority to solicit or receive contributions in violation of this
Section.

(d) Solicitation misconduct (State government) is a Class A misdemeanor.
An employee of an executive branch constitutional officer convicted of
committing solicitation misconduct (State government) forfeits his or
her employment.

(e) An employee of an executive branch constitutional officer who is
discharged, demoted, suspended, threatened, harassed, or in any other
manner discriminated against in the terms and conditions of employment
because of lawful acts done by the employee or on behalf of the employee
or others in furtherance of the enforcement of this Section shall be
entitled to all relief necessary to make the employee whole.

(f) Any person who knowingly makes a false report of solicitation
misconduct (State government) to the Illinois State Police, the Attorney
General, a State's Attorney, or any law enforcement official is guilty
of a Class C misdemeanor.

(Source: P.A. 102-538, eff. 8-20-21.)

\hypertarget{ilcs-533-3.2}{%
\subsection*{(720 ILCS 5/33-3.2)}\label{ilcs-533-3.2}}
\addcontentsline{toc}{subsection}{(720 ILCS 5/33-3.2)}

\hypertarget{sec.-33-3.2.-solicitation-misconduct-local-government.}{%
\section*{Sec. 33-3.2. Solicitation misconduct (local
government).}\label{sec.-33-3.2.-solicitation-misconduct-local-government.}}
\addcontentsline{toc}{section}{Sec. 33-3.2. Solicitation misconduct
(local government).}

\markright{Sec. 33-3.2. Solicitation misconduct (local government).}

(a) An employee of a chief executive officer of a local government
commits solicitation misconduct (local government) when, at any time, he
or she knowingly solicits or receives contributions, as that term is
defined in Section 9-1.4 of the Election Code, from a person engaged in
a business or activity over which the person has regulatory authority.

(b) For the purpose of this Section, ``chief executive officer of a
local government'' means an executive officer of a county, township or
municipal government or any administrative subdivision under
jurisdiction of the county, township, or municipal government including
but not limited to: chairman or president of a county board or
commission, mayor or village president, township supervisor, county
executive, municipal manager, assessor, auditor, clerk, coroner,
recorder, sheriff or State's Attorney; ``employee of a chief executive
officer of a local government'' means a full-time or part-time salaried
employee, full-time or part-time salaried appointee, or any contractual
employee of any office, board, commission, agency, department,
authority, administrative unit, or corporate outgrowth under the
jurisdiction of a chief executive officer of a local government; and
``regulatory authority'' means having the responsibility to investigate,
inspect, license, or enforce regulatory measures necessary to the
requirements of any State, local, or federal statute or regulation
relating to the business or activity.

(c) An employee of a chief executive officer of a local government,
including one who does not have regulatory authority, commits a
violation of this Section if that employee knowingly acts in concert
with an employee of a chief executive officer of a local government who
does have regulatory authority to solicit or receive contributions in
violation of this Section.

(d) Solicitation misconduct (local government) is a Class A misdemeanor.
An employee of a chief executive officer of a local government convicted
of committing solicitation misconduct (local government) forfeits his or
her employment.

(e) An employee of a chief executive officer of a local government who
is discharged, demoted, suspended, threatened, harassed, or in any other
manner discriminated against in the terms and conditions of employment
because of lawful acts done by the employee or on behalf of the employee
or others in furtherance of the enforcement of this Section shall be
entitled to all relief necessary to make the employee whole.

(f) Any person who knowingly makes a false report of solicitation
misconduct (local government) to the Illinois State Police, the Attorney
General, a State's Attorney, or any law enforcement official is guilty
of a Class C misdemeanor.

(Source: P.A. 102-538, eff. 8-20-21.)

\hypertarget{ilcs-533-4}{%
\subsection*{(720 ILCS 5/33-4)}\label{ilcs-533-4}}
\addcontentsline{toc}{subsection}{(720 ILCS 5/33-4)}

\hypertarget{sec.-33-4.-peace-officer-or-correctional-officer-gang-related-activity-prohibited.}{%
\section*{Sec. 33-4. Peace officer or correctional officer; gang-related
activity
prohibited.}\label{sec.-33-4.-peace-officer-or-correctional-officer-gang-related-activity-prohibited.}}
\addcontentsline{toc}{section}{Sec. 33-4. Peace officer or correctional
officer; gang-related activity prohibited.}

\markright{Sec. 33-4. Peace officer or correctional officer;
gang-related activity prohibited.}

(a) It is unlawful for a peace officer or correctional officer to
knowingly commit any act in furtherance of gang-related activities,
except when acting in furtherance of an undercover law enforcement
investigation.

(b) In this Section, ``gang-related'' has the meaning ascribed to it in
Section 10 of the Illinois Streetgang Terrorism Omnibus Prevention Act.

(c) Sentence. A violation of this Section is a Class 3 felony.

(Source: P.A. 90-131, eff. 1-1-98 .)

\hypertarget{ilcs-533-5}{%
\subsection*{(720 ILCS 5/33-5)}\label{ilcs-533-5}}
\addcontentsline{toc}{subsection}{(720 ILCS 5/33-5)}

\hypertarget{sec.-33-5.-preservation-of-evidence.}{%
\section*{Sec. 33-5. Preservation of
evidence.}\label{sec.-33-5.-preservation-of-evidence.}}
\addcontentsline{toc}{section}{Sec. 33-5. Preservation of evidence.}

\markright{Sec. 33-5. Preservation of evidence.}

(a) It is unlawful for a law enforcement agency or an agent acting on
behalf of the law enforcement agency to intentionally fail to comply
with the provisions of subsection (a) of Section 116-4 of the Code of
Criminal Procedure of 1963.

(b) Sentence. A person who violates this Section is guilty of a Class 4
felony.

(c) For purposes of this Section, ``law enforcement agency'' has the
meaning ascribed to it in subsection (e) of Section 116-4 of the Code of
Criminal Procedure of 1963.

(Source: P.A. 91-871, eff. 1-1-01; 92-459, eff. 8-22-01.)

\hypertarget{ilcs-533-6}{%
\subsection*{(720 ILCS 5/33-6)}\label{ilcs-533-6}}
\addcontentsline{toc}{subsection}{(720 ILCS 5/33-6)}

\hypertarget{sec.-33-6.-bribery-to-obtain-driving-privileges.}{%
\section*{Sec. 33-6. Bribery to obtain driving
privileges.}\label{sec.-33-6.-bribery-to-obtain-driving-privileges.}}
\addcontentsline{toc}{section}{Sec. 33-6. Bribery to obtain driving
privileges.}

\markright{Sec. 33-6. Bribery to obtain driving privileges.}

(a) A person commits the offense of bribery to obtain driving privileges
when:

(1) with intent to influence any act related to the issuance of any
driver's license or permit by an employee of the Illinois Secretary of
State's Office, or the owner or employee of any commercial driver
training school licensed by the Illinois Secretary of State, or any
other individual authorized by the laws of this State to give driving
instructions or administer all or part of a driver's license
examination, he or she promises or tenders to that person any property
or personal advantage which that person is not authorized by law to
accept; or

(2) with intent to cause any person to influence any act related to the
issuance of any driver's license or permit by an employee of the
Illinois Secretary of State's Office, or the owner or employee of any
commercial driver training school licensed by the Illinois Secretary of
State, or any other individual authorized by the laws of this State to
give driving instructions or administer all or part of a driver's
license examination, he or she promises or tenders to that person any
property or personal advantage which that person is not authorized by
law to accept; or

(3) as an employee of the Illinois Secretary of

State's Office, or the owner or employee of any commercial driver
training school licensed by the Illinois Secretary of State, or any
other individual authorized by the laws of this State to give driving
instructions or administer all or part of a driver's license
examination, solicits, receives, retains, or agrees to accept any
property or personal advantage that he or she is not authorized by law
to accept knowing that such property or personal advantage was promised
or tendered with intent to influence the performance of any act related
to the issuance of any driver's license or permit; or

(4) as an employee of the Illinois Secretary of

State's Office, or the owner or employee of any commercial driver
training school licensed by the Illinois Secretary of State, or any
other individual authorized by the laws of this State to give driving
instructions or administer all or part of a driver's license
examination, solicits, receives, retains, or agrees to accept any
property or personal advantage pursuant to an understanding that he or
she shall improperly influence or attempt to influence the performance
of any act related to the issuance of any driver's license or permit.

(b) Sentence. Bribery to obtain driving privileges is a Class 2 felony.

(Source: P.A. 96-740, eff. 1-1-10; 96-962, eff. 7-2-10.)

\hypertarget{ilcs-533-7}{%
\subsection*{(720 ILCS 5/33-7)}\label{ilcs-533-7}}
\addcontentsline{toc}{subsection}{(720 ILCS 5/33-7)}

\hypertarget{sec.-33-7.-public-contractor-misconduct.}{%
\section*{Sec. 33-7. Public contractor
misconduct.}\label{sec.-33-7.-public-contractor-misconduct.}}
\addcontentsline{toc}{section}{Sec. 33-7. Public contractor misconduct.}

\markright{Sec. 33-7. Public contractor misconduct.}

(a) A public contractor; a person seeking a public contract on behalf of
himself, herself, or another; an employee of a public contractor; or a
person seeking a public contract on behalf of himself, herself, or
another commits public contractor misconduct when, in the performance
of, or in connection with, a contract with the State, a unit of local
government, or a school district or in obtaining or seeking to obtain
such a contract he or she commits any of the following acts:

(1) intentionally or knowingly makes, uses, or causes to be made or used
a false record or statement to conceal, avoid, or decrease an obligation
to pay or transmit money or property;

(2) knowingly performs an act that he or she knows he or she is
forbidden by law to perform;

(3) with intent to obtain a personal advantage for himself, herself, or
another, he or she performs an act in excess of his or her contractual
responsibility;

(4) solicits or knowingly accepts for the performance of any act a fee
or reward that he or she knows is not authorized by law; or

(5) knowingly or intentionally seeks or receives compensation or
reimbursement for goods and services he or she purported to deliver or
render, but failed to do so pursuant to the terms of the contract, to
the unit of State or local government or school district.

(b) Sentence. Any person who violates this Section commits a Class 3
felony. Any person convicted of this offense or a similar offense in any
state of the United States which contains the same elements of this
offense shall be barred for 10 years from the date of conviction from
contracting with, employment by, or holding public office with the State
or any unit of local government or school district. No corporation shall
be barred as a result of a conviction under this Section of any employee
or agent of such corporation if the employee so convicted is no longer
employed by the corporation and (1) it has been finally adjudicated not
guilty or (2) it demonstrates to the government entity with which it
seeks to contract, and that entity finds, that the commission of the
offense was neither authorized, requested, commanded, nor performed by a
director, officer or high managerial agent on behalf of the corporation
as provided in paragraph (2) of subsection (a) of Section 5-4 of this
Code.

(c) The Attorney General or the State's Attorney in the county where the
principal office of the unit of local government or school district is
located may bring a civil action on behalf of any unit of State or local
government to recover a civil penalty from any person who knowingly
engages in conduct which violates subsection (a) of this Section in
treble the amount of the monetary cost to the unit of State or local
government or school district involved in the violation. The Attorney
General or State's Attorney shall be entitled to recover reasonable
attorney's fees as part of the costs assessed to the defendant. This
subsection (c) shall in no way limit the ability of any unit of State or
local government or school district to recover moneys or damages
regarding public contracts under any other law or ordinance. A civil
action shall be barred unless the action is commenced within 6 years
after the later of (1) the date on which the conduct establishing the
cause of action occurred or (2) the date on which the unit of State or
local government or school district knew or should have known that the
conduct establishing the cause of action occurred.

(d) This amendatory Act of the 96th General Assembly shall not be
construed to create a private right of action.

(Source: P.A. 96-575, eff. 8-18-09.)

\hypertarget{ilcs-533-8}{%
\subsection*{(720 ILCS 5/33-8)}\label{ilcs-533-8}}
\addcontentsline{toc}{subsection}{(720 ILCS 5/33-8)}

\hypertarget{sec.-33-8.-legislative-misconduct.}{%
\section*{Sec. 33-8. Legislative
misconduct.}\label{sec.-33-8.-legislative-misconduct.}}
\addcontentsline{toc}{section}{Sec. 33-8. Legislative misconduct.}

\markright{Sec. 33-8. Legislative misconduct.}

(a) A member of the General Assembly commits legislative misconduct when
he or she knowingly accepts or receives, directly or indirectly, any
money or other valuable thing, from any corporation, company or person,
for any vote or influence he or she may give or withhold on any bill,
resolution or appropriation, or for any other official act.

(b) Sentence. Legislative misconduct is a Class 3 felony.

(Source: P.A. 97-1108, eff. 1-1-13.)

\hypertarget{ilcs-533-9}{%
\subsection*{(720 ILCS 5/33-9)}\label{ilcs-533-9}}
\addcontentsline{toc}{subsection}{(720 ILCS 5/33-9)}

\hypertarget{sec.-33-9.-law-enforcement-misconduct.}{%
\section*{Sec. 33-9. Law enforcement
misconduct.}\label{sec.-33-9.-law-enforcement-misconduct.}}
\addcontentsline{toc}{section}{Sec. 33-9. Law enforcement misconduct.}

\markright{Sec. 33-9. Law enforcement misconduct.}

(a) A law enforcement officer or a person acting under color of law
commits law enforcement misconduct when, in the performance of his or
her official duties with intent to prevent the apprehension or obstruct
the prosecution or defense of any person, he or she:

(1) knowingly and intentionally misrepresents or fails to provide
material facts describing an incident in any report or during any
investigations regarding the law enforcement employee's conduct;

(2) knowingly and intentionally withholds any knowledge of the material
misrepresentations of another law enforcement officer from the law
enforcement employee's supervisor, investigator, or other person or
entity tasked with holding the law enforcement officer accountable; or

(3) knowingly and intentionally fails to comply with paragraphs (3),
(5), (6), and (7) of subsection (a) of Section 10-20 of the Law
Enforcement Officer-Worn Body Camera Act.

(b) Sentence. Law enforcement misconduct is a Class 3 felony.

(Source: P.A. 101-652, eff. 7-1-21; 102-28, eff. 6-25-21.)

\hypertarget{ilcs-5tit.-iii-pt.-f-heading}{%
\subsection*{(720 ILCS 5/Tit. III Pt. F
heading)}\label{ilcs-5tit.-iii-pt.-f-heading}}
\addcontentsline{toc}{subsection}{(720 ILCS 5/Tit. III Pt. F heading)}

PART F.

CERTAIN AGGRAVATED OFFENSES

\bookmarksetup{startatroot}

\hypertarget{article-33a.-armed-violence}{%
\chapter*{Article 33a. Armed
Violence}\label{article-33a.-armed-violence}}
\addcontentsline{toc}{chapter}{Article 33a. Armed Violence}

\markboth{Article 33a. Armed Violence}{Article 33a. Armed Violence}

\hypertarget{ilcs-533a-1-from-ch.-38-par.-33a-1}{%
\subsection*{(720 ILCS 5/33A-1) (from Ch. 38, par.
33A-1)}\label{ilcs-533a-1-from-ch.-38-par.-33a-1}}
\addcontentsline{toc}{subsection}{(720 ILCS 5/33A-1) (from Ch. 38, par.
33A-1)}

\hypertarget{sec.-33a-1.-legislative-intent-and-definitions.}{%
\section*{Sec. 33A-1. Legislative intent and
definitions.}\label{sec.-33a-1.-legislative-intent-and-definitions.}}
\addcontentsline{toc}{section}{Sec. 33A-1. Legislative intent and
definitions.}

\markright{Sec. 33A-1. Legislative intent and definitions.}

(a) Legislative findings. The legislature finds and declares the
following:

(1) The use of a dangerous weapon in the commission of a felony offense
poses a much greater threat to the public health, safety, and general
welfare, than when a weapon is not used in the commission of the
offense.

(2) Further, the use of a firearm greatly facilitates the commission of
a criminal offense because of the more lethal nature of a firearm and
the greater perceived threat produced in those confronted by a person
wielding a firearm. Unlike other dangerous weapons such as knives and
clubs, the use of a firearm in the commission of a criminal felony
offense significantly escalates the threat and the potential for bodily
harm, and the greater range of the firearm increases the potential for
harm to more persons. Not only are the victims and bystanders at greater
risk when a firearm is used, but also the law enforcement officers whose
duty is to confront and apprehend the armed suspect.

(3) Current law does contain offenses involving the use or discharge of
a gun toward or against a person, such as aggravated battery with a
firearm, aggravated discharge of a firearm, and reckless discharge of a
firearm; however, the General Assembly has legislated greater penalties
for the commission of a felony while in possession of a firearm because
it deems such acts as more serious.

(b) Legislative intent.

(1) In order to deter the use of firearms in the commission of a felony
offense, the General Assembly deems it appropriate for a greater penalty
to be imposed when a firearm is used or discharged in the commission of
an offense than the penalty imposed for using other types of weapons and
for the penalty to increase on more serious offenses.

(2) With the additional elements of the discharge of a firearm and great
bodily harm inflicted by a firearm being added to armed violence and
other serious felony offenses, it is the intent of the General Assembly
to punish those elements more severely during commission of a felony
offense than when those elements stand alone as the act of the offender.

(3) It is the intent of the 91st General Assembly that should Public Act
88-680 be declared unconstitutional for a violation of Article 4,
Section 8 of the 1970 Constitution of the State of Illinois, the
amendatory changes made by Public Act 88-680 to Article 33A of the
Criminal Code of 1961 and which are set forth as law in this amendatory
Act of the 91st General Assembly are hereby reenacted by this amendatory
Act of the 91st General Assembly.

(c) Definitions.

(1) ``Armed with a dangerous weapon''. A person is considered armed with
a dangerous weapon for purposes of this Article, when he or she carries
on or about his or her person or is otherwise armed with a Category I,
Category II, or Category III weapon.

(2) A Category I weapon is a handgun, sawed-off shotgun, sawed-off
rifle, any other firearm small enough to be concealed upon the person,
semiautomatic firearm, or machine gun. A Category II weapon is any other
rifle, shotgun, spring gun, other firearm, stun gun or taser as defined
in paragraph (a) of Section 24-1 of this Code, knife with a blade of at
least 3 inches in length, dagger, dirk, switchblade knife, stiletto,
axe, hatchet, or other deadly or dangerous weapon or instrument of like
character. As used in this subsection (b) ``semiautomatic firearm''
means a repeating firearm that utilizes a portion of the energy of a
firing cartridge to extract the fired cartridge case and chamber the
next round and that requires a separate pull of the trigger to fire each
cartridge.

(3) A Category III weapon is a bludgeon, black-jack, slungshot,
sand-bag, sand-club, metal knuckles, billy, or other dangerous weapon of
like character.

(Source: P.A. 91-404, eff. 1-1-00; 91-696, eff. 4-13-00.)

\hypertarget{ilcs-533a-2-from-ch.-38-par.-33a-2}{%
\subsection*{(720 ILCS 5/33A-2) (from Ch. 38, par.
33A-2)}\label{ilcs-533a-2-from-ch.-38-par.-33a-2}}
\addcontentsline{toc}{subsection}{(720 ILCS 5/33A-2) (from Ch. 38, par.
33A-2)}

\hypertarget{sec.-33a-2.-armed-violence-elements-of-the-offense.}{%
\section*{Sec. 33A-2. Armed violence-Elements of the
offense.}\label{sec.-33a-2.-armed-violence-elements-of-the-offense.}}
\addcontentsline{toc}{section}{Sec. 33A-2. Armed violence-Elements of
the offense.}

\markright{Sec. 33A-2. Armed violence-Elements of the offense.}

(a) A person commits armed violence when, while armed with a dangerous
weapon, he commits any felony defined by Illinois Law, except first
degree murder, attempted first degree murder, intentional homicide of an
unborn child, second degree murder, involuntary manslaughter, reckless
homicide, predatory criminal sexual assault of a child, aggravated
battery of a child as described in Section 12-4.3 or subdivision (b)(1)
of Section 12-3.05, home invasion, or any offense that makes the
possession or use of a dangerous weapon either an element of the base
offense, an aggravated or enhanced version of the offense, or a
mandatory sentencing factor that increases the sentencing range.

(b) A person commits armed violence when he or she personally discharges
a firearm that is a Category I or Category II weapon while committing
any felony defined by Illinois law, except first degree murder,
attempted first degree murder, intentional homicide of an unborn child,
second degree murder, involuntary manslaughter, reckless homicide,
predatory criminal sexual assault of a child, aggravated battery of a
child as described in Section 12-4.3 or subdivision (b)(1) of Section
12-3.05, home invasion, or any offense that makes the possession or use
of a dangerous weapon either an element of the base offense, an
aggravated or enhanced version of the offense, or a mandatory sentencing
factor that increases the sentencing range.

(c) A person commits armed violence when he or she personally discharges
a firearm that is a Category I or Category II weapon that proximately
causes great bodily harm, permanent disability, or permanent
disfigurement or death to another person while committing any felony
defined by Illinois law, except first degree murder, attempted first
degree murder, intentional homicide of an unborn child, second degree
murder, involuntary manslaughter, reckless homicide, predatory criminal
sexual assault of a child, aggravated battery of a child as described in
Section 12-4.3 or subdivision (b)(1) of Section 12-3.05, home invasion,
or any offense that makes the possession or use of a dangerous weapon
either an element of the base offense, an aggravated or enhanced version
of the offense, or a mandatory sentencing factor that increases the
sentencing range.

(d) This Section does not apply to violations of the Fish and Aquatic
Life Code or the Wildlife Code.

(Source: P.A. 95-688, eff. 10-23-07; 96-1551, eff. 7-1-11 .)

\hypertarget{ilcs-533a-3-from-ch.-38-par.-33a-3}{%
\subsection*{(720 ILCS 5/33A-3) (from Ch. 38, par.
33A-3)}\label{ilcs-533a-3-from-ch.-38-par.-33a-3}}
\addcontentsline{toc}{subsection}{(720 ILCS 5/33A-3) (from Ch. 38, par.
33A-3)}

\hypertarget{sec.-33a-3.-sentence.}{%
\section*{Sec. 33A-3. Sentence.}\label{sec.-33a-3.-sentence.}}
\addcontentsline{toc}{section}{Sec. 33A-3. Sentence.}

\markright{Sec. 33A-3. Sentence.}

(a) Violation of Section 33A-2(a) with a Category I weapon is a Class X
felony for which the defendant shall be sentenced to a minimum term of
imprisonment of 15 years.

(a-5) Violation of Section 33A-2(a) with a Category II weapon is a Class
X felony for which the defendant shall be sentenced to a minimum term of
imprisonment of 10 years.

(b) Violation of Section 33A-2(a) with a Category III weapon is a Class
2 felony or the felony classification provided for the same act while
unarmed, whichever permits the greater penalty. A second or subsequent
violation of Section 33A-2(a) with a Category III weapon is a Class 1
felony or the felony classification provided for the same act while
unarmed, whichever permits the greater penalty.

(b-5) Violation of Section 33A-2(b) with a firearm that is a Category I
or Category II weapon is a Class X felony for which the defendant shall
be sentenced to a minimum term of imprisonment of 20 years.

(b-10) Violation of Section 33A-2(c) with a firearm that is a Category I
or Category II weapon is a Class X felony for which the defendant shall
be sentenced to a term of imprisonment of not less than 25 years nor
more than 40 years.

(c) Unless sentencing under subsection (a) of Section 5-4.5-95 of the
Unified Code of Corrections (730 ILCS 5/5-4.5-95) is applicable, any
person who violates subsection (a) or (b) of Section 33A-2 with a
firearm, when that person has been convicted in any state or federal
court of 3 or more of the following offenses: treason, first degree
murder, second degree murder, predatory criminal sexual assault of a
child, aggravated criminal sexual assault, criminal sexual assault,
robbery, burglary, arson, kidnaping, aggravated battery resulting in
great bodily harm or permanent disability or disfigurement, a violation
of the Methamphetamine Control and Community Protection Act, or a
violation of Section 401(a) of the Illinois Controlled Substances Act,
when the third offense was committed after conviction on the second, the
second offense was committed after conviction on the first, and the
violation of Section 33A-2 was committed after conviction on the third,
shall be sentenced to a term of imprisonment of not less than 25 years
nor more than 50 years.

(c-5) Except as otherwise provided in paragraph (b-10) or (c) of this
Section, a person who violates Section 33A-2(a) with a firearm that is a
Category I weapon or Section 33A-2(b) in any school, in any conveyance
owned, leased, or contracted by a school to transport students to or
from school or a school related activity, or on the real property
comprising any school or public park, and where the offense was related
to the activities of an organized gang, shall be sentenced to a term of
imprisonment of not less than the term set forth in subsection (a) or
(b-5) of this Section, whichever is applicable, and not more than 30
years. For the purposes of this subsection (c-5), ``organized gang'' has
the meaning ascribed to it in Section 10 of the Illinois Streetgang
Terrorism Omnibus Prevention Act.

(d) For armed violence based upon a predicate offense listed in this
subsection (d) the court shall enter the sentence for armed violence to
run consecutively to the sentence imposed for the predicate offense. The
offenses covered by this provision are:

(i) solicitation of murder,

(ii) solicitation of murder for hire,

(iii) heinous battery as described in Section 12-4.1 or subdivision
(a)(2) of Section 12-3.05,

(iv) aggravated battery of a senior citizen as described in Section
12-4.6 or subdivision (a)(4) of Section 12-3.05,

(v) (blank),

(vi) a violation of subsection (g) of Section 5 of the Cannabis Control
Act,

(vii) cannabis trafficking,

(viii) a violation of subsection (a) of Section 401 of the Illinois
Controlled Substances Act,

(ix) controlled substance trafficking involving a Class X felony amount
of controlled substance under Section 401 of the Illinois Controlled
Substances Act,

(x) calculated criminal drug conspiracy,

(xi) streetgang criminal drug conspiracy, or

(xii) a violation of the Methamphetamine Control and

Community Protection Act.

(Source: P.A. 95-688, eff. 10-23-07; 95-1052, eff. 7-1-09; 96-1551, eff.
7-1-11 .)

\bookmarksetup{startatroot}

\hypertarget{article-33b.-mandatory-life-sentence}{%
\chapter*{Article 33b. Mandatory Life
Sentence}\label{article-33b.-mandatory-life-sentence}}
\addcontentsline{toc}{chapter}{Article 33b. Mandatory Life Sentence}

\markboth{Article 33b. Mandatory Life Sentence}{Article 33b. Mandatory
Life Sentence}

A THIRD OR SUBSEQUENT FORCIBLE OFFENSE

(Repealed)

(Source: Repealed by P.A. 95-1052, eff. 7-1-09.)

\bookmarksetup{startatroot}

\hypertarget{article-33c.-deception}{%
\chapter*{Article 33c. Deception}\label{article-33c.-deception}}
\addcontentsline{toc}{chapter}{Article 33c. Deception}

\markboth{Article 33c. Deception}{Article 33c. Deception}

RELATING TO CERTIFICATION

OF DISADVANTAGED BUSINESS ENTERPRISES

(Repealed)

(Article repealed by P.A. 96-1551, eff. 7-1-11)

\bookmarksetup{startatroot}

\hypertarget{article-33d.-contributing-to-the}{%
\chapter*{Article 33d. Contributing To
The}\label{article-33d.-contributing-to-the}}
\addcontentsline{toc}{chapter}{Article 33d. Contributing To The}

\markboth{Article 33d. Contributing To The}{Article 33d. Contributing To
The}

CRIMINAL DELINQUENCY OF A JUVENILE

(Repealed)

(Source: P.A. 85-906. Repealed by P.A. 97-1109, eff. 1-1-13.)

\hypertarget{ilcs-533d-1-from-ch.-38-par.-33d-1}{%
\subsection*{(720 ILCS 5/33D-1) (from Ch. 38, par.
33D-1)}\label{ilcs-533d-1-from-ch.-38-par.-33d-1}}
\addcontentsline{toc}{subsection}{(720 ILCS 5/33D-1) (from Ch. 38, par.
33D-1)}

(This Section was renumbered as Section 12C-30 by P.A. 97-1109.)

\hypertarget{sec.-33d-1.-renumbered.}{%
\section*{Sec. 33D-1. (Renumbered).}\label{sec.-33d-1.-renumbered.}}
\addcontentsline{toc}{section}{Sec. 33D-1. (Renumbered).}

\markright{Sec. 33D-1. (Renumbered).}

(Source: P.A. 91-337, eff. 1-1-00. Renumbered by P.A. 97-1109, eff.
1-1-13.)

\bookmarksetup{startatroot}

\hypertarget{article-33e.-public-contracts}{%
\chapter*{Article 33e. Public
Contracts}\label{article-33e.-public-contracts}}
\addcontentsline{toc}{chapter}{Article 33e. Public Contracts}

\markboth{Article 33e. Public Contracts}{Article 33e. Public Contracts}

\hypertarget{ilcs-533e-1-from-ch.-38-par.-33e-1}{%
\subsection*{(720 ILCS 5/33E-1) (from Ch. 38, par.
33E-1)}\label{ilcs-533e-1-from-ch.-38-par.-33e-1}}
\addcontentsline{toc}{subsection}{(720 ILCS 5/33E-1) (from Ch. 38, par.
33E-1)}

\hypertarget{sec.-33e-1.-interference-with-public-contracting.}{%
\section*{Sec. 33E-1. Interference with public
contracting.}\label{sec.-33e-1.-interference-with-public-contracting.}}
\addcontentsline{toc}{section}{Sec. 33E-1. Interference with public
contracting.}

\markright{Sec. 33E-1. Interference with public contracting.}

It is the finding of the General Assembly that the cost to the public is
increased and the quality of goods, services and construction paid for
by public monies is decreased when contracts for such goods, services or
construction are obtained by any means other than through independent
noncollusive submission of bids or offers by individual contractors or
suppliers, and the evaluation of those bids or offers by the
governmental unit pursuant only to criteria publicly announced in
advance.

(Source: P.A. 85-1295.)

\hypertarget{ilcs-533e-2-from-ch.-38-par.-33e-2}{%
\subsection*{(720 ILCS 5/33E-2) (from Ch. 38, par.
33E-2)}\label{ilcs-533e-2-from-ch.-38-par.-33e-2}}
\addcontentsline{toc}{subsection}{(720 ILCS 5/33E-2) (from Ch. 38, par.
33E-2)}

\hypertarget{sec.-33e-2.-definitions.}{%
\section*{Sec. 33E-2. Definitions.}\label{sec.-33e-2.-definitions.}}
\addcontentsline{toc}{section}{Sec. 33E-2. Definitions.}

\markright{Sec. 33E-2. Definitions.}

In this Act:

(a) ``Public contract'' means any contract for goods, services or
construction let to any person with or without bid by any unit of State
or local government.

(b) ``Unit of State or local government'' means the State, any unit of
state government or agency thereof, any county or municipal government
or committee or agency thereof, or any other entity which is funded by
or expends tax dollars or the proceeds of publicly guaranteed bonds.

(c) ``Change order'' means a change in a contract term other than as
specifically provided for in the contract which authorizes or
necessitates any increase or decrease in the cost of the contract or the
time to completion.

(d) ``Person'' means any individual, firm, partnership, corporation,
joint venture or other entity, but does not include a unit of State or
local government.

(e) ``Person employed by any unit of State or local government'' means
any employee of a unit of State or local government and any person
defined in subsection (d) who is authorized by such unit of State or
local government to act on its behalf in relation to any public
contract.

(f) ``Sheltered market'' has the meaning ascribed to it in Section 8b of
the Business Enterprise for Minorities, Women, and Persons with
Disabilities Act; except that, with respect to State contracts set aside
for award to service-disabled veteran-owned small businesses and
veteran-owned small businesses pursuant to Section 45-57 of the Illinois
Procurement Code, ``sheltered market'' means procurements pursuant to
that Section.

(g) ``Kickback'' means any money, fee, commission, credit, gift,
gratuity, thing of value, or compensation of any kind which is provided,
directly or indirectly, to any prime contractor, prime contractor
employee, subcontractor, or subcontractor employee for the purpose of
improperly obtaining or rewarding favorable treatment in connection with
a prime contract or in connection with a subcontract relating to a prime
contract.

(h) ``Prime contractor'' means any person who has entered into a public
contract.

(i) ``Prime contractor employee'' means any officer, partner, employee,
or agent of a prime contractor.

(i-5) ``Stringing'' means knowingly structuring a contract or job order
to avoid the contract or job order being subject to competitive bidding
requirements.

(j) ``Subcontract'' means a contract or contractual action entered into
by a prime contractor or subcontractor for the purpose of obtaining
goods or services of any kind under a prime contract.

(k) ``Subcontractor'' (1) means any person, other than the prime
contractor, who offers to furnish or furnishes any goods or services of
any kind under a prime contract or a subcontract entered into in
connection with such prime contract; and (2) includes any person who
offers to furnish or furnishes goods or services to the prime contractor
or a higher tier subcontractor.

(l) ``Subcontractor employee'' means any officer, partner, employee, or
agent of a subcontractor.

(Source: P.A. 100-391, eff. 8-25-17.)

\hypertarget{ilcs-533e-3-from-ch.-38-par.-33e-3}{%
\subsection*{(720 ILCS 5/33E-3) (from Ch. 38, par.
33E-3)}\label{ilcs-533e-3-from-ch.-38-par.-33e-3}}
\addcontentsline{toc}{subsection}{(720 ILCS 5/33E-3) (from Ch. 38, par.
33E-3)}

\hypertarget{sec.-33e-3.-bid-rigging.}{%
\section*{Sec. 33E-3. Bid-rigging.}\label{sec.-33e-3.-bid-rigging.}}
\addcontentsline{toc}{section}{Sec. 33E-3. Bid-rigging.}

\markright{Sec. 33E-3. Bid-rigging.}

A person commits the offense of bid-rigging when he knowingly agrees
with any person who is, or but for such agreement would be, a competitor
of such person concerning any bid submitted or not submitted by such
person or another to a unit of State or local government when with the
intent that the bid submitted or not submitted will result in the award
of a contract to such person or another and he either (1) provides such
person or receives from another information concerning the price or
other material term or terms of the bid which would otherwise not be
disclosed to a competitor in an independent noncollusive submission of
bids or (2) submits a bid that is of such a price or other material term
or terms that he does not intend the bid to be accepted.

Bid-rigging is a Class 3 felony. Any person convicted of this offense or
any similar offense of any state or the United States which contains the
same elements as this offense shall be barred for 5 years from the date
of conviction from contracting with any unit of State or local
government. No corporation shall be barred from contracting with any
unit of State or local government as a result of a conviction under this
Section of any employee or agent of such corporation if the employee so
convicted is no longer employed by the corporation and: (1) it has been
finally adjudicated not guilty or (2) if it demonstrates to the
governmental entity with which it seeks to contract and that entity
finds that the commission of the offense was neither authorized,
requested, commanded, nor performed by a director, officer or a high
managerial agent in behalf of the corporation as provided in paragraph
(2) of subsection (a) of Section 5-4 of this Code.

(Source: P.A. 86-150.)

\hypertarget{ilcs-533e-4-from-ch.-38-par.-33e-4}{%
\subsection*{(720 ILCS 5/33E-4) (from Ch. 38, par.
33E-4)}\label{ilcs-533e-4-from-ch.-38-par.-33e-4}}
\addcontentsline{toc}{subsection}{(720 ILCS 5/33E-4) (from Ch. 38, par.
33E-4)}

\hypertarget{sec.-33e-4.-bid-rotating.}{%
\section*{Sec. 33E-4. Bid rotating.}\label{sec.-33e-4.-bid-rotating.}}
\addcontentsline{toc}{section}{Sec. 33E-4. Bid rotating.}

\markright{Sec. 33E-4. Bid rotating.}

A person commits the offense of bid rotating when, pursuant to any
collusive scheme or agreement with another, he engages in a pattern over
time (which, for the purposes of this Section, shall include at least 3
contract bids within a period of 10 years, the most recent of which
occurs after the effective date of this amendatory Act of 1988) of
submitting sealed bids to units of State or local government with the
intent that the award of such bids rotates, or is distributed among,
persons or business entities which submit bids on a substantial number
of the same contracts. Bid rotating is a Class 2 felony. Any person
convicted of this offense or any similar offense of any state or the
United States which contains the same elements as this offense shall be
permanently barred from contracting with any unit of State or local
government. No corporation shall be barred from contracting with any
unit of State or local government as a result of a conviction under this
Section of any employee or agent of such corporation if the employee so
convicted is no longer employed by the corporation and: (1) it has been
finally adjudicated not guilty or (2) if it demonstrates to the
governmental entity with which it seeks to contract and that entity
finds that the commission of the offense was neither authorized,
requested, commanded, nor performed by a director, officer or a high
managerial agent in behalf of the corporation as provided in paragraph
(2) of subsection (a) of Section 5-4 of this Code.

(Source: P.A. 86-150.)

\hypertarget{ilcs-533e-5-from-ch.-38-par.-33e-5}{%
\subsection*{(720 ILCS 5/33E-5) (from Ch. 38, par.
33E-5)}\label{ilcs-533e-5-from-ch.-38-par.-33e-5}}
\addcontentsline{toc}{subsection}{(720 ILCS 5/33E-5) (from Ch. 38, par.
33E-5)}

\hypertarget{sec.-33e-5.-acquisition-or-disclosure-of-bidding-information-by-public-official.-a-any-person-who-is-an-official-of-or-employed-by-any-unit-of-state-or-local-government-who-knowingly-opens-a-sealed-bid-at-a-time-or-place-other-than-as-specified-in-the-invitation-to-bid-or-as-otherwise-designated-by-the-state-or-unit-of-local-government-or-outside-the-presence-of-witnesses-required-by-the-applicable-statute-or-ordinance-commits-a-class-4-felony.}{%
\section*{Sec. 33E-5. Acquisition or disclosure of bidding information
by public official. (a) Any person who is an official of or employed by
any unit of State or local government who knowingly opens a sealed bid
at a time or place other than as specified in the invitation to bid or
as otherwise designated by the State or unit of local government, or
outside the presence of witnesses required by the applicable statute or
ordinance, commits a Class 4
felony.}\label{sec.-33e-5.-acquisition-or-disclosure-of-bidding-information-by-public-official.-a-any-person-who-is-an-official-of-or-employed-by-any-unit-of-state-or-local-government-who-knowingly-opens-a-sealed-bid-at-a-time-or-place-other-than-as-specified-in-the-invitation-to-bid-or-as-otherwise-designated-by-the-state-or-unit-of-local-government-or-outside-the-presence-of-witnesses-required-by-the-applicable-statute-or-ordinance-commits-a-class-4-felony.}}
\addcontentsline{toc}{section}{Sec. 33E-5. Acquisition or disclosure of
bidding information by public official. (a) Any person who is an
official of or employed by any unit of State or local government who
knowingly opens a sealed bid at a time or place other than as specified
in the invitation to bid or as otherwise designated by the State or unit
of local government, or outside the presence of witnesses required by
the applicable statute or ordinance, commits a Class 4 felony.}

\markright{Sec. 33E-5. Acquisition or disclosure of bidding information
by public official. (a) Any person who is an official of or employed by
any unit of State or local government who knowingly opens a sealed bid
at a time or place other than as specified in the invitation to bid or
as otherwise designated by the State or unit of local government, or
outside the presence of witnesses required by the applicable statute or
ordinance, commits a Class 4 felony.}

(b) Any person who is an official of or employed by any unit of State or
local government who knowingly discloses to any interested person any
information related to the terms of a sealed bid whether that
information is acquired through a violation of subsection (a) or by any
other means except as provided by law or necessary to the performance of
such official's or employee's responsibilities relating to the bid,
commits a Class 3 felony.

(c) It shall not constitute a violation of subsection (b) of this
Section for any person who is an official of or employed by any unit of
State or local government to make any disclosure to any interested
person where such disclosure is also made generally available to the
public.

(d) This Section only applies to contracts let by sealed bid.

(Source: P.A. 86-150.)

\hypertarget{ilcs-533e-6-from-ch.-38-par.-33e-6}{%
\subsection*{(720 ILCS 5/33E-6) (from Ch. 38, par.
33E-6)}\label{ilcs-533e-6-from-ch.-38-par.-33e-6}}
\addcontentsline{toc}{subsection}{(720 ILCS 5/33E-6) (from Ch. 38, par.
33E-6)}

\hypertarget{sec.-33e-6.-interference-with-contract-submission-and-award-by-public-official.}{%
\section*{Sec. 33E-6. Interference with contract submission and award by
public
official.}\label{sec.-33e-6.-interference-with-contract-submission-and-award-by-public-official.}}
\addcontentsline{toc}{section}{Sec. 33E-6. Interference with contract
submission and award by public official.}

\markright{Sec. 33E-6. Interference with contract submission and award
by public official.}

(a) Any person who is an official of or employed by any unit of State or
local government who knowingly conveys, either directly or indirectly,
outside of the publicly available official invitation to bid, pre-bid
conference, solicitation for contracts procedure or such procedure used
in any sheltered market procurement adopted pursuant to law or ordinance
by that unit of government, to any person any information concerning the
specifications for such contract or the identity of any particular
potential subcontractors, when inclusion of such information concerning
the specifications or contractors in the bid or offer would influence
the likelihood of acceptance of such bid or offer, commits a Class 4
felony. It shall not constitute a violation of this subsection to convey
information intended to clarify plans or specifications regarding a
public contract where such disclosure of information is also made
generally available to the public.

(b) Any person who is an official of or employed by any unit of State or
local government who, either directly or indirectly, knowingly informs a
bidder or offeror that the bid or offer will be accepted or executed
only if specified individuals are included as subcontractors commits a
Class 3 felony.

(c) It shall not constitute a violation of subsection (a) of this
Section where any person who is an official of or employed by any unit
of State or local government follows procedures established (i) by
federal, State or local minority or female owned business enterprise
programs or (ii) pursuant to Section 45-57 of the Illinois Procurement
Code.

(d) Any bidder or offeror who is the recipient of communications from
the unit of government which he reasonably believes to be proscribed by
subsections (a) or (b), and fails to inform either the Attorney General
or the State's Attorney for the county in which the unit of government
is located, commits a Class A misdemeanor.

(e) Any public official who knowingly awards a contract based on
criteria which were not publicly disseminated via the invitation to bid,
when such invitation to bid is required by law or ordinance, the pre-bid
conference, or any solicitation for contracts procedure or such
procedure used in any sheltered market procurement procedure adopted
pursuant to statute or ordinance, commits a Class 3 felony.

(f) It shall not constitute a violation of subsection (a) for any person
who is an official of or employed by any unit of State or local
government to provide to any person a copy of the transcript or other
summary of any pre-bid conference where such transcript or summary is
also made generally available to the public.

(Source: P.A. 97-260, eff. 8-5-11.)

\hypertarget{ilcs-533e-7-from-ch.-38-par.-33e-7}{%
\subsection*{(720 ILCS 5/33E-7) (from Ch. 38, par.
33E-7)}\label{ilcs-533e-7-from-ch.-38-par.-33e-7}}
\addcontentsline{toc}{subsection}{(720 ILCS 5/33E-7) (from Ch. 38, par.
33E-7)}

\hypertarget{sec.-33e-7.-kickbacks.}{%
\section*{Sec. 33E-7. Kickbacks.}\label{sec.-33e-7.-kickbacks.}}
\addcontentsline{toc}{section}{Sec. 33E-7. Kickbacks.}

\markright{Sec. 33E-7. Kickbacks.}

(a) A person violates this Section when he knowingly either:

(1) provides, attempts to provide or offers to provide any kickback;

(2) solicits, accepts or attempts to accept any kickback; or

(3) includes, directly or indirectly, the amount of any kickback
prohibited by paragraphs (1) or (2) of this subsection (a) in the
contract price charged by a subcontractor to a prime contractor or a
higher tier subcontractor or in the contract price charged by a prime
contractor to any unit of State or local government for a public
contract.

(b) Any person violates this Section when he has received an offer of a
kickback, or has been solicited to make a kickback, and fails to report
it to law enforcement officials, including but not limited to the
Attorney General or the State's Attorney for the county in which the
contract is to be performed.

(c) A violation of subsection (a) is a Class 3 felony. A violation of
subsection (b) is a Class 4 felony.

(d) Any unit of State or local government may, in a civil action,
recover a civil penalty from any person who knowingly engages in conduct
which violates paragraph (3) of subsection (a) of this Section in twice
the amount of each kickback involved in the violation. This subsection
(d) shall in no way limit the ability of any unit of State or local
government to recover monies or damages regarding public contracts under
any other law or ordinance. A civil action shall be barred unless the
action is commenced within 6 years after the later of (1) the date on
which the conduct establishing the cause of action occurred or (2) the
date on which the unit of State or local government knew or should have
known that the conduct establishing the cause of action occurred.

(Source: P.A. 85-1295.)

\hypertarget{ilcs-533e-8-from-ch.-38-par.-33e-8}{%
\subsection*{(720 ILCS 5/33E-8) (from Ch. 38, par.
33E-8)}\label{ilcs-533e-8-from-ch.-38-par.-33e-8}}
\addcontentsline{toc}{subsection}{(720 ILCS 5/33E-8) (from Ch. 38, par.
33E-8)}

\hypertarget{sec.-33e-8.-bribery-of-inspector-employed-by-contractor.}{%
\section*{Sec. 33E-8. Bribery of inspector employed by
contractor.}\label{sec.-33e-8.-bribery-of-inspector-employed-by-contractor.}}
\addcontentsline{toc}{section}{Sec. 33E-8. Bribery of inspector employed
by contractor.}

\markright{Sec. 33E-8. Bribery of inspector employed by contractor.}

(a) A person commits bribery of an inspector when he offers to any
person employed by a contractor or subcontractor on any public project
contracted for by any unit of State or local government any property or
other thing of value with the intent that such offer is for the purpose
of obtaining wrongful certification or approval of the quality or
completion of any goods or services supplied or performed in the course
of work on such project. Violation of this subsection is a Class 4
felony.

(b) Any person employed by a contractor or subcontractor on any public
project contracted for by any unit of State or local government who
accepts any property or other thing of value knowing that such was
intentionally offered for the purpose of influencing the certification
or approval of the quality or completion of any goods or services
supplied or performed under subcontract to that contractor, and either
before or afterwards issues such wrongful certification, commits a Class
3 felony. Failure to report such offer to law enforcement officials,
including but not limited to the Attorney General or the State's
Attorney for the county in which the contract is performed, constitutes
a Class 4 felony.

(Source: P.A. 85-1295.)

\hypertarget{ilcs-533e-9-from-ch.-38-par.-33e-9}{%
\subsection*{(720 ILCS 5/33E-9) (from Ch. 38, par.
33E-9)}\label{ilcs-533e-9-from-ch.-38-par.-33e-9}}
\addcontentsline{toc}{subsection}{(720 ILCS 5/33E-9) (from Ch. 38, par.
33E-9)}

\hypertarget{sec.-33e-9.-change-orders.}{%
\section*{Sec. 33E-9. Change orders.}\label{sec.-33e-9.-change-orders.}}
\addcontentsline{toc}{section}{Sec. 33E-9. Change orders.}

\markright{Sec. 33E-9. Change orders.}

Any change order authorized under this Section shall be made in writing.
Any person employed by and authorized by any unit of State or local
government to approve a change order to any public contract who
knowingly grants that approval without first obtaining from the unit of
State or local government on whose behalf the contract was signed, or
from a designee authorized by that unit of State or local government, a
determination in writing that (1) the circumstances said to necessitate
the change in performance were not reasonably foreseeable at the time
the contract was signed, or (2) the change is germane to the original
contract as signed, or (3) the change order is in the best interest of
the unit of State or local government and authorized by law, commits a
Class 4 felony. The written determination and the written change order
resulting from that determination shall be preserved in the contract's
file which shall be open to the public for inspection. This Section
shall only apply to a change order or series of change orders which
authorize or necessitate an increase or decrease in either the cost of a
public contract by a total of \$25,000 or more or the time of completion
by a total of 180 days or more.

(Source: P.A. 102-1119, eff. 1-23-23.)

\hypertarget{ilcs-533e-10-from-ch.-38-par.-33e-10}{%
\subsection*{(720 ILCS 5/33E-10) (from Ch. 38, par.
33E-10)}\label{ilcs-533e-10-from-ch.-38-par.-33e-10}}
\addcontentsline{toc}{subsection}{(720 ILCS 5/33E-10) (from Ch. 38, par.
33E-10)}

\hypertarget{sec.-33e-10.-rules-of-evidence.}{%
\section*{Sec. 33E-10. Rules of
evidence.}\label{sec.-33e-10.-rules-of-evidence.}}
\addcontentsline{toc}{section}{Sec. 33E-10. Rules of evidence.}

\markright{Sec. 33E-10. Rules of evidence.}

(a) The certified bid is prima facie evidence of the bid.

(b) It shall be presumed that in the absence of practices proscribed by
this Article 33E, all persons who submit bids in response to an
invitation to bid by any unit of State or local government submit their
bids independent of all other bidders, without information obtained from
the governmental entity outside the invitation to bid, and in a good
faith effort to obtain the contract.

(Source: P.A. 85-1295.)

\hypertarget{ilcs-533e-11-from-ch.-38-par.-33e-11}{%
\subsection*{(720 ILCS 5/33E-11) (from Ch. 38, par.
33E-11)}\label{ilcs-533e-11-from-ch.-38-par.-33e-11}}
\addcontentsline{toc}{subsection}{(720 ILCS 5/33E-11) (from Ch. 38, par.
33E-11)}

\hypertarget{sec.-33e-11.-a-every-bid-submitted-to-and-public-contract-executed-pursuant-to-such-bid-by-the-state-or-a-unit-of-local-government-shall-contain-a-certification-by-the-prime-contractor-that-the-prime-contractor-is-not-barred-from-contracting-with-any-unit-of-state-or-local-government-as-a-result-of-a-violation-of-either-section-33e-3-or-33e-4-of-this-article.-the-state-and-units-of-local-government-shall-provide-the-appropriate-forms-for-such-certification.}{%
\section*{Sec. 33E-11. (a) Every bid submitted to and public contract
executed pursuant to such bid by the State or a unit of local government
shall contain a certification by the prime contractor that the prime
contractor is not barred from contracting with any unit of State or
local government as a result of a violation of either Section 33E-3 or
33E-4 of this Article. The State and units of local government shall
provide the appropriate forms for such
certification.}\label{sec.-33e-11.-a-every-bid-submitted-to-and-public-contract-executed-pursuant-to-such-bid-by-the-state-or-a-unit-of-local-government-shall-contain-a-certification-by-the-prime-contractor-that-the-prime-contractor-is-not-barred-from-contracting-with-any-unit-of-state-or-local-government-as-a-result-of-a-violation-of-either-section-33e-3-or-33e-4-of-this-article.-the-state-and-units-of-local-government-shall-provide-the-appropriate-forms-for-such-certification.}}
\addcontentsline{toc}{section}{Sec. 33E-11. (a) Every bid submitted to
and public contract executed pursuant to such bid by the State or a unit
of local government shall contain a certification by the prime
contractor that the prime contractor is not barred from contracting with
any unit of State or local government as a result of a violation of
either Section 33E-3 or 33E-4 of this Article. The State and units of
local government shall provide the appropriate forms for such
certification.}

\markright{Sec. 33E-11. (a) Every bid submitted to and public contract
executed pursuant to such bid by the State or a unit of local government
shall contain a certification by the prime contractor that the prime
contractor is not barred from contracting with any unit of State or
local government as a result of a violation of either Section 33E-3 or
33E-4 of this Article. The State and units of local government shall
provide the appropriate forms for such certification.}

(b) A contractor who knowingly makes a false statement, material to the
certification, commits a Class 3 felony.

(Source: P.A. 97-1108, eff. 1-1-13.)

\hypertarget{ilcs-533e-12-from-ch.-38-par.-33e-12}{%
\subsection*{(720 ILCS 5/33E-12) (from Ch. 38, par.
33E-12)}\label{ilcs-533e-12-from-ch.-38-par.-33e-12}}
\addcontentsline{toc}{subsection}{(720 ILCS 5/33E-12) (from Ch. 38, par.
33E-12)}

\hypertarget{sec.-33e-12.-it-shall-not-constitute-a-violation-of-any-provisions-of-this-article-for-any-person-who-is-an-official-of-or-employed-by-a-unit-of-state-or-local-government-to-1-disclose-the-name-of-any-person-who-has-submitted-a-bid-in-response-to-or-requested-plans-or-specifications-regarding-an-invitation-to-bid-or-who-has-been-awarded-a-public-contract-to-any-person-or-2-to-convey-information-concerning-acceptable-alternatives-or-substitute-to-plans-or-specifications-if-such-information-is-also-made-generally-available-to-the-public-and-mailed-to-any-person-who-has-submitted-a-bid-in-response-to-or-requested-plans-or-specifications-regarding-an-invitation-to-bid-on-a-public-contract-or-3-to-negotiate-with-the-lowest-responsible-bidder-a-reduction-in-only-the-price-term-of-the-bid.}{%
\section*{Sec. 33E-12. It shall not constitute a violation of any
provisions of this Article for any person who is an official of or
employed by a unit of State or local government to (1) disclose the name
of any person who has submitted a bid in response to or requested plans
or specifications regarding an invitation to bid or who has been awarded
a public contract to any person or, (2) to convey information concerning
acceptable alternatives or substitute to plans or specifications if such
information is also made generally available to the public and mailed to
any person who has submitted a bid in response to or requested plans or
specifications regarding an invitation to bid on a public contract or,
(3) to negotiate with the lowest responsible bidder a reduction in only
the price term of the
bid.}\label{sec.-33e-12.-it-shall-not-constitute-a-violation-of-any-provisions-of-this-article-for-any-person-who-is-an-official-of-or-employed-by-a-unit-of-state-or-local-government-to-1-disclose-the-name-of-any-person-who-has-submitted-a-bid-in-response-to-or-requested-plans-or-specifications-regarding-an-invitation-to-bid-or-who-has-been-awarded-a-public-contract-to-any-person-or-2-to-convey-information-concerning-acceptable-alternatives-or-substitute-to-plans-or-specifications-if-such-information-is-also-made-generally-available-to-the-public-and-mailed-to-any-person-who-has-submitted-a-bid-in-response-to-or-requested-plans-or-specifications-regarding-an-invitation-to-bid-on-a-public-contract-or-3-to-negotiate-with-the-lowest-responsible-bidder-a-reduction-in-only-the-price-term-of-the-bid.}}
\addcontentsline{toc}{section}{Sec. 33E-12. It shall not constitute a
violation of any provisions of this Article for any person who is an
official of or employed by a unit of State or local government to (1)
disclose the name of any person who has submitted a bid in response to
or requested plans or specifications regarding an invitation to bid or
who has been awarded a public contract to any person or, (2) to convey
information concerning acceptable alternatives or substitute to plans or
specifications if such information is also made generally available to
the public and mailed to any person who has submitted a bid in response
to or requested plans or specifications regarding an invitation to bid
on a public contract or, (3) to negotiate with the lowest responsible
bidder a reduction in only the price term of the bid.}

\markright{Sec. 33E-12. It shall not constitute a violation of any
provisions of this Article for any person who is an official of or
employed by a unit of State or local government to (1) disclose the name
of any person who has submitted a bid in response to or requested plans
or specifications regarding an invitation to bid or who has been awarded
a public contract to any person or, (2) to convey information concerning
acceptable alternatives or substitute to plans or specifications if such
information is also made generally available to the public and mailed to
any person who has submitted a bid in response to or requested plans or
specifications regarding an invitation to bid on a public contract or,
(3) to negotiate with the lowest responsible bidder a reduction in only
the price term of the bid.}

(Source: P.A. 86-150.)

\hypertarget{ilcs-533e-13-from-ch.-38-par.-33e-13}{%
\subsection*{(720 ILCS 5/33E-13) (from Ch. 38, par.
33E-13)}\label{ilcs-533e-13-from-ch.-38-par.-33e-13}}
\addcontentsline{toc}{subsection}{(720 ILCS 5/33E-13) (from Ch. 38, par.
33E-13)}

\hypertarget{sec.-33e-13.-contract-negotiations-under-the-local-government-professional-services-selection-act-shall-not-be-subject-to-the-provisions-of-this-article.}{%
\section*{Sec. 33E-13. Contract negotiations under the Local Government
Professional Services Selection Act shall not be subject to the
provisions of this
Article.}\label{sec.-33e-13.-contract-negotiations-under-the-local-government-professional-services-selection-act-shall-not-be-subject-to-the-provisions-of-this-article.}}
\addcontentsline{toc}{section}{Sec. 33E-13. Contract negotiations under
the Local Government Professional Services Selection Act shall not be
subject to the provisions of this Article.}

\markright{Sec. 33E-13. Contract negotiations under the Local Government
Professional Services Selection Act shall not be subject to the
provisions of this Article.}

(Source: P.A. 87-855.)

\hypertarget{ilcs-533e-14}{%
\subsection*{(720 ILCS 5/33E-14)}\label{ilcs-533e-14}}
\addcontentsline{toc}{subsection}{(720 ILCS 5/33E-14)}

\hypertarget{sec.-33e-14.-false-statements-on-vendor-applications.}{%
\section*{Sec. 33E-14. False statements on vendor
applications.}\label{sec.-33e-14.-false-statements-on-vendor-applications.}}
\addcontentsline{toc}{section}{Sec. 33E-14. False statements on vendor
applications.}

\markright{Sec. 33E-14. False statements on vendor applications.}

(a) A person commits false statements on vendor applications when he or
she knowingly makes any false statement or report with the intent to
influence in any way the action of any unit of local government or
school district in considering a vendor application.

(b) Sentence. False statements on vendor applications is a Class 3
felony.

(Source: P.A. 99-78, eff. 7-20-15.)

\hypertarget{ilcs-533e-15}{%
\subsection*{(720 ILCS 5/33E-15)}\label{ilcs-533e-15}}
\addcontentsline{toc}{subsection}{(720 ILCS 5/33E-15)}

\hypertarget{sec.-33e-15.-false-entries.}{%
\section*{Sec. 33E-15. False
entries.}\label{sec.-33e-15.-false-entries.}}
\addcontentsline{toc}{section}{Sec. 33E-15. False entries.}

\markright{Sec. 33E-15. False entries.}

(a) An officer, agent, or employee of, or anyone who is affiliated in
any capacity with any unit of local government or school district
commits false entries when he or she makes a false entry in any book,
report, or statement of any unit of local government or school district
with the intent to defraud the unit of local government or school
district.

(b) Sentence. False entries is a Class 3 felony.

(Source: P.A. 97-1108, eff. 1-1-13.)

\hypertarget{ilcs-533e-16}{%
\subsection*{(720 ILCS 5/33E-16)}\label{ilcs-533e-16}}
\addcontentsline{toc}{subsection}{(720 ILCS 5/33E-16)}

\hypertarget{sec.-33e-16.-misapplication-of-funds.}{%
\section*{Sec. 33E-16. Misapplication of
funds.}\label{sec.-33e-16.-misapplication-of-funds.}}
\addcontentsline{toc}{section}{Sec. 33E-16. Misapplication of funds.}

\markright{Sec. 33E-16. Misapplication of funds.}

(a) An officer, director, agent, or employee of, or affiliated in any
capacity with any unit of local government or school district commits
misapplication of funds when he or she knowingly misapplies any of the
moneys, funds, or credits of the unit of local government or school
district.

(b) Sentence. Misapplication of funds is a Class 3 felony.

(Source: P.A. 97-1108, eff. 1-1-13.)

\hypertarget{ilcs-533e-17}{%
\subsection*{(720 ILCS 5/33E-17)}\label{ilcs-533e-17}}
\addcontentsline{toc}{subsection}{(720 ILCS 5/33E-17)}

\hypertarget{sec.-33e-17.-unlawful-participation.}{%
\section*{Sec. 33E-17. Unlawful
participation.}\label{sec.-33e-17.-unlawful-participation.}}
\addcontentsline{toc}{section}{Sec. 33E-17. Unlawful participation.}

\markright{Sec. 33E-17. Unlawful participation.}

Whoever, being an officer, director, agent, or employee of, or
affiliated in any capacity with any unit of local government or school
district participates, shares in, or receiving directly or indirectly
any money, profit, property, or benefit through any contract with the
unit of local government or school district, with the intent to defraud
the unit of local government or school district is guilty of a Class 3
felony.

(Source: P.A. 90-800, eff. 1-1-99.)

\hypertarget{ilcs-533e-18}{%
\subsection*{(720 ILCS 5/33E-18)}\label{ilcs-533e-18}}
\addcontentsline{toc}{subsection}{(720 ILCS 5/33E-18)}

\hypertarget{sec.-33e-18.-unlawful-stringing-of-bids.}{%
\section*{Sec. 33E-18. Unlawful stringing of
bids.}\label{sec.-33e-18.-unlawful-stringing-of-bids.}}
\addcontentsline{toc}{section}{Sec. 33E-18. Unlawful stringing of bids.}

\markright{Sec. 33E-18. Unlawful stringing of bids.}

(a) A person commits unlawful stringing of bids when he or she, with the
intent to evade the bidding requirements of any unit of local government
or school district, knowingly strings or assists in stringing or
attempts to string any contract or job order with the unit of local
government or school district.

(b) Sentence. Unlawful stringing of bids is a Class 4 felony.

(Source: P.A. 97-1108, eff. 1-1-13; 98-756, eff. 7-16-14.)

\bookmarksetup{startatroot}

\hypertarget{article-33f.-unlawful-use-of-body-armor}{%
\chapter*{Article 33f. Unlawful Use Of Body
Armor}\label{article-33f.-unlawful-use-of-body-armor}}
\addcontentsline{toc}{chapter}{Article 33f. Unlawful Use Of Body Armor}

\markboth{Article 33f. Unlawful Use Of Body Armor}{Article 33f. Unlawful
Use Of Body Armor}

\hypertarget{ilcs-533f-1-from-ch.-38-par.-33f-1}{%
\subsection*{(720 ILCS 5/33F-1) (from Ch. 38, par.
33F-1)}\label{ilcs-533f-1-from-ch.-38-par.-33f-1}}
\addcontentsline{toc}{subsection}{(720 ILCS 5/33F-1) (from Ch. 38, par.
33F-1)}

\hypertarget{sec.-33f-1.-definitions.}{%
\section*{Sec. 33F-1. Definitions.}\label{sec.-33f-1.-definitions.}}
\addcontentsline{toc}{section}{Sec. 33F-1. Definitions.}

\markright{Sec. 33F-1. Definitions.}

For purposes of this Article:

(a) ``Body Armor'' means any one of the following:

(1) A military style flak or tactical assault vest which is made of
Kevlar or any other similar material or metal, fiberglass, plastic, and
nylon plates and designed to be worn over one's clothing for the
intended purpose of stopping not only missile fragmentation from mines,
grenades, mortar shells and artillery fire but also fire from rifles,
machine guns, and small arms.

(2) Soft body armor which is made of Kevlar or any other similar
material or metal or any other type of insert and which is lightweight
and pliable and which can be easily concealed under a shirt.

(3) A military style recon/surveillance vest which is made of Kevlar or
any other similar material and which is lightweight and designed to be
worn over one's clothing.

(4) Protective casual clothing which is made of

Kevlar or any other similar material and which was originally intended
to be used by undercover law enforcement officers or dignitaries and is
designed to look like jackets, coats, raincoats, quilted or three piece
suit vests.

(b) ``Dangerous weapon'' means a Category I, Category II, or Category
III weapon as defined in Section 33A-1 of this Code.

(Source: P.A. 91-696, eff. 4-13-00.)

\hypertarget{ilcs-533f-2-from-ch.-38-par.-33f-2}{%
\subsection*{(720 ILCS 5/33F-2) (from Ch. 38, par.
33F-2)}\label{ilcs-533f-2-from-ch.-38-par.-33f-2}}
\addcontentsline{toc}{subsection}{(720 ILCS 5/33F-2) (from Ch. 38, par.
33F-2)}

\hypertarget{sec.-33f-2.-unlawful-use-of-body-armor.}{%
\section*{Sec. 33F-2. Unlawful use of body
armor.}\label{sec.-33f-2.-unlawful-use-of-body-armor.}}
\addcontentsline{toc}{section}{Sec. 33F-2. Unlawful use of body armor.}

\markright{Sec. 33F-2. Unlawful use of body armor.}

A person commits the offense of unlawful use of body armor when he
knowingly wears body armor and is in possession of a dangerous weapon,
other than a firearm, in the commission or attempted commission of any
offense.

(Source: P.A. 93-906, eff. 8-11-04.)

\hypertarget{ilcs-533f-3-from-ch.-38-par.-33f-3}{%
\subsection*{(720 ILCS 5/33F-3) (from Ch. 38, par.
33F-3)}\label{ilcs-533f-3-from-ch.-38-par.-33f-3}}
\addcontentsline{toc}{subsection}{(720 ILCS 5/33F-3) (from Ch. 38, par.
33F-3)}

\hypertarget{sec.-33f-3.-sentence.}{%
\section*{Sec. 33F-3. Sentence.}\label{sec.-33f-3.-sentence.}}
\addcontentsline{toc}{section}{Sec. 33F-3. Sentence.}

\markright{Sec. 33F-3. Sentence.}

A person convicted of unlawful use of body armor for a first offense
shall be guilty of a Class A misdemeanor and for a second or subsequent
offense shall be guilty of a Class 4 felony.

(Source: P.A. 87-521.)

\bookmarksetup{startatroot}

\hypertarget{article-33g.-illinois-street-gang}{%
\chapter*{Article 33g. Illinois Street
Gang}\label{article-33g.-illinois-street-gang}}
\addcontentsline{toc}{chapter}{Article 33g. Illinois Street Gang}

\markboth{Article 33g. Illinois Street Gang}{Article 33g. Illinois
Street Gang}

AND RACKETEER INFLUENCED AND CORRUPT ORGANIZATIONS LAW

(Article scheduled to be repealed on June 11, 2023)

(Source: P.A. 97-686, eff. 6-11-12 .)

\hypertarget{ilcs-533g-1}{%
\subsection*{(720 ILCS 5/33G-1)}\label{ilcs-533g-1}}
\addcontentsline{toc}{subsection}{(720 ILCS 5/33G-1)}

(Section scheduled to be repealed on June 11, 2023)

\hypertarget{sec.-33g-1.-short-title.}{%
\section*{Sec. 33G-1. Short title.}\label{sec.-33g-1.-short-title.}}
\addcontentsline{toc}{section}{Sec. 33G-1. Short title.}

\markright{Sec. 33G-1. Short title.}

This Article may be cited as the Illinois Street Gang and Racketeer
Influenced and Corrupt Organizations Law (or ``RICO'').

(Source: P.A. 97-686, eff. 6-11-12 .)

\hypertarget{ilcs-533g-2}{%
\subsection*{(720 ILCS 5/33G-2)}\label{ilcs-533g-2}}
\addcontentsline{toc}{subsection}{(720 ILCS 5/33G-2)}

(Section scheduled to be repealed on June 11, 2023)

\hypertarget{sec.-33g-2.-legislative-declaration.}{%
\section*{Sec. 33G-2. Legislative
declaration.}\label{sec.-33g-2.-legislative-declaration.}}
\addcontentsline{toc}{section}{Sec. 33G-2. Legislative declaration.}

\markright{Sec. 33G-2. Legislative declaration.}

The substantial harm inflicted on the people and economy of this State
by pervasive violent street gangs and other forms of enterprise
criminality, is legitimately a matter of grave concern to the people of
this State who have a basic right to be protected from that criminal
activity and to be given adequate remedies to redress its harms. Whereas
the current laws of this State provide inadequate remedies, procedures
and punishments, the Illinois General Assembly hereby gives the
supplemental remedies of the Illinois Street Gang and Racketeer
Influenced and Corrupt Organizations Law full force and effect under law
for the common good of this State and its people.

(Source: P.A. 97-686, eff. 6-11-12 .)

\hypertarget{ilcs-533g-3}{%
\subsection*{(720 ILCS 5/33G-3)}\label{ilcs-533g-3}}
\addcontentsline{toc}{subsection}{(720 ILCS 5/33G-3)}

(Section scheduled to be repealed on June 11, 2023)

\hypertarget{sec.-33g-3.-definitions.}{%
\section*{Sec. 33G-3. Definitions.}\label{sec.-33g-3.-definitions.}}
\addcontentsline{toc}{section}{Sec. 33G-3. Definitions.}

\markright{Sec. 33G-3. Definitions.}

As used in this Article:

(a) ``Another state'' means any State of the United States (other than
the State of Illinois), or the District of Columbia, or the Commonwealth
of Puerto Rico, or any territory or possession of the United States, or
any political subdivision, or any department, agency, or instrumentality
thereof.

(b) ``Enterprise'' includes:

(1) any partnership, corporation, association, business or charitable
trust, or other legal entity; and

(2) any group of individuals or other legal entities, or any combination
thereof, associated in fact although not itself a legal entity. An
association in fact must be held together by a common purpose of
engaging in a course of conduct, and it may be associated together for
purposes that are both legal and illegal. An association in fact must:

(A) have an ongoing organization or structure, either formal or
informal;

(B) the various members of the group must function as a continuing unit,
even if the group changes membership by gaining or losing members over
time; and

(C) have an ascertainable structure distinct from that inherent in the
conduct of a pattern of predicate activity.

As used in this Article, ``enterprise'' includes licit and illicit
enterprises.

(c) ``Labor organization'' includes any organization, labor union, craft
union, or any voluntary unincorporated association designed to further
the cause of the rights of union labor that is constituted for the
purpose, in whole or in part, of collective bargaining or of dealing
with employers concerning grievances, terms or conditions of employment,
or apprenticeships or applications for apprenticeships, or of other
mutual aid or protection in connection with employment, including
apprenticeships or applications for apprenticeships.

(d) ``Operation or management'' means directing or carrying out the
enterprise's affairs and is limited to any person who knowingly serves
as a leader, organizer, operator, manager, director, supervisor,
financier, advisor, recruiter, supplier, or enforcer of an enterprise in
violation of this Article.

(e) ``Predicate activity'' means any act that is a Class 2 felony or
higher and constitutes a violation or violations of any of the following
provisions of the laws of the State of Illinois (as amended or revised
as of the date the activity occurred or, in the instance of a continuing
offense, the date that charges under this Article are filed in a
particular matter in the State of Illinois) or any act under the law of
another jurisdiction for an offense that could be charged as a Class 2
felony or higher in this State:

(1) under the Criminal Code of 1961 or the Criminal

Code of 2012: 8-1.2 (solicitation of murder for hire), 9-1 (first degree
murder), 9-3.3 (drug-induced homicide), 10-1 (kidnapping), 10-2
(aggravated kidnapping), 10-3.1 (aggravated unlawful restraint), 10-4
(forcible detention), 10-5(b)(10) (child abduction), 10-9 (trafficking
in persons, involuntary servitude, and related offenses), 11-1.20
(criminal sexual assault), 11-1.30 (aggravated criminal sexual assault),
11-1.40 (predatory criminal sexual assault of a child), 11-1.60
(aggravated criminal sexual abuse), 11-6 (indecent solicitation of a
child), 11-6.5 (indecent solicitation of an adult), 11-14.3(a)(2)(A) and
(a)(2)(B) (promoting prostitution), 11-14.4 (promoting juvenile
prostitution), 11-18.1 (patronizing a minor engaged in prostitution;
patronizing a juvenile prostitute), 12-3.05 (aggravated battery), 12-6.4
(criminal street gang recruitment), 12-6.5 (compelling organization
membership of persons), 12-7.3 (stalking), 12-7.4 (aggravated stalking),
12-7.5 (cyberstalking), 12-11 or 19-6 (home invasion), 12-11.1 or 18-6
(vehicular invasion), 18-1 (robbery; aggravated robbery), 18-2 (armed
robbery), 18-3 (vehicular hijacking), 18-4 (aggravated vehicular
hijacking), 18-5 (aggravated robbery), 19-1 (burglary), 19-3
(residential burglary), 20-1 (arson; residential arson; place of worship
arson), 20-1.1 (aggravated arson), 20-1.2 (residential arson), 20-1.3
(place of worship arson), 24-1.2 (aggravated discharge of a firearm),
24-1.2-5 (aggravated discharge of a machine gun or silencer equipped
firearm), 24-1.8 (unlawful possession of a firearm by a street gang
member), 24-3.2 (unlawful discharge of firearm projectiles), 24-3.9
(aggravated possession of a stolen firearm), 24-3A (gunrunning), 26-5 or
48-1 (dog-fighting), 29D-14.9 (terrorism), 29D-15 (soliciting support
for terrorism), 29D-15.1 (causing a catastrophe), 29D-15.2 (possession
of a deadly substance), 29D-20 (making a terrorist threat), 29D-25
(falsely making a terrorist threat), 29D-29.9 (material support for
terrorism), 29D-35 (hindering prosecution of terrorism), 31A-1.2
(unauthorized contraband in a penal institution), or 33A-3 (armed
violence);

(2) under the Cannabis Control Act: Sections 5

(manufacture or delivery of cannabis), 5.1 (cannabis trafficking), or 8
(production or possession of cannabis plants), provided the offense
either involves more than 500 grams of any substance containing cannabis
or involves more than 50 cannabis sativa plants;

(3) under the Illinois Controlled Substances Act:

Sections 401 (manufacture or delivery of a controlled substance), 401.1
(controlled substance trafficking), 405 (calculated criminal drug
conspiracy), or 405.2 (street gang criminal drug conspiracy); or

(4) under the Methamphetamine Control and Community

Protection Act: Sections 15 (methamphetamine manufacturing), or 55
(methamphetamine delivery).

(f) ``Pattern of predicate activity'' means:

(1) at least 3 occurrences of predicate activity that are in some way
related to each other and that have continuity between them, and that
are separate acts. Acts are related to each other if they are not
isolated events, including if they have similar purposes, or results, or
participants, or victims, or are committed a similar way, or have other
similar distinguishing characteristics, or are part of the affairs of
the same enterprise. There is continuity between acts if they are
ongoing over a substantial period, or if they are part of the regular
way some entity does business or conducts its affairs; and

(2) which occurs after the effective date of this

Article, and the last of which falls within 3 years (excluding any
period of imprisonment) after the first occurrence of predicate
activity.

(g) ``Unlawful death'' includes the following offenses: under the Code
of 1961 or the Criminal Code of 2012: Sections 9-1 (first degree murder)
or 9-2 (second degree murder).

(Source: P.A. 97-686, eff. 6-11-12; 97-1150, eff. 1-25-13 .)

\hypertarget{ilcs-533g-4}{%
\subsection*{(720 ILCS 5/33G-4)}\label{ilcs-533g-4}}
\addcontentsline{toc}{subsection}{(720 ILCS 5/33G-4)}

(Section scheduled to be repealed on June 11, 2023)

\hypertarget{sec.-33g-4.-prohibited-activities.}{%
\section*{Sec. 33G-4. Prohibited
activities.}\label{sec.-33g-4.-prohibited-activities.}}
\addcontentsline{toc}{section}{Sec. 33G-4. Prohibited activities.}

\markright{Sec. 33G-4. Prohibited activities.}

(a) It is unlawful for any person, who intentionally participates in the
operation or management of an enterprise, directly or indirectly, to:

(1) knowingly do so, directly or indirectly, through a pattern of
predicate activity;

(2) knowingly cause another to violate this Article; or

(3) knowingly conspire to violate this Article.

Notwithstanding any other provision of law, in any prosecution for a
conspiracy to violate this Article, no person may be convicted of that
conspiracy unless an overt act in furtherance of the agreement is
alleged and proved to have been committed by him, her, or by a
coconspirator, but the commission of the overt act need not itself
constitute predicate activity underlying the specific violation of this
Article.

(b) It is unlawful for any person knowingly to acquire or maintain,
directly or indirectly, through a pattern of predicate activity any
interest in, or control of, to any degree, any enterprise, real
property, or personal property of any character, including money.

(c) Nothing in this Article shall be construed as to make unlawful any
activity which is arguably protected or prohibited by the National Labor
Relations Act, the Illinois Educational Labor Relations Act, the
Illinois Public Labor Relations Act, or the Railway Labor Act.

(d) The following organizations, and any officer or agent of those
organizations acting in his or her official capacity as an officer or
agent, may not be sued in civil actions under this Article:

(1) a labor organization; or

(2) any business defined in Division D, E, F, G, H, or I of the Standard
Industrial Classification as established by the Occupational Safety and
Health Administration, U.S. Department of Labor.

(e) Any person prosecuted under this Article may be convicted and
sentenced either:

(1) for the offense of conspiring to violate this

Article, and for any other particular offense or offenses that may be
one of the objects of a conspiracy to violate this Article; or

(2) for the offense of violating this Article, and for any other
particular offense or offenses that may constitute predicate activity
underlying a violation of this Article.

(f) The State's Attorney, or a person designated by law to act for him
or her and to perform his or her duties during his or her absence or
disability, may authorize a criminal prosecution under this Article.
Prior to any State's Attorney authorizing a criminal prosecution under
this Article, the State's Attorney shall adopt rules and procedures
governing the investigation and prosecution of any offense enumerated in
this Article. These rules and procedures shall set forth guidelines
which require that any potential prosecution under this Article be
subject to an internal approval process in which it is determined, in a
written prosecution memorandum prepared by the State's Attorney's
Office, that (1) a prosecution under this Article is necessary to ensure
that the indictment adequately reflects the nature and extent of the
criminal conduct involved in a way that prosecution only on the
underlying predicate activity would not, and (2) a prosecution under
this Article would provide the basis for an appropriate sentence under
all the circumstances of the case in a way that a prosecution only on
the underlying predicate activity would not. No State's Attorney, or
person designated by law to act for him or her and to perform his or her
duties during his or her absence or disability, may authorize a criminal
prosecution under this Article prior to reviewing the prepared written
prosecution memorandum. However, any internal memorandum shall remain
protected from disclosure under the attorney-client privilege, and this
provision does not create any enforceable right on behalf of any
defendant or party, nor does it subject the exercise of prosecutorial
discretion to judicial review.

(g) A labor organization and any officer or agent of that organization
acting in his or her capacity as an officer or agent of the labor
organization are exempt from prosecution under this Article.

(Source: P.A. 97-686, eff. 6-11-12; 98-463, eff. 8-16-13 .)

\hypertarget{ilcs-533g-5}{%
\subsection*{(720 ILCS 5/33G-5)}\label{ilcs-533g-5}}
\addcontentsline{toc}{subsection}{(720 ILCS 5/33G-5)}

(Section scheduled to be repealed on June 11, 2023)

\hypertarget{sec.-33g-5.-penalties.}{%
\section*{Sec. 33G-5. Penalties.}\label{sec.-33g-5.-penalties.}}
\addcontentsline{toc}{section}{Sec. 33G-5. Penalties.}

\markright{Sec. 33G-5. Penalties.}

Under this Article, notwithstanding any other provision of law:

(a) Any violation of subsection (a) of Section 33G-4 of this Article
shall be sentenced as a Class X felony with a term of imprisonment of
not less than 7 years and not more than 30 years, or the sentence
applicable to the underlying predicate activity, whichever is higher,
and the sentence imposed shall also include restitution, and/or a
criminal fine, jointly and severally, up to \$250,000 or twice the gross
amount of any intended proceeds of the violation, if any, whichever is
higher.

(b) Any violation of subsection (b) of Section 33G-4 of this Article
shall be sentenced as a Class X felony, and the sentence imposed shall
also include restitution, and/or a criminal fine, jointly and severally,
up to \$250,000 or twice the gross amount of any intended proceeds of
the violation, if any, whichever is higher.

(c) Wherever the unlawful death of any person or persons results as a
necessary or natural consequence of any violation of this Article, the
sentence imposed on the defendant shall include an enhanced term of
imprisonment of at least 25 years up to natural life, in addition to any
other penalty imposed by the court, provided:

(1) the death or deaths were reasonably foreseeable to the defendant to
be sentenced; and

(2) the death or deaths occurred when the defendant was otherwise
engaged in the violation of this Article as a whole.

(d) A sentence of probation, periodic imprisonment, conditional
discharge, impact incarceration or county impact incarceration, court
supervision, withheld adjudication, or any pretrial diversionary
sentence or suspended sentence, is not authorized for a violation of
this Article.

(Source: P.A. 97-686, eff. 6-11-12; 98-463, eff. 8-16-13 .)

\hypertarget{ilcs-533g-6}{%
\subsection*{(720 ILCS 5/33G-6)}\label{ilcs-533g-6}}
\addcontentsline{toc}{subsection}{(720 ILCS 5/33G-6)}

(Section scheduled to be repealed on June 11, 2023)

\hypertarget{sec.-33g-6.-remedial-proceedings-procedures-and-forfeiture.}{%
\section*{Sec. 33G-6. Remedial proceedings, procedures, and
forfeiture.}\label{sec.-33g-6.-remedial-proceedings-procedures-and-forfeiture.}}
\addcontentsline{toc}{section}{Sec. 33G-6. Remedial proceedings,
procedures, and forfeiture.}

\markright{Sec. 33G-6. Remedial proceedings, procedures, and
forfeiture.}

(a) Under this Article, the circuit court shall have jurisdiction to
prevent and restrain violations of this Article by issuing appropriate
orders, including:

(1) ordering any person to disgorge illicit proceeds obtained by a
violation of this Article or divest himself or herself of any interest,
direct or indirect, in any enterprise or real or personal property of
any character, including money, obtained, directly or indirectly, by a
violation of this Article;

(2) imposing reasonable restrictions on the future activities or
investments of any person or enterprise, including prohibiting any
person or enterprise from engaging in the same type of endeavor as the
person or enterprise engaged in, that violated this Article; or

(3) ordering dissolution or reorganization of any enterprise, making due
provision for the rights of innocent persons.

(b) Any violation of this Article is subject to the remedies,
procedures, and forfeiture as set forth in Article 29B of this Code.

(c) Property seized or forfeited under this Article is subject to
reporting under the Seizure and Forfeiture Reporting Act.

(Source: P.A. 100-512, eff. 7-1-18; 100-699, eff. 8-3-18; 101-81, eff.
7-12-19 .)

\hypertarget{ilcs-533g-7}{%
\subsection*{(720 ILCS 5/33G-7)}\label{ilcs-533g-7}}
\addcontentsline{toc}{subsection}{(720 ILCS 5/33G-7)}

(Section scheduled to be repealed on June 11, 2023)

\hypertarget{sec.-33g-7.-construction.}{%
\section*{Sec. 33G-7. Construction.}\label{sec.-33g-7.-construction.}}
\addcontentsline{toc}{section}{Sec. 33G-7. Construction.}

\markright{Sec. 33G-7. Construction.}

In interpreting the provisions of this Article, the court shall construe
them in light of the applicable model jury instructions set forth in the
Federal Criminal Jury Instructions for the Seventh Circuit (1999) for
Title IX of Public Law 91-452, 84 Stat. 922 (as amended in Title 18,
United States Code, Sections 1961 through 1968), except to the extent
that they are inconsistent with the plain language of this Article.

(Source: P.A. 97-686, eff. 6-11-12; 98-463, eff. 8-16-13 .)

\hypertarget{ilcs-533g-8}{%
\subsection*{(720 ILCS 5/33G-8)}\label{ilcs-533g-8}}
\addcontentsline{toc}{subsection}{(720 ILCS 5/33G-8)}

(Section scheduled to be repealed on June 11, 2023)

\hypertarget{sec.-33g-8.-limitations.}{%
\section*{Sec. 33G-8. Limitations.}\label{sec.-33g-8.-limitations.}}
\addcontentsline{toc}{section}{Sec. 33G-8. Limitations.}

\markright{Sec. 33G-8. Limitations.}

Under this Article, notwithstanding any other provision of law, but
otherwise subject to the periods of exclusion from limitation as
provided in Section 3-7 of this Code, the following limitations apply:

(a) Any action, proceeding, or prosecution brought under this Article
must commence within 5 years of one of the following dates, whichever is
latest:

(1) the date of the commission of the last occurrence of predicate
activity in a pattern of that activity, in the form of an act underlying
the alleged violation of this Article; or

(2) in the case of an action, proceeding, or prosecution, based upon a
conspiracy to violate this Article, the date that the last objective of
the alleged conspiracy was accomplished, defeated or abandoned
(whichever is later); or

(3) the date any minor victim of the violation attains the age of 18
years or the date any victim of the violation subject to a legal
disability thereafter gains legal capacity.

(b) Any action, proceeding, or prosecution brought under this Article
may be commenced at any time against all defendants if the conduct of
any defendant, or any part of the overall violation, resulted in the
unlawful death of any person or persons.

(Source: P.A. 97-686, eff. 6-11-12 .)

\hypertarget{ilcs-533g-9}{%
\subsection*{(720 ILCS 5/33G-9)}\label{ilcs-533g-9}}
\addcontentsline{toc}{subsection}{(720 ILCS 5/33G-9)}

(Section scheduled to be repealed on June 11, 2023)

\hypertarget{sec.-33g-9.-repeal.}{%
\section*{Sec. 33G-9. Repeal.}\label{sec.-33g-9.-repeal.}}
\addcontentsline{toc}{section}{Sec. 33G-9. Repeal.}

\markright{Sec. 33G-9. Repeal.}

This Article is repealed on June 11, 2023.

(Source: P.A. 102-918, eff. 5-27-22.)

\hypertarget{ilcs-5tit.-iv-heading}{%
\subsection*{(720 ILCS 5/Tit. IV heading)}\label{ilcs-5tit.-iv-heading}}
\addcontentsline{toc}{subsection}{(720 ILCS 5/Tit. IV heading)}

TITLE IV.

CONSTRUCTION, EFFECTIVE DATE AND REPEAL

\bookmarksetup{startatroot}

\hypertarget{article-34.-construction-and-effective-date}{%
\chapter*{Article 34. Construction And Effective
Date}\label{article-34.-construction-and-effective-date}}
\addcontentsline{toc}{chapter}{Article 34. Construction And Effective
Date}

\markboth{Article 34. Construction And Effective Date}{Article 34.
Construction And Effective Date}

\hypertarget{ilcs-534-1-from-ch.-38-par.-34-1}{%
\subsection*{(720 ILCS 5/34-1) (from Ch. 38, par.
34-1)}\label{ilcs-534-1-from-ch.-38-par.-34-1}}
\addcontentsline{toc}{subsection}{(720 ILCS 5/34-1) (from Ch. 38, par.
34-1)}

\hypertarget{sec.-34-1.-effect-of-headings.}{%
\section*{Sec. 34-1. Effect of
headings.}\label{sec.-34-1.-effect-of-headings.}}
\addcontentsline{toc}{section}{Sec. 34-1. Effect of headings.}

\markright{Sec. 34-1. Effect of headings.}

Section, Article, and Title headings contained herein shall not be
deemed to govern, limit, modify or in any manner affect the scope,
meaning, or intent of the provisions of any Section, Article, or Title
hereof.

(Source: P.A. 91-357, eff. 7-29-99.)

\hypertarget{ilcs-534-2-from-ch.-38-par.-34-2}{%
\subsection*{(720 ILCS 5/34-2) (from Ch. 38, par.
34-2)}\label{ilcs-534-2-from-ch.-38-par.-34-2}}
\addcontentsline{toc}{subsection}{(720 ILCS 5/34-2) (from Ch. 38, par.
34-2)}

\hypertarget{sec.-34-2.-partial-invalidity.}{%
\section*{Sec. 34-2. Partial
invalidity.}\label{sec.-34-2.-partial-invalidity.}}
\addcontentsline{toc}{section}{Sec. 34-2. Partial invalidity.}

\markright{Sec. 34-2. Partial invalidity.}

The invalidity of any provision of this Code shall not affect the
validity of the remainder of this Code.

(Source: Laws 1961, p.~1983.)

\hypertarget{ilcs-534-3-from-ch.-38-par.-34-3}{%
\subsection*{(720 ILCS 5/34-3) (from Ch. 38, par.
34-3)}\label{ilcs-534-3-from-ch.-38-par.-34-3}}
\addcontentsline{toc}{subsection}{(720 ILCS 5/34-3) (from Ch. 38, par.
34-3)}

\hypertarget{sec.-34-3.-savings-provisions-continuation-of-prior-statutes.}{%
\section*{Sec. 34-3. Savings provisions; continuation of prior
Statutes.}\label{sec.-34-3.-savings-provisions-continuation-of-prior-statutes.}}
\addcontentsline{toc}{section}{Sec. 34-3. Savings provisions;
continuation of prior Statutes.}

\markright{Sec. 34-3. Savings provisions; continuation of prior
Statutes.}

The provisions of Sections 2, 3 and 4 of ``An Act to revise the law in
relation to the construction of the Statutes'', approved March 5, 1874,
as amended, shall apply in all constructions of this Code.

(Source: Laws 1961, p.~1983.)

\hypertarget{ilcs-534-4-from-ch.-38-par.-34-4}{%
\subsection*{(720 ILCS 5/34-4) (from Ch. 38, par.
34-4)}\label{ilcs-534-4-from-ch.-38-par.-34-4}}
\addcontentsline{toc}{subsection}{(720 ILCS 5/34-4) (from Ch. 38, par.
34-4)}

\hypertarget{sec.-34-4.-effective-date.}{%
\section*{Sec. 34-4. Effective date.}\label{sec.-34-4.-effective-date.}}
\addcontentsline{toc}{section}{Sec. 34-4. Effective date.}

\markright{Sec. 34-4. Effective date.}

This Code shall take effect January 1, 1962.

(Source: Laws 1961, p.~1983.)

\hypertarget{ilcs-5tit.-v-heading}{%
\subsection*{(720 ILCS 5/Tit. V heading)}\label{ilcs-5tit.-v-heading}}
\addcontentsline{toc}{subsection}{(720 ILCS 5/Tit. V heading)}

TITLE V.

ADDED ARTICLES

\bookmarksetup{startatroot}

\hypertarget{article-36.-seizure-and-forfeiture}{%
\chapter*{Article 36. Seizure And
Forfeiture}\label{article-36.-seizure-and-forfeiture}}
\addcontentsline{toc}{chapter}{Article 36. Seizure And Forfeiture}

\markboth{Article 36. Seizure And Forfeiture}{Article 36. Seizure And
Forfeiture}

OF VESSELS, VEHICLES AND AIRCRAFT

\hypertarget{ilcs-536-1-from-ch.-38-par.-36-1}{%
\subsection*{(720 ILCS 5/36-1) (from Ch. 38, par.
36-1)}\label{ilcs-536-1-from-ch.-38-par.-36-1}}
\addcontentsline{toc}{subsection}{(720 ILCS 5/36-1) (from Ch. 38, par.
36-1)}

(Text of Section before amendment by P.A. 102-982

)

\hypertarget{sec.-36-1.-property-subject-to-forfeiture.}{%
\section*{Sec. 36-1. Property subject to
forfeiture.}\label{sec.-36-1.-property-subject-to-forfeiture.}}
\addcontentsline{toc}{section}{Sec. 36-1. Property subject to
forfeiture.}

\markright{Sec. 36-1. Property subject to forfeiture.}

(a) Any vessel or watercraft, vehicle, or aircraft is subject to
forfeiture under this Article if the vessel or watercraft, vehicle, or
aircraft is used with the knowledge and consent of the owner in the
commission of or in the attempt to commit as defined in Section 8-4 of
this Code:

(1) an offense prohibited by Section 9-1 (first degree murder), Section
9-3 (involuntary manslaughter and reckless homicide), Section 10-2
(aggravated kidnaping), Section 11-1.20 (criminal sexual assault),
Section 11-1.30 (aggravated criminal sexual assault), Section 11-1.40
(predatory criminal sexual assault of a child), subsection (a) of
Section 11-1.50 (criminal sexual abuse), subsection (a), (c), or (d) of
Section 11-1.60 (aggravated criminal sexual abuse), Section 11-6
(indecent solicitation of a child), Section 11-14.4 (promoting juvenile
prostitution except for keeping a place of juvenile prostitution),
Section 11-20.1 (child pornography), paragraph (a)(1), (a)(2), (a)(4),
(b)(1), (b)(2), (e)(1), (e)(2), (e)(3), (e)(4), (e)(5), (e)(6), or
(e)(7) of Section 12-3.05 (aggravated battery), Section 12-7.3
(stalking), Section 12-7.4 (aggravated stalking), Section 16-1 (theft if
the theft is of precious metal or of scrap metal), subdivision (f)(2) or
(f)(3) of Section 16-25 (retail theft), Section 18-2 (armed robbery),
Section 19-1 (burglary), Section 19-2 (possession of burglary tools),
Section 19-3 (residential burglary), Section 20-1 (arson; residential
arson; place of worship arson), Section 20-2 (possession of explosives
or explosive or incendiary devices), subdivision (a)(6) or (a)(7) of
Section 24-1 (unlawful use of weapons), Section 24-1.2 (aggravated
discharge of a firearm), Section 24-1.2-5 (aggravated discharge of a
machine gun or a firearm equipped with a device designed or used for
silencing the report of a firearm), Section 24-1.5 (reckless discharge
of a firearm), Section 28-1 (gambling), or Section 29D-15.2 (possession
of a deadly substance) of this Code;

(2) an offense prohibited by Section 21, 22, 23, 24 or 26 of the
Cigarette Tax Act if the vessel or watercraft, vehicle, or aircraft
contains more than 10 cartons of such cigarettes;

(3) an offense prohibited by Section 28, 29, or 30 of the Cigarette Use
Tax Act if the vessel or watercraft, vehicle, or aircraft contains more
than 10 cartons of such cigarettes;

(4) an offense prohibited by Section 44 of the

Environmental Protection Act;

(5) an offense prohibited by Section 11-204.1 of the

Illinois Vehicle Code (aggravated fleeing or attempting to elude a peace
officer);

(6) an offense prohibited by Section 11-501 of the

Illinois Vehicle Code (driving while under the influence of alcohol or
other drug or drugs, intoxicating compound or compounds or any
combination thereof) or a similar provision of a local ordinance, and:

(A) during a period in which his or her driving privileges are revoked
or suspended if the revocation or suspension was for:

(i) Section 11-501 (driving under the influence of alcohol or other drug
or drugs, intoxicating compound or compounds or any combination
thereof),

(ii) Section 11-501.1 (statutory summary suspension or revocation),

(iii) paragraph (b) of Section 11-401 (motor vehicle accidents involving
death or personal injuries), or

(iv) reckless homicide as defined in Section

9-3 of this Code;

(B) has been previously convicted of reckless homicide or a similar
provision of a law of another state relating to reckless homicide in
which the person was determined to have been under the influence of
alcohol, other drug or drugs, or intoxicating compound or compounds as
an element of the offense or the person has previously been convicted of
committing a violation of driving under the influence of alcohol or
other drug or drugs, intoxicating compound or compounds or any
combination thereof and was involved in a motor vehicle accident that
resulted in death, great bodily harm, or permanent disability or
disfigurement to another, when the violation was a proximate cause of
the death or injuries;

(C) the person committed a violation of driving under the influence of
alcohol or other drug or drugs, intoxicating compound or compounds or
any combination thereof under Section 11-501 of the Illinois Vehicle
Code or a similar provision for the third or subsequent time;

(D) he or she did not possess a valid driver's license or permit or a
valid restricted driving permit or a valid judicial driving permit or a
valid monitoring device driving permit; or

(E) he or she knew or should have known that the vehicle he or she was
driving was not covered by a liability insurance policy;

(7) an offense described in subsection (g) of Section

6-303 of the Illinois Vehicle Code;

(8) an offense described in subsection (e) of Section

6-101 of the Illinois Vehicle Code; or

(9)(A) operating a watercraft under the influence of alcohol, other drug
or drugs, intoxicating compound or compounds, or combination thereof
under Section 5-16 of the Boat Registration and Safety Act during a
period in which his or her privileges to operate a watercraft are
revoked or suspended and the revocation or suspension was for operating
a watercraft under the influence of alcohol, other drug or drugs,
intoxicating compound or compounds, or combination thereof; (B)
operating a watercraft under the influence of alcohol, other drug or
drugs, intoxicating compound or compounds, or combination thereof and
has been previously convicted of reckless homicide or a similar
provision of a law in another state relating to reckless homicide in
which the person was determined to have been under the influence of
alcohol, other drug or drugs, intoxicating compound or compounds, or
combination thereof as an element of the offense or the person has
previously been convicted of committing a violation of operating a
watercraft under the influence of alcohol, other drug or drugs,
intoxicating compound or compounds, or combination thereof and was
involved in an accident that resulted in death, great bodily harm, or
permanent disability or disfigurement to another, when the violation was
a proximate cause of the death or injuries; or (C) the person committed
a violation of operating a watercraft under the influence of alcohol,
other drug or drugs, intoxicating compound or compounds, or combination
thereof under Section 5-16 of the Boat Registration and Safety Act or a
similar provision for the third or subsequent time.

(b) In addition, any mobile or portable equipment used in the commission
of an act which is in violation of Section 7g of the Metropolitan Water
Reclamation District Act shall be subject to seizure and forfeiture
under the same procedures provided in this Article for the seizure and
forfeiture of vessels or watercraft, vehicles, and aircraft, and any
such equipment shall be deemed a vessel or watercraft, vehicle, or
aircraft for purposes of this Article.

(c) In addition, when a person discharges a firearm at another
individual from a vehicle with the knowledge and consent of the owner of
the vehicle and with the intent to cause death or great bodily harm to
that individual and as a result causes death or great bodily harm to
that individual, the vehicle shall be subject to seizure and forfeiture
under the same procedures provided in this Article for the seizure and
forfeiture of vehicles used in violations of clauses (1), (2), (3), or
(4) of subsection (a) of this Section.

(d) If the spouse of the owner of a vehicle seized for an offense
described in subsection (g) of Section 6-303 of the Illinois Vehicle
Code, a violation of subdivision (d)(1)(A), (d)(1)(D), (d)(1)(G),
(d)(1)(H), or (d)(1)(I) of Section 11-501 of the Illinois Vehicle Code,
or Section 9-3 of this Code makes a showing that the seized vehicle is
the only source of transportation and it is determined that the
financial hardship to the family as a result of the seizure outweighs
the benefit to the State from the seizure, the vehicle may be forfeited
to the spouse or family member and the title to the vehicle shall be
transferred to the spouse or family member who is properly licensed and
who requires the use of the vehicle for employment or family
transportation purposes. A written declaration of forfeiture of a
vehicle under this Section shall be sufficient cause for the title to be
transferred to the spouse or family member. The provisions of this
paragraph shall apply only to one forfeiture per vehicle. If the vehicle
is the subject of a subsequent forfeiture proceeding by virtue of a
subsequent conviction of either spouse or the family member, the spouse
or family member to whom the vehicle was forfeited under the first
forfeiture proceeding may not utilize the provisions of this paragraph
in another forfeiture proceeding. If the owner of the vehicle seized
owns more than one vehicle, the procedure set out in this paragraph may
be used for only one vehicle.

(e) In addition, property subject to forfeiture under Section 40 of the
Illinois Streetgang Terrorism Omnibus Prevention Act may be seized and
forfeited under this Article.

(Source: P.A. 99-78, eff. 7-20-15; 100-512, eff. 7-1-18 .)

(Text of Section after amendment by P.A. 102-982

)

\hypertarget{sec.-36-1.-property-subject-to-forfeiture.-1}{%
\section*{Sec. 36-1. Property subject to
forfeiture.}\label{sec.-36-1.-property-subject-to-forfeiture.-1}}
\addcontentsline{toc}{section}{Sec. 36-1. Property subject to
forfeiture.}

\markright{Sec. 36-1. Property subject to forfeiture.}

(a) Any vessel or watercraft, vehicle, or aircraft is subject to
forfeiture under this Article if the vessel or watercraft, vehicle, or
aircraft is used with the knowledge and consent of the owner in the
commission of or in the attempt to commit as defined in Section 8-4 of
this Code:

(1) an offense prohibited by Section 9-1 (first degree murder), Section
9-3 (involuntary manslaughter and reckless homicide), Section 10-2
(aggravated kidnaping), Section 11-1.20 (criminal sexual assault),
Section 11-1.30 (aggravated criminal sexual assault), Section 11-1.40
(predatory criminal sexual assault of a child), subsection (a) of
Section 11-1.50 (criminal sexual abuse), subsection (a), (c), or (d) of
Section 11-1.60 (aggravated criminal sexual abuse), Section 11-6
(indecent solicitation of a child), Section 11-14.4 (promoting juvenile
prostitution except for keeping a place of juvenile prostitution),
Section 11-20.1 (child pornography), paragraph (a)(1), (a)(2), (a)(4),
(b)(1), (b)(2), (e)(1), (e)(2), (e)(3), (e)(4), (e)(5), (e)(6), or
(e)(7) of Section 12-3.05 (aggravated battery), Section 12-7.3
(stalking), Section 12-7.4 (aggravated stalking), Section 16-1 (theft if
the theft is of precious metal or of scrap metal), subdivision (f)(2) or
(f)(3) of Section 16-25 (retail theft), Section 18-2 (armed robbery),
Section 19-1 (burglary), Section 19-2 (possession of burglary tools),
Section 19-3 (residential burglary), Section 20-1 (arson; residential
arson; place of worship arson), Section 20-2 (possession of explosives
or explosive or incendiary devices), subdivision (a)(6) or (a)(7) of
Section 24-1 (unlawful use of weapons), Section 24-1.2 (aggravated
discharge of a firearm), Section 24-1.2-5 (aggravated discharge of a
machine gun or a firearm equipped with a device designed or used for
silencing the report of a firearm), Section 24-1.5 (reckless discharge
of a firearm), Section 28-1 (gambling), or Section 29D-15.2 (possession
of a deadly substance) of this Code;

(2) an offense prohibited by Section 21, 22, 23, 24 or 26 of the
Cigarette Tax Act if the vessel or watercraft, vehicle, or aircraft
contains more than 10 cartons of such cigarettes;

(3) an offense prohibited by Section 28, 29, or 30 of the Cigarette Use
Tax Act if the vessel or watercraft, vehicle, or aircraft contains more
than 10 cartons of such cigarettes;

(4) an offense prohibited by Section 44 of the

Environmental Protection Act;

(5) an offense prohibited by Section 11-204.1 of the

Illinois Vehicle Code (aggravated fleeing or attempting to elude a peace
officer);

(6) an offense prohibited by Section 11-501 of the

Illinois Vehicle Code (driving while under the influence of alcohol or
other drug or drugs, intoxicating compound or compounds or any
combination thereof) or a similar provision of a local ordinance, and:

(A) during a period in which his or her driving privileges are revoked
or suspended if the revocation or suspension was for:

(i) Section 11-501 (driving under the influence of alcohol or other drug
or drugs, intoxicating compound or compounds or any combination
thereof),

(ii) Section 11-501.1 (statutory summary suspension or revocation),

(iii) paragraph (b) of Section 11-401 (motor vehicle crashes involving
death or personal injuries), or

(iv) reckless homicide as defined in Section

9-3 of this Code;

(B) has been previously convicted of reckless homicide or a similar
provision of a law of another state relating to reckless homicide in
which the person was determined to have been under the influence of
alcohol, other drug or drugs, or intoxicating compound or compounds as
an element of the offense or the person has previously been convicted of
committing a violation of driving under the influence of alcohol or
other drug or drugs, intoxicating compound or compounds or any
combination thereof and was involved in a motor vehicle crash that
resulted in death, great bodily harm, or permanent disability or
disfigurement to another, when the violation was a proximate cause of
the death or injuries;

(C) the person committed a violation of driving under the influence of
alcohol or other drug or drugs, intoxicating compound or compounds or
any combination thereof under Section 11-501 of the Illinois Vehicle
Code or a similar provision for the third or subsequent time;

(D) he or she did not possess a valid driver's license or permit or a
valid restricted driving permit or a valid judicial driving permit or a
valid monitoring device driving permit; or

(E) he or she knew or should have known that the vehicle he or she was
driving was not covered by a liability insurance policy;

(7) an offense described in subsection (g) of Section

6-303 of the Illinois Vehicle Code;

(8) an offense described in subsection (e) of Section

6-101 of the Illinois Vehicle Code; or

(9)(A) operating a watercraft under the influence of alcohol, other drug
or drugs, intoxicating compound or compounds, or combination thereof
under Section 5-16 of the Boat Registration and Safety Act during a
period in which his or her privileges to operate a watercraft are
revoked or suspended and the revocation or suspension was for operating
a watercraft under the influence of alcohol, other drug or drugs,
intoxicating compound or compounds, or combination thereof; (B)
operating a watercraft under the influence of alcohol, other drug or
drugs, intoxicating compound or compounds, or combination thereof and
has been previously convicted of reckless homicide or a similar
provision of a law in another state relating to reckless homicide in
which the person was determined to have been under the influence of
alcohol, other drug or drugs, intoxicating compound or compounds, or
combination thereof as an element of the offense or the person has
previously been convicted of committing a violation of operating a
watercraft under the influence of alcohol, other drug or drugs,
intoxicating compound or compounds, or combination thereof and was
involved in an accident that resulted in death, great bodily harm, or
permanent disability or disfigurement to another, when the violation was
a proximate cause of the death or injuries; or (C) the person committed
a violation of operating a watercraft under the influence of alcohol,
other drug or drugs, intoxicating compound or compounds, or combination
thereof under Section 5-16 of the Boat Registration and Safety Act or a
similar provision for the third or subsequent time.

(b) In addition, any mobile or portable equipment used in the commission
of an act which is in violation of Section 7g of the Metropolitan Water
Reclamation District Act shall be subject to seizure and forfeiture
under the same procedures provided in this Article for the seizure and
forfeiture of vessels or watercraft, vehicles, and aircraft, and any
such equipment shall be deemed a vessel or watercraft, vehicle, or
aircraft for purposes of this Article.

(c) In addition, when a person discharges a firearm at another
individual from a vehicle with the knowledge and consent of the owner of
the vehicle and with the intent to cause death or great bodily harm to
that individual and as a result causes death or great bodily harm to
that individual, the vehicle shall be subject to seizure and forfeiture
under the same procedures provided in this Article for the seizure and
forfeiture of vehicles used in violations of clauses (1), (2), (3), or
(4) of subsection (a) of this Section.

(d) If the spouse of the owner of a vehicle seized for an offense
described in subsection (g) of Section 6-303 of the Illinois Vehicle
Code, a violation of subdivision (d)(1)(A), (d)(1)(D), (d)(1)(G),
(d)(1)(H), or (d)(1)(I) of Section 11-501 of the Illinois Vehicle Code,
or Section 9-3 of this Code makes a showing that the seized vehicle is
the only source of transportation and it is determined that the
financial hardship to the family as a result of the seizure outweighs
the benefit to the State from the seizure, the vehicle may be forfeited
to the spouse or family member and the title to the vehicle shall be
transferred to the spouse or family member who is properly licensed and
who requires the use of the vehicle for employment or family
transportation purposes. A written declaration of forfeiture of a
vehicle under this Section shall be sufficient cause for the title to be
transferred to the spouse or family member. The provisions of this
paragraph shall apply only to one forfeiture per vehicle. If the vehicle
is the subject of a subsequent forfeiture proceeding by virtue of a
subsequent conviction of either spouse or the family member, the spouse
or family member to whom the vehicle was forfeited under the first
forfeiture proceeding may not utilize the provisions of this paragraph
in another forfeiture proceeding. If the owner of the vehicle seized
owns more than one vehicle, the procedure set out in this paragraph may
be used for only one vehicle.

(e) In addition, property subject to forfeiture under Section 40 of the
Illinois Streetgang Terrorism Omnibus Prevention Act may be seized and
forfeited under this Article.

(Source: P.A. 102-982, eff. 7-1-23.)

\hypertarget{ilcs-536-1.1}{%
\subsection*{(720 ILCS 5/36-1.1)}\label{ilcs-536-1.1}}
\addcontentsline{toc}{subsection}{(720 ILCS 5/36-1.1)}

\hypertarget{sec.-36-1.1.-seizure.}{%
\section*{Sec. 36-1.1. Seizure.}\label{sec.-36-1.1.-seizure.}}
\addcontentsline{toc}{section}{Sec. 36-1.1. Seizure.}

\markright{Sec. 36-1.1. Seizure.}

(a) Any property subject to forfeiture under this Article may be seized
and impounded by the Director of the Illinois State Police or any peace
officer upon process or seizure warrant issued by any court having
jurisdiction over the property.

(b) Any property subject to forfeiture under this Article may be seized
and impounded by the Director of the Illinois State Police or any peace
officer without process if there is probable cause to believe that the
property is subject to forfeiture under Section 36-1 of this Article and
the property is seized under circumstances in which a warrantless
seizure or arrest would be reasonable.

(c) If the seized property is a conveyance, an investigation shall be
made by the law enforcement agency as to any person whose right, title,
interest, or lien is of record in the office of the agency or official
in which title to or interest in the conveyance is required by law to be
recorded.

(d) After seizure under this Section, notice shall be given to all known
interest holders that forfeiture proceedings, including a preliminary
review, may be instituted and the proceedings may be instituted under
this Article.

(Source: P.A. 102-538, eff. 8-20-21.)

\hypertarget{ilcs-536-1.2}{%
\subsection*{(720 ILCS 5/36-1.2)}\label{ilcs-536-1.2}}
\addcontentsline{toc}{subsection}{(720 ILCS 5/36-1.2)}

\hypertarget{sec.-36-1.2.-receipt-for-seized-property.}{%
\section*{Sec. 36-1.2. Receipt for seized
property.}\label{sec.-36-1.2.-receipt-for-seized-property.}}
\addcontentsline{toc}{section}{Sec. 36-1.2. Receipt for seized
property.}

\markright{Sec. 36-1.2. Receipt for seized property.}

If a law enforcement officer seizes property for forfeiture under this
Article, the officer shall provide an itemized receipt to the person
possessing the property or, in the absence of a person to whom the
receipt could be given, shall leave the receipt in the place where the
property was found, if possible.

(Source: P.A. 100-512, eff. 7-1-18 .)

\hypertarget{ilcs-536-1.3}{%
\subsection*{(720 ILCS 5/36-1.3)}\label{ilcs-536-1.3}}
\addcontentsline{toc}{subsection}{(720 ILCS 5/36-1.3)}

\hypertarget{sec.-36-1.3.-safekeeping-of-seized-property-pending-disposition.}{%
\section*{Sec. 36-1.3. Safekeeping of seized property pending
disposition.}\label{sec.-36-1.3.-safekeeping-of-seized-property-pending-disposition.}}
\addcontentsline{toc}{section}{Sec. 36-1.3. Safekeeping of seized
property pending disposition.}

\markright{Sec. 36-1.3. Safekeeping of seized property pending
disposition.}

(a) Property seized under this Article is deemed to be in the custody of
the Director of the Illinois State Police, subject only to the order and
judgments of the circuit court having jurisdiction over the forfeiture
proceedings and the decisions of the State's Attorney under this
Article.

(b) If property is seized under this Article, the seizing agency shall
promptly conduct an inventory of the seized property and estimate the
property's value and shall forward a copy of the inventory of seized
property and the estimate of the property's value to the Director of the
Illinois State Police. Upon receiving notice of seizure, the Director of
the Illinois State Police may:

(1) place the property under seal;

(2) remove the property to a place designated by the

Director of the Illinois State Police;

(3) keep the property in the possession of the seizing agency;

(4) remove the property to a storage area for safekeeping;

(5) place the property under constructive seizure by posting notice of
pending forfeiture on it, by giving notice of pending forfeiture to its
owners and interest holders, or by filing notice of pending forfeiture
in any appropriate public record relating to the property; or

(6) provide for another agency or custodian, including an owner, secured
party, or lienholder, to take custody of the property upon the terms and
conditions set by the seizing agency.

(c) The seizing agency shall exercise ordinary care to protect the
subject of the forfeiture from negligent loss, damage, or destruction.

(d) Property seized or forfeited under this Article is subject to
reporting under the Seizure and Forfeiture Reporting Act.

(Source: P.A. 102-538, eff. 8-20-21.)

\hypertarget{ilcs-536-1.4}{%
\subsection*{(720 ILCS 5/36-1.4)}\label{ilcs-536-1.4}}
\addcontentsline{toc}{subsection}{(720 ILCS 5/36-1.4)}

\hypertarget{sec.-36-1.4.-notice-to-states-attorney.}{%
\section*{Sec. 36-1.4. Notice to State's
Attorney.}\label{sec.-36-1.4.-notice-to-states-attorney.}}
\addcontentsline{toc}{section}{Sec. 36-1.4. Notice to State's Attorney.}

\markright{Sec. 36-1.4. Notice to State's Attorney.}

The law enforcement agency seizing property for forfeiture under this
Article shall, as soon as practicable but not later than 28 days after
the seizure, notify the State's Attorney for the county in which an act
or omission giving rise to the seizure occurred or in which the property
was seized and the facts and circumstances giving rise to the seizure
and shall provide the State's Attorney with the inventory of the
property and its estimated value. The notice shall be by the delivery of
Illinois State Police Notice/Inventory of Seized Property (Form 4-64).
If the property seized for forfeiture is a vehicle, the law enforcement
agency seizing the property shall immediately notify the Secretary of
State that forfeiture proceedings are pending regarding the vehicle.

(Source: P.A. 100-512, eff. 7-1-18; 100-699, eff. 8-3-18; 100-1163, eff.
12-20-18.)

\hypertarget{ilcs-536-1.5}{%
\subsection*{(720 ILCS 5/36-1.5)}\label{ilcs-536-1.5}}
\addcontentsline{toc}{subsection}{(720 ILCS 5/36-1.5)}

\hypertarget{sec.-36-1.5.-preliminary-review.}{%
\section*{Sec. 36-1.5. Preliminary
review.}\label{sec.-36-1.5.-preliminary-review.}}
\addcontentsline{toc}{section}{Sec. 36-1.5. Preliminary review.}

\markright{Sec. 36-1.5. Preliminary review.}

(a) Within 14 days of the seizure, the State's Attorney of the county in
which the seizure occurred shall seek a preliminary determination from
the circuit court as to whether there is probable cause that the
property may be subject to forfeiture.

(b) The rules of evidence shall not apply to any proceeding conducted
under this Section.

(c) The court may conduct the review under subsection (a) of this
Section simultaneously with a proceeding under Section 109-1 of the Code
of Criminal Procedure of 1963 for a related criminal offense if a
prosecution is commenced by information or complaint.

(d) The court may accept a finding of probable cause at a preliminary
hearing following the filing of an information or complaint charging a
related criminal offense or following the return of indictment by a
grand jury charging the related offense as sufficient evidence of
probable cause as required under subsection (a) of this Section.

(e) Upon making a finding of probable cause as required under this
Section, the circuit court shall order the property subject to the
provisions of the applicable forfeiture Act held until the conclusion of
any forfeiture proceeding.

For seizures of conveyances, within 28 days of a finding of probable
cause under subsection (a) of this Section, the registered owner or
other claimant may file a motion in writing supported by sworn
affidavits claiming that denial of the use of the conveyance during the
pendency of the forfeiture proceedings creates a substantial hardship
and alleges facts showing that the hardship was not due to his or her
culpable negligence. The court shall consider the following factors in
determining whether a substantial hardship has been proven:

(1) the nature of the claimed hardship;

(2) the availability of public transportation or other available means
of transportation; and

(3) any available alternatives to alleviate the hardship other than the
return of the seized conveyance.

If the court determines that a substantial hardship has been proven, the
court shall then balance the nature of the hardship against the State's
interest in safeguarding the conveyance. If the court determines that
the hardship outweighs the State's interest in safeguarding the
conveyance, the court may temporarily release the conveyance to the
registered owner or the registered owner's authorized designee, or both,
until the conclusion of the forfeiture proceedings or for such shorter
period as ordered by the court provided that the person to whom the
conveyance is released provides proof of insurance and a valid driver's
license and all State and local registrations for operation of the
conveyance are current. The court shall place conditions on the
conveyance limiting its use to the stated hardship and providing
transportation for employment, religious purposes, medical needs, child
care, and restricting the conveyance's use to only those individuals
authorized to use the conveyance by the registered owner. The use of the
vehicle shall be further restricted to exclude all recreational and
entertainment purposes. The court may order additional restrictions it
deems reasonable and just on its own motion or on motion of the People.
The court shall revoke the order releasing the conveyance and order that
the conveyance be reseized by law enforcement if the conditions of
release are violated or if the conveyance is used in the commission of
any offense identified in subsection (a) of Section 6-205 of the
Illinois Vehicle Code.

If the court orders the release of the conveyance during the pendency of
the forfeiture proceedings, the court may order the registered owner or
his or her authorized designee to post a cash security with the clerk of
the court as ordered by the court. If cash security is ordered, the
court shall consider the following factors in determining the amount of
the cash security:

(A) the full market value of the conveyance;

(B) the nature of the hardship;

(C) the extent and length of the usage of the conveyance;

(D) the ability of the owner or designee to pay; and

(E) other conditions as the court deems necessary to safeguard the
conveyance.

If the conveyance is released, the court shall order that the registered
owner or his or her designee safeguard the conveyance, not remove the
conveyance from the jurisdiction, not conceal, destroy, or otherwise
dispose of the conveyance, not encumber the conveyance, and not diminish
the value of the conveyance in any way. The court shall also make a
determination of the full market value of the conveyance prior to it
being released based on a source or sources defined in 50 Ill. Adm. Code
919.80(c)(2)(A) or 919.80(c)(2)(B).

If the conveyance subject to forfeiture is released under this Section
and is subsequently forfeited, the person to whom the conveyance was
released shall return the conveyance to the law enforcement agency that
seized the conveyance within 7 days from the date of the declaration of
forfeiture or order of forfeiture. If the conveyance is not returned
within 7 days, the cash security shall be forfeited in the same manner
as the conveyance subject to forfeiture. If the cash security was less
than the full market value, a judgment shall be entered against the
parties to whom the conveyance was released and the registered owner,
jointly and severally, for the difference between the full market value
and the amount of the cash security. If the conveyance is returned in a
condition other than the condition in which it was released, the cash
security shall be returned to the surety who posted the security minus
the amount of the diminished value, and that amount shall be forfeited
in the same manner as the conveyance subject to forfeiture.
Additionally, the court may enter an order allowing any law enforcement
agency in the State of Illinois to seize the conveyance wherever it may
be found in the State to satisfy the judgment if the cash security was
less than the full market value of the conveyance.

(Source: P.A. 100-512, eff. 7-1-18; 100-699, eff. 8-3-18; 100-1163, eff.
12-20-18.)

\hypertarget{ilcs-536-1a-from-ch.-38-par.-36-1a}{%
\subsection*{(720 ILCS 5/36-1a) (from Ch. 38, par.
36-1a)}\label{ilcs-536-1a-from-ch.-38-par.-36-1a}}
\addcontentsline{toc}{subsection}{(720 ILCS 5/36-1a) (from Ch. 38, par.
36-1a)}

\hypertarget{sec.-36-1a.-repealed.}{%
\section*{Sec. 36-1a. (Repealed).}\label{sec.-36-1a.-repealed.}}
\addcontentsline{toc}{section}{Sec. 36-1a. (Repealed).}

\markright{Sec. 36-1a. (Repealed).}

(Source: P.A. 98-699, eff. 1-1-15. Repealed by P.A. 100-512, eff. 7-1-18
.)

\hypertarget{ilcs-536-2-from-ch.-38-par.-36-2}{%
\subsection*{(720 ILCS 5/36-2) (from Ch. 38, par.
36-2)}\label{ilcs-536-2-from-ch.-38-par.-36-2}}
\addcontentsline{toc}{subsection}{(720 ILCS 5/36-2) (from Ch. 38, par.
36-2)}

\hypertarget{sec.-36-2.-complaint-for-forfeiture.}{%
\section*{Sec. 36-2. Complaint for
forfeiture.}\label{sec.-36-2.-complaint-for-forfeiture.}}
\addcontentsline{toc}{section}{Sec. 36-2. Complaint for forfeiture.}

\markright{Sec. 36-2. Complaint for forfeiture.}

(a) If the State's Attorney of the county in which such seizure occurs
finds that the alleged violation of law giving rise to the seizure was
incurred without willful negligence or without any intention on the part
of the owner of the vessel or watercraft, vehicle, or aircraft or any
person whose right, title, or interest is of record as described in
Section 36-1 of this Article, to violate the law, or finds the existence
of such mitigating circumstances as to justify remission of the
forfeiture, he or she may cause the law enforcement agency having
custody of the property to return the property to the owner within a
reasonable time not to exceed 7 days. The State's Attorney shall
exercise his or her discretion under this subsection (a) prior to or
promptly after the preliminary review under Section 36-1.5.

(b) If, after review of the facts surrounding the seizure, the State's
Attorney is of the opinion that the seized property is subject to
forfeiture and the State's Attorney does not cause the forfeiture to be
remitted under subsection (a) of this Section, he or she shall bring an
action for forfeiture in the circuit court within whose jurisdiction the
seizure and confiscation has taken place by filing a verified complaint
for forfeiture in the circuit court within whose jurisdiction the
seizure occurred, or within whose jurisdiction an act or omission giving
rise to the seizure occurred, subject to Supreme Court Rule 187. The
complaint shall be filed as soon as practicable but not later than 28
days after the State's Attorney receives notice from the seizing agency
as provided under Section 36-1.4 of this Article. A complaint of
forfeiture shall include:

(1) a description of the property seized;

(2) the date and place of seizure of the property;

(3) the name and address of the law enforcement agency making the
seizure; and

(4) the specific statutory and factual grounds for the seizure.

The complaint shall be served upon each person whose right, title, or
interest is of record in the office of the Secretary of State, the
Secretary of Transportation, the Administrator of the Federal Aviation
Agency, or any other department of this State, or any other state of the
United States if the vessel or watercraft, vehicle, or aircraft is
required to be so registered, as the case may be, the person from whom
the property was seized, and all persons known or reasonably believed by
the State to claim an interest in the property, as provided in this
Article. The complaint shall be accompanied by the following written
notice:

``This is a civil court proceeding subject to the Code of Civil
Procedure. You received this Complaint of Forfeiture because the State's
Attorney's office has brought a legal action seeking forfeiture of your
seized property. This complaint starts the court process where the State
seeks to prove that your property should be forfeited and not returned
to you. This process is also your opportunity to try to prove to a judge
that you should get your property back. The complaint lists the date,
time, and location of your first court date. You must appear in court on
that day, or you may lose the case automatically. You must also file an
appearance and answer. If you are unable to pay the appearance fee, you
may qualify to have the fee waived. If there is a criminal case related
to the seizure of your property, your case may be set for trial after
the criminal case has been resolved. Before trial, the judge may allow
discovery, where the State can ask you to respond in writing to
questions and give them certain documents, and you can make similar
requests of the State. The trial is your opportunity to explain what
happened when your property was seized and why you should get the
property back.''

(c) (Blank).

(d) (Blank).

(e) (Blank).

(f) (Blank).

(g) (Blank).

(h) (Blank).

(Source: P.A. 99-78, eff. 7-20-15; 100-512, eff. 7-1-18; 100-699, eff.
8-3-18; 100-1163, eff. 12-20-18.)

\hypertarget{ilcs-536-2.1}{%
\subsection*{(720 ILCS 5/36-2.1)}\label{ilcs-536-2.1}}
\addcontentsline{toc}{subsection}{(720 ILCS 5/36-2.1)}

\hypertarget{sec.-36-2.1.-notice-to-owner-or-interest-holder.}{%
\section*{Sec. 36-2.1. Notice to owner or interest
holder.}\label{sec.-36-2.1.-notice-to-owner-or-interest-holder.}}
\addcontentsline{toc}{section}{Sec. 36-2.1. Notice to owner or interest
holder.}

\markright{Sec. 36-2.1. Notice to owner or interest holder.}

The first attempted service of notice shall be commenced within 28 days
of the receipt of the notice from the seizing agency by Form 4-64. If
the property seized is a conveyance, notice shall also be directed to
the address reflected in the office of the agency or official in which
title to or interest in the conveyance is required by law to be
recorded. A complaint for forfeiture shall be served upon the property
owner or interest holder in the following manner:

(1) If the owner's or interest holder's name and current address are
known, then by either:

(A) personal service; or

(B) mailing a copy of the notice by certified mail, return receipt
requested, and first class mail to that address.

(i) If notice is sent by certified mail and no signed return receipt is
received by the State's Attorney within 28 days of mailing, and no
communication from the owner or interest holder is received by the
State's Attorney documenting actual notice by said parties, the State's
Attorney shall, within a reasonable period of time, mail a second copy
of the notice by certified mail, return receipt requested, and first
class mail to that address.

(ii) If no signed return receipt is received by the State's Attorney
within 28 days of the second attempt at service by certified mail, and
no communication from the owner or interest holder is received by the
State's Attorney documenting actual notice by said parties, the State's
Attorney shall have 60 days to attempt to serve the notice by personal
service, which also includes substitute service by leaving a copy at the
usual place of abode with some person of the family or a person residing
there, of the age of 13 years or upwards. If, after 3 attempts at
service in this manner, no service of the notice is accomplished, then
the notice shall be posted in a conspicuous manner at this address and
service shall be made by the posting.

The attempts at service and the posting, if required, shall be
documented by the person attempting service and said documentation shall
be made part of a return of service returned to the State's Attorney.

The State's Attorney may utilize a Sheriff or

Deputy Sheriff, any peace officer, a private process server or
investigator, or any employee, agent, or investigator of the State's
Attorney's office to attempt service without seeking leave of court.

After the procedures are followed, service shall be effective on an
owner or interest holder on the date of receipt by the State's Attorney
of a return receipt, or on the date of receipt of a communication from
an owner or interest holder documenting actual notice, whichever is
first in time, or on the date of the last act performed by the State's
Attorney in attempting personal service under item (ii) of this
paragraph (1). If notice is to be shown by actual notice from
communication with a claimant, then the State's Attorney shall file an
affidavit providing details of the communication, which shall be
accepted as sufficient proof of service by the court.

For purposes of notice under this Section, if a person has been arrested
for the conduct giving rise to the forfeiture, the address provided to
the arresting agency at the time of arrest shall be deemed to be that
person's known address. Provided, however, if an owner or interest
holder's address changes prior to the effective date of the complaint
for forfeiture, the owner or interest holder shall promptly notify the
seizing agency of the change in address or, if the owner or interest
holder's address changes subsequent to the effective date of the notice
of pending forfeiture, the owner or interest holder shall promptly
notify the State's Attorney of the change in address; or if the property
seized is a conveyance, to the address reflected in the office of the
agency or official in which title to or interest in the conveyance is
required by law to be recorded.

(2) If the owner's or interest holder's address is not known, and is not
on record, then notice shall be served by publication for 3 successive
weeks in a newspaper of general circulation in the county in which the
seizure occurred.

(3) Notice to any business entity, corporation, limited liability
company, limited liability partnership, or partnership shall be
completed by a single mailing of a copy of the notice by certified mail,
return receipt requested, and first class mail to that address. This
notice is complete regardless of the return of a signed return receipt.

(4) Notice to a person whose address is not within the State shall be
completed by a single mailing of a copy of the notice by certified mail,
return receipt requested, and first class mail to that address. This
notice is complete regardless of the return of a signed return receipt.

(5) Notice to a person whose address is not within the United States
shall be completed by a single mailing of a copy of the notice by
certified mail, return receipt requested, and first class mail to that
address. This notice shall be complete regardless of the return of a
signed return receipt. If certified mail is not available in the foreign
country where the person has an address, then notice shall proceed by
publication under paragraph (2) of this Section.

(6) Notice to any person whom the State's Attorney reasonably should
know is incarcerated within the State shall also include mailing a copy
of the notice by certified mail, return receipt requested, and first
class mail to the address of the detention facility with the inmate's
name clearly marked on the envelope.

(Source: P.A. 100-512, eff. 7-1-18; 100-699, eff. 8-3-18; 100-1163, eff.
12-20-18.)

\hypertarget{ilcs-536-2.2}{%
\subsection*{(720 ILCS 5/36-2.2)}\label{ilcs-536-2.2}}
\addcontentsline{toc}{subsection}{(720 ILCS 5/36-2.2)}

\hypertarget{sec.-36-2.2.-replevin-prohibited-return-of-personal-property-inside-seized-conveyance.}{%
\section*{Sec. 36-2.2. Replevin prohibited; return of personal property
inside seized
conveyance.}\label{sec.-36-2.2.-replevin-prohibited-return-of-personal-property-inside-seized-conveyance.}}
\addcontentsline{toc}{section}{Sec. 36-2.2. Replevin prohibited; return
of personal property inside seized conveyance.}

\markright{Sec. 36-2.2. Replevin prohibited; return of personal property
inside seized conveyance.}

(a) Property seized under this Article shall not be subject to replevin,
but is deemed to be in the custody of the Director of the Illinois State
Police, subject only to the order and judgments of the circuit court
having jurisdiction over the forfeiture proceedings and the decisions of
the State's Attorney.

(b) A claimant or a party interested in personal property contained
within a seized conveyance may file a motion with the court in a
judicial forfeiture action for the return of any personal property
contained within a conveyance seized under this Article. The return of
personal property shall not be unreasonably withheld if the personal
property is not mechanically or electrically coupled to the conveyance,
needed for evidentiary purposes, or otherwise contraband. A law
enforcement agency that returns property under a court order under this
Section shall not be liable to any person who claims ownership to the
property if the property is returned to an improper party.

(Source: P.A. 102-538, eff. 8-20-21.)

\hypertarget{ilcs-536-2.5}{%
\subsection*{(720 ILCS 5/36-2.5)}\label{ilcs-536-2.5}}
\addcontentsline{toc}{subsection}{(720 ILCS 5/36-2.5)}

\hypertarget{sec.-36-2.5.-judicial-in-rem-procedures.}{%
\section*{Sec. 36-2.5. Judicial in rem
procedures.}\label{sec.-36-2.5.-judicial-in-rem-procedures.}}
\addcontentsline{toc}{section}{Sec. 36-2.5. Judicial in rem procedures.}

\markright{Sec. 36-2.5. Judicial in rem procedures.}

(a) The laws of evidence relating to civil actions shall apply to
judicial in rem proceedings under this Article.

(b) Only an owner of or interest holder in the property may file an
answer asserting a claim against the property in the action in rem. For
purposes of this Section, the owner or interest holder shall be referred
to as claimant. A person not named in the forfeiture complaint who
claims to have an interest in the property may petition to intervene as
a claimant under Section 2-408 of the Code of Civil Procedure.

(c) The answer shall be filed with the court within 45 days after
service of the civil in rem complaint.

(d) The trial shall be held within 60 days after filing of the answer
unless continued for good cause.

(e) In its case in chief, the State shall show by a preponderance of the
evidence that:

(1) the property is subject to forfeiture; and

(2) at least one of the following:

(i) the claimant knew or should have known that the conduct was likely
to occur; or

(ii) the claimant is not the true owner of the property that is subject
to forfeiture.

In any forfeiture case under this Article, a claimant may present
evidence to overcome evidence presented by the State that the property
is subject to forfeiture.

(f) Notwithstanding any other provision of this Section, the State's
burden of proof at the trial of the forfeiture action shall be by clear
and convincing evidence if:

(1) a finding of not guilty is entered as to all counts and all
defendants in a criminal proceeding relating to the conduct giving rise
to the forfeiture action; or

(2) the State receives an adverse finding at a preliminary hearing and
fails to secure an indictment in a criminal proceeding related to the
factual allegations of the forfeiture action.

(g) If the State does not meet its burden of proof, the court shall
order the interest in the property returned or conveyed to the claimant
and shall order all other property in which the State does meet its
burden of proof forfeited to the State. If the State does meet its
burden of proof, the court shall order all property forfeited to the
State.

(h) A defendant convicted in any criminal proceeding is precluded from
later denying the essential allegations of the criminal offense of which
the defendant was convicted in any proceeding under this Article
regardless of the pendency of an appeal from that conviction. However,
evidence of the pendency of an appeal is admissible.

(i) An acquittal or dismissal in a criminal proceeding shall not
preclude civil proceedings under this Act; however, for good cause
shown, on a motion by either party, the court may stay civil forfeiture
proceedings during the criminal trial for a related criminal indictment
or information alleging a violation of law authorizing forfeiture under
Section 36-1 of this Article.

(j) Title to all property declared forfeited under this Act vests in
this State on the commission of the conduct giving rise to forfeiture
together with the proceeds of the property after that time. Except as
otherwise provided in this Article, any property or proceeds
subsequently transferred to any person remain subject to forfeiture
unless a person to whom the property was transferred makes an
appropriate claim under or has the claim adjudicated at the judicial in
rem hearing.

(k) No property shall be forfeited under this Article from a person who,
without actual or constructive notice that the property was the subject
of forfeiture proceedings, obtained possession of the property as a bona
fide purchaser for value. A person who purports to transfer property
after receiving actual or constructive notice that the property is
subject to seizure or forfeiture is guilty of contempt of court and
shall be liable to the State for a penalty in the amount of the fair
market value of the property.

(l) A civil action under this Article shall be commenced within 5 years
after the last conduct giving rise to forfeiture became known or should
have become known or 5 years after the forfeitable property is
discovered, whichever is later, excluding any time during which either
the property or claimant is out of the State or in confinement or during
which criminal proceedings relating to the same conduct are in progress.

(m) If property is ordered forfeited under this Article from a claimant
who held title to the property in joint tenancy or tenancy in common
with another claimant, the court shall determine the amount of each
owner's interest in the property according to principles of property
law.

(Source: P.A. 100-512, eff. 7-1-18; 100-699, eff. 8-3-18; 100-1163, eff.
12-20-18.)

\hypertarget{ilcs-536-2.7}{%
\subsection*{(720 ILCS 5/36-2.7)}\label{ilcs-536-2.7}}
\addcontentsline{toc}{subsection}{(720 ILCS 5/36-2.7)}

\hypertarget{sec.-36-2.7.-innocent-owner-hearing.}{%
\section*{Sec. 36-2.7. Innocent owner
hearing.}\label{sec.-36-2.7.-innocent-owner-hearing.}}
\addcontentsline{toc}{section}{Sec. 36-2.7. Innocent owner hearing.}

\markright{Sec. 36-2.7. Innocent owner hearing.}

(a) After a complaint for forfeiture has been filed and all claimants
have appeared and answered, a claimant may file a motion with the court
for an innocent owner hearing prior to trial. This motion shall be made
and supported by sworn affidavit and shall assert the following along
with specific facts that support each assertion:

(1) that the claimant filing the motion is the true owner of the
conveyance as interpreted by case law; and

(2) that the claimant did not know or did not have reason to know the
conduct giving rise to the forfeiture was likely to occur.

The claimant's motion shall include specific facts that support these
assertions.

(b) Upon the filing, a hearing may only be conducted after the parties
have been given the opportunity to conduct limited discovery as to the
ownership and control of the property, the claimant's knowledge, or any
matter relevant to the issues raised or facts alleged in the claimant's
motion. Discovery shall be limited to the People's requests in these
areas but may proceed by any means allowed in the Code of Civil
Procedure.

(c) After discovery is complete and the court has allowed for sufficient
time to review and investigate the discovery responses, the court shall
conduct a hearing. At the hearing, the fact that the conveyance is
subject to forfeiture shall not be at issue. The court shall only hear
evidence relating to the issue of innocent ownership.

(d) At the hearing on the motion, the claimant shall bear the burden of
proving each of the assertions listed in subsection (a) of this Section
by a preponderance of the evidence. If a claimant meets the burden of
proof, the court shall grant the motion and order the conveyance
returned to the claimant. If the claimant fails to meet the burden of
proof, the court shall deny the motion and the forfeiture case shall
proceed according to the Code of Civil Procedure.

(Source: P.A. 100-512, eff. 7-1-18; 100-699, eff. 8-3-18.)

\hypertarget{ilcs-536-3-from-ch.-38-par.-36-3}{%
\subsection*{(720 ILCS 5/36-3) (from Ch. 38, par.
36-3)}\label{ilcs-536-3-from-ch.-38-par.-36-3}}
\addcontentsline{toc}{subsection}{(720 ILCS 5/36-3) (from Ch. 38, par.
36-3)}

\hypertarget{sec.-36-3.-exemptions-from-forfeiture.}{%
\section*{Sec. 36-3. Exemptions from
forfeiture.}\label{sec.-36-3.-exemptions-from-forfeiture.}}
\addcontentsline{toc}{section}{Sec. 36-3. Exemptions from forfeiture.}

\markright{Sec. 36-3. Exemptions from forfeiture.}

(a) No vessel or watercraft, vehicle, or aircraft used by any person as
a common carrier in the transaction of business as such common carrier
may be forfeited under the provisions of Section 36-2 unless the State
proves by a preponderance of the evidence that (1) in the case of a
railway car or engine, the owner, or (2) in the case of any other such
vessel or watercraft, vehicle or aircraft, the owner or the master of
such vessel or watercraft or the owner or conductor, driver, pilot, or
other person in charge of such vehicle or aircraft was at the time of
the alleged illegal act a consenting party or privy thereto.

(b) No vessel or watercraft, vehicle, or aircraft shall be forfeited
under the provisions of Section 36-2 of this Article by reason of any
act or omission committed or omitted by any person other than such owner
while such vessel or watercraft, vehicle, or aircraft was unlawfully in
the possession of a person who acquired possession thereof in violation
of the criminal laws of the United States, or of any state.

(Source: P.A. 100-512, eff. 7-1-18 .)

\hypertarget{ilcs-536-3.1}{%
\subsection*{(720 ILCS 5/36-3.1)}\label{ilcs-536-3.1}}
\addcontentsline{toc}{subsection}{(720 ILCS 5/36-3.1)}

\hypertarget{sec.-36-3.1.-proportionality.}{%
\section*{Sec. 36-3.1.
Proportionality.}\label{sec.-36-3.1.-proportionality.}}
\addcontentsline{toc}{section}{Sec. 36-3.1. Proportionality.}

\markright{Sec. 36-3.1. Proportionality.}

Property forfeited under this Article shall be subject to an 8th
Amendment to the United States Constitution disproportionate penalties
analysis, and the property forfeiture may be denied in whole or in part
if the court finds that the forfeiture would constitute an excessive
fine in violation of the 8th Amendment to the United States
Constitution, as interpreted by case law.

(Source: P.A. 100-512, eff. 7-1-18 .)

\hypertarget{ilcs-536-4-from-ch.-38-par.-36-4}{%
\subsection*{(720 ILCS 5/36-4) (from Ch. 38, par.
36-4)}\label{ilcs-536-4-from-ch.-38-par.-36-4}}
\addcontentsline{toc}{subsection}{(720 ILCS 5/36-4) (from Ch. 38, par.
36-4)}

\hypertarget{sec.-36-4.-remission-by-attorney-general.}{%
\section*{Sec. 36-4. Remission by Attorney
General.}\label{sec.-36-4.-remission-by-attorney-general.}}
\addcontentsline{toc}{section}{Sec. 36-4. Remission by Attorney
General.}

\markright{Sec. 36-4. Remission by Attorney General.}

Whenever any owner of, or other person interested in, a vessel or
watercraft, vehicle, or aircraft seized under the provisions of this Act
files with the Attorney General before the sale or destruction of such
vessel or watercraft, vehicle, or aircraft, a petition for the remission
of such forfeiture the Attorney General if he finds that such forfeiture
was incurred without willful negligence or without any intention on the
part of the owner or any person whose right, title or interest is of
record as described in Section 36-1, to violate the law, or finds the
existence of such mitigating circumstances as to justify the remission
of forfeiture, may cause the same to be remitted upon such terms and
conditions as he deems reasonable and just, or order discontinuance of
any forfeiture proceeding relating thereto.

(Source: P.A. 98-699, eff. 1-1-15 .)

\hypertarget{ilcs-536-5}{%
\subsection*{(720 ILCS 5/36-5)}\label{ilcs-536-5}}
\addcontentsline{toc}{subsection}{(720 ILCS 5/36-5)}

\hypertarget{sec.-36-5.-repealed.}{%
\section*{Sec. 36-5. (Repealed).}\label{sec.-36-5.-repealed.}}
\addcontentsline{toc}{section}{Sec. 36-5. (Repealed).}

\markright{Sec. 36-5. (Repealed).}

(Source: P.A. 98-1020, eff. 8-22-14. Repealed by P.A. 100-512, eff.
7-1-18 .)

\hypertarget{ilcs-536-6}{%
\subsection*{(720 ILCS 5/36-6)}\label{ilcs-536-6}}
\addcontentsline{toc}{subsection}{(720 ILCS 5/36-6)}

\hypertarget{sec.-36-6.-return-of-property-damages-and-costs.}{%
\section*{Sec. 36-6. Return of property, damages and
costs.}\label{sec.-36-6.-return-of-property-damages-and-costs.}}
\addcontentsline{toc}{section}{Sec. 36-6. Return of property, damages
and costs.}

\markright{Sec. 36-6. Return of property, damages and costs.}

(a) The law enforcement agency that holds custody of property seized for
forfeiture shall return to the claimant, within a reasonable period of
time not to exceed 7 days unless the order is stayed by the trial court
or a reviewing court pending an appeal, motion to reconsider, or other
reason after the court orders the property to be returned or conveyed to
the claimant:

(1) property ordered by the court to be conveyed or returned to the
claimant; and

(2) property ordered by the court to be conveyed or returned to the
claimant under subsection (d) of Section 36-3.1 of this Article.

(b) The law enforcement agency that holds custody of property seized
under this Article is responsible for any damages, storage fees, and
related costs applicable to property returned to a claimant under this
Article. The claimant shall not be subject to any charges by the State
for storage of the property or expenses incurred in the preservation of
the property. Charges for the towing of a conveyance shall be borne by
the claimant unless the conveyance was towed for the sole reason of
seizure for forfeiture. This subsection does not prohibit the imposition
of any fees or costs by a home rule unit of local government related to
the impoundment of a conveyance under an ordinance enacted by the unit
of government.

(Source: P.A. 100-512, eff. 7-1-18 .)

\hypertarget{ilcs-536-7}{%
\subsection*{(720 ILCS 5/36-7)}\label{ilcs-536-7}}
\addcontentsline{toc}{subsection}{(720 ILCS 5/36-7)}

\hypertarget{sec.-36-7.-distribution-of-proceeds-selling-or-retaining-seized-property-prohibited.}{%
\section*{Sec. 36-7. Distribution of proceeds; selling or retaining
seized property
prohibited.}\label{sec.-36-7.-distribution-of-proceeds-selling-or-retaining-seized-property-prohibited.}}
\addcontentsline{toc}{section}{Sec. 36-7. Distribution of proceeds;
selling or retaining seized property prohibited.}

\markright{Sec. 36-7. Distribution of proceeds; selling or retaining
seized property prohibited.}

(a) Except as otherwise provided in this Section, the court shall order
that property forfeited under this Article be delivered to the Illinois
State Police within 60 days.

(b) The Illinois State Police or its designee shall dispose of all
property at public auction and shall distribute the proceeds of the
sale, together with any moneys forfeited or seized, under subsection (c)
of this Section.

(c) All moneys and the sale proceeds of all other property forfeited and
seized under this Act shall be distributed as follows:

(1) 65\% shall be distributed to the drug task force, metropolitan
enforcement group, local, municipal, county, or State law enforcement
agency or agencies that conducted or participated in the investigation
resulting in the forfeiture. The distribution shall bear a reasonable
relationship to the degree of direct participation of the law
enforcement agency in the effort resulting in the forfeiture, taking
into account the total value of the property forfeited and the total law
enforcement effort with respect to the violation of the law upon which
the forfeiture is based. Amounts distributed to the agency or agencies
shall be used, at the discretion of the agency, for the enforcement of
criminal laws; or for public education in the community or schools in
the prevention or detection of the abuse of drugs or alcohol; or for
security cameras used for the prevention or detection of violence,
except that amounts distributed to the Secretary of State shall be
deposited into the Secretary of State Evidence Fund to be used as
provided in Section 2-115 of the Illinois Vehicle Code.

Any local, municipal, or county law enforcement agency entitled to
receive a monetary distribution of forfeiture proceeds may share those
forfeiture proceeds pursuant to the terms of an intergovernmental
agreement with a municipality that has a population in excess of 20,000
if:

(A) the receiving agency has entered into an intergovernmental agreement
with the municipality to provide police services;

(B) the intergovernmental agreement for police services provides for
consideration in an amount of not less than \$1,000,000 per year;

(C) the seizure took place within the geographical limits of the
municipality; and

(D) the funds are used only for the enforcement of criminal laws; for
public education in the community or schools in the prevention or
detection of the abuse of drugs or alcohol; or for security cameras used
for the prevention or detection of violence or the establishment of a
municipal police force, including the training of officers, construction
of a police station, the purchase of law enforcement equipment, or
vehicles.

(2) 12.5\% shall be distributed to the Office of the

State's Attorney of the county in which the prosecution resulting in the
forfeiture was instituted, deposited in a special fund in the county
treasury and appropriated to the State's Attorney for use, at the
discretion of the State's Attorney, in the enforcement of criminal laws;
or for public education in the community or schools in the prevention or
detection of the abuse of drugs or alcohol; or at the discretion of the
State's Attorney, in addition to other authorized purposes, to make
grants to local substance abuse treatment facilities and half-way
houses. In counties over 3,000,000 population, 25\% will be distributed
to the Office of the State's Attorney for use, at the discretion of the
State's Attorney, in the enforcement of criminal laws; or for public
education in the community or schools in the prevention or detection of
the abuse of drugs or alcohol; or at the discretion of the State's
Attorney, in addition to other authorized purposes, to make grants to
local substance abuse treatment facilities and half-way houses. If the
prosecution is undertaken solely by the Attorney General, the portion
provided shall be distributed to the Attorney General for use in the
enforcement of criminal laws governing cannabis and controlled
substances or for public education in the community or schools in the
prevention or detection of the abuse of drugs or alcohol.

12.5\% shall be distributed to the Office of the

State's Attorneys Appellate Prosecutor and shall be used at the
discretion of the State's Attorneys Appellate Prosecutor for additional
expenses incurred in the investigation, prosecution and appeal of cases
arising in the enforcement of criminal laws; or for public education in
the community or schools in the prevention or detection of the abuse of
drugs or alcohol. The Office of the State's Attorneys Appellate
Prosecutor shall not receive distribution from cases brought in counties
with over 3,000,000 population.

(3) 10\% shall be retained by the Illinois State

Police for expenses related to the administration and sale of seized and
forfeited property.

(d) A law enforcement agency shall not retain forfeited property for its
own use or transfer the property to any person or entity, except as
provided under this Section. A law enforcement agency may apply in
writing to the Director of the Illinois State Police to request that
forfeited property be awarded to the agency for a specifically
articulated official law enforcement use in an investigation. The
Director of the Illinois State Police shall provide a written
justification in each instance detailing the reasons why the forfeited
property was placed into official use, and the justification shall be
retained for a period of not less than 3 years.

(Source: P.A. 102-538, eff. 8-20-21.)

\hypertarget{ilcs-536-9}{%
\subsection*{(720 ILCS 5/36-9)}\label{ilcs-536-9}}
\addcontentsline{toc}{subsection}{(720 ILCS 5/36-9)}

\hypertarget{sec.-36-9.-reporting.}{%
\section*{Sec. 36-9. Reporting.}\label{sec.-36-9.-reporting.}}
\addcontentsline{toc}{section}{Sec. 36-9. Reporting.}

\markright{Sec. 36-9. Reporting.}

Property seized or forfeited under this Article is subject to reporting
under the Seizure and Forfeiture Reporting Act.

(Source: P.A. 100-512, eff. 7-1-18 .)

\bookmarksetup{startatroot}

\hypertarget{article-36.5.-vehicle-impoundment}{%
\chapter*{Article 36.5. Vehicle
Impoundment}\label{article-36.5.-vehicle-impoundment}}
\addcontentsline{toc}{chapter}{Article 36.5. Vehicle Impoundment}

\markboth{Article 36.5. Vehicle Impoundment}{Article 36.5. Vehicle
Impoundment}

(Source: P.A. 96-1551, eff. 7-1-11.)

\hypertarget{ilcs-536-10}{%
\subsection*{(720 ILCS 5/36-10)}\label{ilcs-536-10}}
\addcontentsline{toc}{subsection}{(720 ILCS 5/36-10)}

\hypertarget{sec.-36-10.-applicability-savings-clause.}{%
\section*{Sec. 36-10. Applicability; savings
clause.}\label{sec.-36-10.-applicability-savings-clause.}}
\addcontentsline{toc}{section}{Sec. 36-10. Applicability; savings
clause.}

\markright{Sec. 36-10. Applicability; savings clause.}

(a) The changes made to this Article by Public Act 100-512 and Public
Act 100-699 only apply to property seized on and after July 1, 2018.

(b) The changes made to this Article by Public Act 100-699 are subject
to Section 4 of the Statute on Statutes.

(Source: P.A. 100-699, eff. 8-3-18; 100-1163, eff. 12-20-18.)

\hypertarget{ilcs-536.5-5}{%
\subsection*{(720 ILCS 5/36.5-5)}\label{ilcs-536.5-5}}
\addcontentsline{toc}{subsection}{(720 ILCS 5/36.5-5)}

\hypertarget{sec.-36.5-5.-vehicle-impoundment.}{%
\section*{Sec. 36.5-5. Vehicle
impoundment.}\label{sec.-36.5-5.-vehicle-impoundment.}}
\addcontentsline{toc}{section}{Sec. 36.5-5. Vehicle impoundment.}

\markright{Sec. 36.5-5. Vehicle impoundment.}

(a) In addition to any other penalty, fee or forfeiture provided by law,
a peace officer who arrests a person for a violation of Section 10-9,
11-14, 11-14.1, 11-14.3, 11-14.4, 11-18, or 11-18.1 of this Code or
related municipal ordinance, may tow and impound any vehicle used by the
person in the commission of the violation. The person arrested for one
or more such violations shall be charged a \$1,000 fee, to be paid to
the law enforcement agency that made the arrest or its designated
representative. The person may recover the vehicle from the impound
after a minimum of 2 hours after arrest upon payment of the fee.

(b) \$500 of the fee shall be distributed to the law enforcement agency
whose peace officers made the arrest, for the costs incurred by the law
enforcement agency to investigate and to tow and impound the vehicle.
Upon the defendant's conviction of one or more of the violations in
connection with which the vehicle was impounded and the fee imposed
under this Section, the remaining \$500 of the fee shall be deposited
into the Specialized Services for Survivors of Human Trafficking Fund
and disbursed in accordance with subsections (d), (e), and (f) of
Section 5-9-1.21 of the Unified Code of Corrections.

(c) Upon the presentation by the defendant of a signed court order
showing that the defendant has been acquitted of all of the violations
in connection with which a vehicle was impounded and a fee imposed under
this Section, or that the charges against the defendant for those
violations have been dismissed, the law enforcement agency shall refund
the \$1,000 fee to the defendant.

(Source: P.A. 97-333, eff. 8-12-11; 97-897, eff. 1-1-13; 97-1109, eff.
1-1-13; 98-463, eff. 8-16-13; 98-1013, eff. 1-1-15 .)

\bookmarksetup{startatroot}

\hypertarget{article-37.-property-forfeiture}{%
\chapter*{Article 37. Property
Forfeiture}\label{article-37.-property-forfeiture}}
\addcontentsline{toc}{chapter}{Article 37. Property Forfeiture}

\markboth{Article 37. Property Forfeiture}{Article 37. Property
Forfeiture}

\hypertarget{ilcs-537-1-from-ch.-38-par.-37-1}{%
\subsection*{(720 ILCS 5/37-1) (from Ch. 38, par.
37-1)}\label{ilcs-537-1-from-ch.-38-par.-37-1}}
\addcontentsline{toc}{subsection}{(720 ILCS 5/37-1) (from Ch. 38, par.
37-1)}

\hypertarget{sec.-37-1.-maintaining-public-nuisance.}{%
\section*{Sec. 37-1. Maintaining Public
Nuisance.}\label{sec.-37-1.-maintaining-public-nuisance.}}
\addcontentsline{toc}{section}{Sec. 37-1. Maintaining Public Nuisance.}

\markright{Sec. 37-1. Maintaining Public Nuisance.}

Any building used in the commission of offenses prohibited by Sections
9-1, 10-1, 10-2, 11-14, 11-15, 11-16, 11-17, 11-20, 11-20.1, 11-20.1B,
11-20.3, 11-21, 11-22, 12-5.1, 16-1, 20-2, 23-1, 23-1(a)(1), 24-1(a)(7),
24-3, 28-1, 28-3, 31-5 or 39A-1, or subdivision (a)(1), (a)(2)(A), or
(a)(2)(B) of Section 11-14.3, of this Code, or prohibited by the
Illinois Controlled Substances Act, the Methamphetamine Control and
Community Protection Act, or the Cannabis Control Act, or used in the
commission of an inchoate offense relative to any of the aforesaid
principal offenses, or any real property erected, established,
maintained, owned, leased, or used by a streetgang for the purpose of
conducting streetgang related activity as defined in Section 10 of the
Illinois Streetgang Terrorism Omnibus Prevention Act is a public
nuisance.

(b) Sentence. A person convicted of knowingly maintaining such a public
nuisance commits a Class A misdemeanor. Each subsequent offense under
this Section is a Class 4 felony.

(Source: P.A. 96-1551, eff. 7-1-11; 97-1150, eff. 1-25-13.)

\hypertarget{ilcs-537-2-from-ch.-38-par.-37-2}{%
\subsection*{(720 ILCS 5/37-2) (from Ch. 38, par.
37-2)}\label{ilcs-537-2-from-ch.-38-par.-37-2}}
\addcontentsline{toc}{subsection}{(720 ILCS 5/37-2) (from Ch. 38, par.
37-2)}

\hypertarget{sec.-37-2.-enforcement-of-lien-upon-public-nuisance.}{%
\section*{Sec. 37-2. Enforcement of lien upon public
nuisance.}\label{sec.-37-2.-enforcement-of-lien-upon-public-nuisance.}}
\addcontentsline{toc}{section}{Sec. 37-2. Enforcement of lien upon
public nuisance.}

\markright{Sec. 37-2. Enforcement of lien upon public nuisance.}

Any building, used in the commission of an offense specified in Section
37-1 of this Act with the intentional, knowing, reckless or negligent
permission of the owner thereof, or the agent of the owner managing the
building, shall, together with the underlying real estate, all fixtures
and other property used to commit such an offense, be subject to a lien
and may be sold to pay any unsatisfied judgment that may be recovered
and any unsatisfied fine that may be levied under any Section of this
Article and to pay to any person not maintaining the nuisance his
damages as a consequence of the nuisance; provided, that the lien herein
created shall not affect the rights of any purchaser, mortgagee,
judgment creditor or other lien holder arising prior to the filing of a
notice of such lien in the office of the recorder of the county in which
the real estate subject to the lien is located, or in the office of the
registrar of titles of such county if that real estate is registered
under ``An Act concerning land titles'' approved May 1, 1897, as
amended; which notice shall definitely describe the real estate and
property involved, the nature and extent of the lien claimed, and the
facts upon which the same is based. An action to enforce such lien may
be commenced in any circuit court by the State's Attorney of the county
of the nuisance or by the person suffering damages or both, except that
a person seeking to recover damages must pursue his remedy within 6
months after the damages are sustained or his cause of action becomes
thereafter exclusively enforceable by the State's Attorney of the county
of the nuisance.

(Source: P.A. 83-358.)

\hypertarget{ilcs-537-3-from-ch.-38-par.-37-3}{%
\subsection*{(720 ILCS 5/37-3) (from Ch. 38, par.
37-3)}\label{ilcs-537-3-from-ch.-38-par.-37-3}}
\addcontentsline{toc}{subsection}{(720 ILCS 5/37-3) (from Ch. 38, par.
37-3)}

\hypertarget{sec.-37-3.-revocation-of-licenses-permits-and-certificates.}{%
\section*{Sec. 37-3. Revocation of licenses, permits and
certificates.}\label{sec.-37-3.-revocation-of-licenses-permits-and-certificates.}}
\addcontentsline{toc}{section}{Sec. 37-3. Revocation of licenses,
permits and certificates.}

\markright{Sec. 37-3. Revocation of licenses, permits and certificates.}

All licenses, permits or certificates issued by the State of Illinois or
any subdivision or political agency thereof authorizing the serving of
food or liquor on any premises found to constitute a public nuisance as
described in Section 37-1 shall be void and shall be revoked by the
issuing authority; and no license, permit or certificate so revoked
shall be reissued for such premises for a period of 60 days thereafter;
nor shall any person convicted of knowingly maintaining such nuisance be
reissued such license, permit or certificate for one year from his
conviction. No license, permit or certificate shall be revoked pursuant
to this Section without a full hearing conducted by the commission or
agency which issued the license.

(Source: Laws 1965, p.~403.)

\hypertarget{ilcs-537-4-from-ch.-38-par.-37-4}{%
\subsection*{(720 ILCS 5/37-4) (from Ch. 38, par.
37-4)}\label{ilcs-537-4-from-ch.-38-par.-37-4}}
\addcontentsline{toc}{subsection}{(720 ILCS 5/37-4) (from Ch. 38, par.
37-4)}

\hypertarget{sec.-37-4.-abatement-of-nuisance.-the-attorney-general-of-this-state-or-the-states-attorney-of-the-county-wherein-the-nuisance-exists-may-commence-an-action-to-abate-a-public-nuisance-as-described-in-section-37-1-of-this-act-in-the-name-of-the-people-of-the-state-of-illinois-in-the-circuit-court.-upon-being-satisfied-by-affidavits-or-other-sworn-evidence-that-an-alleged-public-nuisance-exists-the-court-may-without-notice-or-bond-enter-a-temporary-restraining-order-or-preliminary-injunction-to-enjoin-any-defendant-from-maintaining-such-nuisance-and-may-enter-an-order-restraining-any-defendant-from-removing-or-interfering-with-all-property-used-in-connection-with-the-public-nuisance.-if-during-the-proceedings-and-hearings-upon-the-merits-which-shall-be-in-the-manner-of-an-act-in-relation-to-places-used-for-the-purpose-of-using-keeping-or-selling-controlled-substances-or-cannabis-approved-july-5-1957-the-existence-of-the-nuisance-is-established-and-it-is-found-that-such-nuisance-was-maintained-with-the-intentional-knowing-reckless-or-negligent-permission-of-the-owner-or-the-agent-of-the-owner-managing-the-building-the-court-shall-enter-an-order-restraining-all-persons-from-maintaining-or-permitting-such-nuisance-and-from-using-the-building-for-a-period-of-one-year-thereafter-except-that-an-owner-lessee-or-other-occupant-thereof-may-use-such-place-if-the-owner-shall-give-bond-with-sufficient-security-or-surety-approved-by-the-court-in-an-amount-between-1000-and-5000-inclusive-payable-to-the-people-of-the-state-of-illinois-and-including-a-condition-that-no-offense-specified-in-section-37-1-of-this-act-shall-be-committed-at-in-or-upon-the-property-described-and-a-condition-that-the-principal-obligor-and-surety-assume-responsibility-for-any-fine-costs-or-damages-resulting-from-such-an-offense-thereafter.}{%
\section*{Sec. 37-4. Abatement of nuisance.) The Attorney General of
this State or the State's Attorney of the county wherein the nuisance
exists may commence an action to abate a public nuisance as described in
Section 37-1 of this Act, in the name of the People of the State of
Illinois, in the circuit court. Upon being satisfied by affidavits or
other sworn evidence that an alleged public nuisance exists, the court
may without notice or bond enter a temporary restraining order or
preliminary injunction to enjoin any defendant from maintaining such
nuisance and may enter an order restraining any defendant from removing
or interfering with all property used in connection with the public
nuisance. If during the proceedings and hearings upon the merits, which
shall be in the manner of ``An Act in relation to places used for the
purpose of using, keeping or selling controlled substances or
cannabis'', approved July 5, 1957, the existence of the nuisance is
established, and it is found that such nuisance was maintained with the
intentional, knowing, reckless or negligent permission of the owner or
the agent of the owner managing the building, the court shall enter an
order restraining all persons from maintaining or permitting such
nuisance and from using the building for a period of one year
thereafter, except that an owner, lessee or other occupant thereof may
use such place if the owner shall give bond with sufficient security or
surety approved by the court, in an amount between \$1,000 and \$5,000
inclusive, payable to the People of the State of Illinois, and including
a condition that no offense specified in Section 37-1 of this Act shall
be committed at, in or upon the property described and a condition that
the principal obligor and surety assume responsibility for any fine,
costs or damages resulting from such an offense
thereafter.}\label{sec.-37-4.-abatement-of-nuisance.-the-attorney-general-of-this-state-or-the-states-attorney-of-the-county-wherein-the-nuisance-exists-may-commence-an-action-to-abate-a-public-nuisance-as-described-in-section-37-1-of-this-act-in-the-name-of-the-people-of-the-state-of-illinois-in-the-circuit-court.-upon-being-satisfied-by-affidavits-or-other-sworn-evidence-that-an-alleged-public-nuisance-exists-the-court-may-without-notice-or-bond-enter-a-temporary-restraining-order-or-preliminary-injunction-to-enjoin-any-defendant-from-maintaining-such-nuisance-and-may-enter-an-order-restraining-any-defendant-from-removing-or-interfering-with-all-property-used-in-connection-with-the-public-nuisance.-if-during-the-proceedings-and-hearings-upon-the-merits-which-shall-be-in-the-manner-of-an-act-in-relation-to-places-used-for-the-purpose-of-using-keeping-or-selling-controlled-substances-or-cannabis-approved-july-5-1957-the-existence-of-the-nuisance-is-established-and-it-is-found-that-such-nuisance-was-maintained-with-the-intentional-knowing-reckless-or-negligent-permission-of-the-owner-or-the-agent-of-the-owner-managing-the-building-the-court-shall-enter-an-order-restraining-all-persons-from-maintaining-or-permitting-such-nuisance-and-from-using-the-building-for-a-period-of-one-year-thereafter-except-that-an-owner-lessee-or-other-occupant-thereof-may-use-such-place-if-the-owner-shall-give-bond-with-sufficient-security-or-surety-approved-by-the-court-in-an-amount-between-1000-and-5000-inclusive-payable-to-the-people-of-the-state-of-illinois-and-including-a-condition-that-no-offense-specified-in-section-37-1-of-this-act-shall-be-committed-at-in-or-upon-the-property-described-and-a-condition-that-the-principal-obligor-and-surety-assume-responsibility-for-any-fine-costs-or-damages-resulting-from-such-an-offense-thereafter.}}
\addcontentsline{toc}{section}{Sec. 37-4. Abatement of nuisance.) The
Attorney General of this State or the State's Attorney of the county
wherein the nuisance exists may commence an action to abate a public
nuisance as described in Section 37-1 of this Act, in the name of the
People of the State of Illinois, in the circuit court. Upon being
satisfied by affidavits or other sworn evidence that an alleged public
nuisance exists, the court may without notice or bond enter a temporary
restraining order or preliminary injunction to enjoin any defendant from
maintaining such nuisance and may enter an order restraining any
defendant from removing or interfering with all property used in
connection with the public nuisance. If during the proceedings and
hearings upon the merits, which shall be in the manner of ``An Act in
relation to places used for the purpose of using, keeping or selling
controlled substances or cannabis'', approved July 5, 1957, the
existence of the nuisance is established, and it is found that such
nuisance was maintained with the intentional, knowing, reckless or
negligent permission of the owner or the agent of the owner managing the
building, the court shall enter an order restraining all persons from
maintaining or permitting such nuisance and from using the building for
a period of one year thereafter, except that an owner, lessee or other
occupant thereof may use such place if the owner shall give bond with
sufficient security or surety approved by the court, in an amount
between \$1,000 and \$5,000 inclusive, payable to the People of the
State of Illinois, and including a condition that no offense specified
in Section 37-1 of this Act shall be committed at, in or upon the
property described and a condition that the principal obligor and surety
assume responsibility for any fine, costs or damages resulting from such
an offense thereafter.}

\markright{Sec. 37-4. Abatement of nuisance.) The Attorney General of
this State or the State's Attorney of the county wherein the nuisance
exists may commence an action to abate a public nuisance as described in
Section 37-1 of this Act, in the name of the People of the State of
Illinois, in the circuit court. Upon being satisfied by affidavits or
other sworn evidence that an alleged public nuisance exists, the court
may without notice or bond enter a temporary restraining order or
preliminary injunction to enjoin any defendant from maintaining such
nuisance and may enter an order restraining any defendant from removing
or interfering with all property used in connection with the public
nuisance. If during the proceedings and hearings upon the merits, which
shall be in the manner of ``An Act in relation to places used for the
purpose of using, keeping or selling controlled substances or
cannabis'', approved July 5, 1957, the existence of the nuisance is
established, and it is found that such nuisance was maintained with the
intentional, knowing, reckless or negligent permission of the owner or
the agent of the owner managing the building, the court shall enter an
order restraining all persons from maintaining or permitting such
nuisance and from using the building for a period of one year
thereafter, except that an owner, lessee or other occupant thereof may
use such place if the owner shall give bond with sufficient security or
surety approved by the court, in an amount between \$1,000 and \$5,000
inclusive, payable to the People of the State of Illinois, and including
a condition that no offense specified in Section 37-1 of this Act shall
be committed at, in or upon the property described and a condition that
the principal obligor and surety assume responsibility for any fine,
costs or damages resulting from such an offense thereafter.}

(Source: P.A. 83-342.)

\hypertarget{ilcs-537-5-from-ch.-38-par.-37-5}{%
\subsection*{(720 ILCS 5/37-5) (from Ch. 38, par.
37-5)}\label{ilcs-537-5-from-ch.-38-par.-37-5}}
\addcontentsline{toc}{subsection}{(720 ILCS 5/37-5) (from Ch. 38, par.
37-5)}

\hypertarget{sec.-37-5.-enforcement-by-private-person.}{%
\section*{Sec. 37-5. Enforcement by private
person.}\label{sec.-37-5.-enforcement-by-private-person.}}
\addcontentsline{toc}{section}{Sec. 37-5. Enforcement by private
person.}

\markright{Sec. 37-5. Enforcement by private person.}

A private person may, after 30 days and within 90 days of giving the
Attorney General and the State's Attorney of the county of nuisance
written notice by certified or registered mail of the fact that a public
nuisance as described in Section 37-1 of this Act, commence an action
pursuant to Section 37-4 of this Act, provided that the Attorney General
or the State's Attorney of the county of nuisance has not already
commenced said action.

(Source: Laws 1965, p.~403.)

\bookmarksetup{startatroot}

\hypertarget{article-37.5.-animal-fighting-forfeiture}{%
\chapter*{Article 37.5. Animal Fighting;
Forfeiture}\label{article-37.5.-animal-fighting-forfeiture}}
\addcontentsline{toc}{chapter}{Article 37.5. Animal Fighting;
Forfeiture}

\markboth{Article 37.5. Animal Fighting; Forfeiture}{Article 37.5.
Animal Fighting; Forfeiture}

\hypertarget{ilcs-537.5-5}{%
\subsection*{(720 ILCS 5/37.5-5)}\label{ilcs-537.5-5}}
\addcontentsline{toc}{subsection}{(720 ILCS 5/37.5-5)}

\hypertarget{sec.-37.5-5.-repealed.}{%
\section*{Sec. 37.5-5. (Repealed).}\label{sec.-37.5-5.-repealed.}}
\addcontentsline{toc}{section}{Sec. 37.5-5. (Repealed).}

\markright{Sec. 37.5-5. (Repealed).}

(Source: P.A. 93-192, eff. 7-14-03. Repealed by P.A. 96-712, eff.
1-1-10.)

\hypertarget{ilcs-537.5-10}{%
\subsection*{(720 ILCS 5/37.5-10)}\label{ilcs-537.5-10}}
\addcontentsline{toc}{subsection}{(720 ILCS 5/37.5-10)}

\hypertarget{sec.-37.5-10.-repealed.}{%
\section*{Sec. 37.5-10. (Repealed).}\label{sec.-37.5-10.-repealed.}}
\addcontentsline{toc}{section}{Sec. 37.5-10. (Repealed).}

\markright{Sec. 37.5-10. (Repealed).}

(Source: P.A. 93-192, eff. 7-14-03. Repealed by P.A. 96-712, eff.
1-1-10.)

\hypertarget{ilcs-537.5-15}{%
\subsection*{(720 ILCS 5/37.5-15)}\label{ilcs-537.5-15}}
\addcontentsline{toc}{subsection}{(720 ILCS 5/37.5-15)}

\hypertarget{sec.-37.5-15.-repealed.}{%
\section*{Sec. 37.5-15. (Repealed).}\label{sec.-37.5-15.-repealed.}}
\addcontentsline{toc}{section}{Sec. 37.5-15. (Repealed).}

\markright{Sec. 37.5-15. (Repealed).}

(Source: P.A. 93-192, eff. 7-14-03. Repealed by P.A. 96-712, eff.
1-1-10.)

\hypertarget{ilcs-537.5-20}{%
\subsection*{(720 ILCS 5/37.5-20)}\label{ilcs-537.5-20}}
\addcontentsline{toc}{subsection}{(720 ILCS 5/37.5-20)}

\hypertarget{sec.-37.5-20.-repealed.}{%
\section*{Sec. 37.5-20. (Repealed).}\label{sec.-37.5-20.-repealed.}}
\addcontentsline{toc}{section}{Sec. 37.5-20. (Repealed).}

\markright{Sec. 37.5-20. (Repealed).}

(Source: P.A. 93-192, eff. 7-14-03. Repealed by P.A. 96-712, eff.
1-1-10.)

\hypertarget{ilcs-537.5-25}{%
\subsection*{(720 ILCS 5/37.5-25)}\label{ilcs-537.5-25}}
\addcontentsline{toc}{subsection}{(720 ILCS 5/37.5-25)}

\hypertarget{sec.-37.5-25.-repealed.}{%
\section*{Sec. 37.5-25. (Repealed).}\label{sec.-37.5-25.-repealed.}}
\addcontentsline{toc}{section}{Sec. 37.5-25. (Repealed).}

\markright{Sec. 37.5-25. (Repealed).}

(Source: P.A. 93-192, eff. 7-14-03. Repealed by P.A. 96-712, eff.
1-1-10.)

\hypertarget{ilcs-537.5-30}{%
\subsection*{(720 ILCS 5/37.5-30)}\label{ilcs-537.5-30}}
\addcontentsline{toc}{subsection}{(720 ILCS 5/37.5-30)}

\hypertarget{sec.-37.5-30.-repealed.}{%
\section*{Sec. 37.5-30. (Repealed).}\label{sec.-37.5-30.-repealed.}}
\addcontentsline{toc}{section}{Sec. 37.5-30. (Repealed).}

\markright{Sec. 37.5-30. (Repealed).}

(Source: P.A. 93-192, eff. 7-14-03. Repealed by P.A. 96-712, eff.
1-1-10.)

\hypertarget{ilcs-537.5-35}{%
\subsection*{(720 ILCS 5/37.5-35)}\label{ilcs-537.5-35}}
\addcontentsline{toc}{subsection}{(720 ILCS 5/37.5-35)}

\hypertarget{sec.-37.5-35.-repealed.}{%
\section*{Sec. 37.5-35. (Repealed).}\label{sec.-37.5-35.-repealed.}}
\addcontentsline{toc}{section}{Sec. 37.5-35. (Repealed).}

\markright{Sec. 37.5-35. (Repealed).}

(Source: P.A. 93-192, eff. 7-14-03. Repealed by P.A. 96-712, eff.
1-1-10.)

\hypertarget{ilcs-537.5-40}{%
\subsection*{(720 ILCS 5/37.5-40)}\label{ilcs-537.5-40}}
\addcontentsline{toc}{subsection}{(720 ILCS 5/37.5-40)}

\hypertarget{sec.-37.5-40.-repealed.}{%
\section*{Sec. 37.5-40. (Repealed).}\label{sec.-37.5-40.-repealed.}}
\addcontentsline{toc}{section}{Sec. 37.5-40. (Repealed).}

\markright{Sec. 37.5-40. (Repealed).}

(Source: P.A. 93-192, eff. 7-14-03. Repealed by P.A. 96-712, eff.
1-1-10.)

\hypertarget{ilcs-537.5-45}{%
\subsection*{(720 ILCS 5/37.5-45)}\label{ilcs-537.5-45}}
\addcontentsline{toc}{subsection}{(720 ILCS 5/37.5-45)}

\hypertarget{sec.-37.5-45.-repealed.}{%
\section*{Sec. 37.5-45. (Repealed).}\label{sec.-37.5-45.-repealed.}}
\addcontentsline{toc}{section}{Sec. 37.5-45. (Repealed).}

\markright{Sec. 37.5-45. (Repealed).}

(Source: P.A. 93-192, eff. 7-14-03. Repealed by P.A. 96-712, eff.
1-1-10.)

\bookmarksetup{startatroot}

\hypertarget{article-38.-criminally-operated-businesses}{%
\chapter*{Article 38. Criminally Operated
Businesses}\label{article-38.-criminally-operated-businesses}}
\addcontentsline{toc}{chapter}{Article 38. Criminally Operated
Businesses}

\markboth{Article 38. Criminally Operated Businesses}{Article 38.
Criminally Operated Businesses}

\hypertarget{ilcs-538-1-from-ch.-38-par.-38-1}{%
\subsection*{(720 ILCS 5/38-1) (from Ch. 38, par.
38-1)}\label{ilcs-538-1-from-ch.-38-par.-38-1}}
\addcontentsline{toc}{subsection}{(720 ILCS 5/38-1) (from Ch. 38, par.
38-1)}

\hypertarget{sec.-38-1.-forfeiture-of-charter-and-revocation-of-certificate.}{%
\section*{Sec. 38-1. Forfeiture of charter and revocation of
certificate.}\label{sec.-38-1.-forfeiture-of-charter-and-revocation-of-certificate.}}
\addcontentsline{toc}{section}{Sec. 38-1. Forfeiture of charter and
revocation of certificate.}

\markright{Sec. 38-1. Forfeiture of charter and revocation of
certificate.}

The State's Attorney is authorized to institute civil proceedings in the
Circuit Court to forfeit the charter of a corporation organized under
the laws of this State or to revoke the certificate authorizing a
foreign corporation to conduct business in this State. The Court may
order the charter forfeited or the certificate revoked upon finding (a)
that a director, officer, employee, agent or stockholder acting in
behalf of the corporation has, in conducting the corporation's affairs,
purposely engaged in a persistent course of intimidation, coercion,
bribery or other such illegal conduct with the intent to compel other
persons, firms, or corporations to deal with such corporation, and (b)
that for the prevention of future illegal conduct of the same character,
the public interest requires the charter of the corporation to be
forfeited and the corporation to be dissolved or the certificate to be
revoked.

(Source: Laws 1965, p.~1222.)

\hypertarget{ilcs-538-2-from-ch.-38-par.-38-2}{%
\subsection*{(720 ILCS 5/38-2) (from Ch. 38, par.
38-2)}\label{ilcs-538-2-from-ch.-38-par.-38-2}}
\addcontentsline{toc}{subsection}{(720 ILCS 5/38-2) (from Ch. 38, par.
38-2)}

\hypertarget{sec.-38-2.-enjoining-operation-of-a-business.}{%
\section*{Sec. 38-2. Enjoining operation of a
business.}\label{sec.-38-2.-enjoining-operation-of-a-business.}}
\addcontentsline{toc}{section}{Sec. 38-2. Enjoining operation of a
business.}

\markright{Sec. 38-2. Enjoining operation of a business.}

The State's Attorney is authorized to institute civil proceedings in the
Circuit Court to enjoin the operation of any business other than a
corporation, including a partnership, joint venture or sole
proprietorship. The Court may grant the injunction upon finding that (a)
any person in control of any such business, who may be a partner in a
partnership, a participant in a joint venture, the owner of a sole
proprietorship, an employee or agent of any such business, or a person
who, in fact, exercises control over the operations of any such
business, has, in conducting its business affairs, purposely engaged in
a persistent course of intimidation, coercion, bribery or other such
illegal conduct with the intent to compel other persons, firms, or
corporations to deal with such business, and (b) that for the prevention
of future illegal conduct of the same character, the public interest
requires the operation of the business to be enjoined.

(Source: Laws 1965, p.~1222.)

\hypertarget{ilcs-538-3-from-ch.-38-par.-38-3}{%
\subsection*{(720 ILCS 5/38-3) (from Ch. 38, par.
38-3)}\label{ilcs-538-3-from-ch.-38-par.-38-3}}
\addcontentsline{toc}{subsection}{(720 ILCS 5/38-3) (from Ch. 38, par.
38-3)}

\hypertarget{sec.-38-3.-institution-and-conduct-of-proceedings.-a-the-proceedings-authorized-by-section-38-1-may-be-instituted-against-a-corporation-in-any-county-in-which-it-is-doing-business-and-the-proceedings-shall-be-conducted-in-accordance-with-the-civil-practice-law-and-all-existing-and-future-amendments-of-that-law-and-the-supreme-court-rules-now-or-hereafter-adopted-in-relation-to-that-law.-such-proceedings-shall-be-deemed-additional-to-any-other-proceeding-authorized-by-law-for-the-purpose-of-forfeiting-the-charter-of-a-corporation-or-revoking-the-certificate-of-a-foreign-corporation.}{%
\section*{Sec. 38-3. Institution and conduct of proceedings.) (a) The
proceedings authorized by Section 38-1 may be instituted against a
corporation in any county in which it is doing business and the
proceedings shall be conducted in accordance with the Civil Practice Law
and all existing and future amendments of that Law and the Supreme Court
Rules now or hereafter adopted in relation to that Law. Such proceedings
shall be deemed additional to any other proceeding authorized by law for
the purpose of forfeiting the charter of a corporation or revoking the
certificate of a foreign
corporation.}\label{sec.-38-3.-institution-and-conduct-of-proceedings.-a-the-proceedings-authorized-by-section-38-1-may-be-instituted-against-a-corporation-in-any-county-in-which-it-is-doing-business-and-the-proceedings-shall-be-conducted-in-accordance-with-the-civil-practice-law-and-all-existing-and-future-amendments-of-that-law-and-the-supreme-court-rules-now-or-hereafter-adopted-in-relation-to-that-law.-such-proceedings-shall-be-deemed-additional-to-any-other-proceeding-authorized-by-law-for-the-purpose-of-forfeiting-the-charter-of-a-corporation-or-revoking-the-certificate-of-a-foreign-corporation.}}
\addcontentsline{toc}{section}{Sec. 38-3. Institution and conduct of
proceedings.) (a) The proceedings authorized by Section 38-1 may be
instituted against a corporation in any county in which it is doing
business and the proceedings shall be conducted in accordance with the
Civil Practice Law and all existing and future amendments of that Law
and the Supreme Court Rules now or hereafter adopted in relation to that
Law. Such proceedings shall be deemed additional to any other proceeding
authorized by law for the purpose of forfeiting the charter of a
corporation or revoking the certificate of a foreign corporation.}

\markright{Sec. 38-3. Institution and conduct of proceedings.) (a) The
proceedings authorized by Section 38-1 may be instituted against a
corporation in any county in which it is doing business and the
proceedings shall be conducted in accordance with the Civil Practice Law
and all existing and future amendments of that Law and the Supreme Court
Rules now or hereafter adopted in relation to that Law. Such proceedings
shall be deemed additional to any other proceeding authorized by law for
the purpose of forfeiting the charter of a corporation or revoking the
certificate of a foreign corporation.}

(b) The proceedings authorized by Section 38-2 may be instituted against
a business other than a corporation in any county in which it is doing
business and the proceedings shall be conducted in accordance with the
Civil Practice Law and all existing and future amendments of that Law
and the Supreme Court Rules now or hereafter adopted in relation to that
Law.

(c) Whenever proceedings are instituted against a corporation or
business pursuant to Section 38-1 or 38-2, the State's Attorney shall
give written notice of the institution of such proceedings to the
corporation or business against which the proceedings are brought.

(Source: P.A. 82-783.)

\bookmarksetup{startatroot}

\hypertarget{article-39.-criminal-usury}{%
\chapter*{Article 39. Criminal Usury}\label{article-39.-criminal-usury}}
\addcontentsline{toc}{chapter}{Article 39. Criminal Usury}

\markboth{Article 39. Criminal Usury}{Article 39. Criminal Usury}

(Repealed)

(Article heading repealed by P.A. 96-1551, eff. 7-1-11)

\hypertarget{ilcs-539-1-from-ch.-38-par.-39-1}{%
\subsection*{(720 ILCS 5/39-1) (from Ch. 38, par.
39-1)}\label{ilcs-539-1-from-ch.-38-par.-39-1}}
\addcontentsline{toc}{subsection}{(720 ILCS 5/39-1) (from Ch. 38, par.
39-1)}

(This Section was renumbered as Section 17-59 by P.A. 96-1551.)

\hypertarget{sec.-39-1.-renumbered.}{%
\section*{Sec. 39-1. (Renumbered).}\label{sec.-39-1.-renumbered.}}
\addcontentsline{toc}{section}{Sec. 39-1. (Renumbered).}

\markright{Sec. 39-1. (Renumbered).}

(Source: P.A. 76-1879. Renumbered by P.A. 96-1551, eff. 7-1-11 .)

\hypertarget{ilcs-539-2-from-ch.-38-par.-39-2}{%
\subsection*{(720 ILCS 5/39-2) (from Ch. 38, par.
39-2)}\label{ilcs-539-2-from-ch.-38-par.-39-2}}
\addcontentsline{toc}{subsection}{(720 ILCS 5/39-2) (from Ch. 38, par.
39-2)}

\hypertarget{sec.-39-2.-repealed.}{%
\section*{Sec. 39-2. (Repealed).}\label{sec.-39-2.-repealed.}}
\addcontentsline{toc}{section}{Sec. 39-2. (Repealed).}

\markright{Sec. 39-2. (Repealed).}

(Source: P.A. 77-2638. Repealed by P.A. 96-1551, eff. 7-1-11 .)

\hypertarget{ilcs-539-3-from-ch.-38-par.-39-3}{%
\subsection*{(720 ILCS 5/39-3) (from Ch. 38, par.
39-3)}\label{ilcs-539-3-from-ch.-38-par.-39-3}}
\addcontentsline{toc}{subsection}{(720 ILCS 5/39-3) (from Ch. 38, par.
39-3)}

\hypertarget{sec.-39-3.-repealed.}{%
\section*{Sec. 39-3. (Repealed).}\label{sec.-39-3.-repealed.}}
\addcontentsline{toc}{section}{Sec. 39-3. (Repealed).}

\markright{Sec. 39-3. (Repealed).}

(Source: P.A. 84-1004. Repealed by P.A. 96-1551, eff. 7-1-11 .)

\bookmarksetup{startatroot}

\hypertarget{article-42.-looting}{%
\chapter*{Article 42. Looting}\label{article-42.-looting}}
\addcontentsline{toc}{chapter}{Article 42. Looting}

\markboth{Article 42. Looting}{Article 42. Looting}

\hypertarget{ilcs-542-1}{%
\subsection*{(720 ILCS 5/42-1)}\label{ilcs-542-1}}
\addcontentsline{toc}{subsection}{(720 ILCS 5/42-1)}

\hypertarget{sec.-42-1.-repealed.}{%
\section*{Sec. 42-1. (Repealed).}\label{sec.-42-1.-repealed.}}
\addcontentsline{toc}{section}{Sec. 42-1. (Repealed).}

\markright{Sec. 42-1. (Repealed).}

(Source: Laws 1967, p.~2598. Repealed by P.A. 96-710, eff. 1-1-10.)

\hypertarget{ilcs-542-2}{%
\subsection*{(720 ILCS 5/42-2)}\label{ilcs-542-2}}
\addcontentsline{toc}{subsection}{(720 ILCS 5/42-2)}

\hypertarget{sec.-42-2.-repealed.}{%
\section*{Sec. 42-2. (Repealed).}\label{sec.-42-2.-repealed.}}
\addcontentsline{toc}{section}{Sec. 42-2. (Repealed).}

\markright{Sec. 42-2. (Repealed).}

(Source: P.A. 87-1170. Repealed by P.A. 96-710, eff. 1-1-10.)

\bookmarksetup{startatroot}

\hypertarget{article-44.-telecommunications-devices}{%
\chapter*{Article 44. Telecommunications
Devices}\label{article-44.-telecommunications-devices}}
\addcontentsline{toc}{chapter}{Article 44. Telecommunications Devices}

\markboth{Article 44. Telecommunications Devices}{Article 44.
Telecommunications Devices}

(Repealed)

(Source: P.A. 86-811. Repealed by P.A. 97-1109, eff. 1-1-13.)

\hypertarget{ilcs-544-1-from-ch.-38-par.-44-1}{%
\subsection*{(720 ILCS 5/44-1) (from Ch. 38, par.
44-1)}\label{ilcs-544-1-from-ch.-38-par.-44-1}}
\addcontentsline{toc}{subsection}{(720 ILCS 5/44-1) (from Ch. 38, par.
44-1)}

\hypertarget{sec.-44-1.-repealed.}{%
\section*{Sec. 44-1. (Repealed).}\label{sec.-44-1.-repealed.}}
\addcontentsline{toc}{section}{Sec. 44-1. (Repealed).}

\markright{Sec. 44-1. (Repealed).}

(Source: P.A. 86-811. Repealed by P.A. 97-1109, eff. 1-1-13.)

\hypertarget{ilcs-544-2-from-ch.-38-par.-44-2}{%
\subsection*{(720 ILCS 5/44-2) (from Ch. 38, par.
44-2)}\label{ilcs-544-2-from-ch.-38-par.-44-2}}
\addcontentsline{toc}{subsection}{(720 ILCS 5/44-2) (from Ch. 38, par.
44-2)}

(This Section was renumbered as Section 12C-65 by P.A. 97-1109.)

\hypertarget{sec.-44-2.-renumbered.}{%
\section*{Sec. 44-2. (Renumbered).}\label{sec.-44-2.-renumbered.}}
\addcontentsline{toc}{section}{Sec. 44-2. (Renumbered).}

\markright{Sec. 44-2. (Renumbered).}

(Source: P.A. 94-556, eff. 9-11-05. Renumbered by P.A. 97-1109, eff.
1-1-13.)

\hypertarget{ilcs-544-3-from-ch.-38-par.-44-3}{%
\subsection*{(720 ILCS 5/44-3) (from Ch. 38, par.
44-3)}\label{ilcs-544-3-from-ch.-38-par.-44-3}}
\addcontentsline{toc}{subsection}{(720 ILCS 5/44-3) (from Ch. 38, par.
44-3)}

\hypertarget{sec.-44-3.-repealed.}{%
\section*{Sec. 44-3. (Repealed).}\label{sec.-44-3.-repealed.}}
\addcontentsline{toc}{section}{Sec. 44-3. (Repealed).}

\markright{Sec. 44-3. (Repealed).}

(Source: P.A. 95-331, eff. 8-21-07. Repealed by P.A. 97-1109, eff.
1-1-13.)

\bookmarksetup{startatroot}

\hypertarget{article-45.-disclosing-location}{%
\chapter*{Article 45. Disclosing
Location}\label{article-45.-disclosing-location}}
\addcontentsline{toc}{chapter}{Article 45. Disclosing Location}

\markboth{Article 45. Disclosing Location}{Article 45. Disclosing
Location}

OF DOMESTIC VIOLENCE VICTIM

(Repealed)

(Article heading repealed by P.A. 96-1551, eff. 7-1-11)

\hypertarget{ilcs-545-1-from-ch.-38-par.-45-1}{%
\subsection*{(720 ILCS 5/45-1) (from Ch. 38, par.
45-1)}\label{ilcs-545-1-from-ch.-38-par.-45-1}}
\addcontentsline{toc}{subsection}{(720 ILCS 5/45-1) (from Ch. 38, par.
45-1)}

(This Section was renumbered as Section 12-3.6 by P.A. 96-1551.)

\hypertarget{sec.-45-1.-renumbered.}{%
\section*{Sec. 45-1. (Renumbered).}\label{sec.-45-1.-renumbered.}}
\addcontentsline{toc}{section}{Sec. 45-1. (Renumbered).}

\markright{Sec. 45-1. (Renumbered).}

(Source: P.A. 88-45. Renumbered by P.A. 96-1551, eff. 7-1-11 .)

\hypertarget{ilcs-545-2-from-ch.-38-par.-45-2}{%
\subsection*{(720 ILCS 5/45-2) (from Ch. 38, par.
45-2)}\label{ilcs-545-2-from-ch.-38-par.-45-2}}
\addcontentsline{toc}{subsection}{(720 ILCS 5/45-2) (from Ch. 38, par.
45-2)}

(This Section was renumbered as Section 12-3.6 by P.A. 96-1551.)

\hypertarget{sec.-45-2.-renumbered.}{%
\section*{Sec. 45-2. (Renumbered).}\label{sec.-45-2.-renumbered.}}
\addcontentsline{toc}{section}{Sec. 45-2. (Renumbered).}

\markright{Sec. 45-2. (Renumbered).}

(Source: P.A. 88-45. Renumbered by P.A. 96-1551, eff. 7-1-11 .)

\bookmarksetup{startatroot}

\hypertarget{article-46.-insurance-fraud-fraud-on-the-government}{%
\chapter*{Article 46. Insurance Fraud, Fraud On The
Government,}\label{article-46.-insurance-fraud-fraud-on-the-government}}
\addcontentsline{toc}{chapter}{Article 46. Insurance Fraud, Fraud On The
Government,}

\markboth{Article 46. Insurance Fraud, Fraud On The Government,}{Article
46. Insurance Fraud, Fraud On The Government,}

AND RELATED OFFENSES

(Repealed)

(Article repealed by P.A. 96-1551, eff. 7-1-11)

\bookmarksetup{startatroot}

\hypertarget{article-47.-nuisance}{%
\chapter*{Article 47. Nuisance}\label{article-47.-nuisance}}
\addcontentsline{toc}{chapter}{Article 47. Nuisance}

\markboth{Article 47. Nuisance}{Article 47. Nuisance}

\hypertarget{ilcs-547-5}{%
\subsection*{(720 ILCS 5/47-5)}\label{ilcs-547-5}}
\addcontentsline{toc}{subsection}{(720 ILCS 5/47-5)}

\hypertarget{sec.-47-5.-public-nuisance.}{%
\section*{Sec. 47-5. Public
nuisance.}\label{sec.-47-5.-public-nuisance.}}
\addcontentsline{toc}{section}{Sec. 47-5. Public nuisance.}

\markright{Sec. 47-5. Public nuisance.}

It is a public nuisance:

(1) To cause or allow the carcass of an animal or offal, filth, or a
noisome substance to be collected, deposited, or to remain in any place
to the prejudice of others.

(2) To throw or deposit offal or other offensive matter or the carcass
of a dead animal in a water course, lake, pond, spring, well, or common
sewer, street, or public highway.

(3) To corrupt or render unwholesome or impure the water of a spring,
river, stream, pond, or lake to the injury or prejudice of others.

(4) To obstruct or impede, without legal authority, the passage of a
navigable river or waters.

(5) To obstruct or encroach upon public highways, private ways, streets,
alleys, commons, landing places, and ways to burying places.

(6) To carry on the business of manufacturing gunpowder, nitroglycerine,
or other highly explosive substances, or mixing or grinding the
materials for those substances, in a building within 20 rods of a
valuable building erected at the time the business is commenced.

(7) To establish powder magazines near incorporated towns, at a point
different from that appointed according to law by the corporate
authorities of the town, or within 50 rods of an occupied dwelling
house.

(8) To erect, continue, or use a building or other place for the
exercise of a trade, employment, or manufacture that, by occasioning
noxious exhalations, offensive smells, or otherwise, is offensive or
dangerous to the health of individuals or of the public.

(9) To advertise wares or occupation by painting notices of the wares or
occupation on or affixing them to fences or other private property, or
on rocks or other natural objects, without the consent of the owner, or
if in the highway or other public place, without permission of the
proper authorities.

(10) To permit a well drilled for oil, gas, salt water disposal, or any
other purpose in connection with the production of oil and gas to remain
unplugged after the well is no longer used for the purpose for which it
was drilled.

(11) To construct or operate a salt water pit or oil field refuse pit,
commonly called a ``burn out pit'', so that salt water, brine, or oil
field refuse or other waste liquids may escape from the pit in a manner
except by the evaporation of the salt water or brine or by the burning
of the oil field waste or refuse.

(12) To permit concrete bases, discarded machinery, and materials to
remain around an oil or gas well, or to fail to fill holes, cellars,
slush pits, and other excavations made in connection with the well or to
restore the surface of the lands surrounding the well to its condition
before the drilling of the well, upon abandonment of the oil or gas
well.

(13) To permit salt water, oil, gas, or other wastes from a well drilled
for oil, gas, or exploratory purposes to escape to the surface, or into
a mine or coal seam, or into an underground fresh water supply, or from
one underground stratum to another.

(14) To harass, intimidate, or threaten a person who is about to sell or
lease or has sold or leased a residence or other real property or is
about to buy or lease or has bought or leased a residence or other real
property, when the harassment, intimidation, or threat relates to a
person's attempt to sell, buy, or lease a residence, or other real
property, or refers to a person's sale, purchase, or lease of a
residence or other real property.

(15) To store, dump, or permit the accumulation of debris, refuse,
garbage, trash, tires, buckets, cans, wheelbarrows, garbage cans, or
other containers in a manner that may harbor mosquitoes, flies, insects,
rodents, nuisance birds, or other animal pests that are offensive,
injurious, or dangerous to the health of individuals or the public.

(16) To create a condition, through the improper maintenance of a
swimming pool or wading pool, or by causing an action that alters the
condition of a natural body of water, so that it harbors mosquitoes,
flies, or other animal pests that are offensive, injurious, or dangerous
to the health of individuals or the public.

(17) To operate a tanning facility without a valid permit under the
Tanning Facility Permit Act.

Nothing in this Section shall be construed to prevent the corporate
authorities of a city, village, or incorporated town, or the county
board of a county, from declaring what are nuisances and abating them
within their limits. Counties have that authority only outside the
corporate limits of a city, village, or incorporated town.

(Source: P.A. 89-234, eff. 1-1-96.)

\hypertarget{ilcs-547-10}{%
\subsection*{(720 ILCS 5/47-10)}\label{ilcs-547-10}}
\addcontentsline{toc}{subsection}{(720 ILCS 5/47-10)}

\hypertarget{sec.-47-10.-dumping-garbage.}{%
\section*{Sec. 47-10. Dumping
garbage.}\label{sec.-47-10.-dumping-garbage.}}
\addcontentsline{toc}{section}{Sec. 47-10. Dumping garbage.}

\markright{Sec. 47-10. Dumping garbage.}

It is unlawful for a person to dump or place garbage or another
offensive substance within the corporate limits of a city, village, or
incorporated town other than (1) the city, village, or incorporated town
within the corporate limits of which the garbage or other offensive
substance originated or (2) a city, village, or incorporated town that
has contracted with the city, village, or incorporated town within which
the garbage originated, for the joint collection and disposal of
garbage; nor shall the garbage or other offensive substance be dumped or
placed within a distance of one mile of the corporate limits of any
other city, village, or incorporated town.

A person violating this Section is guilty of a petty offense.

(Source: P.A. 89-234, eff. 1-1-96.)

\hypertarget{ilcs-547-15}{%
\subsection*{(720 ILCS 5/47-15)}\label{ilcs-547-15}}
\addcontentsline{toc}{subsection}{(720 ILCS 5/47-15)}

\hypertarget{sec.-47-15.-dumping-garbage-upon-real-property.}{%
\section*{Sec. 47-15. Dumping garbage upon real
property.}\label{sec.-47-15.-dumping-garbage-upon-real-property.}}
\addcontentsline{toc}{section}{Sec. 47-15. Dumping garbage upon real
property.}

\markright{Sec. 47-15. Dumping garbage upon real property.}

(a) It is unlawful for a person to dump, deposit, or place garbage,
rubbish, trash, or refuse upon real property not owned by that person
without the consent of the owner or person in possession of the real
property.

(b) A person who violates this Section is liable to the owner or person
in possession of the real property on which the garbage, rubbish, trash,
or refuse is dumped, deposited, or placed for the reasonable costs
incurred by the owner or person in possession for cleaning up and
properly disposing of the garbage, rubbish, trash, or refuse, and for
reasonable attorneys' fees.

(c) A person violating this Section is guilty of a Class B misdemeanor
for which the court must impose a minimum fine of \$500. A second
conviction for an offense committed after the first conviction is a
Class A misdemeanor for which the court must impose a minimum fine of
\$500. A third or subsequent violation, committed after a second
conviction, is a Class 4 felony for which the court must impose a
minimum fine of \$500. A person who violates this Section and who has an
equity interest in a motor vehicle used in violation of this Section is
presumed to have the financial resources to pay the minimum fine not
exceeding his or her equity interest in the vehicle. Personal property
used by a person in violation of this Section shall on the third or
subsequent conviction of the person be forfeited to the county where the
violation occurred and disposed of at a public sale. Before the
forfeiture, the court shall conduct a hearing to determine whether
property is subject to forfeiture under this Section. At the forfeiture
hearing the State has the burden of establishing by a preponderance of
the evidence that property is subject to forfeiture under this Section.
Property seized or forfeited under this Section is subject to reporting
under the Seizure and Forfeiture Reporting Act.

(d) The statutory minimum fine required by subsection (c) is not subject
to reduction or suspension unless the defendant is indigent. If the
defendant files a motion with the court asserting his or her inability
to pay the mandatory fine required by this Section, the court must set a
hearing on the motion before sentencing. The court must require an
affidavit signed by the defendant containing sufficient information to
ascertain the assets and liabilities of the defendant. If the court
determines that the defendant is indigent, the court must require that
the defendant choose either to pay the minimum fine of \$500 or to
perform 100 hours of community service.

(Source: P.A. 100-512, eff. 7-1-18 .)

\hypertarget{ilcs-547-20}{%
\subsection*{(720 ILCS 5/47-20)}\label{ilcs-547-20}}
\addcontentsline{toc}{subsection}{(720 ILCS 5/47-20)}

\hypertarget{sec.-47-20.-unplugged-well.}{%
\section*{Sec. 47-20. Unplugged
well.}\label{sec.-47-20.-unplugged-well.}}
\addcontentsline{toc}{section}{Sec. 47-20. Unplugged well.}

\markright{Sec. 47-20. Unplugged well.}

It is a Class A misdemeanor for a person to permit a water well, located
on property owned by him or her, to be in an unplugged condition at any
time after the abandonment of the well for obtaining water. No well is
in an unplugged condition, however, that is plugged in conformity with
the rules and regulations of the Department of Natural Resources issued
under Section 6 and Section 19 of the Illinois Oil and Gas Act. This
Section does not apply to a well drilled or used for observation or any
other purpose in connection with the development or operation of a gas
storage project.

(Source: P.A. 89-234, eff. 1-1-96; 89-445, eff. 2-7-96.)

\hypertarget{ilcs-547-25}{%
\subsection*{(720 ILCS 5/47-25)}\label{ilcs-547-25}}
\addcontentsline{toc}{subsection}{(720 ILCS 5/47-25)}

\hypertarget{sec.-47-25.-penalties.}{%
\section*{Sec. 47-25. Penalties.}\label{sec.-47-25.-penalties.}}
\addcontentsline{toc}{section}{Sec. 47-25. Penalties.}

\markright{Sec. 47-25. Penalties.}

Whoever causes, erects, or continues a nuisance described in this
Article, for the first offense, is guilty of a petty offense and shall
be fined not exceeding \$100, and for a subsequent offense is guilty of
a Class B misdemeanor. Every nuisance described in this Article, when a
conviction for that nuisance is had, may, by order of the court before
which the conviction is had, be abated by the sheriff or other proper
officer, at the expense of the defendant. It is not a defense to a
proceeding under this Section that the nuisance is erected or continued
by virtue or permission of a law of this State.

(Source: P.A. 89-234, eff. 1-1-96.)

\bookmarksetup{startatroot}

\hypertarget{article-48.-animals}{%
\chapter*{Article 48. Animals}\label{article-48.-animals}}
\addcontentsline{toc}{chapter}{Article 48. Animals}

\markboth{Article 48. Animals}{Article 48. Animals}

(Source: P.A. 97-1108, eff. 1-1-13.)

\hypertarget{ilcs-548-1}{%
\subsection*{(720 ILCS 5/48-1)}\label{ilcs-548-1}}
\addcontentsline{toc}{subsection}{(720 ILCS 5/48-1)}

(was 720 ILCS 5/26-5)

\hypertarget{sec.-48-1.-dog-fighting.}{%
\section*{Sec. 48-1. Dog fighting.}\label{sec.-48-1.-dog-fighting.}}
\addcontentsline{toc}{section}{Sec. 48-1. Dog fighting.}

\markright{Sec. 48-1. Dog fighting.}

(For other provisions that may apply to dog fighting, see the Humane
Care for Animals Act. For provisions similar to this Section that apply
to animals other than dogs, see in particular Section 4.01 of the Humane
Care for Animals Act.)

(a) No person may own, capture, breed, train, or lease any dog which he
or she knows is intended for use in any show, exhibition, program, or
other activity featuring or otherwise involving a fight between the dog
and any other animal or human, or the intentional killing of any dog for
the purpose of sport, wagering, or entertainment.

(b) No person may promote, conduct, carry on, advertise, collect money
for or in any other manner assist or aid in the presentation for
purposes of sport, wagering, or entertainment of any show, exhibition,
program, or other activity involving a fight between 2 or more dogs or
any dog and human, or the intentional killing of any dog.

(c) No person may sell or offer for sale, ship, transport, or otherwise
move, or deliver or receive any dog which he or she knows has been
captured, bred, or trained, or will be used, to fight another dog or
human or be intentionally killed for purposes of sport, wagering, or
entertainment.

(c-5) No person may solicit a minor to violate this Section.

(d) No person may manufacture for sale, shipment, transportation, or
delivery any device or equipment which he or she knows or should know is
intended for use in any show, exhibition, program, or other activity
featuring or otherwise involving a fight between 2 or more dogs, or any
human and dog, or the intentional killing of any dog for purposes of
sport, wagering, or entertainment.

(e) No person may own, possess, sell or offer for sale, ship, transport,
or otherwise move any equipment or device which he or she knows or
should know is intended for use in connection with any show, exhibition,
program, or activity featuring or otherwise involving a fight between 2
or more dogs, or any dog and human, or the intentional killing of any
dog for purposes of sport, wagering or entertainment.

(f) No person may knowingly make available any site, structure, or
facility, whether enclosed or not, that he or she knows is intended to
be used for the purpose of conducting any show, exhibition, program, or
other activity involving a fight between 2 or more dogs, or any dog and
human, or the intentional killing of any dog or knowingly manufacture,
distribute, or deliver fittings to be used in a fight between 2 or more
dogs or a dog and human.

(g) No person may knowingly attend or otherwise patronize any show,
exhibition, program, or other activity featuring or otherwise involving
a fight between 2 or more dogs, or any dog and human, or the intentional
killing of any dog for purposes of sport, wagering, or entertainment.

(h) No person may tie or attach or fasten any live animal to any machine
or device propelled by any power for the purpose of causing the animal
to be pursued by a dog or dogs. This subsection (h) applies only when
the dog is intended to be used in a dog fight.

(i) Sentence.

(1) Any person convicted of violating subsection (a),

(b), (c), or (h) of this Section is guilty of a Class 4 felony for a
first violation and a Class 3 felony for a second or subsequent
violation, and may be fined an amount not to exceed \$50,000.

(1.5) A person who knowingly owns a dog for fighting purposes or for
producing a fight between 2 or more dogs or a dog and human or who
knowingly offers for sale or sells a dog bred for fighting is guilty of
a Class 3 felony and may be fined an amount not to exceed \$50,000, if
the dog participates in a dogfight and any of the following factors is
present:

(i) the dogfight is performed in the presence of a person under 18 years
of age;

(ii) the dogfight is performed for the purpose of or in the presence of
illegal wagering activity; or

(iii) the dogfight is performed in furtherance of streetgang related
activity as defined in Section 10 of the Illinois Streetgang Terrorism
Omnibus Prevention Act.

(1.7) A person convicted of violating subsection

(c-5) of this Section is guilty of a Class 4 felony.

(2) Any person convicted of violating subsection (d) or (e) of this
Section is guilty of a Class 4 felony for a first violation. A second or
subsequent violation of subsection (d) or (e) of this Section is a Class
3 felony.

(2.5) Any person convicted of violating subsection

(f) of this Section is guilty of a Class 4 felony. Any person convicted
of violating subsection (f) of this Section in which the site,
structure, or facility made available to violate subsection (f) is
located within 1,000 feet of a school, public park, playground, child
care institution, day care center, part day child care facility, day
care home, group day care home, or a facility providing programs or
services exclusively directed toward persons under 18 years of age is
guilty of a Class 3 felony for a first violation and a Class 2 felony
for a second or subsequent violation.

(3) Any person convicted of violating subsection (g) of this Section is
guilty of a Class 4 felony for a first violation. A second or subsequent
violation of subsection (g) of this Section is a Class 3 felony. If a
person under 13 years of age is present at any show, exhibition,
program, or other activity prohibited in subsection (g), the parent,
legal guardian, or other person who is 18 years of age or older who
brings that person under 13 years of age to that show, exhibition,
program, or other activity is guilty of a Class 3 felony for a first
violation and a Class 2 felony for a second or subsequent violation.

(i-5) A person who commits a felony violation of this Section is subject
to the property forfeiture provisions set forth in Article 124B of the
Code of Criminal Procedure of 1963.

(j) Any dog or equipment involved in a violation of this Section shall
be immediately seized and impounded under Section 12 of the Humane Care
for Animals Act when located at any show, exhibition, program, or other
activity featuring or otherwise involving a dog fight for the purposes
of sport, wagering, or entertainment.

(k) Any vehicle or conveyance other than a common carrier that is used
in violation of this Section shall be seized, held, and offered for sale
at public auction by the sheriff's department of the proper
jurisdiction, and the proceeds from the sale shall be remitted to the
general fund of the county where the violation took place.

(l) Any veterinarian in this State who is presented with a dog for
treatment of injuries or wounds resulting from fighting where there is a
reasonable possibility that the dog was engaged in or utilized for a
fighting event for the purposes of sport, wagering, or entertainment
shall file a report with the Department of Agriculture and cooperate by
furnishing the owners' names, dates, and descriptions of the dog or dogs
involved. Any veterinarian who in good faith complies with the
requirements of this subsection has immunity from any liability, civil,
criminal, or otherwise, that may result from his or her actions. For the
purposes of any proceedings, civil or criminal, the good faith of the
veterinarian shall be rebuttably presumed.

(m) In addition to any other penalty provided by law, upon conviction
for violating this Section, the court may order that the convicted
person and persons dwelling in the same household as the convicted
person who conspired, aided, or abetted in the unlawful act that was the
basis of the conviction, or who knew or should have known of the
unlawful act, may not own, harbor, or have custody or control of any dog
or other animal for a period of time that the court deems reasonable.

(n) A violation of subsection (a) of this Section may be inferred from
evidence that the accused possessed any device or equipment described in
subsection (d), (e), or (h) of this Section, and also possessed any dog.

(o) When no longer required for investigations or court proceedings
relating to the events described or depicted therein, evidence relating
to convictions for violations of this Section shall be retained and made
available for use in training peace officers in detecting and
identifying violations of this Section. Such evidence shall be made
available upon request to other law enforcement agencies and to schools
certified under the Illinois Police Training Act.

(p) For the purposes of this Section, ``school'' has the meaning
ascribed to it in Section 11-9.3 of this Code; and ``public park'',
``playground'', ``child care institution'', ``day care center'', ``part
day child care facility'', ``day care home'', ``group day care home'',
and ``facility providing programs or services exclusively directed
toward persons under 18 years of age'' have the meanings ascribed to
them in Section 11-9.4 of this Code.

(Source: P.A. 96-226, eff. 8-11-09; 96-712, eff. 1-1-10; 96-1000, eff.
7-2-10; 96-1091, eff. 1-1-11; 97-1108, eff. 1-1-13.)

\hypertarget{ilcs-548-2}{%
\subsection*{(720 ILCS 5/48-2)}\label{ilcs-548-2}}
\addcontentsline{toc}{subsection}{(720 ILCS 5/48-2)}

\hypertarget{sec.-48-2.-animal-research-and-production-facilities-protection.}{%
\section*{Sec. 48-2. Animal research and production facilities
protection.}\label{sec.-48-2.-animal-research-and-production-facilities-protection.}}
\addcontentsline{toc}{section}{Sec. 48-2. Animal research and production
facilities protection.}

\markright{Sec. 48-2. Animal research and production facilities
protection.}

(a) Definitions.

``Animal'' means every living creature, domestic or wild, but does not
include man.

``Animal facility'' means any facility engaging in legal scientific
research or agricultural production of or involving the use of animals
including any organization with a primary purpose of representing
livestock production or processing, any organization with a primary
purpose of promoting or marketing livestock or livestock products, any
person licensed to practice veterinary medicine, any institution as
defined in the Impounding and Disposition of Stray Animals Act, and any
organization with a primary purpose of representing any such person,
organization, or institution. ``Animal facility'' shall include the
owner, operator, and employees of any animal facility and any premises
where animals are located.

``Director'' means the Director of the Illinois

Department of Agriculture or the Director's authorized representative.

(b) Legislative Declaration. There has been an increasing number of
illegal acts committed against animal research and production facilities
involving injury or loss of life to humans or animals, criminal trespass
and damage to property. These actions not only abridge the property
rights of the owner of the facility, they may also damage the public
interest by jeopardizing crucial scientific, biomedical, or agricultural
research or production. These actions can also threaten the public
safety by possibly exposing communities to serious public health
concerns and creating traffic hazards. These actions may substantially
disrupt or damage publicly funded research and can result in the
potential loss of physical and intellectual property. Therefore, it is
in the interest of the people of the State of Illinois to protect the
welfare of humans and animals as well as productive use of public funds
to require regulation to prevent unauthorized possession, alteration,
destruction, or transportation of research records, test data, research
materials, equipment, research and agricultural production animals.

(c) It shall be unlawful for any person:

(1) to release, steal, or otherwise intentionally cause the death,
injury, or loss of any animal at or from an animal facility and not
authorized by that facility;

(2) to damage, vandalize, or steal any property in or on an animal
facility;

(3) to obtain access to an animal facility by false pretenses for the
purpose of performing acts not authorized by that facility;

(4) to enter into an animal facility with an intent to destroy, alter,
duplicate, or obtain unauthorized possession of records, data,
materials, equipment, or animals;

(5) by theft or deception knowingly to obtain control or to exert
control over records, data, material, equipment, or animals of any
animal facility for the purpose of depriving the rightful owner or
animal facility of the records, material, data, equipment, or animals or
for the purpose of concealing, abandoning, or destroying these records,
material, data, equipment, or animals; or

(6) to enter or remain on an animal facility with the intent to commit
an act prohibited under this Section.

(d) Sentence.

(1) Any person who violates any provision of subsection (c) shall be
guilty of a Class 4 felony for each violation, unless the loss, theft,
or damage to the animal facility property exceeds \$300 in value.

(2) If the loss, theft, or damage to the animal facility property
exceeds \$300 in value but does not exceed \$10,000 in value, the person
is guilty of a Class 3 felony.

(3) If the loss, theft, or damage to the animal facility property
exceeds \$10,000 in value but does not exceed \$100,000 in value, the
person is guilty of a Class 2 felony.

(4) If the loss, theft, or damage to the animal facility property
exceeds \$100,000 in value, the person is guilty of a Class 1 felony.

(5) Any person who, with the intent that any violation of any provision
of subsection (c) be committed, agrees with another to the commission of
the violation and commits an act in furtherance of this agreement is
guilty of the same class of felony as provided in paragraphs (1) through
(4) of this subsection for that violation.

(6) Restitution.

(A) The court shall conduct a hearing to determine the reasonable cost
of replacing materials, data, equipment, animals and records that may
have been damaged, destroyed, lost or cannot be returned, and the
reasonable cost of repeating any experimentation that may have been
interrupted or invalidated as a result of a violation of subsection (c).

(B) Any persons convicted of a violation shall be ordered jointly and
severally to make restitution to the owner, operator, or both, of the
animal facility in the full amount of the reasonable cost determined
under paragraph (A).

(e) Private right of action. Nothing in this Section shall preclude any
animal facility injured in its business or property by a violation of
this Section from seeking appropriate relief under any other provision
of law or remedy including the issuance of a permanent injunction
against any person who violates any provision of this Section. The
animal facility owner or operator may petition the court to permanently
enjoin the person from violating this Section and the court shall
provide this relief.

(f) The Director shall have authority to investigate any alleged
violation of this Section, along with any other law enforcement agency,
and may take any action within the Director's authority necessary for
the enforcement of this Section. State's Attorneys, State police and
other law enforcement officials shall provide any assistance required in
the conduct of an investigation and prosecution. Before the Director
reports a violation for prosecution he or she may give the owner or
operator of the animal facility and the alleged violator an opportunity
to present his or her views at an administrative hearing. The Director
may adopt any rules and regulations necessary for the enforcement of
this Section.

(Source: P.A. 97-1108, eff. 1-1-13.)

\hypertarget{ilcs-548-3}{%
\subsection*{(720 ILCS 5/48-3)}\label{ilcs-548-3}}
\addcontentsline{toc}{subsection}{(720 ILCS 5/48-3)}

\hypertarget{sec.-48-3.-hunter-or-fisherman-interference.}{%
\section*{Sec. 48-3. Hunter or fisherman
interference.}\label{sec.-48-3.-hunter-or-fisherman-interference.}}
\addcontentsline{toc}{section}{Sec. 48-3. Hunter or fisherman
interference.}

\markright{Sec. 48-3. Hunter or fisherman interference.}

(a) Definitions. As used in this Section:

``Aquatic life'' means all fish, reptiles, amphibians, crayfish, and
mussels the taking of which is authorized by the Fish and Aquatic Life
Code.

``Interfere with'' means to take any action that physically impedes,
hinders, or obstructs the lawful taking of wildlife or aquatic life.

``Taking'' means the capture or killing of wildlife or aquatic life and
includes travel, camping, and other acts preparatory to taking which
occur on lands or waters upon which the affected person has the right or
privilege to take such wildlife or aquatic life.

``Wildlife'' means any wildlife the taking of which is authorized by the
Wildlife Code and includes those species that are lawfully released by
properly licensed permittees of the Department of Natural Resources.

(b) A person commits hunter or fisherman interference when he or she
intentionally or knowingly:

(1) obstructs or interferes with the lawful taking of wildlife or
aquatic life by another person with the specific intent to prevent that
lawful taking;

(2) drives or disturbs wildlife or aquatic life for the purpose of
disrupting a lawful taking of wildlife or aquatic life;

(3) blocks, impedes, or physically harasses another person who is
engaged in the process of lawfully taking wildlife or aquatic life;

(4) uses natural or artificial visual, aural, olfactory, gustatory, or
physical stimuli to affect wildlife or aquatic life behavior in order to
hinder or prevent the lawful taking of wildlife or aquatic life;

(5) erects barriers with the intent to deny ingress or egress to or from
areas where the lawful taking of wildlife or aquatic life may occur;

(6) intentionally interjects himself or herself into the line of fire or
fishing lines of a person lawfully taking wildlife or aquatic life;

(7) affects the physical condition or placement of personal or public
property intended for use in the lawful taking of wildlife or aquatic
life in order to impair the usefulness of the property or prevent the
use of the property;

(8) enters or remains upon or over private lands without the permission
of the owner or the owner's agent, with the intent to violate this
subsection;

(9) fails to obey the order of a peace officer to desist from conduct in
violation of this subsection (b) if the officer observes the conduct, or
has reasonable grounds to believe that the person has engaged in the
conduct that day or that the person plans or intends to engage in the
conduct that day on a specific premises; or

(10) uses a drone in a way that interferes with another person's lawful
taking of wildlife or aquatic life. For the purposes of this paragraph
(10), ``drone'' means any aerial vehicle that does not carry a human
operator.

(c) Exemptions; defenses.

(1) This Section does not apply to actions performed by authorized
employees of the Department of Natural Resources, duly accredited
officers of the U.S. Fish and Wildlife Service, sheriffs, deputy
sheriffs, or other peace officers if the actions are authorized by law
and are necessary for the performance of their official duties.

(2) This Section does not apply to landowners, tenants, or lease holders
exercising their legal rights to the enjoyment of land, including, but
not limited to, farming and restricting trespass.

(3) It is an affirmative defense to a prosecution for a violation of
this Section that the defendant's conduct is protected by his or her
right to freedom of speech under the constitution of this State or the
United States.

(4) Any interested parties may engage in protests or other free speech
activities adjacent to or on the perimeter of the location where the
lawful taking of wildlife or aquatic life is taking place, provided that
none of the provisions of this Section are being violated.

(d) Sentence. A first violation of paragraphs (1) through (8) of
subsection (b) is a Class B misdemeanor. A second or subsequent
violation of paragraphs (1) through (8) of subsection (b) is a Class A
misdemeanor for which imprisonment for not less than 7 days shall be
imposed. A person guilty of a second or subsequent violation of
paragraphs (1) through (8) of subsection (b) is not eligible for court
supervision. A violation of paragraph (9) or (10) of subsection (b) is a
Class A misdemeanor. A court shall revoke, for a period of one year to 5
years, any Illinois hunting, fishing, or trapping privilege, license or
permit of any person convicted of violating any provision of this
Section. For purposes of this subsection, a ``second or subsequent
violation'' means a conviction under paragraphs (1) through (8) of
subsection (b) of this Section within 2 years of a prior violation
arising from a separate set of circumstances.

(e) Injunctions; damages.

(1) Any court may enjoin conduct which would be in violation of
paragraphs (1) through (8) or (10) of subsection (b) upon petition by a
person affected or who reasonably may be affected by the conduct, upon a
showing that the conduct is threatened or that it has occurred on a
particular premises in the past and that it is not unreasonable to
expect that under similar circumstances it will be repeated.

(2) A court shall award all resulting costs and damages to any person
adversely affected by a violation of paragraphs (1) through (8) or (10)
of subsection (b), which may include an award for punitive damages. In
addition to other items of special damage, the measure of damages may
include expenditures of the affected person for license and permit fees,
travel, guides, special equipment and supplies, to the extent that these
expenditures were rendered futile by prevention of the taking of
wildlife or aquatic life.

(Source: P.A. 97-1108, eff. 1-1-13; 98-402, eff. 8-16-13.)

\hypertarget{ilcs-548-4}{%
\subsection*{(720 ILCS 5/48-4)}\label{ilcs-548-4}}
\addcontentsline{toc}{subsection}{(720 ILCS 5/48-4)}

\hypertarget{sec.-48-4.-obtaining-certificate-of-registration-by-false-pretenses.}{%
\section*{Sec. 48-4. Obtaining certificate of registration by false
pretenses.}\label{sec.-48-4.-obtaining-certificate-of-registration-by-false-pretenses.}}
\addcontentsline{toc}{section}{Sec. 48-4. Obtaining certificate of
registration by false pretenses.}

\markright{Sec. 48-4. Obtaining certificate of registration by false
pretenses.}

(a) A person commits obtaining certificate of registration by false
pretenses when he or she, by any false pretense, obtains from any club,
association, society or company for improving the breed of cattle,
horses, sheep, swine, or other domestic animals, a certificate of
registration of any animal in the herd register, or other register of
any club, association, society or company, or a transfer of the
registration.

(b) A person commits obtaining certificate of registration by false
pretenses when he or she knowingly gives a false pedigree of any animal.

(c) Sentence. Obtaining certificate of registration by false pretenses
is a Class A misdemeanor.

(Source: P.A. 97-1108, eff. 1-1-13.)

\hypertarget{ilcs-548-5}{%
\subsection*{(720 ILCS 5/48-5)}\label{ilcs-548-5}}
\addcontentsline{toc}{subsection}{(720 ILCS 5/48-5)}

\hypertarget{sec.-48-5.-horse-mutilation.}{%
\section*{Sec. 48-5. Horse
mutilation.}\label{sec.-48-5.-horse-mutilation.}}
\addcontentsline{toc}{section}{Sec. 48-5. Horse mutilation.}

\markright{Sec. 48-5. Horse mutilation.}

(a) A person commits horse mutilation when he or she cuts the solid part
of the tail of any horse in the operation known as docking, or by any
other operation performed for the purpose of shortening the tail, and
whoever shall cause the same to be done, or assist in doing this
cutting, unless the same is proved to be a benefit to the horse.

(b) Sentence. Horse mutilation is a Class A misdemeanor.

(Source: P.A. 97-1108, eff. 1-1-13.)

\hypertarget{ilcs-548-6}{%
\subsection*{(720 ILCS 5/48-6)}\label{ilcs-548-6}}
\addcontentsline{toc}{subsection}{(720 ILCS 5/48-6)}

\hypertarget{sec.-48-6.-horse-racing-false-entry.}{%
\section*{Sec. 48-6. Horse racing false
entry.}\label{sec.-48-6.-horse-racing-false-entry.}}
\addcontentsline{toc}{section}{Sec. 48-6. Horse racing false entry.}

\markright{Sec. 48-6. Horse racing false entry.}

(a) That in order to encourage the breeding of and improvement in
trotting, running and pacing horses in the State, it is hereby made
unlawful for any person or persons knowingly to enter or cause to be
entered for competition, or knowingly to compete with any horse, mare,
gelding, colt or filly under any other than its true name or out of its
proper class for any purse, prize, premium, stake or sweepstakes offered
or given by any agricultural or other society, association, person or
persons in the State where the prize, purse, premium, stake or
sweepstakes is to be decided by a contest of speed.

(b) The name of any horse, mare, gelding, colt or filly, for the purpose
of entry for competition or performance in any contest of speed, shall
be the name under which the horse has publicly performed, and shall not
be changed after having once so performed or contested for a prize,
purse, premium, stake or sweepstakes, except as provided by the code of
printed rules of the society or association under which the contest is
advertised to be conducted.

(c) The official records shall be received in all courts as evidence
upon the trial of any person under the provisions of this Section.

(d) Sentence. A violation of subsection (a) is a Class 4 felony.

(Source: P.A. 97-1108, eff. 1-1-13.)

\hypertarget{ilcs-548-7}{%
\subsection*{(720 ILCS 5/48-7)}\label{ilcs-548-7}}
\addcontentsline{toc}{subsection}{(720 ILCS 5/48-7)}

\hypertarget{sec.-48-7.-feeding-garbage-to-animals.}{%
\section*{Sec. 48-7. Feeding garbage to
animals.}\label{sec.-48-7.-feeding-garbage-to-animals.}}
\addcontentsline{toc}{section}{Sec. 48-7. Feeding garbage to animals.}

\markright{Sec. 48-7. Feeding garbage to animals.}

(a) Definitions. As used in this Section:

``Department'' means the Department of Agriculture of the State of
Illinois.

``Garbage'' has the same meaning as in the federal Swine Health
Protection Act (7 U.S.C. 3802) and also includes putrescible vegetable
waste. ``Garbage'' does not include the contents of the bovine digestive
tract.

``Person'' means any person, firm, partnership, association,
corporation, or other legal entity, any public or private institution,
the State, or any municipal corporation or political subdivision of the
State.

(b) A person commits feeding garbage to animals when he or she feeds or
permits the feeding of garbage to swine or any animals or poultry on any
farm or any other premises where swine are kept.

(c) Establishments licensed under the Animal Mortality Act or under
similar laws in other states are exempt from the provisions of this
Section.

(d) Nothing in this Section shall be construed to apply to any person
who feeds garbage produced in his or her own household to animals or
poultry kept on the premises where he or she resides except this garbage
if fed to swine shall not contain particles of meat.

(e) Sentence. Feeding garbage to animals is a Class B misdemeanor, and
for the first offense shall be fined not less than \$100 nor more than
\$500 and for a second or subsequent offense shall be fined not less
than \$200 nor more than \$500 or imprisoned in a penal institution
other than the penitentiary for not more than 6 months, or both.

(f) A person violating this Section may be enjoined by the Department
from continuing the violation.

(g) The Department may make reasonable inspections necessary for the
enforcement of this Section, and is authorized to enforce, and
administer the provisions of this Section.

(Source: P.A. 102-216, eff. 1-1-22 .)

\hypertarget{ilcs-548-8}{%
\subsection*{(720 ILCS 5/48-8)}\label{ilcs-548-8}}
\addcontentsline{toc}{subsection}{(720 ILCS 5/48-8)}

\hypertarget{sec.-48-8.-service-animal-access.}{%
\section*{Sec. 48-8. Service animal
access.}\label{sec.-48-8.-service-animal-access.}}
\addcontentsline{toc}{section}{Sec. 48-8. Service animal access.}

\markright{Sec. 48-8. Service animal access.}

(a) When a person with a physical, mental, or intellectual disability
requiring the use of a service animal is accompanied by a service animal
or when a trainer of a service animal is accompanied by a service
animal, neither the person nor the service animal shall be denied the
right of entry and use of facilities of any public place of
accommodation as defined in Section 5-101 of the Illinois Human Rights
Act.

For the purposes of this Section, ``service animal'' means a dog or
miniature horse trained or being trained as a hearing animal, a guide
animal, an assistance animal, a seizure alert animal, a mobility animal,
a psychiatric service animal, an autism service animal, or an animal
trained for any other physical, mental, or intellectual disability.
``Service animal'' includes a miniature horse that a public place of
accommodation shall make reasonable accommodation so long as the public
place of accommodation takes into consideration: (1) the type, size, and
weight of the miniature horse and whether the facility can accommodate
its features; (2) whether the handler has sufficient control of the
miniature horse; (3) whether the miniature horse is housebroken; and (4)
whether the miniature horse's presence in the facility compromises
legitimate safety requirements necessary for operation.

(b) A person who knowingly violates this Section commits a Class C
misdemeanor.

(Source: P.A. 97-1108, eff. 1-1-13; incorporates 97-956, eff. 8-14-12;
97-1150, eff. 1-25-13.)

\hypertarget{ilcs-548-9}{%
\subsection*{(720 ILCS 5/48-9)}\label{ilcs-548-9}}
\addcontentsline{toc}{subsection}{(720 ILCS 5/48-9)}

\hypertarget{sec.-48-9.-misrepresentation-of-stallion-and-jack-pedigree.}{%
\section*{Sec. 48-9. Misrepresentation of stallion and jack
pedigree.}\label{sec.-48-9.-misrepresentation-of-stallion-and-jack-pedigree.}}
\addcontentsline{toc}{section}{Sec. 48-9. Misrepresentation of stallion
and jack pedigree.}

\markright{Sec. 48-9. Misrepresentation of stallion and jack pedigree.}

(a) The owner or keeper of any stallion or jack kept for public service
commits misrepresentation of stallion and jack pedigree when he or she
misrepresents the pedigree or breeding of the stallion or jack, or
represents that the animal, so kept for public service, is registered,
when in fact it is not registered in a published volume of a society for
the registry of standard and purebred animals, or who shall post or
publish, or cause to be posted or published, any false pedigree or
breeding of this animal.

(b) Sentence. Misrepresentation of stallion and jack pedigree is a petty
offense, and for a second or subsequent offense is a Class B
misdemeanor.

(Source: P.A. 97-1108, eff. 1-1-13.)

\hypertarget{ilcs-548-10}{%
\subsection*{(720 ILCS 5/48-10)}\label{ilcs-548-10}}
\addcontentsline{toc}{subsection}{(720 ILCS 5/48-10)}

\hypertarget{sec.-48-10.-dangerous-animals.}{%
\section*{Sec. 48-10. Dangerous
animals.}\label{sec.-48-10.-dangerous-animals.}}
\addcontentsline{toc}{section}{Sec. 48-10. Dangerous animals.}

\markright{Sec. 48-10. Dangerous animals.}

(a) Definitions. As used in this Section, unless the context otherwise
requires:

``Dangerous animal'' means a lion, tiger, leopard, ocelot, jaguar,
cheetah, margay, mountain lion, lynx, bobcat, jaguarundi, bear, hyena,
wolf or coyote. Dangerous animal does not mean any herptiles included in
the Herptiles-Herps Act.

``Owner'' means any person who (1) has a right of property in a
dangerous animal or primate, (2) keeps or harbors a dangerous animal or
primate, (3) has a dangerous animal or primate in his or her care, or
(4) acts as custodian of a dangerous animal or primate.

``Person'' means any individual, firm, association, partnership,
corporation, or other legal entity, any public or private institution,
the State, or any municipal corporation or political subdivision of the
State.

``Primate'' means a nonhuman member of the order primate, including but
not limited to chimpanzee, gorilla, orangutan, bonobo, gibbon, monkey,
lemur, loris, aye-aye, and tarsier.

(b) Dangerous animal or primate offense. No person shall have a right of
property in, keep, harbor, care for, act as custodian of or maintain in
his or her possession any dangerous animal or primate except at a
properly maintained zoological park, federally licensed exhibit, circus,
college or university, scientific institution, research laboratory,
veterinary hospital, hound running area, or animal refuge in an
escape-proof enclosure.

(c) Exemptions.

(1) This Section does not prohibit a person who had lawful possession of
a primate before January 1, 2011, from continuing to possess that
primate if the person registers the animal by providing written
notification to the local animal control administrator on or before
April 1, 2011. The notification shall include:

(A) the person's name, address, and telephone number; and

(B) the type of primate, the age, a photograph, a description of any
tattoo, microchip, or other identifying information, and a list of
current inoculations.

(2) This Section does not prohibit a person who has a permanent
disability with a severe mobility impairment from possessing a single
capuchin monkey to assist the person in performing daily tasks if:

(A) the capuchin monkey was obtained from and trained at a licensed
nonprofit organization described in Section 501(c)(3) of the Internal
Revenue Code of 1986, the nonprofit tax status of which was obtained on
the basis of a mission to improve the quality of life of severely
mobility-impaired individuals; and

(B) the person complies with the notification requirements as described
in paragraph (1) of this subsection (c).

(d) A person who registers a primate shall notify the local animal
control administrator within 30 days of a change of address. If the
person moves to another locality within the State, the person shall
register the primate with the new local animal control administrator
within 30 days of moving by providing written notification as provided
in paragraph (1) of subsection (c) and shall include proof of the prior
registration.

(e) A person who registers a primate shall notify the local animal
control administrator immediately if the primate dies, escapes, or
bites, scratches, or injures a person.

(f) It is no defense to a violation of subsection (b) that the person
violating subsection (b) has attempted to domesticate the dangerous
animal. If there appears to be imminent danger to the public, any
dangerous animal found not in compliance with the provisions of this
Section shall be subject to seizure and may immediately be placed in an
approved facility. Upon the conviction of a person for a violation of
subsection (b), the animal with regard to which the conviction was
obtained shall be confiscated and placed in an approved facility, with
the owner responsible for all costs connected with the seizure and
confiscation of the animal. Approved facilities include, but are not
limited to, a zoological park, federally licensed exhibit, humane
society, veterinary hospital or animal refuge.

(g) Sentence. Any person violating this Section is guilty of a Class C
misdemeanor. Any corporation or partnership, any officer, director,
manager or managerial agent of the partnership or corporation who
violates this Section or causes the partnership or corporation to
violate this Section is guilty of a Class C misdemeanor. Each day of
violation constitutes a separate offense.

(Source: P.A. 98-752, eff. 1-1-15; 99-143, eff. 7-27-15.)

\hypertarget{ilcs-548-11}{%
\subsection*{(720 ILCS 5/48-11)}\label{ilcs-548-11}}
\addcontentsline{toc}{subsection}{(720 ILCS 5/48-11)}

\hypertarget{sec.-48-11.-unlawful-use-of-an-elephant-in-a-traveling-animal-act.}{%
\section*{Sec. 48-11. Unlawful use of an elephant in a traveling animal
act.}\label{sec.-48-11.-unlawful-use-of-an-elephant-in-a-traveling-animal-act.}}
\addcontentsline{toc}{section}{Sec. 48-11. Unlawful use of an elephant
in a traveling animal act.}

\markright{Sec. 48-11. Unlawful use of an elephant in a traveling animal
act.}

(a) Definitions. As used in this Section:

``Mobile or traveling animal housing facility'' means a transporting
vehicle such as a truck, trailer, or railway car used to transport or
house animals while traveling to an exhibition or other performance.

``Performance'' means an exhibition, public showing, presentation,
display, exposition, fair, animal act, circus, ride, trade show, petting
zoo, carnival, parade, race, or other similar undertaking in which
animals are required to perform tricks, give rides, or participate as
accompaniments for entertainment, amusement, or benefit of a live
audience.

``Traveling animal act'' means any performance of animals where animals
are transported to, from, or between locations for the purpose of a
performance in a mobile or traveling animal housing facility.

(b) A person commits unlawful use of an elephant in a traveling animal
act when he or she knowingly allows for the participation of an African
elephant (Loxodonta africana) or Asian elephant (Elephas maximus)
protected under the federal Endangered Species Act of 1973 in a
traveling animal act.

(c) This Section does not apply to an exhibition of elephants at a
non-mobile, permanent institution, or other facility.

(d) Sentence. Unlawful use of an elephant in a traveling animal act is a
Class A misdemeanor.

(Source: P.A. 100-90, eff. 1-1-18 .)

\bookmarksetup{startatroot}

\hypertarget{article-49.-miscellaneous-offenses}{%
\chapter*{Article 49. Miscellaneous
Offenses}\label{article-49.-miscellaneous-offenses}}
\addcontentsline{toc}{chapter}{Article 49. Miscellaneous Offenses}

\markboth{Article 49. Miscellaneous Offenses}{Article 49. Miscellaneous
Offenses}

(Source: P.A. 97-1108, eff. 1-1-13.)

\hypertarget{ilcs-549-1}{%
\subsection*{(720 ILCS 5/49-1)}\label{ilcs-549-1}}
\addcontentsline{toc}{subsection}{(720 ILCS 5/49-1)}

\hypertarget{sec.-49-1.-flag-desecration.}{%
\section*{Sec. 49-1. Flag
desecration.}\label{sec.-49-1.-flag-desecration.}}
\addcontentsline{toc}{section}{Sec. 49-1. Flag desecration.}

\markright{Sec. 49-1. Flag desecration.}

(a) Definition. As used in this Section:

``Flag'', ``standard'', ``color'' or ``ensign'' shall include any flag,
standard, color, ensign or any picture or representation of either
thereof, made of any substance or represented on any substance and of
any size evidently purporting to be either of said flag, standard, color
or ensign of the United States of America, or a picture or a
representation of either thereof, upon which shall be shown the colors,
the stars, and the stripes, in any number of either thereof, of the
flag, colors, standard, or ensign of the United States of America.

(b) A person commits flag desecration when he or she knowingly:

(1) for exhibition or display, places or causes to be placed any word,
figure, mark, picture, design, drawing, or any advertisement of any
nature, upon any flag, standard, color or ensign of the United States or
State flag of this State or ensign;

(2) exposes or causes to be exposed to public view any such flag,
standard, color or ensign, upon which has been printed, painted or
otherwise placed, or to which has been attached, appended, affixed, or
annexed, any word, figure, mark, picture, design or drawing or any
advertisement of any nature;

(3) exposes to public view, manufactures, sells, exposes for sale, gives
away, or has in possession for sale or to give away or for use for any
purpose, any article or substance, being an article of merchandise, or a
receptacle of merchandise or article or thing for carrying or
transporting merchandise upon which has been printed, painted, attached,
or otherwise placed a representation of any such flag, standard, color,
or ensign, to advertise, call attention to, decorate, mark or
distinguish the article or substance on which so placed; or

(4) publicly mutilates, defaces, defiles, tramples, or intentionally
displays on the ground or floor any such flag, standard, color or
ensign.

(c) All prosecutions under this Section shall be brought by any person
in the name of the People of the State of Illinois, against any person
or persons violating any of the provisions of this Section, before any
circuit court. The State's Attorneys shall see that this Section is
enforced in their respective counties, and shall prosecute all offenders
on receiving information of the violation of this Section. Sheriffs,
deputy sheriffs, and police officers shall inform against and prosecute
all persons whom there is probable cause to believe are guilty of
violating this Section. One-half of the amount recovered in any penal
action under this Section shall be paid to the person making and filing
the complaint in the action, and the remaining 1/2 to the school fund of
the county in which the conviction is obtained.

(d) All prosecutions under this Section shall be commenced within six
months from the time the offense was committed, and not afterwards.

(e) Sentence. A violation of paragraphs (1) through (3) of subsection
(b) is a Class C misdemeanor. A violation of paragraph (4) of subsection
(b) is a Class 4 felony.

(Source: P.A. 97-1108, eff. 1-1-13.)

\hypertarget{ilcs-549-1.5}{%
\subsection*{(720 ILCS 5/49-1.5)}\label{ilcs-549-1.5}}
\addcontentsline{toc}{subsection}{(720 ILCS 5/49-1.5)}

\hypertarget{sec.-49-1.5.-draft-card-mutilation.}{%
\section*{Sec. 49-1.5. Draft card
mutilation.}\label{sec.-49-1.5.-draft-card-mutilation.}}
\addcontentsline{toc}{section}{Sec. 49-1.5. Draft card mutilation.}

\markright{Sec. 49-1.5. Draft card mutilation.}

(a) A person commits draft card mutilation when he or she knowingly
destroys or mutilates a valid registration certificate or any other
valid certificate issued under the federal ``Military Selective Service
Act of 1967''.

(b) Sentence. Draft card mutilation is a Class 4 felony.

(Source: P.A. 97-1108, eff. 1-1-13.)

\hypertarget{ilcs-549-2}{%
\subsection*{(720 ILCS 5/49-2)}\label{ilcs-549-2}}
\addcontentsline{toc}{subsection}{(720 ILCS 5/49-2)}

\hypertarget{sec.-49-2.-business-use-of-military-terms.}{%
\section*{Sec. 49-2. Business use of military
terms.}\label{sec.-49-2.-business-use-of-military-terms.}}
\addcontentsline{toc}{section}{Sec. 49-2. Business use of military
terms.}

\markright{Sec. 49-2. Business use of military terms.}

(a) It is unlawful for any person, concern, firm or corporation to use
in the name, or description of the name, of any privately operated
mercantile establishment which may or may not be engaged principally in
the buying and selling of equipment or materials of the Government of
the United States or any of its departments, agencies or military
services, the terms ``Army'', ``Navy'', ``Marine'', ``Coast Guard'',
``Government'', ``GI'', ``PX'' or any terms denoting a branch of the
government, either independently or in connection or conjunction with
any other word or words, letter or insignia which import or imply that
the products so described are or were made for the United States
government or in accordance with government specifications or
requirements, or of government materials, or that these products have
been disposed of by the United States government as surplus or rejected
stock.

(b) Sentence. A violation of this Section is a petty offense with a fine
of not less than \$25.00 nor more than \$500 for the first conviction,
and not less than \$500 or more than \$1000 for each subsequent
conviction.

(Source: P.A. 97-1108, eff. 1-1-13.)

\hypertarget{ilcs-549-3}{%
\subsection*{(720 ILCS 5/49-3)}\label{ilcs-549-3}}
\addcontentsline{toc}{subsection}{(720 ILCS 5/49-3)}

\hypertarget{sec.-49-3.-governmental-uneconomic-practices.}{%
\section*{Sec. 49-3. Governmental uneconomic
practices.}\label{sec.-49-3.-governmental-uneconomic-practices.}}
\addcontentsline{toc}{section}{Sec. 49-3. Governmental uneconomic
practices.}

\markright{Sec. 49-3. Governmental uneconomic practices.}

(a) It is unlawful for the State of Illinois, any political subdivision
thereof, or any municipality therein, or any officer, agent or employee
of the State of Illinois, any political subdivision thereof or any
municipality therein, to sell to or procure for sale or have in its or
his or her possession or under its or his or her control for sale to any
officer, agent or employee of the State or any political subdivision
thereof or municipality therein any article, material, product or
merchandise of whatsoever nature, excepting meals, public services and
such specialized appliances and paraphernalia as may be required for the
safety or health of such officers, agents or employees.

(b) The provisions of this Section shall not apply to the State, any
political subdivision thereof or municipality therein, nor to any
officer, agent or employee of the State, or of any such subdivision or
municipality while engaged in any recreational, health, welfare, relief,
safety or educational activities furnished by the State, or any such
political subdivision or municipality.

(c) Sentence. A violation of this Section is a Class B misdemeanor.

(Source: P.A. 97-1108, eff. 1-1-13.)

\hypertarget{ilcs-549-4}{%
\subsection*{(720 ILCS 5/49-4)}\label{ilcs-549-4}}
\addcontentsline{toc}{subsection}{(720 ILCS 5/49-4)}

\hypertarget{sec.-49-4.-sale-of-maps.}{%
\section*{Sec. 49-4. Sale of maps.}\label{sec.-49-4.-sale-of-maps.}}
\addcontentsline{toc}{section}{Sec. 49-4. Sale of maps.}

\markright{Sec. 49-4. Sale of maps.}

(a) The sale of current Illinois publications or highway maps published
by the Secretary of State is prohibited except where provided by law.

(b) Sentence. A violation of this Section is a Class B misdemeanor.

(Source: P.A. 97-1108, eff. 1-1-13.)

\hypertarget{ilcs-549-5}{%
\subsection*{(720 ILCS 5/49-5)}\label{ilcs-549-5}}
\addcontentsline{toc}{subsection}{(720 ILCS 5/49-5)}

\hypertarget{sec.-49-5.-video-movie-sales-and-rentals-rating-violation.}{%
\section*{Sec. 49-5. Video movie sales and rentals rating
violation.}\label{sec.-49-5.-video-movie-sales-and-rentals-rating-violation.}}
\addcontentsline{toc}{section}{Sec. 49-5. Video movie sales and rentals
rating violation.}

\markright{Sec. 49-5. Video movie sales and rentals rating violation.}

(a) Definitions. As used in this Section, unless the context otherwise
requires:

``Person'' means an individual, corporation, partnership, or any other
legal or commercial entity.

``Official rating'' means an official rating of the

Motion Picture Association of America.

``Video movie'' means a videotape or video disc copy of a motion picture
film.

(b) A person may not sell at retail or rent, or attempt to sell at
retail or rent, a video movie in this State unless the official rating
of the motion picture from which it is copied is clearly displayed on
the outside of any cassette, case, jacket, or other covering of the
video movie.

(c) This Section does not apply to any video movie of a motion picture
which:

(1) has not been given an official rating; or

(2) has been altered in any way subsequent to receiving an official
rating.

(d) Sentence. A violation of this Section is a Class C misdemeanor.

(Source: P.A. 97-1108, eff. 1-1-13.)

\hypertarget{ilcs-549-6}{%
\subsection*{(720 ILCS 5/49-6)}\label{ilcs-549-6}}
\addcontentsline{toc}{subsection}{(720 ILCS 5/49-6)}

\hypertarget{sec.-49-6.-container-label-obliteration-prohibited.}{%
\section*{Sec. 49-6. Container label obliteration
prohibited.}\label{sec.-49-6.-container-label-obliteration-prohibited.}}
\addcontentsline{toc}{section}{Sec. 49-6. Container label obliteration
prohibited.}

\markright{Sec. 49-6. Container label obliteration prohibited.}

(a) No person shall sell or offer for sale any product, article or
substance in a container on which any statement of weight, quantity,
quality, grade, ingredients or identification of the manufacturer,
supplier or processor is obliterated by any other labeling unless the
other labeling correctly restates the obliterated statement.

(b) This Section does not apply to any obliteration which is done in
order to comply with subsection (c) of this Section.

(c) No person shall utilize any used container for the purpose of sale
of any product, article or substance unless the original marks of
identification, weight, grade, quality and quantity have first been
obliterated.



\end{document}
